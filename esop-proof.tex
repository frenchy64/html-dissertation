\chapter{Soundness for \lambdatc{}}

{\javaassumptionsall{appendix}}

\begin{lemma} \label{appendix:lemma:envagree}
  If \openv{} and \openvp{} agree on \fv{\prop{}}
  and \satisfies{\openv{}}{\prop{}}
  then \satisfies{\openvp{}}{\prop{}}.
\begin{proof}
  Since the relevant parts of \openv{} and \openvp{} agree, the proof follows trivially.
\end{proof}
\end{lemma}

\begin{lemma} \label{appendix:lemma:substfilter}
  If 
  \begin{itemize}
    \item \prop{1} = {\replacefor {\prop{2}} {\object{}} {\x{}}},
    \item
  {\satisfies{\openv{2}}{\prop{2}}},
    \item
  $\forall v \in \fv{\prop{2}} - \x{}$.
                              {\inopenvnoeq{\openv{1}}{v}} = {\inopenvnoeq {\openv{2}}{v}},
    \item
  and {\inopenvnoeq{\openv{2}}{\x{}}} = {\inopenvnoeq{\openv{1}}{\object{}}}
  \end{itemize}
  then \satisfies{\openv{1}}{\prop{1}}.

  \begin{proof}
    By induction on the derivation of the model judgement.
  \end{proof}
\end{lemma}

\begin{lemma} \label{appendix:lemma:satisfies}
  If \satisfies{\openv{}}{\propenv{}} and \inpropenv{\propenv{}}{\prop{}} then \satisfies{\openv{}}{\prop{}}.

  \begin{proof}
    By structural induction on \inpropenv{\propenv{}}{\prop{}}.
%    \begin{itemize}
%      \item[]
%        \begin{case}[L-True]
%
%          Holds by M-Top.
%        \end{case}
%      \item[]
%        \begin{case}[L-False]
%          {\inpropenv{\propenv{}}{\botprop{}}}
%
%          ??? TODO
%        \end{case}
%      \item[]
%        \begin{case}[L-AndI]
%          \inpropenv{\propenv{}}{\andprop{\prop{1}}{\prop{2}}}, \satisfies{\openv{}}{\propenv{}}
%
%          By inversion on the proof system we know \inpropenv{\propenv{}}{\prop{1}}
%          and
%          \inpropenv{\propenv{}}{\prop{2}}.
%
%          By the induction hypothesis we know \satisfies{\openv{}}{\prop{1}}
%          and
%          \satisfies{\openv{}}{\prop{2}}.
%
%          By M-And we know \satisfies{\openv{}}{\andprop{\prop{1}}{\prop{2}}}
%          and we are done.
%        \end{case}
%      \item[]
%        \begin{case}[L-AndE]
%          \inpropenv{\propenv{},{\andprop{\prop{1}}{\prop{2}}}}{\prop{}}, \satisfies{\openv{}}{\propenv{},{\andprop{\prop{1}}{\prop{2}}}}
%
%
%          By inversion on the proof system we know  either
%          \inpropenv{\propenv{},{\prop{1}}}{\prop{}}
%          or
%          \inpropenv{\propenv{},{\prop{2}}}{\prop{}}.
%
%          %TODO
%         % By the induction hypothesis we know 
%         % either
%         % \satisfies{\openv{}}{\prop{1}}
%         % and
%         % \satisfies{\openv{}}{\prop{2}}.
%        \end{case}
%    \end{itemize}
  \end{proof}
\end{lemma}

\begin{lemma} \label{appendix:lemma:goodobjects+ve}
  If \inpropenv{\propenv{}}{\isprop{\ty{}}{\pth{\pathelem{}}{\x{}}}},
  \satisfies{\openv{}}{\propenv{}}
  and \inopenv{\openv{}}{\pth{\pathelem{}}{\x{}}}{\val{}}
  then
  \judgementselfrewrite{}{\val{}}{\ty{}}{\filterset{\thenprop{\propp{}}}{\elseprop{\propp{}}}}{\objectp{}}
  for some {\thenprop{\propp{}}}, {\elseprop{\propp{}}} and {\objectp{}}.
  \begin{proof}
    Corollary of lemma~\ref{appendix:lemma:satisfies}.
  \end{proof}
\end{lemma}

\begin{lemma}[Paths are independent] \label{appendix:lemma:pathindependent}
  If \inopenvnoeq{\openv{}}{\object{}} = \inopenvnoeq{\openv{1}}{\objectp{}}
  then \inopenvnoeq{\openv{}}{\pth{\pathelem{}}{\object{}}} =
       \inopenvnoeq{\openv{1}}{\pth{\pathelem{}}{\objectp{}}}
 \begin{proof}
   By induction on \pathelem{}.
   % FIXME
%   \begin{case}[\pathelem{} = \emptypath{}]
%     \inopenvnoeq{\openv{}}{\object{}} = {\inopenvnoeq{\openv{}}{\objectp{}}}
%
%     As 
%     \inopenvnoeq{\openv{}}{\pth{\emptypath{}}{\object{}}} = \inopenvnoeq{\openv{}}{\object{}}
%     and
%     \inopenvnoeq{\openv{}}{\pth{\emptypath{}}{\objectp{}}} = \inopenvnoeq{\openv{}}{\objectp{}}
%     we can conclude 
%     \inopenvnoeq{\openv{}}{\pth{\emptypath{}}{\object{}}} = \inopenvnoeq{\openv{}}{\pth{\emptypath{}}{\objectp{}}}.
%   \end{case}
%   \begin{case}[\pathelem{} = {\destructpath{\pesyntax{}}{\pathelem{1}}}]
%     \inopenvnoeq{\openv{}}{\object{}} = {\inopenvnoeq{\openv{}}{\objectp{}}}
%
%     By cases on \pesyntax{}.
%
%     \begin{itemize}
%       \item[]
%   \begin{subcase}[\pesyntax{} = {\keype{\kw{}}}] 
%
%%     TODO
%     By the induction hypothesis on {\pathelem{1}}
%     we know {\inopenvnoeq{\openv{}}{\pth{\pathelem{1}}{\object{}}}} =
%             {\inopenvnoeq{\openv{1}}{\pth{\pathelem{1}}{\objectp{}}}}.
%             By the definition of pth translation 
%             {\inopenvnoeq{\openv{}}{\pth{\pathelem{1}}{\object{}}}} = {\getexp {{\openv{}}(\object{})}{\kw{}}}
%             and 
%             {\inopenvnoeq{\openv{}}{\pth{\pathelem{1}}{\objectp{}}}} = {\getexp {{\openv{}}(\objectp{})}{\kw{}}}
%   \end{subcase} 
%     \end{itemize}
%%     TODO
%   \end{case}
 \end{proof}
\end{lemma}

\begin{lemma}[\classconst]\label{appendix:lemma:classconst}
  If
  {\opsem{\openv{}}{\appexp{\classconst{}}{\openv{}({\pth{\pathelem{}}{\x{}}})}}{\class{}}} then
  {\satisfies{\openv{}}{\isprop{\class{}}{\pth{\pathelem{}}{\x{}}}}}.

  \begin{proof}
    Induction on the definition of {\classconst{}}.
  \end{proof}
\end{lemma}

{\consistentwithdefinition{appendix}}

{\istruefalsedefinitions{appendix}}

%\begin{lemma}[Path substitution] \label{appendix:lemma:pathsubustitution}
%  If \satisfies{\openv{}}{\prop{}} and 
%  \openv(\object{}) = \openv(\objectp{})
%  then \satisfies{\openv{}}{\replacefor{\prop{}}{\object{}}{\objectp{}}}.
%  \begin{proof}
%    By straightforward induction on \prop{}.
%  \end{proof}
%\end{lemma}
%

\begin{lemma}[isa? has correct propositions] \label{appendix:lemma:isa}
  If
  \begin{itemize}
    \item
  \judgementrewrite {\propenv{}} {\val{1}} {\ty{1}}
             {\filterset {\thenprop {\prop{1}}}
                         {\elseprop {\prop{1}}}}
                       {\object{1}}
                       {\val{1}},
    \item
  \judgementrewrite {\propenv{}} {\val{2}} {\ty{2}}
             {\filterset {\thenprop {\prop{2}}}
                         {\elseprop {\prop{2}}}}
                       {\object{2}}
                       {\val{2}},
    \item
        \isaopsem{\val{1}}{\val{2}} = {\val{}}, 
    \item
        \satisfies{\openv{}}{\propenv{}},
    \item
  \isacompare{\ty{1}}{\object{1}}{\ty{2}}{\filterset {\thenprop {\propp{}}} {\elseprop {\propp{}}}},
    \item
        \inpropenv{\thenprop{\propp{}}}{\thenprop{\prop{}}}, and
    \item
        \inpropenv{\elseprop{\propp{}}}{\elseprop{\prop{}}},
    \end{itemize}
  then either
\begin{itemize}
  \item
        if
        \istrueval{\val{}}
        then {\satisfies{\openv{}}{\thenprop{\prop{}}}}, or
  \item
        if
        \isfalseval{\val{}}
        then {\satisfies{\openv{}}{\elseprop{\prop{}}}}.
\end{itemize}
\begin{proof}
        By cases on the definition of \isaopsemliteral
        and subcases on \isaopsemliteral.

        \begin{itemize} % isaopsem
          \item[]
            \begin{subcase}[\isaopsem{\val{1}}{\val{1}} = {\true{}}, \text{if} \val{1} \notequal\ {\class{}}]
              \ 

              \val{1} = \val{2}, \val{1} \notequal\ {\class{}}, \val{2} \notequal\ {\class{}}, \istrueval{\val{}}
              
              Since \istrueval{\val{}} we prove {\satisfies{\openv{}}{\thenprop{\prop{}}}}
              by cases on the definition of \isacompareliteral{}:
              \begin{itemize} % isacompare
                \item[]
                  \begin{subcase}[\isacompare{\s{}}{\pth{\classpe{}}{\pth{\pathelem{}}{\x{}}}}{\Value{\class{}}}
                                 {\filterset{\isprop{\class{}} {\pth{\pathelem{}}{\x{}}}}
                                            {\notprop{\class{}}{\pth{\pathelem{}}{\x{}}}}}]
                    \ 


                    \object{1} = {\pth{\classpe{}}{\pth{\pathelem{}}{\x{}}}},
                    \ty{2} = {\Value{\class{}}},
                    \inpropenv{\isprop{\class{}} {\pth{\pathelem{}}{\x{}}}}{\thenprop{\prop{}}}

                    Unreachable by inversion on the typing relation, since \ty{2} = {\Value{\class{}}},
                    yet \val{2} \notequal\ {\class{}}.

%                    By inversion on the typing relation, since \classpe{} is the last path element of \object{1}
%                    then \opsem{\openv{}}{\appexp{\classconst{}}{\openv{}({\pth{\pathelem{}}{\x{}}})}}{\val{1}}.
%
%                    Since {\val{1}} = {\val{2}} then {\ty{1}} = {\ty{2}}, and because {\ty{2}} = {\Value{\class{}}}
%                    then {\ty{1}} = {\Value{\class{}}}.
%
%                    By inversion {\val{1}} = {\class{}}, via T-Class.
%
%                    Since {\opsem{\openv{}}{\appexp{\classconst{}}{\openv{}({\pth{\pathelem{}}{\x{}}})}}{\class{}}}
%                    we conclude by lemma~\ref{appendix:lemma:classconst}
%                    with {\satisfies{\openv{}}{\isprop{\class{}} {\pth{\pathelem{}}{\x{}}}}}.

                  \end{subcase}
                \item[]
                  \begin{subcase}[\isacompare{\s{}}{\object{}}{\Value{\singletonmeta{}}}
                    {\replacefor
                      {\filtersetparen{\isprop{\Value{\singletonmeta{}}} {\x{}}}
                        {\notprop{\Value{\singletonmeta{}}}{\x{}}}}
                      {\object{}}
                      {\x{}}}\ 
                    \text{if}\ {\singletonmeta{}} \notequal \class{}]
                    \ 

                    \ty{2} = {\Value{\singletonmeta{}}}, 
                    {\singletonmeta{}} \notequal \class{},
                    \inpropenv{\replacefor{\isprop{\Value{\singletonmeta{}}} {\x{}}}
                                           {\object{1}}
                                           {\x{}}}{\thenprop{\prop{}}}
                    %\elseprop{\prop{}} = {\replacefor{\notprop{\Value{\singletonmeta{}}} {\x{}}}
                    %                       {\object{1}}
                    %                       {\x{}}}

                    Since \ty{2} = {\Value{\singletonmeta{}}} where {\singletonmeta{}} \notequal \class{},
                    by inversion on the typing judgement 
                    {\val{2}} is either \true{}, \false{}, \nil{} or \kw{}
                    by T-True, T-False, T-Nil or T-Kw.

                    Since \val{1} = {\val{2}} then \ty{1} = \ty{2}, and since \ty{2} = {\Value{\singletonmeta{}}}
                      then \ty{1} = {\Value{\singletonmeta{}}}, so
                    \judgementtwo {} {\val{1}} {\Value{\singletonmeta{}}}

                    If \object{1} = \emptyobject{} then \thenprop{\prop{}} = \topprop{} and
                    we derive
                    {\satisfies{\openv{}}{\topprop{}}} with M-Top.

                    Otherwise \object{1} = \pth{\pathelem{}}{\x{}} and 
                    \inpropenv{\isprop{\Value{\singletonmeta{}}}{\pth{\pathelem{}}{\x{}}}}{\thenprop{\prop{}}},
                    and since
                    \judgementtwo {} {\val{1}} {\Value{\singletonmeta{}}}
                    then
                    \judgementtwo {} {{\openv{}}(\pth{\pathelem{}}{\x{}})} {\Value{\singletonmeta{}}},
                    which we can use M-Type to derive
                    {\satisfies{\openv{}}{\isprop{\Value{\singletonmeta{}}}{\pth{\pathelem{}}{\x{}}}}}.
                  \end{subcase}
                \item[]
                  \begin{subcase}[\isacompare{\s{}}{\object{}}{\ty{}} {\filterset{\topprop{}} {\topprop{}}}]
                    \ 

                    {\thenprop{\prop{}}} = {\topprop{}}

                    {\satisfies{\openv{}}{\topprop{}}} holds by M-Top.

                  \end{subcase}
              \end{itemize}
            \end{subcase}
          \item[]
            \begin{subcase}[\isaopsem{\class{1}}{\class{2}} = {\true{}}, \text{if}\ \issubtypein{}{\class{1}}{\class{2}}]
              \ 

              \val{1} = \class{1}, \val{2} = \class{2},
              \issubtypein{}{\class{1}}{\class{2}},
              \istrueval{\val{}}
              
              Since \istrueval{\val{}} we prove {\satisfies{\openv{}}{\thenprop{\prop{}}}}
              by cases on the definition of \isacompareliteral{}:
              \begin{itemize} % isacompare
                \item[]
                  \begin{subcase}[\isacompare{\s{}}{\pth{\classpe{}}{\pth{\pathelem{}}{\x{}}}}{\Value{\class{}}}
                                 {\filterset{\isprop{\class{}} {\pth{\pathelem{}}{\x{}}}}
                                            {\notprop{\class{}}{\pth{\pathelem{}}{\x{}}}}}]
                    \ 


                    \object{1} = {\pth{\classpe{}}{\pth{\pathelem{}}{\x{}}}},
                    \ty{2} = {\Value{\class{2}}},
                    \inpropenv{\isprop{\class{2}} {\pth{\pathelem{}}{\x{}}}}{\thenprop{\prop{}}}

                    By inversion on the typing relation, since \classpe{} is the last path element of \object{1}
                    then \opsem{\openv{}}{\appexp{\classconst{}}{\openv{}({\pth{\pathelem{}}{\x{}}})}}{\val{1}}.

                    Since {\opsem{\openv{}}{\appexp{\classconst{}}{\openv{}({\pth{\pathelem{}}{\x{}}})}}{\class{1}}},
                    as {\val{1}} = {\class{1}},
                    we can derive from lemma~\ref{appendix:lemma:classconst}
                    {\satisfies{\openv{}}{\isprop{\class{1}} {\pth{\pathelem{}}{\x{}}}}}.

                    By the induction hypothesis we can derive 
                    {\inpropenv{\propenv{}}{\isprop{\class{1}} {\pth{\pathelem{}}{\x{}}}}},
                    and with the fact {\issubtypein{}{\class{1}}{\class{2}}}
                    we can use L-Sub to conclude 
                    {\inpropenv{\propenv{}}{\isprop{\class{2}} {\pth{\pathelem{}}{\x{}}}}},
                    and finally by lemma~\ref{appendix:lemma:satisfies}
                    we derive
                    {\satisfies{\openv{}}{\isprop{\class{2}} {\pth{\pathelem{}}{\x{}}}}}.

                  \end{subcase}
                \item[]
                  \begin{subcase}[\isacompare{\s{}}{\object{}}{\Value{\singletonmeta{}}}
                    {\replacefor
                      {\filtersetparen{\isprop{\Value{\singletonmeta{}}} {\x{}}}
                        {\notprop{\Value{\singletonmeta{}}}{\x{}}}}
                      {\object{}}
                      {\x{}}}\ 
                    \text{if}\ {\singletonmeta{}} \notequal \class{}]
                    \ 

                    \ty{2} = {\Value{\singletonmeta{}}}, 
                    {\singletonmeta{}} \notequal \class{},
                    \inpropenv{\replacefor{\isprop{\Value{\singletonmeta{}}} {\x{}}}
                                           {\object{1}}
                                           {\x{}}}{\thenprop{\prop{}}}

                    Unreachable case since 
                    \ty{2} = {\Value{\singletonmeta{}}} where 
                    {\singletonmeta{}} \notequal \class{},
                    but \val{2} = \class{2}.
                  \end{subcase}
                \item[]
                  \begin{subcase}[\isacompare{\s{}}{\object{}}{\ty{}} {\filterset{\topprop{}} {\topprop{}}}]
                    \ 

                    {\thenprop{\prop{}}} = {\topprop{}}

                    {\satisfies{\openv{}}{\topprop{}}} holds by M-Top.

                  \end{subcase}
              \end{itemize}
            \end{subcase}
          \item[]
            \begin{subcase}[\isaopsem{\val{1}}{\val{2}} = {\false{}}, otherwise]
              \ 

              \val{1} \notequal\ \val{2},
              \isfalseval{\val{}}
              
              Since \isfalseval{\val{}} we prove {\satisfies{\openv{}}{\elseprop{\prop{}}}}
              by cases on the definition of \isacompareliteral{}:
              \begin{itemize} % isacompare
                \item[]
                  \begin{subcase}[\isacompare{\s{}}{\pth{\classpe{}}{\pth{\pathelem{}}{\x{}}}}{\Value{\class{}}}
                                 {\filterset{\isprop{\class{}} {\pth{\pathelem{}}{\x{}}}}
                                            {\notprop{\class{}}{\pth{\pathelem{}}{\x{}}}}}]
                    \ 


                    \object{1} = {\pth{\classpe{}}{\pth{\pathelem{}}{\x{}}}},
                    \ty{2} = {\Value{\class{}}},
                    \inpropenv{\notprop{\class{}} {\pth{\pathelem{}}{\x{}}}}{\elseprop{\prop{}}}

                    By inversion on the typing relation, since \classpe{} is the last path element of \object{1}
                    then \opsem{\openv{}}{\appexp{\classconst{}}{\openv{}({\pth{\pathelem{}}{\x{}}})}}{\val{1}}.
                    
                    By the definition of {\classconst{}} either {\val{1}} = {\class{}} or {\val{1}} = \nil{}.

                    If {\val{1}} = \nil{}, then we know from the definition of \isaopsemliteral that 
                    {\openv{}({\pth{\pathelem{}}{\x{}}})} = \nil{}.

                    Since \judgementtwo{}{\openv{}({\pth{\pathelem{}}{\x{}}})}{\Nil},
                    and there is no \val{1} such that both \judgementtwo{}{\openv{}({\pth{\pathelem{}}{\x{}}})}{\class} and
                    \judgementtwo{}{\openv{}({\pth{\pathelem{}}{\x{}}})}{\Nil{}},
                    we use M-NotType to derive 
                    \satisfies{\openv{}}{\notprop{\class{}} {\pth{\pathelem{}}{\x{}}}}.

                    Similarly if {\val{1}} = \class{1}, by the definition of \isacompareliteral
                    we know {\notsubtypein{}{\class{1}}{\class{}}} and 
                    {\openv{}({\pth{\pathelem{}}{\x{}}})} = \class{1}.

                    Since \judgementtwo{}{\openv{}({\pth{\pathelem{}}{\x{}}})}{\class{1}},
                    and there is no \val{1} such that both 
                    \judgementtwo{}{\val{1}}{\class{}} and
                    \judgementtwo{}{\val{1}}{\class{1}},
                    we use M-NotType to derive 
                    \satisfies{\openv{}}{\notprop{\class{}} {\pth{\pathelem{}}{\x{}}}}.


                  \end{subcase}
                \item[]
                  \begin{subcase}[\isacompare{\s{}}{\object{}}{\Value{\singletonmeta{}}}
                    {\replacefor
                      {\filtersetparen{\isprop{\Value{\singletonmeta{}}} {\x{}}}
                        {\notprop{\Value{\singletonmeta{}}}{\x{}}}}
                      {\object{}}
                      {\x{}}}\ 
                    \text{if}\ {\singletonmeta{}} \notequal \class{}]
                    \ 

                    \ty{2} = {\Value{\singletonmeta{}}}, 
                    {\singletonmeta{}} \notequal \class{},
                    %\thenprop{\prop{}} = {\replacefor{\isprop{\Value{\singletonmeta{}}} {\x{}}}
                    %                       {\object{1}}
                    %                       {\x{}}}
                    \inpropenv{\replacefor{\notprop{\Value{\singletonmeta{}}} {\x{}}}
                                           {\object{1}}
                                           {\x{}}}{\elseprop{\prop{}}}

                    Since \ty{2} = {\Value{\singletonmeta{}}} where {\singletonmeta{}} \notequal \class{},
                    by inversion on the typing judgement 
                    {\val{2}} is either \true{}, \false{}, \nil{} or \kw{}
                    by T-True, T-False, T-Nil or T-Kw.

                    If \object{1} = \emptyobject{} then \elseprop{\prop{}} = \topprop{} and
                    we derive
                    {\satisfies{\openv{}}{\topprop{}}} with M-Top.

                    Otherwise \object{1} = \pth{\pathelem{}}{\x{}} and 
                    \inpropenv{\notprop{\Value{\singletonmeta{}}}{\pth{\pathelem{}}{\x{}}}}{\elseprop{\prop{}}}.
                    Noting that \val{1} \notequal\ \val{2},
                    we know \judgementtwo{}{\openv{}({\pth{\pathelem{}}{\x{}}})}{\s{}}
                    where \s{} \notequal\ {\Value{\singletonmeta{}}},
                    and there is no \val{1} such that both 
                    \judgementtwo{}{\val{1}}{\Value{\singletonmeta{}}} and
                    \judgementtwo{}{\val{1}}{\s{}}
                    so we can use M-NotType to derive
                    {\satisfies{\openv{}}{\notprop{\Value{\singletonmeta{}}}{\pth{\pathelem{}}{\x{}}}}}.
                  \end{subcase}
                \item[]
                  \begin{subcase}[\isacompare{\s{}}{\object{}}{\ty{}} {\filterset{\topprop{}} {\topprop{}}}]
                    \ 

                    {\elseprop{\prop{}}} = {\topprop{}}

                    {\satisfies{\openv{}}{\topprop{}}} holds by M-Top.

                  \end{subcase}
              \end{itemize}
            \end{subcase}
        \end{itemize}
      \end{proof}
\end{lemma}

\begin{lemma} \label{appendix:lemma:soundness}
{\soundnesslemmahypothesis}
\begin{proof}
By induction and cases on the derivation of \opsem {\openv{}} {\e{}} {\a{}},
and subcases on the penultimate rule of the derivation of
\judgementrewrite{\propenv{}}{\ep{}}{\ty{}}{\filterset{\thenprop{\prop{}}}{\elseprop{\prop{}}}}{\object{}}{\e{}}
followed by T-Subsume as the final rule.

% induction on the derivation of the evaluation semantics because we want to apply
% the induction hypothesis to subderivations of the eval sem. If eg. used the typing
% judgement, we couldn't use the induction hypothesis on applications of higher-order
% functions, since the subderivation of T-Abs wouldn't be present.

\begin{case}[B-Val]

  \begin{itemize}
    \item[] 
      \begin{subcase}[T-True]
        \val{} = \true{},
  \ep{} = \true{},
  \e{} = \true{}, \issubtypein{}{\True}{\ty{}}, \inpropenv{\topprop{}}{\thenprop{\prop{}}}, 
  \inpropenv{\botprop{}}{\elseprop{\prop{}}}, \issubtypein{}{\emptyobject{}}{\object{}}

        Proving part 1 is trivial: \object{} is a superobject of \emptyobject{}, which can only be \emptyobject{}.

        To prove part 2, we note that \val{} = \true{}
        and \inpropenv{\topprop{}}{\thenprop{\prop{}}},
        so \satisfies{\openv{}}{\thenprop{\prop{}}} by M-Top.

        Part 3 holds as \e{} can only be reduced to itself via B-Val.

        Part 4 holds vacuously.
      \end{subcase}
    \item[]
      \begin{subcase}[T-HMap] \val{} = {\curlymapvaloverright{\val{k}}{\val{v}}},
  \ep{} = {\curlymapvaloverright{\val{k}}{\val{v}}},
  \e{} = {\curlymapvaloverright{\val{k}}{\val{v}}},
  \issubtypein{}{\HMapc {\mandatory{}}}{\ty{}},
  \inpropenv{\topprop{}}{\thenprop{\prop{}}},
  \inpropenv{\botprop{}}{\elseprop{\prop{}}},
  \issubtypein{}{\emptyobject{}}{\object{}},
  $\overrightarrow{\judgementtwo {} {\val{k}}{\Value \kw{}}}$,
  $\overrightarrow{\judgementtwo {} {\val{v}}{\ty{v}}}$,
  \mandatory{} = \mandatorysetoverright{\kw{}}{\ty{v}}

        Similar to T-True.

        Part 4 holds by the induction hypothese on {\overr{\val{k}}} and {\overr{\val{v}}}.
      \end{subcase}
    \item[]
      \begin{subcase}[T-Kw] \val{} = {\kw{}},
  \ep{} = {\kw{}},
  \e{} = {\kw{}},
  \issubtypein{}{\Value{\kw{}}}{\ty{}},
  \inpropenv{\topprop{}}{\thenprop{\prop{}}},
  \inpropenv{\botprop{}}{\elseprop{\prop{}}},
  \issubtypein{}{\emptyobject{}}{\object{}}

        Similar to T-True.
      \end{subcase}
      \begin{subcase}[T-Str]
        Similar to T-Kw.
      \end{subcase}
  \item[] 
    \begin{subcase}[T-False]
      \val{} = \false{},
\ep{} = \false, 
\e{} = \false, 
\issubtypein{}{\False}{\ty{}},
\inpropenv{\botprop{}}{\thenprop{\prop{}}},
\inpropenv{\topprop{}}{\elseprop{\prop{}}},
\issubtypein{}{\emptyobject{}}{\object{}}

Proving part 1 is trivial: \object{} is a superobject of \emptyobject{}, which must be \emptyobject{}. 

To prove part 2, we note that \val{} = \false{}
and \inpropenv{\topprop{}}{\elseprop{\prop{}}}, so \satisfies{\openv{}}{\elseprop{\prop{}}} by M-Top. 

Part 3 holds as \e{} can only be reduced to itself via B-Val.

Part 4 holds vacuously.
\end{subcase}
    \item[]
      \begin{subcase}[T-Class] \val{} = {\class{}},
  \ep{} = {\class{}},
  \e{} = {\class{}},
  \issubtypein{}{\Value{\class{}}}{\ty{}},
  \inpropenv{\topprop{}}{\thenprop{\prop{}}},
  \inpropenv{\botprop{}}{\elseprop{\prop{}}},
  \issubtypein{}{\emptyobject{}}{\object{}}

        Similar to T-True.
      \end{subcase}
    \item[]
      \begin{subcase}[T-Instance]
        \val{} = {\classvalue{\classhint{}} {\overrightarrow {\classfieldpair{\fld{i}} {\val{i}}}}},
        \ep{} = {\classvalue{\classhint{}} {\overrightarrow {\classfieldpair{\fld{}} {\val{}}}}},
        \e{} = {\classvalue{\classhint{}} {\overrightarrow {\classfieldpair{\fld{}} {\val{}}}}},
        \issubtypein{}{\class{}}{\ty{}},
        \inpropenv{\topprop{}}{\thenprop{\prop{}}},
        \inpropenv{\botprop{}}{\elseprop{\prop{}}},
        \issubtypein{}{\emptyobject{}}{\object{}}


        Similar to T-True.

        Part 4 holds by the induction hypotheses on ${\overrightarrow{\val{i}}}$.
      \end{subcase}
  \item[] 
    \begin{subcase}[T-Nil] 
      \val{} = \nil{},
\ep{} = \nil, 
\e{} = \nil, 
\issubtypein{}{\Nil}{\ty{}},
\inpropenv{\botprop{}}{\thenprop{\prop{}}},
\inpropenv{\topprop{}}{\elseprop{\prop{}}},
\issubtypein{}{\emptyobject{}}{\object{}}

      Similar to T-False.
\end{subcase}
\item[]
\begin{subcase}[T-Multi] 
  \val{} = {\multi {\val{1}} {\curlymapvaloverright{\val{k}}{\val{v}}}}
  \ep{} = {\multi {\val{1}} {\curlymapvaloverright{\val{k}}{\val{v}}}},
  \judgementtworewrite {} {\val{1}} {\ty{1}}{\val{1}},
  \overr{\judgementtworewrite{}{\val{k}}{\Top{}}{\val{k}}},
  \overr{\judgementtworewrite{}{\val{v}}{\s{}}{\val{v}}},
  \e{} = {\multi {\val{1}} {\curlymapvaloverright{\val{k}}{\val{v}}}},
  \issubtypein{}{\MultiFntype {\s{}} {\ty{1}}}{\ty{}},
  \inpropenv{\topprop{}}{\thenprop{\prop{}}},
  \inpropenv{\botprop{}}{\elseprop{\prop{}}},
  \issubtypein{}{\emptyobject{}}{\object{}}

        Similar to T-True.
\end{subcase}
\item[]
\begin{subcase}[T-Const]
  \e{} = {\const{}},
  \issubtypein{}{\constanttype{\const{}}}{\ty{}},
  \inpropenv{\topprop{}}{\thenprop{\prop{}}},
  \inpropenv{\botprop{}}{\elseprop{\prop{}}},
  \issubobjin{}{\emptyobject{}}{\object{}}

        Parts 1, 2 and 3 hold for the same reasons as T-True. 
\end{subcase}


  \end{itemize}
\end{case}



\begin{case}[B-Local]
{ \inopenv {\openv{}} {\x{}} {\val{}} },
{ \opsem {\openv{}} {\x{}} {\val{}} }

\begin{itemize}
  \item[]
\begin{subcase}[T-Local]
  \ep{} = \x{}, 
  \e{} = \x{}, 
  \inpropenv{\notprop {\falsy{}} {\x{}}}{\thenprop{\prop{}}},
  \inpropenv{\isprop {\falsy{}} {\x{}}}{\elseprop{\prop{}}},
\issubtypein{}{\x{}}{\object{}},
\inpropenv{\propenv{}}{\isprop{\ty{}}{\x{}}}

Part 1 follows from \inopenv{\openv{}}{\object{}} {\val{}}, since either {\object{}} = \x{}
and \inopenv{\openv{}}{\x{}} {\val{}} is a premise of B-Local, or {\object{}} = {\emptyobject{}} which also
satisfies the goal.

Part 2 considers two cases: if \istrueval{\val{}}, then 
\satisfies{\openv{}}{\notprop{\falsy}{\x{}}} holds by M-NotType; if \isfalseval{\val{}}, then 
\satisfies{\openv{}}{\isprop{\falsy}{\x{}}} holds by M-Type.

We prove part 3 by observing
\inpropenv{\propenv{}}{\isprop{\ty{}}{\x{}}},
\satisfies{\openv{}}{\propenv{}},
and
\inopenv {\openv{}} {\x{}} {\val{}}
(by B-Local)
which gives us the desired result.

Part 4 holds vacuously.
\end{subcase}
\end{itemize}

\end{case}

\begin{case}[B-Do]
  \opsem {\openv{}} {\e{1}} {\val{1}},
  \opsem {\openv{}} {\e{2}} {\val{}}

\begin{itemize}
  \item[] \begin{subcase}[T-Do]
      \ep{} = {\doexp {\ep{1}} {\ep{2}}},
  \judgementrewrite {\propenv{}} 
             {\ep{1}} {\ty{1}}
             {\filterset {\thenprop {\prop{1}}} {\elseprop {\prop1}}} 
             {\object{1}}
             {\e{1}},
\judgementrewrite {\propenv{}, {\orprop {\thenprop {\prop{1}}} {\elseprop {\prop{1}}}}}
           {\ep{}} {\ty{}}
           {\filterset {\thenprop {\prop{}}} {\elseprop {\prop{}}}} 
           {\object{}}
           {\e{}},
    \e{} = {\doexp {\e{1}} {\e{2}}}

For all parts we note 
    since {\e{1}} can be either a true or false value
    then
    {\satisfies{\openv{}}{{\propenv{}},{\orprop {\thenprop {\prop{1}}} {\elseprop {\prop{1}}}}}}
    by M-Or,
    which together with 
\judgement {\propenv{}, {\orprop {\thenprop {\prop{1}}} {\elseprop {\prop{1}}}}}
           {\e{2}} {\ty{}}
           {\filterset {\thenprop {\prop{}}} {\elseprop {\prop{}}}} 
           {\object{}},
    and
  \opsem {\openv{}} {\e{2}} {\val{}}
    allows us to apply the induction hypothesis on \e{2}.

To prove part 1 we use the induction hypothesis on \e{2}
to show either \object{} = \emptyobject{} 
or \inopenv {\openv{}} {\object{}} {\val{}}, since \e{} always
evaluates to the result of \e{2}.

For part 2 we use the induction hypothesis on \e{2}
to show if \istrueval{\val{}} then
        {\satisfies{\openv{}}{\thenprop{\prop{}}}}
        or
  if \isfalseval{\val{}} then
        {\satisfies{\openv{}}{\elseprop{\prop{}}}}.

Parts 3 and 4 follow from the induction hypothesis on \e{2}.
    \end{subcase}
\end{itemize}
\end{case}

\begin{case}[BE-Do1]
  \opsem {\openv{}} {\e{1}} {\errorval{\val{e}}},
  \opsem {\openv{}} {\e{}} {\errorval{\val{}}}

        Trivially reduces to an error.
\end{case}

\begin{case}[BE-Do2]
  \opsem {\openv{}} {\e{1}} {\val{1}},
  \opsem {\openv{}} {\e{2}} {\errorvalv{}},
  \opsem {\openv{}} {\e{}} {\errorvalv{}}

        As above.
\end{case}

\begin{case}[B-New]
  $
  \overrightarrow{
  \opsem {\openv{}}
         {\e{i}}
         {\val{i}}
       }$,
         $\newjava {\classhint{1}}
                  {\overrightarrow{\classhint{i}}}
                  {\overrightarrow{\val{i}}}
                  {\val{}}$

\begin{itemize}
  \item[]
\begin{subcase}[T-New]
  \ep{} = {\newexp {\class{}} {\overrightarrow{\ep{i}}}},
  \inct{\ctctorentry{\overr{\class{i}}}}{\ctlookupctors{\ct{}}{\classhint{}}},
  \overr{\javatotcnil{\classhint{i}}{\ty{i}}},
  \overr{
  \judgementtworewrite {\propenv{}}
                    {\ep{i}} {\ty{i}}
                    {\e{i}}
                  },
  \e{} = {\newstaticexp {\overrightarrow{\classhint{i}}} {\classhint{}} 
                                                          {\class{}} {\overrightarrow{\e{i}}}},
  \issubtypein{}{\javatotcexp{\classhint{}}}{\ty{}},
  \inpropenv{\topprop{}}{\thenprop{\prop{}}},
  \inpropenv{\botprop{}}{\elseprop{\prop{}}},
  \issubobjin{}{\emptyobject{}}{\object{}}

Part 1 follows \object{} = \emptyobject{}.

Part 2 requires some explanation. The two false values in Typed Clojure
cannot be constructed with \newliteral{}, so the only case is \val{} $\not=$ \false\ (or \nil)
where \thenprop{\prop{}} = \topprop{} so \satisfies{\openv{}}{\thenprop{\prop{}}}.
\Void{} also lacks a constructor.

Part 3 holds as B-New reduces to a \emph{non-nilable}
instance of \class{} via \newjavaliteral (by assumption~\ref{appendix:assumption:new}), 
and {\ty{}} is a supertype of \javatotcexp{\classhint{}}.

\end{subcase}
\item[]
\begin{subcase}[T-NewStatic]
  {\ep{}} = {\newstaticexp {\overrightarrow{\classhint{i}}} {\classhint{}}
                                                          {\class{}} {\overrightarrow{\e{i}}}}

  Non-reflective constructors cannot be written directly by the user, so we can assume
  the class information attached to the syntax corresponds to an actual constructor by inversion
  from T-New.

  The rest of this case progresses like T-New.
\end{subcase}
\end{itemize}
\end{case}

\begin{case}[BE-New1] $\overrightarrow{
  \opsem {\openv{}}
         {\e{i-1}}
         {\val{i-1}}
       }$,
  \opsem {\openv{}}
         {\e{i}}
         {\errorvalv{}},
  \opsem {\openv{}} {\e{}} {\errorvalv{}}

        Trivially reduces to an error.

\end{case}

\begin{case}[BE-New2] 
  $\overrightarrow{
  \opsem {\openv{}}
         {\e{i}}
         {\val{i}}
       }$,
         \newjava {\classhint{1}}
                  {\overrightarrow{\classhint{i}}}
                  {\overrightarrow{\val{i}}}
                  {\errorvalv{}},
        \opsem {\openv{}} {\e{}} {\errorvalv{}}

        As above.

\end{case}

\begin{case}[B-Field]
  \opsem {\openv{}}
         {\e{1}} 
         {\classvalue{\classhint{1}} {\classfieldpair{\fld{}} {\val{}}}}

\begin{itemize}
  \item[]
\begin{subcase}[T-Field]
  \ep{} = {\fieldexp {\fld{}} {\ep{1}}},
  \judgementtworewrite {\propenv{}} {\ep{}} {\s{}} {\e{}},
  \issubtypein{}{\s{}}{\Object{}},
  \tctojava{\s{}}{\classhint{1}},
  \inct{\ctfldentry{\fld{}}{\classhint{2}}}{\ctlookupfields{\ct{}}{\classhint{1}}},
  \e{} = {\fieldstaticexp {\classhint{1}} {\classhint{2}} {\fld{}} {\e{1}}}
  \issubtypein{}{\javatotcnilexp{\classhint{2}}}{\ty{}},
  \inpropenv{\topprop{}}{\thenprop{\prop{}}},
  \inpropenv{\topprop{}}{\elseprop{\prop{}}},
  \issubobjin{}{\emptyobject{}}{\object{}}


Part 1 is trivial as \object{} is always \emptyobject{}.

Part 2 holds trivially; \val{} can be either a true or false value
and both {\thenprop{\prop{}}} and {\elseprop{\prop{}}}
are \topprop{}.

Part 3 relies on the semantics of \getfieldliteral (assumption~\ref{appendix:assumption:field})
in B-Field, which returns a \emph{nilable} instance of \classhint{2},
and \ty{} is a supertype of \javatotcnilexp{\classhint{2}}.
Notice \issubtypein{}{\s{}}{\Object{}} is required to guard from dereferencing \nil{},
as {\classhint{1}} erases occurrences of \Nil{} in \s{} via  \tctojava{\s{}}{\classhint{1}}.
\end{subcase}
  \item[]

\begin{subcase}[T-FieldStatic]
  {\ep{}} = {\fieldstaticexp {\classhint{1}} {\classhint{2}} {\fld{}} {\e{1}}}

  Non-reflective field lookups cannot be written directly by the user, so we can assume
  the class information attached to the syntax corresponds to an actual field by inversion
  from T-Field.

  The rest of this case progresses like T-Field.
\end{subcase}

\end{itemize}
\end{case}

\begin{case}[BE-Field]
  \opsem {\openv{}}
         {\e{1}} 
         {\errorvalv{}},
  \opsem {\openv{}}
         {\e{}}
         {\errorvalv{}}

         Trivially reduces to an error.

\end{case}

\begin{case}[B-Method]
  \opsem {\openv{}}
         {\e{m}}
         {\val{m}},
  $\overrightarrow{
  \opsem {\openv{}}
         {\e{a}}
         {\val{a}}}$,
  \invokejavamethod {\classhint{1}} {\val{m}} {mth}
                    {\overrightarrow{\classhint{a}}} {\overrightarrow{\val{a}}}
                    {\classhint{2}}
                    {\val{}}

\begin{itemize}
  \item[]
\begin{subcase}[T-Method]
  \judgementtworewrite {\propenv{}} {\ep{}} {\s{}} {\e{}},
             \issubtypein{}{\s{}}{\Object{}},
  \tctojava{\s{}}{\classhint{1}},
                  \inct{\ctmthentry{\mth{}}{\overrightarrow{\classhint{i}}}{\classhint{2}}}{\ctlookupmethods{\ct{}}{\classhint{1}}},
                  \overr{\javatotcnil{\classhint{i}}{\ty{i}}},
             \overr{
  \judgementtworewrite {\propenv{}} {\ep{i}} {\ty{i}} {\e{i}}
                  },
  \e{} = {\methodstaticexp {\classhint{1}} 
                          {\overr {\classhint{i}}} 
                          {\classhint{2}}
                          {\mth{}} {\e{m}} {\overr{\e{a}}}},
                        \issubtypein{}{\javatotcnilexp{\classhint{2}}}{\ty{}},
  \inpropenv{\topprop{}}{\thenprop{\prop{}}},
  \inpropenv{\topprop{}}{\elseprop{\prop{}}},
  \issubobjin{}{\emptyobject{}}{\object{}}


Part 1 is trivial as \object{} is always \emptyobject{}.

Part 2 holds trivially, \val{} can be either a true or false value
and both {\thenprop{\prop{}}} and {\elseprop{\prop{}}}
are \topprop{}.

Part 3 relies on the semantics of \invokejavamethodliteral (assumption~\ref{appendix:assumption:method})
in B-Method, which returns a \emph{nilable} instance of \classhint{2},
and \ty{} is a supertype of \javatotcnil{\classhint{2}}.
Notice \issubtypein{}{\s{}}{\Object{}} is required to guard from dereferencing \nil{},
as {\classhint{1}} erases occurrences of \Nil{} in \s{} via  \tctojava{\s{}}{\classhint{1}}.
\end{subcase}
\item[]
\begin{subcase}[T-MethodStatic]
  \ep{} = 
  {\methodstaticexp {\classhint{1}} 
        {\overrightarrow {\classhint{i}}} 
        {\classhint{2}}
        {\mth{}} {\e{1}} {\overrightarrow{\e{i}}}}

  Non-reflective method invocations cannot be written directly by the user, so we can assume
  the class information attached to the syntax corresponds to an actual method by inversion
  from T-Method.

  The rest of this case progresses like T-Method.
\end{subcase}


\end{itemize}

\end{case}

\begin{case}[BE-Method1]
  \opsem {\openv{}}
         {\e{m}}
         {\errorval{\val{}}},
  \opsem {\openv{}}
         {\e{}}
         {\errorval{\val{}}}

         Trivially reduces to an error.
\end{case}
\begin{case}[BE-Method2]
  \opsem {\openv{}}
         {\e{m}}
         {\val{m}},
 $\overrightarrow{
  \opsem {\openv{}}
         {\e{n-1}}
         {\val{n-1}}
       }$,
  \opsem {\openv{}}
         {\e{n}}
         {\errorval{\val{}}},
  \opsem {\openv{}}
         {\e{}}
         {\errorval{\val{}}}

  As above.
\end{case}
\begin{case}[BE-Method3]
  \opsem {\openv{}}
         {\e{m}}
         {\val{m}},
  $\overrightarrow{
  \opsem {\openv{}}
         {\e{a}}
         {\val{a}}
       }$,
  \invokejavamethod {\classhint{1}} {\val{m}} {mth}
                    {\overrightarrow{\classhint{a}}} {\overrightarrow{\val{a}}}
                    {\classhint{2}}
                    {\errorvalv{}},
  \opsem {\openv{}} {\e{}} {\errorvalv{}}

  As above.

\end{case}

\begin{case}[B-DefMulti]
  \val{} = {\multi {\val{d}} {\emptydisptable}},
  \opsem {\openv{}} {\e{d}} {\val{d}}



\begin{itemize}
  \item[]
\begin{subcase}[T-DefMulti]
  \ep{} = {\createmultiexp {\s{}} {\ep{d}}},
  \s{} = {\ArrowOne {\x{}} {\ty{1}} {\ty{2}}
                          {\filterset {\thenprop {\prop{1}}}
                                      {\elseprop {\prop{1}}}}
                          {\object{1}}},
  \ty{d} = {\ArrowOne {\x{}} {\ty{1}} {\ty{3}}
                          {\filterset {\thenprop {\prop{2}}}
                                      {\elseprop {\prop{2}}}}
                          {\object{2}}},
\judgementtworewrite {\propenv{}} {\ep{}} {\sp{}} {\e{}},
  \e{} = {\createmultiexp {\s{}} {\e{d}}},
  \issubtypein{}{\MultiFntype {\s{}} {\ty{d}}}{\ty{}},
  \inpropenv{\topprop{}}{\thenprop{\prop{}}},
  \inpropenv{\botprop{}}{\elseprop{\prop{}}},
  \issubobjin{}{\emptyobject{}}{\object{}}


Part 1 and 2 hold for the same reasons as T-True.
For part 3 we show \judgementtwo{}{\multi {\val{d}} {\emptydisptable}}{\MultiFntype {\s{}} {\ty{d}}}
by T-Multi, since \judgementtwo {} {\val{d}} {\ty{d}} by the inductive hypothesis on {\e{d}}
and {\emptydisptable} vacuously satisfies the other premises of T-Multi, so we are done.

\end{subcase}
\end{itemize}
\end{case}

\begin{case}[BE-DefMulti] \opsem {\openv{}} {\e{d}} {\errorvalv{}},
        \opsem {\openv{}} {\e{}} {\errorvalv{}}

        Trivially reduces to an error.

\end{case}

\begin{case}[B-DefMethod]

        \ 

        \begin{enumerate}
          \item
       \val{} = {\multi {\val{d}} {\disptablep{}}},
          \item
        \opsem {\openv{}}
               {\e{m}}
               {\multi {\val{d}} {\disptable{}}},
          \item
  \opsem {\openv{}}
         {\e{v}}
         {\val{v}},
          \item
  \opsem {\openv{}}
         {\e{f}}
         {\val{f}},
          \item
         \disptablep{} = {\extenddisptable {\disptable{}} 
                                {\val{v}}
                                {\val{f}}}
        \end{enumerate}

  \begin{itemize}
    \item[]
      \begin{subcase}[T-DefMethod]
        \ 
        
        \begin{enumerate}[resume]

          \item
  \ep{} = {\extendmultiexp {\ep{m}} {\ep{v}} {\ep{f}}},
          \item
  \ty{m} = {\ArrowOne {\x{}} {\ty{1}} {\s{}}
                     {\filterset {\thenprop {\prop{m}}}
                                 {\elseprop {\prop{m}}}}
                     {\object{m}}},
          \item
  \ty{d} = {\ArrowOne {\x{}} {\ty{1}} {\sp{}}
                     {\filterset {\thenprop {\prop{d}}}
                                 {\elseprop {\prop{d}}}}
                     {\object{d}}},
          \item
\judgementtworewrite {\propenv{}}
                  {\ep{m}} {\MultiFntype {\ty{m}} {\ty{d}}}
                  {\e{m}}
          \item
  \isacompare{\sp{}}{\object{d}}{\ty{v}}{\filterset {\thenprop {\prop{i}}} {\elseprop {\prop{i}}}},
          \item
\judgementtworewrite {\propenv{}}
           {\e{v}} {\ty{v}}
           {\e{v}}
          \item
  \judgementrewrite {\propenv{}, {\isprop{\ty{1}} {\x{}}}, {\thenprop {\prop{i}}}}
           {\ep{f}} {\s{}}
           {\filterset {\thenprop {\prop{m}}}
                       {\elseprop {\prop{m}}}}
           {\object{m}}
           {\e{f}}
          \item
  \e{} = {\extendmultiexp {\e{m}} {\e{v}} {\e{f}}},
          \item
  \e{f} = {\abs {\x{}} {\ty{1}} {\e{b}}},
\item
  \issubtypein{}{\MultiFntype {\ty{m}} {\ty{d}}}{\ty{}},
\item
  \inpropenv{\topprop{}}{\thenprop{\prop{}}},
\item
  \inpropenv{\botprop{}}{\elseprop{\prop{}}},
\item
  \issubobjin{}{\emptyobject{}}{\object{}}
        \end{enumerate}

                                Part 1 and 2 hold for the same reasons as T-True, noting that the propositions
                                and object agree with T-Multi.

For part 3 we show
\judgementtwo{}{\multi {\val{d}} {\extenddisptable {\disptable{}}{\val{v}}{\val{f}}}}{\MultiFntype {\ty{m}} {\ty{d}}}
by noting \judgementtwo {} {\val{d}} {\ty{d}},
  \judgementtwo{}{\val{v}}{\Top{}}
  and
  \judgementtwo{}{\val{f}}{\ty{m}}, and since \disptable{} is in the correct form by the inductive
  hypothesis on {\e{m}} we can satisfy all premises of T-Multi, so we are done.


      \end{subcase}

  \end{itemize}
\end{case}

      \begin{case}[BE-DefMethod1]
        \opsem {\openv{}}
               {\e{m}}
               {\errorval{\val{}}},
        \opsem {\openv{}}
                  {\e{}}
                {\errorval{\val{}}}

                Trivially reduces to an error.

      \end{case}
      \begin{case}[BE-DefMethod2]
        \opsem {\openv{}}
         {\e{m}}
         {\multi {\val{d}} {\disptable{}}},
  \opsem {\openv{}}
         {\e{v}}
         {\errorval{\val{}}},
        \opsem {\openv{}}
                  {\e{}}
                {\errorval{\val{}}}

                Trivially reduces to an error.
      \end{case}
      \begin{case}[BE-DefMethod3]
        \opsem {\openv{}}
         {\e{m}}
         {\multi {\val{d}} {\disptable{}}},
  \opsem {\openv{}}
         {\e{v}}
         {\val{v}},
  \opsem {\openv{}}
         {\e{f}}
         {\errorval{\val{}}},
        \opsem {\openv{}}
                  {\e{}}
                {\errorval{\val{}}}

                Trivially reduces to an error.

      \end{case}

\begin{case}[B-BetaClosure]
  \ 

  \begin{itemize}
    \item
  \opsem{\openv{}}{\e{}}{\val{}},
    \item
  \opsem {\openv{}}
         {\e{1}}
         {\closure {\openv{c}} {\abs {\x{}} {\s{}} {\e{b}}}},
    \item
  \opsem {\openv{}}
         {\e{2}}
         {\val{2}},
    \item
  \opsem {\extendopenv {\openv{c}} {\x{}} {\val{2}}}
         {\e{b}}
         {\val{}}
     \end{itemize}


\begin{itemize}
  \item[]
\begin{subcase}[T-App]
  \ 

  \begin{itemize}
    \item
  \ep{} = {\appexp {\ep{1}} {\ep{2}}},
    \item
  \judgementrewrite {\propenv{}} {\ep{1}} {\ArrowOne {\x{}} {\s{}}
                                                       {\ty{f}}
                                                       {\filterset {\thenprop {\prop{f}}}
                                                                   {\elseprop {\prop{f}}}}
                                                       {\object{f}}}
                {\filterset {\thenprop {\prop{1}}}
                            {\elseprop {\prop{1}}}}
                {\object{1}}
                {\e{1}},
    \item
  \judgementrewrite {\propenv{}}
                 {\ep{2}} {\s{}}
                 {\filterset {\thenprop {\prop{2}}}
                             {\elseprop {\prop{2}}}}
                 {\object{2}}
                 {\e{2}},
    \item
  \e{} = {\appexp {\e{1}} {\e{2}}},
    \item
      \issubtypein{}  {\replacefor {\ty{f}} {\object{2}} {\x{}}}{\ty{}},
    \item
      \inpropenv{\replacefor {\thenprop {\prop{f}}} {\object{2}} {\x{}}} {\thenprop {\prop{}}},
    \item
      \inpropenv{\replacefor {\elseprop {\prop{f}}} {\object{2}} {\x{}}} {\elseprop {\prop{}}},
    \item
      \issubobjin{}{\replacefor {\object{f}} {\object{2}} {\x{}}} {\object{}}
  \end{itemize}

         By inversion on \e{1} from T-Clos
         there is some environment {\propenvc{}} such that
         \begin{itemize}
           \item
              \satisfies{\openv{c}}{\propenvc{}} and
            \item
  \judgement {\propenvc{}} {\abs {\x{}} {\s{}} {\e{b}}} {\ArrowOne {\x{}} {\s{}}
                                                       {\ty{f}}
                                                       {\filterset {\thenprop {\prop{f}}}
                                                                   {\elseprop {\prop{f}}}}
                                                       {\object{f}}}
                {\filterset {\thenprop {\prop{1}}}
                            {\elseprop {\prop{1}}}}
                {\object{1}},
         \end{itemize}
         and also by inversion on \e{1} from T-Abs
         \begin{itemize}
           \item
  { \judgementrewrite {\propenvc{}, {\isprop {\s{}} {\x{}}}}
              {\ep{b}} {\ty{f}}
               {\filterset {\thenprop {\prop{f}}}
                           {\elseprop {\prop{f}}}}
               {\object{f}}
               {\e{b}}}.
         \end{itemize}

          From 
          \begin{itemize}
            \item
              \satisfies{\openv{c}}{\propenvc{}},
            \item
  \judgementrewrite {\propenv{}}
                 {\ep{2}} {\s{}}
                 {\filterset {\thenprop {\prop{2}}}
                             {\elseprop {\prop{2}}}}
                 {\object{2}}
                 {\e{2}} and 
            \item
  \opsem {\openv{}}
         {\e{2}}
         {\val{2}},
     \end{itemize}
              we know (by substitution)
              \satisfies{\extendopenv {\openv{c}} {\x{}} {\val{2}}}{\propenvc{},{\isprop{\s{}}{\x{}}}}.

              We want to prove
        \judgementrewrite {\propenvc{}}
                          {\replacefor{\ep{b}}{\val{2}}{\x{}}}
                          {\replacefor{\ty{f}}{\object{2}}{\x{}}}
               {\replacefor{\filterset {\thenprop {\prop{f}}}
                                       {\elseprop {\prop{f}}}}{\object{2}}{\x{}}}
                          {\replacefor{\object{f}}{\object{2}}{\x{}}}
                          {\replacefor{\e{b}}{\val{2}}{\x{}}}, 
                          which can be justified by noting 
          \begin{itemize}
            \item
  \judgementtworewrite {\propenvc{},{\isprop{\s{}}{\x{}}}}{\ep{b}}{\ty{f}}{\e{b}},
            \item
  \judgementrewrite {\propenv{}}
                 {\ep{2}} {\s{}}
                 {\filterset {\thenprop {\prop{2}}}
                             {\elseprop {\prop{2}}}}
                 {\object{2}}
                 {\e{2}} and 
            \item
  \opsem {\openv{}}
         {\e{2}}
         {\val{2}}.
     \end{itemize}

     From the previous fact and \satisfies{\openv{c}}{\propenvc{}},
              we know
  \opsem {\openv{c}}
         {\replacefor{\e{b}}{\val{2}}{\x{}}}
         {\val{}}.

                    Noting that 
      \issubtypein{}  {\replacefor {\ty{f}} {\object{2}} {\x{}}}{\ty{}},
      \inpropenv{\replacefor {\thenprop {\prop{f}}} {\object{2}} {\x{}}} {\thenprop {\prop{}}},
      \inpropenv{\replacefor {\elseprop {\prop{f}}} {\object{2}} {\x{}}} {\elseprop {\prop{}}}
      and
      \issubobjin{}{\replacefor {\object{f}} {\object{2}} {\x{}}} {\object{}},
                    we can use
         \begin{itemize}
           \item
        \judgementrewrite {\propenvc{}}
                          {\replacefor{\ep{b}}{\val{2}}{\x{}}}
                          {\replacefor{\ty{f}}{\object{2}}{\x{}}}
               {\replacefor{\filterset {\thenprop {\prop{f}}}
                                       {\elseprop {\prop{f}}}}{\object{2}}{\x{}}}
                          {\replacefor{\object{f}}{\object{2}}{\x{}}}
                          {\replacefor{\e{b}}{\val{2}}{\x{}}}, 
           \item
              \satisfies{\openv{c}}{\propenvc{}},
           \item
\isconsistent{\openv{c}} (via induction hypothesis on {\ep{1}}), and
           \item 
  \opsem {\openv{c}}
         {\replacefor{\e{b}}{\val{2}}{\x{}}}
         {\val{}}.
         \end{itemize}
         to apply the induction hypothesis on {\replacefor{\ep{b}}{\val{2}}{\x{}}} and satisfy
         all conditions.

\end{subcase}
\end{itemize}
\end{case}

\begin{case}[B-Delta]
  \opsem {\openv{}} {\e{1}} {\const{}},
  \opsem {\openv{}} {\e{2}} {\val{2}},
  \constantopsem{\const{}}{\val{2}} = \val{}

\begin{itemize}
  \item[]
\begin{subcase}[T-App]
  \ 

  \begin{itemize}
    \item
  \ep{} = {\appexp {\ep{1}} {\ep{2}}},
    \item
  \judgementrewrite {\propenv{}} {\ep{1}} {\ArrowOne {\x{}} {\s{}}
                                                       {\ty{f}}
                                                       {\filterset {\thenprop {\prop{f}}}
                                                                   {\elseprop {\prop{f}}}}
                                                       {\object{f}}}
                {\filterset {\thenprop {\prop{1}}}
                            {\elseprop {\prop{1}}}}
                {\object{1}}
                {\e{1}},
    \item
  \judgementrewrite {\propenv{}}
                 {\ep{2}} {\s{}}
                 {\filterset {\thenprop {\prop{2}}}
                             {\elseprop {\prop{2}}}}
                 {\object{2}}
                 {\e{2}},
    \item
  \e{} = {\appexp {\e{1}} {\e{2}}},
    \item
      \issubtypein{}  {\replacefor {\ty{f}} {\object{2}} {\x{}}}{\ty{}},
    \item
      \inpropenv{\replacefor {\thenprop {\prop{f}}} {\object{2}} {\x{}}} {\thenprop {\prop{}}},
    \item
      \inpropenv{\replacefor {\elseprop {\prop{f}}} {\object{2}} {\x{}}} {\elseprop {\prop{}}},
    \item
      \issubobjin{}{\replacefor {\object{f}} {\object{2}} {\x{}}} {\object{}}
  \end{itemize}

  % TODO do I need to prove anything about the argument in the definition
  % of the constant being under \s{}?

  Prove by cases on \const{}.
  \begin{itemize}
    \item[] \begin{subcase}[\const{} = \classconst]
        \issubtypein{}
  {\ArrowOne {\x{}} {\Top{}}
                                      {\Union{\Nil}{\Class}}
                                      {\filterset {\topprop{}}
                                                  {\topprop{}}}
                                      {\pth {\classpe{}} {\x{}}}}
    {\ArrowOne {\x{}} {\s{}}
                                                       {\ty{f}}
                                                       {\filterset {\thenprop {\prop{f}}}
                                                                   {\elseprop {\prop{f}}}}
                                                       {\object{f}}}

    Prove by cases on \val{2}.

        \begin{itemize}
          \item[] \begin{subcase}[\val{2} = \classvalue{\class{}} {\protect\overrightarrow {\classfieldpair{\fld{i}} {\val{i}}}}]
                    \val{} = \class{}

                    To prove part 1,
                    note
                    \issubobjin{}{\replacefor {\object{f}} {\object{2}} {\x{}}} {\object{}},
                    and \issubobjin{}{\pth {\classpe{}} {\x{}}}{\object{f}}.
                    Then either \object{} = \emptyobject{} and we are done,
                    or \object{} = {\pth {\classpe{}}{\object{2}}} and
                    by the induction hypothesis on \e{2} we know \inopenv {\openv{}} {\object{2}} {\val{2}}
                    and by the definition of path translation we know
                    {\openv{}}({\pth {\classpe{}} {\object{2}}}) = {\appexp {\classconst{}} {{\openv{}}(\object{2})}},
                    which evaluates to \val{}.

                    Part 2 is trivial since both propositions can only be \topprop{}.
                    
                    Part 3 holds because 
                    \val{} = \class{},
                    \issubtypein{}{\Union{\Nil}{\Class}}{\replacefor {\ty{f}} {\object{2}} {\x{}}}
                    and
                    \issubtypein{}{\replacefor {\ty{f}} {\object{2}} {\x{}}}{\ty{}},
                    so
                    {\judgementtwo{}{\val{}}{\ty{}}}
                    since
                    {\judgementtwo{}{\class{}}{\Union{\Nil}{\Class}}}.
                  \end{subcase}
          \item[] \begin{subcase}[\val{2} = \class{}] \val{} = \Class{}

              As above.
                  \end{subcase}
          \item[] \begin{subcase}[\val{2} = \true{}] \val{} = \Boolean{}

              As above.
                  \end{subcase}
          \item[] \begin{subcase}[\val{2} = \false{}] \val{} = \Boolean{}


              As above.
                  \end{subcase}
          \item[] \begin{subcase}[\val{2} = {\closure {\openv{}} {\abs {\x{}} {\ty{}} {\e{}}}}] \val{} = \IFn{}


              As above.
                  \end{subcase}
          \item[] \begin{subcase}[\val{2} = {\multi {\val{d}} {\disptable{}}}] \val{} = \HMapInstance{}


              As above.
                  \end{subcase}
          \item[] \begin{subcase}[\val{2} = {\protect\curlymapvaloverright{\val{1}}{\val{2}}}] \val{} = \Keyword{}


              As above.
                  \end{subcase}
          \item[] \begin{subcase}[\val{2} = {\nil{}}] \val{} = \nil{}

             Parts 1 and 2 as above.
                    Part 3 holds because \val{} = \nil{}
                    and {\judgementtwo{}{\nil{}}{\Union{\Nil}{\Class}}}.
                  \end{subcase}
        \end{itemize}
      \end{subcase}
    %\item[]
    %  \begin{subcase}[\const{} = \throwconst]
    %    {\ArrowOne {\x{}} {\s{}}
    %                                                   {\ty{f}}
    %                                                   {\filterset {\thenprop {\prop{f}}}
    %                                                               {\elseprop {\prop{f}}}}
    %                                                   {\object{f}}}
    %                                                   =
    %    {\ArrowOne {\x{}} {\Top{}}
    %                                  {\Bot{}}
    %                                  {\filterset {\botprop{}}
    %                                              {\botprop{}}}
    %                                  {\emptyobject{}}}

    %                                  Part 1 is trivial since \object{} = \emptyobject{} after substitution.
    %                                  Part 2 holds vacuously as both propositions are \botprop{} after substitution.
    %                                  Finally part 3 holds since {\judgementtwo{}{\hastype{\errorval{\val{2}}}{\Bot{}}}}.

    %  \end{subcase}
  \end{itemize}

\end{subcase}
\end{itemize}
\end{case}

\begin{case}[B-BetaMulti]
  \ 

  \begin{itemize}
    \item
  \opsem {\openv{}}
         {\e{1}}
         {\multi {\val{d}} {\disptable{}}},
    \item
  \opsem {\openv{}}
         {\e{2}}
         {\val{2}},
    \item
  \opsem {\openv{}}
         {\appexp {\val{d}} {\val{2}}}
         {\val{e}},
    \item
  \getmethod {\disptable{}}
             {\val{e}}
             {\val{l}}
             {\val{g}},
    \item
  \opsem {\openv{}}
         {\appexp {\val{g}} {\val{2}}}
         {\val{}},
       \item {\disptable{}} = {\curlymapvaloverright{\val{k}}{\val{v}}}
     \end{itemize}
     \begin{itemize}
       \item[]
\begin{subcase}[T-App]
  \ 

  \begin{itemize}
    \item
  \ep{} = {\appexp {\ep{1}} {\ep{2}}},
    \item
  \judgementrewrite {\propenv{}} {\ep{1}} {\ArrowOne {\x{}} {\s{}}
                                                       {\ty{f}}
                                                       {\filterset {\thenprop {\prop{f}}}
                                                                   {\elseprop {\prop{f}}}}
                                                       {\object{f}}}
                {\filterset {\thenprop {\prop{1}}}
                            {\elseprop {\prop{1}}}}
                {\object{1}}
                {\e{1}},
    \item
  \judgementrewrite {\propenv{}}
                 {\ep{2}} {\s{}}
                 {\filterset {\thenprop {\prop{2}}}
                             {\elseprop {\prop{2}}}}
                 {\object{2}}
                 {\e{2}},
    \item
  \e{} = {\appexp {\e{1}} {\e{2}}},
    \item
      \issubtypein{}  {\replacefor {\ty{f}} {\object{2}} {\x{}}}{\ty{}},
    \item
      \inpropenv{\replacefor {\thenprop {\prop{f}}} {\object{2}} {\x{}}} {\thenprop {\prop{}}},
    \item
      \inpropenv{\replacefor {\elseprop {\prop{f}}} {\object{2}} {\x{}}} {\elseprop {\prop{}}},
    \item
      \issubobjin{}{\replacefor {\object{f}} {\object{2}} {\x{}}} {\object{}},
  \end{itemize}

     By inversion on \e{1} via T-Multi we know 
     \begin{itemize}
       \item
         \judgementrewrite{\propenv{}}{\ep{1}}{\MultiFntype{\s{t}}{\s{d}}}
                {\filterset {\thenprop {\prop{1}}}
                            {\elseprop {\prop{1}}}}
                {\object{1}}{\e{1}},
           
         \item \s{t} = {\ArrowOne {\x{}} {\s{}}
                                                       {\ty{f}}
                                                       {\filterset {\thenprop {\prop{f}}}
                                                                   {\elseprop {\prop{f}}}}
                                                       {\object{f}}},
         \item \s{d} = {\ArrowOne {\x{}} {\s{}}
                                                       {\ty{d}}
                                                       {\filterset {\thenprop {\prop{d}}}
                                                                   {\elseprop {\prop{d}}}}
                                                       {\object{d}}},
       \item
         \judgementtwo{}{\val{d}}{\s{d}}
              \item
  $\overrightarrow{\judgementtwo{}{\val{k}}{\Top{}}}$, and 
\item
  $\overrightarrow{\judgementtwo{}{\val{v}}{\s{t}}}$.
  \end{itemize}

  % FIXME do we really know this? seems obvious but the IH says something subtly different, might need
  % a lemma to bridge this. Same problem in T-IsA case.
  By the inductive hypothesis on 
  \opsem {\openv{}}
         {\e{2}}
         {\val{2}}
  we know 
  \judgementrewrite {\propenv{}} {\val{2}} {\s{}}
             {\filterset {\thenprop {\prop{2}}}
                         {\elseprop {\prop{2}}}}
                       {\object{2}}
                       {\val{2}}.

We then consider applying the evaluated argument to the dispatch function:
  \opsem {\openv{}}
         {\appexp {\val{d}} {\val{2}}}
         {\val{e}}.

         Since we can satisfy T-App with
       \begin{itemize}
         \item
         \judgementtwo{}{\val{d}}{\ArrowOne {\x{}} {\s{}}
                                                       {\ty{d}}
                                                       {\filterset {\thenprop {\prop{d}}}
                                                                   {\elseprop {\prop{d}}}}
                                                       {\object{d}}}, and
         \item
  \judgementrewrite {\propenv{}} {\val{2}} {\s{}}
             {\filterset {\thenprop {\prop{2}}}
                         {\elseprop {\prop{2}}}}
                       {\object{2}}
                       {\val{2}},
       \end{itemize}
       we can apply the inductive hypothesis
       to derive
  \judgementrewrite {\propenv{}} {\val{e}} 
  {\replacefor{\ty{d}}
              {\object{2}}
              {\x{}}}
             {\replacefor{\filterset {\thenprop {\prop{d}}}
                                     {\elseprop {\prop{d}}}}
                         {\object{2}}
                         {\x{}}}
                       {\replacefor
                         {\object{d}}
                         {\object{2}}
                         {\x{}}}
                       {\val{e}}.

 Now we consider how we choose which method to dispatch to.

 As 
  \getmethod {\disptable{}}
             {\val{e}}
             {\val{l}}
             {\val{g}}, by inversion on \getmethodliteral
             we know
   there exists exactly one \val{k} such that 
   \entryinmap{\mapvalentry{\val{k}}{\val{g}}}{\disptable{}} and \isaopsem{\val{e}}{\val{k}} = {\true{}}.

   By inversion we know T-DefMethod must have extended \disptable{} 
   with the well-typed dispatch value \val{k},
   thus {\judgementtwo{}{\val{k}}{\ty{k}}}, and
   the well-typed method \val{g},
   so {\judgementtwo{}{\val{g}}{\s{t}}}.

  We can also prove that given
        \begin{itemize}
          \item
  \judgementrewrite {\propenv{}} {\val{e}} 
  {\replacefor{\ty{d}}
              {\object{2}}
              {\x{}}}
             {\replacefor{\filterset {\thenprop {\prop{d}}}
                                     {\elseprop {\prop{d}}}}
                         {\object{2}}
                         {\x{}}}
                       {\replacefor
                         {\object{d}}
                         {\object{2}}
                         {\x{}}}
                       {\val{e}}.
    \item
  \judgementtwo {\propenv{}} {\val{k}} {\ty{k}},
                     \item
        \isaopsem{\val{e}}{\val{k}} = {\true{}}, 
      \item
        \satisfies{\openv{}}{\propenv{}},
    \item
  \isacompare{\ty{d}}
                       {\replacefor
                         {\object{d}}
                         {\object{2}}
                         {\x{}}}
  {\ty{k}}{\filterset {\thenprop {\propp{}}} {\elseprop {\propp{}}}},
          \item
        \inpropenv{\thenprop{\propp{}}}{\thenprop{\propp{}}}, and
    \item
        \inpropenv{\elseprop{\propp{}}}{\elseprop{\propp{}}}.
        \end{itemize}
  we can apply \lemref{appendix:lemma:isa} to derive
  then {\satisfies{\openv{}}{\thenprop{\propp{}}}}.

   Now we consider applying the evaluated argument to the chosen method:
  \opsem {\openv{}}
         {\appexp {\val{g}} {\val{2}}}
         {\val{}}.

  By inversion via B-DefMethod we can assume {\val{g}} = {\abs{\x{}}{\s{}}{\e{b}}}, 
  ie. that we have chosen a method to dispatch to that is a closure.

  Because 
  \opsem {\openv{}}
         {\appexp {\val{g}} {\val{2}}}
         {\val{}}
         and
    \judgementtwo{\propenv{}}{\val{2}}{\s{}},
  by inversion via B-BetaClosure we know {\val{}} = {\replacefor{\e{b}}{\val{2}}{\x{}}}.

  With the following premises:
\begin{itemize}
  \item
{\judgementrewrite{{\propenv{}},{\thenprop{\propp{}}}}
                  {\replacefor{\ep{b}}{\val{2}}{\x{}}}
                  {\replacefor{\ty{f}} {\object{2}}{\x{}}}
                  {\replacefor
                   {\filterset {\thenprop {\prop{f}}}
                               {\elseprop {\prop{f}}}}
                             {\object{2}}
                             {\x{}}}
                  {\replacefor
                          {\object{f}}
                             {\object{2}}
                             {\x{}}}
                  {\replacefor{\e{b}}{\val{2}}{\x{}}}
                        },
    \begin{itemize}
      \item From
{\judgementrewrite{{\propenv{}},{\isprop{\s{}}{\x{}}}}
                  {\e{b}}
                  {\ty{f}}
               {\filterset {\thenprop {\prop{f}}}
                                       {\elseprop {\prop{f}}}}
                          {\object{f}}
                          {\e{b}}}
          via the inductive hypothesis on 
  \opsem {\openv{}}
         {\appexp {\abs{\x{}}{\s{}}{\e{b}}} {\val{2}}}
         {\val{}},
      \item then we can derive
{\judgementrewrite{{\propenv{}}}
                  {\replacefor{\ep{b}}{\val{2}}{\x{}}}
                  {\replacefor{\ty{f}} {\object{2}}{\x{}}}
                  {\replacefor
                   {\filterset {\thenprop {\prop{f}}}
                               {\elseprop {\prop{f}}}}
                             {\object{2}}
                             {\x{}}}
                  {\replacefor
                          {\object{f}}
                             {\object{2}}
                             {\x{}}}
                  {\replacefor{\e{b}}{\val{2}}{\x{}}}
                        } via substitution and the fact that {\x{}} is fresh 
                        therefore \x{} $\not\in$ \fv{\propenv{}} so we do not need to substitution for \x{} in \propenv{}. %TODO lemma for this
                        
      \item 
        \satisfies{\openv{}}{\propenv{}, {\thenprop{\propp{}}}}
        because
        \satisfies{\openv{}}{\propenv{}} and {\satisfies{\openv{}}{\thenprop{\propp{}}}} via M-And.
    \end{itemize}
  \item
              \satisfies{\openv{}}{{\propenv{}},{\thenprop{\propp{}}}},
    \begin{itemize}
      \item From \satisfies{\openv{}}{\propenv{}} and \item {\satisfies{\openv{}}{\thenprop{\propp{}}}}  via M-And.
    \end{itemize}
           \item
\isconsistent{\openv{}}, and
           \item 
  \opsem {\openv{}}
         {\replacefor{\e{b}}{\val{2}}{\x{}}}
         {\val{}}.
\end{itemize}

we can apply the inductive hypothesis to satisfy our overall goal for this subcase.
\end{subcase}
     \end{itemize}
\end{case}

\begin{case}[BE-Beta1]
  \ 

  Reduces to an error.
\end{case}
\begin{case}[BE-Beta2]
  \ 

  Reduces to an error.
\end{case}
\begin{case}[BE-BetaClosure]
  \ 

  Reduces to an error.
\end{case}
\begin{case}[BE-BetaMulti1]
  \ 

  Reduces to an error.
\end{case}
\begin{case}[BE-BetaMulti2]
  \ 

  Reduces to an error.
\end{case}
\begin{case}[BE-Delta]
  \ 

  Reduces to an error.
\end{case}

\begin{case}[B-IsA]
        \opsem {\openv{}} {\e{1}} {\val{1}},
        \opsem {\openv{}} {\e{2}} {\val{2}},
        \isaopsem{\val{1}}{\val{2}} = {\val{}}


  \begin{itemize}
    \item[]
      \begin{subcase}[T-IsA]
  \ep{} = {\isaapp {\ep{1}} {\ep{2}}},
  \judgementrewrite {\propenv{}} {\ep{1}} {\ty{1}}
             {\filterset {\thenprop {\prop{1}}}
                         {\elseprop {\prop{1}}}}
                       {\object{1}}
                       {\e{1}},
  \judgementrewrite {\propenv{}} {\ep{2}} {\ty{2}}
             {\filterset {\thenprop {\prop{2}}}
                         {\elseprop {\prop{2}}}}
                       {\object{2}}
                       {\e{2}},
  \e{} = {\isaapp {\e{1}} {\e{2}}},
  \issubtypein{}{\Boolean}{\ty{}},
  \isacompare{\ty{1}}{\object{1}}{\ty{2}}{\filterset {\thenprop {\propp{}}} {\elseprop {\propp{}}}},
  \inpropenv{\thenprop {\propp{}}}{\thenprop {\prop{}}},
  \inpropenv{\elseprop {\propp{}}}{\elseprop {\prop{}}},
  \issubobjin{}{\emptyobject{}}{\object{}}

        Part 1 holds trivially with \object{} = \emptyobject{}.

        For part 2, by the induction hypothesis on \e{1} and \e{2}
        we know
  \judgementrewrite {\propenv{}} {\val{1}} {\ty{1}}
             {\filterset {\thenprop {\prop{1}}}
                         {\elseprop {\prop{1}}}}
                       {\object{1}}
                       {\val{1}} and
  \judgementrewrite {\propenv{}} {\val{2}} {\ty{2}}
             {\filterset {\thenprop {\prop{2}}}
                         {\elseprop {\prop{2}}}}
                       {\object{2}}
                       {\val{2}},
                       so we can then apply
        \lemref{appendix:lemma:isa}
        to reach our goal.

        Part 3 holds because by the definition of \isaopsemliteral
        \val{} can only be \true\ or \false, 
        and since \judgementtwo{\propenv{}}{\true}{\ty{}}
        and
        \judgementtwo{\propenv{}}{\false}{\ty{}}
        we are done.
      \end{subcase}
  \end{itemize}
\end{case}

      \begin{case}[BE-IsA1]
        \opsem {\openv{}} {\e{1}} {\errorvalv{}}

        Trivially reduces to an error.
      \end{case}
      \begin{case}[BE-IsA2]
       \opsem {\openv{}} {\e{1}} {\val{1}},
       \opsem {\openv{}} {\e{2}} {\errorvalv{}}

        Trivially reduces to an error.
      \end{case}

\begin{case}[B-Get]
      $\opsem {\openv{}} {\e{m}}{\val{m}}$,
        $\val{m} = {\curlymap{\overrightarrow{({\val{a}}\ {\val{b}})}}}$,
         \opsem {\openv{}} {\e{k}} {\kw{}},
         $\keyinmap{\kw{}}{\curlymap{\overrightarrow{({\val{a}}\ {\val{b}})}}}$,
         \getmap{\curlymap{\overrightarrow{({\val{a}}\ {\val{b}})}}} {\kw{}} = {\val{}}

  \begin{itemize}
    \item[]
      \begin{subcase}[T-GetHMap]
  \ep{} = {\getexp {\ep{m}} {\ep{k}}},
  \judgementrewrite {\propenv{}} {\ep{m}} {\Unionsplice {\overrightarrow {\HMapgeneric {\mandatory{}} {\absent{}}}}}
           {\filterset {\thenprop {\prop{m}}} {\elseprop {\prop{m}}}}
           {\object{m}}
           {\e{m}},
  \judgementtworewrite {\propenv{}} {\ep{k}} {\Value {k}}{\e{k}},
  \overr{\inmandatory{\kw{}}{\ty{i}}{\mandatory{}},}
  \e{} = {\getexp {\e{m}} {\e{k}}},
  \issubtypein{}{\Unionsplice {\overrightarrow {\ty{i}}}}{\ty{}} ,
  \thenprop{\prop{}} = {\topprop{}},
  \elseprop{\prop{}} = {\topprop{}},
  \issubobjin{}{\replacefor {\pth {\keype{k}} {\x{}}}
                          {\object{m}}
                          {\x{}}}
                        {\object{}}


         To prove part 1 we consider two cases on the form of \object{m}: 
         \begin{itemize}
           \item
         if {\object{m}} = \emptyobject{}
         then \object{} = \emptyobject{} by substitution, which gives the desired result;
           \item
         if \object{m} = {\pth {\pathelem{m}} {\x{m}}}
         then \issubobjin{}{\pth {\keype{k}} {\object{m}}}{\object{}} by substitution.
         We note by the definition of path translation
         {\openv{}}({\pth {\keype{k}} {\object{m}}}) =
         {\getexp {{\openv{}}(\object{m})}{\kw{}}}
         and by the induction hypothesis on \e{m}
         {{\openv{}}(\object{m})} = {\curlymap{\overrightarrow{({\val{a}}\ {\val{b}})}}},
         which together imply 
         \inopenv {\openv{}} {\object{}} {\getexp {\curlymap{\overrightarrow{({\val{a}}\ {\val{b}})}}} {\kw{}}}.
         Since this is the same form as B-Get, we can apply the premise
         \getmap{\curlymap{\overrightarrow{({\val{a}}\ {\val{b}})}}} {\kw{}} = {\val{}}
         to derive \inopenv {\openv{}} {\object{}} {\val{}}.
         \end{itemize}
         
         Part 2 holds trivially as \thenprop{\prop{}} = {\topprop{}}
         and \elseprop{\prop{}} = {\topprop{}}.

         To prove part 3 we note that (by the induction hypothesis on \e{m})
         $\judgementtwo{}{\val{m}}{\Unionsplice{\overrightarrow {\HMapgeneric {\mandatory{}} {\absent{}}}}}$,
         where $\overrightarrow{\inmandatory{\kw{}}{\ty{i}}{\mandatory{}}}$, and 
         both
         $\keyinmap{\kw{}}{\curlymap{\overrightarrow{({\val{a}}\ {\val{b}})}}}$
         and
         \getmap{\curlymap{\overrightarrow{({\val{a}}\ {\val{b}})}}} {\kw{}} = {\val{}}
         imply \judgementtwo{}{\val{}}{\Unionsplice {\overrightarrow {\ty{i}}}}.

      \end{subcase}
    \item[]
      \begin{subcase}[T-GetHMapAbsent]
  \ep{} = {\getexp {\ep{m}} {\ep{k}}},
  \judgementtworewrite {\propenv{}} {\ep{k}} {\Value {k}} {\e{k}},
  \\
  \judgementrewrite {\propenv{}} {\ep{m}} {\HMapgeneric {\mandatory{}} {\absent}}
           {\filterset {\thenprop {\prop{m}}} {\elseprop {\prop{m}}}}
           {\object{m}}
           {\e{m}},
  {\inabsent{\kw{}}{\absent{}}},
  \e{} = {\getexp {\e{m}} {\e{k}}},
  \issubtypein{}{\Nil}{\ty{}},
  \thenprop{\prop{}} = {\topprop{}},
  \elseprop{\prop{}} = {\topprop{}},
  \issubobjin{}{\replacefor
               {\pth {\keype{k}} {\x{}}}
                          {\object{m}}
                          {\x{}}}
                        {\object{}}

       Unreachable subcase because 
         $\keyinmap{\kw{}}{\curlymap{\overrightarrow{({\val{a}}\ {\val{b}})}}}$,
         contradicts
                {\inabsent{\kw{}}{\absent{}}}.
      \end{subcase}
    \item[]
      \begin{subcase}[T-GetHMapPartialDefault]
  \ep{} = {\getexp {\ep{m}} {\ep{k}}},
  \judgementtworewrite {\propenv{}} {\ep{k}} {\Value {k}}{\e{k}},
  \\
 \judgementrewrite {\propenv{}} {\ep{m}} {\HMapp {\mandatory{}} {\absent}}
           {\filterset {\thenprop {\prop{m}}} {\elseprop {\prop{m}}}}
           {\object{m}}
           {\e{m}},
             {\notinmandatory{\kw{}}{\ty{}}{\mandatory{}}},
             {\notinabsent{\kw{}}{\absent{}}},
  \e{} = {\getexp {\e{m}} {\e{k}}},
  \ty{} = \Top,
  \thenprop{\prop{}} = {\topprop{}},
  \elseprop{\prop{}} = {\topprop{}},
  \issubobjin{}{\replacefor
               {\pth {\keype{k}} {\x{}}}
                          {\object{m}}
                          {\x{}}}{\object{}}

         Parts 1 and 2 are the same as the B-Get subcase.
         Part 3 is trivial as \ty{} = \Top.
      \end{subcase}
  \end{itemize}
\end{case}

\begin{case}[B-GetMissing]
        \val{} = \nil,
        $\opsem {\openv{}}
        {\e{m}} {\curlymap{\overrightarrow{({\val{a}}\ {\val{b}})}}}$,
       \opsem {\openv{}} {\e{k}} {\kw{}},
       \keynotinmap{\kw{}}{\curlymap{\overrightarrow{({\val{a}}\ {\val{b}})}}}

  \begin{itemize}
    \item[]
      \begin{subcase}[T-GetHMap]
  \ep{} = {\getexp {\ep{m}} {\ep{k}}},
  \judgementrewrite {\propenv{}} {\ep{m}} {\Unionsplice {\overrightarrow {\HMapgeneric {\mandatory{}} {\absent{}}}}}
           {\filterset {\thenprop {\prop{m}}} {\elseprop {\prop{m}}}}
           {\object{m}}
           {\e{m}},
  \judgementtworewrite {\propenv{}} {\ep{k}} {\Value {k}}{\e{k}},
  \overr{\inmandatory{\kw{}}{\ty{i}}{\mandatory{}},}
  \e{} = {\getexp {\e{m}} {\e{k}}},
  \issubtypein{}{\Unionsplice {\overrightarrow {\ty{i}}}}{\ty{}},
  \thenprop{\prop{}} = {\topprop{}},
  \elseprop{\prop{}} = {\topprop{}},
  \issubobjin{}{\replacefor {\pth {\keype{k}} {\x{}}}
                          {\object{m}}
                          {\x{}}}{\object{}}

       Unreachable subcase because 
       \keynotinmap{\kw{}}{\curlymap{\overrightarrow{({\val{a}}\ {\val{b}})}}}
       contradicts ${\inmandatory{\kw{}}{\ty{}}{\mandatory{}}}$.
      \end{subcase}
    \item[]
      \begin{subcase}[T-GetHMapAbsent]
  \ep{} = {\getexp {\ep{m}} {\ep{k}}},
  \judgementtworewrite {\propenv{}} {\ep{k}} {\Value {k}} {\e{k}},
  \\
  \judgementrewrite {\propenv{}} {\ep{m}} {\HMapgeneric {\mandatory{}} {\absent}}
           {\filterset {\thenprop {\prop{m}}} {\elseprop {\prop{m}}}}
           {\object{m}}
           {\e{m}},
  {\inabsent{\kw{}}{\absent{}}},
  \e{} = {\getexp {\e{m}} {\e{k}}},
  \issubtypein{}{\Nil}{\ty{}},
  \thenprop{\prop{}} = {\topprop{}},
  \elseprop{\prop{}} = {\topprop{}},
  \issubobjin{}{\replacefor
               {\pth {\keype{k}} {\x{}}}
                          {\object{m}}
                          {\x{}}}{\object{}}

         To prove part 1 we consider two cases on the form of \object{m}: 
         \begin{itemize}
           \item
         if {\object{m}} = \emptyobject{}
         then \object{} = \emptyobject{} by substitution, which gives the desired result;
           \item
         if \object{m} = {\pth {\pathelem{m}} {\x{m}}}
         then \issubobjin{}{\pth {\keype{k}} {\object{m}}}{\object{}} by substitution.
         We note by the definition of path translation
         {\openv{}}({\pth {\keype{k}} {\object{m}}}) =
         {\getexp {{\openv{}}(\object{m})}{\kw{}}}
         and by the induction hypothesis on \e{m}
         {{\openv{}}(\object{m})} = {\curlymap{\overrightarrow{({\val{a}}\ {\val{b}})}}},
         which together imply 
         \inopenv {\openv{}} {\object{}} {\getexp {\curlymap{\overrightarrow{({\val{a}}\ {\val{b}})}}} {\kw{}}}.
         Since this is the same form as B-GetMissing, we can apply the premise
        \val{} = \nil\ 
         to derive \inopenv {\openv{}} {\object{}} {\val{}}.
         \end{itemize}
         
         Part 2 holds trivially as \thenprop{\prop{}} = {\topprop{}}
         and \elseprop{\prop{}} = {\topprop{}}.

         To prove part 3 we note that \e{m} has type {\HMapgeneric {\mandatory{}} {\absent{}}}
         where {\inabsent{\kw{}}{\absent{}}}, and
         the premises of B-GetMissing
         \keynotinmap{\kw{}}{\curlymap{\overrightarrow{({\val{a}}\ {\val{b}})}}}
         and
          \val{} = \nil\ 
         tell us {\val{}} must be of type {\ty{}}.
      \end{subcase}
    \item[]
      \begin{subcase}[T-GetHMapPartialDefault]
  \ep{} = {\getexp {\ep{m}} {\ep{k}}},
  \judgementtworewrite {\propenv{}} {\ep{k}} {\Value {k}}{\e{k}},
  \\
 \judgementrewrite {\propenv{}} {\ep{m}} {\HMapp {\mandatory{}} {\absent}}
           {\filterset {\thenprop {\prop{m}}} {\elseprop {\prop{m}}}}
           {\object{m}}
           {\e{m}},
             {\notinmandatory{\kw{}}{\ty{}}{\mandatory{}}},
             {\notinabsent{\kw{}}{\absent{}}},
  \e{} = {\getexp {\e{m}} {\e{k}}},
  \ty{} = \Top,
  \thenprop{\prop{}} = {\topprop{}},
  \elseprop{\prop{}} = {\topprop{}},
  \issubobjin{}{\replacefor
               {\pth {\keype{k}} {\x{}}}
                          {\object{m}}
                          {\x{}}}{\object{}}

         Parts 1 and 2 are the same as the B-GetMissing subcase of T-GetHMapAbsent.
         Part 3 is trivial, since \ty{} = \Top.
      \end{subcase}
  \end{itemize}
\end{case}

\begin{case}[BE-Get1]
  \ 

  Reduces to an error.
\end{case}

\begin{case}[BE-Get2]
  \ 

  Reduces to an error.
\end{case}

\begin{case}[B-Assoc]
        \val{} = 
        {\extendmap{\curlymap{\overrightarrow{({\val{a}}\ {\val{b}})}}}
                {\kw{}}{\val{v}}},
        \opsem {\openv{}}
        {\e{m}} {\curlymap{\overrightarrow{({\val{a}}\ {\val{b}})}}},
        \opsem {\openv{}} {\e{k}} {\kw{}},
        \opsem {\openv{}} {\e{v}} {\val{v}}

  \begin{itemize}
    \item[]
      \begin{subcase}[T-AssocHMap]
  \judgementtworewrite {\propenv{}} {\ep{m}} {\HMapgeneric {\mandatory{}} {\absent}} {\e{m}},
  \judgementtworewrite {\propenv{}} {\ep{k}} {\Value{\kw{}}}{\e{k}},\\
  \judgementtworewrite {\propenv{}} {\ep{v}} {\ty{}}{\e{v}},
  {\kw{}} $\not\in$ {\absent{}},
  \ep{} = {\assocexp {\ep{m}} {\ep{k}} {\ep{v}}},\\
  \e{} = {\assocexp {\e{m}} {\e{k}} {\e{v}}},
  \issubtypein{}{\HMapgeneric {\extendmandatoryset {\mandatory{}}{\kw{}}{\ty{}}} {\absent}}{\ty{}},
  \thenprop{\prop{}} = {\topprop{}},
  \elseprop{\prop{}} = {\botprop{}},
  \object{} = \emptyobject{}

        Parts 1 and 2 hold for the same reasons as T-True.
        %TODO part 3
      \end{subcase}
  \end{itemize}
\end{case}

\begin{case}[BE-Assoc1]
  \ 

  Reduces to an error.
\end{case}

\begin{case}[BE-Assoc2]
  \ 

  Reduces to an error.
\end{case}

\begin{case}[BE-Assoc3]
  \ 

  Reduces to an error.
\end{case}

\begin{case}[B-IfFalse]
        \opsem {\openv{}} {\e{1}} {\false}
        \ \ \text{or}\ \ 
        \opsem {\openv{}} {\e{1}} {\nil},
        \opsem {\openv{}} {\e{3}} {\val{}}

  \begin{itemize}
    \item[]
      \begin{subcase}[T-If]
        \ep{} = {\ifexp {\ep{1}} {\ep{2}} {\ep{3}}},
        \judgementrewrite {\propenv{}} {\ep{1}} {\ty{1}} {\filterset {\thenprop {\prop{1}}} {\elseprop {\prop{1}}}}
                 {\object{1}}
                 {\e{1}},
  \judgementrewrite {\propenv{}, {\thenprop {\prop{1}}}}
                 {\ep{2}} {\ty{}} {\filterset {\thenprop {\prop{2}}} {\elseprop {\prop{2}}}}
                 {\object{}}
                 {\e{2}},
  \judgementrewrite {\propenv{}, {\elseprop {\prop{1}}}}
                 {\ep{3}} {\ty{}} {\filterset {\thenprop {\prop{3}}} {\elseprop {\prop{3}}}}
                 {\object{}}
                 {\e{3}},
        \e{} = {\ifexp {\e{1}} {\e{2}} {\e{3}}},
  \inpropenv{\orprop {\thenprop {\prop{2}}} {\thenprop {\prop{3}}}}{\thenprop{\prop{}}},
  \inpropenv{\orprop {\elseprop {\prop{2}}} {\elseprop {\prop{3}}}}{\elseprop{\prop{}}}

              For part 1, either \object{} = \emptyobject{}, or \e{} evaluates to the
              result of \e{3}.

              To prove part 2, we consider two cases:
              \begin{itemize}
                \item if \isfalseval{\val{}}
                  then \e{3} evaluates to a false value so {\satisfies{\openv{}}{\elseprop {\prop{3}}}}, and thus
                  {\satisfies{\openv{}}{\orprop {\elseprop {\prop{2}}} {\elseprop {\prop{3}}}}} by M-Or, 
                \item otherwise
                  \istrueval{\val{}},
                  so \e{3} evaluates to a true value so {\satisfies{\openv{}}{\thenprop {\prop{3}}}}, and thus
                  {\satisfies{\openv{}}{\orprop {\thenprop {\prop{2}}} {\thenprop {\prop{3}}}}} by M-Or.
              \end{itemize}

              Part 3 is trivial as
              \opsem {\openv{}} {\e{3}} {\val{}}
              and {\judgementtwo{}{\val{}}{\ty{}}} by the induction hypothesis on {\e{3}}.
      \end{subcase}
  \end{itemize}
\end{case}

\begin{case}[B-IfTrue]
        \opsem {\openv{}} {\e{1}} {\val{1}},
              ${\val{1}} \not= {\false}$,
              ${\val{1}} \not= {\nil}$,
              \opsem {\openv{}} {\e{2}} {\val{}}

  \begin{itemize}
    \item[]
      \begin{subcase}[T-If]
        \ep{} = {\ifexp {\ep{1}} {\ep{2}} {\ep{3}}},
        \judgementrewrite {\propenv{}} {\ep{1}} {\ty{1}} {\filterset {\thenprop {\prop{1}}} {\elseprop {\prop{1}}}}
                 {\object{1}}
                 {\e{1}},
  \judgementrewrite {\propenv{}, {\thenprop {\prop{1}}}}
                 {\ep{2}} {\ty{}} {\filterset {\thenprop {\prop{2}}} {\elseprop {\prop{2}}}}
                 {\object{}}
                 {\e{2}},
  \judgementrewrite {\propenv{}, {\elseprop {\prop{1}}}}
                 {\ep{3}} {\ty{}} {\filterset {\thenprop {\prop{3}}} {\elseprop {\prop{3}}}}
                 {\object{}}
                 {\e{3}},
        \e{} = {\ifexp {\e{1}} {\e{2}} {\e{3}}},
  \inpropenv{\orprop {\thenprop {\prop{2}}} {\thenprop {\prop{3}}}}{\thenprop{\prop{}}},
  \inpropenv{\orprop {\elseprop {\prop{2}}} {\elseprop {\prop{3}}}}{\elseprop{\prop{}}}

              For part 1, either \object{} = \emptyobject{}, or \e{} evaluates to the
              result of \e{2}.

              To prove part 2, we consider two cases:
              \begin{itemize}
                \item if \isfalseval{\val{}}
                  then \e{2} evaluates to a false value so {\satisfies{\openv{}}{\elseprop {\prop{2}}}}, and thus
                  {\satisfies{\openv{}}{\orprop {\elseprop {\prop{2}}} {\elseprop {\prop{3}}}}} by M-Or, 
                \item otherwise
                  \istrueval{\val{}},
                  so \e{2} evaluates to a true value so {\satisfies{\openv{}}{\thenprop {\prop{2}}}}, and thus
                  {\satisfies{\openv{}}{\orprop {\thenprop {\prop{2}}} {\thenprop {\prop{3}}}}} by M-Or.
              \end{itemize}

              Part 3 is trivial as
              \opsem {\openv{}} {\e{2}} {\val{}}
              and {\judgementtwo{}{\val{}}{\ty{}}} by the induction hypothesis on {\e{2}}.

      \end{subcase}
  \end{itemize}
\end{case}

\begin{case}[BE-If]
  \ 

  Reduces to an error.
\end{case}

\begin{case}[BE-IfFalse]
  \ 

  Reduces to an error.
\end{case}

\begin{case}[BE-IfTrue]
  \ 

  Reduces to an error.
\end{case}

\begin{case}[B-Let]
  \e{} = {\letexp {\x{}} {\e{1}} {\e{2}}},
        \opsem {\openv{}} {\e{1}} {\val{1}},
        \opsem {\extendopenv{\openv{}}{\x{}}{\val{1}}} {\e{2}} {\val{}}


  \begin{itemize}
    \item[]
      \begin{subcase}[T-Let]
  \ep{} = {\letexp {\x{}} {\ep{1}} {\ep{2}}},
  \judgementrewrite {\propenv{}} {\ep{1}} {\s{}} {\filterset {\thenprop {\prop{1}}} {\elseprop {\prop{1}}}}
             {\object{1}}
             {\ep{1}},
             \propp{} = {\impprop {\notprop {\falsy{}} {\x{}}} {\thenprop {\prop{1}}}},
             \proppp{} = {\impprop {\isprop {\falsy{}} {\x{}}} {\elseprop {\prop{1}}}},
  \judgementrewrite
       {\propenv{}, {\isprop {\s{}} {\x{}}},
         {\propp{}},
         {\proppp{}}}
             {\ep{2}} {\ty{}} {\filterset {\thenprop {\prop{}}} {\elseprop {\prop{}}}}
             {\object{}} 
             {\e{2}}

        For all the following cases (with a reminder that \x{} is fresh)
        we apply the induction hypothesis on \e{2}. We justify this by noting
        that occurrences of \x{} inside \e{2} have the same type as \e{1} and 
        simulate the propositions of \e{1}
        because 
        \begin{itemize}
          \item
  \judgementrewrite
       {\propenv{}, {\isprop {\s{}} {\x{}}},
         {\propp{}},
         {\proppp{}}}
             {\ep{2}} {\ty{}} {\filterset {\thenprop {\prop{}}} {\elseprop {\prop{}}}}
             {\object{}} 
             {\e{2}},
           \item
        \satisfies{\extendopenv{\openv{}}{\x{}}{\val{1}}}{\propenv{}, {\isprop {\s{}} {\x{}}}, \propp{}, \proppp{}},
           \item
        {\isconsistent{\extendopenv{\openv{}}{\x{}}{\val{1}}}},
        and
           \item
        \opsem {\extendopenv{\openv{}}{\x{}}{\val{1}}} {\e{2}} {\val{}}.
    \end{itemize}

        We prove parts 1, 2 and 3 by directly using the induction hypothesis on \e{2}.
      \end{subcase}
  \end{itemize}
\end{case}

\begin{case}[BE-Let]
  \ 
  
  Reduces to an error.
\end{case}

\begin{case}[B-Abs] 
        \val{} = {\closure {\openv{}} {\abs {\x{}} {\s{}} {\e{1}}}}

  \begin{itemize}
    \item[]
      \begin{subcase}[T-Clos]
  \ep{} = {\closure {\openv{}} {\abs {\x{}} {\s{}} {\e{1}}}},
  $\exists {\propenvp{}}. \satisfies{\openv{}}{\propenvp{}}$
  \ \text{and}\ 
\judgementrewrite {\propenvp{}} {\abs {\x{}} {\s{}} {\e{1}}} {\ty{}}
                 {\filterset {\thenprop {\prop{f}}}
                             {\elseprop {\prop{f}}}}
                 {\object{f}}
                 {\abs {\x{}} {\s{}} {\e{1}}},
  \e{} = {\closure {\openv{}} {\abs {\x{}} {\s{}} {\e{1}}}},
                 {\thenprop{\prop{}}} = \topprop{},
                 {\elseprop{\prop{}}} = \botprop{},
                 {\object{}} = \emptyobject{}

        We assume some \propenvp{}, such that
        \begin{itemize}
          \item \satisfies{\openv{}}{\propenvp{}}
          \item \judgement {\propenvp{}} {\abs {\x{}} {\s{}} {\e{1}}} {\ty{}}
                           {\filterset {\thenprop {\prop{}}}
                                       {\elseprop {\prop{}}}}
                           {\object{}}.
       \end{itemize}
       Note the last rule in the derivation of
          \judgement {\propenvp{}} {\abs {\x{}} {\s{}} {\e{1}}} {\ty{}}
                           {\filterset {\thenprop {\prop{}}}
                                       {\elseprop {\prop{}}}}
                           {\object{}}
                           must be T-Abs, so 
                           {\thenprop {\prop{}}} = {\topprop{}},
                           {\elseprop {\prop{}}} = {\botprop{}}
                           and {\object{}} = {\emptyobject{}}.
         Thus parts 1 and 2 hold for the same reasons as T-True.
         Part 3 holds as \val{} has the same type as {\abs {\x{}} {\s{}} {\e{1}}}
         under \propenvp{}.

      \end{subcase} 
  \end{itemize}
\end{case}

\begin{case}[B-Abs]
        \val{} = ${\closure {\openv{}} {\abs {\x{}} {\s{}} {\e{1}}}}$,
          { \opsem {\openv{}}
                   {\abs {\x{}} {\ty{}} {\e{1}}}
                   {\closure {\openv{}} {\abs {\x{}} {\s{}} {\e{1}}}}}

  \begin{itemize}
    \item[]
      %TODO
      \begin{subcase}[T-Abs]
  \ep{} = {\abs {\x{}} {\s{}} {\ep{1}}},
{ \judgementrewrite {\propenv{}, {\isprop {\s{}} {\x{}}}}
            {\ep{1}} {\ty{}}
             {\filterset {\thenprop {\prop{1}}}
                         {\elseprop {\prop{1}}}}
             {\object{1}}
             {\e{1}}},
           \issubtypein{}
           {\ArrowOne {\x{}} {\s{}}
                      {\ty{1}}
                      {\filterset {\thenprop {\prop{1}}}
                                  {\elseprop {\prop{1}}}}
                      {\object{1}}}
          {\ty{}},
          \inpropenv{\topprop{}}{\thenprop{\prop{}}},
          \inpropenv{\botprop{}}{\elseprop{\prop{}}},
          {\object{}} = {\emptyobject{}}

        Parts 1 and 2 hold for the same reasons as T-True.
        Part 3 holds directly via T-Clos, since \val{} must be a closure.
      \end{subcase}
  \end{itemize}
\end{case}

\begin{case}[BE-Error]
        \opsem {\openv{}} {\e{}} {\errorval{\val{1}}}


  \begin{itemize}
    \item[]
      \begin{subcase}[T-Error] 
  \ep{} = \errorval{\val{1}},
  \e{} = \errorval{\val{1}},
  \ty{} = \Bot,
  \thenprop{\prop{}} = \botprop{}, \elseprop{\prop{}} = \botprop{}, \object{} = \emptyobject{}

        Trivially reduces to an error.
      \end{subcase}
  \end{itemize}
\end{case}

\end{proof}

\end{lemma}

{\wrongtheorem{appendix}}

{\soundnesstheorem{appendix}}
