%\documentclass[11pt]{iuthesis}
\documentclass{book}

%TODO use something like https://bitbucket.org/amiede/classicthesis/wiki/Home

% this is the dissertation, not the paper
\newcommand{\DISSERTATION}{}
\newcommand{\either}[2]{#1}

\usepackage{savesym}
\savesymbol{r}
\savesymbol{AA}
\usepackage{esop-common}
\usepackage{infer-common}
\usepackage{symb-common}
%\usepackage{quals-common}

\usepackage{hyperref}

\newcommand{\thesisauthor}[0]{Ambrose Bonnaire-Sergeant}
\newcommand{\thesistitle}[0]{Typed Clojure in Theory and Practice} % copied to \title for all-caps
%\newcommand{\thesiskeywords}[0]{Kwd1, Kwd2, Kwd3}

%% Setup for hyperref.
\hypersetup{
  pdftitle={\thesistitle{}},
  pdfauthor={\thesisauthor{}},
  colorlinks=true,
  linkcolor=black,
  citecolor=black,
  urlcolor=black,
}


\title{Typed Clojure in Theory and Practice}
\author{Ambrose Bonnaire-Sergeant}

% pdflatex
%\usepackage[backend=bibtex]{biblatex}
% latexmk
\usepackage[backend=biber]{biblatex}

% https://tex.stackexchange.com/questions/134191/line-breaks-of-long-urls-in-biblatex-bibliography
% wrap url's without spilling the margins
\usepackage{url}
\setcounter{biburllcpenalty}{7000}
\setcounter{biburlucpenalty}{8000}
\addbibresource{bibliography.bib}

\usepackage{thmtools}
%\declaretheorem[numberwithin=chapter]{example}
%\declaretheorem[numberwithin=chapter]{theorem}
%\declaretheorem[numberwithin=chapter]{lemma}
%\declaretheorem[numberwithin=chapter]{corollary}
%\declaretheorem[numberwithin=chapter]{definition}

\begin{document}

\frontmatter %turns off chapter numbering and uses roman numerals for page numbers

\maketitle

\chapter*{Abstract}
%\chapter*{Abstract}

Typed Clojure is an optional type system for the Clojure programming language that aims to type check idiomatic Clojure code.
This dissertation presents the design of Typed Clojure, formalizes Typed Clojure's underlying theory, studies its effectiveness
in real-world code bases, and proposes several extensions to help address its shortcomings.

I develop a formal model of Typed Clojure that includes
key features like hash-maps, multimethods, Java interoperability, and occurrence typing,
and prove the model type sound.
Then, I demonstrate that Typed Clojure's design is useful and corresponds to actual usage patterns
with an empirical study of real-world Typed Clojure usage in over 19,000 lines of code.
This experience also revealed several usability shortcomings in Typed Clojure.

First, the top-level annotation burden needed to port untyped code is prohibitively high.
We present an automatic annotator for Typed Clojure to ease this burden, using runtime
observations to synthesize heterogeneous, recursive type annotations. We evaluate our
experience using the annotator by porting several open-source projects.
%First, I address a major usability flaw in Typed Clojure: users must \emph{manually}
%write annotations.
%To remedy this, 
%I present a tool that automatically generates Typed Clojure annotations based on observed
%program behavior, including
%a formal model of the tool, consisting of its runtime instrumentation phase that
%collects samples from a running program, and type reconstruction phase
%that creates useful annotations from these samples.
%Then, I give an overview of a practical implementation that generates Typed Clojure annotations for
%real programs.
%Next, I study the effectiveness, accuracy, and usability of these annotations
%by generating annotations for several projects, and then manually amending the annotations
%until they type check.

Second, pre-expanding macros before type checking makes type checking brittle.
We describe and implement a new analyzer for Clojure code that can provide the
foundation of an alternative approach where the user provides custom type rules for macros.

Third, too many local functions require annotations. We present a hybrid approach of symbolic 
execution and type checking that helps check some common higher-order Clojure idioms.

%The final part of this thesis will either:
%\begin{itemize}
%  \item increase the number of type checkable Clojure programs, especially those
%    combining polymorphic higher-order and anonymous functions, by combining
%    an extensible typing rule system with symbolic execution, and study its effectiveness
%    in reducing the changes needed to port Clojure programs to Typed Clojure, or
%  \item repurpose the automatic annotation tool to generate clojure.spec annotations,
%    study its effectiveness in generating good specs over several
%    hundred open source projects, and use it to help answer more
%    general questions about Clojure usage.
%\end{itemize}

%Third, we conduct a study of clojure.spec, the recently released runtime verification
%system that comes bundled with Clojure, and compare its feature set to Typed Clojure's.
%I present an empirical study of the use of Clojure's core.spec contract system in several
%real world code bases, observing which features are used, and the precision of
%specifications.
%I then present models of several subsets of clojure.spec, concentrating on its interesting
%handling of higher-order function checking, and precisely identifying its intentional
%unsoundness compared to traditional higher-order contract checking.
%
%I repurpose my automatic annotation tool to generate clojure.spec annotations (``specs'')
%and subsequently test their effectiveness over hundreds of open source Clojure projects.
%I outline clojure.spec, the official runtime verification
%library bundled with Clojure, and present a formal model of clojure.spec that highlights its
%``generative testing'' function checking semantics.
%Next, I discuss how to extend my annotation tool to generate specs.
%Finally, I verify the effectiveness of generated specs in hundreds of open-source Clojure projects.

%Third, I will conduct a larger scale investigation of Clojure usage patterns by
%repurposing my automatic annotation tool to generate clojure.spec annotations (``specs'')
%and subsequently use them to enforce.
%I will outline clojure.spec, the official runtime verification
%library bundled with Clojure, and present a formal model of clojure.spec that highlights its
%``generative testing'' function checking semantics.
%Next, I will discuss how to extend my annotation tool to generate specs.
%Finally, I will automatically generate specs for hundreds of open-source Clojure projects,
%and use this data to investigate general questions like the effectiveness of unit and generative testing,
%the evolution of code over time, and the prevalence of idioms that Typed Clojure and clojure.spec
%have been designed around.


\chapter*{Acknowledgements}
I first touched down in Bloomington ready to start my graduate career---that is, until 
my puzzled cab driver Juannita informed me with a gasp
we were in Bloomington, \emph{Illinois}.
She kindly agreed to drive me to IU, a 12-hour roundtrip---a humbling start to graduate school.
The next day, reconnecting with Dan Friedman and the inspirational Will Byrd came full circle.
Thanks Jason Hemann, Cam \& Rebecca Swords, Jaime Guerrero, 
and Tori Vollmer for helping me find my feet.

Since then, Andrew Kent and I have done most things together, like deciphering 
set-theoretic types, floating down a Eugene river,
weekly Zelda gaming, and dissertating.
Andrew and his delightful family took me under their wing---thanks
Carrie, Charlotte, Harrison, Sydney, and Dwight.

Thanks to my donators, talk attendees, and users, including
Chas Emerick, Brandon Bloom, David Nolen,
Ghadi Shayban, Nicola Mometto, Devin Walters, Reid McKenzie, Nathan Sorenson, Eric
Normand, Craig Andera, Jim Duey, Kyle Kingsbury, Colin Fleming, Paula Gearon,
Nikko Patten, and Claire Alvis who were very encouraging.
CircleCI enabled discussion of Typed Clojure ``in practice,'' and their post-mortem
summarized my frustrations with Typed Clojure.

Sam taught me how (and when) to turn my implementations into research.
He has been supportive, trusting, and always game to clarify my bizarre ideas.
Mike Vollmer helped brainstorm the annotator;
I'm grateful for Mike's presense and support. I fondly remember our
conversations, nagivating Mumbai together, and bad-movie nights.
The annotator took shape at weekly meetings---thanks
Matteo Cimini, Rajan Walia, Spenser Bauman, Jeremy Siek, Mike Vitousek, Deyaaeldeen Almahallawi,
Caner Derici, Sarah Spall, and David Christiansen for your attention and suggestions.

Symbolic closures were the result of many conversations.
Andre Kuhlenschmidt's interest in them during my final PL Wonks talk
helped me through the last weeks before my defense.
Thanks to Wonks regulars  
Ryan Scott, Kyle Carter, 
Vikraman Choudhury, Matthew Heimerdinger, Paulette Koronkevich, Aaron Hsu, Praveen Narayanan,
and Chaitanya Koparkar for your engagement.

Thanks to my committee Ryan, Larry, and Ken for your feedback.
I found Ryan's compilers course and Larry's logic course invigorating,
and I'm grateful to Ken for always asking the right questions.

Thanks to my friends and family, especially Mum, Dad, and Aaron.
Having a partner with experience in Computer Science research has been wonderful, 
and my wife Marcela Poffald is my greatest advocate.
Her feedback has improved and clarified my work,
and her continuous requests
to be referred to as ``Mrs.~Typed Clojure'' with a matching cap never
fail to make me blush.


\tableofcontents

\newpage

%turns on chapter numbering, resets page numbering and uses arabic numerals for page numbers;
\mainmatter

\chapter{Introduction}

\section{My Thesis}

\emph{Typed Clojure is a sound and practical optional type system for Clojure.}

\section{Structure of this Dissertation}

This document progresses in several parts that support my thesis statement.

\partref{part:types} motivates and presents the design of Typed Clojure.
It addresses both parts of my thesis statement.

\begin{itemize}
  \item \emph{Typed Clojure is sound} I formalize Typed Clojure, including
    its characteristic features like hash-maps, multimethods, and Java interoperability,
    and prove the model type sound.
  \item \emph{Typed Clojure is practical} 
      I present an empirical study of real-world Typed Clojure usage
        in over 19,000 lines of code, showing its features correspond to actual usage patterns.
\end{itemize}

The results and industry feedback of this work inspired three distinct research directions
to help improve the experience of using Typed Clojure.

\begin{itemize}
  \item
\partref{part:autoann} presents a solution to lower the annotation burden in real-world Typed Clojure programs.
I formalize and implement a tool to automatically annotate types for top-level
user and library definitions, and empirically study the manual changes needed for the generated annotations
to pass type checking.
  \item
\partref{part:implementations} describes the design and implementation of a 
new code analyzer for Clojure, in service of enabling user-provided type rules for Clojure macros
    to help make type checking complex macro usages more robust.
\item \partref{part:symbolic-closures} motivates and describes \emph{symbolic closure types},
      a technique that enhances type checking with symbolic execution, that helps check some
      common Clojure idioms via a compatible extension of Typed Clojure's original design.
\end{itemize}

Finally, \partref{part:related-future-work} presents the related work and future directions for each part.

\section{Previously Published Work}

\partref{part:types} has been published:

\begin{itemize}
  \item Ambrose Bonnaire-Sergeant, Rowan Davies, and Sam Tobin-Hochstadt. 
        Practical Optional Types for Clojure. In
        \emph{Proceedings of the 25th European Symposium on Programming}, 2016.
        (ESOP '16)
\end{itemize}

\partref{part:autoann} is in submission:

\begin{itemize}
  \item Ambrose Bonnaire-Sergeant, and Sam Tobin-Hochstadt.
        Squash the work: A Workflow for Typing Untyped Programs that use Ad-Hoc Data Structures.
        \emph{In Submission}
\end{itemize}


\part{Practical Optional Types for Clojure}
\label{part:types}

\chapter{Abstract}

Typed Clojure is an optional type system for Clojure, a dynamic
language in the Lisp family that targets the JVM. Typed Clojure
enables Clojure programmers to gain greater confidence in the
correctness of their code via static type checking while remaining in
the Clojure world, and has acquired significant adoption in the
Clojure community. Typed Clojure repurposes Typed Racket's
\emph{occurrence typing}, an approach to statically reasoning about
predicate tests, and also includes several new type system features to
handle existing Clojure idioms.

In this part, we describe Typed Clojure and present these type system
extensions, focusing on three features widely used in Clojure. 
%
 First, multimethods provide extensible
operations, and their Clojure semantics turns out to have a surprising
synergy with the underlying occurrence typing framework.
%
Second, Java
interoperability is central to Clojure's mission but introduces
challenges such as ubiquitous \texttt{null}; Typed Clojure handles
Java interoperability while ensuring the absence of null-pointer
exceptions in typed programs. 
%
Third, Clojure programmers
idiomatically use immutable dictionaries for data structures; Typed
Clojure handles this with multiple forms of
heterogeneous dictionary types.
%

We provide a formal model of the Typed Clojure type system
incorporating these and other features, with a proof of
soundness. Additionally, Typed Clojure is now in use by numerous
corporations and developers working with Clojure, and we present
a quantitative analysis on the use of type system features
in two substantial code bases.


%% Clojure is a dynamically typed language hosted on the Java 
%% Virtual Machine.
%% Typed Racket is a valuable starting point for
%% a gradual type system that targets Clojure.
%% Building a similar type system for a new language gives the
%% designer some flexibility to repurpose and extend features.
%% This paper gives an overview of Typed Clojure, concentrating
%% on the extensions and differences from Typed Racket. We also
%% show where Typed Racket's features were particularly useful
%% for type checking non-trivial Clojure idioms.

\chapter{Background: Clojure with static typing}

% current situation 

The popularity of dynamically-typed languages in software
development, combined with a recognition that types often improve
programmer productivity, software reliability, and performance, has
led to the recent development of a wide variety of optional and
gradual type systems aimed at checking existing programs written in
existing languages.  These include  TypeScript~\cite{typescript} and Flow~\cite{flow} for
JavaScript, Hack~\cite{hack} for PHP, and mypy~\cite{mypy}
for Python among the optional systems, and Typed Racket~\cite{TF08}, Reticulated
Python~\cite{Vitousek14}, and Gradualtalk~\cite{gradualtalk} among gradually-typed systems.\footnote{We
  use ``gradual typing'' for systems like Typed Racket with sound
  interoperation between typed and untyped code; Typed Clojure or
 TypeScript which don't
  enforce type invariants we describe as ``optionally typed''.}

One key lesson of these systems, indeed a lesson known to early
developers of optional type systems such as Strongtalk, is that type
systems for existing languages must be designed to work with the
features and idioms of the target language. Often this takes the form
of a core language, be it of functions or classes and objects,
together with extensions to handle distinctive language features.


We synthesize these lessons to present \emph{Typed Clojure}, an
optional type system for Clojure. 
%
Clojure is a dynamically
typed language in the Lisp family---built on the Java Virtual
Machine (JVM)---which has recently gained popularity as an alternative
JVM language.  It offers the flexibility of a Lisp dialect, including
macros, emphasizes a functional style via
immutable data structures, and provides
interoperability with existing Java code, allowing programmers to use
existing Java libraries without leaving Clojure.
%
Since its initial release in 2007, Clojure has been widely adopted for
``backend'' development in places where its support for parallelism,
functional programming, and Lisp-influenced abstraction is desired on
the JVM. As a result, there is an extensive base of existing untyped
programs whose developers can benefit from Typed Clojure,
an experience we discuss in this paper.

Since Clojure is a language in the
Lisp family, we apply the lessons of Typed Racket, an existing gradual type
system for Racket, to the core of Typed Clojure, consisting of an extended
$\lambda$-calculus over a variety of base types shared between all Lisp systems.
%
Furthermore, Typed Racket's \emph{occurrence typing} has proved
necessary for type checking realistic Clojure programs.

However, Clojure goes beyond Racket in many ways, requiring several
new type system features which we detail in this paper.
%
Most significantly, Clojure supports, and Clojure developers use,
\textbf{multimethods} to structure their code in extensible
fashion. Furthermore, since Clojure is an untyped language, dispatch
within multimethods is determined by application of dynamic predicates
to argument values. 
%
Fortunately, the dynamic dispatch used by multimethods has surprising
symmetry with the conditional dispatch handled by occurrence
typing. Typed Clojure is therefore able to effectively handle complex
and highly dynamic dispatch as present in existing Clojure programs. 

But multimethods are not the only Clojure feature crucial to type
checking existing programs. As a language built on the Java Virtual
Machine, Clojure provides flexible and transparent access to existing
Java libraries, and \textbf{Clojure/Java interoperation} is found in almost
every significant Clojure code base. Typed Clojure therefore builds in
an understanding of the Java type system and handles interoperation
appropriately. Notably, \texttt{null} is a distinct type in Typed Clojure,
designed to automatically rule out null-pointer exceptions.

An example of these features is given in
\figref{fig:ex1}. Here, the \clj{pname} multimethod dispatches
on the \clj{class} of the argument---for \clj{String}s,
the first method implementation is called, for \clj{File}s, the
second. The \clj{String} method calls
a \clj{File} constructor, returning a non-nil \clj{File} instance---the 
\clj{getName} method 
on \clj{File} requires a non-nil target, returning a nilable
type.  
%Typed Clojure offers an opt-in mode that
%resolves JVM overloading, avoiding expensive runtime reflective calls.

Finally, flexible, high-performance immutable dictionaries
are the most common Clojure data structure.
Simply treating them as uniformly-typed
key-value mappings would be insufficient for existing
programs and programming styles. Instead, Typed Clojure provides a
flexible \textbf{heterogenous map} type, in which specific entries can be specified. 

While these features may seem disparate, they are unified in important
ways. First, they leverage the type system mechanisms
inherited from Typed Racket---multimethods when using 
dispatch via predicates, Java interoperation for handling
\texttt{null} tests, and heterogenous maps using union types and
reasoning about subcomponents of data. Second,
they are crucial features for handling Clojure code in
practice. Typed Clojure's use in real Clojure deployments would not be
possible without effective handling of these three Clojure features. 

\begin{figure*}[t!]
  %\normalsize
\begin{cljlisting}
(*typed ann pname [(U File String) -> (U nil String)] typed*)
(defmulti pname class)  ; multimethod dispatching on class of argument
(defmethod pname String [s] (*invoke pname (*interop new File s interop*) invoke*)) ; String case 
(defmethod pname File [f] (*interop .getName f interop*)) ; File case, static null check
(*invoke pname "STAINS/JELLY" invoke*) ;=> "JELLY" :- (U nil Str)
\end{cljlisting}
\caption{A simple Typed Clojure program (delimiters: {\color{interop}Java interoperation (green)}, 
  {\color{types}type annotation (blue)},
  {\color{invoke}function invocation (black)}, {\color{red}collection literal (red)}, {\color{mygray}other (gray)})}
\label{fig:ex1}
\end{figure*}


\section{Contributions}

Our main contributions are as follows:

\begin{enumerate}
  \item We motivate and describe  Typed Clojure, an optional
    type system for Clojure that understands existing Clojure idioms.
  \item We present a sound formal model for three crucial type
    system features: multi-methods, Java
    interoperability, and heterogenous maps.
  \item We evaluate the use of Typed Clojure features on existing
    Typed Clojure code, including both open source and in-house systems.
\end{enumerate}

%
%
%%% \begin{figure}
%%% \inputminted[firstline=5]{clojure}{code/demo/src/demo/parent2.clj}
%%% \caption{A simple Typed Clojure program}
%%% \label{fig:ex1}
%%% \end{figure}
%
%%% Figure~\ref{fig:ex1} presents a simple program demonstrating many
%%% aspects of our system, from simple type annotations to explicit
%%% handling of Java's \java{null} (written \clj{nil}) in interoperation, as well as an
%%% extended form of occurrence typing and method resolution of
%%% Java interoperability based on static type information.
%
%%% The \clj{parent} function has the type 
%%% $$
%%% \clj{['{:file (U nil File)} -> (U nil Str)]}
%%% $$
%%% which means that it takes a hash table whose \clj{:file} key maps to either
%%% \clj{nil} or a \clj{File}, and it produces either \clj{nil} or a
%%% \clj{String}. The \clj{parent} function uses the \clj{:file} keyword
%%% as an accessor to get the file, checks that it isn't \clj{nil}, and
%%% then obtains the parent by making a Java method call.

\noindent
 The remainder of this part begins with an example-driven
 presentation of the main type system features in
 \secref{sec:overview}. We then incrementally present a core calculus
 for Typed Clojure covering all of these features together in
 \secref{sec:formal} and prove type soundness
 (\secref{sec:metatheory}). We then 
 %discuss the full implementation of
 %Typed Clojure, \coretyped{}, which extends the formal model in many ways, 
 present an empirical analysis of significant code bases written
 in \coretyped{}---the full implementation of Typed Clojure---in \secref{sec:experience}. 
 Finally, we discuss related work and conclude.


\chapter{Overview of Typed Clojure}

\label{sec:overview}

We now begin a tour of the central features of Typed Clojure,
beginning with Clojure itself. Our presentation
uses the full Typed Clojure system to illustrate key type system
ideas,\footnote{Full examples: \url{https://github.com/typedclojure/esop16}} before studying the core features in detail in
\secref{sec:formal}.

\section{Clojure}

Clojure~\cite{Hic08} is a Lisp that runs on the
Java Virtual Machine with support for concurrent programming
and immutable data structures in a mostly-functional
style.
%, restricting imperative updates to a limited set of
%structures each with specific thread synchronization behaviour. 
%Fast implementations of immutable vectors, and hash tables are featured,
%and a means for defining new records.
%
Clojure provides easy interoperation with existing Java libraries, with Java values being like any other Clojure value. 
However, this smooth interoperability comes at the cost of pervasive \java{null}, which leads to the possibility of null pointer exceptions---a drawback we address in Typed Clojure.

%\paragraph{Clojure Syntax}
%
%We describe new syntax as they appear in each example, but
%begin with include the essential basics of Clojure syntax.
%
%\clj{nil} is exactly Java's \java{null}.
%Parentheses indicate \emph{applications}, brackets
%delimit
%\emph{vectors}, braces
%delimit
%\emph{hash-maps}
%and double quotes delimit \emph{Java strings}.
%\emph{Symbols} begin with an alphabetic character,
%and a colon prefixed symbol like \clj{:a} is a \emph{keyword}.
%
%\emph{Commas} are always \emph{whitespace}.

\section{Typed Clojure}

A simple one-argument function \clj{greet} is annotated with \clj{ann} to take and return strings.

\begin{lstlisting}[language=Clojure]
(*typed ann  greet [Str -> Str] typed*)
(defn greet [n] (*invoke str "Hello, " n "!" invoke*))
(*invoke greet "Grace" invoke*) ;=> "Hello, Grace!" :- Str
\end{lstlisting}

Providing \clj{nil} (exactly Java's \java{null})
is a static type error---\clj{nil} is not a string.

\begin{lstlisting}[language=Clojure]
(*invoke greet nil invoke*) ; Type Error: Expected Str, given nil
\end{lstlisting}

\paragraph{Unions} To allow \clj{nil}, we use \emph{ad-hoc unions} (\clj{nil} and \clj{false}
are logically false).

\begin{lstlisting}[language=Clojure]
(*typed ann  greet-nil [(U nil Str) -> Str] typed*)
(defn greet-nil [n] (*invoke str "Hello" (when n (*invoke str ", " n invoke*)) "!" invoke*))
(*invoke greet-nil "Donald" invoke*) ;=> "Hello, Donald!" :- Str 
(*invoke greet-nil nil invoke*)      ;=> "Hello!"         :- Str
\end{lstlisting}
%

Typed Clojure prevents well-typed code from dereferencing \clj{nil}.
%This is important for Clojure programs---\clj{nil}
%is treated like any other distinct datum in Clojure.

\paragraph{Flow analysis} Occurrence typing~\cite{TF10}
models type-based control flow.
In \clj{greetings}, a branch ensures \clj{repeat}
is never passed \clj{nil}.

\begin{lstlisting}[language=Clojure]
(*typed ann  greetings [Str (U nil Int) -> Str] typed*)
(defn greetings [n i]
  (*invoke str "Hello, " (when i (*invoke apply str (*invoke repeat i "hello, " invoke*) invoke*)) n "!" invoke*))
(*invoke greetings "Donald" 2 invoke*)  ;=> "Hello, hello, hello, Donald!" :- Str
(*invoke greetings "Grace" nil invoke*) ;=> "Hello, Grace!"                :- Str
\end{lstlisting}

Removing the branch is a static type error---\clj{repeat} cannot be passed \clj{nil}.

\begin{lstlisting}[language=Clojure]
(*typed ann  greetings-bad [Str (U nil Int) -> Str] typed*)
(defn greetings-bad [n i]           ; Expected Int, given (U nil Int)
  (*invoke str "Hello, " (*invoke apply str (*invoke repeat i "hello, " invoke*) invoke*) n "!" invoke*))
\end{lstlisting}


%\subsection{Type System Basics}
%
%\cite{TF10}
%presented Typed Racket with occurrence typing,
%a technique for deriving type information from conditional control flow.
%They introduced the concept of occurrence typing 
%with the following example.
%
%\inputminted[firstline=3]{racket}{code/tr/example1.rkt}
%
%This function takes a value that is either \emph{\#f} % mintinline really hates #
%or a number, represented by an \emph{untagged} union type.
%The `then' branch has an implicit invariant
%that \rkt{x} is a number, which is automatically inferred with occurrence typing
%and type checked without further annotations.
%
%We chose to build on the ideas and implementation
%of Typed Racket to implement a type system targeting Clojure for several reasons.
%Initially, the similarities between Racket and Clojure drew us to
%investigate the effectiveness of repurposing occurrence typing
%for a Clojure type system---both languages share a Lisp heritage,
%similar standard functions 
%(for instance \clj{map}
%in both languages is variable-arity)
%and idioms.
%While Typed Racket is gradually typed and has sophisticated
%dynamic semantics for cross-language interaction, we 
%chose to first implement
%the static semantics
%with the hope to extend Typed Clojure to be gradually typed at a future date.
%Finally,
%Typed Racket's combination of bidirectional checking
%and occurrence typing presents a successful model for 
%type checking dynamically typed programs without compromising
%soundness, which is appealing over success typing~\cite{Lindahl:2006:PTI}
%which cannot prove strong properties about programs
%and soft typing~\cite{CF91}
%which has proved too complicated in practice.
%
%Here is the above program in Typed Clojure.
%\begin{exmp}
%\inputminted[firstline=5]{clojure}{code/demo/src/demo/eg1.clj}
%\label{example:conditionalflow}
%\end{exmp}
%
%The \clj{fn} macro (provided by core.typed) supports optional annotations by 
%adding
%\clj{:-} and a type after a parameter
%position
%or binding vector 
%to annotate parameter types
%and return types respectively.
%\clj{number?} is
%a Java \java{instanceof} test of \clj{java.lang.Number}.
%As in Typed Racket, \clj{U} creates an \emph{untagged union} type, which can take
%any number of types.
%
%Typed Clojure can already check all of the examples in~\cite{TF10}---the 
%rest of this section describes the extensions necessary
%to check Clojure code.


\section{Java interoperability}
\label{sec:overviewjavainterop}

Clojure can interact with Java constructors, methods, and fields.
This program calls the \clj{getParent} on a constructed
\clj{File}
instance, returning a nullable string.

\begin{exmp}
\begin{lstlisting}[language=Clojure]
(*interop .getParent (*interop new File "a/b" interop*) interop*)  ;=> "a" :- (U nil Str)
\end{lstlisting}
\label{example:getparent-direct-constructor}
\end{exmp}
%
Typed Clojure can integrate with the Clojure compiler to avoid expensive reflective 
calls like \clj{getParent}, however if a specific overload cannot be found based on the
surrounding static context, a type error is thrown.

\begin{lstlisting}[language=Clojure]
(fn [f] (*interop .getParent f interop*)) ; Type Error: Unresolved interop: getParent
\end{lstlisting}

Function arguments default to \clj{Any}, which is similar to a union of all types. Ascribing
a parameter type allows Typed Clojure to find a specific method.

%Calls to Java methods and fields have prefix notation
%like \clj{(.method target args*)} and \clj{(.field target)} respectively,
%with method and field names prefixed with a dot and methods taking some number of arguments.

\begin{exmp}
\begin{lstlisting}[language=Clojure]
(*typed ann parent [(U nil File) -> (U nil Str)] typed*)
(defn parent [f] (if f (*interop .getParent f interop*) nil))
\end{lstlisting}
\label{example:parent-if}
\end{exmp}

%\begin{exmp}
%\inputminted[firstline=18,lastline=19]{clojure}{code/demo/src/demo/parent3.clj}
%\end{exmp}

The conditional guards from dereferencing \clj{nil}, and---as before---removing 
it is a static type error, as typed code could possibly dereference \clj{nil}.

\begin{lstlisting}[language=Clojure]
(defn parent-bad-in [f :- (U nil File)]
  (*interop .getParent f interop*)) ; Type Error: Cannot call instance method on nil.
\end{lstlisting}

Typed Clojure rejects programs that assume methods cannot return \clj{nil}.

\begin{lstlisting}[language=Clojure]
(defn parent-bad-out [f :- File] :- Str
  (*interop .getParent f interop*)) ; Type Error: Expected Str, given (U nil Str).
\end{lstlisting}

Method targets can never be \clj{nil}.
Typed Clojure also prevents passing \clj{nil} as Java method or
constructor arguments by default---this restriction can be
adjusted per method.

%
%Typed Clojure and Java treat \java{null} differently.
%In Clojure, where it is known as \clj{nil}, Typed Clojure assigns it an explicit type
%called \clj{nil}. In Java \java{null} is implicitly a member of any reference type.
%This means the Java static type \java{String} is equivalent to
%\clj{(U nil String)} in Typed Clojure.
%
%Reference types in Java are nullable, so to guarantee a method call does not
%leak \java{null} into a Typed Clojure program we
%must assume methods can return \clj{nil}.
%
In contrast, JVM invariants guarantee constructors return non-null.\footnote{\url{http://docs.oracle.com/javase/specs/jls/se7/html/jls-15.html\#jls-15.9.4}}
%
\begin{exmp}
\begin{lstlisting}[language=Clojure]
(*invoke parent (*interop new File s interop*) invoke*)
\end{lstlisting}
\end{exmp}


\section{Multimethods}

\label{sec:multioverview}

\emph{Multimethods} are a kind of extensible function---combining a \emph{dispatch function} with 
one or more \emph{methods}---widely used to define Clojure operations.

\paragraph{Value-based dispatch}
This simple multimethod takes a keyword (\clj{Kw}) and says hello in different languages.%, as
%specified by a keyword argument.

\begin{exmp}
\begin{lstlisting}[language=Clojure]
(*typed ann hi [Kw -> Str] typed*) ; multimethod type
(defmulti hi identity) ; dispatch function `identity`
(defmethod hi :en [_] "hello") ; method for `:en`
(defmethod hi :fr [_] "bonjour") ; method for `:fr`
(defmethod hi :default [_] "um...") ; default method
\end{lstlisting}
\label{example:hi-multimethod}
\end{exmp}

When invoked, the arguments are first supplied to the dispatch function---\clj{identity}---yielding
a \emph{dispatch value}. A method is then chosen
based on the dispatch value, to which the arguments are then passed to return a value.

\begin{lstlisting}[language=Clojure]
(*invoke map hi [*vec :en :fr :bocce vec*] invoke*) ;=> (*list "hello" "bonjour" "um..." list*)
\end{lstlisting}

For example, 
\clj{(*invoke hi :en invoke*)} evaluates to \clj{"hello"}---it executes
the \clj{:en} method
because 
\begin{itemize}
  \item \clj{(*invoke = (*invoke identity :en invoke*) :en invoke*)} is true
and 
    \item \clj{(*invoke = (*invoke identity :en invoke*) :fr invoke*)} is false.
\end{itemize}

Dispatching based on literal values enables certain forms of method
definition, but this is only part of the story for multimethod dispatch.

\paragraph{Class-based dispatch}
For class values, multimethods can choose methods based on subclassing
relationships.
%
Recall the multimethod from \figref{fig:ex1}. %, reproduced here.
%\begin{minted}{clj}
%(ann pname [(U File String) -> (U nil String)])
%(defmulti pname class)
%(defmethod pname String [s] (pname (new File s)))
%(defmethod pname File [f] (.getName f))
%\end{minted}
%
%Its dispatching function is
%\clj{class}, with two methods associated with dispatch values \clj{java.lang.String} and \clj{java.io.File}
%respectively.
%\noindent
The dispatch function \clj{class}
%---associated at multimethod creation with \clj{defmulti}---
dictates 
whether the \clj{String} or \clj{File} method is chosen.
%---both installed via \clj{defmethod}
%
The multimethod dispatch rules use
\clj{isa?}, a hybrid predicate which is both a subclassing check for classes and
an equality check for other values.

%(isa? (class "STAINS/JELLY") Object) ;=> true
%(isa? (identity :en) :fr) ;=> false
%(isa? (class (new File "JELLY")) String) ;=> false
\begin{lstlisting}[language=Clojure]
(*invoke isa? :en :en invoke*)       ;=> true
(*invoke isa? String Object invoke*) ;=> true
\end{lstlisting}

The current dispatch value and---in turn---each method's associated dispatch value
is supplied to \clj{isa?}. If exactly one method returns true, it is chosen.
For example,
  \clj{(*invoke pname "STAINS/JELLY" invoke*)}
picks the \clj{String} method because \clj{(*invoke isa? String String invoke*)}
is true, and
\clj{(*invoke isa? String File invoke*)}
is not.
%---\clj{(class "STAINS/JELLY")}
%is \clj{String}.
%
%The \clj{String} method body
%\clj{(pname (new File "STAINS/JELLY"))}
%chooses the \clj{File} method for opposite reasons.

%The following Typed Clojure program is semantically identical to figure~\ref{fig:ex1}.
%
%\begin{minted}{clj}
%(ann pname [(U Str File) -> (U nil Str)])
%(defn pname [x]
%  ; dispatch value calculated by applying dispatch
%  ; function `class` to argument `x`.
%  (cond
%    ; if (class x) subclasses String, but not File
%    (and (isa? (class x) String)
%         (not (isa? (class x) File)))
%    ; then choose the String method
%    (pname (new File x))
%
%    ; else if (class x) subclasses File, but not String
%    (and (isa? (class x) File)         
%         (not (isa? (class x) String)))
%    ; then choose the File method
%    (.getName x)
%    :else (throw (Exception. "No match"))))
%\end{minted}
%
%An unambiguous match leads to the corresponding method being applied to the arguments,
%giving the final result.

%\subsection{Multimethods}
%
%A multimethod in Clojure is a function with a \emph{dispatch
%function} and a \emph{dispatch table} of methods. Multimethods are created with {\clj{defmulti}}.
%\inputminted[firstline=5,lastline=6]{clojure}{code/demo/src/demo/rep.clj}
%The multimethod \clj{path} has type \clj{[Any -> (U nil String)]}, an initially empty \emph{dispatch table}
%and \emph{dispatch function} \clj{class}, a function that
%returns the class of its argument or \clj{nil} if passed \clj{nil}.
%
%We can use {\clj{defmethod}} to install a method to \clj{path}.
%\inputminted[firstline=7,lastline=7]{clojure}{code/demo/src/demo/rep.clj}
%Now the dispatch table maps
%the \emph{dispatch value} \clj{String} to the function
%\clj{(fn [x] x)}. 
%We add another method
%which maps
%\clj{File} to the function
%\clj{(fn [x] (.getPath x))}
%in the dispatch table.
%\inputminted[firstline=8,lastline=8]{clojure}{code/demo/src/demo/rep.clj}
%
%After installing both methods, the call 
%$$
%\clj{(path (new File "dir/a"))}
%$$
%dispatches to the second method we installed because
%$$
%\clj{(isa? (class "dir/a") String)}
%$$
%is true, and finally returns 
%$$
%\clj{((fn [x] (.getPath x)) "dir/a")}.
%$$

%We include the above sequence of definitions as \egref{example:rep}.
%
%\begin{Code}
%\begin{exmp}
%\inputminted[firstline=5,lastline=10]{clojure}{code/demo/src/demo/rep.clj}
%\label{example:rep}
%\end{exmp}
%\end{Code}
%
%Typed Clojure does not predict if a runtime dispatch will be successful---\clj{(path :a)} 
%type checks because \clj{:a} agrees with the parameter type \clj{Any},
%but throws an error at runtime.

%\paragraph{Multiple dispatch} \clj{isa?} is special with vectors---vectors of the
%same length recursively call \clj{isa?} on the elements pairwise.
%\begin{minted}{clojure}
%  (isa? [Keyword Keyword] [Object Object]) ;=> true
%\end{minted}
%
%\inputminted[firstline=6,lastline=23]{clojure}{code/demo/src/demo/eg7.clj}
%
%\egref{example:multidispatch}
%simulates multiple dispatch by dispatching on
%a vector containing the class of both arguments. \clj{open}
%takes two arguments which can be strings or files and returns
%a new file that concatenates their paths.
%
%We call three different \clj{File} constructors, each known at compile-time
%via type hints.
%Multiple dispatch follows the same kind of reasoning as \egref{example:incmap},
%except we update multiple bindings simultaneously.

\section{Heterogeneous hash-maps}

%Beyond primitives and Java objects, 
The most common way to represent compound data in Clojure 
are immutable hash-maps, typicially with keyword keys.
%
Keywords double as functions that
look themselves up in a map, or return \clj{nil} if absent.

\begin{exmp}
\begin{lstlisting}[language=Clojure]
(def breakfast {*map :en "waffles" :fr "croissants" map*})
(*invoke :en breakfast invoke*)    ;=> "waffles" :- Str
(*invoke :bocce breakfast invoke*) ;=> nil       :- nil
\end{lstlisting}
\label{example:breakfastcomplete}
\end{exmp}

\emph{HMap types} describe the most common usages of
keyword-keyed maps.

\begin{lstlisting}[language=Clojure]
breakfast ; :- (HMap :mandatory {:en Str, :fr Str}, :complete? true)
\end{lstlisting}

This says
\clj{:en} and \clj{:fr} are known entries mapped to strings,
and the map is fully specified---that is, no other entries exist---by \clj{:complete?} being \clj{true}.
%
HMap types default to partial specification.
\clj{'\{:en Str :fr Str\}} abbreviates:
\clj{(HMap :mandatory {:en Str, :fr Str})}.

%
\begin{exmp}
\begin{lstlisting}[language=Clojure]
(*typed ann lunch '{:en Str :fr Str} typed*)
(def lunch {*map :en "muffin" :fr "baguette" map*})
(*invoke :bocce lunch invoke*) ;=> nil :- Any ; less accurate type
\end{lstlisting}
\label{example:lunchpartial}
\end{exmp}
%(:en lunch)    ; :- Str
%;=> "muffin"
%(:fr lunch)    ; :- Str
%;=> "baguette"

\paragraph{HMaps in practice} The next example is extracted from a production system at CircleCI,
a company with a large production Typed Clojure system
(\secref{sec:casestudy} presents a case study and empirical
result from this code base).

%\newpage

\begin{minipage}{\linewidth}
\begin{exmp}
\begin{lstlisting}[language=Clojure]
(*typed defalias RawKeyPair ; extra keys disallowed
  (HMap :mandatory {:pub RawKey, :priv RawKey}, 
        :complete? true) typed*)
(*typed defalias EncKeyPair ; extra keys disallowed
  (HMap :mandatory {:pub RawKey, :enc-priv EncKey}, :complete? true) typed*)

(*typed ann enc-keypair [RawKeyPair -> EncKeyPair] typed*)
(defn enc-keypair [kp]
  (*invoke assoc (*invoke dissoc kp :priv invoke*) :enc-priv (*invoke encrypt (*invoke :priv kp invoke*) invoke*) invoke*))
\end{lstlisting}
\label{example:circleci}
\end{exmp}
\end{minipage}
%
%(ann enc-keypair [RawKeyPair -> EncKeyPair])
%(defn enc-keypair [{pk :priv :as kp}] ; original map is kp
%  (assoc (dissoc kp :priv)       ; remove unencrypted private key
%         :enc-priv (encrypt pk))) ; add encrypted private key
%
%\inputminted[firstline=10,lastline=22]{clojure}{code/demo/src/demo/key.clj}

As \clj{EncKeyPair} is fully specified, we remove extra keys like \clj{:priv}
via \clj{dissoc}, which returns a new map that is the first argument without the
entry named by the second argument. Notice removing \clj{dissoc} causes a type error.

\begin{lstlisting}[language=Clojure]
(defn enc-keypair-bad [kp] ; Type error: :priv disallowed
  (*invoke assoc kp :enc-priv (*invoke encrypt (*invoke :priv kp invoke*) invoke*) invoke*))
\end{lstlisting}

%\clj{enc-keypair} takes an unencrypted keypair and returns an encrypted keypair by
%dissociating the raw \clj{:priv} entry with \clj{dissoc}
%and associating an encrypted private key
%as \clj{:enc-priv} on an immutable map with \clj{assoc}.
%The expression \clj{(:priv kp)} shows that keywords are also 
%functions that look themselves up in a map returning the associated value or \nil{} if the key is missing.
%Since \clj{EncKeyPair} is \clj{:complete?}, Typed Clojure enforces the return type
%does not contain an entry \clj{:priv}, and would complain if the \clj{dissoc}
%operation forgot to remove it.

%\egref{example:absentkeys}
%is like \egref{example:circleci}
%except the \clj{:absent-keys} HMap option is used
%instead of \clj{:complete?},
%which takes a \emph{set literal} of keywords that do not appear in the map, written 
%with \emph{\#}-prefixed braces.
%The syntax \clj{(fn [{pkey :priv, :as kp}] ...)}
%aliases \clj{kp} to the first argument and \clj{pkey} to \clj{(:priv m)}
%in the function body.
%
%\begin{exmp}
%\inputminted[firstline=10,lastline=21]{clojure}{code/demo/src/demo/key2.clj}
%\label{example:absentkeys}
%\end{exmp}
%
%Since this example enforces that \clj{:priv} must not appear
%in a \clj{EncKeyPair}
%Typed Clojure would still complain if we forgot to \clj{dissoc} \clj{:priv}
%from the return value.
%Now, however we could stash the raw private key in another entry
%like \clj{:secret-key} which is not mentioned by the partial HMap \clj{EncKeyPair}
%without Typed Clojure noticing.

%\paragraph{Branching on HMaps} Finally, testing on HMap properties
%allows us to refine its type down branches. \clj{dec-map} takes an
%\clj{Expr}, traverses to its nodes and decrements their values by \clj{dec}, then
%builds the \clj{Expr} back up with the decremented nodes.
%
%\begin{exmp}
%\inputminted[linenos,firstnumber=1,firstline=15,lastline=27]{clojure}{code/demo/src/demo/hmap.clj}
%\label{example:decmap}
%\end{exmp}
%
%If we go down the then branch (line 4), since \clj{(= (:op m) :if)} is true
%we remove
%the \clj{:do} and \clj{:const}
%Expr's from the type of \clj{m} (because their respective \clj{:op} entries disagrees with \clj{(= (:op m) :if)})
%and we are left with an \clj{:if} Expr.
%On line 8,
%we instead strike out the \clj{:if} Expr since it contradicts \clj{(= (:op m) :if)} being false. 
%Line 9 know we can
%remove the \clj{:const} Expr from the type of \clj{m} because it contradicts \clj{(= (:op m) :do)} being true,
%and we know \clj{m} is a \clj{:do} Expr.
%Line 12 we strike out \clj{:do} because \clj{(= (:op m) :do)} is false,
%so we are left with \clj{m} being a \clj{:const} Expr.
%
%Section~\ref{sec:formalpaths} discusses how this automatic reasoning is achieved.

\section{HMaps and multimethods, joined at the hip}

HMaps and multimethods are the primary ways for representing
and dispatching on data respectively, and so are intrinsically linked.
As type system designers, we must
search for a compositional approach that can anticipate
any combination of these features.

Thankfully, occurrence typing, originally designed for reasoning about
\clj{if} tests, provides the compositional approach we need.
By extending the system with
a handful of rules based on HMaps and other functions, 
we can automatically cover both easy cases and those
that compose rules in arbitrary ways.

Futhermore, this approach extends to multimethod dispatch by reusing
occurrence typing's approach to conditionals
%, whose branching
%mechanism may appear complex, but
%can be understood in terms of the humble \clj{if} conditional. 
and
encoding a small number of rules to handle
the \clj{isa?}-based dispatch.
In practice, conditional-based control flow typing
extends to multimethod dispatch, and vice-versa.

We first demonstrate a very common, simple dispatch style,
then move on to deeper structural dispatching where occurrence typing's
compositionality shines.

\paragraph{HMaps and unions} Partially specified HMap's with a common dispatch key
combine naturally with ad-hoc unions.
An \clj{Order} is one of three kinds of HMaps.

%FIXME define defalias above and keyword singletons

\begin{lstlisting}[language=Clojure]
(*typed defalias Order "A meal order, tracking dessert quantities."
  (U '{:Meal ':lunch, :desserts Int} '{:Meal ':dinner :desserts Int}
     '{:Meal ':combo :meal1 Order :meal2 Order}) typed*)
\end{lstlisting}

The \clj{:Meal} entry is common to each HMap, always mapped to a known keyword singleton
type.
It's natural to dispatch on the \clj{class} of an instance---it's similarly
natural to dispatch on a known entry like \clj{:Meal}.

\begin{minipage}{\linewidth}
\begin{exmp}
\begin{lstlisting}[language=Clojure]
(*typed ann desserts [Order -> Int] typed*)
(defmulti desserts :Meal)  ; dispatch on :Meal entry
(defmethod desserts :lunch [o] (*invoke :desserts o invoke*))
(defmethod desserts :dinner [o] (*invoke :desserts o invoke*))
(defmethod desserts :combo [o] 
  (*invoke + (*invoke desserts (*invoke :meal1 o invoke*) invoke*) (*invoke desserts (*invoke :meal2 o invoke*) invoke*) invoke*))

(*invoke desserts {*map :Meal :combo, :meal1 {*map :Meal :lunch :desserts 1 map*}, 
           :meal2 {*map :Meal :dinner :desserts 2 map*} map*} invoke*) ;=> 3
\end{lstlisting}
\label{example:desserts-on-meal}
\end{exmp}
\end{minipage}

The \clj{:combo} method is verified to only structurally recur
on \clj{Order}s. This is achieved because we learn the argument \clj{o} must
be of type
\clj{'\{:Meal ':combo\}}
since
\clj{(isa? (:Meal o) :combo)}
is true. Combining this
with the fact that \clj{o} is an \clj{Order}
eliminates possibility of \clj{:lunch} and \clj{:dinner}
orders, simplifying \clj{o} to
\clj{'\{:Meal ':combo :meal1 Order :meal2 Order\}}
which contains appropriate arguments for both recursive calls.

\paragraph{Nested dispatch}
A more exotic dispatch mechanism for \clj{desserts}
might be on the \clj{class} of the \clj{:desserts} key.
If the result is a number, then we know the \clj{:desserts}
key is a number, otherwise the input is a \clj{:combo} meal.
We have already seen dispatch on \clj{class} and on keywords
in isolation---occurrence typing automatically understands
control flow that combines its simple building blocks.

The first method has dispatch value \clj{Long}, a subtype
of \clj{Int}, and the second method has \clj{nil}, the sentinel value for a failed map lookup.
In practice, \clj{:lunch} and \clj{:dinner} meals will dispatch to the \clj{Long}
method, but Typed Clojure infers a slightly more general type due to the definition
of \clj{:combo} meals.

\begin{minipage}{\linewidth}
\begin{exmp}
\begin{lstlisting}[language=Clojure]
(*typed ann desserts' [Order -> Int] typed*)
(defmulti desserts' 
  (fn [o :- Order] (*invoke class (*invoke :desserts o invoke*) invoke*)))
(defmethod desserts' Long [o] 
;o :- (U '{:Meal (U ':dinner ':lunch), :desserts Int}
;       '{:Meal ':combo, :desserts Int, :meal1 Order, :meal2 Order})
  (*invoke :desserts o invoke*))
(defmethod desserts' nil [o]
  ; o :- '{:Meal ':combo, :meal1 Order, :meal2 Order}
  (*invoke + (*invoke desserts' (*invoke :meal1 o invoke*) invoke*) (*invoke desserts' (*invoke :meal2 o invoke*) invoke*) invoke*))
\end{lstlisting}
\label{example:desserts-on-class}
\end{exmp}
\end{minipage}
%
%(desserts' {:Meal :combo 
%            :meal1 {:Meal :lunch :desserts 1}
%            :meal2 {:Meal :dinner :desserts 2}})
%;=> 3

In the \clj{Long} method, Typed Clojure learns that
its argument is at least of type \clj{'\{:desserts Long\}}, since
\clj{(*invoke isa? (*invoke class (*invoke :desserts o invoke*) invoke*) Long invoke*)}
must be true.
%
%Knowing \clj{o} is also an
%\clj{Order},
Here the
\clj{:desserts} entry
\emph{must} be present and mapped to a \clj{Long}---even in a \clj{:combo} meal,
which does not specify \clj{:desserts}
as present or absent.

In the \clj{nil} method,
\clj{(*invoke isa? (*invoke class (*invoke :desserts o invoke*) invoke*) nil invoke*)}
must be true---which implies
%
\clj{(*invoke class (*invoke :desserts o invoke*) invoke*)}
%
is \clj{nil}.
%
Since lookups on missing keys return \clj{nil}, either
\begin{itemize}
  \item \clj{o} maps the \clj{:desserts} entry to \clj{nil}, like the value \clj{\{:desserts nil\}}, or
  \item \clj{o} is missing a \clj{:desserts} entry.%, like \clj{{}}.
\end{itemize}
We can express this type with the \clj{:absent-keys} HMap option
%Equivalently, \clj{o} is of type
% Note: mintedinline doesn't work with hash characters #
\begin{lstlisting}[language=Clojure]
(U '{:desserts nil} (HMap :absent-keys #{:desserts}))
\end{lstlisting}
This eliminates non-\clj{:combo} meals
since their \clj{'\{:desserts Int\}} type does not agree
with this new information (because \clj{:desserts}
is neither \clj{nil} or absent).

%simplifies to a \clj{:combo} meal, 
%\begin{minted}{clojure}
%'{:Meal ':combo :meal1 Order :meal2 Order}
%\end{minted}
%thus allowing both recursive calls to type check.

\paragraph{From multiple to arbitrary dispatch}
Clojure multimethod dispatch, and Typed Clojure's handling of it, goes
even further, supporting dispatch on multiple arguments via vectors.
%
Dispatch on multiple arguments is beyond the scope of this paper,
but the same intuition applies---adding support for multiple dispatch
admits arbitrary combinations and nestings
of it and previous dispatch rules.

%\begin{exmp}
%\inputminted[firstline=6,lastline=13]{clojure}{code/demo/src/demo/hmap.clj}
%\label{example:decleaf}
%\end{exmp}
%
%The \clj{defn} macro defines a top-level function, with syntax like the typed \clj{fn}.
%The function \clj{an-exp} is verified to return an \clj{Expr}.
%
%Here \clj{defalias} defines \clj{Expr}, a type abbreviation
%that describes the structure of a recursively-defined AST as a union of HMaps.
%Keyword singleton types are quoted---\clj{':lunch}.
%A type that is a quoted map like \clj{'{:op ':if}} is a
%HMap type with a fixed number of keyword entries of the specified types
%known to be \emph{present},
%zero entries known to absolutely be \emph{absent},
%and an infinite number of \emph{unknown} entries entries.
%Since only keyword keys are allowed, they do not require quoting.

%\paragraph{HMap dispatch} The flexibility of \clj{isa?} is key to the generality of multimethods. 
%In \egref{example:incmap} we
%dispatch on the \clj{:op} key 
%of our HMap AST \clj{Expr}.
%Since keywords are functions that look themselves up in their argument, we simply
%use \clj{:op} as the dispatch function.
%
%\begin{exmp}
%\inputminted[firstline=5,lastline=18]{clojure}{code/demo/src/demo/eg5.clj}
%\label{example:incmap}
%\end{exmp}
%
%The function \clj{inc-leaf} is like \egref{example:decmap} except the nodes are incremented.
%The reasoning is similar, except we only consider one branch (the current method) by
%locally considering the current \emph{dispatch value} and reasoning about how it relates
%to the \emph{dispatch function}.
%For example, 
%in the \clj{:do} method we learn the \clj{:op} key is a \clj{:do}, which
%narrows our argument type to the \clj{:do} Expr, and similarly for the \clj{:if}
%and \clj{:const} methods.
%
%
%\subsection{Final example}
%
%\egref{example:final}
%combines everything we will cover for the rest of the paper:
%multimethod dispatch, reflection resolution via type hints, Java method
%and constructor calls, conditional and exceptional flow reasoning,
%and HMaps. 
%
%
%\begin{figure}
%\begin{exmp}
%\inputminted[firstline=6,lastline=23]{clojure}{code/demo/src/demo/eg7.clj}
%\label{example:multidispatch}
%\end{exmp}
%\begin{exmp}
%\inputminted[firstline=6,lastline=20]{clojure}{code/demo/src/demo/eg8.clj}
%\label{example:final}
%\end{exmp}
%\caption{Multimethod Examples}
%\end{figure}
%
%We dispatch on \clj{:p} to distinguish the two cases of \clj{FSM}---for example on \clj{:F}
%we know the \clj{:file} is a file.
%The body of the first method uses type hints to resolve reflection
%and conditional control flow to prove null-pointer exceptions are impossible.
%The second method is similar except it uses exceptional control flow.

\chapter{A Formal Model of \lambdatc{}}

\label{sec:formal}

After demonstrating the core features of Typed Clojure, 
we link them together in a formal model called
\lambdatc{}.
%
Building on occurrence typing,
we incrementally add each
novel feature of Typed Clojure to the formalism,
interleaving presentation of syntax, typing rules, operational semantics,
and subtyping.

\section{Core type system}
\label{sec:coretypesystem}

We start with a review of
occurrence typing~\cite{TF10}, the foundation of \lambdatc{}.
%We build up the occurrence typing calculus for illustrative purposes, 
%and present the full syntax at the end of the section.

\paragraph{Expressions} Syntax is given in \figref{main:figure:termsyntax}. Expressions \e{} 
include variables \x{}, values \v{},
applications, abstractions, conditionals, and let expressions.
All binding forms introduce fresh variables---a subtle but important point since our type environments
are not simply dictionaries.
Values include booleans \bool{}, \nil{}, class literals {\class{}}, keywords \kw{},
integers {\nat{}},
constants {\const{}}, and strings \str{}. Lexical closures {\closure {\openv{}} {\abs {\x{}} {\ty{}} {\e{}}}}
close value environments \openv{}---which map bindings to values---over functions.

\paragraph{Types} Types \s{} or \ty{} 
include the top type \Top,
\emph{untagged} unions {\Unionsplice {\overrightarrow{\ty{}}}}, 
singletons ${\Value \singletonmeta{}}$,
and class instances \class{}.
We abbreviate the classes
\Booleanlong{} to \Boolean{}, 
\Keywordlong{} to \Keyword{},
\NumberFull{}  to \Number{},
\StringFull{}  to \String{}, and 
\FileFull{}  to \File{}.
We also abbreviate the types
\EmptyUnion{}     to \Bot{}, 
{\ValueNil}       to \Nil{}, 
{\ValueTrue}      to \True, and
{\ValueFalse} to {\False}.
%
The difference between the types
\Value{\class{}} and \class{} is subtle.
The former is inhabited by class literals like \Keyword{} and the result of 
\appexp{\classconst{}}{\makekw{a}}---the latter by \emph{instances} of classes,
like a keyword literal \makekw{a}, an instance of the type \Keyword{}.
%
Function types 
$
{\ArrowOne {\x{}} {\s{}}
             {\ty{}}
             {\filterset {\prop{}} {\prop{}}}
             {\object{}}}
$
contain \emph{latent} (terminology from~\cite{Lucassen88polymorphiceffect}) propositions \prop{}, object \object{}, and return type
\ty{},
which may refer to the function argument \x{}.
%Latent means they are relevant when the function is applied rather than evaluated.
They are instantiated with the
actual object of the argument in applications. % before they are used in the proposition environment.

\paragraph{Objects}
%As we saw in \secref{sec:overview},
%occurrence typing is capable of reasoning
%about deeply nested expressions.
Each expression is associated with 
a symbolic representation
called an \emph{object}.
%with respect to the current lexical environment. 
For example,
  variable \makelocal{m} has object \makelocal{m};
  $\appexpone{\ccclass{\appexp{\makekw{lunch}}{\makelocal{m}}}}$ has object ${\pth{\classpe{}}{\pth{\keype{\makekw{lunch}}}{\makelocal{m}}}}$; and $42$ has the \emph{empty} object \emptyobject{} since it is unimportant in our system.
%
\figref{main:figure:termsyntax} gives the syntax for objects \object{}---non-empty objects 
\pth{\pathelem{}}{\x{}} combine of a root variable \x{} and a \emph{path} \pathelem{},
which consists of
a possibly-empty sequence of \emph{path elements} (\pesyntax{}) applied right-to-left from the root variable.
We use two path elements---{\classpe{}} and {\keype{k}}---representing the results
of calling \classconst{} and looking up a keyword $k$, respectively.

\paragraph{Propositions with a logical system}
In standard type systems, association lists often
track the types of variables, like in LC-Let and LC-Local.
\begin{singlespacing}
\begin{mathpar}
\infer [LC-Let]
{ \judgementtwo {{\propenv{}}}
                {\e{1}} {\s{}}
  \\
  \judgementtwo {{\propenv{}},\x{} \mapsto {\s{}}}
                {\e{2}} {\ty{}}
}
{ 
  \judgementtwo {\propenv{}} 
            {\letexp{\x{}}{\e{1}}{\e{2}}} {\ty{}}
           }

\infer [LC-Local]
{ {\propenv{}}(\x{}) = {\ty{}}
}
{ \judgementtwo {\propenv{}} 
            {\x{}} {\ty{}}
           }
\end{mathpar}
\end{singlespacing}

Occurrence typing instead pairs \emph{logical formulas},
that can reason about arbitrary non-empty objects,
with a \emph{proof system}.
The logical statement {\isprop{\s{}}{\x{}}} says
variable $x$ is of type \s{}. 
%A \emph{logical system}
%must now \emph{prove}
%a variable's type.
\begin{singlespacing}
\begin{mathpar}
\infer [T0-Let]
{ \judgementtwo {{\propenv{}}}
                {\e{1}} {\s{}}
  \\
  \judgementtwo {{\propenv{}},\isprop{\s{}}{\x{}}}
                {\e{2}} {\ty{}}
}
{ 
  \judgementtwo {\propenv{}} 
            {\letexp{\x{}}{\e{1}}{\e{2}}}
            {\ty{}}
           }
%\judgementtwo{\isprop{\Number{}}{\x{}}}{\appexp{\inc{}}{\x{}}}{\Number}

\infer [T0-Local]
{ \inpropenv {\propenv{}} {\isprop {\ty{}} {\x{}}}}
{ \judgementtwo {\propenv{}} 
            {\x{}} {\ty{}}
           }
\end{mathpar}
\end{singlespacing}
In T0-Local, 
$
{ \inpropenv {\propenv{}} {\isprop {\ty{}}{\x{}}}}
$
appeals to the proof system to solve for \ty{}.
%says under logical assumptions {\propenv{}}, object {\pth{\pathelem{}}{\x{}}} is of type \ty{}.
%We later define the more general T-Local.

\begin{figure}[t!]
  %\footnotesize
$$
\begin{array}{lrll}
  \e{} &::=& \x{}
                      \alt \v{} 
                      \alt {\comb {\e{}} {\e{}}} 
                      \alt {\abs {\x{}} {\ty{}} {\e{}}} 
                      \alt {\ifexp {\e{}} {\e{}} {\e{}}}
                      %\alt {\trdiff{\doexp {\e{}} {\e{}}}}
                      \alt {\letexp {\x{}} {\e{}} {\e{}}}
                      %\alt {\errorvalv{}}
                      &\mbox{Expressions} \\
  \v{} &::=&          \singletonmeta{}
                      \alt {\nat{}}
                      \alt {\const{}}
                      \alt {\str{}}
                      \alt {\closure {\openv{}} {\abs {\x{}} {\ty{}} {\e{}}}}
                &\mbox{Values} \\
                \constantssyntax{}\\
  \s{}, \ty{}    &::=& \Top 
                      \alt {\Unionsplice {\overrightarrow{\ty{}}}}
                      \alt
                      {\ArrowOne {\x{}} {\ty{}}
                                   {\ty{}}
                                   {\filterset {\prop{}} {\prop{}}}
                                   {\object{}}}
                      \alt {\Value \singletonmeta{}} 
                      \alt \trdiff{\class{}}
                &\mbox{Types} \\
  \singletonallsyntax{}
                \\ \\
  \occurrencetypingsyntax{}\\
  \propenvsyntax{}\\
  \openvsyntax{}
  %\\
  %\classliteralallsyntax{}
\end{array}
$$
\caption{Syntax of Terms, Types, Propositions and Objects}
\label{main:figure:termsyntax}
\end{figure}

We further extend logical statements to \emph{propositional logic}.
\figref{main:figure:termsyntax} describes the syntax
for propositions \prop{},
consisting of positive and negative \emph{type propositions} 
about non-empty objects---{\isprop {\ty{}} {\pth {\pathelem{}} {\x{}}}}
and {\notprop {\ty{}} {\pth {\pathelem{}} {\x{}}}}
respectively---the latter pronounced ``the object {\pth {\pathelem{}} {\x{}}} is \emph{not} of type \ty{}''.
The other propositions are standard logical connectives: implications, conjunctions,
disjunctions, and the trivial (\topprop{}) and impossible (\botprop{}) propositions.
%
The full proof system judgement
$
{ \inpropenv {\propenv{}} {\prop{}} }
$
says \emph{proposition environment} {\propenv{}} proves proposition \prop{}.

Each expression is associated with two propositions---when expression
\e{1} is in test position like
\ifexp{\e{1}}{\e{2}}{\e{3}},
the type system extracts \e{1}'s `then' and `else' proposition to check
\e{2} and \e{3} respectively.
For example, in \ifexp{\makelocal{o}}{\e{2}}{\e{3}}
we learn variable {\makelocal{o}} is true in \e{2} via {\makelocal{o}}'s `then' proposition $\notprop{\falsy{}}{\makelocal{o}}$, and 
that {\makelocal{o}} is false in \e{3} via {\makelocal{o}}'s `else' proposition $\isprop{\falsy{}}{\makelocal{o}}$.

To illustrate, recall \egref{example:desserts-on-meal}.
The parameter \makelocal{o} is of type $\Order$,
%by the annotation on $desserts$
written
{\isprop{\Order}{\makelocal{o}}}
as a proposition.
%
In the ${\makekw{combo}}$ method, we know
${\appexp{\makekw{Meal}}{\makelocal{o}}}$ is ${\makekw{combo}}$,
based on multimethod dispatch rules. This is written
  {\isprop{\Value{\makekw{combo}}}{\pth{\keype{\makekw{Meal}}}{\makelocal{o}}}},
pronounced ``the ${\makekw{Meal}}$ path of variable \makelocal{o} is of type
{\Value{\makekw{combo}}}''.

%\paragraph{Logical system in action} 
To attain the type of \makelocal{o}, 
we must solve for \ty{} in
$
{ \inpropenv 
  {\propenv{}}
  {\isprop {\ty{}} {\makelocal{o}}}}
$,
under proposition environment
$
\propenv{} = {{\isprop{\Order}{\makelocal{o}}},
    {\isprop{\Value{\makekw{combo}}}{\pth{\keype{\makekw{Meal}}}{\makelocal{o}}}}}
$
which deduces \ty{} to be a {\makekw{combo}} meal.
The logical
system \emph{combines} pieces of type information to deduce more accurate types for lexical
bindings---this is explained in \secref{formalmodel:proofsystem}.

%The first insight about occurrence typing is that
%logical formulas
%can be used to represent type information about our programs
%by relating parts of the runtime environment to types
%via propositional logic.
%\emph{Type propositions}  make assertions like ``variable \x{} is of type \NumberFull{}'' or
%``variable \x{} is not \nil{}''---in our logical system we write these as
%{\isprop{\NumberFull}{\x{}}}
%and {\notprop{\Nil{}}{\x{}}}. 
%
%The second insight is that we can replace the traditional 
%representation of a
%type environment (eg., a map from variables to types)
%with a set of propositions, written \propenv{}. 
%Instead of mapping \x{} to
%the type \NumberFull{}, we use the proposition {\isprop{\NumberFull}{\x{}}}.





\begin{figure*}[t]
%\footnotesize
    %{\TDo}
    %{\TClass}
    %{\TIf}
    %{\TAbs}
    %\begin{array}{c}
    %  {\TSubsume}\\\\
    %  {\TNum}
    %\end{array}
  \begin{mathpar}
        {\TLocal}
        {\TAbs}
        {\TIf}
    \\
    \begin{array}{c}
    {\TKw}\\
      {\TNum}\\
    \end{array}
    \begin{array}{c}
      {\TNil}\\
      {\TFalse}\\
    {\TConst}
    \end{array}
    \begin{array}{c}
    {\TStr}\\
    {\TClass}\\
    {\TTrue}
  \end{array}
        \\

    {\TLet}
    \\

    {\TApp}\ \ 
    {\TSubsume}
    \\
  \end{mathpar}
    %\begin{array}{c}
    %  {\TSubsume}\\\\
    %  {\TStr}\\\\
    %  {\TNil}\\\\
    %  {\TFalse}
    %\end{array}
  \caption{Core typing rules}
  \label{main:figure:othertypingrules}
\end{figure*}

\begin{figure}%[t!]
  %\footnotesize
  \begin{mathpar}

\SUnionSuper{}\ \ \ 
\SUnionSub{}\ \ \ 
\SFunMono{}\ \ \ 
\begin{array}{l}
\SObject{}\\
\SClass{}\\
\SSBool{}
\end{array}

\SFun{}
\begin{array}{l}
    \SRefl{}\ \ \ 
    \STop{}\\
\SSKw{}
\end{array}


  \end{mathpar}
  \caption{Core subtyping rules}
  \label{main:figure:subtyping}
\end{figure}

\begin{figure}
\begin{mathpar}
    \BIfTrue{}

    \BIfFalse{}
\end{mathpar}
  \caption{Select core semantics}
\label{main:figure:coresemantics}
\end{figure}

\paragraph{Typing judgment}

We formalize our system following Tobin-Hochstadt and Felleisen \cite{TF10}.
%(with differences highlighted in $\trdiff{\text{blue}}$)
The typing judgment 
$
{\judgementshowrewrite   {\propenv}
              {\e{}} {\ty{}}
  {\filterset {\thenprop {\prop{}}}
              {\elseprop {\prop{}}}}
  {\object{}}
  {\ep{}}}
$
says expression \e{} rewrites to \ep{}, which
is of type \ty{} in the 
proposition environment $\propenv{}$, with 
`then' proposition {\thenprop {\prop{}}}, `else' proposition {\elseprop {\prop{}}}
and object \object{}.
For ease of presentation, we omit \ep{} when it is easily inferred.

We write 
{\judgementtworewrite{\propenv}{\e{}} {\ty{}}{\ep{}} 
to mean 
{\judgementshowrewrite   {\propenv}
              {\e{}} {\ty{}}
  {\filterset {\thenprop {\propp{}}}
              {\elseprop {\propp{}}}}
  {\objectp{}}
  {\ep{}}}
for some {\thenprop {\propp{}}}, {\elseprop {\propp{}}}
and {\objectp{}}.
%and abbreviate self rewriting judgements
%{\judgementrewrite   {\propenv}
%              {\e{}} {\ty{}}
%  {\filterset {\thenprop {\prop{}}}
%              {\elseprop {\prop{}}}}
%  {\object{}}
%  {\e{}}}
%to
%{\judgementselfrewrite   {\propenv}
%              {\e{}} {\ty{}}
%  {\filterset {\thenprop {\prop{}}}
%              {\elseprop {\prop{}}}}
%  {\object{}}}.


\paragraph{Typing rules}

The core typing rules are
given as \figref{main:figure:othertypingrules}. We introduce
the interesting rules with the complement number predicate
as a running example.
\begin{equation}
\abs{\makelocal{d}}{\Top}{\ifexp{\appexp{\numberhuh{}}{\makelocal{d}}}{\false{}}{\true{}}}
\end{equation}
%, including
%a subsumption rule T-Subsume and rules for the false values---T-Nil and T-False---encoded 
%as impossible (\botprop{}) `then' propositions.

The lambda rule T-Abs introduces \isprop{\s{}}{\x{}}} = \isprop{\Top}{\makelocal{d}}
to check the body.
With \propenv{} = \isprop{\Top}{\makelocal{d}},
T-If first checks the test \e{1} = {\appexp{\numberhuh{}}{\makelocal{d}}}
via the T-App rule, with three steps.

First, in T-App the operator \e{} = \numberhuh{} is checked with T-Const, which
uses 
\constanttypeliteral{} (\figref{main:figure:constanttyping}, dynamic semantics in the supplemental material)
to type constants.
\numberhuh{} is a predicate over numbers, and
\classconst{} returns its argument's class.

Resuming {\appexp{\numberhuh{}}{\makelocal{d}}},
in T-App the operand \ep{} = \makelocal{d} is checked with
T-Local as
\begin{equation}
\judgementselfrewrite{\propenv{}}
                     {\makelocal{d}}
                     {\Top}
                     {\filterset{\notprop{\falsy}{\makelocal{d}}}
                                {\isprop{\falsy}{\makelocal{d}}}}
                     {\makelocal{d}}
\end{equation}
which encodes the type, proposition, and object information
about variables. The proposition {\notprop{\falsy}{\makelocal{d}}}
says ``it is not the case that variable {\makelocal{d}} is of type {\falsy}'';
{\isprop{\falsy}{\makelocal{d}}} says ``{\makelocal{d}} is of type {\falsy}''.

Finally, the T-App rule substitutes the operand's object \objectp{}
for the parameter \x{} in the latent type, propositions, and object. The proposition
{\isprop{\Number{}}{\makelocal{d}}} says ``{\makelocal{d}} is of type {\Number{}}'';
{\notprop{\Number{}}{\makelocal{d}}} says ``it is not the case that {\makelocal{d}}
is of type {\Number{}}''. The object {\makelocal{d}} is the symbolic representation
of what the expression {\makelocal{d}} evaluates to.
\begin{equation}
\judgementselfrewrite{\propenv{}}
  {\appexp{\numberhuh{}}{\makelocal{d}}}
  {\Boolean{}}
  {\filterset{\isprop{\Number{}}{\makelocal{d}}}
             {\notprop{\Number{}}{\makelocal{d}}}}
  {\emptyobject{}}
\end{equation}
To demonstrate, the `then' proposition---in T-App {\replacefor {\thenprop{\prop{}}} {\objectp{}} {\x{}}}---substitutes
the latent `then' proposition of \constanttype{\numberhuh{}} with 
\makelocal{d}, giving
{\replacefor {\isprop{\Number{}}{\x{}}} {\makelocal{d}} {\x{}}} =
{\isprop{\Number{}}{\makelocal{d}}}.

To check the branches of {\ifexp{\appexp{\numberhuh{}}{\makelocal{d}}}{\false{}}{\true{}}},
T-If
introduces \thenprop{\prop{1}} = \isprop{\Number{}}{\makelocal{d}}
to check \e{2} = {\false{}},
and \elseprop{\prop{1}} = \notprop{\Number{}}{\makelocal{d}}
to check 
\e{3} = \true{}.
%
The branches are first checked with T-False and T-True respectively,
the T-Subsume premises
\inpropenv {\propenv{}, {\thenprop {\prop{}}}} {\thenprop {\propp{}}}
and
\inpropenv {\propenv{}, {\elseprop {\prop{}}}} {\elseprop {\propp{}}}
allow us to pick compatible propositions for both branches.
%$$
%\judgementselfrewrite{\propenv{},{\isprop{\Number{}}{\makelocal{d}}}}
%  {\false{}}
%  {\False{}}
%  {\filterset{\botprop{}}
%             {\topprop{}}}
%  {\emptyobject{}}
%$$
$$
\begin{array}{c}
\judgementselfrewrite{\propenv{},{\isprop{\Number{}}{\makelocal{d}}}}
  {\false{}}
  {\Boolean{}}
  {\filterset{\notprop{\Number{}}{\makelocal{d}}}
             {\isprop{\Number{}}{\makelocal{d}}}}
  {\emptyobject{}}
  \\
\judgementselfrewrite{\propenv{},{\notprop{\Number{}}{\makelocal{d}}}}
  {\true{}}
  {\Boolean{}}
  {\filterset{\notprop{\Number{}}{\makelocal{d}}}
             {\isprop{\Number{}}{\makelocal{d}}}}
  {\emptyobject{}}
\end{array}
$$
%to suit the T-If outputs \ty{} = \Boolean{}, \thenprop{\prop{}}
%= {\notprop{\Number{}}{\makelocal{d}}}, \elseprop{\prop{}} = {\isprop{\Number{}}{\makelocal{d}}},
%and \object{} = {\emptyobject{}}.
%
%In T-Subsume, we can upcast \ty{} = \False{} to \tp{} = \Boolean{} via the premise 
%\issubtypein{}{\ty{}}{\tp{}}.
%and 
%\inpropenv {\propenv{}, {\elseprop {\prop{}}}} {\elseprop {\propp{}}}
%from 
%{\elseprop {\prop{}}} = \topprop{} to {\elseprop {\propp{}}} = {\isprop{\Number{}}{\makelocal{d}}}.

Finally T-Abs assigns a type to the overall function:
$$
{\judgementselfrewrite{}{\abs{\makelocal{d}}{\Top}{\ifexp{\appexp{\numberhuh{}}{\makelocal{d}}}{\false{}}{\true{}}}}
                    {\ArrowOne {\makelocal{d}} {\Top{}}
                                      {\Boolean{}}
                                      {\filterset {\notprop{\Number{}}{\makelocal{d}}}
                                                  {\isprop{\Number{}}{\makelocal{d}}}}
                                      {\emptyobject{}}}
                    {\filterset {\topprop{}}
                                {\botprop{}}}
                    {\emptyobject{}}}
$$

%The object \x{} over latent propositions \thenprop{\prop{}} and
%\elseprop{\prop{}}, latent object \object{}, and latent type \ty{}. The
%actual argument object---\objectp{}---is substituted in at application by T-App.



%The T-App rule instantiates parameters---like \x{} in T-Abs---with their actual object.
%The expression {\appexp{\numberhuh{}}{\makelocal{d}}} replaces its parameter object with \
%\judgementselfrewrite{\isprop{\Top}{\makelocal{d}}}{\appexp{\numberhuh{}}{\makelocal{d}}}
%  {\Boolean{}}
%  {\filterset{\isprop{\Number{}}{\makelocal{d}}}
%             {\notprop{\Number{}}{\makelocal{d}}}}
%  {\emptyobject{}}

%The T-If refines each branch based on the test---a
%{\appexp{\numberhuh{}}{\makelocal{d}}} test introduces \isprop{\Number{}}{\makelocal{d}}
%for checking the `then' branch and \notprop{\Number{}}{\makelocal{d}}
%for checking the `else' branch.
%%Information from each branch is combined via subsumption for the overall type, propositions, and object.
%The T-Subsume rule 

%For example
%$
%\judgementselfrewrite{\isprop{\Top}{\makelocal{d}}}
%  {\ifexp{\appexp{\numberhuh{}}{\makelocal{d}}}{\false{}}{\true{}}}
%  {\Boolean{}}
%  {\filterset{\notprop{\Number{}}{\makelocal{d}}}
%             {\isprop{\Number{}}{\makelocal{d}}}}
%  {\emptyobject{}}
%$
%type checks because T-Subsume allows us to check both branches as
%$
%\judgementselfrewrite{\isprop{\Number{}}{\makelocal{d}}}
%  {\false{}}
%  {\Boolean{}}
%  {\filterset{\notprop{\Number{}}{\makelocal{d}}}
%             {\isprop{\Number{}}{\makelocal{d}}}}
%  {\emptyobject{}}
%$
%and
%$
%\judgementselfrewrite{\notprop{\Number{}}{\makelocal{d}}}
%  {\true{}}
%  {\Boolean{}}
%  {\filterset{\notprop{\Number{}}{\makelocal{d}}}
%             {\isprop{\Number{}}{\makelocal{d}}}}
%  {\emptyobject{}}
%$
%respectively.
%
%For example, if \e{2} is true when variable {\makelocal{y}} is a \File{}
%and \e{3} is true when variable {\makelocal{z}} is a \Number{}, then
%we use T-Subsume on both branches
%to introduce a logical disjunction
%since at least one must be true if the entire expression is true.



%Given a set of propositions, we can use logical reasoning to derive
%new information about our programs
%with the judgment \inpropenv{\propenv{}}{\prop{}}.
%In addition to the standard rules for the logical connectives, the key
%rule is L-Update, which combines multiple propositions about the same variable,
%allowing us to refine its type.
%$$
%  {\LUpdate}
%$$
%For example, with L-Update we can use the knowledge of
%\inpropenv{\propenv{}}{\isprop{\UnionNilNum}{\x{}}}
%and 
%\inpropenv{\propenv{}}{\notprop{\Nil{}}{\x{}}}
%to derive \inpropenv{\propenv{}}{\isprop{\Number}{\x{}}}.
%(The metavariable \propisnotmeta{} ranges over \ty{} and \nottype{\ty{}} (without variables).)
%We cover L-Update in more detail in \secref{sec:formalpaths}.
%
%Finally, this approach allows the type system to track
%programming idioms from 
%dynamic languages
%using implicit type-based reasoning based on the result of
%conditional tests.
%For instance,
%\egref{example:parent-if}
%only utilizes \clj{f} once
%the programmer is convinced it is safe to do so based whether
%\clj{f}
%is
%true or false. 
%To express this in the type system, every expression 
%is described by two propositions: a `then' proposition
%for when it reduces to a true value, and an `else' proposition
%when it reduces to a false value---for \clj{f}
%the then proposition is {\notprop{\falsy}{f}} and 
%the else proposition is {\isprop{\falsy}{f}}.
%\ref{main:figure:typingrules}




%\figref{main:figure:typingrules} contains the core typing rules.
%The key rule for reasoning about conditional control flow is
%T-If. 
%
%\begin{mathpar}
%  {\TIf}
%\end{mathpar}

%The propositions of the test expression \e{1}, \thenprop{\prop{1}} and \elseprop{\prop{1}}, are 
%used as assumptions in the then and else branch respectively.

%The let rule T-Let links inferred information about
%\x{} to the expression used to instantiate \x{}, \ep{1}, via logical implications.
%
%The T-Local rule connects the type system to the proof system over type propositions
%via \inpropenv {\propenv{}} {\isprop {\ty{}} {\x{}}}
%to derive a type for a variable.
%Using this rule, the type system can then appeal to L-Update to refine the type
%assigned to \x{}.
%
%We are now equipped to type check
%\egref{example:parent-if}:
%$$
%\clj{(if f (.getParent f) nil)}
%$$
%
%With {\propenv{}} = {\isprop{\UnionNilFile{}}{f}},
%$$
%\judgement{\propenv{}}{f}{\UnionNilFile{}}{\localfilterset{f}}{f}
%$$
%via T-Local.
%
%Checking the then branch involves extending
%the proposition environment with {\notprop{\falsy}{f}}
%$$
%\judgement{{\propenv{}},{\isprop{\Number}{\x{}}}}{\x{}}{\Number{}}{\filterset{\notprop{\falsy{}}{\x{}}}{\isprop{\falsy{}}{\x{}}}}{\emptyobject{}}
%$$
%because we can now satisfy the premise of T-Local:
%$$
%\inpropenv{{\propenv{}},\isprop{\Number}{\x{}}}{\isprop{\Number}{\x{}}}.
%$$
%\judgement{{\propenv{}},\isprop{\Number}{\x{}}}{\hastype{\appexp{\inc{}}{\x{}}}{\Number{}}}{\filterset{\topprop{}}{\botprop{}}}{\emptyobject{}}
%$$
%$$
%\judgement{{\propenv{}},\notprop{\Number}{\x{}}}{\hastype{\zeroliteral{}}{\Number}}{\filterset{\topprop{}}{\botprop{}}}{\emptyobject{}}
%$$

%\inc{} has type
%$$
%{\ArrowOne{\x{}}{\Number}{\Number}
%        {\filterset{\topprop{}}{\topprop{}}}{\emptyobject{}}}
%$$
%We can now check the conditional with T-If.
%$$
%\judgement{\isprop{\Number}{\x{}}}{\hastype{\ifexp{\appexp{\numberhuh{}}{\x{}}}{\appexp{\inc{}}{\x{}}}{\zeroliteral{}}}{\Number}}{\filterset{\orprop{\isprop{\Number}{\x{}}}{\topprop{}}}{\orprop{\notprop{\Number}{\x{}}}{\topprop{}}}}{\emptyobject{}}
%$$
%Finally the function can be checked with T-Abs
%$$
%\judgement{}{\hastype{\abs{\x{}}{\UnionNilNum}{\ ...}}
%                                             {\ArrowOne{\x{}}{\UnionNilNum}{\Number}
%        {\filterset{\orprop{\isprop{\Number}{\x{}}}{\topprop{}}}{\orprop{\notprop{\Number}{\x{}}}{\topprop{}}}}{\emptyobject{}}}}
%  {\filterset{\topprop{}}{\botprop{}}}{\emptyobject{}}
%$$

\paragraph{Subtyping}
\figref{main:figure:subtyping} presents subtyping
as a reflexive and transitive relation with top type \Top. 
Singleton types are instances of their respective classes---boolean singleton types
are of type \Boolean{}, class literals are instances of \Class{} and keywords are
instances of \Keyword{}.
Instances of classes \class{} are subtypes of \Object{}. Function types 
are subtypes of \IFn{}. All types except for \Nil{} are subtypes of \Object{},
so \Top{} is similar to {\Union{\Nil}{\Object}}.
Function subtyping is contravariant left of the arrow---latent propositions, object
and result type are covariant.
Subtyping for untagged unions is standard.

\paragraph{Operational semantics} We define the dynamic semantics for \lambdatc{}
in a big-step style using an environment, following~\cite{TF10}.
We include both errors and a \wrong{} value, which is provably ruled out by the
type system.
The main judgment is \opsem{\openv{}}{\e{}}{\definedreduction{}}
which states that \e{} evaluates to answer \definedreduction{} in environment
\openv{}. We chose to omit the core rules (included in supplemental material)
however a notable difference is \nil{} is a false value, which affects the
semantics of \ifliteral{} (\figref{main:figure:coresemantics}).

%The definition of \updateliteral{} supports various idioms relating to \classpe{}
%which we introduce in \secref{sec:isaformal}.

%\begin{figure*}
%  \footnotesize
%%%   \judgbox{
%%%{\judgementrewrite   {\propenv}
%%%              {\e{}} {\ty{}}
%%%  {\filterset {\thenprop {\prop{}}}
%%%              {\elseprop {\prop{}}}}
%%%  {\object{}}{\ep{}}}}
%%%           {Under proposition environment $\propenv{}$, 
%%%             expression \e{} is of type \ty{}
%%%             with  \\
%%%
%%%`then' proposition {\thenprop {\prop{}}}, `else' proposition {\elseprop {\prop{}}}
%%%and object \object{} and rewrites to \ep{}.}
%  \begin{mathpar}
%    %{\TDo}
%    %{\TClass}
%    %{\TIf}
%    %{\TAbs}
%    %\begin{array}{c}
%    %  {\TSubsume}\\\\
%    %  {\TNum}
%    %\end{array}
%    \begin{array}{c}
%      {\TNum}\\\\
%      {\TConst}\\\\
%      {\TKw}\\\\
%      {\TClass}\\\\
%      {\TTrue}\\\\
%    \end{array}
%    \begin{array}{c}
%      {\TSubsume}\\\\
%      {\TNil}\\\\
%      {\TFalse}
%    \end{array}
%
%    %{\TLet}
%    %{\TLocal}
%
%    %{\TApp}
%    %{\TError}
%
%  \end{mathpar}
%  \caption{Typing rules}
%  \label{main:figure:typingrules}
%\end{figure*}

%\begin{figure}
%  \footnotesize
%  \begin{mathpar}
%    {\BLocal}
%
%    %{\BDo}
%
%    {\BLet}
%
%    \BVal{}
%
    %\BIfTrue{}

%    \BIfFalse{}
%
%    \BAbs{}
%
%    \BBetaClosure{}
%
%    \BDelta{}
%  \end{mathpar}
%  \caption{Operational Semantics}
%  \label{main:figure:standardopsem}
%\end{figure}

%\subsection{Reasoning about Exceptional Control Flow}
%\label{sec:doformal}
%
%Along with conditional control flow,
%Clojure programmers rely on \emph{exceptions}
%to assert type-related invariants.
%
%\begin{exmp}
%\inputminted[firstline=13,lastline=15]{clojure}{code/demo/src/demo/do.clj}
%\label{example:doexception}
%\end{exmp}
%
%The fully expanded increment function in~\egref{example:doexception}
%guards its final call with a number check, preventing
%a possible null-pointer exception.
%Without this check, the type system would reject the program.
%
%To check this example,
%occurrence typing 
%automatically
%assumes
%\clj{x} is a number when checking the second \clj{do} subexpression
%based on the first subexpression.
%\footnote{See \url{https://github.com/typedclojure/examples}
%  for full examples.}
%We model this formally %(section~\ref{sec:doformal}) 
%and prove
%null-pointer exceptions are impossible in typed code (section~\ref{sec:metatheory}).
%
%
%We extend our model with sequencing expressions and errors, where {\errorvalv{}}
%models the result of calling Clojure's \clj{throw} special form
%with some \clj{Throwable}.
%
%\smallskip
%$
%\begin{altgrammar}
%  \e{} &::=& \ldots \alt {\errorvalv{}} \alt {\doexp {\e{}} {\e{}}} &\mbox{Expressions} 
%\end{altgrammar}
%$
%
%\smallskip
%%
%%B-Do simply evaluates its arguments sequentially and returns the right argument.
%%Since errors are not values, we define error propagation semantics
%%like BE-Do1 (figure~\ref{appendix:figure:errorstuck} for the full rules).
%%
%%\begin{mathpar}
%%    {\BDo}
%%
%%\infer [BE-Error]
%%{}
%%{ \opsem {\openv{}} 
%%         {\errorvalv{}}
%%         {\errorvalv{}}}
%%
%%\infer [BE-Do1]
%%{ \opsem {\openv{}} {\e{1}} {\errorvalv{}} }
%%{ \opsem {\openv{}} {\doexp{\e{1}}{\e{}}} {\errorvalv{}}}
%%\end{mathpar}
%
%Our main insight is as follows: 
%if the first subexpression in a sequence reduces to a value, then it is either true or false.
%If we learn some proposition in both cases then we can use that proposition as an assumption to check the second subexpression.
%T-Do formalizes this intuition.
%
%\begin{mathpar}
%    {\TDo}  
%\end{mathpar}
%
%The introduction of errors, 
%which do not evaluate to either
%a true or false value,
%makes our insight interesting.
%
%\begin{mathpar}
%    {\TError}
%\end{mathpar}
%
%Recall \egref{example:doexception}.
%\begin{minted}{clojure}
%...
%  (do (if (number? x) nil (throw (new Exception)))
%      (inc x)) 
%...
%\end{minted}
%
%As before, checking \appexp{\numberhuh{}}{\x{}} allows us to use the proposition \isprop{\Number}{\x{}}
%when checking the then branch.
%
%By T-Nil and subsumption we can propagate this  information to both propositions.
%$$
%\judgement{\isprop{\Number}{\x{}}}{\nil{}}{\Nil{}}{\filterset{\isprop{\Number}{\x{}}}{\isprop{\Number}{\x{}}}}{\emptyobject{}}
%$$
%Furthermore, using T-Error
%and subsumption we can conclude anything in the else branch.
%$$
%\judgement{\notprop{\Number}{\x{}}}{\errorvalv{}}{\Bot}{\filterset{\isprop{\Number}{\x{}}}{\isprop{\Number}{\x{}}}}{\emptyobject{}}
%$$
%Using the above as premises to T-If we conclude that if the first
%expression in the \doliteral{} evaluates successfully, \isprop{\Number}{\x{}} must be true.
%$$
%\judgement{\isprop{\UnionNilNum}{\x{}}}
%          {\ifexp{\appexp{\numberhuh{}}{\x{}}}{\nil{}}{\errorvalv{}}}{\Boolean}
%          {\filterset{\isprop{\Number}{\x{}}}{\isprop{\Number}{\x{}}}}{\emptyobject{}}
%$$
%We can now use \isprop{\Number}{\x{}} in the environment to check the second subexpression
%{\appexp{\inc{}}{\x{}}}, completing the example.

\section{Java Interoperability}

\begin{figure}
  %\footnotesize
  $$
  \begin{altgrammar}
    \e{} &::=& \ldots \alt  {\fieldexp {\fld{}} {\e{}}} \alt {\methodexp {\mth{}} {\e{}} {\overrightarrow{\e{}}}}
                      \alt {\newexp {\class{}} {\overrightarrow{\e{}}}}
                      &\mbox{Expressions}\\
     &\alt& \nonreflectiveexpsyntax{} &\mbox{Non-reflective Expressions}\\

    \v{} &::=& \ldots \alt {\classvalue{\classhint{}} {\overrightarrow {\classfieldpair{\fld{}} {\v{}}}}}
    &\mbox{Values} \\

    \classtableallsyntax{}
  \end{altgrammar}
  $$
  \begin{mathpar}
    {\TNew}

    {\TMethod}

    {\TField}

    %{\TInstance}
  \end{mathpar}
 %\classtablelookupsyntax{}
 \begin{mathpar}
  \begin{altgrammar}
    \convertjavatypenil{}
  \end{altgrammar}
  \begin{altgrammar}
    \convertjavatypenonnil{}
  \end{altgrammar}
  \begin{altgrammar}
    \converttctype{}
  \end{altgrammar}
\end{mathpar}
  \begin{mathpar}
    \BField{}\ \ \ 
%
    \BNew{}

    \BMethod{}
  \end{mathpar}
  \caption{Java Interoperability Syntax, Typing and Operational Semantics}
  \label{main:figure:javatyping}
\end{figure}

\begin{figure}
  $$
\constanttypefigure{}
  $$
  \caption{Constant typing}
  \label{main:figure:constanttyping}
\end{figure}

We present Java interoperability in a restricted setting without class inheritance,
overloading or Java Generics.
%
We extend the syntax in \figref{main:figure:javatyping} with Java field lookups and calls to
methods and constructors. 
To prevent ambiguity between zero-argument methods and fields,
we use Clojure's primitive ``dot'' syntax:
field accesses are written \fieldexp{\fld{}}{\e{}}
and method calls $\methodexp{\mth{}}{\e{}}{\overrightarrow{e}}$.
%and \clj{(new class es*)} is $\newexp{\class{}}{\overrightarrow{es}}$.

In \egref{example:getparent-direct-constructor}, \clj{(*interop .getParent (*interop new File "a/b" interop*) interop*)}
translates to
\begin{equation}  \label{eq:unresolved}
  \qquad {\methodexp {\getparent{}} {\newexp {\File{}} {\makestr{a/b}}} {}}
\end{equation}

But both the constructor and method are unresolved.
We introduce \emph{non-reflective} expressions for specifying exact Java overloads.
\begin{equation} \label{eq:resolved}
\qquad {\methodstaticexp {\File} {} {\String} {\getparent{}} {\newstaticexp {\String} {\File{}} {\File{}} {\makestr{a/b}}} {}}
\end{equation}
From the left, the one-argument constructor for \File takes a \String, and the 
\getparent{} method of
\File{} takes zero arguments
and
returns a \String.

We now walk through this conversion.% from unresolved expression~\ref{eq:unresolved} to 
%resolved expression~\ref{eq:resolved}.

\paragraph{Constructors} First we check and convert {\newexp {\File{}} {\makestr{a/b}}} to {\newstaticexp {\String} {\File{}} {\File{}} {\makestr{a/b}}}.
The T-New typing rule checks and rewrites constructors.
%$$
%    {\TNew}
%$$
To check
{\newexp {\File{}} {\makestr{a/b}}}
we first resolve the constructor overload in the class table---there is at most one
to simplify presentation.
With \classhint{1} = \String,
we convert to a nilable type the argument with \ty{1} = \Union{\Nil}{\String}
and type check {\makestr{a/b}} against \ty{1}.
Typed Clojure defaults to allowing non-nilable arguments, but this
can be overridden, so we model the more general case.
% which erases nil
The return Java type \File is converted to a non-nil
Typed Clojure type \ty{} = \File for the return type,
and the propositions say constructors can never be false---constructors
can never produce the internal boolean value that Clojure uses for \false{}, or \nil{}.
Finally, the constructor rewrites to {\newstaticexp {\String} {\File{}} {\File{}} {\makestr{a/b}}}.

\paragraph{Methods} Next we convert {\methodexp {\getparent{}} {\newstaticexp {\String} {\File{}} {\File{}} {\makestr{a/b}}} {}}
to the non-reflective expression
{\methodstaticexp {\File} {} {\String} {\getparent{}} {\newstaticexp {\String} {\File{}} {\File{}} {\makestr{a/b}}} {}}.
%The T-Method rule checks unresolved methods.
%$$
%    {\TMethod}
%$$
The T-Method rule for unresolved methods
checks:

{\methodexp {\getparent{}} {\newstaticexp {\String} {\File{}} {\File{}} {\makestr{a/b}}} {}}.

We verify the target type \s{} = \File is non-nil by T-New.
The overload is chosen from the class table based on \classhint{1} = \File---there is at most one.
The nilable return type \ty{} = \Union{\Nil}{\String} is given, and the 
entire expression rewrites to expression \ref{eq:resolved}.
%
%We allow arguments to constructors and methods to be nilable, but not method
%and field targets.

The T-Field rule (\figref{main:figure:javatyping}) is like T-Method, but without arguments.

The evaluation rules B-Field, B-New and B-Method (\figref{main:figure:javatyping}) simply evaluate their
arguments and call the relevant JVM operation, which we do not model---\secref{sec:metatheory}
states our exact assumptions.
There are no evaluation rules for reflective Java interoperability, since there are no typing
rules that rewrite to reflective calls.


%\subsection{Paths}
%\label{sec:formalpaths}
%
%Recall the first insight of occurrence typing---we can reason
%about specific \emph{parts} of the runtime environment
%using propositions.
%We refer to parts of the runtime environment via 
%a \emph{path} that consists of a series of
%\emph{path elements} applied right-to-left to a variable
%written \pth{\pathelem{}}{\x{}}.
%\cite{TF10} introduce the path elements \carpe{} and \cdrpe{}
%to reason about selector operations on cons cells.
%We instead want to reason about HMap lookups and calls to \classconst{}.
%
%\paragraph{Key path element} We introduce our first path element
%{\keype{\kw{}}}, which represents the operation of looking up
%a key \kw{}.
%We directly relate this to our typing rule T-GetHMap
%(\figref{main:figure:hmapsyntax}) by
%checking the then branch of the first conditional test is checked in 
%an equivalent version of \egref{example:decleaf}.
%\begin{minted}{clojure}
%  (fn [m :- Expr]
%    (if (= (get m :op) :if)
%      {:op :if, ...}
%      (if ...)))
%\end{minted}
%
%We do not specifically support \equivliteral{} in our calculus, 
%but on keyword arguments it works identically to \clj{isa?} which we model
%in \secref{sec:isaformal}.
%Intuitively, if {\judgement{\propenv{}}{\e{}}{\ty{}}{\filterset{\thenprop{\prop{}}}{\elseprop{\prop{}}}}{\object{}}}
%then \equivapp{\e{}}{\makekw{if}} has the true and false propositions
%$$
%{\replacefor{\filterset{\isprop{\Value{\makekw{if}}}{\x{}}}{\notprop{\Value{\makekw{if}}}{\x{}}}}{\object{}}{\x{}}}
%$$
%where substitution reduces to \topprop{} if \object{} = \emptyobject{}.
%
%We start with proposition environment \propenv{} = {\isprop{\Expr{}}{m}}.
%Since {\Expr{}} is a union of HMaps, each with the entry \makekw{op}, we can use T-GetHMap.
%$$
%\judgement{\propenv{}}{\getexp{m}{\makekw{op}}}{\Keyword}{\filterset{\topprop{}}{\topprop{}}}{\pth{\keype{\makekw{op}}}{m}}
%$$
%Using our intuitive definition of \equivliteral{} above, we know
%$$
%\judgement{\propenv{}}{\equivapp{\getexp{m}{\makekw{op}}}{\makekw{if}}}{\Boolean}{\filterset{\isprop{\Value{\makekw{if}}}{\pth{\keype{\makekw{op}}}{m}}}{\notprop{\Value{\makekw{if}}}{\pth{\keype{\makekw{op}}}{m}}}}{\emptyobject{}}
%$$
%Going down the then branch gives us the extended environment
%\propenvp{} = {\isprop{\Expr{}}{m}},{\isprop{\Value{\makekw{if}}}{\pth{\keype{\makekw{op}}}{m}}}.
%Using L-Update we can combine what we know about object $m$ and object
%{\pth{\keype{\makekw{op}}}{m}}
%to derive
%$$
%\inpropenv{\propenvp{}}{\isprop{\HMapp{\mandatoryset{{\mandatoryentrykwnoarrow{op}{\makekw{if}}}, {\mandatoryentrykwnoarrow{test}{\Expr{}}},
%                                       {\mandatoryentrykwnoarrow{then}{\Expr{}}},   {\mandatoryentrykwnoarrow{else}{\Expr{}}}}}
%                                   {\emptyabsent{}}}{m}}
%$$
%
%The full definition of \updateliteral{} is given in \figref{main:figure:update}
%which considers both keys a path elements as well as the \classconst{}
%path element described below.
%In the absence of paths, update simply performs set-theoretic operations
%on types; see \figref{main:figure:restrictremove} for details.
%
%\paragraph{Class path element} Our second path element \classpe{} is used in the latent
%object of the constant \classconst{} function. Like Clojure's \clj{class}
%function \classconst{} returns the argument's class or \nil{}
%if passed \nil{}.
%$$
%\begin{array}{lrlr}
%  \pesyntax{}   &::=& \ldots \alt {\classpe{}}
%                &\mbox{Path Elements}
%\end{array}
%$$
%\begin{mathpar}
%\constanttypefigure{}
%\end{mathpar}
%The dynamic semantics are given in \figref{main:figure:primitivesem}.
%The definition of \updateliteral{} supports various idioms relating to \classpe{}
%which we introduce in \secref{sec:isaformal}.

\section{Multimethod preliminaries: \isaliteral}

\label{sec:isaformal}

We now consider the \isaliteral{} operation, a core part of the multimethod dispatch mechanism. 
Recalling the examples in \secref{sec:multioverview},
\isaliteral{} is
a subclassing test for classes, but otherwise is an equality test.
%---we do not model the semantics for vectors
%
The T-IsA rule uses \isacompareliteral{}
(\figref{main:figure:mmsyntax}), a metafunction which produces the propositions for
\isaliteral{} expressions.
%\begin{mathpar}
%  \TIsA{}
%\end{mathpar}

To demonstrate the first \isacompareliteral{} case,
the expression
\isaapp{\appexp{\classconst{}}{\x{}}}{\Keyword}
is true if \x{} is a keyword, otherwise false.
When checked with T-IsA,
the object of the left subexpression \object{} = {\pth{\classpe{}}{\x{}}}
(which starts with the {\classpe{}} path element)
and the type of the right subexpression \ty{} = {\Value{\Keyword}} (a singleton class type)
together trigger the first \isacompareliteral{} case
\isacompare{\s{}}{\pth{\classpe{}}{\x{}}}{\Value{\Keyword}}{\filterset{\isprop{\Keyword}{\x{}}}{\notprop{\Keyword}{\x{}}}},
giving propositions that correspond to our informal description {\filterset{\thenprop{\prop{}}}{\elseprop{\prop{}}}} = {\filterset{\isprop{\Keyword}{\x{}}}{\notprop{\Keyword}{\x{}}}}.

The second \isacompareliteral{} case captures the simple equality mode for non-class singleton types.
For example,
the expression
\isaapp{\x{}}{\makekw{en}} produces true 
when \x{} evaluates to {\makekw{en}}, otherwise it produces false.
Using T-IsA,
it has the propositions {\filterset{\thenprop{\prop{}}}{\elseprop{\prop{}}}} = 
\isacompare{}{\x{}}{\Value{\makekw{en}}}{\filterset {\isprop {\Value{\makekw{en}}}{\x{}}}{\notprop{\Value{\makekw{en}}}{\x{}}}}
since \object{} = {\x{}} and \ty{} = {\Value{\makekw{en}}}.
%
The side condition on the second \isacompareliteral{} case ensures we are in equality mode---if \x{} can possibly be a class in 
\isaapp{\x{}}{\Object{}}, \isacompareliteral{} uses its conservative default case,
since if \x{} is a class literal, subclassing mode could be triggered.
%
Capture-avoiding substitution of objects {\replacefor {} {\object{}} {\x{}}} used in this case erases propositions
that would otherwise have \emptyobject{} substituted in for their objects---it
is defined in the appendix.

The operational behavior of \isaliteral{} is given by B-IsA (\figref{main:figure:mmsyntax}). \isaopsemliteral{} explicitly handles classes in the second case.

%The definition of \isacompareliteral{} (figure~\ref{main:figure:mmsyntax}) is deliberately conservative.
%The first line considers the case where the object of the left argument
%is a non-empty path ending in \classpe{} and the type of the right argument is a singleton class.

%\constantsemfigure{main}

\section{Multimethods}

\begin{figure}
  %\footnotesize
$$
\begin{altgrammar}
  \e{} &::=& \ldots \alt {\createmultiexp {\ty{}} {\e{}}} \alt
             {\extendmultiexp {\e{}} {\e{}} {\e{}}}
             \alt {\isaapp {\e{}} {\e{}}} &\mbox{Expressions} \\
  \v{} &::=& \ldots \alt {\multi {\v{}} {\disptable{}}}
                &\mbox{Values} \\
 \disptablesyntax{} \\
  \s{}, \ty{} &::=& \ldots \alt {\MultiFntype{\ty{}}{\ty{}}}
                &\mbox{Types}
\end{altgrammar}
$$
  \begin{mathpar}
    \TDefMulti{}

    \TDefMethod{}

    \TIsA{}
  \end{mathpar}
  \begin{mathpar}
    \isapropsfigure{}
  \end{mathpar}
  \begin{mathpar}
    \Multisubtyping{}

    \BDefMulti{}
  \end{mathpar}
  \begin{mathpar}
    \BDefMethod{}
    %\BBetaMulti{}
  \end{mathpar}
  \getmethodfigure{}
\begin{mathpar}
  {\BIsA{}}
  {\isaopsemfigure{}}
  \\
\BBetaMulti{}
\end{mathpar}
\caption{Multimethod Syntax, Typing and Operational Semantics}
\label{main:figure:mmsyntax}
\end{figure}


\figref{main:figure:mmsyntax} presents \emph{immutable} multimethods without default methods to ease presentation.
%Syntax and semantics are given in \figref{main:figure:mmsyntax}. 
%Multimethods can error if no matching method is chosen (rules in the supplemental material).
\figref{main:figure:mmexample} translates the mutable \egref{example:hi-multimethod} to \lambdatc{}.
%\begin{minted}{clojure}
%(ann hi [Kw -> Str])
%(defmulti hi identity)
%(defmethod hi :en [_] "hello")
%(defmethod hi :fr [_] "bonjour")
%(hi :en) ;=> "hello"
%\end{minted}

%FIXME formatting
\begin{figure}
\begin{gather*}
\letexp{hi_0} {\createmultiexp {\ArrowOne{\x{}}{\Keyword}{\String}{\filterset{\topprop{}}{\topprop{}}}{\emptyobject{}}} {\abs{\x{}}{\Keyword}{\x{}}}}
  {\\\text{\quad}
    \letexp{hi_1} {\extendmultiexp {hi_0} {\makekw{en}} {\abs {\x{}} {\Keyword} {\makestr{hello}}}}
      {\\\text{\quad\quad}
        \letexp{hi_2} {\extendmultiexp {hi_1} {\makekw{fr}} {\abs {\x{}} {\Keyword} {\makestr{bonjour}}}}
        {\\\text{\quad\quad\quad
          \appexp{hi_2}{\makekw{en}}}}}}
\end{gather*}
\caption{Multimethod example}
\label{main:figure:mmexample}
\end{figure}
%
%For convenience, examples in this section are flattened when they are really nested
%let bindings. We also elide trivial latent propositions and objects.
%The following is an abbreviation of the previous expression.
%\begin{lstlisting}
%${\createmultiexp {\ArrowTwo{\x{}}{\Keyword}{\String}} {\abs{\x{}}{\Keyword}{\x{}}}}$
%${\extendmultiexp {hi} {\makekw{en}} {\abs {\x{}} {\Keyword} {\makestr{hello}}}}$
%${\extendmultiexp {hi} {\makekw{fr}} {\abs {\x{}} {\Keyword} {\makestr{bonjour}}}}$
%$\appexp{hi}{\makekw{en}}$
%\end{lstlisting}
%
%\defmethodliteral{} returns a new extended multimethod
%without changing the original multimethod. 
%
%\begin{minted}{clojure}
%(let [hi (defmulti [Kw -> Str] identity)]
%  (let [hi (defmethod hi :en [_] "hello")]
%    (let [hi (defmethod hi :fr [_] "bonjour")]
%      (hi :en))) ;=> "hello"
%\end{minted}
%
%\paragraph{How to check}
To check 
{\createmultiexp {\ArrowTwo{\x{}}{\Keyword}{\String}} {\abs{\x{}}{\Keyword}{\x{}}}},
%
we note
{\createmultiexp {\s{}} {\e{}}} creates a multimethod with \emph{interface type} \s{}, and dispatch function \e{}
of type \sp{},
producing a value of type
{\MultiFntype {\s{}} {\sp{}}}. % with interface type {\s{}} and dispatch function type {\sp{}}.
The T-DefMulti typing rule checks the dispatch function, and
verifies both the interface and dispatch type's domain agree.
%$$
%    \TDefMulti{}
%$$
Our example checks with \ty{} = \Keyword, interface type \s{} = {\ArrowTwo{\x{}}{\Keyword}{\String}},
dispatch function type \sp{} = {\ArrowOne{\x{}}{\Keyword}{\Keyword}{\filterset{\topprop{}}{\topprop{}}}{\x{}}}, and overall type
$
{\MultiFntype {\ArrowTwo{\x{}}{\Keyword}{\String}}
              {\ArrowOne{\x{}}{\Keyword}{\Keyword}{\filterset{\topprop{}}{\topprop{}}}{\x{}}}}
$.

Next, we show how to check
$
{\extendmultiexp {hi_0} {\makekw{en}} {\abs {\x{}} {\Keyword} {\makestr{hello}}}}
$.
%
The expression 
{\extendmultiexp {\e{m}} {\e{v}} {\e{f}}} creates a new multimethod
that extends multimethod \e{m}'s dispatch table, mapping dispatch value
\e{v} to method \e{f}. The T-DefMulti typing rule
checks \e{m} is a multimethod with dispatch function type \ty{d},
then calculates the extra information we know based on the current
dispatch value {\thenprop{\proppp{}}}, which is assumed when checking the method
body.
%$$
%    \TDefMethod{}
%$$
Our example checks with \e{m} being of type
$
{\MultiFntype {\ArrowTwo{\x{}}{\Keyword}{\String}}
              {\ArrowOne{\x{}}{\Keyword}{\Keyword}{\filterset{\topprop{}}{\topprop{}}}{\x{}}}}
$
with \objectp{} = {\x{}} (from below the arrow on the right argument of the previous type) and \ty{v} = \Value{\makekw{en}}. 
Then {\thenprop{\proppp{}}} = 
{\isprop {\Value{\makekw{en}}}{\x{}}}
from
$
\isacompare{}{\x{}}{\Value{\makekw{en}}}
{\filterset {\isprop {\Value{\makekw{en}}}{\x{}}}{\notprop{\Value{\makekw{en}}}{\x{}}}}
$
(see \secref{sec:isaformal}).
Since \ty{} = \Keyword{}, we check the method body with
$$
\judgement{{\isprop{\Keyword}{\x{}}},{\isprop {\Value{\makekw{en}}}{\x{}}}}
  {\makestr{hello}}
  {\String}{\filterset{\topprop{}}{\topprop{}}}{\emptyobject{}}\text{.}
$$
Finally from the interface type \ty{m}, we know \thenprop{\prop{}} = \elseprop{\prop{}} = \topprop{},
and \object{} = \emptyobject{}, which also agrees with the method body, above.
Notice the overall type of a \defmethodliteral{} is the same as its first subexpression \e{m}.

It is worth noting the lack of special typing rules for overlapping methods---each
method is checked independently based on local type information.

%The expression {\createmultiexp {\s{}} {\e{}}} 
%defines a multimethod
%with interface type \s{} and dispatch function \e{}.
%The expression {\extendmultiexp {\e{m}} {\e{v}}{\e{f}}}
%extends multimethod \e{m} and to map
%dispatch value {\e{v}} to {\e{f}} in an extended dispatch table.
%The value {\multi {\v{}} {\disptable{}}} is the runtime value of a multimethod
%with dispatch function {\v{}} and dispatch table {\disptable{}}.
%
%The T-DefMulti rule ensures that the type of the dispatch function
%has at least as permissive a parameter type
%as the interface type.
%
%For example, we can check the definition from our translation above of \egref{example:rep}
%using T-DefMulti.
%$$
%\judgement{}%{\propenv{}}
%{\createmultiexp 
%      {\s{}}
%      {\classconst{}}}
%  {\MultiFntype {\s{}}{\sp{}}}{\filterset{\topprop{}}{\botprop{}}}{\emptyobject{}}
%$$
%where \s{}  = {\ArrowOne {\x{}} {\Top{}} {\ty{}} {\filterset {\topprop{}} {\topprop{}}} {\emptyobject{}}}
%  and \sp{} = {\ArrowOne {\x{}} {\Top{}} {\Union{\Nil}{\Class}} {\filterset {\topprop{}} {\topprop{}}} {\pth{\classpe{}}{\x{}}}}.
%  Since the parameter types agree, this is well-typed.
%
%The T-DefMethod rule requires a syntactic lambda expression as the method.
%This way we can manually check the body of the lambda under an extended
%environment as sketched in \egref{example:incmap}.
%We use \isacompareliteral{} to compute the proposition for this method,
%since \isaliteral{} is used at runtime in multimethod dispatch.
%
%We continue with the next line of the translation of \egref{example:rep}.
%From the previous line we have \propenv{} = {\isprop{\MultiFntype {\s{}}{\sp{}}}{path}},
%so
%$$
%\judgement{\propenv{}}
%  {\extendmultiexp {prop} {\String}
%                   {\abs {\x{}} {\Top{}} {\x{}}}}
%  {\MultiFntype {\s{}}{\sp{}}}{\filterset{\topprop{}}{\botprop{}}}{\emptyobject{}}
%$$
%We know \emph{prop} is a multimethod by \propenv{}, so now we check the body
%of this method.
%$$
%\judgement{\propenv{},{\isprop{\Top}{\x{}}},{\isprop{\String}{\x{}}}}
%  {\x{}}
%  {\String}{\filterset{\topprop{}}{\botprop{}}}{\emptyobject{}}
%$$
%%This is checked by T-Local since {\inpropenv{\propenv{},{\isprop{\Top}{\x{}}},{\isprop{\String}{\x{}}}}{\isprop{\String}{\x{}}}}.
%The new proposition {\isprop{\String}{\x{}}} is derived by 
%$$
%  \isacompare{\Top{}}{\pth{\classpe{}}{\x{}}}{\Value{\File{}}}
%             {\filterset{\isprop{\String}{\x{}}}
%                        {\notprop{\String}{\x{}}}}.
%$$
%%
%The body of the \clj{let} is checked by T-App because
%{\MultiFntype {\s{}}{\sp{}}} is a subtype of its interface type {\s{}}.

\paragraph{Subtyping}
Multimethods are functions, via S-PMultiFn,
%$$
%\SPMultiFn{}
%$$
which says a multimethod can be upcast to its interface type. 
Multimethod call sites are then handled by T-App via T-Subsume. Other rules are given
in \figref{main:figure:mmsyntax}. 

\paragraph{Semantics}
Multimethod definition semantics are also given 
in \figref{main:figure:mmsyntax}. 
B-DefMulti creates a multimethod with the given dispatch function and an empty dispatch table.
B-DefMethod produces a new multimethod with an extended dispatch table.

The overall dispatch mechanism is summarised by B-BetaMulti.
First the dispatch function \v{d} is applied to the argument \vp{} to obtain
the dispatch value \v{e}.
Based on \v{e},
the \getmethodliteral{} metafunction (\figref{main:figure:mmsyntax})
extracts a method \v{f} from the method table {\disptable{}}
and applies it to the original argument for the final result.

\section{Precise Types for Heterogeneous maps}
\label{sec:hmapformal}

\begin{figure}
  %\footnotesize
  $$
  \begin{altgrammar}
    \e{} &::=& \ldots \alt \hmapexpressionsyntax{}
    &\mbox{Expressions} \\
    \v{} &::=& \ldots \alt {\emptymap{}}
    &\mbox{Values} \\
    \ty{} &::=& \ldots \alt {\HMapgeneric {\mandatory{}} {\absent{}}}
    &\mbox{Types} \\
    \auxhmapsyntax{}\\
%    \pesyntax{}   &::=& \ldots \alt {\keype{\kw{}}}
%                  &\mbox{Path Elements}
  \end{altgrammar}
  $$
  \begin{mathpar}
    {\TAssoc}

    {\TGetHMap}

    {\TGetAbsent}

    {\TGetHMapPartialDefault}
  
    {\SHMapMono}
  \end{mathpar}
  \begin{mathpar}
  {\SHMapP}\ \ 
  {\SHMap}

  \end{mathpar}
  \begin{mathpar}
    {\BAssoc}\ \ 
    {\BGet}\ \ 
    {\BGetMissing}
  \end{mathpar}
  \caption{HMap Syntax, Typing and Operational Semantics}
  \label{main:figure:hmapsyntax}
\end{figure}


\begin{figure}
  %\footnotesize
$$
  \begin{array}{llll}
    \restrictfigure{}\\
    \removefigure{}
  \end{array}
$$
\caption{Restrict and remove}
\label{main:figure:restrictremove}
\end{figure}

\figref{main:figure:hmapsyntax}
presents
heterogeneous map types.
The type \HMapgeneric{\mandatory{}}{\absent{}}
contains {\mandatory{}}, a map of \emph{present} entries (mapping keywords to types),
\absent{}, a set of keyword keys that are known to be \emph{absent}
and
tag \completenessmeta{} which is either {\complete{}} (``complete'') if the map is fully specified by \mandatory{},
and {\partial{}} (``partial'') if there are \emph{unknown} entries.
%
The partially specified map of
\clj{lunch} in \egref{example:lunchpartial}
is written
\HMapp{\mandatoryset{\mandatoryentrynoarrow{\Valkw{en}}{\String}, {\mandatoryentrynoarrow{\Valkw{fr}}{\String}}}}{\emptyabsent{}}
(abbreviated \Lunch).
%
The type of the fully specified map
\clj{breakfast} in \egref{example:breakfastcomplete} elides the absent entries,
written
\HMapc{\mandatoryset{\mandatoryentrynoarrow{\Valkw{en}}{\String}, {\mandatoryentrynoarrow{\Valkw{fr}}{\String}}}}
(abbreviated \Breakfast).
To ease presentation, 
if an HMap has completeness tag \complete{} then \absent{} is elided and implicitly contains all keywords not in the domain of 
\mandatory{}---dissociating keys is not modelled, so the set of absent entries otherwise
never grows.
Keys cannot be both present and absent.
%\HMapcwithabsent{\mandatory{}}{\absent{}} is abbreviated to \HMapc{\mandatory{}}. 

The metavariable \mapval{}
ranges over the runtime value of maps {\curlymapvaloverright{\kw{}}{\v{}}},
usually written {\curlymapvaloverrightnoarrow{\kw{}}{\v{}}}.
We %do not model keywords as functions,
only provide syntax for the empty map literal,
however when convenient we abbreviate non-empty map literals
to be a series of \assocliteral{} operations on the empty map.
We restrict lookup and extension to keyword keys. 

\paragraph{How to check}
A mandatory lookup is checked by T-GetHMap.
$$
\abs{\makelocal{b}}{\Breakfast}{\getexp{\makelocal{b}}{\makekw{en}}}
$$
The result type is \String, and the return object is \pth{\keype{\makekw{en}}}{\makelocal{b}}.
The object {\replacefor {\pth {\keype{k}} {\x{}}} {\object{}} {\x{}}}
is a symbolic representation for a keyword lookup of $k$ in \object{}.
The substitution for {\x{}} handles the case where \object{} is empty.

\begin{mathpar}
\begin{array}{rcl}
{\replacefor {\pth {\keype{k}} {\x{}}} {\y{}} {\x{}}} &=& {\pth {\keype{k}} {\y{}}} \\
\end{array}
\ \ \ \ \ \ \ 
\begin{array}{rcl}
{\replacefor {\pth {\keype{k}} {\x{}}} {\emptyobject{}} {\x{}}} &=& \emptyobject{}
\end{array}
\end{mathpar}

An absent lookup is checked by T-GetHMapAbsent.
$$
\abs{\makelocal{b}}{\Breakfast}{\getexp{\makelocal{b}}{\makekw{bocce}}}
$$
The result type is \Nil---since \Breakfast is fully specified---with return object \pth{\keype{\makekw{bocce}}}{\makelocal{b}}.

A lookup that is not present or absent is checked by
T-GetHMapPartialDefault.
$$
\abs{\makelocal{u}}{\Lunch}{\getexp{\makelocal{u}}{\makekw{bocce}}}
$$
The result type is \Top---since {\Lunch} has an unknown \makekw{bocce} entry---with return object \pth{\keype{\makekw{bocce}}}{\makelocal{u}}.
Notice propositions are erased once they enter a HMap type.

For presentational reasons, lookups on unions of HMaps are only supported in T-GetHMap
and each element of the union must contain the relevant key.
$$
\abs{\makelocal{u}}{\Unionsplice{\Breakfast \Lunch}}{\getexp{\makelocal{u}}{\makekw{en}}}
$$
The result type is \String, and the return object is \pth{\keype{\makekw{en}}}{\makelocal{u}}.
However, lookups of \makekw{bocce} on {\Unionsplice{\Breakfast \Lunch}} maps are unsupported.
This restriction still allows us to check many of the examples in \secref{sec:overview}---in
particular we can check 
\egref{example:desserts-on-meal}, as \makekw{Meal} is in common with both HMaps,
but cannot check \egref{example:desserts-on-class}
because a \makekw{combo} meal lacks a \makekw{desserts} entry.
Adding a rule to handle \egref{example:desserts-on-class} is otherwise straightforward.

Extending a map with T-AssocHMap preserves its completeness.
$$
\abs{\makelocal{b}}{\Breakfast}{\assocexp{\makelocal{b}}{\makekw{au}}{\makestr{beans}}}
$$
The result type is
$
\HMapc{\mandatoryset{\mandatoryentrynoarrow{\Valkw{en}}{\String}, {\mandatoryentrynoarrow{\Valkw{fr}}{\String}}
        ,{\mandatoryentrynoarrow{\Valkw{au}}{\String}}}}
$,
a complete map.
T-AssocHMap also enforces ${\kw{}} \not\in {\absent{}}$ to prevent badly formed types.

%for cases like \egref{example:desserts-on-meal}
%where every element in the union
%contains the key we are looking up.

\paragraph{Subtyping}
Subtyping for HMaps
designate \MapLiteral{} as a common supertype for all HMaps.
S-HMap says that HMaps are subtypes if they agree
on \completenessmeta{}, agree on mandatory entries with subtyping
and at least cover the absent keys of the supertype.
Complete maps are subtypes of partial maps
as long as they agree on the mandatory entries of the partial map via subtyping (S-HMapP).

%The typing rules for \getliteral{} consider three possible cases. T-GetHMap models a lookup
%that will certainly succeed, T-GetHMapAbsent a lookup that will certainly fail
%and T-GetHMapPartialDefault a lookup with unknown results.

%The objects in the T-Get rules are more complicated than those in T-Local---the 
%next section discusses this in detail.
%Finally T-AssocHMap extends an HMap with a mandatory entry while preserving completeness
%and absent entries, and enforcing ${\kw{}} \not\in {\absent{}}$ to prevent badly
%formed types.

The semantics for \getliteral{} and \assocliteral{} are straightforward.
%If the entry is missing, B-GetMissing produces \nil{}.

\begin{figure}[t]
  $$
\begin{array}{llll}
\updatefigure{}
\end{array}
$$
\caption{Type update (the metavariable \propisnotmeta{} ranges over \ty{} and \nottype{\ty{}} (without variables), 
  \notsubtypein{}{\Nil{}}{\nottype{\ty{}}} when \issubtypein{}{\Nil{}}{\ty{}}, see
\figref{main:figure:restrictremove} for \restrictliteral{} and \removeliteral{}.
  )}
\label{main:figure:update}
\end{figure}

%\begin{figure}
%  $$
%\begin{array}{llll}
%  \restrictremovefigure{}
%\end{array}
%  $$
%  \caption{Restrict and Remove}
%  \label{main:figure:restrictremove}
%\end{figure}

\section{Proof system}
\label{formalmodel:proofsystem}

The occurrence typing proof system uses standard propositional logic,
except for where nested information is combined. This is
handled by L-Update:
{  %\footnotesize
\singlespacing
  $$
\LUpdate{}
$$
}

It says
under \propenv{}, if object \pth{\pathelemp{}}{\x{}} is of type \ty{}, and 
an extension
\pth{\pathelem{}}{\pth{\pathelemp{}}{\x{}}}
is of possibly-negative type \propisnotmeta{}, then
{\update{\ty{}}{\propisnotmeta{}}{\pathelem{}}}
is \pth{\pathelemp{}}{\x{}}'s type under \propenv{}.

Recall \egref{example:desserts-on-meal}.
%, resuming from
%\secref{sec:coretypesystem}. 
Solving
$
{ \inpropenv 
  {{\isprop{\Order}{\makelocal{o}}},
    {\isprop{\Value{\makekw{combo}}}{\pth{\keype{\makekw{Meal}}}{\makelocal{o}}}}}
  {\isprop {\ty{}} {\makelocal{o}}}}
$
uses L-Update, where \pathelem{} = {\emptypath{}} and \pathelemp{} = [{\keype{\makekw{Meal}}}].
$$
\inpropenv{\propenv{}}{\isprop{\update{\Order}{\Value{\makekw{combo}}}{[{\keype{\makekw{Meal}}}]}}{\makelocal{o}}}
$$
Since {\Order} is a union of HMaps, we structurally recur on the first case of \updateliteral{}
(\figref{main:figure:update}),
which preserves \pathelem{}.
Each initial recursion hits the first HMap case, since there is some \ty{} such that
{\inmandatory{\kw{}}{\ty{}}{\mandatory{}}} and 
\completenessmeta{} accepts partial maps \partial{}.

To demonstrate,
\makekw{lunch} meals are handled by the first HMap case and
update to
$$
{\HMapp {\extendmandatoryset {\mandatory{}}{\Valkw{Meal}}{\sp{}}} {\emptyabsent{}}}
$$
where \sp{} = {\update{\Valkw{lunch}}{\Valkw{combo}}{\emptypath{}}}
and
$$
\mandatory{} = \mandatoryset{\mandatoryentry{\Valkw{Meal}}{\Valkw{lunch}},{\mandatoryentry{\Valkw{desserts}}{\Number{}}}}.
$$
\sp{} updates to \Bot via the penultimate \updateliteral{} case,
because \restrict{\Value{\makekw{lunch}}}{\Value{\makekw{combo}}} = \Bot
by the first \restrictliteral{} case.
The same happens to \makekw{dinner} meals,
leaving just the \makekw{combo} HMap. 

In \egref{example:desserts-on-class},
$
\inpropenv{\propenv{}}{\isprop{\update{\Order}{\Long}{[{\classpe{}}, {\keype{\makekw{desserts}}}]}}{\makelocal{o}}}
$
updates the argument in the {\Long} method.
This recurs twice for each meal to handle the {\classpe{}}
path element.

We describe the other \updateliteral{} cases.
The first \classpe{} case updates
to \class{} if \classconst{} returns \Value{\class{}}.
The second \keype{\kw{}} case detects contradictions in absent
keys. % not overlapping with \Nil{}.
The third \keype{\kw{}} case updates unknown entries to be mapped to \ty{} or absent.
The fourth \keype{\kw{}} case updates unknown entries to be \emph{present}
when they do not overlap with \Nil{}.

%$
%{\update{\Number}{\Long}{[{\classpe{}}]}}}
%$
%
%$
%{\update{\Int}{\Long}{[{\classpe{}}]}}}
%$
%
%

\chapter{Metatheory}
\label{sec:metatheory}

We prove type soundness following Tobin-Hochstadt and Felleisen~\cite{TF10}.  Our model is extended
to include errors \errorvalv{} and a \wrong{} value, and we prove well-typed
programs do not go wrong; this is therefore a stronger theorem than
proved by Tobin-Hochstadt and Felleisen~\cite{TF10}. 
Errors behave like Java exceptions---they can be thrown and propagate ``upwards'' in the evaluation rules
(\errorvalv{} rules are deferred to the appendix).

Rather than modeling Java's dynamic semantics, a task of daunting
complexity, we instead make our assumptions about Java explicit. We
concede that method and constructor calls may diverge or error, but
assume they can never go wrong.
%(other assumptions given in the supplemental material).

%{\javanewassumption{main}}
{\javaassumptionsall{main}}

For readability we define logical truth in Clojure.

{\istruefalsedefinitions{main}}

For the purposes of our soundness proof, we require that all values
are \emph{consistent}.  Consistency 
%(defined in the supplemental material)
states that the types of closures are well-scoped---they do
not claim propositions about variables hidden in their closures.

{\consistentwithonlydef{main}}

We can now state our main lemma and soundness theorem.  The
metavariable \definedreduction{} ranges over \v{}, \errorvalv{} and
\wrong{}. Proofs are deferred to the supplemental material. %\ref{appendix:lemma:soundness}.

\begin{lemma}\label{main:lemma:soundness}

  {\soundnesslemmahypothesis}
\end{lemma}


{\soundnesstheoremnoproof{main}}

{\wrongtheoremnoproof{main}}

%{\nilinvoketheoremnoproof{main}}

\chapter{Experience}
\label{sec:experience}

Typed Clojure is implemented as \coretyped{}~\cite{coretyped},
which has seen wide usage.

\section{Implementation}

\coretyped{} provides preliminary integration with the Clojure compilation
pipeline, primarily to resolve Java interoperability.
%, however
%most usages are entirely optionally typed.

%In contrast to Racket, Clojure does not provide extension
%points to the macroexpander. 
%To satisfy our goals of providing
%Typed Clojure as a library that works with the latest version of the Clojure
%compiler, \coretyped{} is implemented as an external static analysis pass
%that must be explicitly invoked by the programmer, and not as an
%integral part of the Clojure compilation process. 
%%Therefore, \coretyped{} is in a sense a linter.

The \coretyped{} implementation extends this paper in several key areas 
to handle checking real Clojure code, including an implementation
of Typed Racket's variable-arity polymorphism~\cite{stf-esop}, 
and support for other Clojure idioms like datatypes and protocols.
%
There is no integration with Java Generics, so only Java 1.4-style erased types are ``trusted''
by \coretyped{}.
Casts are needed to recover the discarded information, which---for collections---are 
then tracked via Clojure's universal sequence interface~\cite{CljSeqDoc}.

%Recently, steps have been taken to integrate Typed Clojure into 

%This means that type checking is  optional. 
%On the positive side, \coretyped{} is flexible to the needs of a dynamically
%typed programmer, encouraging experimentation with programs that may not
%type check.
%On the negative side, programmers must remember to type check their namespaces.
%Also, the compiler cannot depend on types, making
%type-based optimisation is impossible. 
%If this were not the case, we could dispose of type-hints
%altogether, and simply use static types to resolve reflection.

%\subsection{Let-aliasing}
%
%\begin{mathpar}
%  \footnotesize
%\infer [T-LocalAlias]
%{ \Theta[\x{}] = \object{}
%  \\
%  \inpropenv {\propenv{}} {\isprop {\t{}} {\object{}}}
%  \\\\
%  \s{} = {\falsy} }
%{ \judgement {\Theta; \propenv{}} 
%             {\hastype {\x{}} {\t{}}}
%             {\filterset {{\notprop {\s{}} {\object{}}}} {{\isprop {\s{}} {\object{}}}}}
%             {\object{}}
%                   }
%
%\infer [T-LetAlias]
%{ \judgement {\Theta; \propenv{}} {\hastype {\e{1}} {\s{}}} {\filterset {\thenprop {\prop{1}}} {\elseprop {\prop{1}}}}
%             {\object{1}}
%  \\\\
%  \object{1} \notequal \emptyobject{}
%  \\\\
%  \judgement
%       {\Theta[\x{} \mapsto \object{1}];
%         \propenv{}}
%             {\hastype {\e{}} {\t{}}} {\filterset {\thenprop {\prop{}}} {\elseprop {\prop{}}}}
%             {\object{}} 
%             }
%{ \judgement {\Theta; \propenv{}} {\hastype {\letexp {\x{}} {\e{1}} {\e{}}} {\t{}}}
%             {\filterset {\thenprop {\prop{}}} {\elseprop {\prop{}}}}
%             {\object{}} 
%             }
%\end{mathpar}

%\subsection{Further Extensions}
%
%In addition to the key features we present in this paper,
%\coretyped{} supports other extensions to handle additional Clojure
%features. 
%
%\smallsection{Datatypes, Records and Protocols}
%Clojure features datatypes and protocols. Datatypes are Java classes
%declared final with public final fields. They can implement Java interfaces
%or protocols, which are similar to interfaces but already-defined classes
%and \nil{} may extend protocols.
%%
%Typed Clojure can reason about most of these features,
%including the ability to define polymorphic datatypes and protocols and
%utilising the Java type system to help check implemented interface methods.

%\smallsection{Intersection Types}
%Typed Clojure includes simple intersection types which do no sophisticated
%reasoning with the dual subtyping rules to unions.
%
%In some cases this makes types more expressive. Say we know \clj{x} has some
%universally quantified type \clj{a} and we learn \clj{x} is a \clj{Number}.
%Without intersection types, we must choose which piece of information to forget.
%In Typed Clojure, \clj{x} is simply of type \clj{(I x Number)}.
%
%\smallsection{Mutation and Polymorphism}
%Clojure supports mutable references with software-transactional-memory
%which Typed Clojure defines \emph{bivariantly}---with write and read type parameters
%as in the atomic reference \clj{(Atom2 Int Int)} which can write and read \clj{Int}.
%Typed Clojure also supports parametric polymorphism, including
%Typed Racket's variable-arity polymorphism~\cite{stf-esop}, 
%which enables us to assign a type to functions like \clj{swap!} (\figref{main:fig:swap!}),
%which takes a mutable \emph{atom},
%a function and extra arguments, and swaps into the atom the result of
%applying the function to the atom's current value and the extra arguments.
%
%\begin{figure}
%\begin{minted}{clojure}
%(ann clojure.core/swap! (All [w r b ...] 
%                          [(Atom2 w r) [r b ... b -> w] b ... b -> w]))
%(swap! (atom :- Num 1) + 2 3);=> 6 (atom contains 6)
%\end{minted}
%%\inputminted[firstline=5,lastline=5]{clojure}{code/demo/src/demo/atom.clj}
%\caption{Type annotation and example call of \clj{swap!}}
%\label{main:fig:swap!}
%\end{figure}

\section{Evaluation}
\label{sec:casestudy}

Throughout this paper, we have focused on three interrelated type
system features: heterogeneous maps, Java interoperability, and
multimethods. Our hypothesis is that these features are widely used in
existing Clojure programs in interconnecting ways, and that handling
them as we have done is required to type check realistic Clojure
programs.



To evaluate this hypothesis, we analyzed two existing \coretyped{}
code bases, one from the open-source community, and one from a company
that uses \coretyped{} in production. For our data gathering, we
instrumented the \coretyped{} type checker to record how often
various features were used (summarized in 
\figref{experience:featuretable}). 

\begin{figure*}[t]

\begin{tabular}{lll}
      \toprule


  & feeds2imap & CircleCI \\
  \midrule
  Total number of typed namespaces & 11 (825 LOC) & 87 (19,000 LOC) \\
  Total number of \clj{def} expressions & 93  & 1834 \\
  \tabitem
  checked & 52 (56\%) & 407 (22\%) \\
  \tabitem
  unchecked & 41 (44\%) & 1427 (78\%) \\
  Total number of Java interactions & 32 & 105 \\
  \tabitem
  static methods & 5 (16\%) & 26 (25\%) \\ 
  \tabitem
  instance methods & 20 (62\%) & 36 (34\%) \\
  \tabitem
  constructors & 6 (19\%) & 38 (36\%) \\
  \tabitem
  static fields & 1 (3\%) & 5 (5\%) \\
  Methods overriden to return non-nil & 0 & 35 \\
  Methods overriden to accept nil arguments & 0 & 1 \\
  Total HMap lookups & 27  & 328  \\
  \tabitem
  resolved to mandatory key & 20 (74\%) & 208 (64\%) \\
  \tabitem
  resolved to optional key & 6 (22\%) & 70 (21\%) \\
  \tabitem
  resolved of absent key & 0 (0\%) & 20 (6\%) \\
  \tabitem
  unresolved key & 1 (4\%) & 30 (9\%) \\
  Total number of \clj{defalias} expressions & 18  & 95 \\
  \tabitem
  contained HMap or union of HMap type & 7 (39\%)  & 62 (65\%) \\
  Total number of checked \clj{defmulti} expressions & 0  & 11 \\
  Total number of checked \clj{defmethod} expressions & 0  & 89 \\


\end{tabular}
\caption{Typed Clojure Features used in Practice}
\label{experience:featuretable}
\end{figure*}


\paragraph{feeds2imap}
feeds2imap\footnote{\url{https://github.com/frenchy64/feeds2imap.clj}}
is an open source library written in Typed Clojure. 
It provides an RSS reader using the \emph{javax.mail} framework.

% static call (:check/:static-call) = 74
% - user   5
% - inlined (:check/static-call-clojure-lang-probably-inline) 69
% static field = 13
% - user 1
% - inlined (:check/static-field-clojure-lang-probably-inline) 12
% new (:check/:new) = 11
% - user 6
% - inlined (:check/new-clojure-lang-probably-inline) 5
% instance call = 53
% - body  20
% - inlined (:check/instance-call-clojure-lang-probably-inline) 33
% total 151
% - user  32
Of 11 typed namespaces containing 825 lines of code, there are 32 Java interactions.
The majority are method calls, consisting of 20 (62\%) instance methods and 5 (16\%) static methods. 
The rest consists of 1 (3\%) static field access, and 6 (19\%) constructor calls---there are no instance field accesses.

%  from :check/find-val-type-with-hmap* numbers
There are 27 lookup operations on HMap types, of which 20 (74\%) resolve to mandatory entries, 6 (22\%) to optional entries, and 1 (4\%) is an unresolved lookup. 
No lookups involved fully specified maps.

% :collect/:def     93
% :check/checked-def 52
From 93 \clj{def} expressions in typed code, 52 (56\%) are checked, with a rate of 1 Java interaction for 1.6 checked top-level definitions, and 1 HMap lookup to 1.9 checked top-level definitions.
That leaves 41 (44\%) unchecked vars, mainly due to partially complete porting to Typed Clojure, but in some cases due to unannotated third-party libraries.

No typed multimethods are defined or used. 
% :collect/defalias-is-HMap      7
% :invoke-special-collect/(quote clojure.core.typed/def-alias*)     18
Of 18 total type aliases, 7 (39\%) contained one HMap type, and none contained unions of HMaps---on further inspection there was no HMap entry used to dictate control flow, often handled by multimethods.
This is unusual in our experience, and is perhaps explained by feeds2imap mainly wrapping existing \emph{javax.mail} functionality.

\paragraph{CircleCI}
CircleCI~\cite{CircleCI}
provides continuous integration services built with a mixture of open-
and closed-source software.
Typed Clojure was used at CircleCI in production systems for two years \cite{CircleCIUsesTC},
maintaining
87 namespaces and 19,000 lines of code,
an experience we summarise in \secref{sec:limitations}.
%
%CircleCI provided the first author access to the main closed-source backend system written in Clojure
%and Typed Clojure.
%We conducted a study of the effectiveness of Typed Clojure in practice.
%There is no clear metric for quantifying typed Clojure code, since untyped code
%can be freely mixed and some seemingly typed namespaces are not checked
%regularly. 
%We manually type checked all namespaces that depend on \clj{clojure.core.typed}
%and considered those with type errors as untyped.
%We then searched the remaining typed code for unsafe Typed Clojure operations like
%var annotations with \clj{:no-check} and the \clj{tc-ignore} macro,
%which instruct Typed Clojure to ignore the specified code,
%and also considered those untyped.
%Furthermore, we manually collected and inspected all top-level annotations and
%classified them.
%
%We determined that

%% Out of 588 top-level var annotations, 270 (46\%) were checked annotations of
%% functions defined in typed code,
%% 129 (22\%) annotations assigned types to external libraries 
%% and the remaining 189 (32\%) annotated `unchecked' user code.
%Some of the type-annotated definitions were so annotated by the first
%author and contributed back to CircleCI.
%HMaps were a valuable feature, with 38 (59\%) out of 64 total type aliases
%featuring them; see \egref{example:circleci} for an instance.
%
%Because of various shortcomings of \coretyped{}, all 57 \clj{defmethod}
%expressions in typed namespaces were unchecked.
%
%811 top-level var annotations
%
%% Due to a lack of checked multimethods,
%% the first author ported 11 previously-untyped multimethods to Typed Clojure, also checking 
%% 89 methods.

The CircleCI code base contains 11 checked multimethods.
 All 11 dispatch functions
are on a HMap key containing a keyword, in a similar style to
\egref{example:desserts-on-meal}.
Correspondingly, all 89 methods are associated with a keyword dispatch value.
The argument type was in all cases a single HMap type, however,
rather than a union type.
In our experience from porting other libraries, this is unusual.

% 87 typed namespaces
% :check/gen-analysis     87

% :check/find-val-type-with-hmap    328
% :check/find-val-type-with-hmap-present    208
% :check/find-val-type-with-hmap-with-optional     70
% :check/find-val-type-with-hmap-fall-through     30
% :check/find-val-type-with-hmap-absent     20
% :check/find-val-has-complete      2
% :merge/complete-used-on-right      5

Of 328 lookup operations on HMaps,
208 (64\%) resolve to mandatory keys,
70 (21\%) to optional keys,
20 (6\%) to absent keys, and
30 (9\%) lookups are unresolved.
%
% :collect/defalias-is-HMap     62
% :invoke-special-collect/(quote clojure.core.typed/def-alias*)     95 
Of 95 total type aliases defined with \clj{defalias},
62 (65\%) involved one or more HMap types.
%
%% :new-special/(quote clojure.lang.MultiFn)     11
%
%% :check/:static-call    525
%% :check/static-call-clojure-lang-probably-inline    499
%% = 26 user
%
%% :check/:instance-call    510
%% :check/instance-call-clojure-lang-probably-inline    474
%% = 36 user
%
%% :check/:new    159
%% :check/new-clojure-lang-probably-inline    121
%% = 38
%
%% :check/:static-field     92
%% :check/static-field-clojure-lang-probably-inline     87
%% = 5
%
%% 26 + 36 + 38 + 5 = 105
%
%% :invoke-special-collect/(quote clojure.core.typed/non-nil-return*) 35
%% :invoke-special-collect/(quote clojure.core.typed/nilable-param*)  1
%
Out of 105 Java interactions, 26 (25\%) are static methods, 36 (34\%)
are instance methods, 38 (36\%) are constructors, and 5 (5\%) are static
fields. 35 methods are overriden to return non-nil, and 1 method 
overridden to accept nil---suggesting that
\coretyped{} disallowing \clj{nil} as a method argument by default
is justified.

% :check/checked-def  407
% :check/checked-MultiFn-addMethod 57
% :instance-method-special/(quote clojure.lang.MultiFn/addMethod)     89
% = 464 checked definitions
Of 464 checked top-level definitions (which consists of
57 \clj{defmethod} calls and 407 \clj{def} expressions),
1 HMap lookup occurs per 1.4 top-level definitions,
and 1 Java interaction occurs every 4.4 top-level definitions.

% :check/def-not-checking-definition   1352
% :check/checked-def  407
% = 1759
% :collect/:def   1834
% = 1427 unchecked
From 1834 \clj{def} expressions in typed code,
%87 typed namespaces,
only 407 (22\%) were checked.
That leaves 1427 (78\%) which have unchecked definitions, either by an explicit \clj{:no-check} annotation
or \clj{tc-ignore} to suppress type checking,
or the \clj{warn-on-unannotated-vars} option, which skips \clj{def} expressions
that lack expected types via \clj{ann}.
From a brief investigation,
reasons include unannotated third-party libraries,
work-in-progress conversions to Typed Clojure,
unsupported Clojure idioms, 
and hard-to-check code.

\paragraph{Lessons}
Based on our empirical survey, HMaps and Java interoperability support
are vital features used on average more than once per typed
function. 
%
Multimethods are less common
in our case studies. The CircleCI code base contains only 26 multimethods total
in 55,000 lines of mixed untyped-typed Clojure code,
a low number in our experience.

%The
%data therefore validates our choice of a type system that supports
%expressive multimethod definition and acknowledges the relationship
%between these seemingly-distinct features. 

%
%The other lesson from our case studies and from other interactions
%with Typed Clojure users, it is clear the main barrier to entry to
%Typed Clojure for large systems is the requirement to annotate
%functions outside the borders of typed code.  We hope that this
%can be addressed by making annotations available for popular
%libraries.

\section{Further challenges}
\label{sec:limitations}

After a 2 year trial, the second case study decided to disabled type checking~\cite{CircleCIBlog}.
They were supportive of the fundamental ideas presented in this paper, but primarily
cited issues with the checker implementation in practice and would reconsider
type checking if they were resolved. This is also supported by \figref{experience:featuretable},
where 78\% of \clj{def} expressions are unchecked.

\smallsection{Performance}
Rechecking files with transitive dependencies is expensive since all dependencies must be rechecked.
We conjecture caching type state will significantly
improve re-checking performance,
though preserving static soundness in the context of arbitrary code reloading is a largely unexplored area.

\smallsection{Library annotations}
Annotations for external code are rarely available, so a large part of the
untyped-typed porting process is reverse engineering libraries.

\smallsection{Unsupported idioms}
While the current set of features is vital to checking Clojure code,
there is still much work to do.
For example, common Clojure functions are often too polymorphic for the current implementation
or theory to account for. The post-mortem~\cite{CircleCIBlog} contains more details.

%\smallsection{Java Arrays}
%Java arrays are known to be statically unsound.
%\cite{Bra98} summarises the approach taken to regain runtime soundness, which involves
%checking array writes at runtime.
%
%Typed Clojure implements an experimental partial solution, making arrays \emph{bivariant},
%separating the write and read types into contravariant and covariant parameters.
%If the array originates from typed code, then we may track the write and read
%parameters statically. Currently arrays from foreign sources
%have their write parameter set to to \Bot{}, protecting typed code from writing
%something of incorrect type. However there are currently no casting mechanisms to 
%convince Typed Clojure the foreign array is writeable.

%\smallsection{Array-backed sequences}
%Typed Clojure assumes sequences are immutable. This is almost always true, however
%for performance reasons,
%sequences created from Java arrays (and Iterables) reflect future writes to the array 
%in the `immutable' sequence. While disturbing and a clear unsoundness in Typed Clojure,
%this has not yet been an issue in practice and is strongly discouraged as undefined behavior:
%``Robust programs should not mutate arrays or Iterables that have seqs on them.''~\cite{CljSeqDoc}.
%
%\smallsection{Typed-untyped interoperation}
%Currently, interactions between typed and untyped Clojure code are unchecked
%which can violate the expectations of Typed Clojure.
%Gradual typing~\cite{thf06,siek06:_gradual} ensures sound interoperability between typed and untyped code by enforcing
%invariants of the type system via run-time contracts.
% We hope to add support
%for gradual typing in the future.



%OLD

%\subsection{Using negative filters}
%
%Occurrence typing plays an important role in Typed Racket and Typed Clojure.
%By maintaining a \emph{proposition environment} of propositions relating types to
%bindings, we can update bindings with more accurate types as programs progress.
%It follows that there is some correspondence between propositions and types,
%characterised by the \emph{update} function, which takes a type and a proposition
%and returns a type which updates the input type using the proposition.
%
%There is a straightforward relationship between ``positive'' propositions and types.
%For example 
%{\tt (update Number (is Integer 0))}
%updates Number by Integer, which is Integer, because Integer <: Number.
%
%The relationship between ``negative'' propositions and types is not always obvious.
%A common proposition in Typed Clojure is (! (U nil false) a): the proposition that
%local binding ``a'' is \emph{not} of type (U nil false).
%This problem is most visible in expressions like {\tt (filter identity coll)}, where
%``identity'' has a ``then'' proposition that has negative information: (! (U nil false) 0),
%which reads, the 0th argument of identity does not contain (U nil false).
%
%\subsubsection{Arrays}
%\label{sec:arrays}
%
%Supporting statically sound interactions with Java arrays is a goal
%of Typed Clojure. This is complicated by Java's decision to make
%arrays covariant in their argument, a well documented source of static
%unsoundness. Bracha~\cite{Bra98} summarises Java's approach to maintaining
%soundness at runtime, which involves all array writes being checked by
%runtime assertions.
%
%This approach fits Java's type system, but we can do better in a more powerful
%type system like Typed Clojure. Our goal is to catch all type-incorrect array
%writes at compile time so the type system can do more to help Clojure programmers
%use arrays, especially those being passed from foreign Java code.
%
%Our basic approach is to make our array types \emph{bivariant}. Array types
%look like {\ArrayTwo {\t{w}} {\t{r}}} and
%are reminiscent of functions or pipes: having a contravariant parameter for input (writing)
%and a covariant parameter for output (reading).
%This type can write type {\t{w}} and read type {\t{r}}.
%
%Most commonly, an array type is invariant in its parameter; it can
%write and read input of the same type.
%We can get the same effect by setting our input and output
%parameters to the same type. For example, {\ArrayTwo {\Number} {\Number}}
%(or equivalently, {\Array {\Number}})
%in Typed Clojure is similar to invariant array types of \Number in languages like Scala.
%
%The biggest gain in using a separate input parameter is the ability
%to specify \emph{read-only} arrays. Crucially, our type system features an
%explicit bottom type \lstinline|Nothing|, enabling a read-only \lstinline|Number| array
%to be of type \lstinline|(Array2 Nothing Number)|.
%
%To realise why defining read-only arrays are useful, we need to examine
%what makes array covariance unsound in Java.
%\begin{verbatim}
%FIXME
%Array covariance about the type of an array so the consumer
%of an array cannot tell the actual type of the array when examining a type
%signature.
%\end{verbatim}
%
%\begin{lstlisting}
%...
%public static Number[] getNumberArray() {
%  Number[] n = new Integer[10];
%  return n;
%}
%...
%\end{lstlisting}
%
%To the casual consumer \emph{getNumberArray} returns an array that can both
%read and write \lstinline|Number|s. However it is clear from the implementation
%that attempting to write say a \lstinline|Double| to this array will result
%in a runtime error.
%
%\begin{verbatim}
%...
%Number[] myArray = getNumberArray();
%myArray[0] = 1.1;
%/* Exception in thread "main" 
%   java.lang.ArrayStoreException: 
%   java.lang.Double */
%...
%\end{verbatim}
%
%Notice that this is a runtime error, and Java's type system has not helped
%statically prevent it.
%This could cause a similar issue for other statically-typed languages offering
%interoperability with Java. 
%
%To prevent these sorts of runtime exceptions in Typed Clojure, we declare
%all arrays from unknown sources to be \emph{read-only}. Put differently,
%the only way to define a writeable array is to create it in type-checked Clojure
%code.
%
%\begin{lstlisting}
%(let [n (CovariantArray/getNumberArray)]
%  (aset n 0 1.1))
%
%; Polymorphic static method clojure.lang.RT/aset could not be 
%; applied to arguments:
%; Domains: 
%;         (Array2 i o) clojure.core.typed/AnyInteger i
%; 
%; Arguments:
%;         (Array2 Nothing java.lang.Number) int (Value 1.1)
%; 
%; with expected type:
%;         Any
%\end{lstlisting}
%
%The type inferred for the local \lstinline|n| is \lstinline|(Array2 Nothing Number)|
%which tells the type system: it is never safe to write to this array, but
%it is safe to assume \lstinline|Number|s can be read from this array.
%
%To emphasise, Typed Clojure throws a static type error. Errors like this help Clojure programmers
%use foreign Java libraries more correctly.
%
%\begin{verbatim}
%Note that Java libraries are often large 
%and complex and programmers will probably
%enjoy the extra help from the type system.
%\end{verbatim}

\chapter{Conclusion}
\label{sec:conclusion}

Optional type systems must be designed with close attention to the
language that they are intended to work for.
We have therefore designed Typed Clojure, an optionally-typed version of
Clojure, with a type system that works with a wide variety of distinctive
Clojure idioms and features. Although based on the foundation of Typed
Racket's occurrence typing approach, Typed Clojure both extends the
fundamental control-flow based reasoning as well as applying it to
handle seemingly unrelated features such as multi-methods. In
addition, Typed Clojure supports crucial features such as
heterogeneous maps and Java interoperability while integrating these
features into the core type system. Not only are each of these
features important in isolation to Clojure and Typed Clojure
programmers, but they must fit together smoothly to ensure that
existing untyped programs are easy to convert to Typed Clojure.

The result is a sound, expressive, and useful type system which, as
implemented in \coretyped with appropriate extensions, is suitable for
typechecking a significant amount of existing Clojure programs.
%
As a result, Typed Clojure is already successful: it is used in
the Clojure community among both enthusiasts and professional
programmers.% and receives contributions from many developers.

Our empirical analysis of existing Typed Clojure programs bears out
our design choices. Multimethods, Java interoperation, and
heterogeneous maps are indeed common in both Clojure and Typed Clojure,
meaning that our type system must accommodate them. Furthermore, they
are commonly used together, and the features of each are mutually
reinforcing. Additionally, the choice to make Java's \clj{null}
explicit in the type system is validated by the many Typed Clojure
programs that  specify non-nullable types.

% Delete the following paragraphs if space is needed.

However, there is much more that Typed Clojure can provide. Most
significantly, Typed Clojure currently does not provide \emph{gradual
  typing}---interaction between typed and untyped code is unchecked and
thus unsound. We hope to explore the possibilities of using existing
mechanisms for contracts and proxies in Java and
Clojure to enable sound gradual typing for Clojure.

Additionally, the Clojure compiler is unable to use Typed Clojure's
wealth of static information to optimize programs. Addressing this
requires not only  enabling sound gradual typing, but also
integrating Typed Clojure into the Clojure tool so
that its information can be communicated to the compiler. 

Finally, our case study, evaluation, and broader experience indicate that Clojure
programmers still find themselves unable to use Typed Clojure on some
of their programs for lack of expressiveness. This requires continued
effort to analyze and understand the features and idioms and
develop new type checking approaches.


\part{Automatic Annotations for Typed Clojure}
\label{part:autoann}

\chapter{Abstract}
%This paper shows how to
%generate recursive heterogeneous type annotations for
%programs that manipulate plain data.
%Our approach includes observing an instrumented running program,
%and using a novel algorithm to ``squash'' the observed structure
%of program values into named recursive types.
%
%We apply this approach to generate Typed Clojure annotations,
%and report on experience in using our tool to generate annotations for real-world
%Clojure programs, and enumerate the remaining changes needed to fully port
%them to Typed Clojure.

%%%% V1

%The untyped-typed porting process is costly, but tools to smooth
%this transition are scarce.
%To type check a dynamically-typed program
%with most optional type systems,
%type annotations must be added.
%This burden is sometimes large, and has put off real users 
%attempting to migrate to
%existing optional
%type systems. When not discouraged, programmers often
%annotate tens of thousands of lines of code without assistance.

%Programming languages that encourage programming with plain values over declared classes
%tend follow Alan Perlis' advise:
%``It is better to have 100 functions operate on one data structure than 10 functions on 10 data structures.''
%As a consequence, a rich variety of idioms and 

%As a consequence of this flexibility, the central data structures in such languages
%are used in numerous, implicit 

% Recursive Type Annotations for Ad-Hoc Data Structures
% Semi-Automatic Type Annotations for Recursive Ad-Hoc Data Structures
% Tool-Assisted Type Annotations for Recursive Ad-Hoc Data Structures
% Annotating Recursive Ad-Hoc Data Structures for Optional Type Systems

% A Semi-Automatic Workflow for Type Annotating Ad-Hoc Data Structures

We present a semi-automated workflow for porting
untyped programs to annotation-driven optional type systems.
Unlike previous work, we infer useful types for
recursive heterogeneous entities that have ``ad-hoc''
representations as plain data structures like maps, vectors, and sequences.

Our workflow starts by using dynamic analysis to collect samples from program execution
via test suites or examples.
Then, initial type annotations are inferred by
combining observations across different parts of the program.
Finally, the programmer uses the type system
as a feedback loop to tweak the provided annotations until they
type check.

Since inferring perfect annotations is usually undecidable
and dynamic analysis is necessarily incomplete,
the key to our approach is generating close-enough annotations
that are easy to manipulate to their final form by following static type error
messages.
We explain our philosophy behind achieving this
%, including choosing compact, recursive types with recognizable names,
along with a formal model of the automated stages of our workflow,
featuring maps as the primary ``ad-hoc'' data representation.

We report on using our workflow to convert real untyped Clojure programs to type check with Typed Clojure,
which both feature extensive support for ad-hoc data representations.
First, we visually inspect the initial annotations for conformance to our philosophy.
Second, we quantify the kinds of manual changes needed to amend them.
Third, we verify the initial annotations are meaningfully underprecise by enforcing them at runtime.

We find that the kinds of changes needed are
usually straightforward operations on the initial annotations,
leading to a substantial reduction in the effort required to port such programs.

%Our only requirement of existing
%programs is that they be runnable, with a suite of tests or
%examples. Given a running program, we instrument the execution, record
%type information, summarize it, and annotate the existing program with
%the recovered types.

%We present an approach to lighten the load on programmers moving to
%optional types. 

%We apply our approach to Clojure, a dynamically typed
%language with a culture of unit testing as well as both an existing
%optional type system and a contract system. Given a component under
%consideration, we instrument the source and analyze the behavior of the
%program while running unit tests.
%Equipped with this information, we summarize it by generating compact
%type specifications for all the functions in the component, including
%well-named type definitions. Our tool can also automatically generate
%contracts using the Clojure spec tool. Since Clojure relies
%heavily on ad-hoc data structures in the Lisp tradition, we describe
%an algorithm for automatically inferring recursive structural types
%from data examples, a challenge not considered in prior work.

%Our approach, as must be the case for a testing-driven tool, is
%incomplete---programs may have too few unit tests, and untested
%execution paths can have differing type behavior. We therefore
%evaluate our tool by running it on real Clojure programs and then
%completing the porting to Typed Clojure. We find that while
%some changes are always needed, the generated types are
%valuable and the effort reduction is substantial.

%%%% OLD
%  The untyped-typed porting process is costly, but
%  tools to smooth this transition are scarce.
%  We isolate the process of writing static type
%  annotations for untyped top-level variables, often manual and tedious.
%  Programmers must first annotate their own variablr es.
%  Even then, annotate used libraries. u
%  Worse still, annotate variables in the macroexpansion of imported macros.
%
%  In this paper, we explore a
%  tool dyinfer to generate type annotations.
%  What makes a good annotation?
%  Annotations should be readable, compact,
%  and capture the essential structure of
%  the running program.
%  Annotations should help programmers
%  type check their programs
%  by capturing relevant usages in the current 
%  context.
%  We aim to generate mostly-good annotations
%  that require little engineering effort
%  to correct, of which most is driven by
%  static type errors.
%
%  Our tool instruments running programs
%  and summarises observed values, outputing type annotations.
%  We handle higher-order functions
%  and record-like constructs.
%  To improve readability and make annotations
%  more useful for type checking programs, we generate
%  recursive union types when appropriate.
%
%  We apply our algorithm to Clojure programs
%  to generate Typed Clojure annotations.
%  We show the resulting annotations are often
%  adequate to type check usages.


\Dchapter{Introduction}
\label{infer:chapter:intro}

%\input{infer-old-pldi-intro}

Consider the exercise of counting binary tree nodes using JavaScript.
With a class-based tree representation, we naturally add a method
to each kind of node like so.

\begin{lstlisting}[language=JavaScript]
class Node { nodes() { return 1 + this.left.nodes() + this.right.nodes(); } }
class Leaf { nodes() { return 1; } }
new Node(new Leaf(1), new Leaf(2)).nodes(); //=> 3 (constructors implicit)
\end{lstlisting}

An alternative ``ad-hoc'' representation uses plain JavaScript Objects
with explicit tags, which is less extensible but simpler.
Then, the method becomes a recursive function that explicitly takes a tree as input.

\begin{lstlisting}[language=JavaScript]
function nodes(t) { switch t.op { 
                      case "node": return 1 + nodes(t.left) + nodes(t.right);
                      case "leaf": return 1; } }
nodes({op: "node", left:{op: "leaf", val: 1}, right:{op: "leaf", val: 2}})//=>3
\end{lstlisting}

Now, consider the problem of inferring type annotations for these programs.
The class-based representation is idiomatic to popular dynamic languages
like JavaScript and Python, and so many existing solutions support it.
%~\infercitep{saftoiu2010jstrace,pyannotate,typette18,An10dynamicinference,pytype,kristensen2017inference}
For example, TypeWiz~\infercitep{typewiz} uses dynamic analysis to generate
the following TypeScript annotations from the above example execution of \js{nodes}.

\begin{lstlisting}[language=JavaScript]
class Node { public left: @Leaf@; public right: @Leaf@; ... }
class Leaf { public val: @number@; ... }
\end{lstlisting}

The intuition behind inferring such a type is straightforward.
For example, an instance of \js{Leaf} was observed in \js{Node}'s \js{left} field,
and so the nominal type \js{Leaf} is used for its annotation.

The second ``ad-hoc'' style of programming seems peculiar in JavaScript, Python, and, indeed,
object-oriented style in general.
Correspondingly, existing state-of-the-art automatic annotation tools are not designed
to support them.
There are several ways to trivially handle such cases.
Some enumerate the tree representation ``verbatim'' in a union, like TypeWiz~\infercitep{typewiz}.

\begin{lstlisting}[language=JavaScript]
function nodes(t: {left: {op: string, val: number}, op: string,
                   right: {op: string, val: number}}
                | {op: string, val: number}) ...
\end{lstlisting}

Others ``discard'' most (or all) structure, like Typette~\infercitep{typette18} 
and PyType~\infercitep{pytype} for Python.

\begin{lstlisting}[language=Python]
def nodes(t: Dict[(Sequence, object)]) -> int: ... # Typette
def nodes(t) -> int: ...                           # PyType
\end{lstlisting}

Each annotation is clearly insufficient to meaningfully check both the function definition
and valid usages. To show a desirable annotation for the ``ad-hoc'' program,
we port it to Clojure~\infercitep{Hic08}, where it
enjoys full support from the built-in runtime verification library
clojure.spec and primary optional type system Typed Clojure~\infercitep{bonnaire2016practical}.

\begin{lstlisting}[language=Clojure]
(defn nodes [t] (case (:op t)
                  :node (+ 1 (nodes (:left t)) (nodes (:right t)))
                  :leaf 1))
(nodes {:op :node, :left {:op :leaf, :val 1}, :right {:op :leaf, :val 2}}) ;=>3
\end{lstlisting}

Making this style viable requires a harmony of language features, in particular to
support programming with functions and immutable values, but
none of which comes at the expense of object-orientation. Clojure is hosted on the Java Virtual Machine
and has full interoperability with Java objects and classes---even Clojure's core design embraces
object-orientation by exposing a collection of Java interfaces to create new kinds of data structures.
The \clj{\{k v ...\}} syntax creates a persistent and immutable Hash Array Mapped Trie~\cite{bagwell2001ideal},
which can be efficiently manipulated by dozens of built-in functions.
The leading colon syntax like \clj{:op} creates an interned \emph{keyword}, which are ideal for map keys
for their fast equality checks, and also look themselves up in maps when used as functions
(e.g., \clj{(:op t)} is like JavaScript's \js{t.op}).
\emph{Multimethods} regain the extensibility we lost when abandoning methods, like the following.

\begin{lstlisting}[language=Clojure]
(defmulti nodes-mm :op)
(defmethod nodes-mm :node [t] (+ 1 (nodes-mm (:left t)) (nodes-mm (:right t))))
(defmethod nodes-mm :leaf [t] 1)
\end{lstlisting}

On the type system side, Typed Clojure supports a variety of heterogeneous types,
in particular for maps, along with occurrence typing \cite{TF10} to follow local control flow.
Many key features come together to represent our ``ad-hoc'' binary tree as the following type.

\begin{lstlisting}[language=Clojure]
(defalias Tree
  (U '{:op ':node, :left Tree, :right Tree}
     '{:op ':leaf, :val Int}))
\end{lstlisting}

The \clj{defalias} form introduces an equi-recursive type alias \clj{Tree},
\clj{U} a union type, \clj{'\{:kw Type ...\}} for heterogeneous keyword map types,
and \clj{':node} for keyword singleton types.
With the following function annotation, Typed Clojure can intelligently type check
the definition and usages of \clj{nodes}.

\begin{lstlisting}[language=Clojure]
(ann nodes [Tree -> Int])
\end{lstlisting}

This (manually written) Typed Clojure annotation involving \clj{Tree}
is significantly different from TypeWiz's ``verbatim'' annotation for \js{nodes}.
First, it is recursive, and so supports trees of arbitrary depth (TypeWiz's annotation
supports trees of height $<3$).
Second, it uses singleton types \clj{':leaf} and \clj{':node} to distinguish each case
(TypeWiz upcasts \js{"leaf"} and \js{"node"} to \js{string}).
Third, the tree type is factored out under a name to enhance readability and reusability.
On the other end of the spectrum, the ``discarding'' annotations of Typette and PyType
are too imprecise to use meaningfully (they include trees of arbitrary depth, but
also many other values).

The challenge we overcome in this research is to automatically generate
annotations like Typed Clojure's \clj{Tree}, in such a way that the ease of manual amendment is
only mildly reduced by unresolvable ambiguities and incomplete data collection.

%This research presents an approach based on dynamic analysis to automatically inferring 
%recursive, structural, and compact annotations
%like the Typed Clojure annotation for \clj{nodes} along with \clj{Tree}.
%We show how our approach tolerates unresolvable ambiguities and incomplete data collection
%to generate close-enough annotations that are often straightforward to manually amend.

\Dchapter{Overview}
\label{infer:sec:overview}

%\input{infer-old-pldi-overview}

We demonstrate our approach by synthesizing a Typed Clojure annotation for \clj{nodes}.
The following presentation is somewhat loose to keep from being bogged down by details---interested
readers may follow the pointers to subsequent sections where they are made precise.

We use dynamic analysis to observe the execution of functions, so we give an
explicit test suite for \clj{nodes}.

\begin{lstlisting}[language=Clojure]
(def t1 {:op :node, :left {:op :leaf, :val 1}, :right {:op :leaf, :val 2}})
(deftest nodes-test (is (= (nodes t1) 3)))
\end{lstlisting}

The first step is the instrumentation phase
(formalized in \secref{infer:sec:formal:collection-phase}), which 
monitors the inputs and outputs of \clj{nodes}
by redefining it to use the \clj{track} function like so (where \clj{<nodes-body>} begins the \clj{case} expression of
the original \clj{nodes} definition):

\begin{lstlisting}[language=Clojure]
(def nodes (fn [t'] (track ((fn [t] <nodes-body>) (track t' ['nodes :dom]))
                           ['nodes :rng])))
\end{lstlisting}

The \clj{track} function (given later in \figref{infer:fig:trackmeta})
takes a value to track and a
\emph{path} that represents its origin, and returns an instrumented value
along with recording some runtime samples about the value.
A path is represented as a vector of \emph{path elements},
and describes the source of the value in question.
For example, \clj{(track 3 ['nodes :rng])}
returns \clj{3} and records the sample
\resentrynm{\clj{['nodes :rng]}}{\clj{Int}}
which says ``\clj{Int} was recorded at \clj{nodes}'s range.''
%
Running our test suite \clj{nodes-test} with an instrumented \clj{nodes}
results in more samples like this, most which use the path element
\clj{\{:key :kw\}} which represents a map lookup on the \clj{:kw} entry.

\begin{itemize}
  \item \resentrynm{\clj{['nodes :dom \{:key :op\}]}}{\clj{':leaf}}
  \item \resentrynm{\clj{['nodes :dom \{:key :op\}]}}{\clj{':node}}
  \item \resentrynm{\clj{['nodes :dom \{:key :val\}]}}{\clj{?}}
  \item \resentrynm{\clj{['nodes :dom \{:key :left\}]}}{\clj{?}}
  \item \resentrynm{\clj{['nodes :dom \{:key :right\}]}}{\clj{?}}
  \item \resentrynm{\clj{['nodes :dom \{:key :left\} \{:key :op\}]}}{\clj{':leaf}}
  \item \resentrynm{\clj{['nodes :dom \{:key :right\} \{:key :op\}]}}{\clj{':leaf}}
  \item \resentrynm{\clj{['nodes :dom \{:key :left\} \{:key :val\}]}}{\clj{Int}}
  \item \resentrynm{\clj{['nodes :dom \{:key :right\} \{:key :val\}]}}{\clj{Int}}
  \item \resentrynm{\clj{['nodes :rng]}}{\clj{Int}}
\end{itemize}

Now, our task is to transform these samples into a readable and useful annotation.
This is the function of the inference phase (formalized in \secref{infer:sec:formal:inference-phase}),
which is split into three passes: first it generates a naive type from samples, then it
combines types that occur syntactically near eachother (``squash locally''),
and then aggressively across different function annotations (``squash globally'').

The initial naive type generated from these samples resembles TypeWiz's
``verbatim'' annotation given in \Dchapref{infer:chapter:intro}, except
the \clj{?} placeholder represents incomplete information about a path
(this process is formalized as \generatetenv{} in \figref{infer:fig:generatetenv}).

\begin{lstlisting}[language=Clojure]
(ann nodes [(U '{:op ':leaf,
                 :val ?}
               '{:op ':node,
                 :left '{:op ':leaf,
                         :val Int},
                 :right '{:op ':leaf,
                          :val Int}})
            -> Int])
\end{lstlisting}

Next, the two ``squashing'' phases.
The intuition behind both are based on seeing types as directed graphs,
where vertices are type aliases, and an edge connects
two vertices $u$ and $v$ if $u$ is mentioned in $v$'s type.

Local squashing (\squashlocal{} in \figref{infer:fig:squashlocal})
constructs such a graph by creating type aliases from map types
using a post-order traversal of the original types.
In this example, the previous annotations become:

\begin{lstlisting}[language=Clojure]
(defalias op-leaf1 '{:op ':leaf, :val ?})
(defalias op-leaf2 '{:op ':leaf, :val Int})
(defalias op-leaf3 '{:op ':leaf, :val Int})
(defalias op-node '{:op ':node,
                    :left op-leaf2,
                    :right op-leaf3})
(ann nodes [(U op-leaf1 op-node) -> Int])
\end{lstlisting}

As a graph, this becomes the left-most graph below. The dotted edge
from \clj{op-leaf2} to \clj{op-leaf1} signifies that they are to be merged,
based on the similar structure of the types they point to.

\begin{tikzpicture}[
        > = stealth, % arrow head style
        shorten > = 1pt, % don't touch arrow head to node
        auto,
        %node distance = 1.5cm, % distance between nodes
        semithick % line style
    ]
  \tikzstyle{invalias}=[
      draw = none,
      thick,
      fill = none,
      rounded rectangle,
      text opacity=0
  ]
  \tikzstyle{alias}=[
      draw = black,
      thick,
      fill = white,
      rounded rectangle,
  ]
  \tikzstyle{ann}=[
      draw = gray,
      thick,
      fill = white,
      rectangle,
  ]

  \begin{scope}

    \node[ann]   (nodes) {\clj{nodes}};
    \node[alias] (leaf1) [right = 0.7cm of nodes] {\clj{op-leaf1}};
    \node[alias] (leaf2) [below = 0.1cm of leaf1] {\clj{op-leaf2}};
    \node[alias] (leaf3) [below = 0.1cm of leaf2] {\clj{op-leaf3}};
    \node[alias] (op-node) [below of=nodes]       {\clj{op-node}};

    \path[->] (nodes) edge node {} (leaf1);
    \path[->] (nodes) edge node {} (op-node);
    \path[->] (op-node) edge node  {} (leaf2.west);
    \path[->] (op-node) edge node  {} (leaf3.west);

    \draw[->,dotted,thick,red] (leaf2.east) to [bend right] (leaf1.east);
  \end{scope}

  \begin{scope}[xshift=4.5cm]
    \node[ann]   (nodes) {\clj{nodes}};
    \node[alias] (leaf1) [right = 0.7cm of nodes] {\clj{op-leaf1}};
    \node[invalias] (leaf2) [below = 0.1cm of leaf1] {\clj{op-leaf2}};
    \node[alias] (leaf3) [below = 0.1cm of leaf2] {\clj{op-leaf3}};
    \node[alias] (op-node) [below of=nodes]       {\clj{op-node}};

    \path[->] (nodes) edge node {} (leaf1);
    \path[->] (nodes) edge node {} (op-node);
    \path[->] (op-node) edge node  {} (leaf1);
    \path[->] (op-node) edge node  {} (leaf3);

    \draw[->,dotted,thick,red] (leaf3.east) to [bend right] (leaf1.east);
  \end{scope}

  \begin{scope}[xshift=9cm]
    \node[ann]   (nodes) {\clj{nodes}};
    \node[alias] (leaf1) [right = 0.7cm of nodes] {\clj{op-leaf1}};
    \node[invalias] (leaf2) [below = 0.1cm of leaf1] {\clj{op-leaf2}};
    \node[invalias] (leaf3) [below = 0.1cm of leaf2] {\clj{op-leaf3}};
    \node[alias] (op-node) [below of=nodes]       {\clj{op-node}};

    \path[->] (nodes) edge node {} (leaf1);
    \path[->] (nodes) edge node {} (op-node);
    \path[->] (op-node) edge node  {} (leaf1);
  \end{scope}
\end{tikzpicture}

After several merges (reading the graphs left-to-right), local squashing results in the following:

\begin{lstlisting}[language=Clojure]
(defalias op-leaf '{:op ':leaf, :val Int})
(defalias op-node '{:op ':node, :left op-leaf, :right op-leaf})
(ann nodes [(U op-leaf op-node) -> Int])
\end{lstlisting}

All three duplications of the \clj{':leaf} type in the naive annotation have
been consolidated into their own name,
with the \clj{?} placeholder for the \clj{:val} entry being absorbed into \clj{Int}.

Now, the global squashing phase (\squashglobal{} in \figref{infer:fig:squashglobal})
proceeds similarly, except the notion of a vertex is expanded to also include
\emph{unions} of map types, calculated, again, with a post-order traversal of the types
giving:

\begin{lstlisting}[language=Clojure]
(defalias op-leaf '{:op ':leaf, :val Int})
(defalias op-node '{:op ':node, :left op-leaf, :right op-leaf})
(defalias op-leaf-node (U op-leaf op-node))
(ann nodes [op-leaf-node -> Int])
\end{lstlisting}

This creates \clj{op-leaf-node}, giving the left-most graph below.

\begin{tikzpicture}[
        > = stealth, % arrow head style
        shorten > = 1pt, % don't touch arrow head to node
        auto,
        %node distance = 1.5cm, % distance between nodes
        semithick % line style
    ]
  \tikzstyle{invalias}=[
      draw = none,
      thick,
      fill = none,
      rounded rectangle,
      text opacity=0
  ]
  \tikzstyle{alias}=[
      draw = black,
      thick,
      fill = white,
      rounded rectangle,
  ]
  \tikzstyle{ann}=[
      draw = gray,
      thick,
      fill = white,
      rectangle,
  ]

  \begin{scope}
    \node[ann]   (nodes) {\clj{nodes}};
    \node[alias] (op-leaf-node) [below = 0.7cm of nodes] {\clj{op-leaf-node}};
    \node[alias] (op-leaf) [right = 0.7cm of nodes] {\clj{op-leaf}};
    \node[alias] (op-node) [right = 0.5cm of op-leaf-node] {\clj{op-node}};

    \path[->] (nodes) edge node {} (op-leaf-node);
    \path[->] (op-leaf-node) edge node  {} (op-leaf);
    \path[->] (op-leaf-node) edge node  {} (op-node);
    \path[->] (op-node) edge node  {} (op-leaf);

    \draw[->,dotted,thick,red] (op-leaf-node) to [bend right] (op-leaf);
  \end{scope}

  \begin{scope}[xshift=5cm]
    \node[ann]   (nodes) {\clj{nodes}};
    \node[alias] (op-leaf-node) [below = 0.7cm of nodes] {\clj{op-leaf-node}};
    \node[invalias] (op-leaf) [right = 0.7cm of nodes] {\clj{op-leaf}};
    \node[alias] (op-node) [right = 0.5cm of op-leaf-node] {\clj{op-node}};

    \path[->] (nodes) edge node {} (op-leaf-node);
    \draw[->] (op-leaf-node) to [bend right=20] (op-node);
    \draw[->] (op-node) to [bend right=20] (op-leaf-node);

    \draw[->,dotted,thick,red] (op-node) to (op-leaf-node);
  \end{scope}

  \begin{scope}[xshift=10cm]
    \node[ann]   (nodes) {\clj{nodes}};
    \node[alias] (op-leaf-node) [below = 0.7cm of nodes] {\clj{op-leaf-node}};

    \path[->] (nodes) edge node {} (op-leaf-node);
    \path[->] (op-leaf-node.east) edge [loop right] node {} (op-leaf-node.east);
  \end{scope}
\end{tikzpicture}

Now, type aliases are merged based on overlapping \emph{sets} of top-level keysets and likely tags.
Since \clj{op-leaf} and \clj{op-leaf-node} refer to maps with identical keysets
(\clj{:op} and \clj{:val}) and whose likely tags agree (the \clj{:op} entry
is probably a tag, and they are both \clj{':leaf}),
they are merged and all occurrences of \clj{op-leaf}
are renamed to \clj{op-leaf-node}, creating a \emph{mutually recursive type}
between the remaining aliases in the middle graph:

\begin{lstlisting}[language=Clojure]
(defalias op-node '{:op ':node, :left op-leaf-node, :right op-leaf-node})
(defalias op-leaf-node (U '{:op ':leaf, :val Int} op-node))
(ann nodes [op-leaf-node -> Int])
\end{lstlisting}

In the right-most graph, the aliases \clj{op-node} and \clj{op-leaf-node} are merged for similar reasons:

\begin{lstlisting}[language=Clojure]
(defalias op-leaf-node
  (U '{:op ':leaf, :val Int}
     '{:op ':node, :left op-leaf-node, :right op-leaf-node}))
(ann nodes [op-leaf-node -> Int])
\end{lstlisting}

All that remains is to choose a recognizable name for the alias.
Since all its top-level types seem to use the \clj{:op} entry for
tags, we choose the name \clj{Op} and output the final annotation:

\begin{lstlisting}[language=Clojure]
(defalias Op (U '{:op ':leaf, :val Int}
                '{:op ':node, :left Op, :right Op}))
(ann nodes [Op -> Int])
\end{lstlisting}

The rest of the porting workflow involves the programmer repeatedly type checking
their code and gradually tweaking the generated annotations until they type check.
It turns out that this annotation immediately type checks the definition of \clj{nodes}
and all its valid usages, so we turn to a more complicated function \clj{visit-leaf}
to demonstrate a typical scenario.

\begin{lstlisting}[language=Clojure]
(defn visit-leaf "Updates :leaf nodes in tree t with function f."
  [f t] (case (:op t)
          :node (assoc t :left (visit-leaf f (:left t))
                         :right (visit-leaf f (:right t)))
          :leaf (f t)))
\end{lstlisting}

This higher-order function uses \clj{assoc} to associate new children
as it recurses down a given tree to update leaf nodes with the provided function.
The following test simply increments the leaf values of the previously-defined \clj{t1}.

\begin{lstlisting}[language=Clojure]
(deftest visit-leaf-test 
  (is (= (visit-leaf (fn [leaf] (assoc leaf :val (inc (:val leaf)))) t1)
         {:op :node, :left {:op :leaf, :val 2}, :right {:op :leaf, :val 3}})))
\end{lstlisting}

Running this test under instrumentation yields some interesting runtime samples
whose calculation is made efficient by space-efficient tracking (\secref{infer:sec:space-efficient-tracking}),
which ensures a function is not repeatedly tracked unnecessarily.
The following two samples demonstrate how to handle multiple arguments
(by parameterizing the \clj{:dom} path element)
and higher-order functions
(by nesting \clj{:dom} or \clj{:rng} path elements).

\begin{itemize}
  \item \resentrynm{\clj{['visit-leaf \{:dom 1\} \{:key :op\}]}}{\clj{':leaf}}
  \item \resentrynm{\clj{['visit-leaf \{:dom 0\} \{:dom 0\} \{:key :op\}]}}{\clj{':leaf}}
\end{itemize}

Here is our automatically generated initial annotation.

% initial annotation
\begin{lstlisting}[language=Clojure]
(defalias Op (U '{:op ':leaf, :val t/Int} '{:op ':node, :left Op, :right Op}))
(ann visit-leaf [[Op -> Any] Op -> Any])
\end{lstlisting}

Notice the surprising occurrences of \clj{Any}. They originate
from \clj{?} placeholders due to the lazy tracking of maps 
(\secref{infer:sec:lazy-tracking}).
Since \clj{visit-leaf} does not traverse the results of \clj{f},
nor does anything traverse \clj{visit-leaf}'s results (hash-codes are used for equality checking)
neither tracking is realized.
Also notice the first argument of \clj{visit-leaf} is underprecise.
These could trigger type errors on usages of \clj{visit-leaf},
so manual intervention is needed (highlighted). We factor out and use a new alias \clj{Leaf}
and replace occurrences of \clj{Any} with \clj{Op}.

\begin{lstlisting}[language=Clojure]
(*@\colorbox{pink}{(defalias Leaf}@*) '{:op ':leaf, :val Int}(*@\colorbox{pink}{)}@*)
(defalias Op (U (*@\colorbox{pink}{Leaf}@*) '{:op ':node, :left Op, :right Op}))
(ann visit-leaf [[(*@\colorbox{pink}{Leaf}@*) -> (*@\colorbox{pink}{Op}@*)] Op -> (*@\colorbox{pink}{Op}@*)])
\end{lstlisting}

% TODO Point to experiments
We measure the success of our workflow by using it to type check real Clojure programs.
Experiment 1 (\secref{infer:sec:experiment1}) manually inspects a selection of inferred types.
Experiment 2 (\secref{infer:sec:experiment2}) classifies and quantifies the kinds of changes needed.
Experiment 3 (\secref{infer:sec:experiment3}) enforces initial annotations at runtime to ensure
they are meaningfully underprecise.

%\input{infer-algorithm} % old stuff
\Dchapter{Formalism}
\label{infer:sec:formalism}

We present \lambdatrack{}, an untyped $\lambda$-calculus
describing the essense of our approach to automatic annotations.
We split our model into two phases: the collection phase 
\collectOp{}
that runs an instrumented program and collects observations, and
an inference phase 
\inferanns{}
that derives type annotations from these observations
that can be used to automatically annotate the program.

We define the top-level driver function \annotateOp{} that connects
both pieces.
It says, given a program \e{}
and top-level variables $\ova{\x{}}$ to infer annotations for,
return an annotation environment \atenv{} with possible entries for
$\ova{\x{}}$ based on observations from evaluating
an instrumented \e{}.
%
%\begin{mathpar}
%\infer[]
%{ \collectnoalign{\e{}}{\ova{\x{}}}{\res{}}
%  \\
%  \inferannsnoalign{\res{}}{\atenv{}}
%}
%{ \annotatenoalign{\e{}}{\ova{\x{}}}{\atenv{}} }
%\end{mathpar}
\begin{mathpar}
  \begin{array}{lllll}
    \annotateOp{} : \e{}, {\ova{\x{}}} \rightarrow \atenv{}\\
    \annotateOp{} = \inferanns{} \circ \collectOp{}
  \end{array}
\end{mathpar}

To contextualize the presentation of these phases, we begin a running example:
inferring the type of a top-level function $f$, that takes a map and
returns its {\makekw{a}} entry, 
based on the following usage.
%
\begin{Verbatim}[commandchars=\\\{\}, codes={\catcode`$=3\catcode`^=7}]
  define $f$ = \uabs{m}{\getexp{m}{\makekw{a}}}
  \appexp{f}{\curlymap{\makekw{a} 42}} => 42
\end{Verbatim}
%
Plugging this example into our driver function
we get a candidate annotation for $f$:
$$
\annotatenoalign{\appexp{f}{\curlymap{\makekw{a}\ 42}}}{[f]}{\{\hastype{f}{[\{\makekw{a}\ \IntT{}\} \rightarrow \IntT{}]}\}}
$$

\Dsection{Collection phase}
\label{infer:sec:formal:collection-phase}

\begin{figure}
  $$
  \begin{altgrammar}
    \val{} &::=& \num{}
       \alt {\kw{}}
       \alt {\closure{\uabs{\x{}}{\e{}}}{\openv{}}}
       \alt {\curlymap{\ova{\kw{}\ {\val{}}}}}
       \alt {\const{}}
       &\mbox{Values} \\
   \e{} &::=& \x{}
       \alt \val{}
       \alt \trackE{\e{}}{\inferpath{}}
       \alt {\uabs{\x{}}{\e{}}}
       \alt {\curlymapvaloverrightnoarrow{\e{}}{\e{}}}
       \alt {\appexp{\e{}}{\ova{\e{}}}}
       &\mbox{Expressions} \\
    \openv{} &::=& \{\ova{x \mapsto \val{}}\}
       &\mbox{Runtime environments} \\
   \inferpth{}
      &::=& \x{}
       \alt \dompe{}
       \alt \rngpe{}
       \alt {\inferkeype{\HMapreq{}}{\kw{}}}
       &\mbox{Path Elements} \\
   \inferpath{} &::=& \ova{\inferpth{}}
       &\mbox{Paths} \\
       \res{}
      &::=& \restwoarrow{\inferpath{}}{\ty{}}
      &\mbox{Inference results} \\
    \ty{}, \s{}
      &::=& \IntT{}
       \alt \arrow{\ty{}}{\ty{}}
       %\alt \HMappretty{\ova{\kw{}\ \ty{}}}
       \alt \HMaptwo{\HMapreq{}}{\HMapopt{}}
       \alt \Unionsplice{\ova{\ty{}}}
       \\
       &\alt& \alias{} % type alias
       \alt \kw{}
       \alt \Keyword{}
       \alt \Top{}
       \alt \IPersistentMap{\ty{}}{\ty{}}
       \alt \UnknownT{}
      &\mbox{Types} \\
    \tenv{} &::=& \{\ova{\hastype{\x{}}{\ty{}}}\}
      &\mbox{Type environments} \\
    \HMapreq{}, \HMapopt{}
      &::=& \{ \ova{\kw{}\ {\ty{}}} \}
      &\mbox{HMap entries} \\
    \aenv{} &::=& \{\ova{\alias{} \mapsto \ty{}}\}
      &\mbox{Type alias environments} \\
    \atenv{} &::=& (\aenv{}, \tenv{})
      &\mbox{Annotation environments} \\
  \end{altgrammar}
  $$
\caption{Syntax of Terms, Types, Inference results, and Environments for \lambdatrack{}}
\label{infer:fig:syntax}
\end{figure}

%\begin{figure*}
%  \ifdefined\PAPER
%  \footnotesize
%  \fi
%\begin{mathpar}
%  \begin{array}{llll}
%    \infer [B-Track]
%    { \opsemtrack{\openv{}}{\e{}}{\val{}}{\res{}} \\\\
%    \trackmeta{\val{}}{\inferpath{}}{\vp{}}{\resp{}}}
%    { \opsemtrack{\openv{}}{\trackE{\e{}}{\inferpath{}}}{\vp{}}{\unionres{\res{}}{\resp{}}}
%    }
%  \end{array}
%
%\infer [B-App]
%{ \opsemtrack{\openv{}}{\e{1}}{\closure{\uabs{\x{}}{\e{}}}{\openvp{}}}{\res{1}} \\\\
%  \opsemtrack{\openv{}}{\e{2}}{\val{}}{\res{2}} \\\\
%  \opsemtrack{\extendopenv{\openvp{}}{\x{}}{\val{}}}{\e{}}{\vp{}}{\res{3}} \\
%}
%{ \opsemtrack{\openv{}}{\appexp{\e{1}}{\e{2}}}{\vp{}}{\bigunionres{\ova{\res{i}}}}
%}
%
%  \begin{array}{lll}
%    \infer [B-Clos]
%    {}
%    { \opsemtrack{\openv{}}{\uabs{\x{}}{\e{}}}{\closure{\uabs{\x{}}{\e{}}}{\openv{}}}{\emptyres{}}}
%    \\\\
%    \infer [B-Val]
%    {}
%    { \opsemtrack{\openv{}}{\val{}}{\val{}}{\emptyres{}} }
%    \ \ \ \ \ \ 
%
%    \infer [B-Var]
%    {}
%    { \opsemtrack{\openv{}}{\xvar{}}{\inopenvnoeq{\openv{}}{\xvar{}}}{\emptyres{}}
%    }
%  \end{array}
%
%\infer [B-Delta]
%{ \opsemtrack{\openv{}}{\e{}}{\const{}}{\res{1}}
%  \\
%  \overrightarrow{\opsemtrack {\openv{}}{\ep{}}{\val{}}{\resp{}}}
%  \\\\
%  \inferconstantopsem{\const{}}{\ova{\val{}}}{\vp{}}{\res{2}}
%}
%{ \opsemtrack {\openv{}}
%              {\appexp {\e{}} {\overrightarrow{\ep{}}}}
%              {\vp{}}
%              {\overrightarrow{\unionres{\res{}}{\resp{}}}}
%       }
%
%\end{mathpar}
%\caption{Operational Semantics for \lambdatrack{}}
%\label{infer:fig:semantics}
%\end{figure*}

Now that we have a high-level picture of how these phases interact,
we describe the syntax and semantics of \lambdatrack{}, before
presenting the details of \collectOp{}.
%
\figref{infer:fig:syntax} presents the syntax of \lambdatrack{}.
Values \val{} consist of numbers \num{}, Clojure-style keywords {\kw{}},
closures {\closure{\uabs{\x{}}{\e{}}}{\openv{}}}, constants \const{},
and keyword keyed hash maps {\curlymapvaloverrightnoarrow{\kw{}}{\val{}}}.

Expressions \e{} consist of variables \x{}, values,
functions, maps, and function applications.
The special form
\trackE{\e{}}{\inferpath{}}
observes {\e{}} as related to path {\inferpath{}}.
Paths \inferpath{} 
record the source of a runtime value with respect
to a sequence of path elements \inferpth{}, always starting with
a variable \x{}, and are read left-to-right.
Other path elements are
a function domain \dompe{}, 
a function range \rngpe{},
and a map entry {\inferkeype{\ova{\kw{1}}}{\kw{2}}}
which represents the result of looking up {\kw{2}}
in a map with keyset ${\ova{\kw{1}}}$.

Inference results \restwoarrow{\inferpath{}}{\ty{}}
are pairs of paths {\inferpath{}} and types \ty{}
that say the path \inferpath{} was observed to be 
type \ty{}.
Types \ty{} are numbers \IntT{}, function types \arrow{\ty{}}{\ty{}},
ad-hoc union types \Union{\ty{}}{\ty{}},
type aliases \alias{},
%top type \Top{},
and unknown type \UnknownT{} that represents
a temporary lack of knowledge during the inference process.
Heterogeneous keyword map types \HMappretty{\ova{\kw{}\ \ty{}}}
for now represent a series of required keyword entries---we will extend
them to have optional entries in later phases.

The big-step operational semantics
{\opsemtrack{\openv{}}{\e{}}{\val{}}{\res{}}}
(\figref{infer:fig:trackmeta})
says under runtime environment \openv{}
expression \e{} evaluates to value \val{}
with inference results \res{}.
Most rules are standard, with extensions to correctly
propagate inference results \res{}.
B-Track is the only interesting rule, which instruments
its fully-evaluated argument with the \trackmetaOp{}
metafunction.

The metafunction \trackmeta{\val{}}{\inferpath{}}{\vp{}}{\res{}} (\figref{infer:fig:trackmeta})
says if value \val{} occurs at path {\inferpath{}}, then return a possibly-instrumented
\vp{} paired with inference results {\res{}} that can be immediately derived
from the knowledge that \val{} occurs at path {\inferpath{}}.
It has a case for every kind of value.
The first three cases records the number input as type {\IntT{}}.
The fourth case, for closures, returns a wrapped value
resembling higher-order function contracts~\infercitep{findler2002contracts},
but we track the domain and range rather than verify them.
The remaining rules case, for maps, recursively tracks each map value,
and returns a map with possibly wrapped values.
Immediately accessible inference results are combined
and returned.
A specific rule for the empty map is needed because we otherwise only rely on
recursive calls to \trackEOp{} to gather inference results---in the empty case,
we have no data to recur on.

\begin{figure}
\begin{mathpar}
    \infer [B-Track]
    { \opsemtrack{\openv{}}{\e{}}{\val{}}{\res{}} \\\\
    \trackmeta{\val{}}{\inferpath{}}{\vp{}}{\resp{}}}
    { \opsemtrack{\openv{}}{\trackE{\e{}}{\inferpath{}}}{\vp{}}{\unionres{\res{}}{\resp{}}}
    }

\infer [B-App]
{ \opsemtrack{\openv{}}{\e{1}}{\closure{\uabs{\x{}}{\e{}}}{\openvp{}}}{\res{1}} \\\\
  \opsemtrack{\openv{}}{\e{2}}{\val{}}{\res{2}} \\\\
  \opsemtrack{\extendopenv{\openvp{}}{\x{}}{\val{}}}{\e{}}{\vp{}}{\res{3}} \\
}
{ \opsemtrack{\openv{}}{\appexp{\e{1}}{\e{2}}}{\vp{}}{\bigunionres{\ova{\res{i}}}}
}

    \infer [B-Clos]
    {}
    { \opsemtrack{\openv{}}{\uabs{\x{}}{\e{}}}{\closure{\uabs{\x{}}{\e{}}}{\openv{}}}{\emptyres{}}}

    \infer [B-Val]
    {}
    { \opsemtrack{\openv{}}{\val{}}{\val{}}{\emptyres{}} }


    \infer [B-Var]
    {}
    { \opsemtrack{\openv{}}{\xvar{}}{\inopenvnoeq{\openv{}}{\xvar{}}}{\emptyres{}}
    }

\infer [B-Delta]
{ \opsemtrack{\openv{}}{\e{}}{\const{}}{\res{1}}
  \\
  \overrightarrow{\opsemtrack {\openv{}}{\ep{}}{\val{}}{\resp{}}}
  \\\\
  \inferconstantopsem{\const{}}{\ova{\val{}}}{\vp{}}{\res{2}}
}
{ \opsemtrack {\openv{}}
              {\appexp {\e{}} {\overrightarrow{\ep{}}}}
              {\vp{}}
              {\overrightarrow{\unionres{\res{}}{\resp{}}}}
       }

  \arraycolsep=1.4pt
  \begin{array}{lllll}
    %\trackmeta{\val{}}{\inferpath{}}{\val{}}{\res{}}\\\\

    \trackmetaalign{\num{}}{\inferpath{}}{\num{}}{\singletonres{\inferpath{}}{\IntT{}}}
    \\
    \trackmetaalign{\kw{}}{\inferpath{}}{\kw{}}
                   {\singletonres{\inferpath{}}
                                 {\Keyword{}}}
    \\
    \trackmetaalign{\const{}}{\inferpath{}}{\const{}}{\emptyres{}}
    \\
    \trackmetalhs{\closure{\uabs{\x{}}{\e{}}}{\openv{}}}
                 {\inferpath{}}
                 &=&
                 \trackmetarhs
                   {\closure{\ep{}}{\openv{}}}
                   {\emptyres{}}
                   % let's treat functions as opaque wrt arity as they are in clojure
                   %{\singletonres{\inferpath{}}
                   %              {\arrow{\UnknownT{}}{\UnknownT{}}}}
         \\
    &&
    \begin{array}{@{}llll}
      \text{where } \y{} \text{ is fresh},\\
       \text{ }    \begin{array}{@{}llll}
                      \ep{} =
                        \uabs{\y{}}{\trackE{&\appexp{(\uabs{\x{}}{\e{}})}{\trackE{\yvar{}}{\appendone{\inferpath{}}{\dompe{}}}}}
                                           {\\&\appendone{\inferpath{}}{\rngpe{}}}}
                   \end{array}
    \end{array}
    \\
    \trackmetaalign{\{\}}
                   {\inferpath{}}
                   {\{\}}
                   {\singletonres{\inferpath{}}
                                 {\HMappretty{}}}
    \\
    \trackmetaalign{\{\ova{\kw{1}\ {\kw{2}}}\ 
                      \ova{\kw{}\ {\val{}}}
                    \}}
                   {\inferpath{}}
                   {\{\ova{\kw{1}\ {\kw{2}}}\ 
                      \ova{\kw{}\ {\vp{}}}
                    \}}
                   {\bigunionres{\res{}}}
    \\
    &&
    \text{where } \ova{\trackmeta{\val{}}
                                            {\appendone{\inferpath{}}
                                                       {\inferkeype{\{\ova{\kw{1}\ {\kw{2}}}\ 
                                                                      \ova{\kw{}\ {\UnknownT{}}}
                                                                    \}}
                                                                   {\kw{}}}}
                                            {\vp{}}
                                            {\res{}}}
    \\
  \end{array}
  
  \arraycolsep=1.4pt
  \begin{array}{lllr}
    \inferconstantopsemalign{\assocliteral{}}{\curlymap{\ova{\kw{}\ \val{}}}, \kwp{}, \vp{}}{\updatemap{\curlymap{\ova{\kw{}\ \val{}}}}{\kwp{}}{\vp{}}}
                            {\emptyres{}}\\
    \inferconstantopsemalign{\getliteral{}}{\curlymap{\kw{}\ \val{}, \ova{\kwp{}\ \vp{}}}, \kw{}}{\val{}}
                            {\emptyres{}}\\
    \inferconstantopsemalign{\dissocliteral{}}{\curlymap{\kw{}\ \val{}, \ova{\kwp{}\ \vp{}}}, \kw{}}{\curlymap{\ova{\kwp{}\ \vp{}}}}
                            {\emptyres{}}\\
  \end{array}
\end{mathpar}
\caption{Operational semantics, \trackmeta{\val{}}{\inferpath{}}{\val{}}{\res{}} and constants}
\label{infer:fig:trackmeta}
\end{figure}

Now we have sufficient pieces to describe the initial collection phase of our model.
Given an expression \e{} and variables ${\ova{\x{}}}$ to track,
\instrumentnoalign{\e{}}{\ova{\x{}}}{\ep{}}
returns an instrumented expression \ep{}
that tracked usages of $\ova{\x{}}$.
It is defined via capture-avoiding substitution:
$$
\instrumentnoalign{\e{}}{\ova{\x{}}}{\replacefor{\e{}}{\ova{\trackE{\x{}}{[\x{}]}}}{\ova{\x{}}}}
$$

Then, the overall collection phase 
\collectnoalign{\e{}}{\ova{\x{}}}{\res{}}
says, given an expression \e{}
and variables
$\ova{\x{}}$
to track,
returns inference results {\res{}}
that are the results of evaluating \e{}
with instrumented occurrences of $\ova{\x{}}$.
It is defined as:
%
$$
\collectnoalign{\e{}}{\ova{\x{}}}{\res{}}, \text{ where }
  \opsemtrack{}{\instrument{\e{}}{\ova{\x{}}}}{\val{}}{\res{}}
$$

For our running example
of collecting for the program \appexp{f}{\curlymap{\makekw{a}\ 42}},
we instrument the program by wrapping occurrences of $f$ with \trackEOp{}
with path $[f]$.
$$
\instrumentnoalign{\appexp{f}{\curlymap{\makekw{a}\ 42}}}{[f]}{\appexp{\trackE{f}{[f]}}{\curlymap{\makekw{a}\ 42}}}
$$

Then we evaluate the instrumented program and derive two inference results
(colored in red for readability):
$$
\opsemtrack{}{\appexp{\trackE{f}{[f]}}{\curlymap{\makekw{a}\ 42}}}{42}{\resflatcolor{\resentry{[f, \dompe{}, \inferkeypenokeyset{[\makekw{a}]}{\makekw{a}}]}{\IntT{}}, \resentry{[f, \rngpe{}]}{\IntT{}}}}
$$

Here is the full derivation:
\begin{Verbatim}[commandchars=\\\{\}, codes={\catcode`$=3\catcode`^=7}]
=> \appexp{\trackE{f}{[f]}}{\curlymap{\makekw{a}\ 42}}
=> \trackE{\getexp{\trackE{\curlymap{\makekw{a}\ 42}}{[f, \dompe{}]}}{\makekw{a}}}{[f, \rngpe{}]}
=> \trackE{\getexp{{\curlymap{\makekw{a}\ 42}} ; \resflatcolor{\resentry{[f, \dompe{}, \inferkeypenokeyset{[\makekw{a}]}{\makekw{a}}]}{\IntT{}}}}{\makekw{a}}}{[f, \rngpe{}]}
=> \trackE{42 ; \resflatcolor{\resentry{[f, \dompe{}, \inferkeypenokeyset{[\makekw{a}]}{\makekw{a}}]}{\IntT{}}}}{[f, \rngpe{}]}
=> $42 ; \resflatcolor{\resentry{[f, \dompe{}, \inferkeypenokeyset{[\makekw{a}]}{\makekw{a}}]}{\IntT{}}, \resentry{[f, \rngpe{}]}{\IntT{}}}$
\end{Verbatim}

Notice that intermediate values can have inference results (colored) attached to them with a semicolon,
and the final value has inference results about both $f$'s domain and range.

\Dsection{Inference phase}
\label{infer:sec:formal:inference-phase}

After the collection phase, we have a collection of inference results \res{}
which can be passed to the 
metafunction \inferanns{}(\res{}) = \atenv{} to produce an annotation environment:
\begin{mathpar}
  \begin{array}{lllll}
    \inferanns{} : \res{} \rightarrow \atenv{}\\
    \inferanns{} = \inferrecOp{} \circ \generatetenv{}\\\\
  \end{array}
\end{mathpar}
%
The first pass $\generatetenv{} (\res{}) = \tenv{}$ generates an initial type environment
from inference results \res{}.
%It is defined (in \figref{infer:fig:generatetenv})
%as a fold over \res{}, building a \tenv{} incrementally via the \inferupdateOp{}
%metafunction.
%
The second pass
$$\squashlocal{}(\tenv{}) = \atenvp{}$$
creates individual type aliases
for each HMap type in \tenv{} and then merges aliases that both occur inside the same
nested type into possibly recursive types. % (\figref{infer:fig:squashlocal}).
%
The third pass $\squashglobal{} (\atenv{}) = \atenvp{}$
merges type aliases in \atenv{} based on their similarity. % (\figref{infer:fig:squashglobal}).

\Dsubsection{Pass 1: Generating initial type environment}
\label{infer:sec:formal:inference-phase:genenv}

\begin{figure*}
\begin{mathpar}
  \begin{array}{rllll}
    \joinOp{} : \ty{}, \ty{} \rightarrow \ty{}
    \\
    \joinalign{\Unionsplice{\ova{\s{}}}}{\ty{}}{\Unionsplice{\ova{\joinexpression{\s{}}{\ty{}}}}}
    \\
    \joinalign{\ty{}}{\Unionsplice{\ova{\s{}}}}{\Unionsplice{\ova{\joinexpression{\s{}}{\ty{}}}}}
    \\
    \joinalign{\UnknownT{}}{\ty{}}{\ty{}}
    \\
    \joinalign{\ty{}}{\UnknownT{}}{\ty{}}
    \\
    \joinalign{\arrow{\ty{1}}{\s{1}}}{\arrow{\ty{2}}{\s{2}}}
              {\arrow{\joinexpression{\ty{1}}{\ty{2}}}
                     {\joinexpression{\s{1}}{\s{2}}}}
    \\
    \joinalign{\HMaptwo{\HMapreq{1}}{\HMapopt{1}}}
              {\HMaptwo{\HMapreq{2}}{\HMapopt{2}}}
              {\joinHMapexpression{\HMaptwo{\HMapreq{1}}{\HMapopt{1}}}
                                  {\HMaptwo{\HMapreq{2}}{\HMapopt{2}}}}
                                  \text{,}
                                  \\
    &&\ova{(\kw{}, \kw{i}) \in {\HMapreq{i}}} \Rightarrow \ova{\kw{i-1} = \kw{i}}
    \\
    \joinalign{\ty{}}{\s{}}{\Union{\ty{}}{\s{}}} \text{, otherwise}
  \end{array}
%
  \begin{array}{lllll}
    \joinHMapnoalign{\HMaptwo{\HMapreq{1}}{\HMapopt{1}}}{\HMaptwo{\HMapreq{2}}{\HMapopt{2}}}{\HMaptwo{\HMapreq{}}{\HMapopt{}}}
    \\
    \begin{array}{lllll}
      \text{where}
          &\mathsf{req}  = \bigcup \ova{\textsf{dom}({\HMapreq{i}})} \\
          &\mathsf{opt}  = \bigcup \ova{\textsf{dom}({\HMapopt{i}})} \\
          &\ova{\kw{}^r} = \bigcap \ova{\textsf{dom}({\HMapreq{i}})} \setminus \mathsf{opt}\\
          &\ova{\kw{}^o} = \mathsf{opt} \cup (\mathsf{req} \setminus \ova{\kw{}^r})\\
          &\HMapreq{}    = \{\ova{\kw{}^r\ \joinstarexpression{\ova{\HMapreq{i}[\kw{}^r]}}} \} \\
          &\HMapopt{}    = \{\ova{\kw{}^o\ \joinstarexpression{\ova{\HMapreq{i}[\kw{}^o], \HMapopt{i}[\kw{}^o]}}} \}
    \end{array}
  \end{array}

  \begin{array}{lllll}
    \textsf{fold} : \forall \alpha, \beta. (\alpha, \beta \rightarrow \alpha), \alpha, \ova{\beta} \rightarrow \alpha\\
    \textsf{fold}(\textsf{f}, \textsf{a}_0, \ova{\textsf{b}}^n) = \textsf{a}_n\\
    \begin{array}{llll}
      \text{where } \ova{\textsf{a}_i = \textsf{f}(\textsf{a}_{i-1}, \textsf{b}_{i})}^{1 \leq i \leq n}\\
    \end{array}
    \\\\
    \generatetenv{} : \res{} \rightarrow \tenv{}\\
    \generatetenv{} (\res{}) = \textsf{fold}(\inferupdateOp{}, \{\}, \res{})\\
  \end{array}
  \begin{array}{lllll}
    \inferupdateOp : \tenv{}, \resentry{\inferpath{}}{\ty{}} \rightarrow  \tenv{} 
    \\
    \inferupdatealign{\tenv{}}{\appendone{\inferpath{}}{\inferkeype{\{\ova{\kwp{}\ \s{}} \}}{\kw{}}}}{\ty{}}
            {\inferupdate{\tenv{}}{\inferpath{}}{\{\ova{\kwp{}\ \s{}}\ \kw{}\ \ty{} \}}}
    \\
    \inferupdatealign{\tenv{}}{\appendone{\inferpath{}}{\dompe{}}}{\ty{}}
                     {\inferupdate{\tenv{}}{\inferpath{}}{\arrow{\ty{}}{\UnknownT{}}}}
    \\
    \inferupdatealign{\tenv{}}{\appendone{\inferpath{}}{\rngpe{}}}{\ty{}}
                {\inferupdate{\tenv{}}{\inferpath{}}{\arrow{\UnknownT{}}{\ty{}}}}
    \\
    \inferupdatealign{\updatemap{\tenv{}}{\x{}}{\s{}}}{[x]}{\ty{}}
                     {\updatemap{\tenv{}}
                                {\x{}}
                                {\joinexpression{\ty{}}{\s{}}}
                                 }
    \\
    \inferupdatealign{\tenv{}}{[\xvar{}]}{\ty{}}{\updatemap{\tenv{}}{\x{}}{\ty{}}}
    \\
  \end{array}

\end{mathpar}
\caption{Definition of $\generatetenv{}(\res{}) = \tenv{}$}
\label{infer:fig:generatetenv}
\end{figure*}

The first pass is given in \figref{infer:fig:generatetenv}.
The entry point \generatetenv{} folds over inference results
to create an initial type environment via \inferupdateOp{}.
This style is inspired by occurrence typing~\infercitep{TF10},
from which we also borrow the concepts of paths into types.

We process paths right-to-left in \inferupdateOp{}, building
up types from leaves to root, before joining the fully constructed type with the existing
type environment via \joinOp{}.
The first case handles the \keypeOp{} path element.
The extra map of type information preserves both keyset
information and any entries that might represent tags
(populated by the final case of \trackEOp{}, \figref{infer:fig:trackmeta}).
This information helps us avoid prematurely collapsing tagged maps,
by the side condition of the HMap \joinOp{} case.
The \joinHMapOp{} metafunction aggressively combines two HMaps---required
keys in both maps are joined and stay required, otherwise keys
become optional.

The second and third \inferupdateOp{} cases update the domain and range of a function type,
respectively.
The \joinOp{} case for function types joins covariantly on the domain to yield more useful
annotations. For example, if a function accepts \IntT{} and \Keyword{},
it will have type
\joinnoalign{\arrow{\IntT{}}{\UnknownT{}}}{\arrow{\Keyword{}}{\UnknownT{}}}
{\arrow{\Union{\IntT{}}{\Keyword{}}}{\UnknownT{}}}.

Returning to our running example, we now want to convert our inference results
$$
\res{} = \{\resentry{[f, \dompe{}, \inferkeypenokeyset{[\makekw{a}]}{\makekw{a}}]}{\IntT{}}, \resentry{[f, \rngpe{}]}{\IntT{}}\}.
$$
into a type environment. Via $\generatetenv{}(\res{})$, we start to trace
\inferupdate{\{\}}{[f, \dompe{}, \inferkeypenokeyset{[\makekw{a}]}{\makekw{a}}]}{\IntT{}}

%\begin{figure*}
%\begin{mathpar}
%  \begin{array}{lllll}
%    \inferanns{} : \res{} \rightarrow \atenv{}\\
%    \inferanns{} = \squashglobal{} \circ \squashlocal{} \circ \generatetenv{}\\
%  \end{array}
%\end{mathpar}
%\caption{Algorithm summary: Generate an initial type environment from inference results.
%Then generate an updated type environment paired with a type alias environment by 1: creating
%a recursive type if a HMap contains another HMap whose keysets have a non-empty intersection, 2:
%globally merging type aliases based on identical HMap keysets, 3: cleaning up redundant aliases.}
%\label{infer:fig:inferanns}
%\end{figure*}

\Dsubsection{Pass 2: Squash locally}
\label{infer:sec:formal:inference-phase:squash-local}

\begin{figure*}
\begin{mathpar}
  \begin{array}{lllll}
    \aliashmap{} : \atenv{}, \ty{} \rightarrow (\atenv{}, \ty{})\\
    \aliashmap{}(\atenv{}, \ty{}) = \textsf{postwalk}(\atenv{}, \ty{}, \textsf{f})\\
    \begin{array}{lllll}
      \text{where} %&\textsf{f} : \atenv{}, \ty{} \rightarrow (\atenv{}, \ty{})\\
                   &\textsf{f}(\atenv{}, {\HMaptwo{\HMapreq{1}}{\HMapreq{2}}}) = \register{}(\atenv{},\HMaptwo{\HMapreq{1}}{\HMapreq{2}})\\
                   &\textsf{f}(\atenv{}, \Unionsplice{\ova{\ty{}}}) = \register{}(\atenv{},\Unionsplice{\ova{\fullyresolve{}(\ty{})}}) \text{,}\\
                   %&\text{ if } \exists\ty{}(\ova{\s{}}^m) \in\ova{\ty{}} .\ m > 0\\
                   &\text{ if } \alias{} \in\ova{\ty{}}\\
                   &\textsf{f}(\atenv{}, \ty{}) = (\atenv{}, \ty{}), \text{otherwise}
    \end{array}
  \end{array}
  \begin{array}{lllll}
    \register{} : \atenv{}, \ty{} \rightarrow (\atenv{}, \ty{})\\
    \register{}(\atenv{}, \ty{}) = (\updatemap{\atenv{}}{\alias{}}{\ty{}}, \alias{}), \text{ where } \alias{} \text{ is fresh}
    \\\\
    \fullyresolve{} : \atenv{}, \ty{} \rightarrow \ty{}\\
    \fullyresolve{}(\atenv{}, \alias{}) = \fullyresolve{}(\atenv{}[\alias{}])\\
    \fullyresolve{}(\atenv{}, \ty{}) = \ty{}  \text{, otherwise}
  \end{array}

  \begin{array}{lllll}
    \aliasesin{} : \ty{} \rightarrow \ova{\alias{}}\\
    \aliasesin{}(\alias{}) = [\alias{}]\\
    \aliasesin{}(\ty{}(\ova{\s{}})) = \bigcup{\ova{\aliasesin{}(\s{})}}
    \\\\
    \textsf{postwalk} : \atenv{}, \ty{}, (\atenv{}, \ty{} \rightarrow (\atenv{}, \ty{})) \rightarrow (\atenv{}, \ty{})\\
    \textsf{postwalk}(\atenv{0}, \ty{}(\ova{\s{}}^n), \textsf{w}) = \textsf{w}(\atenv{n}, \ty{}(\ova{\sp{}}))\\
    \begin{array}{lllll}
      \text{where}
        &\ova{(\atenv{i}, \sp{i}) = \textsf{postwalk}(\atenv{i-1}, \s{i}, \textsf{w})}\\
    \end{array}
    \\\\
    \mergealiases{} : \atenv{}, \ova{\alias{}} \rightarrow \atenv{}\\
    \mergealiases{}(\atenv{}, []) = \atenv{}\\
    \mergealiases{}(\atenv{}, [\alias{1} ... \alias{n}]) =
        \updatemap{\updatemapmulti{\atenv{}}{\alias{i}}{\alias{1}}}{\alias{1}}{\s{}}\\\
    \begin{array}{llll}
      \text{where } &\s{} = \joinstarexpression{\ova{\replacefor{\textsf{f}(\fullyresolve{}(\atenv{},\alias{i}))}{\alias{1}}{\alias{i}}}}
                    \\
                    &\textsf{f}(\aliasp{}) = \EmptyUnion{}, \text{ if } \aliasp{} \in \ova{\alias{}}\\
                    &\textsf{f}(\Unionsplice{\ova{\ty{}}}) = \Unionsplice{\ova{\textsf{f}(\ty{})}}\\
                    &\textsf{f}(\ty{}) = \ty{} \text{, otherwise}
    \end{array}
    \\\\
    %\trymergealias{} : \atenv{}, \alias{}, \alias{} \rightarrow \atenv{}
    %\\
    %\trymergealias{}(\atenv{}, \alias{1}, \alias{2}) =\\
    %\begin{array}{llll}
    %  \textsf{if } \neg \shouldmergeOp{}(\ova{\fullyresolve{}(\atenv{}, \alias{})}) \textsf{ then } \atenv{}\\
    %  \textsf{else } \updatemap{\updatemap{\atenv{}}{\alias{2}}{\alias{1}}}
    %                           {\alias{1}}
    %                           % this fixes the "too much garbage" issue
    %                           {\joinstarexpression{\ova{\replacefor{\atenv{}[\alias{i}]}{\alias{1}}{\alias{2}}}}}\\
    %\end{array}\\
      %fold-based version
    \squashlocal{} : \tenv{} \rightarrow \atenv{}\\
    \squashlocal{}(\tenv{}) = \textbf{fold}(\steptwohelper{}, \emptyatenv{}, \tenv{})\\
    \begin{array}{lllll}
      \text{where} &\steptwohelper{} (\atenv{}, \hastype{\x{}}{\ty{}}) = \updatemap{\atenv{2}}{\x{}}{\ty{2}}\\
                   &\begin{array}{lllll}
                      \text{where}
                        &(\atenv{1}, \ty{1}) = \aliashmap{}(\atenv{}, \ty{})\\
                        &(\atenv{2}, \ty{2}) = \squashall{}(\atenv{1}, \ty{1})\\
                    \end{array}
    \end{array}
      %nonfold-based version
    %\squashlocal{} : \atenv{} \rightarrow \atenv{}\\
    %\squashlocal{}(\atenv{0}) = \atenv{n}\\
    %\begin{array}{lllll}
    %  \text{where}\\
    %  \begin{array}{lllll}
    %    \steptwohelper{} (\atenv{}, \x{}, \ty{}) = \replacefor{\atenv{2}}{\ty{2}}{\x{}}\\
    %    \begin{array}{lllll}
    %      \text{where}
    %      &(\atenv{1}, \ty{1}) = \aliashmap{}(\atenv{}, \ty{})\\
    %      &(\atenv{2}, \ty{2}) = \squashall{}(\atenv{1}, \ty{1})\\
    %    \end{array}
    %    \\
    %    \ova{\hastype{\x{}}{\ty{}}}^n = \atenv{0}[\tenv{}]\\
    %    \ova{\atenv{i} = \steptwohelper{}(\atenv{i-1},{\x{i}},{\ty{i}})} \\
    %  \end{array}
    %\end{array}
  \end{array}
  \begin{array}{llll}
    \squashall{} : \atenv{}, \ty{} \rightarrow \atenv{}\\
    \squashall{}(\atenv{0}, \ty{}) = \atenv{n} \\
    \begin{array}{llll}
      \text{where }
      &\ova{\alias{}}^n = \aliasesin{}(\ty{})\\
      &\ova{\atenv{i} = \squash{}(\atenv{i-1}, [\alias{i}], [])}\\
    \end{array}
    \\\\
    \squash : \atenv{}, \ova{\alias{}}, \ova{\alias{}} \rightarrow \atenv{}\\
    \squash(\atenv{}, [], \textsf{d}) = \atenv{}\\
    \squash(\atenv{}, \alias{1} :: \textsf{w}, \textsf{d}) = \\
    \begin{array}{lllll}
      \squash(\atenvp{}, \textsf{w} \cup \textsf{as}, \textsf{d} \cup \{\alias{1}\})\\
      \begin{array}{@{}llll}
        \text{where}\\
        \begin{array}{@{}llll}
          &\textsf{as} = \aliasesin{}(\atenv{}[\alias{1}]) \setminus \textsf{d}\\
          &\textsf{ap} = \textsf{d} \setminus \{\alias{1}\}\\
          &\begin{array}{@{}llll}
             \textsf{f}(\atenv{}, \alias{2}) = 
               &\textsf{if } \neg \shouldmergeOp{}(\ova{\fullyresolve{}(\atenv{}, \alias{})}),\\
               &\textsf{then } \atenv{}\\
               &\textsf{else } \mergealiases{}(\atenv{}, \ova{\alias{i}})\\
          \end{array}\\
          &\begin{array}{@{}llll}
            \atenvp{} = &\textsf{if } \alias{} \in \textsf{d} \textsf{, then } \atenv{} \text{,}\\
            &\textsf{else } \textsf{fold}(\textsf{f}, \atenv{}, \textsf{ap} \cup \textsf{as})
          \end{array}
        \end{array}
      \end{array}
    \end{array}
    \\\\
    \shouldmergeOp{} : \ova{\ty{}} \rightarrow \textbf{Bool}\\
    \shouldmergeOp{}(\ova{\HMaptwo{\HMapreq{i}}{\HMapopt{i}}}) = \exists\kw{}. \ova{(\kw{}, \kw{i}) \in \HMapreq{i}}\\
    \shouldmergeOp{}(\ova{\ty{}}) = \textbf{F}, \text{ otherwise}
  \end{array}

\end{mathpar}
\caption{Definition of $\squashlocal{}(\tenv{}) = \atenv{}$
%\aliasesin{}(\ty{}) returns the set of aliases that syntactically occur in \ty{}.
%  Step 2 summary: Create aliases for HMaps (graph nodes), then squash recursive types locally
%(don't try to merge data examples from different paths). Omitted: follow-aliases call, that erases
%redundant aliases.
  }
  \label{infer:fig:squashlocal}
\end{figure*}

\begin{figure*}
\begin{mathpar}
  \begin{array}{lllll}
    \textsf{req}: \atenv{}, \alias{} \rightarrow \HMapreq{}\\
    \textsf{req}(\atenv{}, \alias{}) = \textsf{req}(\atenv{}, \atenv{}[\alias{}])\\
    \textsf{req}(\atenv{}, \HMaptwo{\HMapreq{}}{\HMapopt{}}) = \HMapreq{}\\
    \\\\
    %\textsf{req}: \atenv{}, \alias{} \rightarrow \ova{\HMapreq{}}\\
    %\textsf{req}(\atenv{}, \alias{}) = \textsf{req}(\atenv{}, \alias{})\\
    %\textsf{req}(\atenv{}, \HMaptwo{\HMapreq{}}{\HMapopt{}}) = [\HMapreq{}]\\
    %\textsf{req}(\atenv{}, \Unionsplice{\ova{\ty{}}}) = \bigcup\ova{\textsf{req}(\atenv{},\ty{})}\\
    %\textsf{req}(\atenv{}, \ty{}) = [] \text{, otherwise}
    \squashhorizonally{} : \atenv{} \rightarrow \atenv{}\\
    \squashhorizonally{}(\atenv{}) = \\
    \ \ \ \ 
      \textsf{fold}(\mergealiases{}, \atenv{}, \textsf{groupSimilarReq}(\atenv{}))\\\\
    \squashglobal{} : \atenv{} \rightarrow \atenv{}\\
    \squashglobal{} = \\
    \ \ \ \ \squashhorizonally{} \circ \aliassinglehmap{}
  \end{array}
  \begin{array}{lllll}
    \aliassinglehmap{} : \atenv{} \rightarrow \atenv{}\\
    \aliassinglehmap{}(\atenv{}) = \textsf{fold}(\textsf{f}, \atenvp{}, \atenvp{}[\aenv{}])\\
    \begin{array}{llll}
      \text{where} \\
      \begin{array}{llll}
                    &\atenvp{} = \textsf{fold}(\singlehmap{}, \atenv{}, \atenv{}[\tenv{}])\\
                    &\textsf{f}(\atenv{0}, \ty{}(\ova{\s{}}^n)) = (\atenv{n}, \ty{}(\sp{})) \text{, if } \ty{} = \HMaptwo{\HMapreq{}}{\HMapopt{}}\\
                    &\begin{array}{lllll}
                      \text{where } \ova{(\atenv{i}, \s{i}) = \singlehmap{}(\atenv{i-1}, \s{i})}
                     \end{array}
                    \\
                    &\textsf{f}(\atenv{}, \ty{}) = \singlehmap{}(\atenv{}, \ty{}) \text{, otherwise}
      \end{array}
    \end{array}
    \\\\
    \singlehmap{} : \forall \alpha. \atenv{}, (\alpha, \ty{}) \rightarrow \atenv{}\\
    \singlehmap{}(\atenv{}, (\textsf{x}, \ty{})) = \updatemap{\atenv{}}{\textsf{x}}{\s{}}\\
    \begin{array}{lllll}
      \text{where} &(\atenvp{}, \s{}) = \textsf{postwalk}(\atenv{}, \ty{}, \textsf{f})\\
                   &\textsf{f}(\atenv{}, {\HMaptwo{\HMapreq{1}}{\HMapreq{2}}}) = \register{}(\atenv{},\HMaptwo{\HMapreq{1}}{\HMapreq{2}})\\
                   &\textsf{f}(\atenv{}, \ty{}) = (\atenv{}, \ty{}), \text{otherwise}
    \end{array}
    %\\\\
    %\inferrecOp{} : \atenv{} \rightarrow \atenv{}\\
    %\inferrecOp{} = \squashglobal{} \circ \squashlocal{}
  \end{array}

  \begin{array}{lllll}
    \textsf{groupSimilarReq} : \atenv{} \rightarrow \ova{\ova{\alias{}}}\\
    \textsf{groupSimilarReq}(\atenv{}) = 
                   [\ova{\alias{}} | \ova{\kw{}} \in \textsf{dom}(\textsf{r}),
                                     \ova{\alias{}} = \textsf{remDiffTag}(\textsf{similarReq}(\ova{\kw{}}))
                                     ]\\
    \begin{array}{lllll}
      \text{where} \\
      \begin{array}{lllll}
                   &\textsf{r} = \{(\ova{\kw{}}, \ova{\alias{}}) | \HMaptwo{\{\ova{\kw{}\ \ty{}}\}}{\HMapopt{}} \in \textsf{rng}(\atenv{}[\aenv{}]),
                                                                              \ova{\alias{}} = \textsf{matchingReq}(\ova{\kw{}})\}\\
                   &\textsf{matchingReq}(\ova{\kw{}}) = [\alias{} | (\alias{}, \HMaptwo{\HMapreq{}}{\HMapopt{}}) \in \atenv{}]\\
                   &\textsf{similarReq}(\ova{\kw{}}) = [\alias{} | \ova{\kwp{}}^n \subseteq \ova{\kw{}}^m,
                                                                   m-n \leq \textsf{thres}(m),
                                                                   \alias{} \in \textsf{r}[\ova{\kwp{}}]]\\
                   &\textsf{remDiffTag}(\ova{\alias{}}) = [\aliasp{} | \aliasp{} \in \ova{\alias{}},
                                                                       \text{ if } (\kw{}, \kwp{}) \in \textsf{req}(\atenv{}, \aliasp{})
                                                                       \text{ and } 
                                                                       \bigvee\ova{(\kw{}, \kwpp{}) \in \textsf{req}(\atenv{}, \alias{})}
                                                                       \text{ then }
                                                                       \ova{\kwp{} = \kwpp{}}
                                                                      ]
      \end{array}
    \end{array}
  \end{array}
\end{mathpar}
\caption{
    Definition of $\squashglobal{}(\atenv{}) = \atenvp{}$ %\\
%  Step 3 summary: First ensure all HMaps correspond to an alias. Then merge
%aliases that point to a HMaps with identical required keysets (aliases must point to exactly one
%top-level HMap, no unions).
  }
  \label{infer:fig:squashglobal}
\end{figure*}


We now describe the algorithm for generating recursive type aliases.
The first step \squashlocal{} creates recursive types from directly nested types.
It folds over each type in the type environment, first
creating aliases with \aliashmap{}, and then
attempting to merge these aliases by \squashall{}.

A type is aliased by \aliashmap{} either if it is a union containing a HMap,
or a HMap that is not a member of a union.
While we will use the structure of HMaps to determine when to create a recursive
type, keeping surrounding type information close to HMaps helps create more
compact and readable recursive types.
The implementation uses a post-order traversal via \textsf{postwalk},
which also threads an annotation environment as it applies
the provided function.

Then, \squashall{} follows each alias \alias{i} reachable from the type environment
and attempts to merge it with any alias reachable from \alias{i}.
The \squash{} function maintains 
a set of already visited aliases to avoid infinite loops.

The logic for merging aliases is contained in \mergealiases{}.
Merging \alias{2} into \alias{1} involves mapping \alias{2}
to \alias{1} and \alias{1} to the join of both definitions.
Crucially, before joining, we rename occurrences of 
\alias{2} to \alias{1}. This avoids a linear increase in the
width of union types, proportional to the number of merged aliases.
%For example,
%without this, if we further merge \alias{3} into \alias{1},
%like \Unionsplice{{\alias{1}}\ {\alias{2}}\ {\alias{3}}} occur
%instead of simply \alias{1}.
The running time of our algorithm is proportional to the
width of union types (due to the quadratic combination of
unions in the join function) and this optimization greatly
helped the running time of several benchmarks.
To avoid introducing infinite types,
top-level references to other aliases we are merging with
are erased with the helper \textsf{f}.

The \shouldmergeOp{} function determines whether two types are related enough
to warrant being merged. We present our current implementation, which is simplistic,
but is fast and effective in practice, but many variations are possible.
Aliases are merged if they are all HMaps (not contained in unions), that
contain a keyword key in common, with possibly disjoint mapped values.
For example, our opening example has the \clj{:op} key mapped to either
\clj{:leaf} or \clj{:node}, and so aliases for each map would be merged.
Notice again, however, the join operator does not collapse differently-tagged
maps, so they will occur recursively in the resulting alias, but separated by union.

Even though this implementation of \shouldmergeOp{} does not directly utilize the aliased
union types carefully created by \aliashmap{}, they still affect the final types.
For example, squashing \clj{T} in
\begin{lstlisting}[language=Clojure]
(defalias T 
  (U nil '{:op :node :left '{:op :leaf ...} ...}))
\end{lstlisting}
results in 
\begin{lstlisting}[language=Clojure]
(defalias T 
  (U nil '{:op :node :left T ...} '{:op :leaf ...}))
\end{lstlisting}
rather than
\begin{lstlisting}[language=Clojure]
(defalias T2 (U '{:op :node :left T ...}
                '{:op :leaf ...}))
(defalias T (U nil T2))
\end{lstlisting}
An alternative implementation of \shouldmergeOp{} we experimented with included computing sets of keysets
for each alias, and merging if the keysets overlapped. This, and many of our early experimentations,
required expensive computations of keyset combinations and traversals over them that could be emulated
with cruder heuristics like the current implementation.

\Dsubsection{Pass 3: Squash globally}
\label{infer:sec:formal:inference-phase:squash-global}

The final step combines aliases
without restriction on whether they occur ``together''.
This step combines type information between different positions
(such as in different arguments or functions) so that any deficiencies
in unit testing coverage are massaged away.

The \squashglobal{} function is the entry point in this pass,
and is similar in structure to the previous pass.
It first creates aliases for each HMap via \aliassinglehmap{}.
Then, HMap aliases are grouped and merged in \squashhorizonally{}.

The \aliassinglehmap{} function first traverses the type environment
to create HMap aliases via \singlehmap{}, and binds the resulting
envionment as \atenvp{}.
Then, alias environment entries are updated with \textsf{f}, whose first
case prevents re-aliasing a top-level HMap, before we call \singlehmap{}
(\singlehmap{}'s second argument accepts both \x{} and \alias{}).
The \ty{}(\ova{\s{}}) syntax represents a type \ty{} whose constructor
takes types \ova{\s{}}.

After that, \squashhorizonally{} creates groups of related
aliases with \textsf{groupSimilarReq}.
Each group contains HMap aliases whose required keysets are similar,
but are never differently-tagged.
The code creates a map \textsf{r} from keysets to groups of HMap
aliases with that (required) keyset.
Then, for every keyset \ova{\kw{}}, \textsf{similarReq} adds aliases to the group
whose keysets are a subset of \ova{\kw{}}. The number of missing
keys permitted is determined by \textsf{thres}, for which we do not provide a
definition.
Finally, \textsf{remDiffTag} removes differently-tagged HMaps from each group,
and the groups are merged via \mergealiases{} as before.

\Dsubsection{Implementation}

Further passes are used in the implementation.
In particular, we trim unreachable aliases and remove aliases
that simply point to another alias (like \alias{2} in \mergealiases{})
between each pass.

% Things we could prove:
% - update is commutative

\Dchapter{Evaluation}
\label{infer:chap:evaluation}

We performed a quantitative evaluation of our workflow on several open source programs
in three experiments.
We ported five programs to Typed Clojure with our workflow,
and merely generated types for one larger program we deemed too difficult to port,
but features interesting data types.

Experiment 1 involves a manual inspection of the types from our automatic algorithm.
We detail our experience in generating types for part of an industrial-grade compiler which
we ultimately decided not to manually port to Typed Clojure.
This was because it uses many programming idioms beyond Typed Clojure's capabilities
(those detailed as ``Further Challenges'' by \infercitet{bonnaire2016practical}),
and so the final part of the workflow mostly involves working around its shortcomings.

Experiment 2 studies the kinds of the manual changes needed to port our five programs
to Typed Clojure, starting from the automatically generated annotations.
Experiment 3 enforces the initially generated annotations for these programs at runtime
to check they are meaningfully underprecise.

%\paragraph{cljs.compiler}
%ClojureScript (CLJS) is a Clojure variant that runs on JavaScript
%virtual machines. We infer types for its compiler (written in Clojure)
%which emits JavaScript from
%a recursively defined map-based abstract syntax tree format.

\Dsection{Experiment 1: Manual inspection}
\label{infer:sec:experiment1}

For the first experiment, we manually inspect the types automatically generated by our tool.
We judge our tool's ability to
use recognizable names,
favor compact annotations, and
not overspecify types.

\begin{figure}
\begin{lstlisting}[language=Clojure, numbers=left, numbersep=-8pt]
(defalias Op(*@\label{infer:listing:cljs:Op}@*) ; omitted some entries and 11 cases
  (U (HMap :mandatory(*@\label{infer:listing:cljs:Op:op:bindingStart}@*)
           {:op ':binding,(*@\label{infer:listing:cljs:Op:op:binding}@*)
            :info (U NameShadowMap(*@\label{infer:listing:cljs:Op:op:binding:NameShadowMap}@*)
                     FnScopeFnSelfNameNsMap(*@\label{infer:listing:cljs:Op:op:binding:FnScopeFnSelfNameNsMap}@*)),
            ...}
           :optional(*@\label{infer:listing:cljs:Op:optional}@*)
           {:env ColumnLineContextMap,
            :init Op,(*@\label{infer:listing:cljs:Op:optional:init:Op}@*)
            :shadow (U nil Op),(*@\label{infer:listing:cljs:Op:optional:shadow:Op}@*)
            ...})(*@\label{infer:listing:cljs:Op:optionalEnd}@*)(*@\label{infer:listing:cljs:Op:op:bindingEnd}@*)
    '{:op ':const,(*@\label{infer:listing:cljs:Op:op:const}@*)
      :env HMap49305,(*@\label{infer:listing:cljs:Op:op:const:HMap49305}@*)
      ...}
    '{:op ':do,(*@\label{infer:listing:cljs:Op:op:do}@*)
      :env HMap49305,(*@\label{infer:listing:cljs:Op:op:do:HMap49305}@*)
      :ret Op,(*@\label{infer:listing:cljs:Op:op:do:Op}@*)
      :statements (Vec Nothing)(*@\label{infer:listing:cljs:Op:op:do:statements}@*),
      ...}
    ...))(*@\label{infer:listing:cljs:Op-End}@*)
(defalias ColumnLineContextMap(*@\label{infer:listing:cljs:ColumnLineContextMap}@*)
  (HMap :mandatory {:column Int, :line Int}
        :optional {:context ':expr}(*@\label{infer:listing:cljs:ColumnLineContextMap:optional}@*)))(*@\label{infer:listing:cljs:ColumnLineContextMapEnd}@*)
(defalias HMap49305 ; omitted some extries(*@\label{infer:listing:cljs:HMap49305}@*)
  (U nil
     '{:context ':statement, :column Int, ...}
     '{:context ':return, :column Int, ...}
     (HMap :mandatory {:context ':expr, :column Int, ...}
           :optional {...})))(*@\label{infer:listing:cljs:HMap49305End}@*)
(ann emit [Op -> nil])(*@\label{infer:listing:cljs:emit}@*)
(ann emit-dot [Op -> nil])(*@\label{infer:listing:cljs:emit-dot}@*)
\end{lstlisting}
\caption{Sample generated types for cljs.compiler.
}
\label{infer:fig:cljs}
  %(ann emit-let [Op Any -> Any])(*@\label{infer:listing:cljs:emit-let}@*)
%    '{:op ':fn-method,
%      :body Op,
%      :children '[':params ':body],
%      :env HMap49305,
%      :fixed-arity Int,
%      :form (Coll (Coll Any)),
%      :name Op,
%      :params '[Op],
%      :recurs nil,
%      :type nil,
%      :variadic? false}
%    '{:op ':host-call,
%      :args '[Op],
%      :children Any,
%      :env context-statement-tmp-HMap-alias20275,
%      :form (Coll Sym),
%      :method Sym,
%      :tag Any,
%      :target Op}
%    '{:op ':host-field,
%      :children '[':target],
%      :env context-statement-tmp-HMap-alias20275,
%      :field Sym,
%      :form (Coll Sym),
%      :tag Sym,
%      :target Op}
%    '{:op ':if,
%      :children '[':test ':then ':else],
%      :else Op,
%      :env context-statement-tmp-HMap-alias20275,
%      :form (Coll Any),
%      :tag (Set (U nil Sym)),
%      :test Op,
%      :then Op,
%      :unchecked Boolean}
%    '{:op ':invoke,
%      :args '[Op],
%      :children '[':fn ':args],
%      :env context-statement-tmp-HMap-alias20275,
%      :fn Op,
%      :form (Coll Any),
%      :tag Sym}
%    (HMap
%      :mandatory
%      {:op ':js,
%       :env context-statement-tmp-HMap-alias20275,
%       :form (Coll (U nil Str Sym)),
%       :js-op Sym,
%       :numeric nil,
%       :tag Sym}
%      :optional
%      {:args '[Op Op],
%       :children '[':args],
%       :code Str,
%       :segs (Coll Str)})
%    (HMap
%      :mandatory
%      {:op ':js-var, :name Sym, :ns Sym}
%      :optional
%      {:tag Sym})
%    '{:op ':let,
%      :bindings '[Op Op Any],
%      :body Any,
%      :children Any,
%      :env context-statement-tmp-HMap-alias20275,
%      :form Any,
%      :tag Any}
%    (HMap
%      :mandatory
%      {:op ':local,
%       :env context-statement-tmp-HMap-alias20275,
%       :form Sym,
%       :info Op,
%       :local (U ':arg ':let),
%       :name Sym}
%      :optional
%      {:arg-id Int, :init Op, :tag Sym})
%    '{:op ':map,
%      :children '[':keys ':vals],
%      :env context-statement-tmp-HMap-alias20275,
%      :form AMap,
%      :keys '[Op],
%      :tag Sym,
%      :vals '[Op]}
%    (HMap
%      :mandatory
%      {:op ':var, :name Sym, :ns Sym}
%      :optional
%      {:arglists (Coll Any),
%       :arglists-meta (Coll nil),
%       :column Int,
%       :doc Str,
%       :end-column Int,
%       :end-line Int,
%       :env context-statement-tmp-HMap-alias20275,
%       :file (U nil Str),
%       :fn-var Boolean,
%       :form Sym,
%       :info (U nil ColumnFileLineMap),
%       :line Int,
%       :max-fixed-arity Int,
%       :meta
%       (U
%         ColumnFileLineMap__0
%         FileArglistsColumnMap
%         ColumnEndColumnEndLineMap),
%       :method-params (Coll (Coll Sym)),
%       :protocol-impl nil,
%       :protocol-inline nil,
%       :ret-tag Sym,
%       :tag Sym,
%       :top-fn ArglistsArglistsMetaMaxFixedArityMap,
%       :variadic? Boolean})))
\end{figure}

We take this opportunity to juxtapose some strengths and weaknessess
of our tool by discussing a somewhat problematic benchmark,
a namespace from the ClojureScript compiler called cljs.compiler
(the code generation phase).
We generate 448 lines of type annotations
for the 1,776 line file, and present a sample
of our tool's output as \figref{infer:fig:cljs}.
We were unable to fully complete the porting to Typed Clojure due to
type system limitations, but the annotations yielded by this benchmark
are interesting nonetheless.

The compiler's AST format is inferred as \clj{Op} (lines \ref{infer:listing:cljs:Op}-\ref{infer:listing:cljs:Op-End})
with 22 recursive references
(like lines \ref{infer:listing:cljs:Op:optional:init:Op}, \ref{infer:listing:cljs:Op:optional:shadow:Op}, \ref{infer:listing:cljs:Op:op:do:Op})
and 14 cases distinguished by \clj{:op} (like lines \ref{infer:listing:cljs:Op:op:binding},
\ref{infer:listing:cljs:Op:op:const}, \ref{infer:listing:cljs:Op:op:do}),
5 of which have optional entries (like lines \ref{infer:listing:cljs:Op:optional}-\ref{infer:listing:cljs:Op:optionalEnd}).
To improve inference time,
only the code emission unit tests were exercised (299 lines containing 39 assertions)
which normally take 40 seconds to run, from which we
generated 448 lines of types and 517 lines of specs
in 2.5 minutes on a 2011 MacBook Pro (16GB RAM, 2.4GHz i5),
in part because of key optimizations discussed in \Dchapref{infer:sec:extensions}.

The main function of the code generation phase is \clj{emit}, which
effectfully converts a map-based AST
to JavaScript.
The AST is created by functions in cljs.analyzer,
a significantly larger 4,366 line Clojure file.
Without inspecting cljs.analyzer,
our tool annotates \clj{emit} on line \ref{infer:listing:cljs:emit}
with a recursive AST type \clj{Op} (lines \ref{infer:listing:cljs:Op}-\ref{infer:listing:cljs:Op-End}).

Similar to our opening example \clj{nodes}, it uses the \clj{:op}
key to disambiguate between (16) cases, and has recursive
references (\clj{Op}).
We just present the first 4 cases.
The first case \clj{':binding} has 4 required
and 8 optional entries, whose
\clj{:info} and \clj{:env} entries refer to
other \clj{HMap} type aliases generated by the tool.
%%deleted this code
%Similar to \clj{:op},
%the \clj{:local} entry maps to a keyword singleton
%type,
%however our tool wisely chose to cluster types 
%based on the \clj{:op} entry since it is common to all cases.

%\Dsection{Philosophy}

An important question to address is ``how accurate are these annotations?''.
Unlike previous work in this area~\infercitep{An10dynamicinference}, we do not aim for soundness guarantees
in our generated types. 
A significant contribution of our work is a tool that Clojure programmers
can use to help learn about and specify their programs.
In that spirit, we strive to generate annotations meeting more qualitative criteria.
Each guideline by itself helps generate more useful annotations, and
they combine in interesting ways help to make up for shortcomings.
%in generated annotations.
%which we outline along with a commentary
%judging \figref{infer:fig:cljs} along these lines.

\paragraph{Choose recognizable names}
%Typed Clojure and clojure.spec annotations are abundant
%with useful names for types.
Assigning a good name for a type increases
readability by succinctly conveying its purpose.
Along those lines, a good name for the AST representation
on lines \ref{infer:listing:cljs:Op}-\ref{infer:listing:cljs:Op-End}
might be \clj{AST} or \clj{Expr}.
However, these kinds of names can be very misleading when incorrect, so
instead of guessing them,
our tool takes a more consistent approach and generates \emph{easily recognizable}
names based on the type the name points to.
Then, those with a passing familiarity with the data flowing through the program
can quickly identify and rename them.
For example,
\begin{itemize}
  \item
    \clj{Op} (lines \ref{infer:listing:cljs:Op}-\ref{infer:listing:cljs:Op-End})
    is chosen because \clj{:op} is
    clearly the dispatch key (the \clj{:op} entry is also helpfully placed
    as the first entry in each case to aid discoverability),
  \item
    \clj{ColumnLineContextMap} (lines \ref{infer:listing:cljs:ColumnLineContextMap}-\ref{infer:listing:cljs:ColumnLineContextMapEnd})
    enumerates the keys of the map type it points to,
  \item
    \clj{NameShadowMap} and \clj{FnScopeFnSelfNameNsMap} (%referenced on
    line
    \ref{infer:listing:cljs:Op:op:binding:NameShadowMap}% and \ref{infer:listing:cljs:Op:op:binding:FnScopeFnSelfNameNsMap}
    )
    similarly, and
  \item
    \clj{HMap49305} (lines \ref{infer:listing:cljs:HMap49305}-\ref{infer:listing:cljs:HMap49305End})
    shows how our tool fails to give names to certain combinations
    of types (we now discuss the severity of this particular situation).
\end{itemize}

A failure of cljs.compiler's
generated types was \clj{HMap49305}.
It clearly fails to be a recognizable name.
However, all is not lost:
the compactness and recognizable names of other adjacent annotations
makes it plausible for a programmer with some
knowledge of the AST representation to 
recover.
In particular 13/14 cases in \clj{Op}
have entries from \clj{:env} to \clj{HMap49305}, 
(like lines \ref{infer:listing:cljs:Op:op:const:HMap49305} and \ref{infer:listing:cljs:Op:op:do:HMap49305}),
and the only exception (line \ref{infer:listing:cljs:Op:optional:init:Op})
maps to \clj{ColumnLineContextMap}. From this information the user can
decide to combine these aliases.



%Good names can sometimes be reconstructed from the program source,
%like function or parameter names, and other times 
%we can use the shape of a type to summarize it.

\paragraph{Favor compact annotations}
Literally translating runtime observations into
annotations without compacting them
leads to unmaintainable and impractical types resembling
TypeWiz's ``verbatim'' annotation for \clj{nodes}.
To avoid this, we
  use optional keys where possible, like line \ref{infer:listing:cljs:ColumnLineContextMap:optional},
  infer recursive types like \clj{Op}, and
  reuse type aliases in function annotations, like
    \clj{emit} and \clj{emit-dot} (lines \ref{infer:listing:cljs:emit}, \ref{infer:listing:cljs:emit-dot}).

One remarkable success in the generated types
was the automatic inference \clj{Op} (lines \ref{infer:listing:cljs:Op}-\ref{infer:listing:cljs:Op-End})
with 14 distinct cases, and other features described in \figref{infer:fig:cljs}.
Further investigation reveals that
the compiler actually features 36 distinct AST nodes---unsurprisingly, 39 assertions was not sufficient
test coverage to discover them all.
However, because of the recognizable name and organization of
\clj{Op}, it's clear where to add the missing nodes
if no further tests are available.

These processes of compacting annotations often makes them more general,
which leads into our next goal.

%Idiomatic Clojure code rarely mixes certain types in the same position,
%unless the program is polymorphic. Using this knowledge---which we observed
%by the annotations and specs assigned to idiomatic Clojure 
%code---we can rule out certain combinations of types to compact our
%resulting output, without losing information that would help us
%type check our programs.

\paragraph{Don't overspecify types}
Poor test coverage can easily skew the results of dynamic analysis tools,
so we choose to err on the side of generalizing types
where possible.
Our opening example \clj{nodes}
is a good example of this---our inferred type
is recursive, despite \clj{nodes} only being tested with a tree of height 2.
This has several benefits.
\begin{itemize}
  \item We avoid exhausting the pool of easily recognizable names
    by generalizing types to communicate the general role
    of an argument or return position.
    For example, \clj{emit-dot} (line \ref{infer:listing:cljs:emit-dot})
    is annotated to take \clj{Op}, but in reality accepts only a subset
    of \clj{Op}.
    Programmers can combine the recognizability of \clj{Op} with the
    suggestive name of \clj{emit-dot} (the dot operator in Clojure handles host interoperability) to decide whether, for instance,
    to split \clj{Op} into smaller type aliases
    or add type casts in the definition of \clj{emit-dot} to please 
    the type checker
    (some libraries require more casts than others to type check, as discussed in \secref{infer:sec:experiment2}).
  \item Generated Clojure spec annotations (an extension discussed in \secref{infer:sec:spec-extension})
        are more likely to accept valid input with specs enabled, even with incomplete unit tests
        (we enable generated specs on several libraries in \secref{infer:sec:experiment3}).
  \item Our approach becomes more amenable to extensions improving the running time
        of runtime observation without significantly deteriorating annotation quality,
        like lazy tracking (\secref{infer:sec:lazy-tracking}).
\end{itemize}

Several instances of overspecification are evident,
such as the \clj{:statements} entry of a \clj{:do} AST node being inferred as an always-empty vector
(line \ref{infer:listing:cljs:Op:op:do:statements}).
In some ways, this is useful information, showing that
test coverage for \clj{:do} nodes could be improved.
To fix the annotation, we could rerun the tool with better tests.
If no such test exists, we would have to fall back
to reverse-engineering code to identify the correct
type of \clj{:statements}, which is \clj{(Vec Op)}.

Finally, 19 functions in cljs.compiler are annotated to 
take or return \clj{Op} (like lines \ref{infer:listing:cljs:emit}, \ref{infer:listing:cljs:emit-dot}).
This kind of alias reuse enables annotations
to be relatively compact (only 16 type aliases are used by the
49 functions that were exercised).

%
%We rate the quality of generated annotations
%on several axes.
%
%\paragraph{Compactness} Type annotations should be succinct,
%        but without sacrificing too much accuracy.
%        Are our type aliases intelligently combined
%        with good choices for optional keys?
%
%  \paragraph{Accuracy} Would executing a program with these
%      type annotations cause an error?
%      Have we too eagerly erased information in favor
%      of compactness?
%
%  \paragraph{Organization} Have we chosen good recursive types?
%      Do they have good names?
%
%
%\figref{infer:fig:gentype} shows our results.
%Our first program is an implementation of a
%1971 Star Trek game.
%It comes with minimal tests, so to complete this experiment,
%we instead played the game for 30 seconds.

%\begin{figure*}
%  \footnotesize
%\begin{tabular}
%{|         l   || l   | l  | l   || l  | l | l | l | l | l | l | l | l | l | l | l | l | l |}
%  Lib           & LOC  & GT  & LA & MD      & C  & I & P & L & S & O & U & N & V & R & K & F & H \\ 
%  \hline
%  \hline
%  sc            & 166  & 133 & 3  & 70/41   & 5  & 0 & 0 & 2 & 13& 1 & 5 & 1 & 1 & 2 & 0 & 0 & 0 \\
%  mc            & 923  & 395 & 147& 124/120 & 23 & 1 & 11& 19& 2 & 5 & 0 & 9 & 3 & 2 & 4 & 1 & 3 \\
%  fs            & 588  & 157 & 1  & 119/86  & 50 & 0 & 0 & 2 & 3 & 4 & 4 & 11& 2 & 9 & 0 & 0 & 0 \\
%  dj            & 528  & 168 & 9  & 94/125 \\
%  mo            & 530  & 49  & 1  & 46/26%\\
% %data.xml      &      & \\
% % cc            & 1776 & 448 & 4  & N/A 
%  %\\
%\end{tabular}
%  \caption{\emph{The number of type annotations generated for each program}:
%  Lib = Abbreviated library names in the order we introduce them on page \pageref{infer:chap:evaluation},
%  LOC = Number of lines of code we generate types for,
%  GT = Total number of lines of generated types after running our tool,
%  LA = The number of local annotations generated by our tools.
%  \emph{Number of manual changes needed to type check, and why they were needed}:
%  MD = Lines added/removed diff from git comparing initial generated types to
%       the manual amendments needed to
%       type check with Typed Clojure (unless it was too difficult to port),
%  C = Casts,
%  I = Instantiation,
%  P = Polymorphic annotation,
%  L = Local annotation,
%  S = Work around type system Shortcoming,
%  O = Overprecise argument type,
%  U = Uncalled function due to bad test coverage,
%  N = Add No-check annotation to skip checking function,
%  V = Add Variable arity argument type,
%  R = Overprecise return type,
%  K = Add Keyword argument types,
%  F = Added filter annotation,
%  H = Erase/upcast HVec annotation.
%  }
%\end{figure*}

\Dsection{Experiment 2: Changes needed to type check}
\label{infer:sec:experiment2}
% TODO examples for all kinds of things
% TODO bucket how many changes are needed for each kind of thing
%      - eg. varargs, polymorphism
% TODO how many lines of code were skipped

We used our workflow to port the following open source Clojure programs to Typed Clojure.

\paragraph{startrek-clojure}
A reimplementation of a Star Trek text adventure game,
created as a way to learn Clojure.

\paragraph{math.combinatorics}
The core library for common combinatorial functions
on collections,
with implementations based on Knuth's Art of Computer
Programming, Volume 4.

\paragraph{fs}
A Clojure wrapper library over common file-system operations.

\paragraph{data.json}
A library for working with JSON.

%\paragraph{data.xml} A library for manipulating and outputting XML in Clojure.

\paragraph{mini.occ}
A model of occurrence typing by an author of the
current paper. It utilizes three mutually recursive
ad-hoc structures to represent expressions, types,
and propositions.

In this experiment, we first generated types with our algorithm
by running the tests, then amended the program so that it
type checks.
\figref{infer:fig:gentype} summarizes our results.
After the lines of code we generate types for, the next two columns show how many lines of
types were generated and the lines manually changed, respectively.
The latter is a git line diff between commits of the initial
generated types and the final manually amended annotations.
While an objectively fair measurement,
it is not a good indication of the effort needed to port annotations
(a 1 character changes on a line is represented by 1 line addition and 1 line deletion)
The rest of the table enumerates the different kinds of changes needed 
and their frequency.

\begin{figure*}
  \begin{tabular}{|r||p{1cm}|p{1cm}|p{1cm}||p{1cm}|p{1cm}|p{1cm}|p{1cm}|p{1cm}|p{1cm}|p{1cm}|p{1cm}|p{1cm}|p{1cm}|p{1cm}|p{1cm}|p{1cm}|p{1cm}|p{1cm}|}
  Library       & Lines of code
                & Lines of Generated Global / Local Types
                & Lines manually added/removed
                & Casts / Instantiations
                & Polymorphic annotation
                & Local annotation
                & Type System Workaround/no-check
                & Overprecise argument/return type
                & Uncalled function (bad test coverage)
                & Variable-arity/keyword arg type
                & Add occurrence typing annotation
                & Erase or upcast HVec annotation
                & Add missing case in defalias
                \\ 
  \hline
  \hline
  startrek   & 166  & 133/3   & 70/41    & 5  / 0 & 0 & 2 & 13/1 & 1 /2 & 5 &  1 /  0 & 0 & 0 & 0\\
  math.comb  & 923  & 395/147 & 124/120  & 23 / 1 & 11& 19& 2 /9 & 5 /2 & 0 &  3 /  4 & 1 & 3 & 0\\
  fs         & 588  & 157/1   & 119/86   & 50 / 0 & 0 & 2 & 3 /11& 4 /9 & 4 &  2 /  0 & 0 & 0 & 0\\
  data.json  & 528  & 168/9   & 94/125   & 6  / 0 & 0 & 2 & 4 /5 & 11/7 & 5 &  0 /  20& 0 & 0 & 0\\
  mini.occ   & 530  & 49/1    & 46/26    & 7  / 0 & 0 & 2 & 5 /2 & 4 /2 & 6 &  0 /  0 & 0 & 1 & 5\\
 % cc            & 1776 & 448 & 4  & N/A 
  %\\
\end{tabular}
  \caption{Lines of generated annotations, git line diff for total manual changes to type check the program,
  and the kinds of manual changes.
  }
  \label{infer:fig:gentype}
\end{figure*}

\paragraph{Uncalled functions}
A function without tests receives a broad type annotation that
must be amended.
%
For example, the startrek-clojure game has several exit
conditions, one of which is running out of time.
Since the tests do not specifically call this function,
nor play the game long enough to invoke this condition,
no useful type is inferred.

\begin{lstlisting}[language=Clojure]
(ann game-over-out-of-time AnyFunction)
\end{lstlisting}

In this case, minimal effort is needed to amend this
type signature: the appropriate type alias
already exists:

\begin{lstlisting}[language=Clojure]
(defalias CurrentKlingonsCurrentSectorEnterpriseMap
  (HMap :mandatory
    {:current-klingons (Vec EnergySectorMap),
     :current-sector (Vec Int), ...}
    :optional {:lrs-history (Vec Str)}))
\end{lstlisting}
%\begin{lstlisting}[language=Clojure]
%(defalias CurrentKlingonsCurrentSectorEnterpriseMap
%  (HMap :mandatory
%    {:current-klingons (Vec EnergySectorMap),
%     :current-sector (Vec Int), 
%     :enterprise EnergyIsDockedQuadrantMap,
%     :quads (Vec BasesKlingonsQuadrantMap), 
%     :stardate CurrentEndStartMap,
%     :starting-klingons Int}
%    :optional {:lrs-history (Vec Str)}))
%\end{lstlisting}

So we amend the signature as

\begin{lstlisting}[language=Clojure]
(ann game-over-out-of-time
  [(Atom1 CurrentKlingonsCurrentSectorEnterpriseMap) 
   -> Boolean])
\end{lstlisting}


\paragraph{Over-precision}
Function types are often too restrictive due to
insufficient unit tests.

There are several instances of this in math.combinatorics.
The \clj{all-different?} function
takes a collection and returns true only if the collection
contains distinct elements.
As evidenced in the generated type, the tests exercise
this functions with collections of integers, atoms,
keywords, and characters.

\begin{lstlisting}[language=Clojure]
(ann all-different?
  [(Coll (U Int (Atom1 Int) ':a ':b Character)) 
   -> Boolean])
\end{lstlisting}

In our experience, the union is very rarely a good candidate
for a Typed Clojure type signature, so a useful heuristic to improve
the generated types would be to upcast such unions to a more permissive
type, like \clj{Any}.
When we performed that case study, we did not yet add that heuristic
to our tool,
so in this case, we manually amend the signature as

\begin{lstlisting}[language=Clojure]
(ann all-different? [(Coll Any) -> Boolean])
\end{lstlisting}

Another example of overprecision is the generated type
of \clj{initial-perm-numbers} a helper function
taking a \emph{frequency map}---a hash map from values
to the number of times they occur---which is the shape
of the return value of the core \clj{frequencies}
function.

The generated type shows only a frequency map where
the values are integers are exercised.
%
\begin{lstlisting}[language=Clojure]
(ann initial-perm-numbers
  [(Map Int Int) -> (Coll Int)])
\end{lstlisting}
%
A more appropriate type instead takes \clj{(Map Any Int)}.
%
%\begin{lstlisting}[language=Clojure]
%(ann initial-perm-numbers
%  [(Map Any Int) -> (Coll Int)])
%\end{lstlisting}
%
In many examples of overprecision, while the generated
type might not be immediately useful to check programs,
they serve as valuable starting points and also provide
an interesting summary of test coverage.

\paragraph{Missing polymorphism}

We do not attempt to infer polymorphic function types, 
so these amendments are expected. However, it is useful
to compare the optimal types with our generated ones.

For example, the \clj{remove-nth} function in \clj{math.combinatorics}
returns a functional delete operation on its argument.
Here we can see the tests only exercise this function with
collections of integers.

\begin{lstlisting}[language=Clojure]
(ann remove-nth [(Coll Int) Int -> (Vec Int)])
\end{lstlisting}

However, the overall shape of the function is intact,
and the manually amended type only requires a few 
keystrokes.

\begin{lstlisting}[language=Clojure]
(ann remove-nth
  (All [a] [(Coll a) Int -> (Vec a)]))
\end{lstlisting}

Similarly, \clj{iter-perm} could be polymorphic, 
but its type is generated as

\begin{lstlisting}[language=Clojure]
(ann iter-perm [(Vec Int) -> (U nil (Vec Int))])
\end{lstlisting}

We decided this function actually works over any number,
and bounded polymorphism was more appropriate, encoding
the fact that the elements of the output collection
are from the input collection.

\begin{lstlisting}[language=Clojure]
(ann iter-perm
  (All [a]
    [(Vec (I a Num)) -> (U nil (Vec (I a Num)))]))
\end{lstlisting}
%
%\paragraph{Missing return}
%Sometimes a function never returns, because of infinite loops
%or exceptions.

\paragraph{Missing argument counts}
Often, variable argument functions are given very precise types.
Our algorithm does not apply any heuristics to approximate
variable arguments --- instead we emit types that reflect
only the arities that were called during the unit tests.

The \clj{math.combinatorics} experiment contains
a good example of this phemonenon in the type inferred
for the \clj{plus} helper function.
From the generated type, we can see the tests exercise this function with 2, 6,
and 7 arguments.

\begin{lstlisting}[language=Clojure]
(ann plus (IFn [Int Int Int Int Int Int Int -> Int]
               [Int Int Int Int Int Int -> Int]
               [Int Int -> Int]))
\end{lstlisting}

Instead, \clj{plus} is actually variadic and works over any number of arguments.
It is better annotated as the following, which is easy to guess based on
both the annotated type and manually viewing the function implementation.

\begin{lstlisting}[language=Clojure]
(ann plus [Int * -> Int])
\end{lstlisting}

A similar issue occurs with \clj{mult}.

\begin{lstlisting}[language=Clojure]
(ann mult [Int Int -> Int]) ;; generated
(ann mult [Int * -> Int])   ;; amended
\end{lstlisting}

A similar issue is inferring keyword arguments. Clojure implements
keyword arguments with normal variadic arguments. Notice
the generated type for \clj{lex-partitions-H},
which takes a fixed argument, followed by some optional integer keyword
arguments. 

\begin{lstlisting}[language=Clojure]
(ann lex-partitions-H
  (IFn [Int -> (Coll (Coll (Vec Int)))]
       [Int ':min Int ':max Int 
        -> (Coll (Coll (Coll Int)))]))
\end{lstlisting}

While the arity of the generated type is too specific,
we can conceivably use the type to help us write a better one.

\begin{lstlisting}[language=Clojure]
(ann lex-partitions-H
  [Int & :optional {:min Int :max Int}
   -> (Coll (Coll (Coll Int)))])
\end{lstlisting}

\paragraph{Weaknesses in Typed Clojure}

We encountered several known weaknesses in Typed Clojure's type system
that we worked around.
%
The most invasive change needed was in startrek-clojure, which
strongly updated the global mutable configuration map on initial
play. We instead initialized the map with a dummy
value when it is first created.

\paragraph{Missing \clj{defalias} cases}

With insufficient test coverage, our tool can miss cases in a recursively defined
type.
In particular, mini.occ features three recursive types---for the representation
of types \clj{T}, propositions \clj{P}, and expressions \clj{E}.
For \clj{T}, three cases were missing, along with having to upcast the \clj{:params}
entry from the singleton vector \clj{'[NameTypeMap]}.
Two cases were missing from \clj{E}.
The manual changes are highlighted (\clj{P} required no changes with five cases).

\begin{minipage}[t]{0.54\linewidth}
\begin{lstlisting}[language=Clojure]
(defalias T
  (U (*@\colorbox{pink}{'\{:T ':not, :type T\}}@*)
     (*@\colorbox{pink}{'\{:T ':refine, :name t/Sym, :prop P\}}@*)
     (*@\colorbox{pink}{'\{:T ':union, :types (t/Set T)\}}@*)
     '{:T ':false}
     '{:T ':fun,
       :params (*@\colorbox{pink}{(t/Vec}@*) NameTypeMap(*@\colorbox{pink}{)}@*),
       :return T}
     '{:T ':intersection, :types (Set T)}
     '{:T ':num}))
\end{lstlisting}
\end{minipage}
%
\begin{minipage}[t]{0.4\linewidth}
\begin{lstlisting}[language=Clojure]
(defalias E
  (U (*@\colorbox{pink}{'\{:E ':add1\}}@*)
     (*@\colorbox{pink}{'\{:E ':n?\}}@*)
     '{:E ':app, :args (Vec E),
       :fun E}
     '{:E ':false}
     '{:E ':if, :else E,
       :test E, :then E}
     '{:E ':lambda, :arg Sym,
       :arg-type T, :body E}
     '{:E ':var, :name Sym}))
\end{lstlisting}
\end{minipage}


%cljs.compiler uses many polymorphic idioms that Typed Clojure is
%poor at checking, so we deemed it too difficult to attempt to
%type check. In particular, there are many of usages of the
%core functions
%\clj{get-in} and \clj{update-in} (functions that deeply lookup
%and manipulate maps) which are not even assigned types
%in Typed Clojure.
%Many function definitions would need to be ignored by the type
%checker to work around this.
%Furthermore, many manual instantiations
%would be needed to check transducers and polymorphic functions
%passed to other polymorphic functions.

%\begin{verbatim}
%  - get/get-in
%  - apply + kw args
%  - strong updates
%\end{verbatim}

%\paragraph{Possible errors in programs}


\Dsection{Experiment 3: Specs pass unit tests}
\label{infer:sec:experiment3}

Our final experiment uses our tool to
generate specs (\secref{infer:sec:spec-extension})
instead of types.
Specs are checked at runtime,
so to verify the utility of generated specs,
we enable spec checking while
rerunning the unit tests that were used
in the process of creating them.

\begin{figure*}
\begin{tabular}
{|         l   || l   | l || l  | l  | l || l |}
  Library       & LOC  &  Lines of specs  & Recursive & Instance & Het. Map & Passed Tests?\\ 
  \hline
  \hline
  startrek      & 166  &  25  & 0  & 10   & 0  & Yes\\
  math.comb     & 923  &  601 & 0  & 320  & 0  & Yes\\
  fs            & 588  &  543 & 0  & 215  & 0  & Yes\\
  data.json     & 528  &  401 & 0  & 174  & 0  & No (1/79 failed)\\ % pprinting related test
  mini.occ      & 530  &  131 & 3  & 25   & 15 & Yes\\
 %data.xml      &      & \\
 % cc            & 1776 & 448 & 4  & N/A 
  %\\
\end{tabular}
  \caption{Summary of the quantity and kinds of generated specs and whether they passed
  unit tests when enabled.
  The one failing test was related to pretty-printing JSON, and seems to be an artifact
  of our testing environment, as it still fails with all specs removed.
  }
\label{infer:fig:genspec}
\end{figure*}


At first this might seem like a trivial property, but it serves as
a valuable test of our inference algorithm.
The aggressive merging strategies to minimize aliases and
maximize recognizability, while unsound transformations,
are based on hypotheses about Clojure idioms and how
Clojure programs are constructed.
If, hypothetically, we generated singleton specs for numbers
like we do for keywords and did not eventually upcast
them to \clj{number?}, the specs might be too strict
to pass its unit tests.
Some function specs also perform generative testing based on
the argument and return types provided.
If we collapse a spec too much and include it in such
a spec, it might feed a function invalid input.

Thankfully, we avoid such pitfalls, and so
our generated specs pass their tests for the benchmarks
we tried.
\figref{infer:fig:genspec} shows
our preliminary results. All inferred specs pass the unit
tests when enforced, which tells us they are at least well formed.
We had some seemingly unrelated difficulty with a test in data.json which we explain
in the caption.
Since hundreds of invariants are checked---mostly ``instance'' checks that a value is of a particular class or interface---we can also be more confident
that the specs are useful.


%\Dsubsection{Experiment 3: Generating generative tests}

% We should generate the card playing specs in this guide:
% http://clojure.org/guides/spec

% # How evaluate
% ## qualitative
% Does it make sense??
% 
% 1. Don't run, gen type, manual inspection
%   - done on something small but real
%   - star trek game?
% 
% - Try different eval methods on different programs
%   - try different projects on different methods
%
% 2. Generate types, try type checking programs
%   - record what changes needed to get it to
%     type check 
%   - (on a different program than 1.) 
% 
% 3. Generate spec, insert the spec, run the test
%    with the spec on, also generate tests
%   - does spec ignore the input??
%     or just generate tests
%   - best situation:
%     - spec all passes
%     - then types check with minimal changes
%   - Q: can we use spec's tests to improve
%        types, iteratively?
%        (could throw away exceptions, throw
%         away bad input etc., different options
%         here)
% (optional)
% 4. Generate types, use gradual typing

\Dchapter{Extensions}
\label{infer:sec:extensions}

Two optimizations are crucial for practical implementations of the collection phase.
First, space-efficient tracking efficiently handles a common case with higher-order
functions where the same function is tracked at multiple paths.
Second, instead of tracking a potentially large value by eagerly traversing it, lazy tracking
offers a pay-as-you-go model by wrapping a value and only tracking subparts as they are accessed.
Both were necessary to collect samples from the compiler implementation we instrumented for
Experiment 1 (\secref{infer:sec:experiment1})
because it used many higher-order functions and its AST representation can be quite large
which made it intractible to eagerly traverse each time it got passed to one of dozens of functions.

%\Dsection{Polymorphic types}
%
%\begin{figure}
%  \ifdefined\PAPER
%  \footnotesize
%  \fi
%\begin{mathpar}
%  \begin{altgrammar}
%   \e{} &::=& ...
%       \alt \trackpolyE{\e{}}{\inferpath{}}{\e{}}
%       \alt \genE{}
%       &\mbox{Expressions}
%  \end{altgrammar}
%
%  \infer [B-Gen]
%  { \num{} \text{ is fresh}
%  }
%  { \opsemtrack{\openv{}}{\genE{}}{\num{}}{\emptyres{}}}
%
%  \infer [B-TrackPoly]
%  { \bigstepgen{\openv{}}{\num0}{\e{}}{\val{}}{\res{1}}{\num1} 
%    \\\\
%    \bigstepgen{\openv{}}{\num1}{\ep{}}{\num{}}{\res{2}}{\num2}
%    \\\\
%    \trackpolymeta{}(\val{}, \inferpath{}, {\num{}}) = \vp{}\ ; \res{3} }
%  { \bigstepgen{\openv{}}{\num0}{\trackpolyE{\e{}}{\inferpath{}}{\ep{}}}{\vp{}}{\bigunionres{\ova{\res{}}}}{\num2} }
%
%  \begin{array}{lllll}
%    \trackpolymeta{}(\val{}, \inferpath{}, \num{}) = \val{}\ ;\ \res{}\\\\
%
%    \trackpolymeta{}(\num{}, \inferpath{}, \num{}')
%    &=&
%    n\ ; \{\inferpath{} : \IntT{}\}
%    \\
%    \trackpolymeta{}([\lambda \xvar{}. \e{}, \openv{}], \inferpath{}, \num{})
%    &=&
%    [
%    \lambda \yvar{}.
%    \inferletliteral (\xvar{n} \genE{})
%    \\&&
%    FIXME
%      %\trackpolyE{((\lambda \xvar{}. \e{}) \trackpolyE{\yvar{}}{\appendone{\inferpath{}}{\dompe{}}}{\xvar{n}})}
%      %       {\appendone{\inferpath{}}{\rngpe{}}}{\xvar{n}}
%      %   , \openv{}]
%      %   \ ; \{\inferpath{} : [\UnknownT{} \rightarrow \UnknownT{}] \}
%         \\
%    &&
%    \text{where}\ \yvar{} \text{ is fresh}
%    \\
%    \trackpolymeta{}(\{\ova{\val1\ \val2}\}, \inferpath{}, \num{})
%    &=&
%    \{\ova{\val1\ \val2{}'}\}
%    \ ;\ \ova{\sqcup\ \res{}}
%      \sqcup
%    \{\inferpath{} : \{\ova{\val1\ \UnknownT{}}\} \}
%    \\
%    &&
%    \text{where}\ \ova{\trackpolymeta{}(\val2, \appendone{\inferpath{}}{\inferkeype{\ova{\val1}}{\val1}}) = \val2{}'\ ;\ \res{}}
%  \end{array}
%\end{mathpar}
%  \caption{Polymorphic type tracking extensions }
%\end{figure}

\Dsection{Space-efficient tracking}
\label{infer:sec:space-efficient-tracking}

To reduce the overhead of runtime tracking, we can borrow
the concept of ``space-efficient'' contract checking from
the gradual typing literature~\infercitep{Herman:2010}.
%
Instead of tracking just one path at once, a space-efficient
implementation of track threads through a set of paths.
When a tracked value flows into another tracked position,
we extract the unwrapped value, and then our new tracked value
tracks the paths that is the set of the old paths with the new path.

To model this, we introduce a new kind of value \ProxyV{\val{}}{\closure{\e{}}{\openv{}}}{\ova{\inferpath{}}}
that tracks old value \val{} as new value \closure{\e{}}{\openv{}} with the paths \ova{\inferpath{}}.
Proxy expressions are introduced when tracking functions, where instead of just returning
a new wrapped function, we return a proxy.
We can think of function proxies as a normal function with some extra metadata, so we
can reuse the existing semantics for function application---in fact we
can support space-efficient function tracking just by extending \trackEOp{}.

We present the extension in \figref{fig:infer:proxyext}.
The first two \trackEOp{} rules simply make inference
results for each of the paths.
The next rule says that a bare closure
reduces to a proxy that tracks the domain and range
of the closure with respect to the list of paths.
Attached to the proxy is everything needed to extend
it with more paths, which is the role of the
final rule. It extracts the original closure from the
proxy and creates a new proxy with updated paths
via the previous rule.

\begin{figure}
  %\ifdefined\PAPER
  %\footnotesize
  %\fi
\begin{mathpar}
  \begin{altgrammar}
    \val{} &::=& ... \alt \ProxyVdiff{\closure{\uabs{\x{}}{\e{}}}{\openv{}}}{\closure{\uabs{\x{}}{\e{}}}{\openv{}}}{\ova{\inferpath{}}}
       &\mbox{Values}
  \end{altgrammar}

  \arraycolsep=1.4pt
  \begin{array}{lllll}
    \trackmetaalign{\num{}}{\ovadiff{\inferpath{}}}{\num{}}{\proxyextdiff{\bigunionres{\ovadiff{\proxyextsame{\singletonres{\inferpath{}}{\IntT{}}}}}}}\\
    \trackmetaalign{\kw{}}{\ovadiff{\inferpath{}}}{\kw{}}
                   {\proxyextdiff{\bigunionres{\ovadiff{\proxyextsame{\singletonres{\inferpath{}}{\Keyword{}}}}}}}\\
    \trackmetaalign{\closure{\uabs{\x{}}{\e{}}}{\openv{}}}
                   {\ovadiff{\inferpath{}}}
                   {\ProxyVdiff{\closure{\uabs{\x{}}{\e{}}}{\openv{}}}
                               {\closure{\ep{}}{\openv{}}}
                               {\ova{\inferpath{}}}}
                   {\emptyres{}}
         \\
    &&
    \begin{array}{@{}llll}
      \text{where } \yvar{} \text{ is fresh},\\
                    \begin{array}{lllll}
                        \ep{} =
                          \uabs{\y{}}{\trackE{&\appexp{(\uabs{\x{}}{\e{}})}{\trackE{\yvar{}}{\ovadiff{\appendone{\inferpath{}}{\dompe{}}}}}}
                                             {\\&\ovadiff{\appendone{\inferpath{}}{\rngpe{}}}}}
                     \end{array}
    \end{array}
                
    \\
    \trackmetaalignsplice{\ProxyV{\closure{\uabs{\x{}}{\e{}}}{\openv{}}}{\closure{\ep{}}{\openvp{}}}{\ova{\inferpathp{}}}}{\ova{\inferpath{}}}
                         {\trackmetalhs{\closure{\uabs{\x{}}{\e{}}}{\openv{}}}{\ova{\inferpath{}} \cup \ova{\inferpathp{}}}}
  \end{array}
\end{mathpar}
  \caption{Space-efficient tracking extensions (\textcolor{red}{changes})}
  \label{fig:infer:proxyext}
\end{figure}

\Dsection{Lazy tracking}
\label{infer:sec:lazy-tracking}

Building further on the extension of space-efficient functions,
we apply a similar idea for tracking maps.
In practice, eagerly walking data structures to gather inference
results is expensive.
Instead,
waiting until a data structure is used and tracking its contents
lazily can help ease this tradeoff, with the side-effect that
fewer inference results are discovered.

\figref{fig:infer:lazy} extends our system with lazy maps.
We add a new kind of value 
\MProxyV{\curlymap{\ova{\kw{}\ \val{}}}}{\curlymap{\ova{\kwp{}\ \curlymap{\ova{\HMapreq{}\ \ova{\inferpath{}}}}}}}
that wraps a map \curlymap{\ova{\kw{}\ \val{}}} with tracking information.
Keyword entries \kwp{} are associated with pairs of type information \HMapreq{}
with paths \ova{\inferpath{}}.
The first \trackEOp{} rule demonstrates how to create
a lazily tracked map.
We calculate the possibly tagged entries in our type information in advance,
much like the equivalent rule in \figref{infer:fig:trackmeta}, and store
them for later use. Notice that non-keyword entries are not yet traversed,
and thus no inference results are derived from them.
The second \trackEOp{} rule adds new paths to track.

The subtleties of lazily tracking maps lie in the \constantopsemliteral{}
rules.
The \assocliteral{} and \dissocliteral{} rules ensure we no longer track overwritten entries.
Then, the \getliteral{} rules perform the tracking that was deferred
from the \trackEOp{} rule for maps in \figref{infer:fig:trackmeta}
(if the entry is still tracked).

In our experience, some combination of lazy and eager tracking of
maps strikes a good balance between performance overhead and
quantity of inference results.
Intuitively, if a function does not access parts of its argument,
they should not contribute to that function's type signature.
However, our inference algorithm combines information 
\emph{across} function signatures to deduce useful, recursive
type aliases.
Some eager tracking helps normalize the quality of function annotations
with respect to unit test coverage.

For example, say functions \textsf{f} and \textsf{g} operate on the same
types of (deeply nested) arguments, and \textsf{f} has complete test coverage (but does not
traverse all of its arguments), and \textsf{g} has incomplete test coverage
(but fully traverses its arguments).
Eagerly tracking \textsf{f} would give better inference results,
but lazily tracking \textsf{g} is more efficient.
Forcing several layers of tracking helps strike this balance, which
our implementation exposes as a parameter.

This can be achieved in our formal system
by adding fuel arguments to \trackEOp{}
that contain depth and breadth tracking limits, and
defer to lazy tracking when out of fuel.

\begin{figure}
  %\ifdefined\PAPER
  %\footnotesize
  %\fi
\begin{mathpar}
  \begin{altgrammar}
    \val{} &::=& ... \alt \MProxyVdiff{\curlymap{\ova{\kw{}\ \val{}}}}{\curlymap{\ova{\kw{}\ \curlymap{\ova{\HMapreq{}\ \ova{\inferpath{}}}}}}}
       &\mbox{Values}
  \end{altgrammar}

  \arraycolsep=1.4pt
  \begin{array}{lllll}
    \trackmetaalign{\curlymap{\ova{\kw{}\ \kwp{}}\ \ova{\kwpp{}\ \val{}}}}
                   {\ovadiff{\inferpath{}}}
                   {\MProxyVdiff{\curlymap{\ova{\kw{}\ \kwp{}}\ \ova{\kwpp{}\ \proxyextdiff{\val{}}}}}
                                {\curlymap{\ova{\kw{}\ \textsf{t}}\ \ova{\kwpp{}\ \textsf{t}}}}}
                   {\proxyextdiff{\emptyres{}}}
                   \\
                   &&
    \begin{array}{@{}llll}
      \text{where } \textsf{t} = \{\curlymap{\ova{\kw{}\ \kwp{}}\ \ova{\kwpp{}\ \UnknownT{}}}\ \ova{\inferpath{}}\}
    \end{array}
    \\
    \trackmetaalign{\MProxyVdiff{\curlymap{\ova{\kw{}\ \val{}}}}
                                {\curlymap{\ova{\kwp{}\ \{\HMapreq{}\ \ova{\inferpathp{}}\}}}}}
                   {\ovadiff{\inferpath{}}}
                   {\MProxyVdiff{\curlymap{\ova{\kw{}\ \val{}}}}
                                {\curlymap{\ova{\kwp{}\ \{\HMapreq{}\ (\ova{\inferpath{} \cup}\ \cup\ {\ova{\inferpathp{} \cup}})\}}}}}
                   {\emptyres{}}
  \end{array}

  \arraycolsep=1.4pt
  \begin{array}{lllr}
    \inferconstantopsemalignnospace
      {\assocliteral{}}
      {\MProxyVdiff{\curlymap{\ova{\kw{}\ \val{}}}}{\curlymap{\kwp{}\ \textsf{t}', \ova{\kwpp{}\ \textsf{t}}}}
       , \kwp{}
       , \vp{}}
      {\MProxyVdiff{\updatemap{\curlymap{\ova{\kw{}\ \val{}}}}{\kwp{}}{\vp{}}}
               {\curlymap{\ova{\kwpp{}\ \textsf{t}}}}}
      {\emptyres{}}\\
    \inferconstantopsemalignnospace
      {\assocliteral{}}
      {\MProxyVdiff{\curlymap{\ova{\kw{}\ \val{}}}}
                   {\curlymap{\ova{\kwpp{}\ \textsf{t}}}}
       , \kwp{}
       , \vp{}}
      {\MProxyVdiff{\updatemap{\curlymap{\ova{\kw{}\ \val{}}}}{\kwp{}}{\vp{}}}
                   {\curlymap{\ova{\kwpp{}\ \textsf{t}}}}}
      {\emptyres{}}\\
    \inferconstantopsemalignsplice
      {\getliteral{}}
      {\MProxyVdiff{\curlymap{\kw{}\ \val{}, \ova{\kwp{}\ \vp{}}}}
                   {\curlymap{\kw{}\ \textsf{t}, \ova{\kwpp{}\ \textsf{t}'}}}
       , \kw{}}
      {\proxyextdiff{\trackmetalhs{\proxyextsame{\val{}}}{\ova{\inferpath{}}}}}\\
      \begin{array}{lllll}
        \text{ where } \ova{\inferpath{}} = \big[\appendone{\inferpath{}}{\inferkeype{\HMapreq{}}{\kw{}}}
                                                \ |\
                                                (\HMapreq{}, \ova{\inferpath{}}) \in \textsf{t}, 
                                                \inferpath{} \in \ova{\inferpath{}}
                                           \big]
      \end{array}
      \\
    \inferconstantopsemalignsplice
      {\getliteral{}}
      {\MProxyVdiff{\curlymap{\kw{}\ \val{}, \ova{\kwp{}\ \vp{}}}}
                   {\curlymap{\ova{\kwpp{}\ \textsf{t}'}}}
       , \kw{}}
      {\val{}}\\
    \inferconstantopsemalign
      {\dissocliteral{}}
      {\MProxyVdiff{\curlymap{\kw{}\ \val{}, \ova{\kwp{}\ \vp{}}}}
                   {\curlymap{\kw{}\ \textsf{t}, \ova{\kwpp{}\ \textsf{t}'}}}
       , \kw{}}
      {\MProxyVdiff{\curlymap{\ova{\kwp{}\ \vp{}}}}
                   {\curlymap{\ova{\kwpp{}\ \textsf{t}'}}}}
      {\emptyres{}}
      \\
    \inferconstantopsemalign
      {\dissocliteral{}}
      {\MProxyVdiff{\curlymap{\kw{}\ \val{}, \ova{\kwp{}\ \vp{}}}}
                   {\curlymap{\ova{\kwpp{}\ \textsf{t}'}}}
       , \kw{}}
      {\MProxyVdiff{\curlymap{\ova{\kwp{}\ \vp{}}}}
                   {\curlymap{\ova{\kwpp{}\ \textsf{t}'}}}}
      {\emptyres{}}
  \end{array}
\end{mathpar}
  \caption{Lazy tracking extensions (\textcolor{red}{changes})}
  \label{fig:infer:lazy}
\end{figure}

\Dsection{Automatic contracts with clojure.spec}
\label{infer:sec:spec-extension}

While we originally designed our tool to generate Typed Clojure annotations,
it also supports generating ``specs'' for clojure.spec, Clojure's runtime
verification system.
There are key similarities between Typed Clojure and clojure.spec,
such as extensive support for potentially-tagged keyword maps,
however spec features a global registry of names
via \clj{s/def} and
an explicit way to declare unions of maps with a common
dispatch key in \clj{s/multi-spec}.
These require differences in both type and name generation.

The following generated specs 
correspond to the first \clj{Op} case
of \figref{infer:fig:cljs}
(lines \ref{infer:listing:cljs:Op:op:bindingStart}-\ref{infer:listing:cljs:Op:op:bindingEnd}).


\begin{minipage}{\linewidth}%added minipage for dissertation
\begin{cljlistingnumbered}
  (defmulti op-multi-spec :op) ;dispatch on :op key
  (defmethod op-multi-spec :binding ;match :binding
    [_] ;s/keys matches keyword maps
    (s/keys :req-un [::op ...] ;required keys
            :opt-un [::column ...])) ;optional keys
  (s/def ::op #{:js :let ...}) ;:op key maps to keywords
  (s/def ::column int?) ;:column key maps to ints
  ; register ::Op as union dispatching on :op entry
  (s/def ::Op (s/multi-spec op-multi-spec :op))
  ; emit's first argument :ast has spec ::Op
  (s/fdef emit :args (s/cat :ast ::Op) :ret nil?)
\end{cljlistingnumbered}
\end{minipage}


%However, there are several non-trivial details to manage.
% specs have fundamental differences.

%As they are checked at runtime, specs are naturally more expressive than types.
%However, our generated specs are roughly the same as types in their precision.

%\paragraph{Registered Specs}
%Central to spec's design is a global registry of named specs.
%The form \clj{(s/def k s)} registers spec \clj{s} under name \clj{k}.
%To avoid conflicts, \emph{namespaced keywords} are used. The \clj{::a} reader syntax
%auto-resolves the keyword under the current namespace, like \clj{:this-ns/a}.
%
%\paragraph{HMaps}
%Heterogeneous map keyword specs are called ``entity maps'' in spec, 
%and are defined as collections
%of registered specs with \clj{s/keys}, and has quite different
%syntax and semantics than many other runtime verification systems.

%The Typed Clojure type
%\begin{lstlisting}[language=Clojure]
%(defalias M (HMap :mandatory {:a Int, ::a Bool}
%                  :optional  {:b Int, ::b Bool}))
%\end{lstlisting}
%is equivalent to
%\begin{lstlisting}[language=Clojure]
%(s/def :int/a integer?)
%(s/def ::a boolean?)
%(s/def :int/b integer?)
%(s/def ::b boolean?)
%(s/def ::M (s/keys :req-un [:int/a], :req [::a],
%                   :opt-un [:int/b], :opt [::b]))
%\end{lstlisting}
%The \clj{:req-un} and \clj{:opt-un} options
%(\clj{un}[qualified] keyword \clj{req}[uired] and \clj{opt}[ional] entries)
%take vectors of registered specs and, for
%each spec \clj{:ns/name}, uses the registered spec \clj{:ns/name}
%to verify values at \clj{:name}.
%
%The \clj{:req} and \clj{:opt} options
%(qualified keyword \clj{req}[uired] and \clj{opt}[ional] entries)
%take vectors of registered specs and, for
%each spec \clj{:ns/name}, uses the registered spec \clj{:ns/name}
%to verify values at \clj{:ns/name}.
%
%Spec has several other subtleties we must account for, but omit
%for brevity.

%\paragraph{Tagged maps}
%
%\paragraph{Functions}

% infer-comparison == performance analysis vs Daikon
% this isn't in the PLDI submission so might not really fit yet
%\input{infer-comparison}
\Dchapter{Conclusion}

This paper shows how to
generate recursive heterogeneous type annotations for
untyped programs that use plain data.
We use a novel algorithm to ``squash'' the observed structure
of program values into named recursive types suitable for
optional type systems,
all without the assistance of record, structure, or class
definitions.
We test this approach on thousands of lines of Clojure code,
optimizing generated annotations
for programmer comprehensibility over soundness.

In our experience, our guidelines
to automatically name, group, and reuse
types yield insightful annotations
for those with some familiarity with
the original programs,
even if the initial annotations are imprecise, incomplete,
and always require some changes to type check.
Most importantly, many of these changes will involve simply rearranging or changing parts
of existing annotations, so programmers are no longer left alone
with the daunting task of reverse-engineering such programs
completely from scratch.

%Our generated annotations
%serve as a valuable starting point for annotating untyped programs
%who rely on plain data, especially

%We indend to convey to programmers
%to understand 
%
%and report on experience in using our tool to generate annotations for real-world
%Clojure programs, and enumerate the remaining changes needed to fully port
%them to Typed Clojure.
%
%Optional type systems 
%were created on the observation
%that even dynamic programming languages have a 
%
%We tackle the problem of 
%Our approach to automatically
%generate recursive types from unrolled examples,
%has proven
%even with the challenge of 
%
%% key insights
%% - recursive annotations for records and classes without declarations.
%% - how to recover the structure from class definitions
%%   without programmer intervention?
%% - generalized types 
%
%We have tested our approach on thousands of lines
%of Clojure code,
%Typed Clojure types for Clojure
%programs. Despite the fact that Clojure programs
%declare minimal structure, we successfully 


% Quals
%\part{Investigation of clojure.spec}
%\label{part:spec}
%
%\input{spec-intro}
%\input{spec-study}
%\input{spec-model}

\part{Typed Clojure Implementations}
\label{part:implementations}
\chapter{Background}

Clojure is a dialect of Lisp, and so supports metaprogramming
via macros.
This immediately poses an interesting problem for Clojure
type systems: how do we check a macro call?
Ideally, we don't want to require special typing rules for each
macro, since that imposes additional burden on the programmer
to define special rules for their own macros.
On the other hand, sometimes its helpful to write custom rules
for customized error messages, or a higher-level specification
for a macro's usage.

In this part we explore several solutions to this problem,
from the standard approach of expanding macros to
primitive forms before checking, to more involved solutions
that allow extensible typing rules for each macro.

Several constraints guide us through our designs.
There is a question of soundness: does what we actually
check match up with the code being evaluated?
There is a natural tension between soundness and user extensibility.
Allowing custom rules for macros gives a kind of flexibility
that makes it hard to relate type checking semantics with
the running code---which is the whole idea behind a soundness result.
On the other hand, expanding code before checking ensures
we check the actual code being run.
In all of these cases, wrappers that communicate information
to the type system are needed, but they interact
with evaluated code differently.

We also consider the experience of using these solutions.
Error messages can be unrelated to the source problem
if pre-expanding code, but we may miss actual errors
by using a poorly written typing rule.
We are interested in the difficultly of extending each system,
including any additional annotation burden,
additional knowledge needed to manage evaluation semantics
in typing rules, and additional type system knowledge required
to write typing rules. Finally, we also consider implications
to type checking performance and amenability to iterative development.

The following chapters present several designs of Typed Clojure,
their extensibility stories, and general implementation concerns for
Clojure type system designers.

%{
%\singlespacing
%\begin{verbatim}
%- Problem
%  - Clojure is a Lisp with macros
%  - don't want to write typing rules for each macro
%    - don't want to burden users
%    - so we expand them before checking
%    - but sometimes it's really helpful to write custom rules
%- Possible solutions
%  - pre expansion
%  - interleaved analysis and evaluation
%- Constraints
%  - Soundness?
%    - tension between soundness and user extensibility
%  - advantages of pre-expanding
%    - can't have "wrong" expansion in pre-expanded code
%      - we check the actual expansion that gets wr
%  - wrapper macros needed in all of these systems
%\end{verbatim}
%}

\chapter{Expand before checking}

Typed Clojure's initial design was inspired by Typed Racket,
which checks Racket code by first expanding until it consists
of only primitives, and then checking using fixed rules for each
primitive.
This chapter goes into this design in more detail, starting
with our choice of analyzer and then how to handle extensibility.

\section{Upfront Analysis with \texttt{tools.analyzer}}

Instead of using Clojure's compiler to analyze code,
we opted to use \texttt{tools.analyzer}, a standalone nano-pass
analyzer providing an idiomatic map-based AST format providing
passes for hygienic transformations and Java reflection resolution.

%{
%\singlespacing
%\begin{verbatim}
%- Pros to expanding up front:
%  - Separation of concerns (expander does expansion)
%- Cons to expanding up front:
%  - Lose contextual information from unexpanded macros while type checking.
%  - Requires wrapper macros which pollute runtime expansions and
%    often require copying implementation details (brittle).
%\end{verbatim}
%}

\figref{fig:analyzer:control-flow-pre-expand} demonstrates 
how Typed Clojure checks code using the pre-expansion approach.
To simplify presentation we assume \texttt{tools.analyzer}
uses only 2 passes. The first pass \texttt{analyze} creates
a bare AST with no platform specific information.
The second pass is composed of two tree traversals.
The first is a pre-traversal \texttt{pre-passes} which
is called before we visit the children of an AST node.
The second is a post-traversal \texttt{post-passes} which
is called after we visit the children of an AST node.

This arrangement is convenient as a type system implementer,
insofar as there is a clean separation of concerns: the analyzer
handles expansion and evaluation, while the type system
merely checks.
However, much contextual information is lost from the expansion
process that is needed for checking.
We now present how we surmount this challenge while still
preserving the pre-expanded checking model.

\begin{figure}
\singlespacing
$$
  \begin{array}{r||l|l|l|}
    \text{Time} & \text{\clj{(let [...]}} & \text{\clj{(cond ...}} & \text{\clj{(+ ...)))}}\\
    \hline
     0          & \text{\clj{analyze}}^{>}    &                             &                      \\
     1          &                             & \text{\clj{analyze}}^{>}    &                      \\
     2          &                             &                             & \text{\clj{analyze}}^{>} \\
     3          &                             &                             & \text{\clj{analyze}}^{<} \\
     4          &                             & \text{\clj{analyze}}^{<}    &                      \\
     5          & \text{\clj{analyze}}^{<}    &                             &                      \\
     6          & \text{\clj{pre-passes}}^{>} &                             &                      \\
     7          &                             & \text{\clj{pre-passes}}^{>} &                      \\
     8          &                             &                             & \text{\clj{pre-passes}}^{>} \\
     9          &                             &                             & \text{\clj{post-passes}}^{<} \\
     10         &                             & \text{\clj{post-passes}}^{<}&                      \\
     11         & \text{\clj{post-passes}}^{<}&                             &                      \\
     12         & \text{\clj{check}}^{>}      &                             &                      \\
     13         &                             & \text{\clj{check}}^{>}      &                      \\
     14         &                             &                             & \text{\clj{check}}^{>} \\
     15         &                             &                             & \text{\clj{check}}^{<} \\
     16         &                             & \text{\clj{check}}^{<}      &                      \\
     17         & \text{\clj{check}}^{<}      &                             &                      \\
  \end{array}
$$
%\begin{verbatim}
%time | (let [...]  | (cond ...   | (+ ...)))
% |   | ---------------------------------------
% v   | analyze    >|             |
%     |             | analyze >   |
%     |             |             | analyze    >
%     |             |             |<analyze
%     |             |<analyze     |
%     |<analyze     |             |
%     | pre-passes >|             |
%     |             | pre-passes >|
%     |             |             | pre-passes >
%     |             |             |<post-passes
%     |             |<post-passes |
%     |<post-passes |             |
%     | check>      |             |
%     |             | check>      |
%     |             |             | check>
%     |             |             |<check
%     |             |<check       |
%     |<check       |             |
%\end{verbatim}
  \caption{Illustrative control flow when
  using \texttt{tools.analyzer} to expand code via \clj{analyze} and several passes,
  followed by Typed Clojure checking.
  The partial expression \clj{(let [...] (cond ... (+ ...)))}
  was chosen since it has at least 3 levels of nesting.
  Many more levels will be revealed after expansion by \clj{analyze}, which we do not picture.
  ${}^>$ and ${}^<$ indicate work done to a node before and after processing its children, respectively.
  }
  \label{fig:analyzer:control-flow-pre-expand}
\end{figure}

\section{Extensibility}

%{
%\singlespacing
%\begin{verbatim}
%- Problem
%  - need to communicate between type system and Clojure runtime
%- Constraints
%  - a "typed" program must evaluate unchanged via normal Clojure compilation
%    - extensions must be done via macros provided by Typed Clojure
%      - imported and used as normal by Clojure programmers
%    - in contrast to #lang system
%      - which always guarantees the type system is in charge of expanding
%      - (both approaches use macros for extension and to share information)
%- how to communicate to type system via expanded code?
%  - eg. tc-ignore, ann-form
%  - in Racket you would use syntax properties, or side effects
%  - Clojure has metadata, but not as robust as syntax properties
%    - how metadata is compiled is implementation dependent (I forgot how?)
%    - we decided to emit special `do` forms to communicate with type system
%      - (do :special-form ...)
%      - "variable protocol" in Advanced Macrology
%  - side effects
%    - Clojure's compilation strategy is straightforward
%      - files are just sequences of top-level forms
%      - evaluate each in turn
%      - side effects of expanding/evaluating a previous form
%        can be used to compile a subsequent form
%      - members of top-level `do` forms are also top-level forms, and thus
%        are evaluated in turn
%    - Typed Clojure collects global type annotations by evaluation side effects
%      - macroexpansion side effects not used in case AOT compiled
%- how to define custom rules?
%  - Approach 1: custom expansions for embedding typing rules in expansion
%  - Approach 2: "typing rules by analogy"
%    - lose ability to check actual expansion
%\end{verbatim}
%}

Now that we have outlined how we use \texttt{tools.analyzer} to pre-expand code before type checking,
we describe Typed Clojure's approach to sharing information between the programs it checks
and the type system.
We deviate significantly from Typed Racket's approach~\cite{Culpepper07advancedmacrology}
mostly because of differences in compilation models between Clojure and Racket.

One constraint we must consider in Typed Clojure is that a ``typed'' Clojure program must
evaluate unchanged under normal Clojure compilation. In Racket, we could instead specify
the language under which a module is compiled using the \texttt{\#lang} directive---this is Typed
Racket's approach. 
In Clojure, there is just one language and no built-in facilities to extend the compilation
process, so Typed Clojure provides a suite of macros for communicating with the type system that
users must explicitly load and use.

These macros come in several flavors:

\begin{itemize}
  \item syntax-based communication to type checker,
  \item side-effectful communication to type checker, and
  \item wrappers for existing untyped macros.
    %to avoid checking complex expansions
    %or provide .
\end{itemize}

We discuss each in the following sections.

\subsection{Syntax-based communication}

A simple macro provided by Typed Clojure that communicates to the checker
via syntax is \clj{tc-ignore}, which takes a number of forms, places
them in a \clj{do} form, and tells the checker to ignore the resulting
form and assign it type \clj{Any}.

\begin{figure*}
\begin{cljlisting}
(defmacro tc-ignore 
  "Ignore forms in body during type checking"
  [& body]
  `(do :clojure.core.typed.special-form/special-form
       :clojure.core.typed/tc-ignore
       ~@(or body [nil])))
\end{cljlisting}
  \caption{Public facing macro definition for \clj{tc-ignore}.}
  \label{fig:analyzer:tc-ignore}
\end{figure*}

\figref{fig:analyzer:tc-ignore} shows the implementation of the \clj{tc-ignore} macro.
It demonstrates the \clj{do}-special-form protocol:
if the first member of a \clj{do} is the keyword
\[
\clj{:clojure.core.typed.special-form/special-form},
\]
the following keyword names a special typing rule to use
to check the entire form.
A corresponding typing rule must then be registered with the type checker under this name,
like in \figref{fig:analyzer:tc-ignore-do-op}.

\begin{figure*}
\begin{cljlisting}
(defmethod internal-special-form :clojure.core.typed/tc-ignore
  [expr expected]
  (tc-ignore/check-tc-ignore check-expr expr expected))
\end{cljlisting}
  \caption{Registering a corresponding typing rule for \clj{tc-ignore} via the \clj{do}-special-form protocol.}
  \label{fig:analyzer:tc-ignore-do-op}
\end{figure*}

Clojure's compilation and runtime models make \clj{do} statements an excellent candidate for the basis of
an extensible syntax-based communication protocol.
First, it naturally inherits the top-level characteristics of \clj{do}, which is key to defining
wrapper macros that operate at the top-level.
A usage of \clj{tc-ignore} that relies on this is demonstrated in \figref{fig:analyzer:tc-ignore-usage}.
Second, it avoids the need to pre-expand its arguments to attach information, or
have special cases for particular arguments.
On the other hand, a communication protocol based on attaching metadata properties
would require pre-expanding arguments, since metadata is lost on macroexpansion,
and in some cases would not be possible, since many common Clojure forms do not support metadata
(such as keywords, numbers, and nil).
Third, the information can be compiled away using standard techniques,
since they are constant statements---extra information can be provided via a map of constant values
placed after the typing rule name, as in
the definition of \clj{ann-form} (\figref{fig:analyzer:ann-form-definition}).

\begin{figure*}
\begin{cljlisting}
(defmacro ann-form
  "Annotate a form with an expected type."
  [form ty]
  `(do :clojure.core.typed.special-form/special-form
       :clojure.core.typed/ann-form
       {:type '~ty}
       ~form))
\end{cljlisting}
  \caption{The definition of \clj{ann-form} shows how to communicate extra information to the type checker}
  \label{fig:analyzer:ann-form-definition}
\end{figure*}

While a strong choice, there are some downsides to basing our communication protocol on \clj{do}
statements.
There is no guarantee the information will be compiled away at runtime, and
thus may contribute to bloating the runtime.
On the other hand, \clj{tools.analyzer} must be carefully configured to not erase these constant
values before Typed Clojure can access them.

Alternative \clj{do}-based protocols could be similarly effective
such as attaching metadata directly to the symbol \clj{do} or list \clj{(do ...)}.
We felt embedding the information directly in programs had the best chance of forward-compatibility,
since the interaction between metadata and compilation is not well documented and
can be platform-dependent (in our experience ClojureScript has handled some cases differently,
like evaluating metadata instead of simply quoting it as in Clojure).

\begin{figure}
\begin{cljlisting}
(tc-ignore
  (defmacro reverse-app [a f] `(~f ~a))
  (reverse-app 1 inc)) ;=> 2
\end{cljlisting}
  \caption{Example top-level usage of \clj{tc-ignore}
           where the second form must expand after the first evaluates.
  It works because \clj{tc-ignore} wraps only with \clj{do}.}
  \label{fig:analyzer:tc-ignore-usage}
\end{figure}

\subsection{Side-effectful communication}
\label{analyzer:extensibility:side-effects}

Racket has a sophisticated system for managing compile-time side effects
to accompany its module system.
Clojure does not have a module system, and instead relies on conventions
and a simple compilation model to write effective programs.

The unit of compilation in Clojure is a top-level form. A top-level Clojure form
is guaranteed to have all previous top-level forms fully expanded
and evaluated before it is expanded and evaluated itself.
This blurs the lines between compile-time and runtime, compared to the
distinct phases of Racket compilation.

When checking a file with Typed Clojure, we have similar guarantees:
when checking a top-level form, we can depend on the fact that all
previous top-level forms have been expanded, evaluated, and checked,
and that the current form has been fully expanded.

Thus, we have a choice of (at least) three times to send side-effectful communication
to the type checker:
expansion-time, evaluation-time, and checking-time.
\figref{fig:analyzer:ann-definition} shows the most frequently used
side-effectful macro \clj{ann}, which registers the type of a var in the
global environment.
It expands to code that uses internal function \clj{ann*}, which does
the registering. This is a \emph{evaluation-time} side effect,
and we similarly perform most communication at this time.
We now elaborate on why this is a good choice.

A previous implementation of Typed Clojure (which was used by CircleCI
in \secref{sec:casestudy}) only collected top-level annotations
from \clj{ann} at checking-time. This forced Typed Clojure to recursively
check other files just to collection annotations.
We decided the natural behavior of rechecking a file would be to
recheck its dependencies so, among other benefits, top-level annotations
would be kept up-to-date.
Unfortunately, the checker was much slower at evaluating files
than the Clojure compiler, meaning iterative development was hampered.
To fix this, we made checking of transitive file dependencies optional, and
so dependencies containing top-level annotations would potentially only 
be evaluated by the Clojure compiler.
Evaluation-time was then the natural time to collect these annotations.

A side-effect of this design choice is that it is no longer a sound idea to
infer types for unannotated top-level bindings. In the aforementioned 
implementation, if the checker finds an unannotated top-level \clj{def}
like \clj{(def a 1)}, it will update the global environment with the 
inferred type of the right-hand-side.
Now that transitive dependencies are optionally checked, it is not guaranteed
the checker will infer these annotations, and so more top-level annotations
via \clj{ann} are needed to recover consistent checking behavior.
This unfortunately increases the annotation burden even more, however the rewards
are great.
We believe that Clojure programmers will enjoy the ability to rapidly recheck
small parts of their code base, just like they are used to in untyped Clojure.

Now, we discuss the merits of collection at evaluation-time over expansion-time.
We avoid side-effects at expansion-time because Clojure code can be
evaluated in two ways: from the original source code in on-the-fly compilation mode, and 
from precompiled JVM bytecode in ahead-of-time compilation mode.
In the latter, code is expanded ahead-of-time (potentially in a different environment)
and thus expansion-time side-effects are lost.
We applied the standard solution to this problem: remove the side-effect from
the macro itself and move it to the evaluation of the code it expands into.


%\begin{verbatim}
%- not forced to recursively check other files just to collect annotations
%  - problem identified with CircleCI
%  - simply need to evaluate a file normally
%  - in turn requires more annotations
%    - tradeof between annotation burden and performance
%- avoid relying on expansion-time side effects
%  - lost with AOT compilation
%- "staged at checking time": under Typed Racket AOT compilation, it stages global type annotations for eval time
%  - we don't have a similar mode
%    - compiling a Typed Clojure file does not require checking
%\end{verbatim}


\begin{figure*}
\begin{cljlisting}
(defmacro ann 
  "Register top-level var with type."
  [varsym typesyn]
  (let [qsym (qualify-in-current-ns varsym)
        opts (meta varsym)
        check? (not (:no-check opts))]
    `(tc-ignore (ann* '~qsym '~typesyn '~check? '~&form))))
(defn ann* 
  "Internal use only. Use ann."
  [qsym typesyn check? form]
  ; omitted - registers `qsym` at type `typesym`
  )
\end{cljlisting}
  \caption{Implementation of \clj{ann}, which expands to code that registers types at evaluation-time.}
  \label{fig:analyzer:ann-definition}
\end{figure*}

\subsection{Wrapper macros}

Several situations call for wrapper macros for existing untyped macros.
In practice, this often means the type system author provides an alternative
implementation for a macro, and the type system user
replaces any usages of the original macro in type-checked code with the alternative implementation.
Sometimes this choice is aesthetic, providing a prettier 
way to write annotations. For example, the \clj{fn} wrapper
enables writing annotations like
\clj{(fn [a :- Int] ...)}
instead of the more verbose
\clj{(ann-form (fn [a] ...) [Int -> Any])}.

The more pressing need for wrapper macros when checking pre-expanded
code is to manage complex expansions.
Some macro expansions are too complex for Typed Clojure to reason about,
so it becomes necessary to rewrite these expansions to be more palatable
for the checker.
For example, the \clj{for} macro is a lazy sequence builder using
a list-comprehension syntax---however it expands into local
loops using local mutable state, which are problematic to check.
The wrapper macro for \clj{for} expands (and thus evaluates) similarly, but inserts user-provided
type annotations strategically into the expansion so it more easily type checks.

The problem with this kind of wrapper macros is that large amounts
of implementation code must be copied to preserve the original semantics.
Instead of checking a higher-level specification of the macro's behavior,
we are tied closely to a particular implementation.
This has the advantage of checking the actual code that gets evaluated, but
unfortunately
requires the type system writer to closely follow the original implementations
(hampering both backwards- and forwards-compatibility with versions of the original macro).
Furthermore, users not only must use wrapper macros where necessary, but
also recognize when they are required---usually attempting to check a complex
expansion yields an incomprehensible error as Typed Clojure fails to check it.
It is rarely apparent that a wrapper macro is needed from such an error message.

\chapter{Interleaved expansion and checking}

The previous chapter outlined a design for Typed Clojure that fully expands code
before checking.
We identified several problems with the user experience of Typed Clojure's initial design,
including bad error messages, and excessive copying of macro implementations for wrapper
macros.
Additionally, we identified several issues with \texttt{tools.analyzer} that we have
not yet discussed.

First, \texttt{tools.analyzer}'s goals of being mostly platform-agnostic made analysis particularly 
slow, and so added an undesirable performance overhead to type checking.
In particular, a copy of the
global scope is maintained for every namespace. While it enables a convenient platform-agnostic API
for symbol resolution,
it comes at a performance cost since it must be updated (from scratch) frequently.
Furthermore, some macroexpansion side effects are not (yet) recognized by the analyzer
which means analysis sometimes deviates from Clojure compiler, an undesirable situation
since Typed Clojure intends to model how code runs \emph{outside} of type checking.
Unfortunately, fixing some of these differences would require even more frequent costly updates.

Second, it is impractical to recover contextual information lost via analysis.
This is both because \texttt{tools.analyzer} has no way of representing unanalyzed
code (so there is no choice but to expand immediately), and
because \texttt{tools.analyzer} uses at least 2 passes over the AST
(so there is no obvious place to recover contextual information since pre-traversal
passes run \emph{after} the entire program has been expanded).
For example, \figref{fig:analyzer:control-flow-pre-expand}
illustrates \texttt{tools.analyzer}'s control flow with just 2 traversals.
Say at time 1 we wished to take advantage of the unexpanded \clj{cond}
form with a special rule (before it expands and contextual information is lost).
In fact, \texttt{tools.analyzer} provides the extension point \clj{macroexpand-1}
for just this purpose, which allows the user to specify exactly how a form is expanded.
Unfortunately, time 0 introduced local bindings that are unhygienic, and the hygienic
transformation pass (required for checking because occurrence typing's propositions do not recognize variable shadowing)
happens at time 6 with \clj{pre-passes}.
So, there is no room for a checking rule for \clj{cond} until time 13, well
after the \clj{cond} is expanded away.

Fortunately, \texttt{tools.analyzer}'s design and implementation
is otherwise brilliant and innovative, and forms a great base to build a new Clojure analyzer better suited to help solve
many of the aforementioned analysis and checking problems---we did exactly that in \texttt{core.typed.analyzer}.

\section{Interleaved Analysis with \texttt{core.typed.analyzer}}

To replace \texttt{tools.analyzer}, we built \texttt{core.typed.analyzer}. In this section,
we describe how \texttt{core.typed.analyzer} works, and outline both the ideas we repurposed
from \texttt{tools.analyzer} and those specific to \texttt{core.typed.analyzer}.

\subsection{Overview}

The main feature of \texttt{core.typed.analyzer} is the ability to stop and resume
analysis at any point, while still supporting the essentials of a general-purpose Clojure analyzer.
Supporting this requires several key innovations and restrictions over \texttt{tools.analyzer}.
First, a new AST node type for partially expanded forms is needed to return a paused analysis.
Second, the analyzer must have the ability to incrementally perform a small amount of analysis
(on the order of expanding one macro) to provide fine-grained control over the AST.
Third, all AST traversals must be fused into one traversal to minimize
the bookkeeping needed to manage the AST.

To this end, \texttt{core.typed.analyzer} provides an API of 4 functions.
First, \clj{(unanalyzed form env)} creates an \clj{:unanalyzed} AST node
that pauses the analysis of \clj{form} in local environment \clj{env}.
Second, \clj{(analyze-outer ast)} analyzes the outermost form represented by \clj{ast}
further by roughly one macroexpansion if possible, otherwise it returns \clj{ast}.
Third, \clj{(run-pre-passes ast)} and \clj{(run-post-passes ast)}
decorate \clj{ast} with extra information, used before and after visiting its children,
respectively.

To sample how it feels to use this API to implement a type checker, we now
walk through checking \clj{(let [...] (cond ... (+ ...)))} in \figref{fig:analyzer:typed-analyzer-overview}.
To check the outermost \clj{let},
we use \clj{unanalyzed} to create an initial AST from a entire form at time 0.
Then at time 1, the checker calls \clj{analyze-outer} zero or more times, either 
until a special rule for partially expanded code is triggered
or to a fixed point.
Next at time 2 and 3 we decorate our AST node with \clj{run-pre-passes} (adding hygienic bindings)
before calling \clj{check}.
After checking its children during time 4-13, at time 14 and 15 we use \clj{run-post-passes}
to add the rest of the decorations (e.g., resolving interop reflection)
before any final checks from \clj{check}.
The interleaving of operations using \texttt{core.typed.analyzer} is clear to see when
compared to the same example using \texttt{tools.analyzer} 
(\figref{fig:analyzer:control-flow-pre-expand}).

Now with the interleaving analyzer, we can solve the problem we posed at the beginning of this chapter
of wanting a custom typing rule for \clj{cond}: we simply
limit the number of expansions done via \clj{analyze-outer} at time 4 
before calling \clj{check} (\figref{fig:analyzer:typed-analyzer-overview}).
The call to \clj{run-pre-passes} at time 2 will make any introduced let bindings hygienic,
and so it's safe to reason about them with occurrence typing, and thus Typed Clojure.


\begin{figure*}
\singlespacing
$$
  \begin{array}{r||l|l|l|}
    \text{Time} & \text{\clj{(let [...]}}            & \text{\clj{(cond ...}}          & \text{\clj{(+ ...)))}}          \\
    \hline
     0          & \text{\clj{unanalyzed}}^{>}        &                                 &                                 \\
     1          & \text{\clj{analyze-outer}}^{*}     &                                 &                                 \\
     2          & \text{\clj{run-pre-passes}}^{>}    &                                 &                                 \\
     3          & \text{\clj{check}}^{>}             &                                 &                                 \\
     4          &                                    & \text{\clj{analyze-outer}}^{*}  &                                 \\
     5          &                                    & \text{\clj{run-pre-passes}}^{>} &                                 \\
     6          &                                    & \text{\clj{check}}^{>}          &                                 \\
     7          &                                    &                                 & \text{\clj{analyze-outer}}^{*}  \\
     8          &                                    &                                 & \text{\clj{run-pre-passes}}^{>} \\
     9          &                                    &                                 & \text{\clj{check}}^{>}          \\
     10         &                                    &                                 & \text{\clj{run-post-passes}}^{<}\\
     11         &                                    &                                 & \text{\clj{check}}^{<}          \\
     12         &                                    & \text{\clj{run-post-passes}}^{<}&                                 \\
     13         &                                    & \text{\clj{check}}^{<}          &                                 \\
     14         & \text{\clj{run-post-passes}}^{<}   &                                 &                                 \\
     15         & \text{\clj{check}}^{<}             &                                 &                                 \\
  \end{array}
$$

  \caption{Illustrative control flow for interleaved checking and analysis using
  \texttt{core.typed.analyzer}. ${}^*$ denotes zero or more calls.
  }
  \label{fig:analyzer:typed-analyzer-overview}
\end{figure*}

\subsection{Implementation}

We now go into more detail about how \texttt{core.typed.analyzer}
is implemented as a modification of \texttt{tools.analyzer}
and the various tradeoffs that were chosen.

To support the requirement of \clj{analyze-outer} performing as little
analysis as possible, we converting the \clj{analyze} function from
a full AST traversal to a pre-traversal that only visits the current node.
This mostly involved substituting recursive
calls to \clj{analyze-form} with \clj{unanalyzed}, as we
can see from porting the \clj{parse-if} helper function
  in \figref{fig:analyze:parse-if-port}.

\begin{figure*}
\begin{cljlisting}
; tools.analyzer version
(defn parse-if
  "Convert a Clojure `(if <test> <then> <else>)` form to an AST."
  [[_ test then else :as form] env]
  {:op      :if
   :form     form
   :env      env
   :test     (__red>analyze-form<red__ test (assoc env :context :ctx/expr))
   :then     (__red>analyze-form<red__ then env)
   :else     (__red>analyze-form<red__ else env)
   :children [:test :then :else]})

; core.typed.analyzer version
(defn parse-if
  "Convert a Clojure `(if <test> <then> <else>)` form to an AST."
  [[_ test then else :as form] env]
  {:op      :if
   :form     form
   :env      env
   :test     (__red>unanalyzed<red__ test (assoc env :context :ctx/expr))
   :then     (__red>unanalyzed<red__ then env)
   :else     (__red>unanalyzed<red__ else env)
   :children [:test :then :else]})
\end{cljlisting}

  \caption{Example of porting a \texttt{tools.analyzer} function
  to \texttt{core.typed.analyzer} using \clj{unanalyzed} (differences highlighted in \textcolor{red}{red}).
  }
  \label{fig:analyze:parse-if-port}
\end{figure*}

Porting the nano-pass machinery was more involved, however
we have a similar goal: passes must perform the minimum possible
work so they can be easily composed as-needed.
Thankfully, passes in \texttt{tools.analyzer} are written modularly,
so we can straightforwardly pick a subset of them we need for \texttt{core.typed.analyzer}.
To connect the passes, metadata declares dependencies on other passes
and the traversal strategy.
We can see this in action for \clj{constant-lift} (\figref{fig:analyzer:constant-lift}),
which is declared to be part of a post-traversal
that must run after \clj{elide-meta} and \clj{analyze-host-expr}.

\begin{figure}
\begin{cljlisting}
(defn constant-lift
  "Like clojure.tools.analyzer.passes.constant-lifter/constant-lift but
   transforms also :var nodes where the var has :const in the metadata
   into :const nodes and preserves tag info"
  {:pass-info __red>{:walk :post, :depends #{},
               :after #{#'elide-meta #'analyze-host-expr}}<red__}
  [ast]
  (merge (constant-lift* ast)
         (select-keys ast [:tag :o-tag :return-tag :arglists])))
\end{cljlisting}
  \caption{Passes in \texttt{tools.analyzer} are defined as regular functions,
  with \clj{:pass-info} metadata (\textcolor{red}{red}) declaring dependencies on other passes and tree walking strategy.}
  \label{fig:analyzer:constant-lift}
\end{figure}

A scheduler compiles the passes according to this metadata into as few traversals as possible.
We reuse this setup of scheduled passes in \texttt{core.typed.analyzer},
with the restriction that all passes compile into one traversal.
We could convert many existing pre- and post-traversal passes without much modification.
Only the most crucial pass required much modification:
the hygienic transformation pass \clj{uniquify-locals}.
It must be a pre-traversal in \texttt{core.typed.analyzer} (for reasons we have already discussed),
and was modified from a full tree walk.

%Furthermore, passes are almost always extensible via Clojure's multimethods, so it is trivial to add
%support for new AST types, like an AST representation for unanalyzed code.

To help support \clj{:unanalyzed} AST nodes, a 
\clj{:clojure.core.typed.analyzer/config} 
entry (abbreviated \clj{::config}) was added to all nodes
to attach data that applies to AST nodes even after they are expanded.
For example, a top-level expression is still top-level after it is expanded.
The implementations of \clj{unanalyzed} and \clj{analyze-outer}
in \figref{fig:analyzer:config-inheritance} show their propagation---\clj{unanalyzed}
initializes \clj{::config} on line 9, and \clj{analyze-outer} propagates it on line 16 after further analysis.

\begin{figure}
\lstset{numbers=left,xleftmargin=2em,framexleftmargin=1.5em}
  \begin{cljlisting}
(defn unanalyzed
  "Create an unanalyzed AST node from form and env"
  [form env]
  {:op :unanalyzed
   :form form
   :env env
   ;; ::config will be inherited by whatever node
   ;; this :unanalyzed node becomes when analyzed
   __red>::config<red__ {}})

(defn analyze-outer
  "If ast is :unanalyzed, call analyze-form on it, otherwise return ast"
  [ast]
  (case (:op ast)
    :unanalyzed (__red>assoc<red__ (analyze-form (:form ast) (:env ast))
                       __red>::config (::config ast)<red__)
    ast))
\end{cljlisting}
  \caption{The initialization and propagation of \clj{::config} (relevant parts \textcolor{red}{highlighted})}
  \label{fig:analyzer:config-inheritance}
\end{figure}

Finally, we revised to platform-agnostic parts of the \texttt{tools.analyzer} API
to allow better performance.
Symbol and namespace resolution are now platform-dependent, which allows us to 
remove the global environment mirroring we identified as a performance issue
at the beginning of this chapter.
This added a slight burden to platform implementers of 
\texttt{core.typed.analyzer}---the JVM support added a dozen lines of code, although
it took several revisions and testing to recover the original behavior.

%{
%\singlespacing
%\begin{verbatim}
%- Goals
%  1. Build a better tools.analyzer
%     - too slow
%     - too many passes
%     - reuse the passes/scheduler/analysis
%       - and :unanalyzed
%         - instead of analyzing children, store context and return
%       - unforce one pass
%  2. Extensibility
%     - we want custom rules for syntax BEFORE expansion
%     - avoid need for wrapper macros
%       - avoid implementation-dependence
%       - better error messages for users
%     - but lose ability to check actual expansions
%- (This is the Turnstile approach)
%  - Except we don't have syntax objects, how to do it?
%- Create a single-pass tools.analyzer variant that can be paused in
%  the middle of analysis
%  - `analyze` now expands absolute minimum (usually 1 macro)
%- now `check` has access to the raw Clojure forms before they are expanded
%  - much power = much responsibility
%    - top-level evaluation side effects
%    - expansion side effects
%      - talk about that in a different chapter
%    - must manually manage local scope
%    - avoiding double macro expansion
%    - avoiding double evaluation
%    - double analysis is OK though, no side effects
%      - so we can "reinsert" a fully analyzed AST back into
%        a macro call so it can be expanded as usual.
%        - eg. (my-macro (unexpanded))
%            =>
%              (my-macro ~(check (unexpanded) ...))
%- Pros
%  - now have access to original macro forms for higher-level reasoning
%- Cons
%  - no longer checking the implementation of macros
%    - although were we ever, really?
%      - wrapper macros are copied implementation details
%  - must carefully manage compile-time side effects
%\end{verbatim}
%}



\section{Extensibility in Interleaved checking}

Now we present the most significant type system feature
enabled by \texttt{core.typed.analyzer}: custom typing rules.
We already hinted at how this support works in
\figref{fig:analyzer:typed-analyzer-overview}---in this section
we make that explicit with a small type system implementation.

\begin{figure}
\lstset{numbers=left,xleftmargin=2em,framexleftmargin=1.5em}
\begin{cljlisting}
(defn check
  "Check an analyzed AST node has the expected type."
  [expr expected]
  (case (:op expr)
    :if (let [ctest (check-expr (:test expr) (*@\emph{<omitted>}@*))](*@\label{analyzer:listing:typed:check-calls-check-expr}@*)
          (*@\emph{<omitted>}@*))
    :lambda (*@\emph{<omitted>}@*)
    (*@\emph{<omitted other cases>}@*)))
(defn check-expr(*@\label{analyzer:listing:typed:check-expr}@*)
  "Check an AST node has the expected type."
  [expr expected]
  (if (= :unanalyzed (:op expr))
    (case (*@\emph{<resolved-op-sym-for-expr>}@*)
      __red>clojure.core/cond (check-special-cond expr expected)<red__(*@\label{analyzer:listing:typed:check-expr:special-cond}@*)
      ; default case
      (check-expr (analyze-outer expr) expected))
    (run-post-passes
      (check (run-pre-passes expr)(*@\label{analyzer:listing:typed:check-expr-calls-check}@*)
             expected))))
(defn check-form(*@\label{analyzer:listing:typed:check-form}@*)
  "Check a Clojure expression has the expected type"
  [form expected]
  (check-expr (unanalyzed form (empty-env))
              expected))
\end{cljlisting}
  \caption{The driver function \clj{check-form} for a type system using \texttt{core.typed.analyzer},
  which dispatches to a special typing rule for an unexpanded \clj{cond} (\textcolor{red}{red}).}
  \label{fig:analyzer:core.typed.analyzer-driver}
\end{figure}

We now present the sample type system in \figref{fig:analyzer:core.typed.analyzer-driver}.
The main entry point is \clj{check-form} (line \ref{analyzer:listing:typed:check-form}),
and we can check our running example has type \clj{expected}
with:

\begin{cljlisting}
(check-form '(let [...] (cond ... (+ ...)))
            expected)
\end{cljlisting}

A pair of mutually recursive helpers assist the main driver: \clj{check-expr} (line \ref{analyzer:listing:typed:check-expr})
handles the analysis machinery along with unanalyzed forms,
and \clj{check}
which type checks an analyzed AST node.
Once \clj{check-expr} has found a fully analyzed AST, it calls
\clj{check} (line \ref{analyzer:listing:typed:check-expr-calls-check})
in between running the analyzer passes.
Correspondingly, any recursive checking of children performed in \clj{check}
could trigger a special rule for unanalyzed forms, and so
calls \clj{check-expr} (for example, checking \clj{:if}'s test on
line \ref{analyzer:listing:typed:check-calls-check-expr}).

Finally, custom typing rules are dispatched by \clj{check-expr}---we have included
an example dispatch to a \clj{cond} rule on line \ref{analyzer:listing:typed:check-expr:special-cond}.
The \clj{check-special-cond} function now has the ability
to define a robust typing rule for \clj{cond}: it has full access to both
the unexpanded \clj{cond} form and its hygienic type context.

This is a far cry from what was possible with \clj{tools.analyzer},
and so \clj{core.typed.analyzer} is a success in that light.
However, with great power comes great responsibility:
handing users
the ability to control the order of analysis
via custom typing rules
requires careful planning in the face of compile-time side effects.
The next chapter is dedicated to discussing this caveat.

\chapter{Managing Analysis Side effects}

To change Clojure's order-of-macroexpansion is to change the semantics
of Clojure---in theory.
This chapter will give an overview of Clojure's evaluation model
so that the full implications of giving Typed Clojure users the responsibility
to handle macroexpansion via custom typing rules becomes apparent.
We also present how both \texttt{tools.analyzer} and \texttt{core.typed.analyzer}
attempt to preserve these semantics.
We will then compare our issues with those in other systems that allow
typing rules.

\section{Clojure's Evaluation Model}

In this section, we describe the subtleties of evaluating Clojure code.
To evaluate a string of Clojure code, it is first parsed (via \clj{read}) into
a Clojure data representation and then macroexpanded until
it consists of only language primitives.
This is then compiled to JVM bytecode which is executed
to produce the result of evaluation.
Loading a file of Clojure code is mostly equivalent to
evaluating each form in the file from top-to-bottom.

A form is given a special status when considered
\emph{top-level}: it will be completely evaluated
before the next top-level form is expanded.
Under evaluation, a form is considered top-level 
unless it is nested under another form.
For example, \clj{(query)} in \clj{(cond (query) ...)}
is not considered top-level, and the entire \clj{cond} form
is top-level (unless nested in a larger form).
The exception to this rule is nesting under
\clj{do} expressions:
arguments of a top-level \clj{do} form inherits its top-level status.
That is, in the top-level expression \clj{(do (def a ...) (def b ...))},
\clj{a} will be completely defined before
\clj{b} is expanded.
This arrangement allows the expansion of one top-level form
to depend on the evaluation (and thus expansion) of all preceding top-level forms.

As described above, Clojure is always compiled (it has no interpreter).
Clojure offers two modes of compilation: on-the-fly and ahead-of-time.
The main distinction is that
on-the-fly mode discards the generated bytecode after executing it, whereas ahead-of-time mode
both executes and saves the bytecode (as JVM \texttt{.class} files) for later execution.
This is different from other Lisps like Chez Scheme~\cite{dybvig2018chez} and
Common Lisp~\cite{steele1990common},
which has distinct semantics for 
interpreted and compiled modes.
In these languages, there is an implicit assumption that expressions are only
compiled development machines,
and so
compilation mode in these languages skips
the evaluation of certain expressions
to avoid production-only side effects (e.g., initializing databases). Programmers must
use \texttt{eval-when} to opt-in to different behavior.
In contrast, Clojure evaluates all code during compilation (and Clojure is always compiled). Programmers
rely on
on Java-style \clj{main} methods (invoked from the command line) to trigger initialization steps only
applicable in production.

The most important consequence of Clojure's ahead-of-time compilation is that
macros are expanded in a different environment than the program is executed in, and
thus state is not necessarily preserved between them.
This is a well-known problem in most Lisps like Chez Scheme and Common Lisp---to work around it,
Steele~\cite{steele1990common} suggests the convention of moving compile-time
side effects into the code that the macro expands to.
This way, the side effects are evaluation-time, and thus always visible in
every mode of compilation.
Clojure also recommends this convention---without it,
it is possible to have accidental dependencies on expansion
side-effects that only cause bugs under ahead-of-time compilation (usually performed 
only as the last step of software deployment).
Racket's module system, on the other hand, avoids these latent bugs~\cite{flatt2002composable}
by erasing compile-time state before evaluation.
This emulates the conditions of ahead-of-time compilation in Racket's interpreted mode,
at the cost of repeated module reinitializations.

\section{Is order-of-expansion defined in Clojure?}

Order of evaluation in Clojure is usually specified where it makes sense~\cite{CljEvalDoc}.
For example, invocations \clj{(f arg*)} are evaluated left-to-right
starting from \clj{f}, whereas the order of evaluation for elements of unordered set literals \clj{#\{k*\}}
is undefined.
On the other hand, the order of \emph{expansion} is not addressed at all in the Clojure
documentation.
It would be extremely convenient for the writers and users of Typed Clojure
to avoid micromanaging the order of expansion, and would
make writing custom typing rules and other Typed Clojure
extensions more viable.
With those biases in mind, we now attempt to give a balanced account
of expansion order in Clojure.

It is worth distinguishing between order of expansion
of top-level forms and inner forms.
Common Lisp asserts~\cite{steele1990common} that the order
of macroexpansion for inner forms is unspecified.
This gives flexibility not only to both platform implementors
but also macro writers, because it grants macros the flexibility to
expand their arguments, a pattern
used by the Clojure core library \texttt{core.async}~\cite{CljCoreAsync}.
This seems to work in practice for \texttt{core.async} users
without any special instruction or warnings.
Also, \texttt{core.async} was designed
by the same team that develops Clojure itself,
so it gives us more confidence that changing expansion order
(by manually expanding a macro's arguments)
is a sound choice.

Even more reason to doubt the importance of expansion
order is its seeming lack of preservation 
across platforms for core macros.
For instance,
Clojure and ClojureScript share the same infrastructure for
writing macros, but only share a subset of core macro \emph{definitions}.
That is, some macros are redefined in ClojureScript to cater to the
JavaScript host---furthermore, some functions in Clojure are turned into macros
in ClojureScript.
While the order of evaluation must be preserved
for compatibility with Clojure,
it seems unlikely that any special measures were taken to preserve
expansion order of arguments---especially for more complicated,
platform-specific macros.
In practice, however, most macros probably do preserve expansion-order:
idiomatic macros do not expand their arguments
and merely forward them to more primitive operators
(often \clj{do} or \clj{let}) that have
more consistent expansions across platforms.
This might be coincidental, since
we are not aware of any special effort to force this style,
and might more be a consequence of following general Clojure idioms.

The popular general-purpose code analyzer \texttt{tools.analyzer} 
ignores particular expansion-time side-effects, without any
apparent downsides.
Specifically, changes to the current namespace are ignored
during macroexpansion. This is
a common \emph{runtime} side-effect in Clojure, and is crucial for
an analyzer to adhere to because analyzing a form in the
wrong namespace is incorrect.
Even so, we are not aware of any cases in practice in which this is a problem.
Given that \texttt{tools.analyzer} is thoroughly tested and used
in industry, it might then be reasonable to conclude that expansion-time side effects
are rare. On the other hand, changing namespaces is a very specific side-effect
whose conventions are perhaps not generalizable to other side-effects.
Usually, changing namespaces is only triggered by the expansion of the \clj{ns} macro,
and \clj{ns} is almost always used exactly once at the top of every Clojure file (to declare
namespace dependencies). There could be other, more common expansion-time
side-effects that are compatible with \texttt{tools.analyzer} that
we are not aware of.

Relatedly, to help measure the practicality of an alternative design of Typed Clojure
that expands macros multiple times,
we talked to Clojure and Racket developers about repeated macro expansions.
One Clojure developer felt
that avoiding repeated expansion was important in Clojure,
but could not show an example of real-world code
that would fail in these circumstances.
In contrast, a Racket researcher was quick to demonstrate complex
assumptions in their macros that would be violated in these
conditions (for example, global counters for identifying specific expansions).

Reflecting on these anecodes, Clojure developers routinely reload
code (REPL's can take minutes to start and so encourage developers to leave
them open for days)
and so are guided to write code that can be re-evaluated
in almost any order (after an initial load), encouraged further by the late-binding semantics
of Clojure Vars.
On the other hand, Racket's module system is much stricter and reinitializes
entire modules in their own sandboxes. Racket programmers can (and do) rely
on these restrictions to support complex top-level invariants.

In both Clojure and Racket, it is idiomatic to avoid ``double expansions''
when writing macros by binding intermediate results to names in a macro's
expansion, usually
for performance reasons.
This also prevents double \emph{evaluation}, an even more serious performance concern
since expressions are run many more times than they are
expanded.
The issue of a third party (like Typed Clojure) expanding macros
multiple times is tangentially related to this idiom, since
the cost of double expansions must be paid.
The main difference, at least in the design we proposed,
was that the extra expansions by the third party would eventually be discarded
and not evaluated, thus avoiding the cost of double evaluation.

It's interesting to note that the backgrounds and daily obligations of each groups varied significantly,
with the Racket programmers being mostly from academia (studying language extensibility)
and the Clojure developers mostly from industry.
While these opinions about repeated expansions
are useful to help contextualize our larger discussion of
order-of-expansion for inner forms,
there are important details to take into account
before prematurely
linking the two subjects.
We must avoid using the fact that Clojure programmers
reload expressions out-of-order as direct evidence 
to support changing the expansion order of inner forms.
This is because only \emph{top-level} expression are reloaded---intuitively,
this does not change the expansion order of inner forms.
We also note that reloading Clojure expressions can be notoriously
buggy in certain circumstances, resulting in desynchronization
between code on disk and code loaded into memory.
Many disparate Clojure libraries and conventions have been developed to
help manage these situations and there is no centralized solution.
On the other hand, this problem is recognized and addressed by Racket's module system.
We do not want Typed Clojure contributing yet another source of
desynchronization, which is the main reason behind this extended discussion.

%Given that information, combined with Clojure's strong Common Lisp heritage,
%we think it is reasonable to assert that the order of expansion
%(of inner forms) is also unspecified in Clojure.

%Clojure tracks the current \emph{namespace} with the dynamic
%variable \clj{*ns*}.

%{
%\singlespacing
%\begin{verbatim}
%- introduce Clojure's evaluation model
%  - `do` children treated as top-level
%    - eg. (do e1 e2)
%      - e1 is completely expanded+evaluated before e2 is expanded.
%      - e2 can depend on evaluation-time side effects of e1
%    - eg. (let [] (do e1 e2))
%      - e1 is ONLY expanded before e2
%      - e2 cannot depend on evaluation-time side effects of e1
%- introduce *ns*, global thread-local variable that holds the current namespace
%  - probably the most important evaluation-time side effect
%    - changing namespaces
%- introduce AOT-compilation
%  - macroexpansions are performed upfront
%    - so relying on expansion side-effects are uncommon and unreliable
%- since checking & evaluation are interleaved, important that
%  order-of-checking == order-of-evaluation
%- but what about order-of-macroexpansion?
%  - unaware of examples where order of macroexpansion side-effects are important
%    outside of top-level forms
%    - we have serious problems otherwise for any analysis tool
%      - eg. (fn [] (change-ns-at-mexpansion-time) (relys-on-previous-macro-to-resolve))
%        - this evaluates just fine in Clojure since there is
%          no special handling of *ns* and things never pause and always just happen
%          in order
%        - but seems intractable for things like typing rules
%          - custom typing rules must check forms in the order they are mexpanded!
%            - seems very restrictive and error prone.
%          - eg. ((fn [] ...) args ...)
%            - imagine delaying checking a `fn` before its arguments.
%              - but if args depend on fn body to expand, then we are forced
%                to fully expand (ie. CHECK) the fn before checking the args.
%                - very restrictive
%- Relevant reading:
%  - Flatt, "You want it, when?"
%  - SamTH, Scheme 2007 (good for lispy background), PLDI 2011 (general audience)
%- Chez
%  - https://www.scheme.com/csug8/system.html#./system:s76
%  - AOT compilation does not evaluate top-level forms by default
%    - special cases for `define-syntax`
%    - use eval-when for special cases where syntax depends on define's
%      - eg. (eval-when (compile) (defn helper ..))
%            (defmacro m ... (helper ...) ...)
%  - Compared to Clojure:
%    - AOT compilation in Clojure evaluates top-level forms
%      - must be more careful with top-level evaluation side effects then,
%        usually 
%      - usually use -main function for side effects to start a program
%- Common Lisp
%  - http://filonenko-mikhail.github.io/cltl2-doc/enpdf/cltl2.pdf
%  - section 5.3.3
%    - top-level `progn` members are also top-level
%    - order of macroexpansion (for inner forms) is unspecified
%    - convention: top-level macros should side-effect during eval, not expand
%\end{verbatim}
%}

\section{Preserving evaluation order during type checking}

Now that we have discussed some details of Clojure's evaluation model,
we demonstrate how to write a type checker that correctly preserves
these semantics.
Most of the details concern top-level expressions. In particular,
a top-level \clj{do} gives top-level status to its arguments, and
that top-level forms must be completely evaluated in order.

The most interesting case to consider is a top-level \clj{(do e1 e2)} form,
where the
\emph{expansion} of \clj{e2}
depends on the \emph{evaluation} of \clj{e1}.
For example,

{
\lstset{numbers=left,xleftmargin=2em,framexleftmargin=1.5em}
\begin{cljlisting}
(do (defmacro mac [] 42)
    (mac))
\end{cljlisting}
}

defines the macro \clj{mac} and then uses it to expand
the macro call on the the last line (the final result is \clj{42}).
This will only execute correctly if top-level semantics of \clj{do} 
are faithfully preserved.
We will use this as a running example in our exploration of
each analyzer.

\subsection{\texttt{tools.analyzer}}

\begin{figure}
\singlespacing
$$
\begin{array}{r||l|l|l|l|}
\text{Time} & \text{\clj{(do}}               & \text{\clj{(defmacro mac [] 42)}} & \text{\clj{(mac))}}\\
\hline
            & \text{\clj{analyze+eval}}^{>}  &                                                   & \\
            & \textcolor{red}{\text{\clj{macroexpand-1}}^{*}} &                                  & \\
            &                                & \text{\clj{analyze+eval}}^{>}                     & \\
            &                                & \textcolor{red}{\text{\clj{macroexpand-1}}^{*}}   & \\
            &                                & \text{\clj{analyze+passes}}                       & \\
            &                                & \text{\clj{check}}                                & \\
            &                                & \textcolor{red}{\text{\clj{eval}}}                & \\
            &                                & \text{\clj{analyze+eval}}^{<}                     & \\
            &                                & & \text{\clj{analyze+eval}}^{>}   \\
            &                                & & \textcolor{red}{\text{\clj{macroexpand-1}}^{*}}    \\
            &                                & & \text{\clj{analyze+passes}}     \\
            &                                & & \text{\clj{check}}              \\
            &                                & & \textcolor{red}{\text{\clj{eval}}}                 \\
            &                                & & \text{\clj{analyze+eval}}^{<}   \\
            & \text{\clj{analyze+eval}}^{<}  &                      \\
\end{array}
$$
  \caption{Using \texttt{tools.analyzer}
  to interleave evaluation of
  top-level forms during the checking of \clj{(do (defmacro mac [] 42) (42))}.
  }
  \label{fig:analyzer:control-flow-pre-expand-side-effects}
\end{figure}

An \clj{analyze+eval} function handles top-level evaluation
concerns in \texttt{tools.analyzer}, as 
\figref{fig:analyzer:control-flow-pre-expand-side-effects}
demonstrates with our
running example.
The type checker's main obligation is to provide a \clj{check}
function that checks a completely analyzed AST before it is evaluated.
When passed to \clj{analyze+eval}, it will only call \clj{check}
on the smallest top-level forms by expanding macros until
a non-\clj{do} expression is reached.

There are several other details that must be handled
that led us to write our own \clj{analyze+eval} variant for Typed Clojure---for
example, expected type propagation and
handling the \clj{do}-special-form protocol. We
do not discuss those details here, and hope that
the original implementation of \clj{analyze+eval} combined with the
descriptions in this section should be sufficient to create other variants.

\subsection{\texttt{core.typed.analyzer}}

We have just described how to preserve evaluation order with \texttt{tools.analyzer}.
The main distinction to keep in mind is that \texttt{tools.analyzer} controls
analysis and checking for us, and so it handles its own internal bookkeeping 
to keep track of top-level forms.
On the other hand, this section describes the same problem with \texttt{core.typed.analyzer},
which has comparatively very little control of when analysis is performed
(since the provided \clj{analyze-outer} function only performs a single expansion).
Some other party---in this case, the type checker---must incorporate analysis into its
main loop.

To handle this, \texttt{core.typed.analyzer} uses markers to
distinguish between
different kinds of top-level expressions.
The first marker is performed by
\clj{mark-top-level} 
and is added to every kind of top-level expression.
The second marker, \clj{mark-eval-top-level},
further marks a top-level expression for evaluation.
The \clj{eval-top-level} function 
then evaluates AST nodes marked by \clj{mark-eval-top-level}
(with other corner cases we will describe later).

We use \figref{fig:analyzer:control-flow-incremental-side-effects}
to demonstrate this communication with our running example.
The entry point here is \clj{check-top-level} at time 0, which has been enhanced
to handle top-level evaluation order.
To set up the rest of checking,
at time 2 the outer \clj{do} expression is marked with \clj{mark-top-level}.
Then, once \clj{analyze-outer} determines the outer expression is a \clj{do},
its children are then marked with \clj{mark-top-level} (times 4 and 5).

For the \clj{defmacro} form, let us assume for simplicity that
it expands to a non-\clj{do} expression.
Under those circumstances, 
once \clj{analyze-outer} at time 8 reaches a fixed point, the resulting AST node is marked
for evaluation with \clj{mark-eval-top-level} at time 9.
Then (after it has been checked) at time 14 \clj{eval-top-level} evaluates the \clj{defmacro}
expression (because it was marked with \clj{mark-eval-top-level}).
This repeats similarly for the final expression \clj{(mac)}.

There are a few notable cases to where \clj{eval-top-level}
does \emph{not} trigger evaluation.
The common case where evaluation is skipped is when an AST node has no top-level markings,
like during time 11 where the main checking loop
should not recursively evaluate expressions.
A less-obvious case is a node marked with \clj{mark-top-level},
but \emph{not} \clj{mark-eval-top-level}.
For example, the outer-most \clj{do} is only marked by
\clj{mark-top-level} (time 2). Evaluation is then skipped at time 24 because
the top-level markers have been passed along to its children
(at times 4 and 5)
and so they themselves
have been evaluated
(at times 14 and 21). If we evaluated at time 24,
it would result in
an incorrect double-evaluation of the \clj{defmacro} and \clj{(mac)} expressions---hence
evaluation is skipped by \clj{eval-top-level} in this case.

\begin{figure*}
\singlespacing
$$
  \begin{array}{r||l|l|l|l|}
    \text{T} & \text{\clj{(do}}                                   & \text{\clj{(defmacro mac [] 42)}} & \text{\clj{(mac))}}\\
    \hline
     0       & \text{\clj{check-top-level}}^{>}                   & &                     \\
     1       & \text{\clj{unanalyzed}}                            & &\\
     2       & \textcolor{red}{\text{\clj{mark-top-level}}}       & &\\
     3       & \text{\clj{analyze-outer}}^{*}                     & &\\
     4       &                                                    & \textcolor{red}{\text{\clj{mark-top-level}}}&\\
     5       &                                                    & & \textcolor{red}{\text{\clj{mark-top-level}}}          \\
     6       & \text{\clj{run-pre-passes}}^{>}                    & &\\
     7       & \text{\clj{check}}^{>}                             & &\\
     8       &                                                    & \text{\clj{analyze-outer}}^{*}    &                           \\
     9       &                                                    & \textcolor{red}{\text{\clj{mark-eval-top-level}}}&\\
     10      &                                                    & \text{\clj{run-pre-passes}}^{>}   &                           \\
     11      &                                                    & \text{\clj{check}}^{>}            &                           \\
     12      &                                                    & \text{\clj{run-post-passes}}^{<}  &                           \\
     13      &                                                    & \text{\clj{check}}^{<}            &                           \\
     14      &                                                    & \textcolor{red}{\text{\clj{eval-top-level}}}^{<}       & \\
     15      &                                                    & & \text{\clj{analyze-outer}}^{*}                               \\
     16      &                                                    & & \textcolor{red}{\text{\clj{mark-eval-top-level}}}          \\
     17      &                                                    & & \text{\clj{run-pre-passes}}^{>}                              \\
     18      &                                                    & & \text{\clj{check}}^{>}                                       \\
     19      &                                                    & & \text{\clj{run-post-passes}}^{<}                             \\
     20      &                                                    & & \text{\clj{check}}^{<}                                       \\
     21      &                                                    & & \textcolor{red}{\text{\clj{eval-top-level}}}^{<}             \\
     22      & \text{\clj{run-post-passes}}^{<}                   & &\\
     23      & \text{\clj{check}}^{<}                             & &\\
     24      & \textcolor{red}{\text{\clj{eval-top-level}}^{<}}   & &\\
     25      & \text{\clj{check-top-level}}^{<}                   & &                     \\
  \end{array}
$$
  \caption{Using \texttt{core.typed.analyzer}
  to interleave checking with the evaluation of
  top-level forms.}
  \label{fig:analyzer:control-flow-incremental-side-effects}
\end{figure*}

Incorporating correct top-level evaluation into a \texttt{core.typed.analyzer}-based
type checking loop is a mostly-straightforward extension, as demonstrated in
  \figref{fig:analyzer:core.typed.analyzer-top-level-driver}.
The main differences are highlighted.
First, \clj{unanalyzed-top-level} combines
\clj{unanalyzed}
and
\clj{mark-top-level} (times 1 and 2\ in our previous discussion).
Then, \clj{eval-top-level} should be added as the final operation in
the main checking loop.
%
Finally,
the handling of unanalyzed nodes 
(line \ref{analyzer:listing:cta-top-level-driver:unanalyzed-cases})
must carefully manage markers to prevent double evaluations.
In particular, \clj{eval-top-level}
has a corner case to handle AST nodes that are not fully analyzed:
nodes that are both \clj{:unanalyzed} and 
are marked with \clj{mark-top-level} \emph{are} evaluated.
The corresponding functions \clj{unmark-top-level}
\clj{unmark-eval-top-level} can be used
to undo the markers if these semantics are undesirable for particular
situations.

\begin{figure}
\lstset{numbers=left,xleftmargin=2em,framexleftmargin=1.5em}
\begin{cljlisting}
(defn check-expr
  "Check an AST node has the expected type."
  [expr expected]
  (if (= :unanalyzed (:op expr))
    __red>...<red__ (*@\label{analyzer:listing:cta-top-level-driver:unanalyzed-cases}@*)
    (__red>eval-top-level<red__
      (run-post-passes
        (check (run-pre-passes expr)
               expected)))))
(defn check-top-level [form expected]
  (check-expr (__red>unanalyzed-top-level<red__ form (empty-env))
              expected))
\end{cljlisting}
  \caption{Handling top-level forms in a checker based on \clj{core.typed.analyzer}}
  \label{fig:analyzer:core.typed.analyzer-top-level-driver}
\end{figure}


\part{%Local Type Argument Synthesis with Symbolic Closures
Symbolic Closures}
\label{part:symbolic-closures}

\chapter{Background}

As is inevitable for an optional type system, there are many
Clojure programs that Typed Clojure was not designed to type check.
These programs contain Clojure idioms that are often either intentionally
not supported by Typed Clojure's initial design, or 
were introduced to Clojure at a later date.
Regardless, programmers will inevitably want to use these features 
in their Typed Clojure programs---but crucially without breaking
support for existing idioms.
In this part, we explore what kinds of idioms are missing
support in Typed Clojure, and propose solutions in the form of
backwards-compatible extensions.

As we discussed in \partref{part:types}, Typed Clojure's initial
design is strongly influenced by Typed Racket. In particular,
Typed Clojure's static semantics of
combinining local type inference and occurrence typing
to check fully-expanded code
comes directly from Typed Racket.
This shared base is appropriate, given the similarities between
the base Clojure and Racket languages.
It is also effective, seamlessly handling many control flow
idioms, capturing many polymorphic idioms, and often yielding
predictable type error messages.
However, there are important tradeoffs to consider in this design---in the following
sections we introduce them and propose extensions to attempt to nullify
their downsides.

\section{Enhancing Local Type Inference}

``Local Type Inference''~\cite{PierceLTI} refers to the combination of
bidirectional type propagation 
and local type argument synthesis.
%specific approach
%to partial type inference for languages with subtyping and impredicative
%polymorphism.
%``Partial type inference'' here is
%problem of inferring 
%(Pfenning~\cite{Pfenning1988partial} describes the distinction more thoroughly).
%
Concerning the limitations of local type inference,
Hosoya and Pierce~\cite{hosoya1999good}
isolate two drawbacks.
The first is dealing with ``hard-to-synthesize arguments''.
To understand this, we must appreciate a key ingredient of local type inference
called \emph{bidirectional propagation}, which 
we use the example of type checking \clj{(inc 42)} to demonstrate.
If we have already checked \clj{inc} to have type \clj{[Int -> Int]}, we
now have a choice of how to check the argument \clj{42} is an \clj{Int}.
The first is to ascribe an expected type to \clj{42} of \clj{Int}
and rely on
bidirectional \emph{checking mode} to ensure \clj{42} has the correct type
once we check it.
The second is to infer the type of \clj{42} (without an expected type) using 
bidirectional \emph{synthesis mode}, and then ensure the inferred type
is compatible with \clj{Int} after the fact.
A useful analogy in terms of expressions is that checking mode propagates
information outside-in, and synthesis mode propagates inside-out.
A similar analogy in terms of a type derivation tree (that grows upwards)
relates checking and synthesis modes to information being passed
up and down the tree, respectively.

To best serve the purposes of local type inference, it is crucial to stay in
bidirectional \emph{checking} mode as much as possible.
The ``hard-to-synthesize arguments'' problem occurs when type argument
inference interferes with the ability to stay in checking mode, and
thus forces the bidirectional propagator into synthesis mode
for arguments that require checking mode.
For example, to type check

\clj{(map (fn [x] (inc x)) [1 2 3])},

where \clj{map} has type

\clj{(All [a b] [[a -> b] (Seqable a) -> (Seqable b)])},

we use type argument inference to determine how to instantiate type variables \clj{a}
and \clj{b} based on \clj{map}'s arguments.
Unfortunately, 
to answer this question,
the naive local type inference algorithm~\cite{PierceLTI}
uses synthesis mode to retrieve the argument types,
and so checks \clj{(fn [x] (inc x))} in synthesis mode.
No information is propagated about the type of \clj{x},
so this expression will fail to type check, demonstrating
why functions are hard-to-synthesize.

The second drawback noted by Hosoya and Pierce are
cases where there is no ``best'' type argument to infer.
This occurs when there is not enough information available
to determine how to instantiate a type such that the program
has the best chance of type checking, and so it must be guessed.
A representative case where this occurs is inferring the
type of a reference from just its instantiation, such
that optimal types are given to reads and writes.
For example, the following code creates a Clojure Atom
(a reference type) with initial value \clj{nil}, writes
\clj{0} to the Atom, and then increments the Atom's value.

{
\begin{lstlisting}[language=Clojure]
(let [r (atom nil)]
  (reset! r 0)
  (inc @r))
\end{lstlisting}
}

What type should \clj{r} be assigned? From its initial binding,
\clj{(Atom nil)} seems appropriate, but the subsequent write
would fail. Alternatively, assigning \clj{(Atom Any)} would
allow the write to succeed, but the the final read would fail
because it expects \clj{Int}.
This demonstrates difficulties of the ``no-best-type-argument'' problem.

Hosoya and Pierce report unsatisfactory results in their attempts to
fix these issues, in both the effectiveness and complexity
in their solutions.
They speculate that these difficulties might be better 
addressed at the language-design level---rather than algorithmically---in ways that
keep the bidirectional propagator in checking mode.
For the ``no-best-type-argument'' problem,
we agree with this assessment, since
addressing the problem mostly amounts to annotating 
all reference constructors.
To this end,
Typed Clojure offers several
wrappers for common functions where this problem
is common---the previous example might use the ``typed''
constructor \clj{(t/atom :- (U nil Int), nil)}.

However, the ``hard-to-synthesize arguments'' problem
is a deeper and more pervasive issue when checking Clojure code.
We don't have the luxury, desire, nor do we think it would be particularly
successful to introduce new core idioms to Clojure,
and so we attempt to solve the this problem algorithmically.

%\begin{lstlisting}[language=Clojure]
%(for [i (range 100)]
%  (map (fn [j] (inc j))
%       (range i)))
%\end{lstlisting}

Hosoya and Pierce outline the two main challenges that must be
addressed to solve the ``hard-to-synthesize arguments'' problem.
First, we must provide a strategy for identifying which arguments 
should be avoided.
For instance,
they provide a simple grammar for identifying hard-to-synthesize arguments,
which includes (for Standard ML) unannotated functions and unqualified constructors.
Second, an alternative (probably more complicated) algorithm
for inferring type arguments is needed that also handles
avoided arguments.
Their experiments show that the naive approach does not suffice,
and hint at the delicate handling needed to effectively maximize or minimize
instantiated types to stay in checking mode.
We will now use these challenges as a presentational framework to outline our own approach.

In our experience, the most common hard-to-synthesize expression in Clojure code
is the function.
Clojure's large standard library of higher-order functions and encouragement
of functional programming result in many usages of anonymous functions, which almost
always require annotations to check with Typed Clojure.
So, to answer Hosoya and Pierce's first challenge, 
we avoid checking hard-to-synthesize function expressions by
introducing a new function type: a \emph{symbolic closure type}.
A symbolic closure does not immediately check the function body. Instead,
the function's code along with its local type context is saved
until enough information is available to check
the function body in checking mode.
We present more details about symbolic closures in \chapref{chapter:symbolic:symbolic-closures}.

Now that we have delayed the checking of hard-to-check arguments,
Hosoya and Pierce's second challenge calls for an enhanced
type argument reconstruction algorithm to soundly handle
them.
We delegate this challenge to future work, and study symbolic closures
with an off-the-shelf
type argument reconstruction algorithm.
%Our investigation led us to create \emph{directed local type inference}
%(\chapref{chapter:symbolic:directed-lti}),
%which determines the possible data flows through a polymorphic function
%by noting the positions of type variable occurrences, and attempts to
%use this information to check its arguments in an optimal order for remaining
%in bidirectional checking mode.

%\section{Custom typing rules}
%
%Besides local type inference,
%another significant feature inherited from Typed Racket is that
%code is fully expanded before checking.
%This unfortunately means that macros with complex expansions
%are often uncheckable, and display cryptic error messages when attempting
%to do so.
%We investigate providing the user with \emph{custom typing rules} (\chapref{chapter:symbolic:custom-rules})
%as an extension point to customize how to type check a macro before
%it is expanded.
%As discussed in 
%\partref{part:implementations}, Typed Clojure's initial design does
%not support custom typing rules, so we exploit the infrastructure
%discussed there,
%and present our investigation into the user interface for the rules in
%\chapref{chapter:symbolic:custom-rules}.

% - "Avoiding hard-to-synthesize arguments"
%   1. need mechanism to decide which arguments to avoid
%   2. more complicated scheme for determining best type arguments

% - Problem
%   - many common idioms cannot be checked
%   - limitations of local type inference
%   - made harder by occurrence typing
%   - want general solutions available to all users
%   - preliminary investigation of several techniques
% - Possible solutions
%   - symbolic analysis
%     - symbolic closures
%       - deal with "obvious" local function annotations
%     - inlining
%   - directed local type inference
%     - derive data flow from polymorphic types for more aggressive local type variable inference
%   - custom typing rules
%     - interface for describing how to check an unexpanded macro call
%       - or complex functions
%     - custom errors
% - Constraints
%   - some speculation of how well they compose together
%   - small models without rigorous proofs
%   - case studies with real Clojure idioms

\chapter{Delayed checking for Unannotated Local Functions}
\label{chapter:symbolic:symbolic-closures}

Using bidirectional type checking, functions are hard-to-synthesize types for.
Put another way, to check a function body successfully using only locally available information,
types for its parameters are needed upfront.
For top-level function definitions, this is not a problem in many
optional type systems since top-level annotations would be provided
for each function.
However, for anonymous functions it's a different story.
The original local type inference algorithm~\cite{PierceLTI}
lacks a synthesis rule for unannotated functions, instead relying on bidirectional
propagation of types,
but due to the prevalence
of hard-to-synthesize anonymous functions in languages like JavaScript, Racket, and Clojure,
optional type systems for the languages add their own rules.

Typed Racket and Typed Clojure implement a simple but sound strategy
to check unannotated functions. The body of the function is checked
in a type context where its parameters are of type \clj{Any},
the \texttt{Top} type.
This helps check functions that don't use their arguments, or only
use them in positions that require type \clj{Any}.
For example, both \clj{(fn [x] "a")} and \clj{(fn [x] (str "x: " x))} 
synthesize to \clj{[Any -> String]} in Typed Clojure.
The downsides to this strategy are that unannotated functions are never
inferred as polymorphic, and functions that use their arguments
at types more specific than \clj{Any} are common.

TypeScript~\cite{typescript}, an optional type system for JavaScript,
takes a similar approach, but instead of annotating parameters with
TypeScript's least permissive type called \js{unknown},
by default it assigns parameters the unsound dynamic type \js{any}.
In TypeScript, \js{any} can be implicitly cast to any other type,
so the type checker will (unsoundly) allow any usage of unannotated arguments.
If this behavior is unsatisfactory,
the \js{noImplicitAny} flag removes special handling for unannotated
functions altogether, and TypeScript will demand explicit annotations for all arguments.

In this chapter, we present an alternative approach to checking unannotated functions
based on the insight that a function's body need only be type checked if and when it is called.
For example, the program \clj{(fn [x] (inc x))} cannot throw a runtime error because
the function is never called, and so a type system may soundly treat the function body as unreachable code.
On the other hand, wrapping the same program in the invocation
\clj{((fn [x] (inc x)) nil)}
makes the runtime error possible, and so a sound static type system must flag the error.

Exploiting this insight in the context of a bidirectional type checker using
local type inference requires many considerations.
First, we must decide in which situations is it desirable to delay checking a function.
Second, we must identify the information that must be saved in order to delay checking a function,
and then choose a suitable format for packaging that information.
Third, we must identify how a function is deemed ``reachable'',
and then which component of the type system is responsible for checking a function body.
Fourth, it is desirable to identify and handle the ways in which 
infinite loops are possible, such as the checking of a delayed function triggering
another delayed function to check, which triggers another delayed check, ad nauseam.
Fifth, we must determine how delayed functions interact with polymorphic types
during type argument reconstruction.

We address all these considerations in the following sections, except
for the final one, which we delegate to future work. %\chapref{chapter:symbolic:directed-lti}.

\section{Overview}
\label{symbolic:section:overview}

In this section, we explore some of the implications that come with delayed checks for local functions,
by example.
To explain the essense of the challenges we actually address, we avoid polymorphic functions in this section
%(we isolate those issues in \chapref{chapter:symbolic:directed-lti})
and restrict ourselves to non-recursive monomorphic functions.

First, let \clj{inc} be of type \clj{[Int -> Int]}.
The following, then, is well typed because \clj{1} is an \clj{Int}.

\begin{lstlisting}[language=Clojure]
(inc 1)
\end{lstlisting}

Using the standard bidirectional application type rule, \clj{inc} is checked first,
followed by \clj{1}.
However, eta-expanding the operator does not behave as nicely.

\begin{lstlisting}[language=Clojure]
((fn [x] (inc x)) 1)
\end{lstlisting}

Like usual, the standard application rule checks the function first.
However, there is no annotation for \clj{x}, so the function body will fail
to check.
This is unfortunate, especially in a type system that claims to be ``bidirectional'',
since the information that \clj{x} is an \clj{Int} is adjacent to the function
in the form of an argument.
One strategy to alleviate this problem is to always check arguments first~\cite{xie2018let}.
However, that nullifies the ability for the operator to propagate information
to its arguments, whose advantages are exploited to good effect in Colored Local Type Inference~\cite{coloredlti01}.

We combine both flavors by keeping the standard operator-first checking order
but delay the checking of unannotated functions.
Then, an additional application rule handles applications of
unannotated functions to force their checking.
So in this case, the checking of \clj{(inc x)}
is delayed until the argument \clj{1} is inferred as \clj{Int},
after which this information is used to check \clj{(inc x)}
in the extended type context where \clj{x : Int}.

We could imagine hard-coding a type rule that manually delays
direct applications of unannotated functions until after checking
its arguments.
However, that does not generalize to more complicated examples.
Take the following illustrative code, identical the previous
example, except the function is let-bound as \clj{f}.

{
%\lstset{numbers=left,xleftmargin=2em,framexleftmargin=1.5em}
\begin{lstlisting}[language=Clojure]
(let [f (fn [x] (inc x))](*@\label{symbolic:example:let-bound:def-f}@*)
  (f 1))(*@\label{symbolic:example:let-bound:app-f}@*)
\end{lstlisting}
}

Instead of following the brittle strategy of creating yet-another special rule to delay checking
let-bound functions, we generalize the idea.
We make a delayed function check a first-class concept in our type-system by
creating a new type for it.
Roughly, \clj{f} would have a delayed function type---introduced by
a type rule for unannotated functions---and \clj{(f 1)}
would force a check for the delayed function---by an application
rule that handles delayed function \emph{types} (not syntax-driven).

Now we must decide what a delayed function type consists of.
Clearly, the \emph{code} of the function must be preserved until
it is checked, otherwise the application rule would have nothing
to work with.
We note that our static semantics of saving
the code of a function to check later
is analogous to the runtime strategy of
evaluating a function as \emph{closure},
and using beta-reduction to extract the original
function from the closure and apply it to its arguments.

The trick in maintaining lexical scope during beta-reduction for closures
is to apply the function under the \emph{function definition's}
environment, instead of the application site's.
For example,
\figref{symbolic:example:closure-red}
evaluates
to \clj{2}
because
the occurrence of
\clj{y} on line \ref{symbolic:example:closure-red:y-usage}
is bound to \clj{1} by line \ref{symbolic:example:closure-red:y-def-site}.
If we used the local environment at the application site (line \ref{symbolic:example:closure-red:f-app}),
\clj{y} would be bound on line \ref{symbolic:example:closure-red:y-app-site}
to \clj{nil},
and would throw a runtime error.

% must save type context
\begin{figure}
{
\lstset{numbers=left,xleftmargin=2em,framexleftmargin=1.5em}
\begin{lstlisting}[language=Clojure]
(let [f (let [y 1](*@\label{symbolic:example:closure-red:y-def-site}@*)
          (fn [x] (+ x y)))](*@\label{symbolic:example:closure-red:y-usage}@*)
  (let [y nil](*@\label{symbolic:example:closure-red:y-app-site}@*)
    (f 1)))(*@\label{symbolic:example:closure-red:f-app}@*)
\end{lstlisting}
}
  \caption{This example evaluates to \clj{2} with lexically scoped variables.}
  \label{symbolic:example:closure-red}
\end{figure}

The crucial insight is that
the same trick applies to \emph{checking} delayed function types,
except at the \emph{type}-level.
Specifically, the occurrence of \clj{y}
on line \ref{symbolic:example:closure-red:y-usage}
must be checked as type \clj{Int} (from line \ref{symbolic:example:closure-red:y-def-site}),
and not type \clj{nil} (from line \ref{symbolic:example:closure-red:y-app-site}).
So, a delayed function type pairs a function's code with the type environment
at the function definition site.
This strongly resembles a ``type-level'' closure that is reduced symbolically,
and so we call this new type a \emph{symbolic closure}.

We can use symbolic closures to inline higher-order-function definitions.
In the following example, \clj{app} would normally need a higher-order
or polymorphic
annotation to handle the application on the final line.
Instead, with symbolic closures, type checking reduces in a few steps to simply checking
\clj{(inc x)} where \clj{x : Int}.

% more beta reduction
\begin{lstlisting}[language=Clojure]
(let [f (fn [x] (inc x))
      app (fn [g y] (g y))]
  (app f 1))
\end{lstlisting}

As alluded to in the previous section, we must identify
all type system components who are responsible for checking symbolic closures,
and ensure they perform their obligations correctly.
The following example uses a higher-order function
\clj{app-int} to increment the value \clj{1}.
Since \clj{app-int} is annotated, it will be checked
by the standard application rule.
However, its first argument will be delayed as a symbolic
closure---now we must identify who is responsible for checking it.

% using subtyping to check symbolic closures.
\begin{lstlisting}[language=Clojure]
(ann app-int [[Int -> Int] Int -> Int])
(defn app-int [f x] (f x))
...
(app-int (fn [x] (inc x)) 1)
\end{lstlisting}

The type signature of \clj{app-int},
clearly says that its first argument may be called with an \clj{Int}.
Therefore, to maintain soundness, applications of \clj{app-int}
must ensure its first argument accepts \clj{Int}.
The standard application type rule uses subtyping to ensure
provided arguments are compatible with the formal parameter types of
the operator.
To handle symbolic closures, we preserve the standard application rule
and instead add a subtyping case for symbolic closures.

In this case, the subtyping relation would be asked to verify if
``the symbolic closure type representing \clj{(fn [x] (inc x))}
is a subtype of \clj{[Int -> Int]}''.
This can be answered by checking the symbolic closure
returns \clj{Int} when 
\clj{x} is type \clj{Int}---and so this subtyping case
delegates to checking if the symbolic closures inhabits the given type.
The subtype relationship is true if the check succeeds without type error,
otherwise it is false.

The correct ``contravariant subtyping left-of-the-arrow''
is naturally preserved.
In this case, the left-of-the-arrow check is ``\clj{Int} is a subtype of \clj{x}'s type'', and
annotating \clj{x} as \clj{Int} turns this statement into the reflexively true ``\clj{Int} is a subtype of \clj{Int}''.
At a glance, it may seem that we are wasting the benefits
of this contravariant rule---after all, it enables \clj{x} to be \emph{any} supertype of
\clj{Int}, such as \clj{Num} or even \clj{Any}.
However, it is in our interest to propagate the most precise parameter types
so then function bodies have the best chance to check without error.
Since symbolic closures are designed to support rechecking their bodies at different argument types,
a symbolic function can simply be rechecked with the less-precise types
when it comes time to broaden its domain.

This scheme extends to subtyping with arbitrarily-nested function types.
To demonstrate nesting to the right of an arrow,
the following code sums 1 with itself via
\clj{curried-app-int}, which accepts a curried
function of two arguments \clj{f} and a number \clj{x}, and 
provides \clj{x} as both arguments to \clj{f}.

% using subtyping to check symbolic closures.
\begin{lstlisting}[language=Clojure]
(ann curried-app-int [[Int -> [Int -> Int]] Int -> Int])
(defn curried-app-int [f x] ((f x) x))
...
(curried-app-int (fn [y] (fn [x] (+ x y))) 1)
\end{lstlisting}

The standard application rule will ensure 
``the symbolic closure of \clj{(fn [y] (fn [x] (+ x y)))}
is a subtype of
\clj{[Int -> [Int -> Int]]}'', which involves assuming
\clj{y : Int} and then checking the \emph{code} \clj{(fn [x] (+ x y))}
at type \clj{[Int -> Int]}---which just uses the standard
function rule.

To demonstrate nesting to the left of an arrow,
\clj{app-inc} again computes \clj{(inc 1)}
in an even more convoluted way with \clj{app-inc}---by accepting a function
\clj{f} that it passes both \clj{inc} and its second argument to.

% using subtyping to check symbolic closures.
\begin{lstlisting}[language=Clojure]
(ann app-inc [[[Int -> Int] Int -> Int] Int -> Int])
(defn app-inc [f x] (f inc x))
...
(app-inc (fn [g y] (g y)) 1)
\end{lstlisting}

Importantly, \clj{app-inc}'s first argument has a function
type to the left on an arrow, in particular \clj{[Int -> Int]}.
Under these conditions, subtyping asserts ``the symbolic
closure \clj{(fn [g y] (g y))} is a subtype of \clj{[[Int -> Int] Int -> Int]}''
by assuming \clj{g : [Int -> Int]} and \clj{y : Int} and
verifying that \clj{(g y)} checks as \clj{Int}---which is almost immediate by
the standard application rule.

We leverage some syntactic restrictions
to avoid the need for further subtyping cases for symbolic closures.
First, symbolic closures cannot be annotated by the programmer,
and can only be introduced by the ``unannotated function'' typing rule.
Second (as discussed in \secref{analyzer:extensibility:side-effects}),
top-level variables are not allowed to inherit the types of their initial
values, and must be explicitly annotated.
These restrictions ensure symbolic closures both cannot occur to the
left of an arrow type, and 
cannot propagate beyond the top-level form it was defined in.
%This stretches the metaphor of ``local'' type inference
%beyond just a single tree walk using
%bidirectional propagation,

\subsection{Performance and error messages}

% FIXME need to be more precise about "undecidable". What problem are
% we deciding? See Wells '94 for some details. I think so far I
% mean "type checking always terminates (conservatively)"

While useful, allowing the type system to perform beta-reduction
requires careful planning: type checking time is now proportional 
to the running time of the program!
Unsurprisingly, this makes type checking with a naive implementation of symbolic
closures undecidable.
Without intervention,
the next program (an infinite loop using the y-combinator that computes \clj{(inc (inc (inc ...)))})
would send the \emph{type system} into an infinite loop.

%TODO much simpler example: ((fn [x] (x x)) (fn [x] (x x)))

% infinite loops
\begin{lstlisting}[language=Clojure]
(let [Y (fn [f]
          ((fn [g] (fn [x] (f (g g) x)))
           (fn [g] (fn [x] (f (g g) x)))))]
  (let [compute (Y (fn [f x] (inc (f x))))]
    (compute 1)))
\end{lstlisting}

To prevent such loops, we limit the number of symbolic reductions
done at type-checking time.
As a conservative solution to the halting
problem, this limit will prematurely halt some programs that would
otherwise fully reduce in a finite number of steps.
For example, if we set the reduction limit to 5\ in
the following code,
during the 6th reduction of \clj{f} the type system will
throw an error.

% premature halting
\begin{lstlisting}[language=Clojure]
(let [f (fn [x] x)]
  (f (f (f (f (f (f 1)))))))
\end{lstlisting}

In simple cases like these, the error message 
can guide the user to fixing the error.
For example, the type system would suggest 
annotating \clj{f} as \clj{[Int -> Int]} (by collecting
argument and return types as the program is reduced),
which would cause the program to check successfully
under the same conditions.
For cases with more heterogeneous argument and return types---like the y-combinator---the 
error message would just note which function caused
the reduction quota to be depleted.

As Wells~\cite{wells1994typability} remarks,
stopgap measures such as this to circumvent undecidable
type inference algorithms negatively affect
program portability.
For example, a different reduction limit may cause
a program to fail to type check that otherwise type checked
in a previous version.
We hope to learn reasonable defaults for the reduction limit
by experience.

%Note that using the type checker to decide subtyping
%has unfortunate implications for 
%the aforementioned annotation suggestions
%for reduction-limit error messages.
%An ``obviously-failing'' subtyping check might trigger a
%check for irrelevant arguments, and then provide them to the user.
%A curious aside: if symbolic closures are identified just by their code and definition
%type environments, suggestions may also be merged for functions with
%identical code and scope.

% TODO performance
% - undecidable 
%   - heuristics needed to halt search
%   - type checking time proportional to running time of program
% - for finitely running programs:
%   - degenerate case checking time complexity becomes at least exponential time in the size of the program because we can recheck a function
%     body multiple times, and a symbolic closure can be duplicated
%
% eg. (let [pair (fn [f g] (f (g) (g)))] (pair (fn [x y] (+ x y)) (fn [] 1)))
% - (fn [] 1) is checked twice
%   - can "stack" these recheckings, worst case is infinite
% - Damas-Milner algorithm checks a function definition once to determine its principle type scheme
%   - exponential time & space
%     - because principle type schemes can become very large
%     - also exponential time to print a type
%     - symbolic closures are also exponential time to print a type (naively)
%       since they can be duplicated
%       - I think these are similar reasons to Milner's algorithm

% do we need a story for runtime casting from Any to [Int -> Int]?
%\begin{lstlisting}[language=Clojure]
%(ann dynapp-int [Any Int -> Int])
%...
%(dynapp-int (fn [x] (inc x)) 1)
%\end{lstlisting}

% no idea what to do with negation function types 
%\begin{lstlisting}[language=Clojure]
%(ann app-int [(U [Int -> Int] (I Any (Not [Int -> Int]))) Int -> Int])
%...
%(app-int (fn [x] (inc x)) 1)
%\end{lstlisting}


\section{Formal model}
\label{symbolic:section:formal-model}

We formalize a restriction of symbolic closure types by defining an explicitly typed internal language,
providing an external languages that can omit annotations,
and formulating a type inference algorithm based on symbolic closures to recover omitted annotations.
This approach is similar to Local Type Inference~\cite{PierceLTI}
and Colored Local Type Inference~\cite{coloredlti01},
except where they utilize bidirectional type propagation to locally determine function parameter types,
we instead use symbolic closures to propagate type information
(since we have a synthesis rule for functions).

\subsection{Internal Language}

\begin{figure}[h]
$$
\begin{array}{lrll}
  \ltiE{}, \ltiF{} &::=& \ltivar{} \alt
                         \ltifuntparamargtype{\ova{\ltitvar{}}}
                                             {\ltivar{}}
                                             {\ltiT{}}
                                             {\ltiE{}}
                         \alt
                         \ltiappinst{\ltiF{}}{\ova{\ltiR{}}}{\ltiE{}} \alt
                         \ltisel{\ltiE{}}{\ltivar{}} \alt
                         \ltiRec{\ova{\ltivar{} = \ltiE{}}}
                      &\mbox{Terms} \\
  \ltiT{}, \ltiS{}, \ltiR{} &::=& \ltitvar{} 
                         \alt
                         \ltiTop
                         \alt
                         \ltiBot
                         \alt \ltiPolyFn{\ltiT{}}{\ova{\ltitvar{}}}{\ltiT{}}
                         \alt
                         \ltiRec{\ova{\hastype{\ltivar{}}{\ltiT{}}}}
                      &\mbox{Types} \\
  \ltiEnv{} &::=& \ltiEmptyEnv \alt
                  \ltiEnvConcat{\ltiEnv{}}{\hastype{\ltivar{}}{\ltiT{}}} \alt
                  \ltiEnvConcat{\ltiEnv{}}{\ltitvar{}}
                      &\mbox{Type Environments} \\
\end{array}
$$
\caption{Internal Language Syntax}
\label{symbolic:figure:internal-language}
\end{figure}

Our internal language is based on System \ltiFsub extended with records, and is functionally identical
to that used to model Colored Local Type Inference~\cite{coloredlti01}, except
our lambda terms require full (return) type annotations.
\figref{symbolic:figure:internal-language} shows the syntax
for the internal language.
Terms \ltiE{} and \ltiF{} range over 
variables \ltivar{},
explicitly typed polymorphic functions
                         \ltifuntparamargtype{\ova{\ltitvar{}}}
                                             {\ltivar{}}
                                             {\ltiT{}}
                                             {\ltiE{}},
function application
with explicit type arguments
\ltiappinst{\ltiF{}}{\ova{\ltiR{}}}{\ltiE{}},
record selectors
\ltisel{\ltiE{}}{\ltivar{}},
and record constructors
\ltiRec{\ova{\ltivar{} = \ltiE{}}}.
Types \ltiT{}, \ltiS{}, and \ltiR{} are 
type variables \ltitvar{},
top type \ltiTop,
bottom type \ltiBot,
polymorphic function types \ltiPolyFn{\ltiT{}}{\ova{\ltitvar{}}}{\ltiS{}}
(where bound type variables are enumerated over the arrow),
and
record types \ltiRec{\ova{\hastype{\ltivar{}}{\ltiT{}}}}.
Type environments \ltiEnv{}
consist of 
the empty environment
\ltiEmptyEnv,
concatenation of variable typings
``\ltiEnvConcat{\ltiEnv{}}{\hastype{\ltivar{}}{\ltiT{}}}'',
and 
concatenation of type variables
``\ltiEnvConcat{\ltiEnv{}}{\ltitvar{}}''.

We assume different term and type variables are distinct,
and treat terms and types that are equal up to alpha-renaming as equivalent.
Record terms and types have unordered fields.
We treat primitive types (like \sml{String}) as free type variables.

\begin{figure}
  \begin{mathpar}

    \boxed{
    \infer[]
    {
      \ltitjudgementNoElab{\ltiEnv{}}{\ltiE{}}{\ltiT{}}
      \\\\
      \text{\ltiE{} is of type \ltiT{}
      }
      \\\\
      \text{ in context \ltiEnv{}.}
                 }
                 {}
                 }

    \infer [\ltiIVar]
    {}
    {
    \ltitjudgementNoElab
                    {\ltiEnv{}}
                    {\ltivar{}}
                    {\ltiEnvLookup{\ltiEnv{}}{\ltivar{}}}
                 }

    \infer [\ltiISel]
    {
    \ltitjudgementNoElab{\ltiEnv{}}
                  {\ltiE{}}
                  {\ltiRec{\hastype{\ltivar{1}}{\ltiT{1}}, ..., \hastype{\ltivar{i}}{\ltiT{i}} , ..., \hastype{\ltivar{n}}{\ltiT{n}}}}
    }
    {
    \ltitjudgementNoElab{\ltiEnv{}}
                  {\ltisel{\ltiE{}}{\ltivar{i}}}
                  {\ltiT{i}}
    }

    \infer [\ltiISelBot]
    {
    \ltitjudgementNoElab{\ltiEnv{}}
                     {\ltiE{}}
                     {\ltiBot}
    }
    {
    \ltitjudgementNoElab{\ltiEnv{}}
                  {\ltisel{\ltiE{}}{\ltivar{i}}}
                  {\ltiBot}
    }

    \infer [\ltiIAbs]
    { 
    \ltitjudgementNoElab{\ltiEnvConcat{\ltiEnv{}}
                                {\ltiEnvConcat{\ova{\ltitvar{}}}
                                              {\hastype{\ltivar{}}{\ltiT{}}}}}
                  {\ltiE{}}
                  {\ltiS{}}
    }
    {
    \ltitjudgementNoElab{\ltiEnv{}}
                  {\ltifuntparamargtype{\ova{\ltitvar{}}}
                                   {\ltivar{}}
                                   {\ltiT{}}
                                   {\ltiE{}}}
                  {\ltiPoly{\ova{\ltitvar{}}}{\ltiFn{\ltiT{}}{\ltiS{}}}}
                 }
                 \ \ \ \
%
    \infer [\ltiIAppInst]
    {
    \ltitjudgementNoElab{\ltiEnv{}}
                  {\ltiF{}}
                  {\ltiPolyFn{\ltiT{}}{\ova{\ltitvar{}}}{\ltiS{}}}
                    \\\\
    \ltitjudgementNoElab{\ltiEnv{}}
                  {\ltiE{}}
                  {\ltiTp{}}
                  \\
                  \ltiisubtype{\ltiEnv{}}{\ltiTp{}}{\ltireplace{\ova{\ltiR{}}}{\ova{\ltitvar{}}}{\ltiT{}}}
    }
    {
      \ltitjudgementNoElab{\ltiEnv{}}
                    {\ltiappinst{\ltiF{}}
                                {\ova{\ltiR{}}}
                                {\ltiE{}}}
                    {\ltireplace{\ova{\ltiR{}}}{\ova{\ltitvar{}}}{\ltiS{}}}
    }
                 \ \ \ \
%
    \infer [\ltiIAppInstBot]
    {
    \ltitjudgementNoElab{\ltiEnv{}}
                  {\ltiF{}}
                  {\ltiBot}
                  \\\\
    \ltitjudgementNoElab{\ltiEnv{}}
                  {\ltiE{}}
                  {\ltiS{}}
    }
    {
    \ltitjudgementNoElab{\ltiEnv{}}
                  {\ltiappinst{\ltiF{}}{\ova{\ltiR{}}}{\ltiE{}}}
                  {\ltiBot{}}
    }
                 \ \ \ \
%
    \infer [\ltiIRec]
    {
    \overrightarrow{
    \ltitjudgementNoElab{\ltiEnv{}}
                  {\ltiE{}}
                  {\ltiT{}}
                  }
    }
    {
    \ltitjudgementNoElab{\ltiEnv{}}
                  {\ltiRec{\ova{\ltivar{} = \ltiE{}}}}
                  {\ltiRec{\ova{\hastype{\ltivar{}}{\ltiT{}}}}}
    }

  \end{mathpar}
  \caption{Internal language type system
  }
  \label{symbolic:figure:internal-language-type-system}
\end{figure}

\figref{symbolic:figure:internal-language-type-system}
presents the type system for the internal language
\ltitjudgementNoElab{\ltiEnv{}}{\ltiE{}}{\ltiT{}},
pronounced ``\ltiE{} is of type \ltiT{} in context \ltiEnv{}.''
\ltiIVar is the normal variable lookup rule.
\ltiISel selects a field already present in a record.
\ltiISelBot allows selecting fields from \ltiBot.
\ltiIAbs checks a function definition at its annotated type.
\ltiIAppInst checks a function application with explicit type arguments.
\ltiIAppInstBot allows applying operators of type \ltiBot.
\ltiIRec checks record constructors.

\begin{figure}
  \begin{mathpar}
    \boxed{
    \infer[]
    {}
    {
      \ltiisubtype{\ltiEnv{}}{\ltiT{}}{\ltiS{}}
      \\\\
      \text{\ltiT{} is a subtype of \ltiS{}.
      }
                 }
                 }

    \infer [\ltiSTVar]
    {}
    {
    \ltiisubtype{\ltiEnv{}}{\ltitvar{}}{\ltitvar{}}
    }

    \infer [\ltiSTop]
    {}
    { \ltiisubtype{\ltiEnv{}}{\ltiT{}}{\ltiTop}}

    \infer [\ltiSBot]
    {}
    { \ltiisubtype{\ltiEnv{}}{\ltiBot}{\ltiT{}}}

    \infer [\ltiSRec]
    {
    \overrightarrow{\ltiisubtype{\ltiEnv{}}{\ltiT{}}{\ltiS{}}}
    }
    {
    \ltiisubtype{\ltiEnv{}}
                    {\ltiRec{\ova{\hastype{\ltivar{}}{\ltiT{}}},
                             \ova{\hastype{\ltivarp{}}{\ltiTp{}}}}}
                    {\ltiRec{\ova{\hastype{\ltivar{}}{\ltiS{}}}}}
    }

    \infer [\ltiSFn]
    {
          \ltiisubtype{\ltiEnv{}}{\ltiS{}}{\ltiSp{}}
          \\
          \ltiisubtype{\ltiEnv{}}{\ltiT{}}{\ltiTp{}}
          }
    {\ltiisubtype {\ltiEnv{}}
                     {\ltiPolyFn{\ltiSp{}}{\ova{\ltitvar{}}}{\ltiT{}}}
                     {\ltiPolyFn{\ltiS{}}{\ova{\ltitvar{}}}{\ltiTp{}}}
                     }

  \end{mathpar}
  \caption{Internal language subtyping
  }
  \label{symbolic:figure:internal-language-subtyping}
\end{figure}

\figref{symbolic:figure:internal-language-subtyping}
presents the subtyping for the internal language
\ltiisubtype{\ltiEnv{}}{\ltiT{}}{\ltiS{}}, pronounced
``\ltiT{} is a subtype of \ltiS{}.''
\ltiSTVar says type variables are subtypes of themselves.
\ltiSTop and \ltiSBot establish \ltiTop and \ltiBot as maximal and
minimal types.
\ltiSRec says record types may forget or upcast their fields.
\ltiSFn relates types contravariantly to the left of an arrow
and covariantly to the right.

\subsection{External language}

\begin{figure}
$$
\begin{array}{lrll}
  \ltiE{}, \ltiF{} &::=& ... \alt \ltiufun{\ltivar{}}{\ltiE{}}
                         \alt \ltiapp{\ltiF{}}{\ltiE{}}
                      &\mbox{Terms}
\end{array}
$$
\caption{External Language Syntax
  (extends \figref{symbolic:figure:internal-language})
  }
\label{symbolic:figure:external-language-syntax}
\end{figure}

The syntax for the external language
\figref{symbolic:figure:external-language-syntax}
is a superset of the internal language, with unannotated functions 
\ltiufun{\ltivar{}}{\ltiE{}},
and ``lightweight'' applications with implicit type arguments
\ltiapp{\ltiF{}}{\ltiE{}}.

\begin{figure}
  \begin{mathpar}
    \infer [\ltiEAppInf]
    {
    \ltitjudgementNoElab{\ltiEnv{}}
                    {\ltiF{}}
                    {\ltiPolyFn{\ltiT{}}{\ova{\ltitvar{}}}{\ltiS{}}}
                    %{\ltiFp{}}
                    \\
    \ltitjudgementNoElab{\ltiEnv{}}
                    {\ltiE{}}
                    {\ltiSp{}}
                    %{\ltiEp{}}
                  \\
                       |\ova{\ltitvar{}}|>0
                  \\\\
                  \forall \ltiRp{}.
                    \left(
                    \begin{array}{lll}
                      \ltiisubtype{\ltiEnv{}}{\ltiSp{}}{\ltireplace{\ova{\ltiRp{}}}{\ova{\ltitvar{}}}{\ltiT{}}}
                      \text{ implies }
                      %\arcr
                      \ltiisubtype{\ltiEnv{}}{\ltireplace{\ova{\ltiR{}}}{\ova{\ltitvar{}}}{\ltiSp{}}}
                                   {\ltireplace{\ova{\ltiRp{}}}{\ova{\ltitvar{}}}{\ltiSp{}}}
                    \end{array}
                  \right)
    }
    {
    \ltitjudgementNoElab{\ltiEnv{}}
                    {\ltiapp{\ltiF{}}{\ltiE{}}}
                    {\ltireplace{\ova{\ltiR{}}}{\ova{\ltitvar{}}}{\ltiS{}}}
                    %{\ltiappinst{\ltiFp{}}
                    %            {\ova{\ltiR{}}}
                    %            {\ltiEp{}}}
    }

    \infer [\ltiEAppInfBot]
    {
    \ltitjudgementNoElab{\ltiEnv{}}
                  {\ltiF{}}
                  {\ltiBot}
                  \\\\
    \ltitjudgementNoElab{\ltiEnv{}}
                  {\ltiE{}}
                  {\ltiS{}}
    }
    {
    \ltitjudgementNoElab{\ltiEnv{}}
                  {\ltiapp{\ltiF{}}{\ltiE{}}}
                  {\ltiT{}}
    }

    \infer [\ltiEUAbs]
    { 
    \ltitjudgementNoElab{\ltiEnvConcat{\ltiEnv{}}
                                {\ltiEnvConcat{\ova{\ltitvar{}}}
                                              {\hastype{\ltivar{}}{\ltiT{}}}}}
                  {\ltiE{}}
                  {\ltiS{}}
                  \\\\
              \ova{\ltitvar{}} \cap \ltitv{\ltiE{}} = \varnothing
    }
    {
    \ltitjudgementNoElab{\ltiEnv{}}
                  {\ltiufun{\ltivar{}}{\ltiE{}}}
                  {\ltiPolyFn{\ltiT{}}{\ova{\ltitvar{}}}{\ltiS{}}}
                 }
  \end{mathpar}

  \caption{External Language Specification (extends 
  \figref{symbolic:figure:internal-language-type-system})
  }
  \label{symbolic:figure:external-language-declarative-type-system}
\end{figure}

The external language type system is (declaratively) specified in
\figref{symbolic:figure:external-language-declarative-type-system}
as a superset of the (algorithmic) internal language.
\ltiEAppInf says we must pick the most general type arguments when elaborating
an application without explicit type arguments.
This is identical to the corresponding rule in Local Type Inference.
\ltiEUAbs says that an untyped function must type check at the interface
chosen for its elaboration.
Since we also infer type parameters, the rule also requires the type variables chosen must
not capture type variables that occur free in the body of the function.
A similar rule is included in Colored Local Type Inference, except
colored types enforce that parameter types and type parameters only be inherited
from its surrounding context.
In constrast, our rule uses an oracle to synthesize both.
This is because our type inference algorithm based on symbolic closures
is not restricted to local reasoning.

\begin{figure}[h]
$$
\begin{array}{lrlll}
  \ltiFn{\ova{\ltiT{}}^n}{\ltiS{}} &\Leftrightarrow&
  \ltiFn{\ltiRec{\overrightarrowcaption{\hastype{\texttt{arg}i}{\ltiT{i}}}^{1 \leq i \leq n}}}{\ltiS{}}
  & \text{where } n \not= 1
                      &\mbox{Type abbreviations} \\
  \ltiufun{\ova{\ltivar{}}^n}{\ltiE{}} &\Leftrightarrow&
  \ltiufun{\ltivarp{}}{\ltireplaceoverrightarrowcaption{\ltisel{\ltivarp{}}{\texttt{arg}i}}{\ltivar{i}}{1 \leq i \leq n}{\ltiE{}}}
  & \text{where } n \not= 1, \ltivarp{} \not\in \ltifvLHS{\ltiE{}}
                      &\mbox{Term abbreviations}
  \\
  \ltifuntparamargtype{\ova{\ltitvar{}}}{\ova{\ltivar{}}^n}{\ova{\ltiT{}}^n}{\ltiE{}} &\Leftrightarrow&
  \ltifuntparamargtype{\ova{\ltitvar{}}}
                      {\ltivarp{}}
                      {\ltiRec{\overrightarrowcaption{\hastype{\texttt{arg}i}{\ltiT{i}}}^{1 \leq i \leq n}}}
                      {\\ && \ \ \ltireplaceoverrightarrowcaption{\ltisel{\ltivarp{}}{\texttt{arg}i}}{\ltivar{i}}{1 \leq i \leq n}{\ltiE{}}}
  & \text{where } n \not= 1, \ltivarp{} \not\in \ltifvLHS{\ltiE{}}
  \\
  \ltiapp{\ltiF{}}{\ova{\ltiE{}}^n} &\Leftrightarrow&
  \ltiapp{\ltiF{}}{\ltiRec{\overrightarrowcaption{\texttt{arg}i = \ltiE{i}}^{1 \leq i \leq n}}}
  & \text{where } n \not= 1
  \\
  \ltilet{\ova{\ltivar{}}}{\ova{\ltiE{}}}{\ltiF{}} &\Leftrightarrow& \ltiappParens{\ltiufun{\ova{\ltivar{}}}{\ltiF{}}}{\ova{\ltiE{}}}
  \\
  \ltifunargtype{\ova{\ltivar{}}}{\ova{\ltiT{}}}{\ltiE{}} &\Leftrightarrow&
  \ltifuntparamargtype{}{\ova{\ltivar{}}}{\ova{\ltiT{}}}{\ltiE{}}
  \\
  \ltifuntparamargrettype{\ova{\ltitvar{}}}{\ova{\ltivar{}}}{\ova{\ltiT{}}}{\ltiS{}}{\ltiE{}} &\Leftrightarrow&
  \ltifuntparamargtype{\ova{\ltitvar{}}}{\ova{\ltivar{}}}{\ova{\ltiT{}}}{\ltianncolon{\ltiE{}}{\ltiS{}}}
  \\
  \ltianncolon{\ltiE{}}{\ltiS{}} &\Leftrightarrow&
  \ltiappParens{\ltifunargtype{\ltivar{}}{\ltiS{}}{\ltivar{}}}{\ltiE{}}
  \\
  \ltifunargrettype{\ova{\ltivar{}}}{\ova{\ltiT{}}}{\ltiS{}}{\ltiE{}} &\Leftrightarrow&
  \ltifuntparamargrettype{}{\ova{\ltivar{}}}{\ova{\ltiT{}}}{\ltiS{}}{\ltiE{}}
\end{array}
$$
\caption{External Language Syntax abbreviations
  }
\label{symbolic:figure:external-language-syntax-abbreviations}
\end{figure}

We use several syntax abbreviations, enumerated in
\figref{symbolic:figure:external-language-syntax-abbreviations}.
The first four abbreviations use record terms and types to represent multi-parameter functions
and applications.
For example, the function type
\ltiFn{\ltiT{1},\ltiT{2}}{\ltiS{}}
stands for
\ltiFn{\ltiRec{\hastype{\texttt{arg1}}{\ltiT{1}}, {\hastype{\texttt{arg2}}{\ltiT{2}}}}}{\ltiS{}},
the function term
\ltiufun{\text{x}, \text{y}}{\ltiE{}}
stands for 
\ltiufun{\ltivarp{}}{\ltireplaceSingle{\ltireplaceentry{\ltisel{\ltivarp{}}{\texttt{arg1}}}{\text{x}},
                                       \ltireplaceentry{\ltisel{\ltivarp{}}{\texttt{arg2}}}{\text{y}}}
                                      {\ltiE{}}}
(where {\ltivarp{}} does not occur free in {\ltiE{}})
and the application term
\ltiapp{\ltiF{}}{\ltiE{1},\ltiE{2}}
stands for
  \ltiapp{\ltiF{}}{\ltiRec{\texttt{arg1} = \ltiE{1},{\texttt{arg2} = \ltiE{2}}}}.
Symbolic closures permit the lambda-encoding of let, so 
\ltilet{\ltivar{}}{\ltiE{}}{\ltiF{}}
stands for 
\ltiappParens{\ltiufun{\ltivar{}}{\ltiF{}}}{\ltiE{}}.
We allow the type parameters to be omitted from function terms if they are empty.
Type ascription
\ltianncolon{\ltiE{}}{\ltiS{}}
stands for
\ltiappParens{\ltifunargtype{\ltivar{}}{\ltiS{}}{\ltivar{}}}{\ltiE{}},
which we use to represent return types for functions
\ltifuntparamargrettype{\ltitvar{}}{\ltivar{}}{\ltiT{}}{\ltiS{}}{\ltiE{}}
as
\ltifuntparamargtype{\ltitvar{}}{\ltivar{}}{\ltiT{}}{\ltianncolon{\ltiE{}}{\ltiS{}}}.

\subsection{Type Inference Algorithm}

We now define a type inference algorithm based on symbolic closures
that recovers types from terms written in the external language.
First, we give the syntax for symbolic closures, then we describe
the organization of type inference, and finally fill in the missing details.

%In order to reliably elaborate to \ltiFsub, we model a restriction of symbolic closure types.
%
%First, the elaborated type of a symbolic closure is chosen greedily,
%when it is first symbolically executed, and each symbolic closure
%must be exercised at least once to elaborate its body.
%A polymorphic type may also be chosen, but the type arguments bound by
%the function must also be chosen at this time.
%This is for several reasons.
%Most obviously, we lack
%intersection types, and so have no natural way of enumerating more than one interface
%type for a function, with each interface
%corresponding to a symbolic execution.
%Adding intersection types would require further considerations, namely
%the machinery needed to annotate a function that is ascribed many interfaces is quite involved
%(e.g., branching types~\cite{wells2002branching},
%contextual subtyping~\cite{Dunfield2004Tridirectional},
%and ``lazy'' type substitutions~\cite{polyduce1})
%and obscures the (orthogonal) idea of symbolic closures, which we hope to present
%in its essense.
%We might also attempt to infer a single polymorphic function type that combines all interfaces,
%however that is a separate problem we have not attempted.
%
%Second, we restrict a symbolic closure to the type-variable scope in which it was defined.
%Relaxing this restriction raises questions about scoping also solved by
%contextual subtyping~\cite{Dunfield2004Tridirectional}.
%Alternatively, we could quantify over out-of-scope variables
%with a polymorphic type, but this would most likely require also
%inferring type arguments for arguments~\cite{polyduce1},
%which we do not cover here.
%
%Third, we disallow a symbolic closure type to be passed to themselves.
%We must fully erase symbolic closure types to elaborate to \ltiFsub,
%however we do not model the equi-recursive type binders that are the 
%natural encoding for such types. On the other hand, since it does not require
%recursive types to elaborate, symbolic closures
%be passed to \emph{other} symbolic closures, even if the former elaborates to a polymorphic
%type (since \ltiFsub is impredicative, i.e., does not restrict the places a polymorphic type
%may occur).
%
%Fourth, we add a global symbolic reduction limit, called ``fuel'', to make
%type inference decidable.
%Since a symbolic closure may be symbolically executed an unbounded
%number of times, this restriction is also useful for practical implementations
%using symbolic closures.
%
%As we will see, symbolic closures are useful even with these restrictions.

\begin{figure}
$$
\begin{array}{lrll}
  \ltiE{}, \ltiF{} &::=& ... \alt
                         \ltiufunelab{\ltiufunelabentry{\ltiClosureID{}}}
                                     {\ltivar{}}
                                     {\ltiE{}}
                      &\mbox{Terms} \\
  \ltiT{}, \ltiS{}, \ltiR{} &::=& ... \alt \ltiClosureWithStkID{\ltiEnv{}}{\ltiClosureID{}}{\ltiufun{\ltivar{}}{\ltiE{}}}
                      &\mbox{Types} \\
  \ltiClosureID{} &::=& \ltitvar{}
                      &\mbox{Symbolic Closure Identifiers} \\
  \ltiFuel{} &::=& \ltinat{}
                      &\mbox{Symbolic Reduction Fuel} \\
  \ltiClosureCache{} &::=& \ova{\ltiClosureCacheEntry
                                {\ltiClosureID{}}
                                {\ltiClosure{\ltiEnv{}}{\ltiE{}}}}
                      &\mbox{Elaboration Caches}
                      \\
  \ltiCombinedThreadedEnv{} &::=& \ltimakeCombinedThreadedEnv{\ltiFuel{}}{\ltiClosureCache{}}
                      &\mbox{Threaded Environments}
\end{array}
$$
\caption{Symbolic Closure Language (SCL) Syntax (extends \figref{symbolic:figure:external-language-syntax})}
\label{symbolic:figure:SC-language-syntax}
\end{figure}

The syntax for the Symbolic Closure Language (SCL)
is given in 
\figref{symbolic:figure:SC-language-syntax}, which
is a superset of the external language syntax.
We introduce a new term and type, both which act as
a sort of placeholder for an explicitly typed function term or type (respectively)
which will be filled in after type checking.
The term
\ltiufunelab{\ltiufunelabentry{\ltiClosureID{}}}
            {\ltivar{}}
            {\ltiE{}}
is a tagged function,
which says the unannotated function 
\ltiufun{\ltivar{}}{\ltiE{}}
was assigned the symbolic closure type identified by
\ltiClosureID{}.
A symbolic closure type
\ltiClosureWithStkID{\ltiEnv{}}{\ltiClosureID{}}{\ltiufun{\ltivar{}}{\ltiE{}}},
then,
says the unannotated function term \ltiufun{\ltivar{}}{\ltiE{}}
is closed under definition type context \ltiEnv{}, with identifier \ltiClosureID{}.
We say \ltiufun{\ltivar{}}{\ltiE{}} is \emph{closed} because all free type and term variables in \ltiufun{\ltivar{}}{\ltiE{}}
are bound by \ltiEnv{}.
In terms of the internal language,
a loose first-intuition of a symbolic closure type's meaning
is a function type
\ltiPolyFn{\ltiT{}}{\ova{\ltitvar{}}}{\ltiS{}}
where
\ltitjudgementNoElab{\ltiEnvConcat{\ltiEnv{}}
                    {\ltiEnvConcat{\ova{\ltitvar{}}}
                                  {\hastype{\ltivar{}}{\ltiT{}}}}}
                    {\ltiE{}}
                    {\ltiS{}}.
This analogy is inadequate because (in small part) \ltiEnv{}, \ltiT{}, \ltiE{}, and \ltiS{}
can contain SCL types and terms.

An important part of using SCL for type inference is making the meaning of
symbolic closure types explicit by replacing them with concrete internal types and terms.
To this end, the remaining syntax is in service to the bookkeeping necessary to decide
how to achieve this.
To prevent infinite loops when checking symbolic closures, we introduce
symbolic reduction fuel \ltiFuel{}, a natural number that represents
the remaining number of symbolic reductions allowed.
The elaboration of an unannotated function checked with symbolic closures could
be determined at any time, so an elaboration cache \ltiClosureCache{} is maintained 
that associates a symbolic closure \ltiClosureID{}
with its scoped elaboration \ltiClosure{\ltiEnv{}}{\ltiE{}}, which says
SCL term \ltiE{} is closed under SCL type environment \ltiEnv{}.
For convienience, we use threaded environments \ltiCombinedThreadedEnv{} to stand for
the pair \ltimakeCombinedThreadedEnv{\ltiFuel{}}{\ltiClosureCache{}}.

We now describe the organization of type inference.
The typing judgment for SCL is written
    \ltitSstkjudgement{\ltimakeCombinedThreadedEnv{\ltiFuel{}}{\ltiClosureCache{}}}
                      {\ltiEnv{}}
                      {\ltiE{}}
                      {\ltiT{}}
                      {\ltimakeCombinedThreadedEnv{\ltiFuelp{}}{\ltiClosureCachep{}}}
                      {\ltiEp{}}
                      and
says with initial fuel \ltiFuel{} and elaboration cache \ltiClosureCache{},
external term \ltiE{} is of SCL type \ltiT{}
in SCL context \ltiEnv{}, elaborating to SCL term \ltiEp{} with 
updated fuel \ltiFuelp{} and elaboration cache \ltiClosureCachep{}.
It performs a depth-first traversal of the syntax tree.

\begin{figure}[h]
  \begin{mathpar}
  \infer[Infer]
  {
    \exists \ltiFuel{}.
     \ltitSstkjudgement{\ltimakeCombinedThreadedEnv{\ltiFuel{}}{\ltiEmptyClosureCache}}
                       {\ltiEnv{}}
                       {\ltiE{}}
                       {\ltiS{}}
                       {\ltimakeCombinedThreadedEnv{\ltiFuelp{}}{\ltiClosureCache{}}}
                       {\ltiEp{}}
                       \\
                  \ltielimClosT{\varnothing}{\ltiClosureCache{}}{\ltiS{}}{\ltiT{}}
                  \\
                  \ltielimClos{\ltiClosureCache{}}{\ltiEp{}}{\ltiF{}}
  }
  {
    \ltiinferTL{\ltiEnv{}}{\ltiE{}}{\ltiT{}}{\ltiF{}}
  }
  \end{mathpar}
\end{figure}

The top-level driver for type inference
\ltiinferTL{\ltiEnv{}}{\hastype{\ltiE{}}{\ltiT{}}}{\ltiF{}}, presented above,
says external term \ltiE{} has internal type \ltiT{}
in external environment \ltiEnv{}, with internal elaboration \ltiF{}.
It requires an initial fuel \ltiFuel{} to be provided to SCL, 
and then uses the output elaboration cache \ltiClosureCache{}
to erase SCL terms and types using the metafunctions \ltielimClossymbol
and \ltielimClosTsymbol.
Next, we present the SCL type system and subtyping
%(except for the SCL implementation of \ltiEAppInf, described in \chapref{chapter:symbolic:directed-lti}),
then provide definitions for the elaboration metafunctions.

\begin{figure}
  \begin{mathpar}
    \boxed
    {
    \infer[]
    {}
    {
    \ltitSstkjudgement{\ltiCombinedThreadedEnv{}}
                      {\ltiEnv{}}
                      {\ltiE{}}
                      {\ltiT{}}
                      {\ltiCombinedThreadedEnvp{}}
                      {\ltiEp{}}
                     \\\\
                     \text{Given symbolic closure environment \ltiCombinedThreadedEnv{}
                     and SCL context \ltiEnv{}, external term \ltiE{}
                     has SCL type \ltiT{}
                     }
                     \\\\
                     \text{in environment \ltiCombinedThreadedEnvp{},
                     with SCL elaboration \ltiEp{} (omitted when obvious from subderivations).
                     }
                     }
                     }

    \begin{array}{c}
    \infer [\ltiSCVar]
    {}
    {
    \ltitSstkjudgementNoElab{\ltiCombinedThreadedEnv{}}
                      {\ltiEnv{}}
                      {\ltivar{}}
                      {\ltiEnvLookup{\ltiEnv{}}{\ltivar{}}}
                      {\ltiCombinedThreadedEnv{}}
                      {\ltivar{}}
                 }
\ \ \ 
    \infer [\ltiSCRec]
    {
    \overrightarrowcaption{
    \ltitSstkjudgementNoElab{\ltiCombinedThreadedEnv{i-1}}
                      {\ltiEnv{}}
                      {\ltiF{i}}
                      {\ltiT{i}}
                      {\ltiCombinedThreadedEnv{i}}
                      {\ltiFp{i}}
                      }^{ 1 \leq i \leq n}
    }
    {
    \ltitSstkjudgementNoElab{\ltiCombinedThreadedEnv{0}}
                      {\ltiEnv{}}
                      {\ltiRec{\ova{\ltivar{} = \ltiF{}}^n}}
                      {\ltiRec{\ova{\hastype{\ltivar{}}{\ltiT{}}}^n}}
                      {\ltiCombinedThreadedEnv{n}}
                      {\ltiRec{\ova{\ltivar{} = \ltiFp{}}^n}}
    }

                 \\\\
    \infer [\ltiSCSel]
    {
    \ltitSstkjudgementNoElab{\ltiCombinedThreadedEnv{}}
                      {\ltiEnv{}}
                      {\ltiF{}}
                      {\ltiRec{\hastype{\ltivar{1}}{\ltiT{1}},..., \hastype{\ltivar{i}}{\ltiT{i}},..., \hastype{\ltivar{n}}{\ltiT{n}}}}
                      {\ltiCombinedThreadedEnvp{}}
                      {\ltiFp{}}
    }
    {
    \ltitSstkjudgementNoElab{\ltiCombinedThreadedEnv{}}
                      {\ltiEnv{}}
                      {\ltisel{\ltiF{}}{\ltivar{i}}}
                      {\ltiT{i}}
                      {\ltiCombinedThreadedEnvp{}}
                      {\ltisel{\ltiFp{}}{\ltivar{i}}}
    }
    \end{array}

    \begin{array}{ll}
    \infer [\ltiSCAbs]
    {
    \left(
    \begin{array}{llll}
      \text{$|\ova{\ltitvar{}}|>0$ implies \ltiEnv{} and \ltiS{}}
                     \arcr
                     \text{contain no symbolic closures}
    \end{array}
    \right)
                     \\\\
     \ltitSstkjudgementNoElab{\ltiCombinedThreadedEnv{}}
                    {\ltiEnvConcat{\ltiEnv{}}
                                   {\ltiEnvConcat{\ova{\ltitvar{}}}
                                                 {\hastype{\ltivar{}}
                                                          {\ltiT{}}}}}
                     {\ltiE{}}
                     {\ltiS{}}
                     {\ltiCombinedThreadedEnvp{}}
                     {\ltiEp{}}
    }
    {
    \ltitSstkjudgementNoElab{\ltiCombinedThreadedEnv{}}
                    {\ltiEnv{}}
                    {\ltifuntparamargtype{\ova{\ltitvar{}}}
                                           {\ltivar{}}
                                           {\ltiT{}}
                                           {\ltiE{}}}
                    {\ltiPolyFn{\ltiT{}}{\ova{\ltitvar{}}}{\ltiS{}}}
                    {\ltiCombinedThreadedEnvp{}}
                    {\ltifuntparaminterface{\ova{\ltitvar{}}}
                                           {\ltiFn{\ltiT{}}{\ltiS{}}}
                                           {\ltivar{}}
                                           {\ltiEp{}}}
                 }
    \end{array}

    \infer [\ltiSCAppInst]
    {
    \ltitSstkjudgementNoElab{\ltiCombinedThreadedEnv{1}}
                      {\ltiEnv{}}
                      {\ltiF{}}
                      {\ltiPolyFn{\ltiT{}}{\ova{\ltitvar{}}}{\ltiS{}}}
                      {\ltiCombinedThreadedEnv{2}}
                      {\ltiFp{}}
                  \\\\
    \ltitSstkjudgementNoElab{\ltiCombinedThreadedEnv{2}}
                      {\ltiEnv{}}
                      {\ltiE{}}
                      {\ltiTp{}}
                      {\ltiCombinedThreadedEnv{3}}
                      {\ltiEp{}}
    \\\\
                       \ltiSsubtype{\ltiCombinedThreadedEnv{3}}
                                   {\ltiEnv{}}
                                   {\ltiTp{}}
                                   {\ltireplace{\ova{\ltiR{}}}{\ova{\ltitvar{}}}{\ltiT{}}}
                                   {\ltiCombinedThreadedEnv{4}}
    }
    {
    \ltitSstkjudgementNoElab{\ltiCombinedThreadedEnv{1}}
                      {\ltiEnv{}}
                      {\ltiappinst{\ltiF{}}
                                  {\ova{\ltiR{}}}
                                  {\ltiE{}}}
                      {\ltireplace{\ova{\ltiR{}}}{\ova{\ltitvar{}}}{\ltiS{}}}
                      {\ltiCombinedThreadedEnv{4}}
                      {\ltiappinst{\ltiFp{}}
                                  {\ova{\ltiR{}}}
                                  {\ltiEp{}}}
    }
                 \ \ \ \ \ 
                 %
    \infer [\ltiSCAppInstBot]
    {
    \ltitSstkjudgementNoElab{\ltiCombinedThreadedEnv{}}
                      {\ltiEnv{}}
                      {\ltiF{}}
                      {\ltiBot}
                      {\ltiCombinedThreadedEnvpp{}}
                      {\ltiFp{}}
                  \\\\
    \ltitSstkjudgementNoElab{\ltiCombinedThreadedEnvpp{}}
                      {\ltiEnv{}}
                      {\ltiE{}}
                      {\ltiS{}}
                      {\ltiCombinedThreadedEnvp{}}
                      {\ltiEp{}}
    }
    {
    \ltitSstkjudgementNoElab{\ltiCombinedThreadedEnv{}}
                      {\ltiEnv{}}
                      {\ltiappinst{\ltiF{}}
                                  {\ova{\ltiR{}}}
                                  {\ltiE{}}}
                      {\ltiT{}}
                      {\ltiCombinedThreadedEnvp{}}
                      {\ltiappinst{\ltiFp{}}
                                  {\ova{\ltiR{}}}
                                  {\ltiEp{}}}
    }
                 \ \ \ \ \ 
                 %
    \infer [\ltiSCSelBot]
    {
    \ltitSstkjudgementNoElab{\ltiCombinedThreadedEnv{}}
                      {\ltiEnv{}}
                      {\ltiF{}}
                      {\ltiBot}
                      {\ltiCombinedThreadedEnvp{}}
                      {\ltiFp{}}
    }
    {
    \ltitSstkjudgementNoElab{\ltiCombinedThreadedEnv{}}
                      {\ltiEnv{}}
                      {\ltisel{\ltiF{}}{\ltivar{}}}
                      {\ltiBot}
                      {\ltiCombinedThreadedEnvp{}}
                      {\ltisel{\ltiFp{}}{\ltivar{}}}
    }
                 \ \ \ \ \ 
                 %
    \infer [\ltiSCAppInfBot]
    {
    \ltitSstkjudgementNoElab{\ltiCombinedThreadedEnv{}}
                      {\ltiEnv{}}
                      {\ltiF{}}
                      {\ltiBot}
                      {\ltiCombinedThreadedEnvpp{}}
                      {\ltiFp{}}
                  \\\\
    \ltitSstkjudgementNoElab{\ltiCombinedThreadedEnvpp{}}
                      {\ltiEnv{}}
                      {\ltiE{}}
                      {\ltiS{}}
                      {\ltiCombinedThreadedEnvp{}}
                      {\ltiEp{}}
    }
    {
    \ltitSstkjudgementNoElab{\ltiCombinedThreadedEnv{}}
                      {\ltiEnv{}}
                      {\ltiapp{\ltiF{}}{\ltiE{}}}
                      {\ltiBot}
                      {\ltiCombinedThreadedEnvp{}}
                      {\ltiappinst{\ltiFp{}}
                                  {}
                                  {\ltiEp{}}}
    }

    \infer[\ltiSCAppInfPT]
    {
    \ltitSstkjudgementNoElab{\ltiCombinedThreadedEnv{1}}
                      {\ltiEnv{}}
                      {\ltiF{}}
                      {\ltiPoly{\ova{\ltitvar{}}}
                               {\ltiFn{\ltiT{}}{\ltiS{}}}}
                      {\ltiCombinedThreadedEnv{2}}
                      {\ltiFp{}}
                  \\
    \ltitSstkjudgementNoElab{\ltiCombinedThreadedEnv{2}}
                      {\ltiEnv{}}
                      {\ltiE{}}
                      {\ltiTp{}}
                      {\ltiCombinedThreadedEnv{3}}
                      {\ltiEp{}}
                      \\
      \ltiT{}, \ltiS{}, \ltiTp{}
                     \text{ contain no symbolic closures}
                  \\
                       |\ova{\ltitvar{}}|>0
           \\
           \ltigenconstraint{\varnothing}{\ova{\ltitvar{}}}{\ltiTp{}}{\ltiT{}}{\ltiC{}}
           \\
           \ltiSubst{\ltiC{}}{\ltiFn{\ltiT{}}{\ltiS{}}}{\ltisubst{}}
    }
    {
    \ltitSstkjudgementNoElab{\ltiCombinedThreadedEnv{1}}
                      {\ltiEnv{}}
                      {\ltiapp{\ltiF{}}{\ltiE{}}}
                      {\ltiApplySubst{\ltisubst{}}{\ltiS{}}}
                      {\ltiCombinedThreadedEnv{3}}
                      {\ltiappinst{\ltiFp{}}
                                  {\ova{\ltiApplySubst{\ltisubst{}}
                                                      {\ltitvar{}}}}
                                  {\ltiEp{}}}
    }


    \infer [\ltiSCUAbs]
    {
    \ltiCombinedThreadedEnv{} = \ltimakeCombinedThreadedEnv{\ltiFuel{}}{\ltiClosureCache{}}
    \\
    \ltiClosureID{} \not\in dom(\ltiClosureCache{})
    \\\\
    \ltiCombinedThreadedEnvp{}
    =
    \ltimakeCombinedThreadedEnv{\ltiFuel{}}
    {\ltimapsto{\ltiClosureCache{}}
               {\ltiClosureID{}}
               {\ltiClosure{\ltiEnv{}}{\ltiufun{\ltivar{}}{\ltiE{}}}}}
    }
    {
    \ltitSstkjudgement{\ltiCombinedThreadedEnv{}}
                      {\ltiEnv{}}
                      {\ltiufun{\ltivar{}}{\ltiE{}}}
                      {\ltiClosureWithStkID{\ltiEnv{}}
                                           {\ltiClosureID{}}
                                           {\ltiufun{\ltivar{}}{\ltiE{}}}}
                      {\ltiCombinedThreadedEnvp{}}
                      {\ltiufunelab{\ltiClosureID{}}
                                   {\ltivar{}}
                                   {\ltiE{}}}
                 }
    \ \ \ \ 
%
    \infer [\ltiSCAppInfClosure]
    {
    \ltitSstkjudgement{\ltiCombinedThreadedEnv{1}}
                      {\ltiEnv{}}
                      {\ltiF{}}
                      {\ltiClosureWithStkID{\ltiEnvp{}}
                                           {\ltiClosureID{}}
                                           {\ltiufun{\ltivar{}}{\ltiEp{}}}}
                      {\ltiCombinedThreadedEnv{2}}
                      {\ltiFp{}}
                  \\
    \ltitSstkjudgement{\ltiCombinedThreadedEnv{2}}
                      {\ltiEnv{}}
                      {\ltiE{}}
                      {\ltiT{}}
                      {\ltimakeCombinedThreadedEnv{\ltiFuel{3}}{\ltiClosureCache{3}}}
                      {\ltiEpp{}}
                  \\\\
    0 < \ltiFuel{3}
    \\
    \ltitSstkjudgement{\ltimakeCombinedThreadedEnv
                       {\ltiFuel{3}-1}
                       {\ltiClosureCache{3}}}
                      {\ltiEnvConcat{\ltiEnvp{}}{\hastype{\ltivar{}}{\ltiT{}}}}
                      {\ltiEp{}}
                      {\ltiS{}}
                      {\ltimakeCombinedThreadedEnv{\ltiFuel{4}}{\ltiClosureCache{4}}}
                      {\ltiFpp{}}
    }
    {
    \ltitSstkjudgement{\ltiCombinedThreadedEnv{1}}
                      {\ltiEnv{}}
                      {\ltiapp{\ltiF{}}{\ltiE{}}}
                      {\ltiS{}}
                      {\ltimakeCombinedThreadedEnv{\ltiFuel{4}}
                          {\ltiupdateClosureCacheSingleLHS{\ltiClosureCache{4}}
                                {\ltiClosureID{}}
                                {\ltifuntparamargrettype
                                 {}
                                 {\ltivar{}}
                                 {\ltiT{}}
                                 {\ltiS{}}
                                 {\ltiFpp{}}}}}
                      {\ltiappinst{\ltiFp{}}
                                  {}
                                  {\ltiEpp{}}}
    }
  \end{mathpar}
  \caption{Type inference algorithm% (\textsc{AppInf} omitted)
  }
  \label{symbolic:figure:SC-language-algorithmic-type-system}
\end{figure}

The SCL type system is given in \figref{symbolic:figure:SC-language-algorithmic-type-system}.
We abbreviate
    \ltitSstkjudgement{\ltimakeCombinedThreadedEnv{\ltiFuel{}}{\ltiClosureCache{}}}
                      {\ltiEnv{}}
                      {\ltiE{}}
                      {\ltiT{}}
                      {\ltimakeCombinedThreadedEnv{\ltiFuelp{}}{\ltiClosureCachep{}}}
                      {\ltiEp{}}
                      as
    \ltitSstkjudgement{\ltiCombinedThreadedEnv{}}
                      {\ltiEnv{}}
                      {\ltiE{}}
                      {\ltiT{}}
                      {\ltiCombinedThreadedEnvp{}}
                      {\ltiEp{}}
                      where
${\ltiCombinedThreadedEnv{}} = {\ltimakeCombinedThreadedEnv{\ltiFuel{}}{\ltiClosureCache{}}}$
and 
                      ${\ltiCombinedThreadedEnvp{}} = {\ltimakeCombinedThreadedEnv{\ltiFuelp{}}{\ltiClosureCachep{}}}$.
Further, we sometimes omit the elaborated term as
    \ltitSstkjudgementNoElab{\ltiCombinedThreadedEnv{}}
                      {\ltiEnv{}}
                      {\ltiE{}}
                      {\ltiT{}}
                      {\ltiCombinedThreadedEnvp{}}
                      {\ltiEp{}},
                      but only when
                      {\ltiEp{}}
                      can be obviously derived from subderivations.
The first seven rules correspond to the internal language type system,
straightfowardly extended with threaded environments.
The extra condition in \ltiSCAbs
helps ensure a symbolic closure type only reasons about type variables
in its definition scope.
The first rule for lightweight applications \ltiSCAppInfBot implements \ltiEAppInfBot.
The \ltiSCAppInfPT rule uses Pierce and Turner's type argument synthesis
algorithm~\cite{PierceLTI} off-the-shelf. Since it does not handle them,
we ensure that it cannot be fed a symbolic closure.

The remaining rules are more interesting.
The symbolic closure introduction rule
\ltiSCUAbs creates a symbolic closure type with a fresh identifier \ltiClosureID{}
for the unannotated function term \ltiufun{\ltivar{}}{\ltiE{}}.
The return type is symbolic closure type
                       \ltiClosureWithStkID{\ltiEnv{}}
                                           {\ltiClosureID{}}
                                           {\ltiufun{\ltivar{}}{\ltiE{}}},
which holds enough information to both check its body at some later time
and link its elaboration to the originating term.
The elaboration cache entry for \ltiClosureID{} is initialized 
with an unannotated function term, signifying that the body has yet to be type checked,
and is passed on as part of \ltiCombinedThreadedEnvp{}.
Finally, the elaboration
                      {\ltiufunelab{\ltiClosureID{}}
                                   {\ltivar{}}
                                   {\ltiE{}}}
tags the original term with its symbolic closure identifier \ltiClosureID{}.
Intuitively, this rule is sound because, from the type checker's perspective,
there is no witness (yet) to \ltiufun{\ltivar{}}{\ltiE{}}
being called, and so there is no opportunity to ``get stuck'' or
``go wrong''. 
The next rule handles one kind of witness to its invocation: application.

The application rule for symbolic closures
\ltiSCAppInfClosure
checks term \ltiapp{\ltiF{}}{\ltiE{}}
where \ltiF{} has symbolic closure type
                      {\ltiClosureWithStkID{\ltiEnvp{}}
                                           {\ltiClosureID{}}
                                           {\ltiufun{\ltivar{}}{\ltiEp{}}}}
                                           and 
\ltiE{} has type \ltiT{}.
As with the normal application rule,
we must ensure \ltiF{}'s domain is permissive enough to be
applied to terms of type \ltiT{}, which
we verify with a symbolic reduction.
After consuming fuel, the final premise checks \ltiClosureID{}'s
function body \ltiEp{} in its definition context \ltiEnvp{},
extended with \hastype{\ltivar{}}{\ltiT{}} (to account for
\ltiE{} of type \ltiT{} being passed as an argument), giving result type \ltiS{}
and elaboration \ltiFpp{}.
The type of the entire application is simply \ltiS{}, since
the condition in \ltiSCAbs (and in other rules, given later) ensure that
\ltiEnv{} and \ltiEnvp{} share the same type variable scope---there
is no opportunity for \ltiS{} to introduce an out-of-scope type variable.
We now pick the elaboration for \ltiClosureID{}
to be
{\ltifuntparamargrettype
                                 {}
                                 {\ltivar{}}
                                 {\ltiT{}}
                                 {\ltiS{}}
                                 {\ltiFpp{}}}
(an abbreviation for a function with return type \ltiS{}, defined in
\figref{symbolic:figure:external-language-syntax-abbreviations})
using the \ltiupdateClosureCacheSinglesymbol metafunction.
Since we choose \ltiClosureID{} to be monomorphic, it does
not require type arguments
and so
the entire application elaborates to
                      {\ltiappinst{\ltiFp{}}
                                  {}
                                  {\ltiEpp{}}}.
A symbolic closure's elaboration may be picked exactly once,
so it is an error for any past or future elaborations decide
on different numbers of type arguments.
We elide the almost-identical rule \ltiSCAppClosure for applying a symbolic closure
with explicit type arguments,
since it can only be of the form \ltiappinst{\ltiF{}}{}{\ltiE{}}.

To illustrate how \ltiSCUAbs and \ltiSCAppInfClosure interact,
we give simplified definitions for each, using judgments resembling the internal language
and a simplified symbolic closure
type \ltiClosure{\ltiEnv{}}{\ltiufun{\ltivar{}}{\ltiF{}}} that omits identifiers.
\ltiSimpUAbs immediately packages up a function with its environment in a symbolic closure type.
\ltiSimpAppInfClosure unpacks the symbolic closure, and checks its body in a context extended
with the argument's type.
Highlighting is used to convey how a symbolic closure is assembled, disassembled, and checked.

\begin{mathpar}
  \inferrule*[lab=\boxed{\ltiSimpUAbs}]
   {}
   {
   \ltitSstkjudgementNoElabCombined{\varnothing}
                                   {\colorbox{pink}{\ltiEnv{}}}
                                   {\colorbox{pink}{\ltiufun{\ltivar{}}{\ltiF{}}}}
                                   {\colorbox{pink}{
                                           {\ltiClosure{\ltiEnv{}}{\ltiufun{\ltivar{}}{\ltiF{}}}}}}
                                   {\ltiCombinedThreadedEnv{}}
                                   {\ltiufun{\ltivar{}}{\ltiF{}}}
   }

  \inferrule*[lab=\boxed{\ltiSimpAppInfClosure}]
  {
   \ltitSstkjudgementNoElabCombined{\varnothing}
                      {\ltiEnvp{}}
                      {\ltiE{1}}
                      {\colorbox{pink}{\ltiClosure{{\ltiEnv{}}}
                                           {\ltiufun{{\ltivar{}}}
                                                    {\ltiF{}}}}}
                      {\ltiClosureCache{}}
                      {\ltiufunelab{\ltiInferred{\text{c1}}}{\text{x}}{\text{x}}}
   \\
   \ltitSstkjudgementNoElabCombined{\ltiClosureCache{}}
                      {\ltiEnvp{}}
                      {\ltiE{2}}
                      {{\colorbox{pink}{\ltiS{}}}}
                      {\ltiClosureCache{}}
                      {\ltiEp{}}
   \\
    \ltitSstkjudgementNoElabCombined{\ltiClosureCache{}}
                      {\ltiEnvConcat{\colorbox{pink}{\ltiEnv{}}
                                     }
                                    {\hastypesmall{\colorbox{pink}{\ltivar{}}}
                                                  {{\colorbox{pink}{\ltiS{}}}}}}
                      {\colorbox{pink}{\ltiF{}}}
                      {\colorbox{pink}{\ltiT{}}}
                      {\ltiClosureCache{}}
                      {\ltiFp{}}
  }
  {
    \ltitSstkjudgementNoElabCombined{\ltiCombinedThreadedEnv{}}
                      {\ltiEnvp{}}
                      {\ltiapp{\ltiE{1}}
                              {\ltiE{2}}}
                      {\colorbox{pink}{\ltiT{}}}
                      {\ltiCombinedThreadedEnvp{}}
                      {\ltiapp{\ltiFp{}}{\ltiEp{}}}
  }
\end{mathpar}

Using these rules, we illustrate inferring the \emph{obfuscated} term
                      ``{\ltilet{\text{g}}{\ltiRec{\text{d}={{\ltiufun{\ltivar{}}{\ltiF{}}}}}}
                              {\ltiapp{\ltisel{\text{g}}{\text{d}}}{\ltiE{}}}}'',
which is like the \emph{unobfuscated} term ``{\ltiappParens{\ltiufun{\ltivar{}}{\ltiF{}}}{\ltiE{}}}'',
except the function is briefly stored in a let-bound record.
Bidirectional propagation of types is insufficient to recover the type of {\ltivar{}},
however symbolic closures provide the necessary indirection to successfully infer the term.
To simplify its presentation, the following derivation uses the standard explicit typing rule for let,
where the left premise checks the bound term and the right premise checks the body in an extended context---we use 
the lambda encoding of lets everywhere else (\figref{symbolic:figure:external-language-syntax-abbreviations}).
We also liberally elide \ltiEnv{} (when uninteresting)
and the subderivation trees of \ltiSimpAppInfClosure.

\begin{mathpar}
  \inferrule*[]
  {
  \inferrule*
  {
  \inferrule*[lab=\boxed{\ltiSimpUAbs}]
   {}
   {
   \ltitSstkjudgementNoElabCombined{\varnothing}
                                   {\colorbox{pink}{\ltiEnv{}}^{\tikz[overlay,remember picture] \node [] (c1) {};}}
                                   {{\tikz[overlay,remember picture] \node [] (c) {};}\colorbox{pink}{\ltiufun{\ltivar{}}{\ltiF{}}}}
                                   {\colorbox{pink}{
                                           {\ltiClosure{\text{}^{\tikz[overlay,remember picture] \node [] (d1) {};}\ltiEnv{}}
                                                                {{\tikz[overlay,remember picture] \node [] (d2) {};}\ltiufun{\ltivar{}}{\ltiF{}}}}}
                                           {\tikz[overlay,remember picture] \node [] (d) {};}}
                                   {\ltiCombinedThreadedEnv{}}
                                   {\ltiufun{\ltivar{}}{\ltiF{}}}
   }
   }
   {
   \ltitSstkjudgementJustType{\varnothing}
                                   {...}
                                   {\ltiRec{\text{d}={{\tikz[overlay,remember picture] \node [] (b) {};}\colorbox{pink}{\ltiufun{\ltivar{}}{\ltiF{}}}}}}
                                   {\ltiRec{\hastype{\text{d}}
                                           {\colorbox{pink}
                                            {\ltiClosure{\ltiEnv{}}
                                                                {\ltiufun{\ltivar{}}
                                                                         {\ltiF{}}}}
                                                                         ^
                                                                         {\tikz[overlay,remember picture] \node [] (e) {};}
                                                                         _
                                                                         {\tikz[overlay,remember picture] \node [] (e1) {};}}}}
                                   {\ltiCombinedThreadedEnv{}}
                                   {\ltiufun{\ltivar{}}{\ltiF{}}}
                                   }
   \\
  \inferrule*[lab={\boxed{\ltiSimpAppInfClosure}}]
  {
  \inferrule*[]
   {}
   {
   \ltitSstkjudgementJustType{\varnothing}
                      {...}
                      {{\ltisel{\text{g}}{\text{d}}}}
                      {\text{}_{\tikz[overlay,remember picture] \node [] (i) {};}
                      \colorbox{pink}{\ltiClosure{{\ltiEnv{}}^{\tikz[overlay,remember picture] \node [] (j) {};}}
                                           {\ltiufun{{\ltivar{}}_{\tikz[overlay,remember picture] \node [] (p) {};}}
                                                    {\ltiF{}^{\tikz[overlay,remember picture] \node [] (l) {};}}}}
                                                    {\tikz[overlay,remember picture] \node [] (h) {};}}
                      {\ltiClosureCache{}}
                      {\ltiufunelab{\ltiInferred{\text{c1}}}{\text{x}}{\text{x}}}
   }
   \\
   \inferrule*
   {}
   { \ltitSstkjudgementJustType{\ltiClosureCache{}}
                      {...}
                      {\ltiE{}}
                      {{\colorbox{pink}{\ltiS{}}}^{\tikz[overlay,remember picture] \node [] (n) {};}}
                      {\ltiClosureCache{}}
                      {\ltiEp{}}
   }
   \\
   \inferrule*[]
   {}
   {
    \ltitSstkjudgementNoElabCombined{\ltiClosureCache{}}
                      {\ltiEnvConcat{\text{}^{\tikz[overlay,remember picture] \node [] (k) {};}
                                     \colorbox{pink}{\ltiEnv{}}
                                     }
                                    {\hastypesmall{\colorbox{pink}{\ltivar{}}_{\tikz[overlay,remember picture] \node [] (q) {};}}
                                                  {{\colorbox{pink}{\ltiS{}}}^{\tikz[overlay,remember picture] \node [] (o) {};}}}}
                      {\text{}^{\tikz[overlay,remember picture] \node [] (m) {};}
                       \colorbox{pink}{\ltiF{}}}
                      {\colorbox{pink}{\ltiT{}}{\tikz[overlay,remember picture] \node [] (r) {};}}
                      {\ltiClosureCache{}}
                      {\ltisel{\text{g}}{\text{d}}}
                      }
  }
  {
    \ltitSstkjudgementNoElabCombined{\ltiCombinedThreadedEnv{}}
                      {%\ltiEnvConcat{...}
                                    {\hastype{\text{g}}{\ltiRec{\hastype{\text{d}}
                                                                        {{}^{\tikz[overlay,remember picture] \node [] (g) {};}
                                                                                _{\tikz[overlay,remember picture] \node [] (f) {};}
                                                                        \colorbox{pink}{\ltiClosure{\ltiEnv{}}
                                                                                             {\ltiufun{\ltivar{}}{\ltiF{}}}}}}}}}
                      {\ltiapp{
                               {\ltisel{\text{g}}{\text{d}}}}
                              {\ltiE{}}}
                      {\colorbox{pink}{\ltiT{}}{\tikz[overlay,remember picture] \node [] (s) {};}}
                      {\ltiCombinedThreadedEnvp{}}
                      {\ltiapp{\ltiFp{}}{\ltiEp{}}}
  }
  }
  {
    \ltitSstkjudgementJustType{\ltiCombinedThreadedEnv{}}
                      {...}
                      {\ltilet{\text{g}}{\ltiRec{\text{d}={{\tikz[overlay,remember picture] \node [] (a) {};}
                                                           \colorbox{pink}{\ltiufun{\ltivar{}}{\ltiF{}}}}}}
                              {\ltiapp{\ltisel{\text{g}}{\text{d}}}{\ltiE{}}}}
                      {\colorbox{pink}{\ltiT{}}{\tikz[overlay,remember picture] \node [] (t) {};}}
                      {\ltiCombinedThreadedEnvp{}}
                      {\ltiapp{\ltiFp{}}{\ltiEp{}}}
  }
\begin{tikzpicture}[remember picture, overlay,
                  text width = 2.5cm ]
  \coordinate (Start1) at (a);
  \coordinate (End1) at (b);
  \coordinate (Start2) at (b);
  \coordinate (End2) at (c);
  \coordinate (Start3) at (c);
  \coordinate (End3) at (d2);
  \coordinate (Start3p1) at (c1);
  \coordinate (End3p1) at (d1);
  \coordinate (Start4) at (d);
  \coordinate (End4) at (e);
  \coordinate (Start6) at (e1);
  \coordinate (End6) at (f);
  \coordinate (Start7) at (g);
  \coordinate (End7) at (h);
  \coordinate (Start8) at (j);
  \coordinate (End8) at (k);
  \coordinate (Start9) at (l);
  \coordinate (End9) at (m);
  \coordinate (Start10) at (n);
  \coordinate (End10) at (o);
  \coordinate (Start11) at (p);
  \coordinate (End11) at (q);
  \coordinate (Start12) at (r);
  \coordinate (End12) at (s);
  \coordinate (End13) at (t);
  \draw[pink,-](Start1.north) to [bend left] (End1.south);
  \draw[pink,-](Start2.north) to (End2.south);
  \draw[pink,-](Start3.north) to [bend right] (End3.south);
  \draw[pink,-](Start3p1.north) to [bend left] (End3p1.south);
  \draw[pink,-](Start4.north) to (End4.south);
  \draw[pink,-](Start6.north) to (End6.south);
  \draw[pink,-](Start7.north) to (End7.south);
  \draw[pink,-](Start8.north) to [bend left] (End8.south);
  \draw[pink,-](Start9.north) to [bend left] (End9.south);
  \draw[pink,-](Start10.north) to [bend left] (End10.south);
  \draw[pink,-](Start11.north) to [out=356, in=184] (End11.south);
  \draw[pink,-](Start12.north) to (End12.south);
  \draw[pink,-](End12.north) to (End13.south);
\end{tikzpicture} 
\end{mathpar}
% instructions for in/out https://cremeronline.com/LaTeX/minimaltikz.pdf

The derivation proceeds as follows.
First, the record term 
``{\ltiRec{\text{d}={\colorbox{pink}{\ltiufun{\ltivar{}}{\ltiF{}}}}}}''
is checked using the left subderivation, which ends with \ltiSimpUAbs.
There, to delay its checking, {\colorbox{pink}{\ltiufun{\ltivar{}}{\ltiF{}}}}
is packaged with its definition context {\colorbox{pink}{\ltiEnv{}}}
in the symbolic closure {\colorbox{pink}{\ltiClosure{\ltiEnv{}}{\ltiufun{\ltivar{}}{\ltiF{}}}}},
which finds itself in the resulting record type of the left subderivation
``{\ltiRec{\hastype{\text{d}}{\colorbox{pink}{\ltiClosure{\ltiEnv{}}{\ltiufun{\ltivar{}}{\ltiF{}}}}}}}''.
The right subderivation then binds this record type as $\text{g}$ in the type environment
to check the body ``{\ltiapp{\ltisel{\text{g}}{\text{d}}}{\ltiE{}}}''.
The first rule in the right subderivation is \ltiSimpAppInfClosure,
since the operator {\ltisel{\text{g}}{\text{d}}}
has our symbolic closure type
{\colorbox{pink}{\ltiClosure{\ltiEnv{}}{\ltiufun{\ltivar{}}{\ltiF{}}}}}
via a rule like \ltiISel.
Next, the argument \ltiE{} is checked as type {\colorbox{pink}{\ltiS{}}}.
Now, {\colorbox{pink}{\ltiEnv{}}}, {\colorbox{pink}{\ltivar{}}},
and {\colorbox{pink}{\ltiF{}}}
are extracted from the symbolic closure
and, along with {\colorbox{pink}{\ltiS{}}}, used to check (effectively)
our type-decorated unobfuscated term
``{\ltiappParens{\ltifunargtype{\ltivar{}}{\colorbox{pink}{\ltiS{}}}{\ltiF{}}}{\ltiE{}}}''
with the derivation
    ``\ltitSstkjudgementNoElabCombined{\ltiClosureCache{}}
                      {\ltiEnvConcat{\colorbox{pink}{\ltiEnv{}}
                                     }
                                    {\hastypesmall{\colorbox{pink}{\ltivar{}}}
                                                  {{\colorbox{pink}{\ltiS{}}}}}}
                      {\colorbox{pink}{\ltiF{}}}
                      {\colorbox{pink}{\ltiT{}}}
                      {\ltiClosureCache{}}
                      {\ltisel{\text{g}}{\text{d}}}.''
The result type {\colorbox{pink}{\ltiT{}}}
is then propagated to be the type of the entire derivation.

Local Type Inference and Colored Local Type Inference
would have failed to infer the obfuscated term exactly
at \ltiSimpUAbs, because it requires the type of \ltivar{}
to be known from its context at the point \ltiufun{\ltivar{}}{\ltiF{}}
is checked.
Indeed, they also fail to check our unobfuscated term ``{\ltiappParens{\ltiufun{\ltivar{}}{\ltiF{}}}{\ltiE{}}}''
for similar reasons.
In that case, their application rules require a type for the operator before checking its operand,
and so no information may be propagated from operand to operator.
This is fatal to checking the unobfuscated term, since they cannot synthesize the type of functions
from nothing (unlike \ltiSimpUAbs).


\begin{figure}
  \begin{mathpar}
    \boxed{
    \infer[]
    {}
    {\ltiSsubtype{\ltiCombinedThreadedEnv{}}
                 {\ltiEnv{}}
                 {\ltiS{}}
                 {\ltiT{}}
                 {\ltiCombinedThreadedEnvp{}}
                 \\\\
                 \text{
                 With symbolic closure environment \ltiCombinedThreadedEnv{},
                 \ltiS{} is a subtype of \ltiT{}
                 in updated environment \ltiCombinedThreadedEnvp{}.
                 }
    }
    }

    \infer [\ltiSCSTVar]
    {}
    {
     \ltiSsubtype{\ltiCombinedThreadedEnv{}}
                 {\ltiEnv{}}
                 {\ltitvar{}}
                 {\ltitvar{}}
                 {\ltiCombinedThreadedEnv{}}
    }

    \infer [\ltiSCSTop]
    {}
    { \ltiSsubtype{\ltiCombinedThreadedEnv{}}{\ltiEnv{}}{\ltiT{}}{\ltiTop}{\ltiCombinedThreadedEnv{}}}

    \infer [\ltiSCSBot]
    {}
    { \ltiSsubtype{\ltiCombinedThreadedEnv{}}{\ltiEnv{}}{\ltiBot}{\ltiT{}}{\ltiCombinedThreadedEnv{}}}

    \infer [\ltiSCSRec]
    {
    \overrightarrowcaption{\ltiSsubtype{\ltiCombinedThreadedEnv{i-1}}{\ltiEnv{}}
                                {\ltiT{}}
                                {\ltiS{}}
                                {\ltiCombinedThreadedEnv{i}}
                                }^{1 \leq i \leq n}
    }
    {
    \ltiSsubtype{\ltiCombinedThreadedEnv{0}}
                {\ltiEnv{}}
                {\ltiRec{\ova{\hastype{\ltivar{}}{\ltiT{}}}^n,
                         \ova{\hastype{\ltivarp{}}{\ltiTp{}}}}}
                {\ltiRec{\ova{\hastype{\ltivar{}}{\ltiS{}}}^n}}
                {\ltiCombinedThreadedEnv{n}}
    }

    % eg (IFn [Int -> Int] [Number -> Number]) <: [Nothing -> Any]
    \infer [\ltiSCSFn]
    {
    \left(
    \begin{array}{lll}
      |\ova{\ltitvar{}}|>0 \text{ implies \ltiT{}, \ltiTp{}, \ltiS{}, \ltiSp{}}
    \arcr
      \text{contain no symbolic closures}
    \end{array}
    \right)
    \\\\
    \ltiSsubtype{\ltiCombinedThreadedEnv{}}{\ltiEnv{}}{\ltiS{}}{\ltiSp{}}{\ltiCombinedThreadedEnvpp{}}
      \\\\
      \ltiSsubtype{\ltiCombinedThreadedEnvpp{}}{\ltiEnv{}}{\ltiT{}}{\ltiTp{}}{\ltiCombinedThreadedEnvp{}}
    }
    { \ltiSsubtype{\ltiCombinedThreadedEnv{}}{\ltiEnv{}}
                  {\ltiPolyFn{\ltiSp{}}{\ova{\ltitvar{}}}{\ltiT{}}}
                  {\ltiPolyFn{\ltiS{}}{\ova{\ltitvar{}}}{\ltiTp{}}}
                  {\ltiCombinedThreadedEnvp{}}
       }

    \infer [\ltiSCSClosure]
    {
    \ltiCombinedThreadedEnv{1} = {\ltimakeCombinedThreadedEnv{\ltiFuel{1}}{\ltiClosureCache{1}}}
    \\
    \ltitv{\ltiE{}} \cap \ova{\ltitvar{}} = \varnothing
    \\\\
    0 < \ltiFuel{1}
    \\
    \ltitSstkjudgement{\ltimakeCombinedThreadedEnv{\ltiFuel{1}-1}{\ltiClosureCache{1}}}
                      {\ltiEnvConcat{\ltiEnv{}}
                                    {\ltiEnvConcat{\ova{\ltitvar{}}}
                                                  {\hastype{\ltivar{}}{\ltiT{}}}}}
                      {\ltiE{}}
                      {\ltiSp{}}
                      {\ltimakeCombinedThreadedEnv{\ltiFuel{2}}{\ltiClosureCache{2}}}
                      {\ltiEp{}}
                      \\\\
    \ltiSsubtype{\ltimakeCombinedThreadedEnv
                 {\ltiFuel{2}}
                 {\ltiupdateClosureCacheSingleLHS{\ltiClosureCache{2}}
                                                  {\ltiClosureID{}}
                                                  {\ltifuntparamargrettype
                                                   {\ova{\ltitvar{}}}
                                                   {\ltivar{}}
                                                   {\ltiT{}}
                                                   {\ltiSp{}}
                                                   {\ltiEp{}}}}}
                {\ltiEnv{}}{\ltiSp{}}{\ltiS{}}
                {\ltiCombinedThreadedEnv{3}}
    }
    { \ltiSsubtype{\ltiCombinedThreadedEnv{1}}
                  {\ltiEnvp{}}
                  {\ltiClosureWithStkID{\ltiEnv{}}
                                       {\ltiClosureID{}}
                                       {\ltiufun{\ltivar{}}{\ltiE{}}}}
                  {\ltiPolyFn{\ltiT{}}{\ova{\ltitvar{}}}{\ltiS{}}}
                  {\ltiCombinedThreadedEnv{3}}
                  }
  \end{mathpar}

  \caption{Symbolic Closure Language Subtyping}
  \label{symbolic:figure:SC-language-subtype}
\end{figure}

Subtyping for SCL is given in
  \figref{symbolic:figure:SC-language-subtype}.
The judgment
\ltiSsubtype{\ltimakeCombinedThreadedEnv{\ltiFuel{}}{\ltiClosureCache{}}}
            {\ltiEnv{}}
            {\ltiS{}}
            {\ltiT{}}
            {\ltimakeCombinedThreadedEnv{\ltiFuelp{}}{\ltiClosureCachep{}}}
            says with fuel \ltiFuel{} and elaboration cache \ltiClosureCache{},
            \ltiS{} is a subtype of 
            {\ltiT{}}
            with updated fuel \ltiFuelp{} and elaboration cache \ltiClosureCachep{}.
Similar to the typing judgment, we abbreviate subtyping
with threaded environments as
\ltiSsubtype{\ltiCombinedThreadedEnv{}}
            {\ltiEnv{}}
            {\ltiS{}}
            {\ltiT{}}
            {\ltiCombinedThreadedEnvp{}}.
The first five rules correspond to the internal language subtyping rules,
extended with threaded environments.
The extra condition in \ltiSCSFn helps contain symbolic closures
to the type-variable scope they were defined in.
The rule \ltiSCSClosure relates symbolic closures with polymorphic function types.
It follows the idea that \ltiClosureWithStkID{\ltiEnv{}}
                             {\ltiClosureID{}}
                             {\ltiufun{\ltivar{}}{\ltiE{}}}
                             is a subtype of
\ltiPolyFn{\ltiT{}}{\ova{\ltitvar{}}}{\ltiS{}}
if \ltifuntparamargrettype{\ova{\ltitvar{}}}
                          {\ltivar{}}
                          {\ltiT{}}
                          {\ltiSp{}}
                          {\ltiE{}}
is well typed under \ltiEnv{} and 
\ltiSp{}
is a subtype of
\ltiS{}.
The rule proceeds similarly to \ltiSCAppInfClosure, except we may choose a polymorphic
type for \ltiClosureID{}, and we must check the return type is under \ltiS{}.
Adding a type binder to a term invites the possibility of unintentional 
variable capture, and so 
the condition on \ova{\ltitvar{}} avoids capturing free type variables in \ltiE{}.

To illustrate \ltiSCSClosure's role, we again aggressively simplify our presentation.
The simplified rule \ltiSimpSClosure extracts symbolic closure's terms and definition environment
to check it with the provided input type and type arguments.

\begin{mathpar}
    \infer [\ltiSimpSClosure]
    {
    \ltitSstkjudgementNoElabCombined{\ltimakeCombinedThreadedEnv{\ltiFuel{1}-1}{\ltiClosureCache{1}}}
                      {\ltiEnvConcat{\colorbox{pink}{\ltiEnv{}}}
                                    {\ltiEnvConcat{\colorbox{pink}{\ova{\ltitvar{}}}}
                                                  {\hastype{\colorbox{pink}{\ltivar{}}}
                                                           {\colorbox{pink}{\ltiT{}}}}}}
                      {\colorbox{pink}{\ltiE{}}}
                      {\colorbox{pink}{\ltiSp{}}}
                      {\ltimakeCombinedThreadedEnv{\ltiFuel{2}}{\ltiClosureCache{2}}}
                      {\ltiEp{}}
                      \\
    \ltiSsubtypeJustTypes{\ltimakeCombinedThreadedEnv
                 {\ltiFuel{2}}
                 {\ltiupdateClosureCacheSingleLHS{\ltiClosureCache{2}}
                                                  {\ltiClosureID{}}
                                                  {\ltifuntparamargrettype
                                                   {\ova{\ltitvar{}}}
                                                   {\ltivar{}}
                                                   {\ltiT{}}
                                                   {\ltiSp{}}
                                                   {\ltiEp{}}}}}
                {\ltiEnv{}}{\colorbox{pink}{\ltiSp{}}}{\ltiS{}}
                {\ltiCombinedThreadedEnv{3}}
    }
    { \ltiSsubtypeJustTypes{\ltiCombinedThreadedEnv{1}}
                  {\ltiEnvp{}}
                  {\colorbox{pink}{\ltiClosure{\ltiEnv{}}
                                       {\ltiufun{\ltivar{}}{\ltiE{}}}}}
                  {\ltiPolyFn{\colorbox{pink}{\ltiT{}}}{\colorbox{pink}{\ova{\ltitvar{}}}}{\ltiS{}}}
                  {\ltiCombinedThreadedEnv{3}}
                  }
\end{mathpar}

We now illustrate how to check
\ltianncolon{\ltiRec{\text{d}=\ltiufun{\ltivar{}}{\ltiF{}}}}
            {\ltiPolyFn{\ltiT{}}{\ova{\ltitvar{}}}{\ltiS{}}}
a slightly more complicated version of
\ltianncolon{(\ltiufun{\ltivar{}}{\ltiF{}})}{\ltiPolyFn{\ltiT{}}{\ova{\ltitvar{}}}{\ltiS{}}}.
To streamline presentation, we temporarily ignore the fact that type ascription
is syntactic sugar (defined \figref{symbolic:figure:external-language-syntax-abbreviations})
and use its standard typing rule, which first synthesizes a type for the term, then checks
it is a subtype of the provided type.

\begin{mathpar}
  \inferrule*[]
  {
  \inferrule*
  {
  \inferrule*[lab=\boxed{\ltiSimpUAbs}]
   {}
   {
   \ltitSstkjudgementNoElabCombined{\varnothing}
                                   {\colorbox{pink}{\ltiEnv{}}^{\tikz[overlay,remember picture] \node [] (c1) {};}}
                                   {{\tikz[overlay,remember picture] \node [] (c) {};}\colorbox{pink}{\ltiufun{\ltivar{}}{\ltiF{}}}}
                                   {\colorbox{pink}{
                                           {\ltiClosure{\text{}^{\tikz[overlay,remember picture] \node [] (d1) {};}\ltiEnv{}}
                                                                {{\tikz[overlay,remember picture] \node [] (d2) {};}\ltiufun{\ltivar{}}{\ltiF{}}}}}
                                           {\tikz[overlay,remember picture] \node [] (d) {};}}
                                   {\ltiCombinedThreadedEnv{}}
                                   {\ltiufun{\ltivar{}}{\ltiF{}}}
   }
   }
   {
   \ltitSstkjudgementJustType{\varnothing}
                                   {...}
                                   {\ltiRec{\text{d}={{\tikz[overlay,remember picture] \node [] (b) {};}\colorbox{pink}{\ltiufun{\ltivar{}}{\ltiF{}}}}}}
                                   {\ltiRec{\hastype{\text{d}}
                                           {\colorbox{pink}
                                            {\ltiClosure{\ltiEnv{}}
                                                                {\ltiufun{\ltivar{}}
                                                                         {\ltiF{}}}}
                                                                         ^
                                                                         {\tikz[overlay,remember picture] \node [] (e) {};}
                                                                         _
                                                                         {\tikz[overlay,remember picture] \node [] (e1) {};}}}}
                                   {\ltiCombinedThreadedEnv{}}
                                   {\ltiufun{\ltivar{}}{\ltiF{}}}
                                   }
   \\
  \inferrule*[%T-Ann
  ]
  {
  \inferrule*[]
   {}
   {
   \ltitSstkjudgementJustType{\varnothing}
                      {...}
                      {{\ltisel{\text{g}}{\text{d}}}}
                      {\colorbox{pink}{\ltiClosure{{\ltiEnv{}}}
                                           {\ltiufun{{\ltivar{}}}
                                                    {\ltiF{}}}}
                                                    {\tikz[overlay,remember picture] \node [] (h) {};}}
                      {\ltiClosureCache{}}
                      {\ltiufunelab{\ltiInferred{\text{c1}}}{\text{x}}{\text{x}}}
                      \\
    \inferrule* [lab=\boxed{\ltiSimpSClosure}]
    {
    \ltitSstkjudgementNoElabCombined{\ltiClosureCache{}}
                      {\ltiEnvConcat{\colorbox{pink}{\ltiEnv{}}_{\tikz[overlay,remember picture] \node [] (k) {};}}
                                    {\ltiEnvConcat
                                      {{\colorbox{pink}{\ova{\ltitvar{}}}}_{\tikz[overlay,remember picture] \node [] (u) {};}}
                                      {\hastypesmall{\colorbox{pink}{\ltivar{}}_{\tikz[overlay,remember picture] \node [] (q) {};}}
                                                    {{\colorbox{pink}{\ltiT{}}}_{\tikz[overlay,remember picture] \node [] (o) {};}}}}}
                      {\text{}_{\tikz[overlay,remember picture] \node [] (m) {};}
                       \colorbox{pink}{\ltiF{}}}
                      {{\ltiSp{}}}
                      {\ltiClosureCache{}}
                      {\ltisel{\text{g}}{\text{d}}}
                      \\
    \ltiSsubtypeJustTypes{\ltimakeCombinedThreadedEnv
                 {\ltiFuel{2}}
                 {\ltiupdateClosureCacheSingleLHS{\ltiClosureCache{2}}
                                                  {\ltiClosureID{}}
                                                  {\ltifuntparamargrettype
                                                   {\ova{\ltitvar{}}}
                                                   {\ltivar{}}
                                                   {\ltiT{}}
                                                   {\ltiSp{}}
                                                   {\ltiEp{}}}}}
                {\ltiEnv{}}{{\ltiSp{}}}
                           {\colorbox{pink}{\ltiS{}}{\tikz[overlay,remember picture] \node [] (r) {};}}
                {\ltiCombinedThreadedEnv{3}}
    }
    { 
      \ltiSsubtypeJustTypes{\ltiCombinedThreadedEnv{1}}
                  {\ltiEnvp{}}
                  {\text{}_{\tikz[overlay,remember picture] \node [] (i) {};}
                      \colorbox{pink}{\ltiClosure{{\tikz[overlay,remember picture] \node [] (j) {};}{\ltiEnv{}}}
                                           {\ltiufun{{\ltivar{}}^{\tikz[overlay,remember picture] \node [] (p) {};}}
                                                    {{\tikz[overlay,remember picture] \node [] (l) {};}\ltiF{}}}}}
                  {{\colorbox{pink}
                    {\ltiPolyFn{{\ltiT{}}^{\tikz[overlay,remember picture] \node [] (n) {};}}
                               {{\tikz[overlay,remember picture] \node [] (t) {};}{\ova{\ltitvar{}}}}
                               {{\ltiS{}}{\tikz[overlay,remember picture] \node [] (s) {};}}}}
                    _{\tikz[overlay,remember picture] \node [] (v) {};}}
                  {\ltiCombinedThreadedEnv{3}}
                  }
   }
  }
  {
    \ltitSstkjudgementNoElabCombined{\ltiCombinedThreadedEnv{}}
                      {%\ltiEnvConcat{...}
                                    {\hastype{\text{g}}{\ltiRec{\hastype{\text{d}}
                                                                        {{}^{\tikz[overlay,remember picture] \node [] (g) {};}
                                                                                _{\tikz[overlay,remember picture] \node [] (f) {};}
                                                                        \colorbox{pink}{\ltiClosure{\ltiEnv{}}
                                                                                             {\ltiufun{\ltivar{}}{\ltiF{}}}}}}}}}
                      {\ltianncolon{\ltisel{\text{g}}{\text{d}}}
                                   {{}_{\tikz[overlay,remember picture] \node [] (x) {};}
                                   {\colorbox{pink}{\ltiPolyFn{\ltiT{}}
                                                              {\ova{\ltitvar{}}}
                                                              {\ltiS{}}}}^{\tikz[overlay,remember picture] \node [] (w) {};}}}
                      {{\ltiPolyFn{\ltiT{}}{\ova{\ltitvar{}}}{\ltiS{}}}}
                      {\ltiCombinedThreadedEnvp{}}
                      {\ltiapp{\ltiFp{}}{\ltiEp{}}}
  }
  }
  {
    \ltitSstkjudgementJustType{\ltiCombinedThreadedEnv{}}
                      {...}
                      {\ltilet{\text{g}}{\ltiRec{\text{d}={{\tikz[overlay,remember picture] \node [] (a) {};}
                                                           {\colorbox{pink}{\ltiufun{\ltivar{}}{\ltiF{}}}}}}}
                              {\ltianncolon{\ltisel{\text{g}}{\text{d}}}
                                           {{\colorbox{pink}{\ltiPolyFn{\ltiT{}}
                                                                      {\ova{\ltitvar{}}}
                                                                      {\ltiS{}}}}
                                                                      ^{\tikz[overlay,remember picture] \node [] (y) {};}}}}
                      {\ltiPolyFn{\ltiT{}}{\ova{\ltitvar{}}}{\ltiS{}}}
                      {\ltiCombinedThreadedEnvp{}}
                      {\ltiapp{\ltiFp{}}{\ltiEp{}}}
  }
\begin{tikzpicture}[remember picture, overlay,
                  text width = 2.5cm ]
  \coordinate (Start1) at (a);
  \coordinate (End1) at (b);
  \coordinate (Start2) at (b);
  \coordinate (End2) at (c);
  \coordinate (Start3) at (c);
  \coordinate (End3) at (d2);
  \coordinate (Start3p1) at (c1);
  \coordinate (End3p1) at (d1);
  \coordinate (Start4) at (d);
  \coordinate (End4) at (e);
  \coordinate (Start6) at (e1);
  \coordinate (End6) at (f);
  \coordinate (Start7) at (g);
  \coordinate (End7) at (h);
  \coordinate (Start7p1) at (h);
  \coordinate (End7p1) at (i);
  \coordinate (Start8) at (j);
  \coordinate (End8) at (k);
  \coordinate (Start9) at (l);
  \coordinate (End9) at (m);
  \coordinate (Start10) at (n);
  \coordinate (End10) at (o);
  \coordinate (Start11) at (p);
  \coordinate (End11) at (q);
  \coordinate (Start12) at (r);
  \coordinate (End12) at (s);
  \coordinate (Start13) at (t);
  \coordinate (End13) at (u);
  \coordinate (Start14) at (v);
  \coordinate (End14) at (w);
  \coordinate (Start15) at (x);
  \coordinate (End15) at (y);
  \draw[pink,-](Start1.north) to [bend left] (End1.south);
  \draw[pink,-](Start2.north) to (End2.south);
  \draw[pink,-](Start3.north) to [bend right] (End3.south);
  \draw[pink,-](Start3p1.north) to [bend left] (End3p1.south);
  \draw[pink,-](Start4.north) to (End4.south);
  \draw[pink,-](Start6.north) to (End6.south);
  \draw[pink,-](Start7.north) to (End7.south);
  \draw[pink,-](Start7p1.north) to (End7p1.south);
  \draw[pink,-](Start8.north) to (End8.south);
  \draw[pink,-](Start9.north) to (End9.south);
  \draw[pink,-](Start10.north) to (End10.south);
  \draw[pink,-](Start11.north) to (End11.south);
  \draw[pink,-](Start12.north) to (End12.south);
  \draw[pink,-](Start13.north) to (End13.south);
  \draw[pink,-](Start14.north) to (End14.south);
  \draw[pink,-](Start15.north) to (End15.south);
\end{tikzpicture} 
\end{mathpar}

The derivation begins in a similar fashion to the previous one, 
with the left subderivation using \ltiSimpUAbs
to delay the type checking of {\colorbox{pink}{\ltiufun{\ltivar{}}{\ltiF{}}}}
via the symbolic closure
\colorbox{pink}
{\ltiClosure{{\ltiEnv{}}}
           {\ltiufun{{\ltivar{}}}
                    {\ltiF{}}}}.
The right subderivation
begins with a rule checking type ascription, specifically that  term
``{\ltisel{\text{g}}{\text{d}}}''
has function type
{\colorbox{pink}{\ltiPolyFn{\ltiT{}}
                           {\ova{\ltitvar{}}}
                           {\ltiS{}}}}.
First, a type is synthesized for ``{\ltisel{\text{g}}{\text{d}}}'',
which reveals our symbolic closure type from the left subderivation.
Then, we check that the symbolic closure is a subtype of 
our desired function type via the symbolic reduction
    ``\ltitSstkjudgementNoElabCombined{\ltiClosureCache{}}
                      {\ltiEnvConcat{\colorbox{pink}{\ltiEnv{}}}
                                    {\ltiEnvConcat
                                      {{\colorbox{pink}{\ova{\ltitvar{}}}}}
                                      {\hastypesmall{\colorbox{pink}{\ltivar{}}}
                                                    {{\colorbox{pink}{\ltiT{}}}}}}}
                      {\colorbox{pink}{\ltiF{}}}
                      {{\ltiSp{}}}
                      {\ltiClosureCache{}}
                      {\ltisel{\text{g}}{\text{d}}}.''
Finally, the return type of the symbolic reduction is checked to be compatible
with our desired function type with
    \ltiSsubtypeJustTypes{\ltimakeCombinedThreadedEnv
                 {\ltiFuel{2}}
                 {\ltiupdateClosureCacheSingleLHS{\ltiClosureCache{2}}
                                                  {\ltiClosureID{}}
                                                  {\ltifuntparamargrettype
                                                   {\ova{\ltitvar{}}}
                                                   {\ltivar{}}
                                                   {\ltiT{}}
                                                   {\ltiSp{}}
                                                   {\ltiEp{}}}}}
                {\ltiEnv{}}{{\ltiSp{}}}
                           {\colorbox{pink}{\ltiS{}}}
                {\ltiCombinedThreadedEnv{3}}.


\begin{figure}

%  \[
%    \boxed{\ltielabDriver{\ltiE{}}{\ltiEp{}}{\ltiT{}}
%    \text{ Elaborates external language term \ltiE{} to internal language term \ltiEp{} and type \ltiT{}.
%    }
%    }
%  \]
%
%  \[
%  \begin{array}{lll}
%    \ltielabDriver{\ltiE{1}}
%                  {\ltielimClosLHS{\ltiClosureCache{}}{\ltiE{2}}}
%                  {\ltielimClosTLHS{\varnothing}{\ltiClosureCache{}}{\ltiT{}}}
%                  , &\text{where }
%    \exists \ltiFuel{}.
%     \ltitSstkjudgement{\ltimakeCombinedThreadedEnv{\ltiFuel{}}{\ltiEmptyClosureCache}}
%                       {\ltiEmptyEnv}
%                       {\ltiE{1}}
%                       {\ltiT{}}
%                       {\ltimakeCombinedThreadedEnv{\ltiFuelp{}}{\ltiClosureCache{}}}
%                       {\ltiE{2}}
%  \end{array}
%  \]

  \begin{mathpar}
    \boxed{
    \infer[]
    {}
    {\ltiupdateClosureCacheSingle{\ltiClosureCache{}}{\ltiClosureID{}}{\ltiE{}}{\ltiClosureCachep{}}
    \\\\
    \text{Pick SCL elaboration \ltiE{} for symbolic closure identifier \ltiClosureID{}.
    }
    }
    }

    \begin{array}{llll}
      \ltiupdateClosureCacheSinglealign{\ltiClosureCache{}}{\ltiClosureID{}}{\ltiE{}}
                           {\ltimapsto{\ltiClosureCache{}}
                                      {\ltiClosureID{}}
                                      {\ltiClosure{\ltiEnv{}}
                                                  {\ltiE{}}}}
                                                 , &
                                                 \text{where }
    (\ltilookup{\ltiClosureCache{}}{\ltiClosureID{}} = 
                       {\ltiClosure{\ltiEnv{}}
                                   {\ltiufun{\ltivar{}}{\ltiF{}}}})
                                   \text{ or }
    (\ltilookup{\ltiClosureCache{}}{\ltiClosureID{}} = 
              {\ltiClosure{\ltiEnv{}}{\ltiE{}}})
    \end{array}
  \end{mathpar}

  \[
    \boxed{\ltielimClos{\ltiClosureCache{}}{\ltiE{}}{\ltiEp{}}
    \text{ Converts symbolic closures in \ltiE{} to explicit types in \ltiEp{}}
    }
  \]

  \[
  \begin{array}{llll}
    \ltielimClosalign{\ltiClosureCache{}}{\ltivar{}}
                     {\ltivar{}}
                     \\
    \ltielimClosalign{\ltiClosureCache{}}
                     {\ltiappinst{\ltiF{}}
                                 {\ova{\ltiR{}}}
                                 {\ltiE{}}}
                     {\ltiappinst{\ltielimClosLHS{\ltiClosureCache{}}{\ltiF{}}}
                                 {\ova{\ltielimClosTLHS{\varnothing}{\ltiClosureCache{}}{\ltiR{}}}}
                                 {\ltielimClosLHS{\ltiClosureCache{}}{\ltiE{}}}}
                             \\
    \ltielimClosalign{\ltiClosureCache{}}{\ltisel{\ltiF{}}{\ltivar{}}}
                     {\ltisel{\ltielimClosLHS{\ltiClosureCache{}}{\ltiF{}}}{\ltivar{}}}
                     \\
    \ltielimClosalign{\ltiClosureCache{}}{\ltiRec{\ova{\ltivar{} = \ltiF{}}}}
                     {\ltiRec{\ova{\ltivar{} = \ltielimClosLHS{\ltiClosureCache{}}{\ltiF{}}}}}
                     \\
    \ltielimClosalign{\ltiClosureCache{}}
                     {\ltifuntparamargtype{\ova{\ltitvar{}}}
                                          {\ltivar{}}
                                          {\ltiT{}}
                                          {\ltiE{}}}
                     {\ltifuntparamargtype{\ova{\ltitvar{}}}
                                          {\ltivar{}}
                                          {\ltielimClosTLHS{\varnothing}{\ltiClosureCache{}}{\ltiT{}}}
                                          {\ltielimClosLHS{\ltiClosureCache{}}{\ltiE{}}}}
                     \\
    \ltielimClosalign{\ltiClosureCache{}}
                     {\ltiufunelab{\ltiufunelabentry{\ltiClosureID{}}}
                                  {\ltivar{}}
                                  {\ltiE{}}}
                     {\ltielimClosLHS{\ltiClosureCache{}}
                                     {\ltiF{}}},
                     &\text{where } \ltilookup{\ltiClosureCache{}}{\ltiClosureID{}}
                                      = \ltiClosure{\ltiEnv{}}{\ltiF{}}
  \end{array}
  \]


  \begin{mathpar}
    % spread boxed over two lines for dissertation
    \boxed{
    \infer[]
    {}
    {\ltielimClosT{\ova{\ltiClosureID{}}}{\ltiClosureCache{}}{\ltiT{}}{\ltiTp{}}
    \\
    \text{ Converts symbolic closures in \ltiT{} to explicit types in \ltiTp{},
    with seen symbolic closures \ova{\ltiClosureID{}}.}
    }}

  \begin{array}{llll}
    \ltielimClosTalign{\ova{\ltiClosureID{}}}{\ltiClosureCache{}}{\ltiTop}{\ltiTop}
                      \\
    \ltielimClosTalign{\ova{\ltiClosureID{}}}{\ltiClosureCache{}}{\ltiBot}{\ltiBot}
                      \\
    \ltielimClosTalign{\ova{\ltiClosureID{}}}{\ltiClosureCache{}}{\ltitvar{}}{\ltitvar{}}
                      \\
    \ltielimClosTalign{\ova{\ltiClosureID{}}}{\ltiClosureCache{}}
                      {\ltiPolyFn{\ltiT{}}{\ova{\ltitvar{}}}{\ltiS{}}}
                      {\ltiPolyFn{\ltielimClosTLHS{\ova{\ltiClosureID{}}}{\ltiClosureCache{}}{\ltiT{}}}{\ova{\ltitvar{}}}
                             {\ltielimClosTLHS{\ova{\ltiClosureID{}}}{\ltiClosureCache{}}{\ltiS{}}}}
                                          \\
    \ltielimClosTalign{\ova{\ltiClosureID{}}}{\ltiClosureCache{}}
                      {\ltiRec{\ova{\hastype{\ltivar{}}{\ltiT{}}}}}
                      {\ltiRec{\ova{\hastype{\ltivar{}}{\ltielimClosTLHS{\ova{\ltiClosureID{}}}{\ltiClosureCache{}}{\ltiT{}}}}}}
                      \\
    \ltielimClosTalign{\ova{\ltiClosureID{}}}{\ltiClosureCache{}}
                      {\ltiClosureWithStkID{\ltiEnv{}}{\ltiClosureIDp{}}{\ltiE{}}}
                      {\ltiPolyFn{\ltielimClosTLHS{\ova{\ltiClosureID{}}\ltiClosureIDp{}}
                                                  {\ltiClosureCache{}}
                                                  {\ltiT{}}}
                                 {\ova{\ltitvar{}}}
                                 {\ltielimClosTLHS{\ova{\ltiClosureID{}}\ltiClosureIDp{}}
                                                  {\ltiClosureCache{}}
                                                  {\ltiS{}}}}
                      , & 
                      \text{where }
                      \ltiClosureIDp{} \not\in \ova{\ltiClosureID{}},
                      \ltilookup{\ltiClosureCache{}}{\ltiClosureIDp{}}
                      = \ltiClosure{\ltiEnv{}}
                                   {\ltifuntparamargrettype{\ova{\ltitvar{}}}{\ltivar{}}{\ltiT{}}{\ltiS{}}{\ltiE{}}}
                                   % this condition is checked in S-Closure and constraint system
                                   %, \ova{\ltitvar{}} \cap \ltitv{\ltiE{}} = \varnothing
  \end{array}
  \end{mathpar}
  \caption{Elaboration Metafunctions for SCL Terms and Types}
  \label{symbolic:figure:SC-language-elaboration}
\end{figure}

Now that we have covered the type system and subtyping, we turn to the elaboration rules,
which are given in \figref{symbolic:figure:SC-language-elaboration}.
They are split into two metafunctions \ltielimClossymbol and \ltielimClosTsymbol,
elaborating terms and types respectively, along with
\ltiupdateClosureCacheSinglesymbol, which manages when a symbolic closure's elaboration may
be chosen.

For terms, \ltielimClos{\ltiClosureCache{}}{\ltiE{}}{\ltiEp{}}
elaborates \ltiE{} to \ltiEp{} using elaboration cache \ltiClosureCache{}.
The case for tagged unannotated functions 
{\ltiufunelab{\ltiufunelabentry{\ltiClosureID{}}}
                                  {\ltivar{}}
                                  {\ltiE{}}}
simply uses the elaboration entry for \ltiClosureID{} to continue elaboration.
The metafunction is undefined for the external language's unannotated terms
\ltiufun{\ltivar{}}{\ltiE{}} and \ltiapp{\ltiF{}}{\ltiE{}}.
This enforces that each symbolic closure must be symbolically executed at least once
to elaborate away these terms.
Unannotated applications \ltiapp{\ltiF{}}{\ltiE{}}
are elaborated away with local type argument synthesis.
%, covered in \chapref{chapter:symbolic:directed-lti}.

For elaborating types, \ltielimClossymbol
uses \ltielimClosT{\ova{\ltiClosureID{}}}{\ltiClosureCache{}}{\ltiT{}}{\ltiTp{}}, which
elaborates symbolic closures in \ltiT{} using \ltiClosureCache{}.
Some extra bookkeeping is needed to prevent infinitely generating
types. Since symbolic closures may be passed to other symbolic closures,
a seen-set \ova{\ltiClosureID{}} handles the case where it is passed to itself.
Since we do not model equi-recursive types, we simply disallow that situation here.

To highlight the elaboration rules,
we demonstrate the lambda-encoding of let in our system by
checking term ``\ltilet{\ltivar{}}{42}{\ltiF{}}''---which desugars to
``{\ltiappParens{\ltiufun{\ltivar{}}{\ltiF{}}}{\text{42}}}''---at some
some overall type \ltiS{}
with $\text{42}$ having type $\text{Int}$.
To streamline presentation, we assume that checking \ltiF{}
does not introduce any symbolic closures
(to keep the elaboration cache compact), liberally
remove uninteresting type environments and elaboration caches,
and omit symbolic reduction fuel from the derivation.

\begin{mathpar}
  \inferrule*[left=\ltiSCAppInfClosure
  ]
  {\infer[\ltiSCUAbs]
   {
   %\colorbox{pink}{\ltiClosureID{}} \not\in dom(\varnothing)\\
   %\\\\
   \ltiClosureCache{} =
                       {\ltiClosureCacheEntry{\colorbox{pink}{\ltiClosureID{}}}
                                           {\ltiClosure{\ltiEmptyEnv}
                                                        {\ltiufun{\ltivar{}}{\ltiF{}}}}}
   }
   { \ltitSstkjudgement{\varnothing}
                      {\ltiEmptyEnv}
                      {\tikz[overlay,remember picture] \node [] (b) {};\ltiufun{\ltivar{}}{\ltiF{}}}
                      {\ltiClosureWithStkID{\ltiEmptyEnv}
                                           {{}_{\tikz[overlay,remember picture] \node [] (m) {};}{\colorbox{pink}{\ltiClosureID{}}}}
                                           {\ltiufun{\ltivar{}^{\tikz[overlay,remember picture] \node [] (e) {};}}
                                                    {\ltiF{}^{\tikz[overlay,remember picture] \node [] (g) {};}}}}
                      {\ltiClosureCache{}}
                      {\colorbox{pink}{\ltiufunelab{\ltiClosureID{}}{\ltivar{}}{{\ltiF{}}_{\tikz[overlay,remember picture] \node [] (o) {};}}}}
   }
   \\
    \begin{array}{ccc}
   \ltitSstkjudgementJustType{\ltiClosureCache{}}
                      {\ltiEmptyEnv}
                      {\tikz[overlay,remember picture] \node [] (d) {};\text{42}}
                      {\colorbox{pink}{\text{Int}\tikz[overlay,remember picture] \node [] (i) {};}}
                      {\ltiClosureCache{}}
                      {\ltiEp{}}
                      \ \ \ \ \ 
    \ltitSstkjudgementNoCombined{\ltiClosureCache{}}
                      {\hastype{\text{}^{\tikz[overlay,remember picture] \node [] (f) {};}
                                \ltivar{}}
                               {{\tikz[overlay,remember picture] \node [] (j) {};}{\text{Int}}}}
                      {\text{}^{\tikz[overlay,remember picture] \node [] (h) {};}
                       \ltiF{}}
                      {{\colorbox{pink}{\ltiS{}}}_{\tikz[overlay,remember picture] \node [] (k) {};}}
                      {\ltiClosureCache{}}
                      {{\colorbox{pink}{\ltiFp{}}}_{\tikz[overlay,remember picture] \node [] (s) {};}}
    \end{array}
  }
  {
    \ltitSstkjudgement{\varnothing}
                      {\ltiEmptyEnv}
                      {\ltiappParens{\text{}^{\tikz[overlay,remember picture] \node [] (a) {};}
                                     \ltiufun{\ltivar{}}{\ltiF{}}}
                                    {\text{42}^{\tikz[overlay,remember picture] \node [] (c) {};}}}
                      {\ltiS{}}
                      {\ltiClosureCacheEntry
                       {{}^{{}^{\tikz[overlay,remember picture] \node [] (n) {};}}{\colorbox{pink}{\ltiClosureID{}}}}
                       {\ltiClosure{\ltiEmptyEnv}
                                         {\ltifunargrettype{\ltivar{}}
                                                           {{\colorbox{pink}
                                                            {\text{Int}}}^{\tikz[overlay,remember picture] \node [] (j2) {};}}
                                                           {
                                                            {\colorbox{pink}{\ltiS{}}}^{\tikz[overlay,remember picture] \node [] (l) {};}
                                                            \ }
                                                           {{\colorbox{pink}{\ltiFp{}}}^{\tikz[overlay,remember picture] \node [] (t) {};}}}}}
                      {\ltiappParens{{\colorbox{pink}{\ltiufunelab{\ltiClosureID{}}{\ltivar{}}{\ltiF{}}}}^{\tikz[overlay,remember picture] \node [] (p) {};}}{\text{42}}}
  }
\begin{tikzpicture}[remember picture, overlay,
                  text width = 2.5cm ]
  \coordinate (Start1) at (a);
  \coordinate (End1) at (b);
  \coordinate (Start2) at (c);
  \coordinate (End2) at (d);
  \coordinate (Start3) at (e);
  \coordinate (End3) at (f);
  \coordinate (Start4) at (g);
  \coordinate (End4) at (h);
  \coordinate (Start5) at (i);
  \coordinate (End5) at (j);
  \coordinate (Start6) at (i);
  \coordinate (End6) at (j2);
  \coordinate (Start7) at (k);
  \coordinate (End7) at (l);
  \coordinate (Start8) at (m);
  \coordinate (End8) at (n);
  \coordinate (Start9) at (o);
  \coordinate (End9) at (p);
  \coordinate (Start11) at (s);
  \coordinate (End11) at (t);
  % (lambda (x) f) -> (lambda (x) f)
  %\draw[pink,thick,dotted,->](Start1.north) to (End1.south);
  % 42 -> 42
  %\draw[pink,thick,dotted,->](Start2.north) to (End2.south);
  \draw[pink,thick,dotted,->](Start3.north) to [bend left] (End3.south);
  \draw[pink,thick,dotted,->](Start4.north) to [bend left] (End4.south);
  \draw[pink,thick,dotted,->](Start5.north) to [bend right] (End5.south);
  % a little to the right (Int -> Int)
  \draw[pink,thick,solid,->,transform canvas={xshift=1pt}](Start6.north) to (End6.south);
  \draw[pink,thick,solid,->](Start7.north) to [bend right=30] (End7.south);
  \draw[pink,thick,solid,->](Start8.north) to (End8.north);
  % a little higher
  \draw[pink,thick,solid,->,transform canvas={yshift=4pt}](Start9.north) to [bend left=10] (End9.north);
  \draw[pink,thick,solid,->](Start11.north) to (End11.north);
\end{tikzpicture} 
\end{mathpar}
%
The derivation starts with the \ltiSCAppInfClosure rule (chosen because the operator
\ltiufun{\ltivar{}}{\ltiF{}}
has a symbolic closure type)
with empty elaboration cache \ltiEmptyClosureCache
and empty type environment \ltiEmptyEnv.
The distinct identifer {{\colorbox{pink}{\ltiClosureID{}}}}
is chosen by \ltiSCUAbs in the left subderivation, and is utilized there in three ways.
First, the elaboration cache \ltiClosureCache{}'s entry for
{{\colorbox{pink}{\ltiClosureID{}}}} is initialized
to
{\ltiClosure{\ltiEmptyEnv}{\ltiufun{\ltivar{}}{\ltiF{}}}},
an untyped ``placeholder'' that signals that the final elaboration of {{\colorbox{pink}{\ltiClosureID{}}}}
has yet to be picked.
Second, it identifies the rule's overall symbolic closure type
                      {\ltiClosureWithStkID{\ltiEmptyEnv}
                                           {{\colorbox{pink}{\ltiClosureID{}}}}
                                           {\ltiufun{\ltivar{}}
                                                    {\ltiF{}}}}
for later elaboration.
Third, it tags the rule's elaboration {\colorbox{pink}{\ltiufunelab{\ltiClosureID{}}{\ltivar{}}{\ltiF{}}}}
which, again, links the term to its elaboration cache entry.
Like a traditional application rule, the middle subderivation checks the argument.
The right subderivation performs a symbolic reduction
    \ltitSstkjudgementNoCombined{\ltiClosureCache{}}
                      {\hastype{\ltivar{}}
                               {{\text{Int}}}}
                      {\ltiF{}}
                      {{\colorbox{pink}{\ltiS{}}}}
                      {\ltiClosureCache{}}
                      {{\colorbox{pink}{\ltiFp{}}}},
where \ltivar{} and \ltiF{} come from the symbolic closure type
and $\text{Int}$ from the argument's type,
with the result type \colorbox{pink}{\ltiS{}} being the type of the entire derivation.
The elaboration \colorbox{pink}{\ltiFp{}}
will (eventually) be inserted in \ltiF{}'s place.

Now the goal is to stash enough information in the elaborated term and elaboration cache
to eliminate unannotated terms using \ltielimClossymbol after type checking.
For the entire derivation's elaborated term,
tagged functions are preserved
by combining the operator and operand elaborations in
{\ltiappParens{{\colorbox{pink}{\ltiufunelab{\ltiClosureID{}}{\ltivar{}}{\ltiF{}}}}}{\text{42}}}.
For the elaboration cache,
the (omitted) call
                 \ltiupdateClosureCacheSingleLHS{\ltiClosureCache{}}
                                                {\ltiClosureID{}}
                       {\ltiClosure{\ltiEmptyEnv}
                                         {\ltifunargrettype{\ltivar{}}
                                                           {{\text{Int}}}
                                                           {{\colorbox{pink}{\ltiS{}}}
                                                            \ }
                                                           {{\colorbox{pink}{\ltiFp{}}}}}}
updates the elaboration cache's ``placeholder'' entry
for {{\colorbox{pink}{\ltiClosureID{}}}}
to be 
                      {\ltiClosureCacheEntry
                       {{\colorbox{pink}{\ltiClosureID{}}}}
                       {\ltiClosure{\ltiEmptyEnv}
                                         {\ltifunargrettype{\ltivar{}}
                                                           {{\colorbox{pink}
                                                            {\text{Int}}}}
                                                           {
                                                            {\colorbox{pink}{\ltiS{}}}
                                                            \ }
                                                           {{\colorbox{pink}{\ltiFp{}}}}}}},
where {{\colorbox{pink}{\ltiClosureID{}}}},
{\ltiEmptyEnv}, and {\ltivar{}} come from the symbolic closure type,
{{\colorbox{pink}{\text{Int}}}} from the argument's type, 
and {\colorbox{pink}{\ltiS{}}},
{{\colorbox{pink}{\ltiFp{}}}}
from the symbolic reduction.

% Now explain how to SC elaborate a term

The following call to \ltielimClossymbol then eliminates all SCL terms
and types, performing a full elaboration into the internal language.
Once a tagged function term is encountered, its elaboration is simply read
off the cache and inserted.
In this case, there is no need
to also traverse \ltiFp{} since we assumed it does not contain symbolic closures,
however in general a recursive \ltielimClossymbol call on
\ltiFp{} would be needed to eliminate its symbolic closures.

\[
\begin{array}{llll}
\ltielimClosalign{\ltiClosureCacheEntry
                       {{\colorbox{pink}{\ltiClosureID{}}}{\tikz[overlay,remember picture] \node [] (b) {};}}
                       {{\ltiClosure{\ltiEmptyEnv}
                                   {\colorbox{pink}
                                         {\ltifunargrettype{\ltivar{}}
                                                           {\text{Int}}
                                                           {\ltiS{} \ }
                                                           {\ltiFp{}}}}}^
                        {\tikz[overlay,remember picture] \node [] (c) {};}}}
{\ \ \ltiappParens{{\tikz[overlay,remember picture] \node [] (a) {};}{\colorbox{pink}{\ltiufunelab{\ltiClosureID{}}{\ltivar{}}{\ltiF{}}}}}{\text{42}}}
{\ltiappParens{{}^{\tikz[overlay,remember picture] \node [] (d) {};}
               \colorbox{pink}{\ltifunargrettype{\ltivar{}}
                                                {\text{Int}}
                                                {\ltiS{} \ }
                                                {\ltiFp{}}}}
              {\text{42}}}
\begin{tikzpicture}[remember picture, overlay,
                  text width = 2.5cm ]
  \coordinate (Start1) at (a);
  \coordinate (End1) at (b);
  \coordinate (Start2) at (c);
  \coordinate (End2) at (d);
  \draw[pink,thick,solid,->](Start1.north) to [bend left] (End1.south);
  \draw[pink,thick,solid,->](Start2.north) to [bend left] (End2.south);
\end{tikzpicture} 
\end{array}
\]

%\section{Examples without Type Argument Synthesis}

%% let x = 1 in x
%{
%\begin{lstlisting}[language=ml,mathescape=true]
%let x = 1 in x
%(* Desugared *)
%$\ltiappParens{\ltiufun{\text{x}}{\text{x}}}{\text{1}}$
%(* SC annotated *)
%(* $\ltiInferred{\ltiClosureCache{} =%
%      \ltiClosureCacheEntry{\text{c1}}%
%                           {\ltiClosure{\ltiEmptyEnv}%
%                                       {\ltiNotInferred%
%                                        {\ltifunargrettype{\text{x}}%
%                                                          {\ltiInferred{\text{Int}}}%
%                                                          {\ltiInferred{\text{Int}},}%
%                                                          {\text{x}}}}}}$ *)
%$\ltiappParens{\ltiufunelab{\ltiInferred{\text{c1}}}{\text{x}}{\text{x}}}{\text{1}}$
%(* fully annotated *)
%$\ltiappParens{\ltifunargrettype{\text{x}}%
%                                {\ltiInferred{\text{Int}}}%
%                                {\ltiInferred{\text{Int}},}%
%                                {\text{x}}}%
%              {\text{1}}$
%\end{lstlisting}
%}


\chapter{Symbolic Closure Metatheory Conjectures}
\label{chapter:symbolic:metatheory}

We now outline the relationships between the internal, external, and symbolic closure
languages that would be desirable as a series of conjectures.
We leave the proofs as future work.

%\begin{lemma}[Subtyping (External Language)]
%   \ltiSdsubtypeseen{\ltiSubtypeSeen{}}{\ltiEnv{}}{\ltiT{}}{\ltiS{}}
%    iff
%   \ltiisubtypeseen{\ltiSubtypeSeen{}}{\ltiEnv{}}{\ltiT{}}{\ltiS{}}
%\end{lemma}

%\begin{lemma}[Soundness (External Language)]
%  If \ltitSdjudgement{\ltiEnv{}}
%                     {\ltiE{}}
%                     {\ltiT{}}
%                     {\ltiEp{}}
%                     and
%                     \ltiSdsubtypeseen{\varnothing}{\ltiEnv{}}{\ltiT{}}{\ltiS{}}
%                     then
%    \ltitjudgement{\ltiEnv{}}
%                  {\ltiEp{}}
%                  {\ltiS{}}
%                  {\ltiEpp{}}
%\end{lemma}

%\begin{lemma}[Completeness (External Language)]
%  If \ltitjudgement{\ltiEnv{}}
%                   {\ltiE{}}
%                   {\ltiT{}}
%                   {\ltiEp{}}
%                  then
%  \ltitSdjudgement{\ltiEnv{}}
%                     {\ltiE{}}
%                     {\ltiT{}}
%                     {\ltiEp{}}
%\end{lemma}

Ideally, SCL always infers sound annotations and types for external terms.

\begin{conjecture}[SCL Soundness]
  If there exists fuel {\ltiFuel{}} such that
     \ltitSstkjudgement{\ltimakeCombinedThreadedEnv{\ltiFuel{}}{\ltiEmptyClosureCache}}
                       {\ltiEmptyEnv}
                       {\ltiE{}}
                       {\ltiT{}}
                       {\ltimakeCombinedThreadedEnv{\ltiFuelp{}}{\ltiClosureCache{}}}
                       {\ltiEp{}}
  for external term \ltiE{},
                       SCL type \ltiT{},
                       and SCL term \ltiEp{},
                      then
    \ltitjudgementNoElab{}
                  {\ltielimClosLHS{\ltiClosureCache{}}{\ltiEp{}}}
                  {\ltiTp{}}
                   in the internal language,
                  where 
                  \ltiisubtype{}{\ltiTp{}}{\ltielimClosTLHS{\varnothing}{\ltiClosureCache{}}{\ltiT{}}}.
\end{conjecture}

%Several key 

%\begin{conjecture}[SCL Completeness 1]
%  If \ltitjudgementNoElab{\ltiEnv{}}
%                   {\ltiE{}}
%                   {\ltiT{}}
%    then
%    either
%    \forall
%    and fuel \ltiFuel{}
%    such that
%    \ltitSstkjudgementNoElab{\ltimakeCombinedThreadedEnv{\ltiFuel{}}{\ltiEmptyClosureCache}}
%                      {\ltiEnv{}}
%                      {\ltiF{}}
%                      {\ltiT{}}
%                      {\ltimakeCombinedThreadedEnv{\ltiFuelp{}}{\ltiClosureCache{}}}
%                      {\ltiFp{}}.
%\end{conjecture}

%% TODO timeout rules for type system
%% TODO 

Completeness for SCL says that there always exists some amount of annotations
we can add to a term that type checks in the external language so it can type
check in SCL.

\begin{conjecture}[SCL Weak Completeness]
  If \ltitjudgementNoElab{}
                     {\ltiE{}}
                     {\ltiT{}}
                     for external term {\ltiE{}},
                     then
    there exists a term \ltiF{} such that \ltiE{} is a partial erasure of \ltiF{}
    and fuel \ltiFuel{}
    such that
    \ltitSstkjudgementNoElab{\ltimakeCombinedThreadedEnv{\ltiFuel{}}{\ltiEmptyClosureCache}}
                      {\ltiEmptyEnv}
                      {\ltiF{}}
                      {\ltiT{}}
                      {\ltimakeCombinedThreadedEnv{\ltiFuelp{}}{\ltiClosureCache{}}}
                      {\ltiFp{}}.
\end{conjecture}

Since SCL essentially contains the rules of the internal language, we can always
use the annotations chosen by the external language's oracle. This way, we can
erase all symbolic closure introduction rules and reuse previous results for systems
based on \ltiFsub, so this theorem does not say much about symbolic closures.
It would be nice to prove a stronger theorem
based on fuel, such as the following.

\begin{conjecture}[SCL Strong Completeness]
  If \ltitjudgementNoElab{}
                     {\ltiE{}}
                     {\ltiT{}}
                     for external term {\ltiE{}},
                     then
    either, for all initial fuel \ltiFuel{}, SCL gets stuck at a symbolic reduction with zero fuel when checking {\ltiE{}},
    or there exists fuel \ltiFuel{}
    such that
    \ltitSstkjudgementNoElab{\ltimakeCombinedThreadedEnv{\ltiFuel{}}{\ltiEmptyClosureCache}}
                      {\ltiEmptyEnv}
                      {\ltiE{}}
                      {\ltiTp{}}
                      {\ltimakeCombinedThreadedEnv{\ltiFuelp{}}{\ltiClosureCache{}}}
                      {\ltiFp{}},
          and \ltiisubtype{\ltiEnv{}}{\ltielimClosTLHS{\varnothing}{\ltiClosureCache{}}{\ltiTp{}}}{\ltiT{}}.
\end{conjecture}

Unfortunately, this does not hold in the presented model of SCL, at least in part because of the restrictions placed on
symbolic closures and our use of ``off-the-shelf'' type argument synthesis.
Furthermore, we would need to distinguish type errors from running out of fuel.
We now discuss some specific issues we face when trying to achieve a system with Strong Completeness,
and speculate on how to fix them.

The restriction in \ltiSCAbs disallows symbolic closures to cross into a new type variable
scope, like
``\ltilet{\text{f}}{\ltiufun{\ltivar{}}{\ltiE{}}}
       {\ltifun{\ltitvar{}}{\ltivar{}}{\ltitvar{}}{\ltiapp{\text{f}}{\ltivar{}}}}.''
We have considered two fixes, both with their own tradeoffs.
The first is to infer a polymorphic type for $\text{f}$---we simply quantify
over each type variable that occurs out-of-scope with respect to $\text{f}$'s
definition context.
In the present example, instead of recording $\text{f}$'s type as
``\ltiPolyFn{\ltitvar{}}{}{...}'' at the application site,
we would record the polymorphic type
``\ltiPolyFn{\ltitvar{}}{\ltitvar{}}{...}'',
since {\ltitvar{}} is not in scope at $\text{f}$'s definition.
The complication here is related to type argument synthesis.
Since $\text{f}$ now has a polymorphic type, 
the application must elaborate to \ltiappinst{\text{f}}{\ltitvar{}}{\ltivar{}}.
Furthermore, if $\text{f}$ is passed as an argument to a polymorphic function
such as \ltiapp{\text{map}}{\text{f},\ltiE{}},
we would have to also infer type arguments for operands
(like CDuce~\cite{polyduce2}).

Another approach to handling out-of-scope type variables is to 
enrich types with type contexts.
Here, $\text{f}$ would have type 
``(\ltistackmapping{\ltitvarp{}}{\ltiPolyFn{\ltitvarp{}}{}{...}})'',
which says ``in a type context that starts with some
type variable \ltitvar{},
expands to \ltiPolyFn{\ltitvar{}}{}{...}''.
This way, checking the body of ``\ltifun{\ltitvar{}}{\ltivar{}}{\ltitvar{}}{\ltiapp{\text{f}}{\ltivar{}}}''
would effectively update $\text{f}$'s type to 
``\ltiPolyFn{\ltitvar{}}{}{...}''.
This approach resembles contextual subtyping~\cite{Dunfield2004Tridirectional},
except their dependence of types on type contexts is purely syntactic.
Type contexts are eliminated too early to check our example,
because the $\text{f}$'s annotation must occur at its definition, but
there the type environment is empty.

Another major restriction is that we may only pick a single
elaboration for each symbolic closure. This disallows simple programs like 
\ltilet{\text{f}}{\ltiufun{\ltivar{}}{\ltiE{}}}
       {\ltiRec{\text{left}=\ltiapp{\text{f}}{1},
                \text{right}=\ltiapp{\text{f}}{\text{``a''}}}}.
We have considered several approaches to lifting this restriction.
First is to infer an \emph{intersection type} for \text{f}.
That way, \text{f}'s type would be
\ltiIFn{\ltiFn{\text{Int}}{\text{Int}},{\ltiFn{\text{Str}}{\text{Str}}}},
which says ``returns an Int when given an Int, and returns a Str
when given a Str.''
Now choosing the final elaboration for {\ltiE{}} becomes more complicated,
but the intersection type checking literature presents several solutions~\cite{wells2002branching,Dunfield2004Tridirectional,polyduce1}.

We could also try and guess a polymorphic type for \text{f}
based on its use sites,
in an approach resembling Trace Typing~\cite{Andreasen2016TraceTA}.
Here, say we know 
\text{f} is type \ltiFn{\text{Int}}{\text{Int}}
and we learn it is also of type \ltiFn{\text{Str}}{\text{Str}},
we could guess a generalization like
``\ltiPolyFn{\ltitvar{}}{\ltitvar{}}{\ltitvar{}}''
based on the shape of both observations.
Since the definition type environment of \text{f} is always handy
in the elaboration cache, we could check if \text{f} inhabits
the generalized type (Trace Typing only checks use sites to
verify a polymorphic type).

The restriction in \ltiSCAppInfPT that type argument synthesis
may not reason about symbolic closures rules out checking
\ltiapp{\text{map}}{\ltiufun{\ltivar{}}{\ltiF{}}, \ltiE{}}.
Supporting symbolic closures in type argument synthesis requires
a following a notion of data flow in polymorphic types.
For example, the free theorems~\cite{wadler1989theorems}
of the type for \text{map} implies that
its function argument may be invoked only with elements of
its collection argument.
We can visualize the data flow \text{map} must adhere to.

\[
\ltiPolyFn{\ltiFn{\text{a}_{\tikz[overlay,remember picture] \node [] (b) {};}}
                 {\text{b}^{\tikz[overlay,remember picture] \node [] (c) {};}}
                 ,
          {\text{List[a}_{\tikz[overlay,remember picture] \node [] (a) {};}
                 \text{]}}
                 }
          {\text{a,b}}
          {\text{List[}^{\tikz[overlay,remember picture] \node [] (d) {};}
           \text{b]}}
\begin{tikzpicture}[remember picture, overlay,
                  text width = 2.5cm ]
  \coordinate (Start1) at (a);
  \coordinate (End1) at (b);
  \coordinate (Start2) at (c);
  \coordinate (End2) at (d);
  \draw[red,->,bend right=-45](Start1.south) to (End1.east);
  \draw[red,->,bend right=-45](Start2.east) to (End2.north east);
\end{tikzpicture} 
\]

Now, the role of type argument synthesis is to integrate symbolic closures
into this data flow.
For our example, the type of \ltiE{} flows to \text{List[a]},
which informs the symbolic closure of its input, and then \text{List[b]}
is derived from the output of the symbolic closure.

\[
\begin{array}{c}
\ltiapp{\text{map}}{\ltiufun{{\tikz[overlay,remember picture] \node [] (b2) {};}\ltivar{}}
                             {\ltiF{}}{\tikz[overlay,remember picture] \node [] (c0) {};},
                             \ltiE{}{\tikz[overlay,remember picture] \node [] (a0) {};}} \\\\
\ltiPolyFn{\ltiFn{\text{a}_{\tikz[overlay,remember picture] \node [] (b) {};}}
                 {\text{b}^{\tikz[overlay,remember picture] \node [] (c) {};}}
                 ,
          {\text{List[a}_{\tikz[overlay,remember picture] \node [] (a) {};}
                 \text{]}}
                 }
          {\text{a,b}}
          {\text{List[}^{\tikz[overlay,remember picture] \node [] (d) {};}
           \text{b]}}
\begin{tikzpicture}[remember picture, overlay,
                  text width = 2.5cm ]
  \coordinate (Start0) at (a0);
  \coordinate (End0) at (a);
  \coordinate (Start1) at (a);
  \coordinate (End1) at (b);
  \coordinate (Start1b) at (b);
  \coordinate (End1b) at (b2);
  \coordinate (Start2a) at (c0);
  \coordinate (End2a) at (c);
  \coordinate (Start2) at (c);
  \coordinate (End2) at (d);
  \draw[blue,dotted,thick,->](Start0.south) to (End0.east);
  \draw[red,->,bend right=-45](Start1.south) to (End1.east);
  \draw[blue,dotted,thick,->](Start1b.south) to (End1b.east);
  \draw[blue,dotted,thick,->](Start2a.south) to (End2a.east);
  \draw[red,->,bend right=-45](Start2.east) to (End2.north east);
\end{tikzpicture} 
\end{array}
\]

Formalizing this intuition into a type-argument synthesis algorithm is work-in-progress.

% Recursive types
Our SCL model does not feature equi-recursive recursive types. Adding them would allow us to
lift the restriction that a symbolic closure may not be passed to itself.
In the elaboration phase, we introduce a recursive type when we discover
a symbolic closure that we are currently elaborating.
For example, the program ``\ltilet{\ltivar{}}{\ltiufun{\ltivar{}}{\ltivar{}}}{\ltiapp{\ltivar{}}{\ltivar{}}}'',
{\ltivar{}} is passed to itself.
Using this method of elaboration, {\ltivar{}} will be
assigned the type
\ltiFn{\ltiMu{\ltitvar{}}{\ltiFn{\ltitvar{}}{\ltitvar{}}}}
      {\ltiMu{\ltitvar{}}{\ltiFn{\ltitvar{}}{\ltitvar{}}}},
where
{\ltiMu{\ltitvar{}}{\ltiFn{\ltitvar{}}{\ltitvar{}}}}
stands for
the given type of {\ltivar{}}.


Distinguishing between a type error
and running out of fuel in SCL would significantly complicate the model.
One way this might be able to be achieved is based on a technique for
proving type soundness for big-step reduction relations.
For each type and subtype rule, we add extra rules to explicitly
handle all cases where the derivation can ``get stuck,''
and return a special fuel value denoting ``ran out of fuel'' in rules
that get stuck when checking for sufficient fuel.
Then, another set of rules will propagate this special fuel value
back to the root of the derivation, which allows us to distinguish
the two cases.

% something totally different? Symbolic closures as primitives + don't go wrong

Up till now, our conjectures and discussion have been centered around compiling SCL to \ltiFsub.
Another direction we could take is to lift all restrictions
on SCL (except symbolic reduction fuel, to keep it decidable),
ignore elaborations, and treat SCL as a type \emph{checker} rather than a
type inferencer.
Then, we could attempt to prove a type soundness theorem
like the following.

\begin{conjecture}[Unrestricted SCL Type Soundness]
  If there exists fuel {\ltiFuel{}} such that
     \ltitSstkjudgementNoElab{\ltiFuel{}}
                       {\ltiEmptyEnv}
                       {\ltiE{}}
                       {\ltiT{}}
                       {\ltiFuelp{}}
                       {\ltiEp{}}
  for external term \ltiE{}, and
                       SCL type \ltiT{},
                      then
  evaluating {\ltiE{}}
                       yields a value $v$, whose type \ltiTp{}
                       is a subtype of
                       {\ltiT{}} up-to symbolic closure types.
\end{conjecture}

This better reflects the original intended use-case of symbolic closures:
optional type systems with evaluation semantics based on type-erasure.
Indeed, if symbolic closures were to be added to a language like Typed Clojure,
there is no compelling reason to implement elaboration rules and so it is
natural to leave symbolic closures unrestricted. While we believe unrestricted
symbolic closures are type sound, we are unsure how to approach proving such
a theorem.

% - Solution
%   - "obvious" function annotations
%     - can be derived from usage context
%   - introduce "symbolic" closure types
%     - a function's type is its code + typed local scope
%   - don't need to check a function that isn't called
% - Constraints
%   - wildcard "?" type
%     - needed to provide argument types while inferring body
%     - from Colored LTI
%   - Infinite loops
%     - subtyping
%     - type generalization
%     - term reduction limits
%   - user-level story
%     - symbolic closures enabled by flag
%     - users cannot write a symbolic closure
%     - that way, global annotations cannot contain a symbolic closure
%       - helps with polymorphism story
%         - constraint solving
%           - hypothesis/goal: only one side of contraint solving can have a symbolic closure
%             - one side is from global annotation, other side from local inference
%     - compatible with occurrence typing
%   - reporting errors
%     - suggesting types
%     - avoid showing inlining to users
%   - checking fn with arguments at type Bot is equivalent (?) to not checking at all
%     - what about strange disjoint ordered intersection types like `into`
%     - do they break? do they need initial Bot arities?
%   - 0-n checks to same function
%     - avoid double expansion
%       - many copies of symbolic closures are made, could be expanded at different times
%         - how to synchronize?
%     - skipping unreachable functions
%       - potential for latent bugs, if type checker turned off in future and fn is made reachable
%     - performance
%   - consistent evaluation results
%     - how to ensure correct inlining?
%     - relationship between inlined and evaluated code?
%       - do we want to "undo" the inlining when finally evaluating?
%   - when to use a closure type?
%     - partial annotations
%   - polymorphism
%     - postpone discussion to next chapter
%   - applying symbolic analysis to infer loop/recur annotations
%     - similar issues
%     - different type generalization story?
%   - comparison to let-polymorphism
%     - expressiveness
%     - performance
%   - help check macros?
%     - not directly applicable, since too much context would be lost
%       - would help check *more* of an expansion, but error messages
%         are still unrelated to original code
%   - is this a sound strategy?
%     - faithfully simulates beta reduction
%     - termination story?
%   - case studies
%     - criteria:
%       - good errors?
%       - predictable behavior?
%       - performance?
%     - simple eta expansions
%       - (+ 1 2)
%       - ((fn [x y] (+ x y)) 1 2)
%     - let-bound functions
%       - (+ 1 2)
%       - (let [plus (fn [x y] (+ x y))]
%           (plus 1 2))
%     - y-combinator
%       - stress test
%     - let-polymorphism worst case (exponential) comparison
%       - stress test
%     - completely inlined transducers
%       - case study: inlining map + comp
%         - why: non recursive polymorphic functions
%           - common idiom
%         - how to report errors?
%   - polymorphic function-intersection types
%     - how to handle, do we need backtracking?
%     - do we need to recheck arguments? 

%\chapter{Type Argument Synthesis with Symbolic Closures}
\label{chapter:symbolic:directed-lti}

\subsection{Type-argument synthesis for the Symbolic Closure Language}

\begin{figure}
  \begin{mathpar}
    \infer[AppInf]
    {
    \ltitSstkjudgementNoElab{\ltiCombinedThreadedEnv{1}}
                      {\ltiEnv{}}
                      {\ltiF{}}
                      {\ltiPoly{\ova{\ltitvar{}}}
                               {\ltiFn{\ltiT{}}{\ltiS{}}}}
                      {\ltiCombinedThreadedEnv{2}}
                      {\ltiFp{}}
                  \\
    \ltitSstkjudgementNoElab{\ltiCombinedThreadedEnv{2}}
                      {\ltiEnv{}}
                      {\ltiE{}}
                      {\ltiTp{}}
                      {\ltiCombinedThreadedEnv{3}}
                      {\ltiEp{}}
                  \\
                       |\ova{\ltitvar{}}|>0
           \\
           \ltigenconstraint{\varnothing}{\ova{\ltitvar{}}}{\ltiTp{}}{\ltiT{}}{\ltiCp{}}
           \\
           \ltiprocessDelays{\ltiCombinedThreadedEnv{3}}
                            %{\ltiEnvConcatParen{\ltiEnv{}}{\ova{\ltitvar{}}}}
                            {\ltiCp{}}
                            {\ltiC{}}
                            {\ltiCombinedThreadedEnv{4}}
           \\
           \ltiSubst{\ltiC{}}{\ltiFn{\ltiT{}}{\ltiS{}}}{\ltisubst{}}
    }
    {
    \ltitSstkjudgementNoElab{\ltiCombinedThreadedEnv{1}}
                      {\ltiEnv{}}
                      {\ltiapp{\ltiF{}}{\ltiE{}}}
                      {\ltiApplySubst{\ltisubst{}}{\ltiS{}}}
                      {\ltiCombinedThreadedEnv{4}}
                      {\ltiappinst{\ltiFp{}}
                                  {\ova{\ltiApplySubst{\ltisubst{}}
                                                      {\ltitvar{}}}}
                                  {\ltiEp{}}}
    }
  \end{mathpar}
  \caption{Type argument synthesis for the Symbolic Closure Language}
\end{figure}

\begin{figure}
$$
\begin{array}{lrll}
  \ltiC{} &::=& \ltiCSet{\ova{\ltiCEntry{\ltiT{}}{\ltitvar{}}{\ltiT{}}}\ 
                         \ova{\ltiDEntryVX{\ltiV{}}{\ova{\ltitvar{}}}{\ltiT{}}{\ltiT{}}}
                          }
                      &\mbox{Constraint sets}\\
                      % TODO talk about X/V constraint sets
   \ltiCEmpty &\Leftrightarrow& \ltiCSet{\ova{\ltiCEntry{\ltiBot}{\ltitvar{}}{\ltiTop}}}
                      &\mbox{Constraint abbreviations}
\end{array}
$$
  \caption{Syntax for Constraint generation}
\end{figure}

\begin{figure}
  \begin{mathpar}
    \infer [CG-Top]
    {}
    {
    \ltigenconstraint{\ltiV{}}{\overline{\ltitvar{}}}{\ltiT{}}{\ltiTop}{\ltiCEmpty}
    }

    \infer [CG-Upper]
    {
    \ltitvar{1} \in \overline{\ltitvar{}}
    \\
    \ltidemote{\ltiS{}}{\ltiV{}}{\ltiT{}}
    \\\\
    \ltitv{\ltiS{}} \cap \overline{\ltitvar{}} = \varnothing
    }
    {
    \ltigenconstraint{\ltiV{}}{\overline{\ltitvar{}}}{\ltitvar{1}}{\ltiS{}}
                     {\ltiCSet{\ltiCEntry{\ltiBot}{\ltitvar{1}}{\ltiT{}}}}
    }

    \infer [CG-Lower]
    {
    \ltitvar{1} \in \overline{\ltitvar{}}
    \\
    \ltipromote{\ltiS{}}{\ltiV{}}{\ltiT{}}
    \\\\
    \ltitv{\ltiS{}} \cap \overline{\ltitvar{}} = \varnothing
    }
    {
    \ltigenconstraint{\ltiV{}}{\overline{\ltitvar{}}}{\ltiS{}}{\ltitvar{1}}
                     {\ltiCSet{\ltiCEntry{\ltiT{}}{\ltitvar{1}}{\ltiTop}}}
    }

    \infer [CG-Bot]
    {}
    {
    \ltigenconstraint{\ltiV{}}{\overline{\ltitvar{}}}{\ltiBot}{\ltiT{}}{\ltiCEmpty}
    }
    \ \ \ 
    %
    \infer [CG-Refl]
    {
      \ltitvarp{}
      \not\in
      \overline{\ltitvar{}}
    }
    {
    \ltigenconstraint{\ltiV{}}
                     {\overline{\ltitvar{}}}
                     {\ltitvarp{}}
                     {\ltitvarp{}}
                     {\ltiCEmpty}
    }

    \infer [CG-Fun]
    {
    |\ova{\ltitvar{}}|>0 \text{ implies \ova{\ltiTp{}}, \ova{\ltiT{}} contain no symbolic closures}
    \\
    \ltigenconstraint{\ltiV{} \cup \overline{\ltitvar{}}}
                     {\ova{\ltitvarp{}}}
                     {\ltiT{1}}
                     {\ltiTp{1}}
                     {\ltiC{1}}
    \\
    \ltigenconstraint{\ltiV{} \cup \overline{\ltitvar{}}}
                     {\ova{\ltitvarp{}}}
                     {\ltiT{2}}
                     {\ltiTp{2}}
                     {\ltiC{2}}
    \\\\
    \overline{\ltitvar{}} \cap (\ltiV{} \cup \overline{\ltitvarp{}}) = \varnothing
    }
    {
    \ltigenconstraint{\ltiV{}}
                     {\overline{\ltitvarp{}}}
                     {\ltiPolyFn{\ltiTp{1}}{\ova{\ltitvar{}}}{\ltiT{2}}}
                     {\ltiPolyFn{\ltiT{1}}{\ova{\ltitvar{}}}{\ltiTp{2}}}
                     {\ltiCIntersect{\ltiC{1}}{\ltiC{2}}}
    }

    \infer [CG-Closure]
    {}
    {
    \ltigenconstraint{\ltiV{}}
                     {\overline{\ltitvarp{}}}
                     {\ltiClosureWithStkID{\ltiEnv{}}{\ltiClosureID{}}{\ltiufun{\ltivar{}}{\ltiE{}}}}
                     {\ltiPolyFn{\ltiS{}}{\ova{\ltitvar{}}}{\ltiT{}}}
                     {\ltiCSet{\ltiDEntryVX{\ltiV{}}
                                         {\overline{\ltitvarp{}}}
                                         {\ltiClosureWithStkID{\ltiEnv{}}{\ltiClosureID{}}{\ltiufun{\ltivar{}}{\ltiE{}}}}
                                         {\ltiPoly{\ova{\ltitvar{}}}{\ltiFn{\ltiS{}}{\ltiT{}}}}}}
    }
  \end{mathpar}
  \caption{Constraint generation system
                 \ltigenconstraint{\ltiV{}}{\overline{\ltitvar{}}}{\ltiT{}}{\ltiS{}}{\ltiC{}}
                 where $\ltiV{} \cap {\overline{\ltitvar{}}} = \varnothing$.
  }
\end{figure}

\begin{figure}
  \begin{mathpar}

    \boxed{
    \infer[]
    {
      \ltiprocessDelays{\ltiCombinedThreadedEnv{}}
                       {\ltiC{}}
                       {\ltiCp{}}
                       {\ltiCombinedThreadedEnvp{}}
      \\\\
      \text{Process all delayed constraints in
                       \ltiC{}, yielding a new constraint set \ltiCp{}.
      }
    }
    {}
    }

    \infer[]
    {
      \ltiorderDelays{\ltiC{0}}
                     {\ova
                       {\ltiDEntryVX{\ltiV{}}
                                    {\ova{\ltitvarp{}}}
                                    {\ltiT{}}
                                    {\ltiS{}}}^n}
                                    \\
      \overrightarrowcaption{
      \ltiprocessDelay{\ltiCombinedThreadedEnv{i-1}}
                      {\ltiDEntryVX{\ltiV{i}}
                                   {\ova{\ltitvarp{}}_i}
                                   {\ltiT{i}}
                                   {\ltiS{i}}}
                      {\ltiC{i-1}}
                      {\ltiC{i}}
                      {\ltiCombinedThreadedEnv{i}}
                      }^{1 \leq i \leq n}
    }
    {
      \ltiprocessDelays{\ltiCombinedThreadedEnv{0}}
                       {\ltiC{0}}
                       {\ltiC{n}}
                       {\ltiCombinedThreadedEnv{n}}
    }

    \boxed{
    \infer[]
    {
      \ltiorderDelays{\ltiC{}}{\ova{\ltiDEntryVX{\ltiV{}}{\ova{\ltitvar{}}}{\ltiT{}}{\ltiS{}}}}
      \\\\
      \text{Returns the delayed constraints in constraint set \ltiC{} topologically
      sorted by type variable dependency.
      }
      \\\\
      \text{eg. \ltiDEntryVX{}{}{...}{\ltiFn{\ltitvar{1}}{\ltitvar{2}}}
      goes before 
            \ltiDEntryVX{}{}{...}{\ltiFn{\ltitvar{2}}{\ltitvar{3}}}
      }
    }
    {}
    }

    \infer[]
    {
      \ova{\ltitvar{}}^m = \ova{\ltitvarp{}}_1 = \ova{\ltitvarp{}}_2 = ... = \ova{\ltitvarp{}}_{i-1} = \ova{\ltitvarp{}}_i
      \\
      \forall i \in 1...n, j \in 1...m.
          \{\ltivariance{\ltitvar{j}}{\ltiS{i}}, \ltivariance{\ltitvar{j}}{\ltiT{i}}\} \subseteq \{\ltivconstant, \ltivcovariant\}
      \\
      \text{let \ova{k} be a permutation of 1...n st. }
        \forall i,j \in 1...n.
        \text{ if }
        \ltitv{\ltiT{{k_i}}} \cap \ltitv{\ltiS{{k_j}}} \cap \ova{\ltitvarp{}} \not= \varnothing
        \text{ then }
        % not \leq, eg. [a -> a] depends on itself
        i < j 
    }
    {
      \ltiorderDelays{\ltiCSet{...,\ova{\ltiDEntryVX{\ltiV{}}
                                             {\ova{\ltitvarp{}}^m}
                                             {\ltiClosureWithStkID{\ltiEnv{}}{\ltiClosureID{}}{\ltiufun{\ltivar{}}{\ltiE{}}}}
                                             {\ltiFn{\ltiS{}}{\ltiT{}}}}^n}}
                                             {
      [\ltiDEntryVX{\ltiV{}}
                   {\ova{\ltitvarp{}}}
                   {\ltiClosureWithStkID{\ltiEnv{}}{\ltiClosureID{}}{\ltiufun{\ltivar{}}{\ltiE{}}}}
                   {\ltiFn{\ltiS{}}{\ltiT{}}}_i
                   |
                   i \in \ova{k}]
                   }
    }

    \boxed{
    \infer[]
    {
      \ltiprocessDelay{\ltiCombinedThreadedEnv{}}
                      {\ltiDEntryVX{\ltiV{}}
                                   {\ova{\ltitvarp{}}}
                                   {\ltiS{}}
                                   {\ltiT{}}}
                      {\ltiC{}}
                      {\ltiCp{}}
                      {\ltiCombinedThreadedEnvp{}}
      \\\\
      \text{Process delayed constraint 
                      {\ltiDEntryVX{\ltiV{}}
                                   {\ova{\ltitvarp{}}}
                                   {\ltiS{}}
                                   {\ltiT{}}},
                                   with current constraint set \ltiC{},
      }
      \\\\
      \text{yielding a new constraint set \ltiCp{}.}
    }
    {}
    }

    \infer[]
    {
    \ltitv{\ltiE{}} \cap \ova{\ltitvarpp{}} = \varnothing
    \\
            0 < \ltiFuel{} \\
            \ltiSubst{\ltiC{}}{\ltiPolyFn{\ltiS{}}{}{\ltiT{}}}{\ltisubst{}}\\
            \ltitSstkjudgement{\ltimakeCombinedThreadedEnv{\ltiFuel{}-1}
                                                          {\ltiClosureCache{1}}}
                              {\ltiEnvConcat{\ltiEnv{}}
                                            {\hastype{\ltivar{}}
                                                     {\ltiApplySubst{\ltisubst{}}{\ltiS{}}}}}
                              {\ltiE{}}
                              {\ltiTp{}}
                              {\ltimakeCombinedThreadedEnv{\ltiFuelp{}}
                                                          {\ltiClosureCache{2}}}
                              {\ltiFpp{}}
                              \\
          \ltiupdateClosureCacheSingle{\ltiClosureCache{2}}
                                {\ltiClosureID{}}
                                {\ltifuntparamargrettype
                                 {\ova{\ltitvarpp{}}}
                                 {\ltivar{}}
                                 {\ltiApplySubst{\ltisubst{}}{\ltiS{}}}
                                 {\ltiTp{}}
                                 {\ltiFpp{}}}
                                {\ltiClosureCache{3}}
                                \\
          \ltigenconstraint{\ltiV{} \cup \overline{\ltitvarpp{}}}
                           {\ova{\ltitvar{}}}
                           {\ltiTp{}}
                           {\ltiApplySubst{\ltisubst{}}{\ltiT{}}}
                           {\ltiCpp{}}
                           \\
                           \ltiCpp{} \text{ does not contain delayed constraints}
    }
    {
      \ltiprocessDelay{\ltimakeCombinedThreadedEnv{\ltiFuel{}}{\ltiClosureCache{1}}}
                      {\ltiDEntryVX{\ltiV{}}
                                   {\ova{\ltitvarp{}}}
                                   {\ltiClosureWithStkID{\ltiEnv{}}{\ltiClosureID{}}{\ltiufun{\ltivar{}}{\ltiE{}}}}
                                   {\ltiPolyFn{\ltiS{}}{\ova{\ltitvarpp{}}}{\ltiT{}}}}
                      {\ltiC{}}
                      {\ltiCIntersect{\ltiC{}}{\ltiCpp{}}}
                      {\ltimakeCombinedThreadedEnv{\ltiFuelp{}}{\ltiClosureCache{3}}}
    }

  \end{mathpar}
  \caption{Processing delayed constraints
  }
\end{figure}

{
\begin{lstlisting}[language=ml,mathescape=true]
type Option[a] = {match : $\ltiPoly{\text{r}}%
                                   {\ltiFn{\text{OptionVisitor[a,r]}}%
                                          {\text{r}}}$}
type OptionVisitor[a,r] =
  {caseNone : $\ltiFn{}{\text{r}}$,
   caseSome : $\ltiFn{\text{a}}{\text{r}}$}
\end{lstlisting}
}

{
\begin{lstlisting}[language=ml,mathescape=true]
None = $\ltifuntparaminterfaceLHS{\text{s}}%
                                 {\ltiFn{}%
                                        {\text{Option[s]}}}%
                                 {}$
         {match = $\ltiufun{\text{v}}%
                           {\text{v.caseNone()}}$}
(* SC annotated *)
(* $\ltiInferred{\ltiClosureCache{} =%
     \ltiClosureCacheEntry{\text{c1}}%
                          {\ltiClosure{\text{s}}%
                                      {\ltiNotInferred{\ltifuntparaminterface{\ltiInferred{\text{r}}}%
                                                                             {\ltiInferred{\ltiFn{\text{OptionVisitor[s,r]}}{\text{r}}}}%
                                                                             {\text{v}}%
                                                                             {\text{v.caseNone()}}}}}}$ *)
None = $\ltifuntparaminterfaceLHS{\text{s}}%
                                 {\ltiFn{}{\text{Option[s]}}}%
                                 {}$
         {match = $\ltiufunelab{\ltiInferred{\text{c1}}}%
                               {\text{v}}%
                               {\text{v.caseNone()}}$}
(* fully annotated *)
None = $\ltifuntparaminterfaceLHS{\text{s}}%
                                 {\ltiFn{}%
                                        {\text{Option[s]}}}%
                                 {}$
         {match = $\ltifuntparaminterface{\ltiInferred{\text{r}}}%
                                         {\ltiInferred{\ltiFn{\text{OptionVisitor[s,r]}}{\text{r}}}}%
                                         {\text{v}}%
                                         {\text{v.caseNone()}}$}
\end{lstlisting}
}

{
\begin{lstlisting}[language=ml,mathescape=true]
Some = $\ltifuntparaminterfaceLHS{\text{t}}{\ltiFn{\text{t}}{\text{Option[t]}}}{\text{y}}$
         {match = $\ltiufun{\text{v}}{\text{v.caseSome(y)}}$}
(* SC annotated *)
(* $\ltiInferred{\ltiClosureCache{} = \ltiClosureCacheEntry{\text{c1}}{\ltiClosure{\ltiEnvConcat{\text{t}}{\hastype{\text{y}}{\text{t}}}}{\ltiNotInferred{\ltifuntparaminterface{\ltiInferred{\text{r}}}{\ltiInferred{\ltiFn{\text{OptionVisitor[t,r]}}{\text{r}}}}{\text{v}}{\text{v.caseSome(y)}}}}}}$ *)
Some = $\ltifuntparaminterfaceLHS{\text{t}}{\ltiFn{\text{t}}{\text{Option[t]}}}{\text{y}}$
         {match = $\ltiufunelab{\ltiInferred{\text{c1}}}{\text{v}}{\text{v.caseSome(y)}}$}
(* fully annotated *)
Some = $\ltifuntparaminterfaceLHS{\text{t}}{\ltiFn{\text{t}}{\text{Option[t]}}}{\text{y}}$
         {match = $\ltifuntparaminterface{\ltiInferred{\text{r}}}{\ltiInferred{\ltiFn{\text{OptionVisitor[t,r]}}{\text{r}}}}{\text{v}}{\text{v.caseSome(y)}}$}
\end{lstlisting}
}

{
\begin{lstlisting}[language=ml,mathescape=true]
map = $\ltifuntparaminterfaceLHS{\text{c,d}}{\ltiFn{\ltiFn{\text{c}}{\text{d}},\text{Option[c]}}{\text{Option[d]}}}{\text{f,x}}$
        x.match({caseNone = $\ltiufun{}{\text{None()}}$,
                 caseSome = $\ltiufun{\text{y}}{\text{Some(f(y))}}$})
(* SC annotated *)
(* $\ltiInferred{\ltiEnv{} = {\ltiEnvConcat{\text{c}}{\ltiEnvConcat{\text{d}}{\ltiEnvConcat{\hastype{\text{f}}{\ltiFn{\text{c}}{\text{d}}}}{\hastype{\text{x}}{\text{Option[c]}}}}}}}$ *)
(* $\ltiInferred{\ltiClosureCache{} =}$
     $\ltiInferred{\ltiClosureCacheEntry{\text{c1}}{\ltiClosure{\ltiEnv{}}{\ltiNotInferred{\ltifuninterface{\ltiInferred{\ltiFn{}{\ltilstOption{\ltiBot}}}}{}{\text{None[\ltiInferred{\ltiBot}]()}}}}}}$,
     $\ltiInferred{\ltiClosureCacheEntry{\text{c2}}{\ltiClosure{\ltiEnv{}}{\ltiNotInferred{\ltifuninterface{\ltiInferred{\ltiFn{\text{c}}{\ltilstOption{\text{d}}}}}{\text{y}}{\text{Some[\ltiInferred{\text{d}}](f(y))}}}}}}$
*)
map = $\ltifuntparaminterfaceLHS{\text{c,d}}{\ltiFn{\ltiFn{\text{c}}{\text{d}},\text{Option[c]}}{\text{Option[d]}}}{\text{f,x}}$
        x.match[$\ltiInferred{\text{d}}$]({caseNone = $\ltiufunelab{\ltiInferred{\text{c1}}}{}{\text{None()}}$,
                    caseSome = $\ltiufunelab{\ltiInferred{\text{c2}}}{\text{y}}{\text{Some(f(y))}}$})
(* fully annotated *)
map = $\ltifuntparaminterfaceLHS{\text{c,d}}{\ltiFn{\ltiFn{\text{c}}{\text{d}},\text{Option[c]}}{\text{Option[d]}}}{\text{f,x}}$
        x.match[$\ltiInferred{\text{d}}$]({caseNone = $\ltifuninterface{\ltiInferred{\ltiFn{}{\ltilstOption{\ltiBot}}}}{}{\text{None[\ltiInferred{\ltiBot}]()}}$,
                    caseSome = $\ltifuninterface{\ltiInferred{\ltiFn{\text{c}}{\ltilstOption{\text{d}}}}}{\text{y}}{\text{Some[\ltiInferred{\text{d}}](f(y))}}$})
\end{lstlisting}
}

{
\begin{lstlisting}[language=ml,mathescape=true]
map($\ltiufun{\text{y}}{\text{1+y}}$, Some(42))
(* SC annotated *)
(* $\ltiInferred{\ltiClosureCache{} =%
      \ltiClosureCacheEntry{\text{c1}}%
                           {\ltiClosure{\ltiEmptyEnv}%
                                       {\ltiNotInferred%
                                        {\ltifuninterface{\ltiInferred{\ltiFn{\text{Int}}{\text{Int}}}}%
                                                         {\text{y}}%
                                                         {\text{1+y}}}}}}$ *)
map[$\ltiInferred{\text{Int, Int}}$]($\ltiufunelab{\ltiInferred{\text{c1}}}{\text{y}}{\text{1+y}}$, Some[$\ltiInferred{\text{Int}}$](42))
(* fully annotated *)
map[$\ltiInferred{\text{Int, Int}}$]($\ltifuninterface{\ltiInferred{\ltiFn{\text{Int}}{\text{Int}}}}{\text{y}}{\text{1+y}}$, Some[$\ltiInferred{\text{Int}}$](42))
\end{lstlisting}
}



{
\begin{lstlisting}[language=ml,mathescape=true]
id = $\ltifuntparaminterface{\text{a}}{\ltiFn{\text{a}}{\text{a}}}{\text{x}}{\text{x}}$

let app = $\ltiufun{\text{f},\text{x}}{\ltiapp{\text{f}}{\text{x}}}$ in
  $\ltiapp{\text{app}}%
          {\text{id}, \text{1}}$
(* SC annotated *)
(* $\ltiInferred{\ltiClosureCache{} =%
      \ltiClosureCacheEntry{\text{c1}}%
                           {\ltiClosure{\ltiEmptyEnv}%
                                       {\ltiNotInferred%
                                        {\ltifuninterface{\ltiInferred{\ltiFn{\ltiPoly{\text{a}}{\ltiFn{\text{a}}{\text{a}}},\text{Int}}%
                                                                             {\text{Int}}}}%
                                                         {\text{f,x}}%
                                                         {\ltiappinst{\text{f}}{\ltiInferred{\text{Int}}}{\text{x}}}}}}}$ *)
let app = $\ltiufunelab{\text{c1}}{\text{f},\text{x}}{\ltiapp{\text{f}}{\text{x}}}$ in
  $\ltiapp{\text{app}}%
          {\text{id}, \text{1}}$
(* Fully annotated *)
let app = $\ltifuninterface{\ltiInferred{\ltiFn{\ltiPoly{\text{a}}{\ltiFn{\text{a}}{\text{a}}},\text{Int}}%
                                               {\text{Int}}}}%
                           {\text{f},\text{x}}%
                           {\ltiappinst{\text{f}}{\ltiInferred{\text{Int}}}{\text{x}}}$ in
  $\ltiapp{\text{app}}%
          {\text{id}, \text{1}}$
\end{lstlisting}
}


%\chapter{Recursive types and Intersection types}

\subsection{Compiling to \ltiFsub}

There are a least two ways to approach our external language
with symbolic closures compiling to \ltiFsub.
The main challenge and incentive is to remove all symbolic
closure types from our program and to replace them with
explicit annotations.
Our first approach is to copy function code and annotate
parameter types at each usage.
This approach seems promising, except there is a tension between
copying function code and creating runtime closures.
Our second approach is to extend \ltiFsub with intersection types
and ascribe each function an intersection type that 
describes all the ways it was used as a symbolic closure.

We now explore our first approach, to copy function code as they
are checked.
First, for those functions ascribed symbolic closure types, we
copy their code and insert them at their usage sites.
For example, \clj{f} in 

\begin{lstlisting}[language=Clojure]
(let [f (fn [x] x)]
  (f 1)
  (f "a"))
\end{lstlisting}

would be given a symbolic closure type, and
the program would be expanded like so:

\begin{lstlisting}[language=Clojure]
(let [f (fn [x] x)]
  ((fn [x] x) 1)
  ((fn [x] x) "a"))
\end{lstlisting}

First, we notice that all occurrences of \clj{f} have disappeared,
so it is safe to assume \clj{f} will never be called. It seems
reasonable to us to annotate its argument as \clj{Bot}.

\begin{lstlisting}[language=Clojure]
(let [f (fn [x :- Bot] x)]
  ((fn [x] x) 1)
  ((fn [x] x) "a"))
\end{lstlisting}

Next, the two unannotated functions would be checked as symbolic closures.
These checks would succeed, and then we could ascribe a parameter type on
each function.

\begin{lstlisting}[language=Clojure]
(let [f (fn [x :- Bot] x)]
  ((fn [x :- Int] x) 1)
  ((fn [x :- Str] x) "a"))
\end{lstlisting}

This program type checks with the rules of \ltiFsub.
However, subtle variations on this program are much
more puzzling to account for.

If \clj{f} closes over a variable, like \clj{v}
here

\begin{lstlisting}[language=Clojure]
(let [f (let [v 1] (fn [x] (print v) x))]
  (f 1)
  (f "a"))
\end{lstlisting}

then simply copying its function code will
not suffice.
If we do so, \clj{v} is no longer in scope:

\begin{lstlisting}[language=Clojure]
(let [f (let [v 1] (fn [x] (print v) x))]
  ((fn [x] (print v) x) 1)
  ((fn [x] (print v) x) "a"))
\end{lstlisting}

We could imagine inlining the value of 
\clj{v} to work around this issue, but this is not a
full solution.
If instead, \clj{v} was an annotated function parameter
as in the next example

\begin{lstlisting}[language=Clojure]
(let [f (fn [v :- Int]
          (fn [x] (print v) x))]
  ((f 42) 1)
  ((f 42) "a"))
\end{lstlisting}

it's even unclear how to inline the
unannotated function at all.
It seems the only choice is to inline \clj{f}
in its entirety, regardless if it was a symbolic closure.

\begin{lstlisting}[language=Clojure]
(let [f (fn [v :- Int]
          (fn [x] (print v) x))]
  (((fn [v :- Int]
      (fn [x] (print v) x))
    42)
   1)
  (((fn [v :- Int]
      (fn [x] (print v) x))
    42)
   "a"))
\end{lstlisting}

This way we can at least annotate the missing \clj{Int} and \clj{Str}
parameter types.
It is now tempting to inline \emph{all} local variables with their
definitions.
This doesn't work in a language with side effects.
For example, inlining \clj{f} to ``fix'' the issues
in one of our previous examples would repeat side
effects, like printing \clj{"I only print once"} in the following
program (it, instead, prints thrice if \clj{f} is inlined).

\begin{lstlisting}[language=Clojure]
(let [f (let [v 1]
          (print "I only print once.")
          (fn [x] (print v) x))]
  (f 1)
  (f "a"))
\end{lstlisting}

Symbolic closures can also get their types by being passed
to annotated functions.
For example, \clj{id}
gets the type \clj{[Int -> Int]}
by being passed to \clj{f}, whose definition we will treat
as opaque, emulating a top-level function.

\begin{lstlisting}[language=Clojure]
(let [f (fn [g :- [Int -> Int]] ...)
      id (fn [x] x)]
  (f id))
\end{lstlisting}

By inlining \clj{id} and annotating its parameter \clj{Int},
this program does not pose any particular challenge.

\subsubsection{Intersection types}

If we allow ordered function intersection types,
featured in several optional type systems,
we arrive at an impasse.
Here, we assert that \clj{g} must \emph{both}
be
\clj{[Int -> Int]}
and
\clj{[Num -> Num]}.

\begin{lstlisting}[language=Clojure]
(let [f (fn [g :- (IFn [Int -> Int] [Num -> Num])] ...)
      id (fn [x] x)]
  (f id))
\end{lstlisting}

Clearly \clj{id} inhabits both these types,
however attempting to inline its definition
gets us nowhere.
Introducing intersection types gives us
one more trick up our sleeve: ascribing
\clj{id} as an \emph{intersection} of function types.

\begin{lstlisting}[language=Clojure]
(let [f (fn [g :- (IFn [Int -> Int] [Num -> Num])] ...)
      id (fn [x :- Bot] x)]
  (f (ann (fn [x] x)
          (IFn [Int -> Int]
               [Num -> Num]))))
\end{lstlisting}

This seems to help immensely, but now it seems a waste
to inline \clj{id} at all.
Instead, we could simplify this program by only annotating \clj{id}'s
right-hand-side.

\begin{lstlisting}[language=Clojure]
(let [f (fn [g :- (IFn [Int -> Int] [Num -> Num])] ...)
      id (ann (fn [x] x)
              (IFn [Int -> Int]
                   [Num -> Num]))]
  (f id))
\end{lstlisting}

We can use this technique to help check our previous examples
without having to work around closed-over variables.
For example, our thrice-printing example requires
a single intersection type annotation, based on
the two symbolic reductions of \clj{f}.

\begin{lstlisting}[language=Clojure]
(let [f (let [v 1]
          (print "I only print once.")
          (ann (fn [x] (print v) x)
               (IFn [Int -> Int]
                    [Str -> Str])))]
  (f 1)
  (f "a"))
\end{lstlisting}

Unfortunately, inferring ordered intersection types for unannotated
functions introduce other issues,
most prominently determining the ``best'' ordering of
arities.

Compared to unordered intersections, ordered intersections
have a simple application rule: first arity wins.
A ``best'' ordering would yield the same or more accurate return type
for every possible application, compared to every other ordering.
For now, we assume that a best ordering
exists for every ordered function intersection type, although
we are not sure.

There are several interesting cases we sketch to give an idea
of the character of this algorithm.
First, if arity 1's domain is a subtype of arity 2's
domain, then arity 1 should come first,
using subtyping of the range to break ties in a similar fashion.
For example, \clj{[Int -> Int]}
goes before  \clj{[Num -> Int]},
but
\clj{[Int -> Int]}
precedes
\clj{[Int -> Num]}.
Second, if the domains of two arities
are incomparable via subtyping, their ordering does not matter.
For example,
\clj{[(Pair Int Num) -> Int]},
\clj{[(Pair Num Int) -> Num]},
and
\clj{[Int -> Num]}
may occur in any order in relation to each other.
This also accounts for multiple arguments,
by considering them as a list passed to a single argument.

Going back to checking programs, we now explore some
other ways in which inferring ordered function intersections has
interesting interactions with \ltiFsub.
The next example is similar to the one that helped motivate
intersection types earlier, except we omit the annotation
on \clj{f}.

\begin{lstlisting}[language=Clojure]
(let [f (fn [g]
          (g 1)
          (g "a"))
      id (fn [x] x)]
  (f id))
\end{lstlisting}

Here, the annotation on \clj{g} is the interesting part.

\begin{lstlisting}[language=Clojure]
(let [f (fn [g :- (IFn [Int -> Int] [Str -> Str])]
          (g 1)
          (g "a"))
      id (ann (fn [x] x)
              (IFn [Int -> Int] [Str -> Str]))]
  (f id))
\end{lstlisting}

Unfortunately, we can't retroactively annotate all programs in this way.
Take the following program.

\begin{lstlisting}[language=Clojure]
(let [f (fn [g]
          (fn [x]
            (g x)))]
  ((f (fn [y] y)) 1)
  ((f (fn [z] z)) "a"))
\end{lstlisting}

Ideally, we would give \clj{f} the polymorphic type
\clj{(All [a] [[a -> a] -> [a -> a]])}.
Then, we would use the type variable to annotate its return
as \clj{[a -> a]}.
Instead, we infer \clj{f} as type

\begin{lstlisting}[language=Clojure]
(IFn [[Int -> Int] -> [Int -> Int]]
     [[Str -> Str] -> [Str -> Str]])
\end{lstlisting}

This means that \clj{f}'s body will be checked twice.
The first time, \clj{g} will be assumed \clj{[Int -> Int]},
and then return checked as \clj{[Int -> Int]}.
The second, \clj{g} will be assumed \clj{[Str -> Str]},
and then return checked as \clj{[Str -> Str]}.

The problem now is annotating the function in \clj{f}'s body once-and-for-all.
It inhabits the type \clj{(U [Int -> Int] [Str -> Str])},
but that is too broad to be compatible with \clj{f}'s return---on the other hand,
it does not inhabit \clj{(IFn [Int -> Int] [Str -> Str])},
because \clj{g} cannot accept both \clj{Int} and \clj{Str}.
To handle these cases, we borrow \emph{conditional types} from TypeScript.

A conditional type is type-level dependency between types.
It is of the form \clj{(if (subtype? S T) U V)},
and returns type \clj{U} if \clj{S} is a subtype of \clj{T},
and \clj{V} otherwise.
This construct is particularly useful in combination with
the ability to reference the types of \emph{variables}.
In TypeScript, the type \clj{typeof f} resolves to the type of \clj{f}
in the current type environment. Here, we equivalently write \clj{(TypeOf f)}.

Applying these new type constructs to our example, we get the following annotation:

\begin{lstlisting}[language=Clojure]
(let [f (ann (fn [g]
               (fn [x :- (if (subtype? (TypeOf g) [Int -> Int]) Int Str)]
                 (g x)))
             (IFn [[Int -> Int] -> [Int -> Int]]
                  [[Str -> Str] -> [Str -> Str]]))]
  ((f (fn [y :- Int] y)) 1)
  ((f (fn [z :- Str] z)) "a"))
\end{lstlisting}

Now when \clj{f}'s first function type is checked,
\clj{g} will be of type \clj{[Int -> Int]}, which annotates
\clj{x} as \clj{Int}.
Correspondingly for the second function type,
\clj{g} will be of type \clj{[Str -> Str]}, which annotates
\clj{x} as \clj{Str} via the conditional type's else-branch.
We note that special consideration of variable shadowing is required when using \clj{TypeOf}---for
example, if \clj{g} was shadowed above, we would be branching on the wrong type.

\subsubsection{Polymorphism}

We have not addressed how symbolic closures interact with polymorphic types.
For now, we consider a restricted subset of polymorphic functions, but which happens to be
common in Clojure code.
Anecdotally, they are higher-order functions that take in functions cannot be iterated.

For example, the \clj{map} function is roughly of type:

\begin{lstlisting}[language=Clojure]
(All [a b]
  [[a -> b] (Seqable a) -> (Seqable b)])
\end{lstlisting}

We can immediately see the data flow by the occurrences of type variables.
The function argument takes an \clj{a} from the collection argument,
and then returns a \clj{b} to the return collection.
The function argument cannot be called on its own output in the body
of \clj{map} because \clj{b} is not compatible with \clj{a}.
We can draw these dependencies as arrows---notice that there is no arrow
from \clj{b} to \clj{a}. There are implicit dependencies from
\clj{a} to \clj{b} because everything to the right of an arrow type depends
on everything to the left of the arrow (or, output values depend
on input values).


\begin{lstlisting}
(All [a b]
  [[(*@\tikz[overlay,remember picture] \node [] (b) {};@*)a -> (*@\tikz[overlay,remember picture] \node [] (c) {};@*)b] (Seqable (*@\tikz[overlay,remember picture] \node [] (a) {};@*)a) -> (Seq (*@\tikz[overlay,remember picture] \node [] (d) {};@*)b)])
\end{lstlisting}
\begin{tikzpicture}[remember picture, overlay,
                  text width = 2.5cm ]
  \coordinate (Start1) at (a);
  \coordinate (End1) at (b);
  \coordinate (Start2) at (c);
  \coordinate (End2) at (d);
  \draw[red,->,bend right=-45](Start1.south) to (End1.east);
  \draw[blue,->,bend right=-45](Start2.east) to (End2.north east);
\end{tikzpicture} 

Inferring the data flow is crucial to checking symbolic closures.
Take the following example, where the function argument is inferred
as a symbolic closure.

\begin{lstlisting}[language=Clojure]
(map (fn [x] x) [1 2 3])
\end{lstlisting}

We now have two jobs: to infer the type arguments to \clj{map}
and to infer the type of \clj{x}.
Both can be found simultaneously by solving constraints
to find optimal instantiations for \clj{a} and \clj{b}.
First, we collect the constraints that make
\clj{(Closure \{\} (fn [x] x))}
a subtype of
\clj{[a -> b]}.
By checking the function with annotations
\clj{(fn [x :- a] x)},
we know that \clj{a} flows into \clj{b}, so
we get the constraint
\clj{Bot <: a <: b}.
For the second argument to \clj{map}
we infer
\clj{Int <: a <: Top}.
Since both type variables occur invariantly, we use their smallest instantiations,
so the optimal solution to these constraints
is \clj{a = Int, b = Int}.
We use this substitution to both
both provide the type arguments to \clj{map} (via \clj{inst})
and function argument (by substituting away the \clj{a} that
exercised the symbolic closure).


\begin{lstlisting}[language=Clojure]
((inst map Int Int) (fn [x :- Int] x) [1 2 3])
\end{lstlisting}

We can combine this inference technique with the same approach
we used to check our previous let-bound function examples,
like in the following code.

\begin{lstlisting}[language=Clojure]
(let [f (fn [x] x)]
  (map f [1 2 3])
  (map f ["a" "b" "c"]))
\end{lstlisting}

Here, we infer type arguments for each usage of \clj{map},
and combine the information collected for \clj{f} from both
inferences into an intersection type, yielding:

\begin{lstlisting}[language=Clojure]
(let [f (ann (fn [x] x)
             (IFn [Int -> Int]
                  [Str -> Str]))]
  ((inst map Int Int) f [1 2 3])
  ((inst map Str Str) f ["a" "b" "c"]))
\end{lstlisting}

Furthermore, this approach plays nicely with inferring conditional types,
like in the next example:

\begin{lstlisting}[language=Clojure]
(let [f (fn [g]
          (fn [x]
            (map g x)))]
  ((f (fn [y] y)) [1 2 3])
  ((f (fn [z] z)) ["a" "b" "c"]))
\end{lstlisting}

Now we must both infer an annotation for \clj{x} and
the type arguments to \clj{map}, but, as
in example motivating conditional types,
this is aggravated by \clj{f} being given a
intersection type, and thus forcing its body to be checked twice.
The solution is to to use more conditional types, particularly
as in the instantiation of \clj{map}:

\begin{lstlisting}[language=Clojure]
(let [f (ann (fn [g]
               (fn [x :- (Seqable (if (subtype? (TypeOf g) [Int -> Int]) Int Str))]
                 ((inst map
                        (if (subtype? (TypeOf x) (Seqable Int)) Int Str)
                        (if (subtype? (TypeOf x) (Seqable Int)) Int Str))
                  g x)))
             (IFn [[Int -> Int] -> [(Seqable Int) -> (Seqable Int)]]
                  [[Str -> Str] -> [(Seqable Str) -> (Seqable Str)]]))]
  ((f (fn [y :- Int] y)) [1 2 3])
  ((f (fn [z :- Str] z)) ["a" "b" "c"]))
\end{lstlisting}

\subsubsection{Inferring Conditional types}

The most attractive use of conditional types is to check the
same piece of code at different types.

\begin{figure}
$$
\begin{array}{lrll}
  \ltiE{}, \ltiF{} &::=& ... \alt
                         \ltifuntparaminterface{\ova{\ltitvar{}}}
                                               {\ova
                                                {\ltistackmapping{\ltiEnv{}}
                                                                 {\ltiIFn{\ova{\ltiFn{\ltiT{}}{\ltiT{}}}}}}}
                                               {\ltivar{}}
                                               {\ltiE{}}
                         \alt
                         \ltiappinst{\ltiF{}}{\ova{\ltistackmapping{\ltiEnv{}}{\ova{\ltiR{}}}}}{\ltiE{}} \alt
                      &\mbox{Terms} \\
  \ltiappinst{\ltiF{}}{\ova{\ltiR{}}}{\ltiE{}} &\Leftrightarrow&
         \ltiappinst{\ltiF{}}{\ltistackmapping{\ltiEmptyEnv{}}{\ova{\ltiR{}}}}{\ltiE{}}\\
   \ltifuntparaminterface{\ova{\ltitvar{}}}
                         {\ltiT{}}
                         {\ltivar{}}
                         {\ltiE{}}
         &\Leftrightarrow&
   \ltifuntparaminterface{\ova{\ltitvar{}}}
                         {\ltistackmapping{\ltiEmptyEnv{}}{\ltiT{}}}
                         {\ltivar{}}
                         {\ltiE{}}
                      &\mbox{Term abbreviations} \\
  \ltiT{}, \ltiS{}, \ltiR{} &::=& ...
                         \alt
                         \ltiMu{\ltitvar{}}{\ltiT{}}
                         \alt 
                         \ltiIFn{\ova{\ltiFn{\ltiT{}}{\ltiT{}}}}
                      &\mbox{Types} \\
  \ltiSubtypeSeen{} &::=& \ova{\ltiSeenEntry{\ltiT{}}{\ltiT{}}}
                      &\mbox{Subtype Seen List} \\

\end{array}
$$
\caption{Internal Language Syntax Extensions}
\label{symbolic:figure:internal-language-mu-intersection}
\end{figure}

\begin{figure}
  \begin{mathpar}
    \infer [I-AppInst]
    {
    \ltitjudgement{\ltiEnv{}}
                  {\ltiF{}}
                  {\ltiT{}^f}
                  {\ltiFp{}}
                    \\
    \ltitjudgement{\ltiEnv{}}
                  {\ltiE{}}
                  {\ltiT{}^a}
                  {\ltiEp{}}
                  \\
         \ltiunfold{\ltiT{}^f}
                   {\ltiPoly{\ova{\ltitvar{}}}{\ltiIFn{\ova{\ltiFn{\ltiT{}}{\ltiS{}}}^n}}}
                  \\\\
                  \exists i \in 1...m.
                        \ltiunifyContexts{\ltiInternalOrExternalLang{}}{\ltistackmapping{\ltiEnvp{i}}{\ova{\ltiRp{}}_i}}{\ltiEnv{}}{\ova{\ltiR{}}} \}
                  \\\\
                  \ova{\ltiSp{}}
                  =
                  \left\{ {\ltireplace{\ova{\ltiR{}}}{\ova{\ltitvar{}}}{\ltiS{i}}}
                  \middle| i \in 1 ... n, 
                  \text{ if }
                  \ltiisubtype{\ltiEnv{}}{\ltiT{}^a}{\ltireplace{\ova{\ltiR{}}}{\ova{\ltitvar{}}}{\ltiT{i}}}
                  \right\}
                  \\
                  |\ova{\ltiSp{}}|>0
    }
    {
      \ltitjudgement{\ltiEnv{}}
                    {\ltiappinst{\ltiF{}}
                                {\ova{\ltistackmapping{\ltiEnvp{}}{\ova{\ltiRp{}}}}^m}
                                {\ltiE{}}}
                    {\ltiMeetMany{\ova{\ltiSp{}}}}
                    {\ltiappinst{\ltiFp{}}
                                {\ova{\ltistackmapping{\ltiEnvp{}}{\ova{\ltiRp{}}}}^m}
                                {\ltiEp{}}}
    }

    \infer [I-App\Bot]
    {
    \ltitjudgement{\ltiEnv{}}
                  {\ltiF{}}
                  {\ltiT{}}
                  {\ltiFp{}}
                  \\\\
    \ltitjudgement{\ltiEnv{}}
                  {\ltiE{}}
                  {\ltiS{}}
                  {\ltiEp{}}
                  \\\\
    \ltiunfold{\ltiT{}}{\ltiBot}
    }
    {
    \ltitjudgement{\ltiEnv{}}
                  {\ltiappinst{\ltiF{}}{\ova{\ltistackmapping{\ltiEnvp{}}{\ova{\ltiR{}}}}}{\ltiE{}}}
                  {\ltiBot{}}
                  {\ltiappinst{\ltiFp{}}{\ova{\ltistackmapping{\ltiEnvp{}}{\ova{\ltiR{}}}}}{\ltiEp{}}}
    }

    \infer [I-Abs]
    { 
    \exists i \in 1...m.
    \ltiunifyContexts{\ltiInternalOrExternalLang{}}
                     {\ltistackmapping{\ltiEnvp{i}}{\ltiTp{i}}}
                     {\ltiEnv{}}
                     {\ltiIFn{\ova{\ltiFn{\ltiT{}}{\ltiS{}}}^n}}
    \\\\
    n>0
    \\
    \overrightarrow{
    \ltitjudgement{\ltiEnvConcat{\ltiEnv{}}
                                {\ltiEnvConcat{\ova{\ltitvar{}}}
                                              {\hastype{\ltivar{}}{\ltiT{i}}}}}
                  {\ltiE{}}
                  {\ltiSp{i}}
                  {\ltiF{i}}
                  \ \ \
          \ltiisubtype{\ltiEnv{}}{\ltiSp{i}}{\ltiS{i}}
                  }^{1 \leq i \leq n}
          \\\\
          \ltimergeTaggedTermsLHS{\ltiE{}}{\ltimergeTaggedTermsLHS{\ltiF{1}}{\ltimergeTaggedTermsLHS{...}{\ltiF{n}}}}
          = \ltiEp{}
    }
    {
    \ltitjudgement{\ltiEnv{}}
                  {\ltifuntparaminterface{\ova{\ltitvar{}}}
                                         {\ova{\ltistackmapping{\ltiEnvp{}}{\ltiTp{}}}^m}
                                         {\ltivar{}}
                                         {\ltiE{}}}
                  {\ltiPoly{\ova{\ltitvar{}}}{\ltiIFn{\ova{\ltiFn{\ltiT{}}{\ltiS{}}}^n}}}
                  {\ltifuntparaminterface{\ova{\ltitvar{}}}
                                         {\ova{\ltistackmapping{\ltiEnvp{}}{\ltiTp{}}}^m}
                                         {\ltivar{}}
                                         {\ltiEp{}}}
                 }

    \infer [I-Sel]
    {
    \ltitjudgement{\ltiEnv{}}
                  {\ltiE{}}
                  {\ltiS{}}
                  {\ltiF{}}
                     \\\\
    \ltiunfold{\ltiS{}}
              {\ltiRec{\hastype{\ltivar{1}}{\ltiT{1}}, ..., \hastype{\ltivar{i}}{\ltiT{i}} , ..., \hastype{\ltivar{n}}{\ltiT{n}}}}
    }
    {
    \ltitjudgement{\ltiEnv{}}
                  {\ltisel{\ltiE{}}{\ltivar{i}}}
                  {\ltiT{i}}
                  {\ltisel{\ltiF{}}{\ltivar{i}}}
    }

    \infer [I-Sel\ltiBot]
    {
    \ltitjudgement{\ltiEnv{}}
                     {\ltiE{}}
                     {\ltiT{}}
                     {\ltiF{}}
                     \\\\
                     \ltiunfold{\ltiT{}}{\ltiBot}
    }
    {
    \ltitjudgement{\ltiEnv{}}
                  {\ltisel{\ltiE{}}{\ltivar{}}}
                  {\ltiBot}
                  {\ltisel{\ltiF{}}{\ltivar{}}}
    }
  \end{mathpar}
  \caption{Internal language type system extensions
  }
  \label{symbolic:figure:internal-language-type-system-mu-intersection}
\end{figure}

\begin{figure}
  \begin{mathpar}
    \boxed{
    \infer[]
    {}
    {
    \ltiunfold{\ltiT{}}{\ltiS{}}
    \\\\
    \text{ Unfold top-level recursion in \ltiT{} to \ltiS{}.}
    }
    }

    \begin{array}{lcl}
      \ltiunfoldalign{\ltiPoly{\ova{\ltitvar{}}}{\ltiT{}}}
                     {\ltiPoly{\ova{\ltitvar{}}}{\ltiunfoldLHS{\ltiT{}}}}\\
      \ltiunfoldalign{\ltiMu{\ltitvar{}}{\ltiT{}}}
                     {\ltireplace{\ltiMu{\ltitvar{}}{\ltiT{}}}
                                 {\ltitvar{}}
                                 {\ltiunfoldLHS{\ltiT{}}}}
                                                                \\
      \ltiunfoldalign{\ltiT{}}{\ltiT{}} \text{, otherwise}\\
    \end{array}

    \boxed{
    \infer[]
    {}
    {
    \ltimergeTaggedTerms{\ltiE{1}}{\ltiE{2}}{\ltiF{}}
    \\\\
    \text{Merge terms \ltiE{i} as \ltiF{}.}
    }
    }

    \boxed{
    \infer[]
    {}
    {
    \ltiunifyContexts{\ltiInternalOrExternalLang{}}{\ltistackmapping{\ltiEnvp{}}{\ova{\ltiT{}}}}{\ltiEnv{}}{\ova{\ltiS{}}}
    \\\\
    \text{Prepare \ova{\ltiT{}} for use in current context \ltiEnv{}.}
    }
    }

    \begin{array}{lllll}
      \ltiunifyContextsalign{\ltiInternalOrExternalLang{}}
                             {\ltistackmapping{\ltiEmptyEnv{}}{\ova{\ltiT{}}}}
                             {\ltiEnv{}}
                             {\ova{\ltiT{}}} \\
      \ltiunifyContextsalign{\ltiInternalOrExternalLang{}}
                             {\ltistackmapping{\ltiEnvConcatParen{\ltitvarp{}}{\ltiEnvp{}}}{\ova{\ltiT{}}}}
                             {\ltiEnvConcatParen{\ltitvar{}}{\ltiEnv{}}}
                             {\ltiunifyContextsLHS{\ltiInternalOrExternalLang{}}
                                                   {\ltireplace{\ltitvar{}}{\ltitvarp{}}
                                                               {(\ltistackmapping{\ltiEnvp{}}{\ova{\ltiT{}}})}}
                                                   {\ltiEnv{}}}\\
      \ltiunifyContextsalign{\ltiInternalOrExternalLang{}}
                             {\ltistackmapping{\ltiEnvConcatParen{\hastype{\ltivarp{}}{\ltiSp{}}}{\ltiEnvp{}}}{\ova{\ltiT{}}}}
                            {\ltiEnvConcatParen{\hastype{\ltivar{}}{\ltiS{}}}{\ltiEnv{}}}
                            {\ltiunifyContextsLHS{\ltiInternalOrExternalLang{}}
                                                  {\ltistackmapping{\ltiEnvp{}}{\ova{\ltiT{}}}}
                                                  {\ltiEnv{}}},
                                                 &\text{ if } \ltiisubtype{\ltiEnvpp{}}{\ltiS{}}{\ltiSp{}}
    \end{array}

  \begin{array}{llll}
    \ltimergeTaggedTermsalign{\ltifuntparaminterface{\ova{\ltitvar{}}}{\ltiT{}}{\ltivar{}}{\ltiE{1}}}
                             {\ltifuntparaminterface{\ova{\ltitvar{}}}{\ltiT{}}
                                              {\ltivar{}}
                                              {\ltiE{2}}}
                             {\ltifuntparaminterface{\ova{\ltitvar{}}}{\ltiT{}}
                                              {\ltivar{}}
                                              {\ltimergeTaggedTermsLHS{\ltiE{1}}{\ltiE{2}}}}
                             \\
    \ltimergeTaggedTermsalign{\ltiappinst{\ltiF{}}{\ova{\ltistackmapping{\ltiEnv{}}{\ova{\ltiR{}}}}}{\ltiE{}}}
                             {\ltiappinst{\ltiFp{}}{\ova{\ltistackmapping{\ltiEnv{}}{\ova{\ltiR{}}}}}{\ltiEp{}}}
                             {\ltiappinst{\ltimergeTaggedTermsLHS{\ltiF{}}{\ltiFp{}}}
                                         {\ova{\ltistackmapping{\ltiEnv{}}{\ova{\ltiR{}}}}}
                                         {\ltimergeTaggedTermsLHS{\ltiE{}}{\ltiEp{}}}}
                                     \\
    \ltimergeTaggedTermsalign{\ltisel{\ltiE{1}}{\ltivar{}}}
                             {\ltisel{\ltiE{2}}{\ltivar{}}}
                             {\ltisel{\ltimergeTaggedTermsLHS{\ltiE{1}}{\ltiE{2}}}{\ltivar{}}}
                                     \\
    \ltimergeTaggedTermsalign{\ltiRec{\ova{\ltivar{} = \ltiE{}}}}
                             {\ltiRec{\ova{\ltivar{} = \ltiF{}}}}
                             {\ltiRec{\ova{\ltivar{} = \ltimergeTaggedTermsLHS{\ltiE{}}{\ltiF{}}}}}
                                     \\
    \ltimergeTaggedTermsalign{\ltivar{}}
                             {\ltivar{}}
                             {\ltivar{}}
  \end{array}

  \end{mathpar}
  \caption{Extended Type System Metafunctions}
  \label{symbolic:figure:internal-language-metafunctions}
\end{figure}

\begin{figure}
  \begin{mathpar}
    \boxed{
    \infer[]
    {}
    {
      \ltiisubtypeseen{\ltiSubtypeSeen{}}{\ltiEnv{}}{\ltiT{}}{\ltiS{}}
      \\\\
      \text{\ltiT{} is a subtype of \ltiS{},
      }
      \\\\
      \text{with seen queries \ltiSubtypeSeen{}.
                 }
                 }
                 }

    \infer [S-MuL]
    {
     \ltiSeenEntry{\ltiMu{\ltitvar{}}{\ltiT{}}}{\ltiS{}} \in \ltiSubtypeSeen{}
     \\\\
    \text{ or }
     \\\\
    \ltiisubtypeseen{\ltiSeenConcat{\ltiSeenEntry{\ltiMu{\ltitvar{}}{\ltiT{}}}
                                                 {\ltiS{}}}
                                   {\ltiSubtypeSeen{}}}
                    {\ltiEnv{}}
                    {\ltireplace{\ltiMu{\ltitvar{}}{\ltiT{}}}{\ltitvar{}}{\ltiT{}}}{\ltiS{}}
    }
    {
    \ltiisubtypeseen{\ltiSubtypeSeen{}}{\ltiEnv{}}
                    {\ltiMu{\ltitvar{}}{\ltiT{}}}
                    {\ltiS{}}
    }

    \infer [S-MuR]
    {
     \ltiSeenEntry{\ltiS{}}{\ltiMu{\ltitvar{}}{\ltiT{}}} \in \ltiSubtypeSeen{}
     \\\\
    \text{ or }
     \\\\
    \ltiisubtypeseen{\ltiSeenConcat{\ltiSeenEntry{\ltiS{}}
                                                 {{\ltiMu{\ltitvar{}}{\ltiT{}}}}}
                                   {\ltiSubtypeSeen{}}}
                    {\ltiEnv{}}
                    {\ltiS{}}
                    {\ltireplace{\ltiMu{\ltitvar{}}{\ltiT{}}}{\ltitvar{}}{\ltiT{}}}
    }
    {
    \ltiisubtypeseen{\ltiSubtypeSeen{}}
                    {\ltiEnv{}}
                    {\ltiS{}}
                    {\ltiMu{\ltitvar{}}{\ltiT{}}}
    }

    \infer [S-IFn]
    { 
      \forall i \in 1...n.\ 
        \exists j \in 1...m.\ 
          \ltiisubtypeseen{\ltiSubtypeSeen{}}{\ltiEnv{}}{\ltiS{j}}{\ltiT{i}}
    }
    { \ltiisubtypeseen{\ltiSubtypeSeen{}}{\ltiEnv{}}
                      {\ltiIFn{\ova{\ltiS{}}^m}}
                      {\ltiIFn{\ova{\ltiT{}}^n}}
                   }

%    \infer [SF-ContextBoth]
%    {
%     \ltiisubtypeseen{\ltiSubtypeSeen{}}
%                     {\ltiEnvpp{}}
%                     {\ltiunifyContextsLHS{\ltiinternallabel}{\ltistackmapping{\ltiEnvpp{}}{\ltiS{}}}{\ltiEnvp{}}}
%                     {\ltiT{}}
%    }
%    {\ltiisubtypeseen{\ltiSubtypeSeen{}}
%                     {\ltiEnv{}}
%                     {(\ltistackmapping{\ltiEnvp{}}{\ltiS{}})}
%                     {(\ltistackmapping{\ltiEnvpp{}}{\ltiT{}})}
%    }
%
%    \infer [SF-ContextL]
%    {
%     \ltiisubtypeseen{\ltiSubtypeSeen{}}
%                     {\ltiEnv{}}
%                     {\ltiunifyContextsLHS{\ltiinternallabel}{\ltistackmapping{\ltiEnvp{}}{\ltiS{}}}{\ltiEnv{}}}
%                     {\ltiT{}}
%    }
%    {\ltiisubtypeseen{\ltiSubtypeSeen{}}
%                     {\ltiEnv{}}
%                     {(\ltistackmapping{\ltiEnvp{}}{\ltiS{}})}
%                     {\ltiT{}}
%    }
%
%    \infer [SF-ContextR]
%    {
%     \ltiisubtypeseen{\ltiSubtypeSeen{}}
%                     {\ltiEnv{}}
%                     {\ltiS{}}
%                     {\ltiunifyContextsLHS{\ltiinternallabel}{\ltistackmapping{\ltiEnvp{}}{\ltiT{}}}{\ltiEnv{}}}
%    }
%    {\ltiisubtypeseen{\ltiSubtypeSeen{}}
%                     {\ltiEnv{}}
%                     {\ltiS{}}
%                     {(\ltistackmapping{\ltiEnvp{}}{\ltiT{}})}
%    }

  \end{mathpar}
  \caption{Internal language subtyping extensions
  }
  \label{symbolic:figure:internal-language-subtyping-mu-intersection}
\end{figure}

\begin{figure}
  \begin{mathpar}

%    \infer [Var]
%    {}
%    {
%       \ltitSdjudgement{\ltiEnv{}}
%                       {\ltivar{}}
%                       {\ltiEnvLookup{\ltiEnv{}}{\ltivar{}}}
%                       {\ltivar{}}
%                 }

%    \infer [Sel]
%    {
%    \ltitSdjudgement{\ltiEnv{}}
%                    {\ltiF{}}
%                    {\ltiS{}}
%                    {\ltiFp{}}
%                     \\\\
%    \ltiSdsubtype{\ltiEnv{}}{\ltiS{}}{\ltiRec{\hastype{\ltivar{1}}{\ltiT{1}},...,\hastype{\ltivar{i}}{\ltiT{i}},...,\hastype{\ltivar{n}}{\ltiT{n}}}}
%    }
%    {
%    \ltitSdjudgement{\ltiEnv{}}
%                  {\ltisel{\ltiF{}}{\ltivar{i}}}
%                  {\ltiT{i}}
%                  {\ltisel{\ltiFp{}}{\ltivar{i}}}
%    }
%
%    \infer [Rec]
%    {
%    \overrightarrow{
%    \ltitSdjudgement{\ltiEnv{}}
%                    {\ltiF{i}}
%                    {\ltiT{i}}
%                    {\ltiFp{i}}
%                    }
%                    ^{1 \leq i \leq n}
%    }
%    {
%    \ltitSdjudgement{\ltiEnv{}}
%                    {\ltiRec{\ova{\ltivar{} = \ltiF{}}^n}}
%                    {\ltiRec{\ova{\hastype{\ltivar{}}{\ltiT{}}}^n}}
%                    {\ltiRec{\ova{\ltivar{} = \ltiFp{}}^n}}
%    }

    \infer [E-AppInf]
    {
    \ltitjudgement{\ltiEnv{}}
                    {\ltiF{}}
                    {\ltiS{}^f}
                    {\ltiFp{}}
                    \\
    \ltitjudgement{\ltiEnv{}}
                    {\ltiE{}}
                    {\ltiS{}^a}
                    {\ltiEp{}}
                    \\\\
          \ltiisubtype{\ltiEnv{}}
                    {\ltiS{}^f}
                    {\ltiPoly{\ova{\ltitvar{}}}
                             {\ltiIFn{\ltiFn{\ltiT{}}{\ltiS{}}}}}
                  \\
                       |\ova{\ltitvar{}}|>0
                  \\\\
                  \forall \ltiRp{}.
                    \left(
                    \begin{array}{lll}
                      \ltiisubtype{\ltiEnv{}}{\ltiS{}^a}{\ltireplace{\ova{\ltiRp{}}}{\ova{\ltitvar{}}}{\ltiT{}}}
                      \text{ implies}
                      \arcr
                      \ltiisubtype{\ltiEnv{}}{\ltireplace{\ova{\ltiR{}}}{\ova{\ltitvar{}}}{\ltiS{}^a}}
                                   {\ltireplace{\ova{\ltiRp{}}}{\ova{\ltitvar{}}}{\ltiS{}^a}}
                    \end{array}
                  \right)
    }
    {
    \ltitjudgement{\ltiEnv{}}
                    {\ltiapp{\ltiF{}}{\ltiE{}}}
                    {\ltireplace{\ova{\ltiR{}}}{\ova{\ltitvar{}}}{\ltiS{}}}
                    {\ltiappinst{\ltiFp{}}
                                {\ltistackmapping{\ltiEnv{}}{\ova{\ltiR{}}}}
                                {\ltiEp{}}}
    }

    \infer [E-UAbs]
    { 
    \exists i \in 1...m.
    \ltiunifyContexts{\ltiInternalOrExternalLang{}}
                     {\ltistackmapping{\ltiEnvp{i}}{\ltiTp{i}}}
                     {\ltiEnv{}}
                     {\ltiIFn{\ova{\ltiFn{\ltiT{}}{\ltiS{}}}^n}}
    \\\\
    n>0
    \\\\
    \overrightarrow{
    \ltitjudgement{\ltiEnvConcat{\ltiEnv{}}
                                {\ltiEnvConcat{\ova{\ltitvar{}}}
                                              {\hastype{\ltivar{}}{\ova{\ltiT{i}}}}}}
                  {\ltiE{}}
                  {\ltiSp{i}}
                  {\ltiF{i}}
                  \ \ \
          \ltiisubtype{\ltiEnv{}}{\ltiSp{i}}{\ltiS{i}}
                  }^{1 \leq i \leq n}
    \\\\
          \ltimergeTaggedTermsLHS{\ltiE{}}{\ltimergeTaggedTermsLHS{\ltiF{1}}{\ltimergeTaggedTermsLHS{...}{\ltiF{n}}}} = \ltiEp{}
    }
    {
    \ltitjudgement{\ltiEnv{}}
                  {\ltiufun{\ltivar{}}{\ltiE{}}}
                  {\ltiPoly{\ova{\ltitvar{}}}{\ltiIFn{\ova{\ltiFn{\ltiT{}}{\ltiS{}}}^n}}}
                  {\ltifuntparaminterface{\ova{\ltitvar{}}}
                                         {\ova{\ltistackmapping{\ltiEnvp{}}{\ltiTp{}}}}
                                         {\ltivar{}}
                                         {\ltiEp{}}}
                 }

%    \infer [AppInst]
%    {
%    \ltitSdjudgement{\ltiEnv{}}
%                    {\ltiF{}}
%                    {\ltiT{}^f}
%                    {\ltiFp{}}
%                    \\
%    \ltitSdjudgement{\ltiEnv{}}
%                    {\ltiE{}}
%                    {\ltiTp{}}
%                    {\ltiEp{}}
%                  \\\\
%                  \ltiSdsubtype{\ltiEnv{}}{
%         \ltiresolveLHS{\ltiexternallanglabel}
%                    {\ltiEnv{}}
%                    {\ltiT{}^f}}
%                    {
%                    {\ltiPoly{\ova{\ltitvar{}}}{\ltiSplitIFn{\ova{\ltiFn{\ltiT{}}{\ltiS{}}}^n}
%                                                            {\ova{\ltiContextualFn{\ltiEnvpp{}}{\ltiTpp{}}{\ltiSpp{}}}}}}
%                                                            }
%                  \\\\
%                  m > 0
%                  \\
%                  \ltiLfindTA{\ltiexternallanglabel}{\ltiEnv{}}{\ova{\ltitvar{}}}{\ova{\ltistackmapping{\ltiEnvp{}}{\ova{\ltiRp{}}}}}{\ova{\ltiR{}}}
%                  \\\\
%                  \ova{\ltiSp{}}^m
%                  =
%                  \{ \ltiS{i}\ |\ i \in 1 ... n, \ltiSdsubtype{\ltiEnv{}}{\ltiT{}^a}{\ltireplace{\ova{\ltiR{}}}{\ova{\ltitvar{}}}{\ltiT{i}}}
%                  \}
%    }
%    {
%    \ltitSdjudgement{\ltiEnv{}}
%                    {\ltiappinst{\ltiF{}}
%                                {\ova{\ltistackmapping{\ltiEnvp{}}{\ova{\ltiRp{}}}}}
%                                {\ltiE{}}}
%                    {\ltiMeetMany{\ova{\ltireplace{\ova{\ltiR{}}}{\ova{\ltitvar{}}}{\ltiSp{}}}}}
%                    {\ltiappinst{\ltiFp{}}
%                                {\ova{\ltistackmapping{\ltiEnvp{}}{\ova{\ltiRp{}}}}}
%                                {\ltiEp{}}}
%    }


%    \infer [Abs]
%    {
%    \ltiunfold{\ltiTp{}}{\ltiPoly{\ova{\ltitvar{}}}
%                                    {\ltiSplitIFn{\ova{\ltiFn{\ltiT{}}{\ltiS{}}}^{1...n}}
%                                                 {\ova{\ltiContextualFn{\ltiEnvp{}}{\ltiT{}}{\ltiS{}}}^{n+1...m}}}}
%    \\\\
%                     m>0
%                     \\
%    \overrightarrowcaption{
%     \ltitSdjudgement{\ltiEnvConcat{\ltiEnv{}}
%                                   {\ltiEnvConcat{\ova{\ltitvar{}}}
%                                                 {\hastype{\ltivar{}}{\ltiT{i}}}}}
%                     {\ltiEp{0}}
%                     {\ltiSp{i}}
%                     {\ltiE{i}}
%                     }^{1 \leq i \leq n}
%                     \\\\
%    \overrightarrowcaption{
%     \ltitSdjudgement{\ltiEnvConcat{\ltiEnvMissingTVarsLHS{\ltiEnv{}}{\ltiEnvp{i}}}
%                                   {\ltiEnvConcat{\ova{\ltitvar{}}}
%                                                 {\hastype{\ltivar{}}
%                                                          {\ltiT{i}}}}}
%                     {\ltiEp{0}}
%                     {\ltiSp{i}}
%                     {\ltiE{i}}
%                     }^{n < i \leq m}
%                     \\\\
%                     \overrightarrowcaption{\ltiSdsubtype{\ltiEnv{}}{\ltiSp{i}}
%                                                  {\ltiS{i}}
%                                                  ,
%                                                  \ 
%                                                  \ltimergeTaggedTerms{\ltiEp{i-1}}{\ltiE{i}}{\ltiEp{i}}
%                                                  }^{1 \leq i \leq m}
%    }
%    {
%    \ltitSdjudgement{\ltiEnv{}}
%                    {\ltifuninterface{\ltiTp{}}{\ltivar{}}{\ltiEp{0}}}
%                    {\ltiTp{}}
%                    {\ltifuninterface{\ltiTp{}}{\ltivar{}}{\ltiEp{m}}}
%                 }
  \end{mathpar}

  \caption{Extended External Language Specification
  }
  \label{symbolic:figure:external-language-declarative-type-system-mu-intersection}
\end{figure}


\begin{figure}
  \begin{mathpar}
    \boxed{
    \infer[]
    {}
    {
    \ltimergeTaggedTerms{\ltiE{1}}{\ltiE{2}}{\ltiF{}}
    \\\\
    \text{Merge terms \ltiE{i} as \ltiF{} (extends \figref{symbolic:figure:internal-language-metafunctions}).}
    }
    }

  \begin{array}{llll}
    \ltimergeTaggedTermsalign{\ltifuntparaminterface{\ova{\ltitvar{}}}
                                                    {\ova{\ltistackmapping{\ltiEnv{}}{\ltiT{}}}^n}
                                                    {\ltivar{}}{\ltiE{1}}}
                             {\ltifuntparaminterface{\ova{\ltitvar{}}}
                                                    {\ova{\ltistackmapping{\ltiEnv{}}{\ltiT{}}}^{n+1...m}}
                                                    {\ltivar{}}
                                                    {\ltiE{2}}}
                             {\ltifuntparaminterface{\ova{\ltitvar{}}}
                                                    {\ova{\ltistackmapping{\ltiEnv{}}{\ltiT{}}}^m}
                                                    {\ltivar{}}
                                                    {\ltimergeTaggedTermsLHS{\ltiE{1}}{\ltiE{2}}}}
                             \\
    \ltimergeTaggedTermsalign{\ltiappinst{\ltiF{}}{\ova{\ltistackmapping{\ltiEnv{}}{\ova{\ltiR{}}}}^n}{\ltiE{}}}
                             {\ltiappinst{\ltiFp{}}{\ova{\ltistackmapping{\ltiEnv{}}{\ova{\ltiR{}}}}^{n+1...m}}{\ltiEp{}}}
                             {\ltiappinst{\ltimergeTaggedTermsLHS{\ltiF{}}{\ltiFp{}}}
                                         {\ova{\ltistackmapping{\ltiEnv{}}{\ova{\ltiR{}}}}^m}
                                         {\ltimergeTaggedTermsLHS{\ltiE{}}{\ltiEp{}}}}
    \end{array}
  \end{mathpar}

  \caption{Extended External Language Metafunctions
  }
  \label{symbolic:figure:external-language-metafunctions}
\end{figure}

\begin{figure}
$$
\begin{array}{lrll}
  \ltiE{}, \ltiF{} &::=& ... \alt
                         \ltiufunelab{\ova{\ltiufunelabentry{\ltiClosureID{}}}}
                                     {\ltivar{}}
                                     {\ltiE{}}
                      &\mbox{Terms} \\
  \ltiClosureCache{} &::=& \ova{\ltiClosureCacheEntry
                                {\ltiClosureID{}}
                                {\ltiClosure{\ltiEnv{}}
                                            {\ltifuntparaminterface
                                             {\ova{\ltitvar{}}}
                                             {\ova{\ltistackmapping{\ltiEnv{}}{\ltiT{}}}}
                                             {\ltiE{}}}}}
                      &\mbox{Closure Cache} \\
  \ltiCombinedThreadedEnv{} &::=& \ltimakeCombinedThreadedEnv{\ltiFuel{}}{\ltiClosureCache{}}
                      &\mbox{Combined Threaded Environments} \\
\end{array}
$$
\caption{Extended Symbolic Closure Language Syntax (extends \figref{symbolic:figure:external-language-syntax-mu-intersection})}
\label{symbolic:figure:SC-language-syntax}
\end{figure}

\begin{figure}
  \begin{mathpar}
    % TODO thread seen subtypings
    \boxed
    {
    \infer[]
    {}
    {
    \ltitSstkjudgement{\ltiClosureCache{}}
                      {\ltiEnv{}}
                      {\ltiE{}}
                      {\ltiT{}}
                      {\ltiClosureCachep{}}
                      {\ltiEp{}}
                     \\\\
                     \text{Given symbolic closure cache \ltiClosureCache{}
                     and context \ltiEnv{}, external term \ltiE{} 
                     is of symbolic-closure-language type \ltiT{}
                     }
                     \\\\
                     \text{
                     with updated symbolic closure cache \ltiClosureCachep{},
                     and elaborated symbolic-closure-language term \ltiEp{}.
                     }
                     }
                     }

    \infer [AppInst]
    {
    \ltitSstkjudgement{\ltiClosureCache{1}}
                      {\ltiEnvp{}}
                      {\ltiF{}}
                      {\ltiT{}^f}
                      {\ltiClosureCache{2}}
                      {\ltiFp{}}
                  \\
    \ltitSstkjudgement{\ltiClosureCache{2}}
                      {\ltiEnvp{}}
                      {\ltiE{}}
                      {\ltiT{}^a}
                      {\ltiClosureCachep{0}}
                      {\ltiEp{}}
                  \\
    \ltiunfold{\ltiT{}^f}
              {\ltiPoly{\ova{\ltitvar{}}}
                       {\ltiIFn{\ova{\ltiFn{\ltiT{}}{\ltiS{}}}^n}}}
                  \\\\
                  \exists i \in 1...m.
                        \ltiunifyContextsSC{\ltiClosureCachep{0}}
                                           {\ltistackmapping{\ltiEnv{i}}{\ova{\ltiRp{}}_i}}
                                           {\ltiEnvp{}}
                                           {\ova{\ltiR{}}}
                                           {\ltiClosureCachep{1}}
    \\
    \ltiClosureCachepp{0} = \ltiClosureCachep{n}
    \\
    \ova{\ltiSpp{}}_0 = \varnothing
                   \\
                   |\ova{\ltiSpp{}}_n| > 0
    \\\\
    \overrightarrowcaption{
      (\ova{\ltiSpp{}}_i, \ltiClosureCachepp{i})
        = \left\{
                     \begin{array}{llll}
                       (\ova{\ltiSpp{}}_{i-1}{\ltireplace{\ova{\ltiR{}}}{\ova{\ltitvar{}}}{\ltiSp{i}}}, \ltiClosureCachepp{i})
                       , &\text{if } 
                       \ltiSsubtype{\ltiClosureCachepp{i-1}}
                                   {\ltiEnvp{}}
                                   {\ltiT{}^a}
                                   {\ltireplace{\ova{\ltiR{}}}{\ova{\ltitvar{}}}{\ltiTp{i}}}
                                   {\ltiClosureCachepp{i}}
                       \arcr
                       (\ova{\ltiSpp{}}_{i-1}, \ltiClosureCachepp{i-1}),  &\text{otherwise}
                     \end{array}
                   \right.
                   }^{1 \leq i \leq n}
    }
    {
    \ltitSstkjudgement{\ltiClosureCache{1}}
                      {\ltiEnvp{}}
                      {\ltiappinst{\ltiF{}}
                                  {\ova{\ltistackmapping{\ltiEnvp{}}{\ova{\ltiRp{}}}}^m}
                                  {\ltiE{}}}
                      {\ltiMeetMany{\ova{\ltiSpp{}}_n}}
                      {\ltiClosureCachepp{n}}
                      {\ltiappinst{\ltiFp{}}
                                  {\ova{\ltistackmapping{\ltiEnvp{}}{\ova{\ltiRp{}}}}^m}
                                  {\ltiEp{}}}
    }

    \infer [AppInf-Closure]
    {
    \ltitSstkjudgement{\ltiClosureCache{1}}
                      {\ltiEnv{}}
                      {\ltiF{}}
                      {\ltiTpp{}}
                      {\ltiClosureCache{2}}
                      {\ltiFp{}}
                  \\
    \ltitSstkjudgement{\ltiClosureCache{2}}
                      {\ltiEnv{}}
                      {\ltiE{}}
                      {\ltiTp{}}
                      {\ltiClosureCache{3}}
                      {\ltiEp{}}
                  \\\\
    \ltiunfold{\ltiTpp{}}{\ltiClosureWithStkID{\ltiEnvp{}}
                                              {\ltiClosureID{}}
                                              {\ltiufun{\ltivar{}}{\ltiEpp{}}}}
                  \\\\
                  \ltilookup{\ltiClosureCache{3}}{\ltiClosureID{}} =
                  \ltiClosureCacheVal{\ltiFuel{}}{\ltiClosureElab{}}
                  \\
    0 < \ltiFuel{}
    \\\\
    \ltitSstkjudgement{\ltimapsto{\ltiClosureCache{3}}{\ltiClosureID{}}{\ltiClosureCacheVal{\ltiFuel{}-1}{\ltiClosureElab{}}}}
                      {\ltiEnvConcat{\ltiEnvp{}}{\hastype{\ltivar{}}{\ltiTp{}}}}
                      {\ltiEpp{}}
                      {\ltiS{}}
                      {\ltiClosureCache{4}}
                      {\ltiFpp{}}
                      \\\\
    \ltiupdateClosureCache{\ltiClosureCache{4}}{\ltiEnv{}}{\ltiClosureID{}}{\varnothing}{\ltiTp{}}{\ltiS{}}{\ltiFpp{}}{\ltiClosureCache{5}}
    }
    {
    \ltitSstkjudgement{\ltiClosureCache{1}}
                      {\ltiEnv{}}
                      {\ltiapp{\ltiF{}}{\ltiE{}}}
                      {\ltiS{}}
                      {\ltiClosureCache{5}}
                      {\ltiappinst{\ltiFp{}}
                                  {}
                                  {\ltiEp{}}}
    }

    \infer [AppInf\Bot]
    {
    \ltitSstkjudgement{\ltiClosureCache{1}}
                      {\ltiEnv{}}
                      {\ltiF{}}
                      {\ltiT{}}
                      {\ltiClosureCache{2}}
                      {\ltiFp{}}
                  \\\\
    \ltitSstkjudgement{\ltiClosureCache{2}}
                      {\ltiEnv{}}
                      {\ltiE{}}
                      {\ltiS{}}
                      {\ltiClosureCache{3}}
                      {\ltiEp{}}
                  \\\\
    \ltiunfold{\ltiT{}}{\ltiBot}
    }
    {
    \ltitSstkjudgement{\ltiClosureCache{1}}
                      {\ltiEnv{}}
                      {\ltiapp{\ltiF{}}{\ltiE{}}}
                      {\ltiTpp{}}
                      {\ltiClosureCache{3}}
                      {\ltiappinst{\ltiFp{}}
                                  {}
                                  {\ltiEp{}}}
    }

    \infer [AppInst\Bot]
    {
    \ltitSstkjudgement{\ltiClosureCache{1}}
                      {\ltiEnv{}}
                      {\ltiF{}}
                      {\ltiT{}}
                      {\ltiClosureCache{2}}
                      {\ltiFp{}}
                  \\
    \ltitSstkjudgement{\ltiClosureCache{2}}
                      {\ltiEnv{}}
                      {\ltiE{}}
                      {\ltiS{}}
                      {\ltiClosureCache{3}}
                      {\ltiEp{}}
                  \\\\
    \ltiunfold{\ltiT{}}{\ltiBot}
    }
    {
    \ltitSstkjudgement{\ltiClosureCache{1}}
                      {\ltiEnv{}}
                      {\ltiappinst{\ltiF{}}
                                  {\ova{\ltistackmapping{\ltiEnv{}}{\ova{\ltiR{}}}}}
                                  {\ltiE{}}}
                      {\ltiTpp{}}
                      {\ltiClosureCache{3}}
                      {\ltiappinst{\ltiFp{}}
                                  {\ova{\ltistackmapping{\ltiEnv{}}{\ova{\ltiR{}}}}}
                                  {\ltiEp{}}}
    }

    \infer [Sel]
    {
    \ltitSstkjudgement{\ltiClosureCache{}}
                      {\ltiEnv{}}
                      {\ltiF{}}
                      {\ltiS{}}
                      {\ltiClosureCachep{}}
                      {\ltiFp{}}
                      \\\\
    \ltiunfold{\ltiS{}}
              {\ltiRec{\hastype{\ltivar{1}}{\ltiT{1}},..., \hastype{\ltivar{i}}{\ltiT{i}},..., \hastype{\ltivar{n}}{\ltiT{n}}}}
    }
    {
    \ltitSstkjudgement{\ltiClosureCache{}}
                      {\ltiEnv{}}
                      {\ltisel{\ltiF{}}{\ltivar{i}}}
                      {\ltiT{i}}
                      {\ltiClosureCachep{}}
                      {\ltisel{\ltiFp{}}{\ltivar{i}}}
    }

    \infer [Sel\Bot]
    {
    \ltitSstkjudgement{\ltiClosureCache{}}
                      {\ltiEnv{}}
                      {\ltiF{}}
                      {\ltiS{}}
                      {\ltiClosureCachep{}}
                      {\ltiFp{}}
                      \\\\
    \ltiunfold{\ltiS{}}{\ltiBot}
    }
    {
    \ltitSstkjudgement{\ltiClosureCache{}}
                      {\ltiEnv{}}
                      {\ltisel{\ltiF{}}{\ltivar{i}}}
                      {\ltiBot}
                      {\ltiClosureCachep{}}
                      {\ltisel{\ltiFp{}}{\ltivar{i}}}
    }

    \infer [Abs]
    {
    \exists i \in 1...m.
    \ltiunifyContexts{\ltiInternalOrExternalLang{}}
                     {\ltistackmapping{\ltiEnvp{i}}{\ltiTp{i}}}
                     {\ltiEnv{}}
                     {\ltiIFn{\ova{\ltiFn{\ltiT{}}{\ltiS{}}}^n}}
                     \\\\
                     n>0
                     \\\\
                     \text{TODO if $|\ova{\ltitvar{}}|>0$, erase SC's in \ltiEnv{}}
                     \\\\
    \overrightarrowcaption{
     \ltitSstkjudgement{\ltiClosureCache{i-1}}
                    {\ltiEnvConcat{\ltiEnv{}}
                                   {\ltiEnvConcat{\ova{\ltitvar{}}}
                                                 {\hastype{\ltivar{}}
                                                          {\ltiT{i}}}}}
                     {\ltiE{}}
                     {\ltiSp{i}}
                     {\ltiF{i}}
                     {\ltiClosureCache{i}}
                     }^{1 \leq i \leq n}
                     \\\\
                     \overrightarrowcaption{
                        \ltiSsubtype{\ltiClosureCache{n+i-1}}
                                                  {\ltiEnv{}}
                                                  {\ltiSp{i}}
                                                  {\ltiS{i}}
                                                  {\ltiClosureCache{n+i}}
                                                  }^{1 \leq i \leq n}
                                                  \\
        \ltimergeTaggedTermsLHS{\ltiE{}}{\ltimergeTaggedTermsLHS{\ltiF{1}}{\ltimergeTaggedTermsLHS{...}{\ltiF{n}}}} = \ltiEp{}
    }
    {
    \ltitSstkjudgement{\ltiClosureCache{0}}
                    {\ltiEnv{}}
                    {\ltifuntparaminterface{\ova{\ltitvar{}}}
                                           {\ova{\ltistackmapping{\ltiEnvp{}}{\ltiTp{}}}^m}
                                           {\ltivar{}}
                                           {\ltiE{}}}
                    {\ltiPoly{\ova{\ltitvar{}}}{\ltiIFn{\ova{\ltiFn{\ltiT{}}{\ltiS{}}}^n}}}
                    {\ltifuntparaminterface{\ova{\ltitvar{}}}
                                           {\ova{\ltistackmapping{\ltiEnvp{}}{\ltiTp{}}}^m}
                                           {\ltivar{}}
                                           {\ltiEp{}}}
                    {\ltiClosureCache{2n}}
                 }

    \infer [UAbs]
    {
    \ltiClosureID{} \not\in dom(\ltiClosureCache{})
    \\\\
    \ltiClosureCachep{}
    =
    \ltimapsto{\ltiClosureCache{}}
              {\ltiClosureID{}}
              {\ltiClosureCacheVal
               {\ltiFuel{0}}
               {\ltiufun{\ltivar{}}{\ltiE{}}}}
               \\
               for some initial fuel {\ltiFuel{0}}
    }
    {
    \ltitSstkjudgement{\ltiClosureCache{}}
                      {\ltiEnv{}}
                      {\ltiufun{\ltivar{}}{\ltiE{}}}
                      {\ltiClosureWithStkID
                                           {\ltiEnv{}}
                                           {\ltiClosureID{}}
                                           {\ltiufun{\ltivar{}}{\ltiE{}}}}
                      {\ltiClosureCachep{}}
                      {\ltiufunelab{\ltiClosureID{}}
                                   {\ltivar{}}
                                   {\ltiE{}}}
                 }
  \end{mathpar}

  \caption{Extended Algorithmic Type system for Symbolic Closure Language (\textsc{AppInf} omitted)
  }
  \label{symbolic:figure:SC-language-algorithmic-type-system-mu-intersection}
\end{figure}

\begin{figure}
  \begin{mathpar}
    \boxed{
    \infer[]
    {}
    {\ltiSsubtypeseen{\ltiSubtypeSeen{}}
                 {\ltiCombinedThreadedEnv{}}
                 {\ltiEnv{}}
                 {\ltiS{}}
                 {\ltiT{}}
                 {\ltiCombinedThreadedEnvp{}}
                 \\\\
                 \text{
                 With closure cache \ltiCombinedThreadedEnv{}
                 and seen subtypings \ltiSubtypeSeen{},
                 \ltiS{} is a subtype of \ltiT{}
                 with updated cache
                 \ltiCombinedThreadedEnvp{}.
                 }
    }
    }

    \infer [S-TVar]
    {}
    {
     \ltiSsubtypeseen{\ltiSubtypeSeen{}}
                 {\ltiCombinedThreadedEnv{}}
                 {\ltiEnv{}}
                 {\ltitvar{}}
                 {\ltitvar{}}
                 {\ltiCombinedThreadedEnv{}}
    }

    \infer [S-Top]
    {}
    { \ltiSsubtypeseen{\ltiSubtypeSeen{}}{\ltiCombinedThreadedEnv{}}{\ltiEnv{}}{\ltiT{}}{\Top}{\ltiCombinedThreadedEnv{}}}

    \infer [S-Bot]
    {}
    { \ltiSsubtypeseen{\ltiSubtypeSeen{}}{\ltiCombinedThreadedEnv{}}{\ltiEnv{}}{\Bot}{\ltiT{}}{\ltiCombinedThreadedEnv{}}}

    \infer [S-MuL]
    {
     (\ltiSeenEntry{\ltiMu{\ltitvar{}}{\ltiT{}}}{\ltiS{}} \in \ltiSubtypeSeen{}
     \text{ and } \ltiCombinedThreadedEnv{} = \ltiCombinedThreadedEnvp{}
     )
     \\\\
    \text{ or }
     \\\\
    \ltiSsubtypeseen{\ltiSeenConcat{\ltiSeenEntry{\ltiMu{\ltitvar{}}{\ltiT{}}}
                                                 {\ltiS{}}}
                                   {\ltiSubtypeSeen{}}}
                    {\ltiCombinedThreadedEnv{}}
                    {\ltiEnv{}}
                    {\ltireplace{\ltiMu{\ltitvar{}}{\ltiT{}}}{\ltitvar{}}{\ltiT{}}}{\ltiS{}}
                    {\ltiCombinedThreadedEnvp{}}
    }
    {
    \ltiSsubtypeseen{\ltiSubtypeSeen{}}
                    {\ltiCombinedThreadedEnv{}}
                    {\ltiEnv{}}
                    {\ltiMu{\ltitvar{}}{\ltiT{}}}
                    {\ltiS{}}
                    {\ltiCombinedThreadedEnvp{}}
    }

    \infer [S-MuR]
    {
     (\ltiSeenEntry{\ltiS{}}{\ltiMu{\ltitvar{}}{\ltiT{}}} \in \ltiSubtypeSeen{}
     \text{ and } \ltiCombinedThreadedEnv{} = \ltiCombinedThreadedEnvp{})
     \\\\
    \text{ or }
     \\\\
    \ltiSsubtypeseen{\ltiSeenConcat{\ltiSeenEntry{\ltiS{}}
                                                 {{\ltiMu{\ltitvar{}}{\ltiT{}}}}}
                                   {\ltiSubtypeSeen{}}}
                    {\ltiCombinedThreadedEnv{}}
                    {\ltiEnv{}}
                    {\ltiS{}}
                    {\ltireplace{\ltiMu{\ltitvar{}}{\ltiT{}}}{\ltitvar{}}{\ltiT{}}}
                    {\ltiCombinedThreadedEnvp{}}
    }
    {
    \ltiSsubtypeseen{\ltiSubtypeSeen{}}
                    {\ltiCombinedThreadedEnv{}}
                    {\ltiEnv{}}
                    {\ltiS{}}
                    {\ltiMu{\ltitvar{}}{\ltiT{}}}
                    {\ltiCombinedThreadedEnvp{}}
    }

    \infer [S-Rec]
    {
    \overrightarrow{\ltiSsubtypeseen{\ltiSubtypeSeen{}}{\ltiCombinedThreadedEnv{i-1}}{\ltiEnv{}}
                                {\ltiT{}}
                                {\ltiS{}}
                                {\ltiCombinedThreadedEnv{i}}
                                }
    }
    {
    \ltiSsubtypeseen{\ltiSubtypeSeen{}}{\ltiCombinedThreadedEnv{0}}
                {\ltiEnv{}}
                {\ltiRec{\ova{\hastype{\ltivar{}}{\ltiT{}}}^n,
                         \ova{\hastype{\ltivarp{}}{\ltiTp{}}}}}
                {\ltiRec{\ova{\hastype{\ltivar{}}{\ltiS{}}}^n}}
                {\ltiCombinedThreadedEnv{n}}
    }

    \infer [S-Poly]
    {
    \text{TODO erase SC's in \ltiS{} and \ltiT{}}
    \\
    \ltiSsubtypeseen{\ltiSubtypeSeen{}}{\ltiCombinedThreadedEnv{}}
                {\ltiEnvConcat{\ltiEnv{}}{\ova{\ltitvar{}}}}
                {\ltiT{}}
                {\ltiS{}}
                {\ltiCombinedThreadedEnvp{}}
    }
    {
    \ltiSsubtypeseen{\ltiSubtypeSeen{}}{\ltiCombinedThreadedEnv{}}
                {\ltiEnv{}}
                {\ltiPoly{\ova{\ltitvar{}}}{\ltiT{}}}
                {\ltiPoly{\ova{\ltitvar{}}}{\ltiS{}}}
                {\ltiCombinedThreadedEnvp{}}
    }

    \infer [S-IFn]
    { 
      \overrightarrowcaption{
        \exists j \in 1...m.\ 
          \ltiSsubtypeseen{\ltiSubtypeSeen{}}
                          {\ltiCombinedThreadedEnv{i-1}}
                          {\ltiEnv{}}
                          {\ltiS{j}}
                          {\ltiT{i}}
                          {\ltiCombinedThreadedEnv{i}}
                          }^{1 \leq i \leq n}
    }
    { \ltiSsubtypeseen{\ltiSubtypeSeen{}}{\ltiCombinedThreadedEnv{0}}{\ltiEnv{}}
                      {\ltiIFn{\ova{\ltiS{}}^m}}
                      {\ltiIFn{\ova{\ltiT{}}^n}}
                      {\ltiCombinedThreadedEnv{n}}
                   }

                   % FIXME don't push in the Poly, just duplicate the logic in SF-Closure
                   % this is because we only get to choose the type arguments once
    \infer [S-Closure]
    {
    \overrightarrowcaption{
    \ltiSsubtypeseen{\ltiSubtypeSeen{}}
                    {\ltiCombinedThreadedEnv{i-1}}
                    {\ltiEnvpp{}}
                    {\ltiClosureWithStkID{\ltiEnv{}}
                                         {\ltiClosureID{}}
                                         {\ltiufun{\ltivar{}}{\ltiE{}}}}
                    {\ltiPoly{\ova{\ltitvar{}}}{\ltiFn{\ltiT{i}}{\ltiS{i}}}}
                    {\ltiCombinedThreadedEnv{i}}
                    }^{1 \leq i \leq n}
    }
    { \ltiSsubtypeseen{\ltiSubtypeSeen{}}{\ltiCombinedThreadedEnv{0}}
                  {\ltiEnvpp{}}
                  {\ltiClosureWithStkID{\ltiEnv{}}
                                       {\ltiClosureID{}}
                                       {\ltiufun{\ltivar{}}{\ltiE{}}}}
                  {\ltiPoly{\ova{\ltitvar{}}}{\ltiIFn{\ova{\ltiFn{\ltiT{}}{\ltiS{}}}^n}}}
                  {\ltiCombinedThreadedEnv{n}}
                  }

    % eg (IFn [Int -> Int] [Number -> Number]) <: [Nothing -> Any]
    \infer [SF-Fn]
    { \ltiSsubtypeseen{\ltiSubtypeSeen{}}{\ltiCombinedThreadedEnv{1}}{\ltiEnv{}}{\ltiS{}}{\ltiSp{}}{\ltiCombinedThreadedEnv{2}}
      \\\\
      \ltiSsubtypeseen{\ltiSubtypeSeen{}}{\ltiCombinedThreadedEnv{2}}{\ltiEnv{}}{\ltiT{}}{\ltiTp{}}{\ltiCombinedThreadedEnv{3}}
    }
    { \ltiSsubtypeseen{\ltiSubtypeSeen{}}{\ltiCombinedThreadedEnv{1}}{\ltiEnv{}}
                  {\ltiFn{\ltiSp{}}{\ltiT{}}}
                  {\ltiFn{\ltiS{}}{\ltiTp{}}}
                  {\ltiCombinedThreadedEnv{3}}
       }

    \infer [SF-Closure]
    {
    0 < \ltiFuel{}
    \\
    \text{TODO propagate \ltiSubtypeSeen{} through type system} 
    \\
    \ltitSstkjudgement{\ltimakeCombinedThreadedEnv{\ltiFuel{}-1}
                                                  {\ltiClosureCache{}}}
                      {\ltiEnvConcat{\ltiEnv{}}{\hastype{\ltivar{}}{\ltiT{}}}}
                      {\ltiE{}}
                      {\ltiSp{}}
                      {\ltiCombinedThreadedEnv{}}
                      {\ltiEp{}}
                      \\\\
    \ltiSsubtypeseen{\ltiSubtypeSeen{}}{\ltiCombinedThreadedEnv{}}{\ltiEnv{}}{\ltiSp{}}{\ltiS{}}
                {\ltimakeCombinedThreadedEnv{\ltiFuelp{}}
                                            {\ltiClosureCachep{}}}
                      \\
    \ltiupdateClosureCache{\ltiClosureCachep{}}{\ltiEnv{}}{\ltiClosureID{}}{\ova{\ltitvar{}}}{\ltiT{}}{\ltiS{}}{\ltiEp{}}{\ltiClosureCachepp{}}
    }
    { \ltiSsubtypeseen{\ltiSubtypeSeen{}}{\ltimakeCombinedThreadedEnv{\ltiFuel{}}
                                              {\ltiClosureCache{}}}
                  {\ltiEnvp{}}
                  {\ltiClosureWithStkID
                                       {\ltiEnv{}}
                                       {\ltiClosureID{}}
                                       {\ltiufun{\ltivar{}}{\ltiE{}}}}
                  {\ltiPoly{\ova{\ltitvar{}}}{\ltiFn{\ltiT{}}{\ltiS{}}}}
                  {\ltimakeCombinedThreadedEnv{\ltiFuelp{}}
                                              {\ltiClosureCachepp{}}}
                  }

%    \infer [SF-ContextBoth]
%    {
%     \ltiunifyContextsSC{\ltiCombinedThreadedEnv{1}}
%                        {\ltistackmapping{\ltiEnvpp{}}{\ltiS{}}}
%                        {\ltiEnvp{}}
%                        {\ltiSp{}}
%                        {\ltiCombinedThreadedEnv{2}}
%     \\\\
%     \ltiSsubtypeseen{\ltiSubtypeSeen{}}
%                     {\ltiCombinedThreadedEnv{2}}
%                     {\ltiEnvpp{}}
%                     {\ltiSp{}}
%                     {\ltiT{}}
%                     {\ltiCombinedThreadedEnv{3}}
%    }
%    {\ltiSsubtypeseen{\ltiSubtypeSeen{}}
%                     {\ltiCombinedThreadedEnv{1}}
%                     {\ltiEnv{}}
%                     {(\ltistackmapping{\ltiEnvp{}}{\ltiS{}})}
%                     {(\ltistackmapping{\ltiEnvpp{}}{\ltiT{}})}
%                     {\ltiCombinedThreadedEnv{3}}
%    }
%    \ 
%%
%    \infer [SF-ContextL]
%    {
%    \ltiunifyContextsSC{\ltiCombinedThreadedEnv{1}}
%                       {\ltistackmapping{\ltiEnvp{}}{\ltiS{}}}
%                       {\ltiEnv{}}
%                       {\ltiSp{}}
%                       {\ltiCombinedThreadedEnv{2}}
%                       \\\\
%     \ltiSsubtypeseen{\ltiSubtypeSeen{}}
%                     {\ltiCombinedThreadedEnv{2}}
%                     {\ltiEnv{}}
%                     {\ltiSp{}}
%                     {\ltiT{}}
%                     {\ltiCombinedThreadedEnv{3}}
%    }
%    {\ltiSsubtypeseen{\ltiSubtypeSeen{}}
%                     {\ltiCombinedThreadedEnv{1}}
%                     {\ltiEnv{}}
%                     {(\ltistackmapping{\ltiEnvp{}}{\ltiS{}})}
%                     {\ltiT{}}
%                     {\ltiCombinedThreadedEnv{3}}
%    }
%    \ 
%%
%    \infer [SF-ContextR]
%    {
%     \ltiunifyContextsSC{\ltiCombinedThreadedEnv{1}}
%                        {\ltistackmapping{\ltiEnvp{}}{\ltiT{}}}
%                        {\ltiEnv{}}
%                        {\ltiTp{}}
%                        {\ltiCombinedThreadedEnv{2}}
%     \\\\
%     \ltiSsubtypeseen{\ltiSubtypeSeen{}}
%                     {\ltiCombinedThreadedEnv{2}}
%                     {\ltiEnv{}}
%                     {\ltiS{}}
%                     {\ltiTp{}}
%                     {\ltiCombinedThreadedEnv{3}}
%    }
%    {\ltiSsubtypeseen{\ltiSubtypeSeen{}}
%                     {\ltiCombinedThreadedEnv{1}}
%                     {\ltiEnv{}}
%                     {\ltiS{}}
%                     {(\ltistackmapping{\ltiEnvp{}}{\ltiT{}})}
%                     {\ltiCombinedThreadedEnv{3}}
%    }

  \end{mathpar}

  \caption{Extended Symbolic Closure Language Subtyping}
  \label{symbolic:figure:SC-language-subtype-mu-intersection}
\end{figure}

\begin{figure}
  \begin{mathpar}
   \boxed{
   \infer[]
   {}
   {
   \ltimergeTaggedTerms{\ltiE{1}}{\ltiE{2}}{\ltiE{3}}
   \\\\
    \text{Merge terms \ltiE{1} and \ltiE{2}}
    \\\\
    \text{
  (extends \figref{symbolic:figure:external-language-metafunctions}).}
   }}

  \begin{array}{llll}
    \ltimergeTaggedTermsalign{\ltiufunelab{\ova{\ltiufunelabentry
                                                            {\ltiClosureID{}}}^n}
                                     {\ltivar{}}
                                     {\ltiE{1}}}
                             {\ltiufunelab{\ova{\ltiufunelabentry
                                                            {\ltiClosureID{}}}^{n+1...m}}
                                     {\ltivar{}}
                                     {\ltiE{2}}}
                             {\ltiufunelab{\ova{\ltiufunelabentry
                                                            {\ltiClosureID{}}}^m
                                                            }
                                     {\ltivar{}}
                                     {\ltimergeTaggedTermsLHS{\ltiE{1}}{\ltiE{2}}}}
  \end{array}

    \boxed{
    \infer[]
    {}
    {\ltiupdateClosureCache{\ltiClosureCache{}}{\ltiEnv{}}{\ltiClosureID{}}{\ova{\ltitvar{}}}{\ltiT{}}{\ltiS{}}{\ltiE{}}{\ltiClosureCachep{}}
    \\\\
    \text{Record symbolic closure \ltiClosureID{} as \ltiPoly{\ova{\ltitvar{}}}{\ltiFn{\ltiT{}}{\ltiS{}}} under}
    \\\\
    \text{application context \ltiEnv{}, with elaboration \ltiE{}.}
    }}

    \infer[]
    {
    % we want to only set type variables once, we probably need to distinguish
    % between zero type variables and a never-exercised closure
    \text{TODO merge type variables}
    \\
    \ltilookup{\ltiClosureCache{}}{\ltiClosureID{}} = 
    \ltiClosure{\ltiEnv{}}
               {\ltifuntparaminterface{\ova{\ltitvar{}}}
                                      {\ltiIFn{\ova{\ltiFn{\ltiTp{}}{\ltiSp{}}}}}
                                      {\ltivar{}}
                                      {\ltiF{}}}
    \\\\
    \ltiClosureCachep{} = 
    \ltimapsto{\ltiClosureCache{}}
              {\ltiClosureID{}}
              {\ltiClosure{\ltiEnv{}}
                          {\ltifuntparaminterface{\ova{\ltitvar{}}}
                                                 {\ltiIFn{\ltiFn{\ltiT{}}{\ltiS{}} \ova{\ltiFn{\ltiTp{}}{\ltiSp{}}}}}
                                                 {\ltivar{}}
                                                 {\ltimergeTaggedTermsLHS{\ltiE{}}{\ltiF{}}}}}
    }
    {\ltiupdateClosureCache{\ltiClosureCache{}}{\ltiEnvp{}}{\ltiClosureID{}}{\ova{\ltitvar{}}}{\ltiT{}}{\ltiS{}}{\ltiE{}}{\ltiClosureCachep{}}
    }

    \boxed{
    \infer[]
    {}
    {
    \ltiunifyContextsSC{\ltiCombinedThreadedEnv{}}{\ltistackmapping{\ltiEnvp{}}{\ova{\ltiT{}}}}{\ltiEnv{}}{\ova{\ltiS{}}}{\ltiCombinedThreadedEnvp{}}
    \text{ Use \ova{\ltiT{}}'s context \ltiEnvp{} to prepare \ova{\ltiT{}} for use in current context \ltiEnv{}.
    }
    }
    }

    \begin{array}{lllll}
      \ltiunifyContextsSCalign{\ltiCombinedThreadedEnv{}}
                              {\ltistackmapping{\ltiEmptyEnv{}}{\ova{\ltiT{}}}}
                              {\ltiEnv{}}
                              {\ltiunifyContextsSCRHS{\ova{\ltiT{}}}
                                                     {\ltiCombinedThreadedEnv{}}}
                                                     \\
      \ltiunifyContextsSCalign{\ltiCombinedThreadedEnv{}}
                              {\ltistackmapping{\ltiEnvConcatParen{\ltitvarp{}}{\ltiEnvp{}}}{\ova{\ltiT{}}}}
                              {\ltiEnvConcatParen{\ltitvar{}}{\ltiEnv{}}}
                              {\ltiunifyContextsSCLHS{\ltiCombinedThreadedEnv{}}
                                                    {\ltireplace{\ltitvar{}}{\ltitvarp{}}
                                                                {(\ltistackmapping{\ltiEnvp{}}{\ova{\ltiT{}}})}}
                                                    {\ltiEnv{}}}\\
      \ltiunifyContextsSCalign{\ltiCombinedThreadedEnv{}}
                              {\ltistackmapping{\ltiEnvConcatParen{\hastype{\ltivarp{}}{\ltiSp{}}}{\ltiEnvp{}}}{\ova{\ltiT{}}}}
                              {\ltiEnvConcatParen{\hastype{\ltivar{}}{\ltiS{}}}{\ltiEnv{}}}
                              {\ltiunifyContextsSCLHS{\ltiCombinedThreadedEnvp{}}
                                                     {\ltistackmapping{\ltiEnvp{}}{\ova{\ltiT{}}}}
                                                     {\ltiEnv{}}},
                                                 \text{ if } \ltiSsubtype{\ltiCombinedThreadedEnv{}}
                                                                          {\ltiEnvpp{}}
                                                                          {\ltiS{}}
                                                                          {\ltiSp{}}
                                                                          {\ltiCombinedThreadedEnvp{}}
    \end{array}
  \end{mathpar}
  \caption{Metafunctions for Extended Symbolic Closure language}
\end{figure}

\begin{figure}

  \[
    \boxed{\ltielabDriver{\ltiE{}}{\ltiEp{}}{\ltiT{}}
    \text{ Elaborates external language term \ltiE{} to internal language term \ltiEp{} and type \ltiT{}.
    }
    }
  \]

  \begin{mathpar}
    \infer[ElabDriver]
    {
     \exists \ltiFuel{}.\ 
     \ltitSstkjudgement{\ltimakeCombinedThreadedEnv{\ltiFuel{}}{\ltiEmptyClosureCache}}
                       {\ltiEmptyEnv}
                       {\ltiE{1}}
                       {\ltiT{}}
                       {\ltimakeCombinedThreadedEnv{\ltiFuelp{}}
                                                   {\ltiClosureCache{}}}
                       {\ltiE{2}}
                       \\
     \ltielimClos{\ltiClosureCache{}}{\ltiE{2}}{\ltiE{3}}
     \\
     \ltielimClosT{\varnothing}{\ltiClosureCache{}}{\ltiT{}}{\ltiTp{}}
    }
    {
    \ltielabDriver{\ltiE{1}}{\ltiE{3}}{\ltiTp{}}
    }
  \end{mathpar}

  \[
    \boxed{\ltielimClos{\ltiClosureCache{}}{\ltiE{}}{\ltiEp{}}
    \text{ Converts symbolic closures in \ltiE{} to explicit types in \ltiEp{}}
    }
  \]

  \[
  \begin{array}{llll}
    \ltielimClosalign{\ltiClosureCache{}}{\ltivar{}}
                     {\ltivar{}}
                     \\
    \ltielimClosalign{\ltiClosureCache{}}
                     {\ltiappinst{\ltiF{}}
                                 {\ova{\ltistackmapping{\ltiEnv{}}{\ova{\ltiR{}}}}}
                                 {\ltiE{}}}
                     {\ltiappinst{\ltielimClosLHS{\ltiClosureCache{}}{\ltiF{}}}
                                 {\ova{\ltistackmapping{\ltielimClosEnvLHS{\ltiClosureCache{}}{\ltiEnv{}}}
                                                       {\ova{\ltielimClosTLHS{\varnothing}{\ltiClosureCache{}}{\ltiR{}}}}}}
                                 {\ltielimClosLHS{\ltiClosureCache{}}{\ltiE{}}}}
                             \\
    \ltielimClosalign{\ltiClosureCache{}}{\ltisel{\ltiF{}}{\ltivar{}}}
                     {\ltisel{\ltielimClosLHS{\ltiClosureCache{}}{\ltiF{}}}{\ltivar{}}}
                     \\
    \ltielimClosalign{\ltiClosureCache{}}{\ltiRec{\ova{\ltivar{} = \ltiF{}}}}
                     {\ltiRec{\ova{\ltivar{} = \ltielimClosLHS{\ltiClosureCache{}}{\ltiF{}}}}}
                     \\
    \ltielimClosalign{\ltiClosureCache{}}
                     {\ltifuntparaminterface{\ova{\ltitvar{}}}
                                            {\ova{\ltistackmapping{\ltiEnv{}}{\ltiT{}}}}
                                            {\ltivar{}}
                                            {\ltiE{}}}
                     {\ltifuntparaminterface{\ova{\ltitvar{}}}
                                            {\ova{\ltistackmapping{\ltielimClosEnvLHS{\ltiClosureCache{}}{\ltiEnv{}}}
                                                                  {\ltielimClosTLHS{\varnothing}{\ltiClosureCache{}}{\ltiT{}}}}}
                                            {\ltivar{}}
                                            {\ltielimClosLHS{\ltiClosureCache{}}{\ltiE{}}}}
                     \\
    \ltielimClosalign{\ltiClosureCache{}}
                     {\ltiufunelab{\ova{\ltiufunelabentry{\ltiClosureID{}}}^n}
                                  {\ltivar{}}
                                  {\ltiE{}}}
                     {\ltielimClosLHS{\ltiClosureCache{}}
                                     {\ltimergeTaggedTermsLHS{\ltiF{1}}
                                                             {\ltimergeTaggedTermsLHS{...}{\ltiF{n}}}}}
                    , &n>0
                     \\
                     &&\text{where } \overrightarrowcaption{
                                      \ltilookup{\ltiClosureCache{}}{\ltiClosureID{i}}
                                      = \ltistackmapping{\ltiEnv{i}}{\ltiF{i}}
                                      }^{1 \leq i \leq n}
  \end{array}
  \]


  \[
    \boxed{\ltielimClosEnv{\ltiClosureCache{}}{\ltiEnv{}}{\ltiEnvp{}}
    \text{ Eliminates symbolic closures in \ltiEnv{} using \ltiClosureCache{}.
    }
    }
  \]

  \[
  \begin{array}{llllll}
    \ltielimClosEnvalign{\ltiClosureCache{}}{\ltiEmptyEnv}{\ltiEmptyEnv}
    \\
    \ltielimClosEnvalign{\ltiClosureCache{}}
                        {\ltiEnvConcatParen{\hastype{\ltivar{}}{\ltiT{}}}{\ltiEnv{}}}
                        {\ltiEnvConcat{\hastype{\ltivar{}}{\ltielimClosTLHS{\varnothing}{\ltiClosureCache{}}{\ltiT{}}}}
                                      {\ltielimClosEnvLHS{\ltiClosureCache{}}{\ltiEnv{}}}}
    \\
    \ltielimClosEnvalign{\ltiClosureCache{}}
                        {\ltiEnvConcatParen{\ltitvar{}}{\ltiEnv{}}}
                        {\ltiEnvConcat{\ltitvar{}}
                                      {\ltielimClosEnvLHS{\ltiClosureCache{}}{\ltiEnv{}}}}
  \end{array}
  \]

  \[
    \boxed{\ltielimClosT{\ova{\ltiClosureID{}}}{\ltiClosureCache{}}{\ltiT{}}{\ltiTp{}}
    \text{ Converts symbolic closures in \ltiT{} to explicit types in \ltiTp{}}
    }
  \]

  \[
  \begin{array}{llll}
    \ltielimClosTalign{\ova{\ltiClosureID{}}}{\ltiClosureCache{}}{\ltitvar{}}{\ltitvar{}}\\
    \ltielimClosTalign{\ova{\ltiClosureID{}}}{\ltiClosureCache{}}{\ltiTop}{\ltiTop}\\
    \ltielimClosTalign{\ova{\ltiClosureID{}}}{\ltiClosureCache{}}{\ltiBot}{\ltiBot}\\
    \ltielimClosTalign{\ova{\ltiClosureID{}}}{\ltiClosureCache{}}
                      {\ltiIFn{\ova{\ltiFn{\ltiT{}}{\ltiS{}}}}}
                      {\ltiIFn{\ova{\ltiFn{\ltielimClosTLHS{\ova{\ltiClosureID{}}}{\ltiClosureCache{}}{\ltiT{}}}
                                          {\ltielimClosTLHS{\ova{\ltiClosureID{}}}{\ltiClosureCache{}}{\ltiS{}}}}}}
                                          \\
    \ltielimClosTalign{\ova{\ltiClosureID{}}}{\ltiClosureCache{}}
                      {\ltiRec{\ova{\hastype{\ltivar{}}{\ltiT{}}}}}
                      {\ltiRec{\ova{\hastype{\ltivar{}}{\ltielimClosTLHS{\ova{\ltiClosureID{}}}{\ltiClosureCache{}}{\ltiT{}}}}}}
                                          \\
    \ltielimClosTalign{\ova{\ltiClosureID{}}}{\ltiClosureCache{}}
                      {\ltiMu{\ltitvar{}}{\ltiT{}}}
                      {\ltiMu{\ltitvar{}}{\ltielimClosTLHS{\ova{\ltiClosureID{}}}{\ltiClosureCache{}}{\ltiT{}}}}
                      , &\ltitvar{} \not\in \ova{\ltiClosureID{}}
                      \\
    \ltielimClosTalign{\ova{\ltiClosureID{}}}{\ltiClosureCache{}}
                      {\ltiPoly{\ova{\ltitvar{}}}{\ltiT{}}}
                      {\ltiPoly{\ova{\ltitvar{}}}{\ltielimClosTLHS{\ova{\ltiClosureID{}}}{\ltiClosureCache{}}{\ltiT{}}}}
                      , &\ova{\ltitvar{}} \cap \ova{\ltiClosureID{}} = \varnothing
                      \\
    \ltielimClosTalign{\ova{\ltiClosureID{}}}{\ltiClosureCache{}}
                      {\ltiClosureWithStkID{\ltiEnv{}}{\ltiClosureIDp{}}{\ltiufun{\ltivar{}}{\ltiE{}}}}
                      {\ltiClosureIDp{}}
                      , & \ltiClosureIDp{} \in \ova{\ltiClosureID{}}
                      \\
    \ltielimClosTalign{\ova{\ltiClosureID{}}}{\ltiClosureCache{}}
                      {\ltiClosureWithStkID{\ltiEnv{}}{\ltiClosureIDp{}}{\ltiufun{\ltivar{}}{\ltiE{}}}}
                      {\ltiPoly{\ova{\ltitvar{}}}
                               {\ltiMu{\ltiClosureIDp{}}
                                      {\ltielimClosTLHS{(\ltiClosureIDp{}, \ova{\ltiClosureID{}})}
                                                       {\ltiClosureCache{}}
                                                       {\ltiT{}}}}}
                      , & \ltiClosureIDp{} \not\in \ova{\ltiClosureID{}},
                          \ltiClosureID{} \not\in \ltifvLHS{\ltiT{}}
                      \\
                      &&\text{where } 
                      \ltilookup{\ltiClosureCache{}}{\ltiClosureIDp{}}
                      = \ltiClosure{\ltiEnv{}}
                                   {\ltifuntparaminterface{\ova{\ltitvar{}}}
                                                          {\ltiT{}}
                                                          {\ltivar{}}
                                                          {\ltiE{}}}

  \end{array}
  \]
  \caption{Elaboration Metafunctions for Extended Symbolic Closure language}
\end{figure}

{
\begin{lstlisting}[language=ml,mathescape=true]
let f = $\ltiufun{\text{x}}{\text{x}}$ in
  {left = $\ltiapp{\text{f}}{\text{1}}$, right = $\ltiapp{\text{f}}{\text{"a"}}$}
(* SC annotated *)
(* $\ltiInferred{\ltiClosureCache{} =%
      \ltiClosureCacheEntry{\text{c1}}%
                           {\ltiClosure{\ltiEmptyEnv}%
                                       {\ltiNotInferred%
                                        {\ltifuninterface{\ltiInferred{\ltiIFn{\ltiFn{\text{Int}}{\text{Int}} \ltiFn{\text{Str}}{\text{Str}}}}}%
                                                         {\text{x}}%
                                                         {\text{x}}}}}}$ *)
let f = $\ltiufunelab{\ltiInferred{\text{c1}}}{\text{x}}{\text{x}}$ in
  {left = $\ltiapp{\text{f}}{\text{1}}$, right = $\ltiapp{\text{f}}{\text{"a"}}$}
(* fully annotated *)
let f = $\ltifuninterface{\ltiInferred{\ltiIFn{\ltiFn{\text{Int}}{\text{Int}} \ltiFn{\text{Str}}{\text{Str}}}}}{\text{x}}{\text{x}}$ in
  {left = $\ltiapp{\text{f}}{\text{1}}$, right = $\ltiapp{\text{f}}{\text{"a"}}$}
\end{lstlisting}
}

{
\begin{lstlisting}[language=ml,mathescape=true]
let f = ${\ltiufun{\text{x}}{\text{x}}}$ in
  $\ltiapp{\text{f}}{\text{f}}$
(* SC annotated *)
(* $\ltiInferred{\ltiClosureCache{} =%
      \ltiClosureCacheEntry{\text{c1}}%
                           {\ltiClosure{\ltiEmptyEnv}%
                                       {\ltiNotInferred%
                                        {\ltifuninterface{\ltiInferred{\ltiFn{\ltiClosureWithStkIDParens{\ltiEmptyEnv}{\text{c1}}{\ltiufun{\text{x}}{\text{x}}}}%
                                                                             {\ltiClosureWithStkIDParens{\ltiEmptyEnv}{\text{c1}}{\ltiufun{\text{x}}{\text{x}}}}}}%
                                                         {\text{x}}%
                                                         {\text{x}}}}}}$ *)
let f = $\ltiufunelab{\ltiInferred{\text{c1}}}{\text{x}}{\text{x}}$ in
  $\ltiapp{\text{f}}{\text{f}}$
(* Fully annotated *)
let f = $\ltifuninterface{\ltiInferred{\ltiFn{\ltiMu{\text{a}}{\ltiFn{\text{a}}{\text{a}}}}{\ltiMu{\text{a}}{\ltiFn{\text{a}}{\text{a}}}}}}{\text{x}}{\text{x}}$ in
  $\ltiapp{\text{f}}{\text{f}}$
\end{lstlisting}
}

{
\begin{lstlisting}[language=ml,mathescape=true]
let f = $\ltiufun{\text{x}}{\text{x}}$ in
  {left  = $\ltiapp{\text{map}}{\text{f},\ltiapp{\text{Some}}{\text{1}}}$,
   right = $\ltiapp{\text{map}}{\text{f},\ltiapp{\text{Some}}{\text{"a"}}}$}
(* SC annotated *)
(* $\ltiInferred{\ltiClosureCache{} =%
      \ltiClosureCacheEntry{\text{c1}}%
                           {\ltiClosure{\ltiEmptyEnv}%
                                       {\ltiNotInferred%
                                        {\ltifuninterface{\ltiInferred{\ltiIFn{\ltiFn{\text{Int}}{\text{Int}}%
                                                                               \ltiFn{\text{Str}}{\text{Str}}}}}%
                                                         {\text{f,x}}%
                                                         {\ltiappinst{\text{f}}{\ltiInferred{\text{Int}}}{\text{x}}}}}}}$ *)
let f = $\ltiufunelab{\text{c1}}{\text{x}}{\text{x}}$ in
  {left  = $\ltiappinst{\text{map}}{\ltiInferred{\text{Int,Int}}}{\text{f},\ltiappinst{\text{Some}}{\ltiInferred{\text{Int}}}{\text{1}}}$,
   right = $\ltiappinst{\text{map}}{\ltiInferred{\text{Str,Str}}}{\text{f},\ltiappinst{\text{Some}}{\ltiInferred{\text{Str}}}{\text{"a"}}}$}
(* Fully annotated *)
let f = $\ltifuninterface{\ltiInferred{\ltiIFn{\ltiFn{\text{Int}}{\text{Int}}%
                                               \ltiFn{\text{Str}}{\text{Str}}}}}%
                         {\text{x}}%
                         {\text{x}}$ in
  {left  = $\ltiappinst{\text{map}}{\ltiInferred{\text{Int,Int}}}{\text{f},\ltiappinst{\text{Some}}{\ltiInferred{\text{Int}}}{\text{1}}}$,
   right = $\ltiappinst{\text{map}}{\ltiInferred{\text{Str,Str}}}{\text{f},\ltiappinst{\text{Some}}{\ltiInferred{\text{Str}}}{\text{"a"}}}$}
\end{lstlisting}
}

{
\begin{lstlisting}[language=ml,mathescape=true]
let f = $\ltiufun{\text{x}}{\ltiapp{\text{map}}{\ltiufun{\text{y}}{\text{y}},\text{x}}}$ in
  {left  = $\ltiapp{\text{f}}{\ltiapp{\text{Some}}{\text{1}}}$,
   right = $\ltiapp{\text{f}}{\ltiapp{\text{Some}}{\text{"a"}}}$}
(* SC annotated *)
(* $\ltiInferred{\ltiEnv{1} = {\hastype{\text{x}}{\text{Option[Int]}}}}$ *)
(* $\ltiInferred{\ltiEnv{2} = {\hastype{\text{x}}{\text{Option[Str]}}}}$ *)
(* $\ltiInferred{\ltiClosureCache{}} =$
     $\ltiInferred{%
      \ltiClosureCacheEntry{\text{c1}}%
                           {\ltiClosure{\ltiEmptyEnv}%
                                       {\ltiNotInferred%
                                        {\ltifuninterfaceLHS{\ltiInferred{\ltiIFn{\ltiFn{\text{Option[Int]}}{\text{Option[Int]}}%
                                                                                  \ltiFn{\text{Option[Str]}}{\text{Option[Str]}}}}}%
                                                         {\text{x}}}}}}$
              $\ltiappinst{\text{map}}% <- do not change indentation!
                           {\ltiInferred%
                            {\ltistackmapping{\ltiEnv{1}}{\text{[Int,Int]}},%
                             \ltistackmapping{\ltiEnv{2}}{\text{[Str,Str]}}}}%
                           {\ltiufunelab{\text{c1-1,c1-2}}{\text{y}}{\text{y}},\text{x}}$
     $\ltiInferred{%
      \ltiClosureCacheEntry{\text{c1-1}}%
                           {\ltiClosure{\ltiEnv{1}}%
                                       {\ltiNotInferred%
                                        {\ltifuninterface{\ltiInferred{\ltiFn{\text{Int}}{\text{Int}}}}%
                                                         {\text{y}}%
                                                         {\text{y}}}}}}$
     $\ltiInferred{%
      \ltiClosureCacheEntry{\text{c1-2}}%
                           {\ltiClosure{\ltiEnv{2}}%
                                       {\ltiNotInferred%
                                        {\ltifuninterface{\ltiInferred{\ltiFn{\text{Str}}{\text{Str}}}}%
                                                         {\text{y}}%
                                                         {\text{y}}}}}}$ *)
let f = $\ltiufunelab{\ltiInferred{\text{c1}}}{\text{x}}{\ltiapp{\text{map}}{\ltiufunelab{\ltiInferred{\text{c1-1,c1-2}}}{\text{y}}{\text{y}},\text{x}}}$ in
  {left  = $\ltiapp{\text{f}}{\ltiappinst{\text{Some}}{\ltiInferred{\text{Int}}}{\text{1}}}$,
   right = $\ltiapp{\text{f}}{\ltiappinst{\text{Some}}{\ltiInferred{\text{Str}}}{\text{"a"}}}$}
(* Fully elaborated *)
let f = $\ltifuninterfaceLHS{\ltiInferred{\ltiIFn{\ltiFn{\text{Option[Int]}}{\text{Option[Int]}}%
                                               \ltiFn{\text{Option[Str]}}{\text{Option[Str]}}}}}%
                         {\text{x}}$
          ${\ltiappinst{\text{map}}%<- do not change indentation!
                       {\ltiInferred%
                        {\ltistackmapping{\ltiEnv{1}}{\text{[Int,Int]}},%
                         \ltistackmapping{\ltiEnv{2}}{\text{[Str,Str]}}}}%
                       {\ltifuninterface{\ltiInferred%
                                         {\ltistackmapping{\ltiEnv{1}}{\ltiFn{\text{Int}}{\text{Int}}},%
                                          \ltistackmapping{\ltiEnv{2}}{\ltiFn{\text{Str}}{\text{Str}}}}}%
                                        {\text{y}}%
                                        {\text{y}},%
                        \text{x}}}$ in
  {left  = $\ltiapp{\text{f}}{\ltiappinst{\text{Some}}{\ltiInferred{\text{Int}}}{\text{1}}}$,
   right = $\ltiapp{\text{f}}{\ltiappinst{\text{Some}}{\ltiInferred{\text{Str}}}{\text{"a"}}}$}
\end{lstlisting}
}

%\input{colored}

% - Solution
%   - extend colored LTI with directed inference
%   - introduce constrained types
%   - derive data flow from (variances of) polymorphic variable occurrences 
%   - simple example
%     - (identity 1)
%     - demonstrate how this is checked with colored LTI
%     - compare to directed LTI:
%       - 2/----v
%         [x -> x]  Int
%          ^--------/1
%
%         1. Int flows to contravariant position
%         2. contravariant position flows to covariant position (because it's on the other side of ->)
%       - no loop, because variables not under different numbers of function types
%         - (we don't know the precise rule yet)
%   - complex example
% - Constraints
%   - advantages over colored LTI
%     - aids symbolic analysis
%       - because we derive potential dataflows, we don't need to over-approximate,
%         and thus trigger unneeded symbolic analysis
%         - which might then fail because of not enough contextual information
%   - disadvantages over colored LTI
%     - significant deviation from LTI
%       - constrained types
%       - aggressive local inference based on data flows
%     - not obvious how to prove soundness
%   - infinite loops
%     - how to manage cycles in inferred data flow 
%   - constraint solving
%     - constrained types
%       - literature (see symb.tex)
%   - flow diagrams
%     - see symb.tex
%   - relationship to colored LTI model
%     - see symb.tex
%   - related work
%     - ML_sub
%     - see: symb.tex
% - investigate implications 
%   - Remy ICFP '05
%     - (seems to) propagate information simultaneously in both directions like CLTI
%     - intro prose does a nice job explaining ML moving towards System F & challenges
%   - Joe B Wells 1994
%     - explains Church vs Curry style System F formulations
%     - some mentions of decidable fragments of System F
%   - Boxy types, ICFP '06
%     - explains higher-rank types
%       - types with forall quantifiers nested inside function types
%     - explains impredicativity
%       - being allowed to instantiate a type variable with a polytype (polymorphic type)
%     - explains "local type inference"
%       - a partial inference technique for a language with bounded, impredicative quantification,
%         and higher-rank types.
%     - explains "CLTI"
%       - reformulated bidirectional checking for F_sub so that the _type_ and not the _judgment form_
%         describes the direction in which type information flows
%     - CLTI's colors inspired their "boxy" types
%       - they outline differences in related work

%\chapter{Custom Typing rules}
%\label{chapter:symbolic:custom-rules}

% - Solution
%   - allow users to provide custom typing rules
%   - 
% - Constraints
%   - wildcard type from colored LTI useful
%   - custom error messages
%     - propagation via expected types
%       - outer-most wins
%   - using clojure.spec to conform/unform
%     - to rip apart and put syntax back together
%     - more robust than manual parsing
%   - differences with Turnstile
%     - in Turnstile, the macro *is* the rule
%       - here, we separate the two
%       - we preserve the macro call until evaluation
%       - use the typing rule to expand "under" the macro as many times as we want
%         - can do this 0-n times, thus compatible with directed LTI & symbolic analysis

%\include{hm-comparison}


\part{Related and Future Work}
\label{part:related-future-work}

\chapter{Related Work to Typed Clojure}

% Cite a few of the early papers here.
%http://www.cs.washington.edu/research/projects/cecil/www/pubs/
\paragraph{Multimethods} 
\cite{MS02} and collaborators present a sequence of
systems~\cite{Chambers:1992:OMC,Chambers:1994:TMM,MS02} with statically-typed multimethods
and modular type checking.  In contrast to Typed Clojure, in these
system methods declare the types of arguments that they expect which
corresponds to exclusively using \clj{class} as the dispatch function
in Typed Clojure. However, Typed Clojure does not attempt to rule out
failed dispatches.

% one sentence
% TC based on TR, already covered

%\paragraph{Occurrence Typing} 
%Occurrence typing~\cite{TF08,TF10} extends the type 
%system with a \emph{proposition environment} that represents 
%the information on the types of bindings down conditional branches.
%These propositions are then used to update the types associated
%with bindings in the \emph{type environment} down branches
%so binding occurrences are given different types 
%depending on the branches they appear in, and the conditionals
%that lead to that branch.

% What's diff about TC from the related work
% small summary for deisel....
% - diesel supports x
%- - calculus supports some subset of x
% we support y, which covers most of x but also foo

% eg. multiple dispatch
%     nominal vs structural

% eg. run abritrary metaprogramming over dispatch in CLOS
%  more expressive

% type systems for mm or rows
% rows vs HMap
% - no poly in HMap
% - based on subtyping
% - rows based on polymorphism

\paragraph{Record Types} Row polymorphism~\cite{Wand89typeinference,CM91,HP91}, used
in systems such as the OCaml object system, provides many of the
features of HMap types, but defined using universally-quantified row
variables. HMaps in Typed Clojure are instead designed to be used with
subtyping, but nonetheless provide similar expressiveness, including
the ability to require presence and absence of certain keys. 

Dependent JavaScript~\cite{Chugh:2012:DTJ} can track similar
invariants as HMaps with types for JS objects. They must deal with
mutable objects, they feature refinement types and strong updates to
the heap to track changes to objects.

TeJaS~\cite{TeJaS}, another type system for JavaScript,
also supports similar HMaps, with the ability to
record the presence and absence of entries, but lacks a compositional
flow-checking approach like occurrence typing.

Typed Lua~\cite{Maidl:2014:TLO} has \emph{table types} which track
entries in a mutable Lua table.  Typed Lua changes the dynamic
semantics of Lua to accommodate mutability: Typed Lua raises a runtime
error for lookups on missing keys---HMaps consider lookups on missing
keys normal.

\paragraph{Java Interoperability in Statically Typed Languages}
Scala~\cite{OCD+} has nullable references for compatibility with Java.
Programmers must manually check for
\java{null} as in Java to avoid null-pointer exceptions. 


\paragraph{Other optional and gradual type systems}
%In addition to Typed Racket, 
Several other gradual type
systems have been developed for existing
dynamically-typed languages.  Reticulated Python~\cite{Vitousek14} is
an experimental gradually typed system for Python, implemented as a
source-to-source translation that inserts dynamic checks at language
boundaries and supporting Python's first-class object system. 
Clojure's nominal classes avoids the need to support
first-class object system in Typed Clojure, however HMaps offer an alternative to
the structural objects offered by Reticulated. Similarly,
Gradualtalk~\cite{gradualtalk} offers gradual typing for Smalltalk,
with nominal classes.

Optional types
%, requiring less implementation effort and avoiding
%runtime cost, 
have been  adopted in industry, including Hack~\cite{hack}, and Flow~\cite{flow} and
TypeScript~\cite{typescript}, two extensions of JavaScript. These
systems  support  limited forms of occurrence typing,
and do not include the other features we
present.

%  \item GradualTalk
%  \item Flow
%\end{itemize}




\Dchapter{Related work\either{ to Automatic Annotations}{}}

%The field of dynamic analysis has a rich history.
%Ball~\cite{ball1999concept}
%introduces frequency spectrum analysis,
%an approach that observes a running program
%that is similar to our instrumentation approach.
%Mock~\cite{mock2003dynamic}
%makes the case for efficient profiling of programs
%to better facilitate usage of instrumentation.
%
%Value Profiling is another related area which characterises
%programs based on their running entities.
%
%Daikon~\cite{ernst2001dynamically}
%uses dynamic analysis to recover likely program invariants
%in C programs.
%
%Dynamic type inference has been attempted in many different
%areas.
%Rubydust~\cite{An10dynamicinference}
%infers static types for Ruby. They prove the generated types
%sound, which we do not. 
%\begin{verbatim}
%Conversely, they do not generate
%recursive types, but recursive types in ruby are probably
%nominal, so how different are we?
%\end{verbatim}
%
%In the context of JavaScript, several usages of this technique
%can be found.
%JSTrace~\cite{saftoiu2010jstrace}
%generates types for (?).
%Separately, work has been done to generate JSDoc-like annotations~\cite{odgaard2014}.
%TypeDevil~\cite{pradel2015typedevil}
%uses dynamic analysis to warn JavaScript programmers of possible inconsistencies
%in their programs.
%
%Work in recovering context-free grammars is most related to our algorithm
%to recover recursive types.
%% TODO Shamir~\cite{shamir1962remark} notes that it is impossible
%% TODO \cite{knobe1976method}
%
%In the context of machine learning, 
%this area is called grammar induction or language learning.  % according to vcrepinvsek2005inferring
%% TODO Wang\cite{wang1998grammar} summarises 
%{\v{C}}repin{\v{s}}ek et. al~\cite{vcrepinvsek2005inferring}
%use genetic programming to infer context-free grammars
%for domain-specific languages.
%Most work in this area assume both positive and negative
%examples. We cannot distinguish between these two in
%our system, so we assume all examples are positive.
%
%Chen~\cite{chen1995bayesian} uses Bayesian inference to converge
%on a suitable grammar, given examples.
%
%There has been recent interest in approximate type inference.
%
%Pluquet et. al~\cite{marot2009fast} investigate heuristics
%to quickly infer types in dynamic programs.
%So does Milojkovi{\'c}
%\cite{milojkovic2016exploring}.
%Spasojevi{\'c} et. al~\cite{spasojevic2014mining}
%compare types across a cross section of projects to improve
%inference.
%
%Adamsen et. al~\cite{adamsen2016analyzing} verify test suite completeness using a hybrid approach of lightweight dependency analysis, static type checking and dynamic instrumentation.
%
%% Inference and Evolution of TypeScript Declaration Files
%% - they submit pull requests from their tool's output
%% https://cs.au.dk/~amoeller/papers/tstools/paper.pdf
%
%% Automatic TS annotations from JSON (including recursive types)
%% https://github.com/shakyShane/json-ts

\paragraph{Automatic annotations}
There are two common implementation strategies for automatic annotation tools. The first
strategy, ``ruling-out'' (for invariant detection), assumes all invariants are true 
and then use runtime analysis results to rule out
impossible invariants. The second ``building-up'' strategy (for dynamic type inference)
assumes nothing and uses runtime analysis results to build up invariant/type knowledge.

Examples of invariant detection tools include Daikon~\infercitep{ernst2001dynamically},
DIDUCE~\infercitep{hangal2002tracking}, and Carrot~\infercitep{pytlik2003automated}, and
typically enhance statically typed languages with more expressive types or contracts.
Examples of dynamic type inference include our tool, Rubydust \infercitep{An10dynamicinference},
JSTrace~\infercitep{saftoiu2010jstrace}, and TypeDevil~\infercitep{pradel2015typedevil},
and typically target untyped languages.

Both strategies have different space behavior with respect to representing
the set of known invariants.
The ruling-out strategy typically uses a lot of memory at the beginning,
but then can free memory as it rules out invariants. For example, if
\texttt{odd(x)} and \texttt{even(x)} are assumed, observing \texttt{x = 1}
means we can delete and free the memory recording \texttt{even(x)}.
Alternatively, the building-up strategy uses the least memory storing
known invariants/types at the beginning, but increases memory usage
as more the more samples are collected. For example, if we know
\texttt{x : Bottom}, and we observe \texttt{x = "a"} and \texttt{x = 1}
at different points in the program, we must use more memory to
store the union \texttt{x : String $\cup$ Integer} in our set of known invariants.

\paragraph{Daikon}
Daikon can reason about very expressive relationships between variables
using properties like ordering ($x < y$), linear relationships ($y = ax + b$),
and containment ($x \in y$). It also supports reasoning with ``derived variables''
like fields ($x.f$), and array accesses ($a[i]$).
%
Typed Clojure's dynamic inference can record heterogeneous data structures
like vectors and hash-maps, but otherwise cannot express relationships
between variables.

There are several reasons for this. The most prominent is that Daikon
primarily targets Java-like languages, so inferring simple type information
would be redundant with the explicit typing disciplines of these languages.
On the other hand, the process of moving from Clojure to Typed Clojure
mostly involves writing simple type signatures without dependencies
between variables. Typed Clojure recovers relevant dependent information
via occurrence typing~\infercitep{TF10}, and gives the option to manually annotate necessary
dependencies in function signatures when needed.

\paragraph{Reverse Engineering Programs with Static Analysis}
Rigi~\infercitep{muller1992reverse} analyzes
the structure of large software systems,
combining static analysis 
with a user-facing graphical environment to allow users to view and manipulate
the in-progress reverse engineering results.
We instead use a static type system as a feedback mechanism,
which forces more aggressive compacting of generated annotations.

Lackwit~\infercitep{o1997lackwit} uses static analysis to identify abstract 
data types in C programs. Like our work, they share representations between
values, except they use type inference with representations encoded as types.
Recursive representations are inferred via Felice and Coppos's
work on type inference with recursive types~\infercitep{cardone1991type},
where we rely on our imprecise ``squashing'' algorithms over incomplete runtime samples.

Soft Typing~\infercitep{CF91} uses static analysis to insert runtime checks into untyped
programs for invariants that cannot be proved statically. Our approach is instead to let
the user check the generated annotations with a static type system, with static type errors
guiding the user to manually add casts when needed.

\paragraph{Schema Inference}
\infercitet{baazizi2017schema}
infer structural properties of JSON data using a custom JSON schema format.
Their schema inference algorithm proceeds in two stages:
schema inference and schema fusion.
This resembles our collection and naive type environment construction phases.
There are slight differences between schema fusion and our approach.
Schema fusion upcasts heterogeneous array types to be homogeneous, where
we maintain heterogeneous vector types until a differently-sized
vector type is found in the same position.
We also support function types, which JSON lacks.
While they support nested data, they do not attempt to factor out common types as names
or create recursive types like our squashing algorithms.

\infercitet{discala2016automatic}
present a machine learning algorithm to translate denormalized
and nested data that is commonly found in NoSQL databases to traditional
relational formats used by standard RDBMS.
A key component is a schema generation algorithm which arranges related
data into tables via a matching algorithm which discovers related attributes.
Phases 1 and 2 of their algorithm are similar to our local and global
squashing algorithms, respectively, in that first locally accessible information
is combined, and then global information.
%TODO do they infer recursive schemas?
They identify groups of attributes that have (possibly cyclic) relationships.
%They choose a loose method of relating entities (soft functional dependencies) to compensate
%for data inconsistencies and to help users learn about their data via higher-level abstractions.
Where our squashing algorithms for map types are based on (sets of) keysets---on the 
assumption that related entities use similar keysets---they also join attributes
based on their similar values.
This enables more effective entity matching via equivalent attributes
with different names (e.g., ``Email'' vs ``UserEmail'').
Our approach instead assumes programs are somewhat internally consistent, and instead
optimizes to handle missing samples from incomplete dynamic analysis.

%TODO Inferring XML Schema Definitions from XML Data - Bex, Neven, Vansummeren

% Inference and Evolution of TypeScript Declaration Files
% - they submit pull requests from their tool's output
% https://cs.au.dk/~amoeller/papers/tstools/paper.pdf
\paragraph{Other Annotation Tools}
Static analyzers
for JavaScript
(TSInfer~\infercitep{kristensen2017inference}) and for Python (Typpete~\infercitep{typette18}
and PyType~\infercitep{pytype})
automatically annotate code with types.
PyType and Typpete inferred \texttt{nodes}
as \texttt{(? -> int)}
and \texttt{Dict[(Sequence, object)] -> int}, respectively---our tool 
infers it as \clj{[Op -> Int]} by also generating a compact recursive
type.
Similarly, a class-based translation of
inferred both \texttt{left} and \texttt{right}
fields
as \texttt{Any} by PyType, and as \texttt{Leaf} by Typpete---our tool
uses \clj{Op},
a compact recursive type containing \emph{both} \clj{Leaf} and \clj{Node}.
This is similar to our experience with TypeWiz in \Dchapref{infer:chapter:intro}.
(We were unable to install TSInfer.)

NoRegrets~\infercitep{noregrets2018} uses dynamic analysis to learn how a program
is used, and automatically runs the tests of downstream projects to
improve test coverage.
Their \emph{dynamic access paths} represented as
a series of \emph{actions} are analogous to our paths of path elements.

% distinguishes public/private API

% Python
% - MaxSMT-Based Type Inference for Python 3
%  - cites other python based projects
%  - https://link.springer.com/content/pdf/10.1007%2F978-3-319-96142-2_2.pdf
% - pytype
%  - static analysis to generate python annotations
%  - https://github.com/google/pytype
% - pyannotate
%   - dynamic analysis
%   - https://github.com/dropbox/pyannotate

% A Survey of Dynamic Analysis and Test Generation for JavaScript
%  - http://mp.binaervarianz.de/js_survey_2017.pdf


%% NOTE: Haven't pursued the following work yet

%\paragraph{How dynamic languages are used}
%Several languages have seen similar investigations
%into their idioms as I am proposing for Clojure.
%
%A popular motivation is to discover which type system features to support
%when retrofitting a type system.
%% FIXME the is \AAkerblom but there's an error.. also in the bibliography
%Akerblom et. al~\cite{Akerblom:2014:TDF:2597073.2597103} trace dynamic features in Python programs
%via instrumentation. They measured the prevalence of dynamic features in startup versus
%user code, and recorded usage frequencies for a set of dynamic features.
%They concluded dynamism is prevalent in Python, and thus should be supported
%in a retrofitted type system for Python.
%A study along similar lines is also applicable to Clojure, in particular analysing Typed
%Clojure's support for Clojure's dynamic features.
%
%Calla{\'u} et al. \cite{Callau2013} also conducted a large-scale study of
%dynamic Smalltalk idioms to inform future language extensions tooling support.
%Notably, they further perform a qualitative analysis aiming to identify
%the reasons why Smalltalk use these features in the first place, and
%whether they can be replaced with more predictable features. They also 
%measure which kinds of projects (e.g., testing frameworks, user-level libraries, or core system libraries) 
%use particular features more frequently.
%Due to the their prevalence in the open-source Clojure ecosystem,
%Typed Clojure has mainly been tested on user-level libraries.
%We could predict Typed Clojure's applicability to other kinds of projects
%by gathering similar data on how frequently different types of Clojure libraries use
%Clojure's various features.
%
%Andreasen et. al~\cite{Andreasen2016TraceTA} developed
%\emph{trace typing} to explore the design space of JavaScript type systems. 
%Using runtime observations, they studied which control flow techniques
%are used most often in JavaScript programs, and thus, which should
%be supported by an effective type system for JavaScript.
%Typed Clojure implements occurrence typing to reason about control
%flow in Clojure which seems to work well in practice, but a similar
%quantitative analysis could reveal further insights.

%Runtime analysis \cite{Mastrangelo:2015:UYO:2814270.2814313}

%\cite{Mastrangelo:2015:UYO:2814270.2814313} 

\chapter{Related Work to Extensible Type Systems%and Interleaving Type Checking with Expansion
}

Turnstile~\cite{Chang2017TSM} type checks a program during expansion
by repurposing the Racket macro system.
Instead of the more standard approach of providing separate rules to check a macro, Turnstile
typing rules specify both the expansion and checking semantics, and so ensuring the
two are compatibile becomes automatic.
On the other hand, Typed Clojure does not have the goal of allowing users to override
how language primitives type check. Instead, our goal is to provide
a simple interface to write type rules for library functions and macros
in a style that hides the necessary bookkeeping surrounding occurrence
typing and scope management.

SugarJ~\cite{Erdweg2011SJ}
adds syntactic language extensibility to languages like Java, such as pair
syntax, embedded XML, and closures.
Desugarings are expressed as rewrite rules to plain Java.
Similarly, work on \emph{type-specific languages}~\cite{omar2014safely}
adds extensible systems for the definitions of specialized syntax literals
to existing languages.
The \emph{type} of an expression determines how it is parsed and elaborated.

% this paper has a great related works section that differentiates
% the strategies of several typed metaprogramming techniques
SoundX~\cite{Lorenzen2016STS} presents a solution to a common
dilemma in typed metaprogramming: whether to desugar before
type checking, or vice-versa.
The authors present a system where a form is type checked before 
being desugared, with a guarantee that only well-typed code is generated.
Programmers specify desugarings with a combination of typing and rewriting rules, 
which are then connected to form a valid type derivation
in a process called \emph{forwarding}.
We will explore whether we can get the same effect in Typed Clojure
without requiring the user to understand typing rules.
%For example, Scala macros~\cite{Burmako2013SML} interleave type checking and
%desugaring

Ziggurat~\cite{Fisher06staticanalysis} allows programmers to define
the static and dynamic semantics of macros separately. To demonstrate its
broad applicability, they choose Scheme-like macros that generate assembly code
for the dynamic semantics.
They advocate building towers of static analyses, so
macros can be statically checked in terms the static semantics of other macros, instead
of just their assembly code expansions which would otherwise be too difficult to check.
This idea resembles our prototypes in defining custom typing rules for functions and macros in Typed Clojure,
where the dynamic semantics are defined by runtime Clojure constructs (\texttt{defn}
and \texttt{defmacro}), and towers of static semantics are progressively specified in terms of the static
analysis of other Clojure forms.

Type Tailoring~\cite{greenmanttailoring} is an approach to provide more information
to a host type system than it might be capable of by itself.
In particular, the authors use the host platform's metaprogramming functionality
to refine the types of calls based on the program syntax alone, as well as improve
error messages by incorporating surface syntax. Their experiments are based in Typed Racket, that fully expands
syntax before checking it. Since Typed Clojure recently changed to interleave macroexpansion
and type checking, we could extend this technique to also refine calls based on the
types of their arguments (like SoundX).

Other work is relevant to our investigations of improving the user experience
of Typed Clojure. SweetT~\cite{pombrio2018inferring} automatically infers type rules
for syntactic sugar. Helium~\cite{Heeren2003STI} provides hooks into the type inference
process for domain-specific type error messages.

\chapter{Related Work to Symbolic Closures% and Combining Type Checking with Symbolic Analysis
}

\paragraph{Local Type Inference}
Symbolic closures were originally designed as an extension of Local Type Inference~\cite{PierceLTI}.
Our presentation omitted their bidirectional checking (we did not propagate type information down the syntax
tree using synthesis/checking rules)
and so was not a superset of Local Type Inference---in
particular, it does not take advantage of the relaxed optimality conditions when inferring type
arguments in checking mode.
However, adding back bidirectional checking should be possible by starting with the rules of Local
Type Inference, adding a \emph{synthesis} rule for functions that introduces a symbolic
closure, application and subtyping rules for symbolic closures, and some side conditions to restrict 
how a symbolic closure may be reasoned about (like our ``must not contain symbolic closures''
conditions scattered in various rules).
This way, a symbolic closure should only be introduced where Local Type Inference fails---when
a type of a function must be synthesized---and so seems more likely to be a superset of 
Local Type Inference.

Colored Local Type Inference~\cite{coloredlti01} extends Local Type Inference with partial
information propagation. Their type inference algorithm does not use explicit synthesis/checking
rules, instead passing ``prototypes'' \emph{P}
down the syntax tree that containing partial expected type information used for type checking.
A prototype is a type \emph{T} extended with the wildcard ``?'', denoting unknown information,
and the specific shape of a prototype denotes which type rule to use.
The rule inferring unannotated functions \ltiufun{\ltivar{}}{\ltiE{}} requires a prototype \ltiPolyFn{T}{}{P}, where
\emph{T} is the fully known expected type for {\ltivar{}}.
A symbolic closure could be introduced when checking an unannotated function with prototype
\ltiPolyFn{P}{}{P'} or ``?'' (the equivalent of Local Type Inference's ``synthesis'' rules).
In the more complicated case of \ltiPolyFn{P}{}{P'}, a symbolic reduction of 
\ltiufun{\ltivar{}}{\ltiE{}} is required to ensure it at least conforms its the prototype.
For example, inferring the type of \ltiapp{\text{map}}{\ltiufun{\ltivar{}}{\ltiE{}},[1,2,3]} with
Colored Local Type Inference (where \text{map} has type ``\ltiPolyFn{\ltiFn{\text{a}}{\text{b}},\text{List[a]}}{\text{a,b}}{\text{List[b]}}'')
checks \ltiufun{\ltivar{}}{\ltiE{}} with prototype \ltiPolyFn{\text{?}}{}{\text{?}}.
We can be optimistic and check the function 
at the largest (most specific) subtype of 
\ltiPolyFn{\ltiBot}{}{\ltiTop}
that matches \ltiPolyFn{\text{?}}{}{\text{?}}, which is \ltiPolyFn{\ltiBot}{}{\ltiTop}.
This ensures that the function at least conforms to the most optimistic interpretation of its
prototype, and then by returning a symbolic closure type instead of \ltiPolyFn{\ltiBot}{}{\ltiTop}
allows us to check more specific requirements later.
Of course, to fully check this example, it requires that we specify how type argument synthesis works with
symbolic closures, but it at least illustrates how symbolic closures relate to the rest of the system.

Spine-local type inference~\cite{Jenkins:2018:STI:3310232.3310233}
explores Local Type Inference in the context of System F (without subtyping).
They present a greedy type argument synthesis algorithm
which more aggressively propagates type information
to an application's arguments.
To check arguments, type variable instantiations are guessed
based on the expected type of the application.
For example, when checking \ltiapp{\text{id}}{\ltiufun{\ltivar{}}{\ltiE{}}}
with expected type \ltiFn{\ltiT{}}{\ltiS{}},
where \text{id} has type \ltiPolyFn{\ltitvar{}}{\ltitvar{}}{\ltitvar{}},
\ltitvar{} would be guessed to have type \ltiFn{\ltiT{}}{\ltiS{}}
and then {\ltiufun{\ltivar{}}{\ltiE{}}} would be checked at that type.
This would fail if the application was in synthesis mode.
In this specific example, symbolic closures would allow the checking
of \ltiufun{\ltivar{}}{\ltiE{}} to be delayed to when more type information
is available, in either checking or synthesis modes.
Unfortunately, it does not seem that their algorithm can check
\ltiapp{\text{map}}{\ltiufun{\ltivar{}}{\ltiE{}},[1,2,3]}
even in checking mode, and so does not seem to assist us in solving similar
problems with symbolic closures.
This case does not check because only the type of \ltiE{}
would be apparent from an expected type, not the type of \ltivar{}.

% Spine-local type inference
% - Judgement \vdash^P digs down an application to find the head
%   - happens naturally with symbolic closures
% - they use metavariables to solve direct applications
%   - can they check things like (let [x (fn [y] (inc y))] (x 1)) ?
% - they have polymorphism but not subtyping (plain System F)
%   - they speculate about extending to Fsub in related works
%   - they mention Hosoya & Pierce's "challenges" to fix hard-to-synthesize terms
% - they type check arguments left-to-right in a polymorphic application
%   - can't tell if that's different from inferring the data flow from a polytype 
%     and then checking in that order
% - their sense of "locality" is less ambitious than symbolic closures
%   - see "Type Inference Failures" section
% - good related works section for "Impredicative Polymorphism"

\paragraph{Mixing Symbolic Execution and Type Checking}
Mix~\cite{Khoo2010MTC} allows an interplay of symbolic execution~\cite{King1976SEP} with type checking
by providing syntactic regions,
with terms
\MixTregion{\ltiE{}} signaling to use type checking for {\ltiE{}},
and
\MixSregion{e} for symbolic execution.
In Mix, for example, the term
%
\[
\MixSregion{\ltilet{\text{id}}{\ltiufun{\ltivar{}}{\ltivar{}}}
                  {\MixTregion{  ...\ \MixSregion{\ltiapp{\text{id}}{3}}
                               \ ...\ \MixSregion{\ltiapp{\text{id}}{3.0}}
                               \ ... }}}
\]
%
symbolically executes \MixSregion{\ltiapp{\text{id}}{3}}
and
\MixSregion{\ltiapp{\text{id}}{3.0}}, propagating result types
\text{Int} and \text{Real} back to the typed regions, respectively.
Comparatively, symbolic closures integrates only a small amount of
symbolic execution with a type system, but in such a way that delayed symbolic computations
may pass between typed regions.
Since Mix cleanly separates symbolic execution and type checking and its formalism does not support
function types, it is difficult to compare the two approaches.
In rough terms, symbolic closures use typed regions by default and automatically adds symbolic regions
around unannotated functions.
%
\[
\MixTregion{\ltilet{\text{id}}{\MixSregion{\ltiufun{\ltivar{}}{\ltivar{}}}}
                   {  ...\ \ltiapp{\text{id}}{3}
                    \ ...\ \ltiapp{\text{id}}{3.0}
                    \ ... }}
\]
%
Typing rules are then added to introduce a symbolic closure type
and also to apply them, which involves checking the delayed body in a typed region.
%
\begin{mathpar}
\infer[]
  {}
  { \ltitjudgementNoElab{\ltiEnv{}}{\MixSregion{\ltiufun{\ltivar{}}{\ltiE{}}}}
                        {\ltiClosure{\ltiEnv{}}{\ltiufun{\ltivar{}}{\ltiE{}}}}
  }

\infer[]
  { \ltitjudgementNoElab{\ltiEnv{}}{\MixTregion{\ltiF{}}}
                        {\ltiClosure{\ltiEnvp{}}{\ltiufun{\ltivar{}}{\ltiEp{}}}}
    \\\\
    \ltitjudgementNoElab{\ltiEnv{}}{\MixTregion{\ltiE{}}}{\ltiS{}}
    \\
    \ltitjudgementNoElab{\ltiEnvConcat{\ltiEnvp{}}{\hastype{\ltivar{}}{\ltiS{}}}}
                        {\MixTregion{\ltiEp{}}}
                        {\ltiT{}}
  }
  { \ltitjudgementNoElab{\ltiEnv{}}{\MixTregion{\ltiapp{\ltiF{}}{\ltiE{}}}}{\ltiT{}}
  }
\end{mathpar}

When forced to delineate type checking from symbolic execution like this, it interesting to ask to what degree symbolic
closures even uses symbolic execution.
Our view is that (at least) symbolic closures symbolically execute the runtime-closure introduction rule.
%
\begin{mathpar}
\infer[]
  {}
  {
  \opsem {\openv{}}
         {\ltiufun{\x{}}{\e{}}}
         {\closurenosuffix{\openv{}}{\ltiufun{\x{}}{\e{}}}}
         }
\end{mathpar}
%
The symbolic closure type
{\ltiClosure{\ltiEnv{}}{\ltiufun{\ltivar{}}{\ltiE{}}}}
is then the symbolic value of the runtime closure
{\closurenosuffix{\openv{}}{\ltiufun{\x{}}{\e{}}}},
related by the following typing rule.
%
\begin{mathpar}
\infer []
{ 
  \overrightarrowcaption{
  \ltitjudgementNoElab{}{\ltiEnvLookup{\openv{}}{y}}{\ltiEnvLookup{\ltiEnv{}}{y}}
  }
  ^{y \in dom(\openv{})}
              }
{ \ltitjudgementNoElab {\ltiEnvp{}}
                       {\closurenosuffix
                        {\openv{}}
                        {\ltiufun{\ltivar{}}{\ltiE{}}}}
                       {\ltiClosure{\ltiEnv{}}{\ltiufun{\ltivar{}}{\ltiE{}}}}
          }
\end{mathpar}
%
As evidenced by the lack of symbolic regions in the above application rule,
a ``symbolic reduction'' of a symbolic closure is not particularly related
to symbolic execution---it merely kicks off some delayed type checking.
However, \ltiEnv{}, \ltiS{}, and \ltiT{} in that rule may contain symbolic closure
types, so symbolic values are being used to reason about the program.

%\begin{mathpar}
%\infer[]
%{ \opsem {\openv{}}
%         {\e{f}}
%         {\closurenosuffix {\openv{c}} {\abs {\x{}} {\t{}} {\e{b}}}}
%         \\
%  \opsem {\openv{}}
%         {\e{a}}
%         {\v{a}}
%         \\
%  \opsem {\extendopenv {\openv{c}} {\x{}} {\v{a}}}
%         {\e{b}}
%         {\v{}}
%       }
%{ \opsem {\openv{}}
%         {\appexp {\e{f}} {\e{a}}}
%         {\v{}}
%       }
%\end{mathpar}

%Symbolic closures type check an anonymous function if annotated, otherwise it is treated symbolically.
%As the authors envision, this is akin to automatically inserting
%the mode of a code region based on its context, with a Mix-like language
%becoming the intermediate language.

Mix also uses symbolic execution to enhance simple type systems with flow-sensitivity.
For example, the following Mix program uses symbolic execution 
to flow-sensitively reason about \text{int?}, a predicate that returns true only for integer values.
%
\[
\MixSregion{\ltilet{\text{f}}{\ltiufun{\ltivar{}}{(\ltiif{\ltiapp{\text{int?}}{\ltivar{}}}{\ltivar{}}{\textsf{nil}})}}
                  {\MixTregion{  ...\ \MixSregion{\ltiapp{\text{f}}{3}}
                               \ ...\ \MixSregion{\ltiapp{\text{f}}{3.0}}
                               \ ... }}}
\]
%
The symbolic regions determine
\MixSregion{\ltiapp{\text{f}}{3}} has type \text{Int} and
\MixSregion{\ltiapp{\text{f}}{3.0}} type \text{Nil} via symbolic execution.
Symbolic closures are instead designed to be compatible with flow-sensitive type systems like occurrence typing~\cite{TF10}.
Here is the analogous program using symbolic closures.
%
\[
\MixTregion{\ltilet{\text{f}}{\MixSregion{\ltiufun{\ltivar{}}{(\ltiif{\ltiapp{\text{int?}}{\ltivar{}}}{\ltivar{}}{\textsf{nil}})}}}
                  {  ...\ {\ltiapp{\text{f}}{3}}
                               \ ...\ {\ltiapp{\text{f}}{3.0}}
                               \ ... }}
\]
%
Now, let us assume occurrence typing is also used to check this program, and
that \text{int?} is typed as a predicate for integers.
The call to
{\ltiapp{\text{f}}{3}}
triggers the symbolic reduction
%
\[
\ltitjudgementNoElab{\hastype{\ltivar{}}{\text{Int}}}
                    {(\ltiif{\ltiapp{\text{int?}}{\ltivar{}}}{\ltivar{}}{\textsf{nil}})}
                    {\text{Int}}
\]
%
which has type \text{Int}, because the else-branch is unreachable, and 
similarly {\ltiapp{\text{f}}{3.0}} triggers
%
\[
\ltitjudgementNoElab{\hastype{\ltivar{}}{\text{Real}}}
                    {(\ltiif{\ltiapp{\text{int?}}{\ltivar{}}}{\ltivar{}}{\textsf{nil}})}
                    {\text{Nil}}
\]
%
which has type \text{Nil}, because the then-branch is unreachable.

% >> talk about occurrence tpying.

% Let arguments go first

% Dunfield works I need to compare to
% - Greedy Bidirectional Polymorphism
% - Sound and complete bidirectional typechecking for higher-rank polymorphism with existentials and indexed types
% - Complete and Easy Bidirectional Typechecking for Higher-Rank Polymorphism

% Other works with undecidable type checking
% - Hybrid type checking - Knowles, Flanagan

\paragraph{Intersection Type Checking}
In hindsight, the idea behind symbolic closures resembles intersection type checking,
where the same code may be checked at multiple types.
Carlier and Wells~\cite{carlier2005expansion} give an approachable explanation of ``expansion'',
a mechanism that informs an intersection type system when it should
check the same term at different types.
This is achieved by splicing typing rules (like intersection-introduction) into existing typing
derivations that are derived from the principal typings of subterms.
In contrast, symbolic closures do not assume principal types are available, and
delays the construction of typing derivation(s) for a delayed term
altogether until it is obvious how to construct it.
Then, it is matter of combining a symbolic closure's typing derivations
to recover the (intersection) type it was used at.

%In Section 3.1, they provide a motivating example for expansion, constructed to be untypable 
%with simple (non-recursive) types, and show how expansion assigns it a type.

%TODO
%With the full power of intersection type systems, a principal type expresses

%TODO how do SC relate to this statement?
% - However, computing these principal typings is as expensive as evaluation, 
%   for the simple reason that principal typings for a term in the full system
%   express all of the information in the term’s β-normal form
% - do SC perform less reductions? eg. (lambda (x) x) does not reduce, but I assume
%   the Coppo algorithm constructs a principal type for it. Then, what if it is
%   use like (lambda (z) (let ([f (lambda (x) x)]) (f (f (f z))))) ?
%   Does it still infer an intersection type for f? Obviously, SC's do nothing here.

% survey
% Expansion: the Crucial Mechanism for Type Inference with Intersection Types: A Survey and Explanation1
% https://www.sciencedirect.com/science/article/pii/S1571066105050656
% - "expansion" seems similar to inferring intersection types via symbolic closures
%   - Section 6.2 talks about type inference for rank-2 intersection types
%     - advantage is that expansion never has to introduce intersections under ->
%       - do we do this with symbolic closures?
% - "Expansion is an operation on typings that simulates the effect of splicing in typing rules
%    uses at nested positions in some derivation of that typing."
% - omega expansion looks very similar to symbolic closures types (Section 4)
%   - at least in that it embeds the term in the type to track its origin 
% - Section 5.2 talks about cost of type inference == beta reduction

% (Intersection type systems)
% Principal Types and Unification for Simple Intersection Type Systems
% https://www.sciencedirect.com/science/article/pii/S0890540185711418

\paragraph{Higher-order Control Flow Analysis}

%% (found via the expansion survey as an application of intersection types)
% http://delivery.acm.org/10.1145/260000/258951/p1-banerjee.pdf?ip=140.182.72.36&id=258951&acc=ACTIVE%20SERVICE&key=EA62C54EFA59E1BA%2EEC3C9CD27046E2ED%2E4D4702B0C3E38B35%2E4D4702B0C3E38B35&__acm__=1553786311_eb5536bb95ba54f3a2979f7a216e898e
% A Modular, Polyvariant, and Type-Based Closure Analysis, Anindya Banerjee

%TODO aka. Closure-analysis 
% http://citeseerx.ist.psu.edu/viewdoc/summary?doi=10.1.1.36.6128
% Analysis and Efficient Implementation of Functional Programs (1991), Peter Sestoft
Closure analysis~\cite{sestoft1991analysis} approximates the set of arguments which a given
function may be applied, as well as which functions a given term
may evaluate to.
Each function term is labelled 
\Sesoftlambda{\Sesoftlabel}{\ltivar{}}{\ltiE{}}, where the label {\Sesoftlabel}
abstracts over the set of all runtime closures
\closurenosuffix{\openv{}}{\Sesoftlambda{}{\ltivar{}}{\ltiE{}}}
where runtime environment {\openv{}} can choose arbitrary bindings for {\ltiE{}}'s
free term variables.
In contrast, a ``tagged'' symbolic closure term
\ltiufunelab{\ltiClosureID{}}{\ltivar{}}{\ltiE{}}
uses identifier {\ltiClosureID{}} to stand for
\closurenosuffix{\openv{}}{\ltiufun{\ltivar{}}{\ltiE{}}}
where the bindings in the runtime environment {\openv{}} are of the types
given in the type environment \ltiEnv{} where the term was encountered
by the type checker.
In an unrestricted setting of symbolic closures, the same term
may be used with different identifiers.
For example, in
%
\[
\ltilet{\text{f}}{\ltiufun{\text{x}}{\ltiufun{\text{y}}{\text{x}}}}{...{\ltiapp{\text{f}}{3}}...{\ltiapp{\text{f}}{3.0}}...}
\]
%
the first call to \text{f} tags the inner function as {\ltiufunelab{\ltiClosureID{1}}{\text{y}}{\text{x}}}
with {\ltiClosureID{1}} standing for the set of closures
whose runtime environments bind \text{x} to a value of type \text{Int},
and the second call tags it as {\ltiufunelab{\ltiClosureID{2}}{\text{y}}{\text{x}}}
with {\ltiClosureID{2}} standing for the set of closures
who similarly bind \text{x} to a value of type \text{Real}.


% https://dl.acm.org/citation.cfm?id=201001
% Closure analysis in constraint form -	Jens Palsberg

Giannini and Rocca~\cite{giannini1988characterization}
provide the following strongly-normalizing term is not typable in System F,
which we write in Clojure and refer to as \GRterm.

%separate to preserve spacing
\begin{cljlisting}
(let [I (fn [a] a)
      K (fn [b] (fn [c] b))
      D (fn [d] (d d))]
  ((fn [x] (fn [y] ((y (x I))
                    (x K))))
   D))
\end{cljlisting}
%separate to preserve spacing

Palsberg~\cite{Palsberg:1995:CAC:200994.201001}
uses \GRterm to motivate program analyses
that answer basic questions like:
\begin{itemize}
  \item For every application point, which abstractions can be applied?
  \item For every abstraction, to which arguments can it be applied?
\end{itemize}
Symbolic closures answer neither of these questions.
Instead, they provide answers relevant to checking and inferring types:
\begin{itemize}
  \item Can \GRterm accept an argument of type \ltiT{}?
  \item When given an argument of type \ltiT{}, what type is the value returned by \GRterm?
  \item Does \GRterm inhabit \ltiFn{\ltiT{}}{\ltiS{}}?
\end{itemize}

To illustrate, we turn to
our preliminary implementation of symbolic closures~\footnote{https://github.com/frenchy64/lti-model}
to explore \GRterm.
It exposes the type checking query \clj{(tc p e)},
which returns the type of checking \clj{e} at expected prototype \clj{p},
where a prototype is a type that can contain ``wildcards'' \clj{?}.
Now we can query the type of \GRterm as if it had a type.
The caveat: without a rich enough prototype,
a benign symbolic closure type may be provided as an answer---you only get out what you put in
(for this reason, symbolic closures perform particularly well when top-level types are always provided).

For example, \clj{(tc ? GR)} asks to synthesize a type for \GRterm.
Unsurprising, a symbolic closure type greets us (below).

% full output
%(Closure
% {I (Closure {} (fn [a] a)),
%  K (Closure {I (Closure {} (fn [a] a))} (fn [b] (fn [c] b))),
%  D
%  (Closure
%   {I (Closure {} (fn [a] a)),
%    K (Closure {I (Closure {} (fn [a] a))} (fn [b] (fn [c] b)))}
%   (fn [d] (d d))),
%  x
%  (Closure
%   {I (Closure {} (fn [a] a)),
%    K (Closure {I (Closure {} (fn [a] a))} (fn [b] (fn [c] b)))}
%   (fn [d] (d d)))}
% (fn [y] ((y (x I)) (x K))))

%abbreviated
%(Closure
% {I (Closure {} I),
%  K (Closure {I (Closure {} I)} K),
%  D (Closure {I (Closure {} I), K (Closure {I (Closure {} I)} K)} D),
%  x (Closure {I (Closure {} I), K (Closure {I (Closure {} I)} K)} D)}
% (fn [y] ((y (x I)) (x K))))

% where Ic = (Closure {} I)
%       Kc = (Closure {I Ic} K)
%       Dc = (Closure {I Ic, K Kc} D)
%(Closure
% {I Ic,
%  K Kc,
%  D Dc,
%  x Dc}
% (fn [y] ((y (x I)) (x K))))
{
\[
\begin{array}{lll}
\ltiClosure{\ltiEnv{}}{\text{\clj{(fn [y] ((y (x I)) (x K)))}}},
              \text{ where }&
{\ltiEnv{}} =  \ltiEnvConcat{\hastype{\text{\clj{I}}}{{\text{\clj{I}}}_c}}
              {\ltiEnvConcat{\hastype{\text{\clj{K}}}{{\text{\clj{K}}}_c}}
              {\ltiEnvConcat{\hastype{\text{\clj{D}}}{{\text{\clj{D}}}_c}}
              {\ltiEnvConcat{\hastype{\text{\clj{x}}}{{\text{\clj{D}}}_c}}}}}\\&
{\GRclosuretag{I}} = \ltiClosure{\ltiEnv{I}}{\text{\clj{I}}}\\&
%{\GRclosuretag{K}} = \ltiClosure{\hastype{\text{\clj{I}}}{{\text{\clj{I}}}_c}}{\text{\clj{K}}}\\&
{\GRclosuretag{K}} = \ltiClosure{\ltiEnv{k}}{\text{\clj{K}}}\\&
%{\GRclosuretag{D}} = \ltiClosure{\ltiEnvConcat{\hastype{\text{\clj{I}}}{{\text{\clj{I}}}_c}}{\hastype{\text{\clj{K}}}{{\text{\clj{K}}}_c}}}
%                                  {\text{\clj{D}}}
{\GRclosuretag{D}} = \ltiClosure{\ltiEnv{D}}{\text{\clj{D}}}\\&
\end{array}
\]
}

The term of \GRclosure is the (call-by-value) normal form of \GRterm, 
derived by applying the symbolic closure of \clj{(fn [x] ...)} to  \clj{D}.
The type environment
\ltiEnv{} captures the type environment at the point the \clj{(fn [y] ...)} term
was type checked.
There, the bindings \clj{I}, \clj{K}, and \clj{D} are all symbolic closure types
{\GRclosuretag{I}},
{\GRclosuretag{K}}, and
{\GRclosuretag{D}}, respectively,
with \clj{x} also having type {\GRclosuretag{D}}
as a result of the application.
%Due to Clojure's left-to-right semantics for \clj{let} bindings, \clj{I} is also bound by
%{\ltiEnv{k}} and {\ltiEnv{D}}, and 
%\clj{K} by {\ltiEnv{D}}.

As explained in \secref{symbolic:section:formal-model}, two ways
to query a symbolic closure are by applying it or using subtyping.
We can experimentally discover what shape of argument \GRterm accepts 
by querying it at different prototypes, and using error messages and
visual inspections of \GRterm and its normal form (calculated by \GRclosure) as guidance.
We started with the query \clj{(tc [Any -> ?] GR)}, which
gives the error message \clj{Cannot invoke Any}.
Then we inspected \GRclosure, and noticed \clj{y}
must have shape \clj{[? -> [? -> ?]]} based on its usage.
Incorporating that information results in our first interesting query result.

\begin{cljlisting}
(tc [[? -> [? -> ?]] -> ?]
    GR)
;=> [[Any -> [Any -> Nothing]] -> Nothing]
\end{cljlisting}

This was calculated by observing a result type of \clj{Nothing}
from the application of \GRclosure to an argument of type \clj{[Any -> [Any -> Nothing]]}
(derived by minimizing/maximising wildcards 
in covariant/contravariant positions, respectively, with respect to the relevant part of the prototype).
We can find other interesting types \GRclosure inhabits by varying the query.

\begin{cljlisting}
(tc [[? -> [? -> Int]] -> ?]
    GR)
;=> [[Any -> [Any -> Int]] -> Int]

(tc (All [a] [[? -> [? -> a]] -> ?])
    GR)
;=> (All [a] [[Any -> [Any -> a]] -> a])
\end{cljlisting}

With the last query, we stumble on how to use \GRterm as a glorified identity function.
We can verify this by evaluating a few terms.

\begin{cljlisting}
(GR (fn [_] (fn [_] 42))) ;=> 42
(GR (fn [_] (fn [_] 24))) ;=> 24
\end{cljlisting}

We can also synthesize the types of these calls.

\begin{cljlisting}
(tc ? (GR (fn [_] (fn [_] 42))))
;=> Int
\end{cljlisting}

The original use case of symbolic closures is to type check
top-level functions against provided types,
but whose bodies are too difficult to type check with traditional means.
The following (extreme) example shows how checking the definition
of a simple identity function can be thwarted, and how symbolic closures
can make checking the definition of functions much more flexible.

\begin{cljlisting}
(tc (All [a] [a -> a])
    (fn [z]
      (GR (fn [_] (fn [_] z)))))
;=> (All [a] [a -> a])
\end{cljlisting}

This illustrates the promise of symbolic closures to treat previously untypable terms
as ``black-boxes'' during type checking, especially in a setting where top-level
type information is always provided.

%http://web.cs.ucla.edu/~palsberg/tba/papers/banerjee-icfp97.pdf
% A modular, polyvariant and type-based closure analysis - Banerjee
Banerjee~\cite{banerjee1997modular}
achieves a similar effect by
instrumenting the rank 2 fragment of the intersection type discipline
with flow information of closure values.
Function terms are labelled and
arrow types are annotated with sets of labels which denote
which functions it may represent.
They demonstrate by analyzing the following term
$(\l f. (\l x. f I)(f 0))I$
where $I$ represents the identity function.
They label the term
$(\l^1 f. (\l^2 x. f (\l^3 u. u))(f 0))(\l^4 v. v)$
and infer overall type
$t \xrightarrow[]{\{3\}} t$,
which says values of this type originate from the lambda labeled $3$,
with fresh type variable $t$.

Their system inherits the principal typing property of intersection
types, which we lack in Typed Clojure.
To compensate for this, our prototype for unrestricted symbolic closures
types returns the full code and type environment for the corresponding closure
of lambda $3$, so it may
be further checked later when more type information is available.
For example, plugging this example into our prototype gives the
following symbolic closure type (using their labelled lambda syntax)
$\ltiClosure{\ltiEnv{}}{\l^3 u. u}$,
where $\ltiEnv{} = \{f : \ltiClosure{\{\}}{\l^4 v. v}, x : \text{Int}\}$.

%\begin{cljlisting}
%(tc ? ((fn [f] ((fn [x] (f (fn [u] u))) (f 0))) (fn [v] v)))
%\end{cljlisting}


\paragraph{Hindley-Milner and Let-polymorphism}
%TODO
%\input{hm-comparison}
Kanellakis and Mitchell~\cite{kanellakis1989polymorphic}
provide a set of (pathological) ML programs that exhibit exponential
growth in the size of their principal type schemes.
%Later, Mairson~\cite{Mairson:1989} confirmed that
%the problem of ML type-checking is \textsc{DExpTime}-complete.
We use their benchmarks to compare symbolic closures with
global type inference in the style of Milner~\cite{milner1978theory}.

Example 3.1 of~\cite{kanellakis1989polymorphic} uses a lambda-encoding of pairs to create an ML principal type
which appears to grow exponentially in length, however has a linear time representation as a directed acyclic graph.
It is designed to avoid ML's let-construct to remove the influence of let-polymorphism.
The idea behind the program is to duplicate types \ltiS{} by placing them in (lambda-encoded) tuples,
following the pattern \ltiS{}, $\langle\ltiS{},\ltiS{}\rangle$, $\langle\langle\ltiS{},\ltiS{}\rangle,\langle\ltiS{},\ltiS{}\rangle\rangle$,
and so on.
To compare with symbolic closures,
let $P$ stand for \clj{(fn [x] (fn [z] (z x x)))}
and $P_z$ for \clj{(fn [z] (z x x))} in the following, where the left
hand side term has the right hand side type.
We can see the size of the type grows linearly in the number of occurrences of $P$---specifically,
the outermost symbolic closure's \emph{environment} increases in size.

{
\[
\begin{array}{c@{}c@{}c@{}c@{}ccr@{}r@{}r@{}r@{}r@{}r@{}r@{}r@{}r}
  &&{(P\ 1)}&&        & : &    \ltiClosure{\hastype{\text{x}}{&&&\text{Int}&&}&}{P_z}\\
  &{(P\ } &{(P\ 1)}& {)}&    & : &   \ltiClosure{\hastype{\text{x}}
                                                 {&&(\ltiClosure{\hastype{\text{x}}{&\text{Int}&}}{{P_z}})&&}}
                                                {P_z}\\
{(P\ }&{(P\ }&{(P\ 1)}&{)}&{)}    & : &   \ltiClosure{\hastype{\text{x}}
                                                      {&(\ltiClosure{\hastype{\text{x}}{&(\ltiClosure{\hastype{\text{x}}{&\text{Int}&}}{P_z})&}}{P_z})}&}
                                                     {P_z}
\end{array}
\]
}

Example 3.4 of~\cite{kanellakis1989polymorphic}
exhibits exponential growth in the size of ML principal types by exploiting
let-polymorphism's ability to copy type variables and assign new names to them.
It follows the following pattern, which is similar to the previous benchmark, except
intermediate values are let-bound.

\begin{minipage}[t]{0.3\linewidth}
\begin{cljlisting}
(let [x0 1
      x1 (P x0)]
  x1)
\end{cljlisting}
\end{minipage}
%
\begin{minipage}[t]{0.3\linewidth}
\begin{cljlisting}
(let [x0 1
      x1 (P x0)
      x2 (P x1)]
  x2)
\end{cljlisting}
\end{minipage}
%
\begin{minipage}[t]{0.3\linewidth}
\begin{cljlisting}
(let [x0 1
      x1 (P x0)
      x2 (P x1)
      x3 (P x2)]
  x3)
\end{cljlisting}
\end{minipage}

Unlike ML, we find that symbolic closures are rather sensitive to whether $P$
is let-bound or copied.
If let-bound at the top of each term, the types of each term are
identical to the previous example, and so grow linearly in size.
This is because the resulting type of each term is a symbolic closure of the
$P_z$ term occuring in $P$, whose definition type environment
never increases to include new variables (in particular, \clj{x1}, \clj{x2}, and \clj{x3}
are never in-scope there).
If $P$ is copied, however, the number of symbolic closures types reachable from
the innermost occurrence of $P$ grows exponentially, and so the resulting type also
grows exponentially.
We can reduce this to linear growth with sharing as below, where \clj{xi} has type \clj{Pci},
because the exponential growth happens by duplicating symbolic closure types.
Each $P_z$ term comes from a different copy of $P$, in particular
the $P_z$ term of symbolic closure type \clj{Pci} originates from the $P$ occurring on the right-hand-side of \clj{xi}.

\begin{cljlisting}
Pc1 = {x0 Int,                         x Int}@Pz
Pc2 = {x0 Int, x1 Pc1,                 x Pc1}@Pz
Pc3 = {x0 Int, x1 Pc1, x2 Pc2,         x Pc2}@Pz
Pc4 = {x0 Int, x1 Pc1, x2 Pc2, x3 Pc3, x Pc3}@Pz
\end{cljlisting}

Example 3.5 of~\cite{kanellakis1989polymorphic} gives a series of terms whose ML principal
type is doubly-exponential in the size of the term,
reduced to exponential when converted to a directed acyclic graph.
The pattern is below, which, for $i>1$ and $j=i-1$, binds \clj{xi} to \clj{(fn [y] (xj (xj y)))}.

\begin{minipage}[t]{0.31\linewidth}
\begin{cljlisting}
(let [x1 (fn [x] (P x))
      x2 (fn [y]
           (x1 (x1 y)))]
  (x2 1))
\end{cljlisting}
\end{minipage}
%
\begin{minipage}[t]{0.31\linewidth}
\begin{cljlisting}
(let [x1 (fn [x] (P x))
      x2 (fn [y]
           (x1 (x1 y)))
      x3 (fn [y]
           (x2 (x2 y)))]
  (x3 1))
\end{cljlisting}
\end{minipage}
%
\begin{minipage}[t]{0.31\linewidth}
\begin{cljlisting}
(let [x1 (fn [x] (P x))
      x2 (fn [y]
           (x1 (x1 y)))
      x3 (fn [y]
           (x2 (x2 y)))
      x4 (fn [y]
           (x3 (x3 y)))]
  (x4 1))
\end{cljlisting}
\end{minipage}

The symbolic closure type for these terms grows exponentially in size, again because 
the number of reachable closures grows exponentially. However, each symbolic closure
is distinct, so no sharing is possible. The term ending in \clj{xi}
is given type \clj{Pci}, below.

\begin{cljlisting}
Pc1 = {x Int}                                      @Pz
Pc2 = {x Pc1}                                      @Pz
Pc3 = {x {x Pc2}@Pz}                               @Pz
Pc4 = {x {x {x {x {x Pc3}@Pz}@Pz}@Pz}@Pz}          @Pz
Pc5 = {x {x {x {x {x {x {x {x {x {x {x {x Pc4}
       @Pz}@Pz}@Pz}@Pz}@Pz}@Pz}@Pz}@Pz}@Pz}@Pz}@Pz}@Pz
                                        
\end{cljlisting}

% https://link.springer.com/content/pdf/10.1007%2FBFb0032745.pdf
% Type-Directed Flow Analysis for Typed Intermediate Languages - Suresh Jagannathan, Stephen Week~s, and Andrew Wright 
% - seems like their "abstract-closures" probably relate to symbolic closures
% - they "exploit types to control a flow analysis algorithm"
\paragraph{Flow analysis for typed languages}
Jagannathan, Weeks and Wright~\cite{jagannathan1997type}
give a flow analysis for a typed intermediate language.
Their ``abstract closures'' resemble our symbolic closures,
and, like ours, their algorithm is not guaranteed to terminate.
Our work explicitly integrates symbolic closures as a new type
in the language and therefore assists in type inference,
whereas their main result is a separate flow analysis that 
exploit types to increase flow accuracy.

% Faithful Translations between Polyvariant Flows and Polymorphic Types
% http://people.cs.ksu.edu/~tamtoft/Papers/Amt+Tur:FTPFPT-2000/short.pdf
% Torben Amtoft and Franklyn Turbak

% Gilray's thesis
% pg 44 has a long list of references to check
% https://thomas.gilray.org/pdf/thesis-gilray.pdf

%TODO
%\paragraph{Directional Polymorphism}

% MLsub
% https://www.cl.cam.ac.uk/~sd601/thesis.pdf
% https://www.cl.cam.ac.uk/~sd601/papers/mlsub-preprint.pdf
%
% Polar type system (Jim)
% http://citeseerx.ist.psu.edu/viewdoc/download?doi=10.1.1.123.8718&rep=rep1&type=pdf

% Pottier
% Simplifying Subtyping Constraints: A Theory
% http://citeseerx.ist.psu.edu/viewdoc/download?doi=10.1.1.41.7032&rep=rep1&type=pdf
% A Framework for Type Inference with Subtyping%
% http://citeseerx.ist.psu.edu/viewdoc/download?doi=10.1.1.55.2364&rep=rep1&type=pdf

% ML_F
% http://gallium.inria.fr/~remy/mlf/mlf-type-inference-long.pdf


%TODO
%Xie and Oliveira~\cite{xie2018let} present a type system where
%argument type information flows to the function position in applications.
%Then, defining `let` as sugar propagates enough information to avoid
%a custom rule for `let`.
%No information is propagated from functions to applications, so the benefits
%of Colored Local Type Inference are negated.



\chapter{Future work}

The most pressing future work for Typed Clojure is part of the ongoing work
presented after \partref{part:types}.

\section{Future work for Automatic Annotations}
A larger scale investigation of Clojure usage patterns is now possible by
repurposing the automatic annotation tool described in \partref{part:autoann}
to generate and enforce clojure.spec annotations.
As well as testing the robustness of the tool's design, 
the resulting data would be
useful in investigating general questions like how effectively Clojure users utilize unit and generative testing,
how Clojure code evolves of code over time, and the prevalence of idioms that Typed Clojure and clojure.spec
have (and have not) been designed around.

\section{Future work for Extensible Types}
\partref{part:implementations} outlines a code analyzer that paves the way to a future implementation
of extensible typing rules for Typed Clojure.
The next steps in this direction involve deciding the user interface for such a system
and performing a survey of commonly used macros to determine which features must be supported.

\section{Future work for Symbolic Closures}
Symbolic closures (\partref{part:symbolic-closures})
show much promise in improving the user-experience of Typed Clojure.
However, our preliminary work is still not well understood.
\chapref{chapter:symbolic:metatheory} outlines several conjectures we hope to first prove.
Finally, the problem of integrating symbolic closures with type argument synthesis
is a crucial piece of future work, that (we hope) will prove symbolic closures
as indispensable in checking many common Clojure problems.

% Possible future work on Higher-rank types
% - see https://www.microsoft.com/en-us/research/publication/practical-type-inference-for-arbitrary-rank-types/
%   - looks like the journal version of boxy types?
%   - some notes from the paper
%     - a predicative type system only allows a polytype to be instantiated with monotypes
%     - ML_F is both impredicative and supports type inference (but costly to implement & formalize)
%       - also infers principal types
%     - higher-kinded types are orthogonal to higher-rank types, and Haskell's implementation of the former
%       happen to work well with higher-rank types (but no explanation)
%     - the concept of "syntax-directed" rules is given lots of a explanation
%     - \vdash^inst compares two _polytypes_
%     - Kfoury and Wells 1994 show that typeability of System F (with completely erased annotations) is decidable for rank 2 
%       but undecidable for rank 3>=
%     - LTI == "partial" type inference
%       - in the sense that it's not-complete (can't check all programs)
%     - nice discussion of partial type inference


\printbibliography


\counterwithin{figure}{section}
\counterwithin{assumption}{section}
\counterwithin{theorem}{section}
\counterwithin{lemma}{section}
\counterwithin{definition}{section}

% instead of writing a wrapper for \part* (see thesis-format.sty),
% I'll just comment this out
%\part*{Appendices}

\appendix

\chapter{Full rules for \lambdatc{}}

\begin{figure}[h]
$$
\begin{altgrammar}
  \expd{}, \e{} &::=& \x{}
                      \alt \val{} 
                      \alt {\comb {\e{}} {\e{}}} 
                      \alt {\abs {\x{}} {\ty{}} {\e{}}}
                      \alt {\ifexp {\e{}} {\e{}} {\e{}}}
                      \alt {\doexp {\e{}} {\e{}}}
                      \\
                      &\alt& {\letexp {\x{}} {\e{}} {\e{}}}
                      \alt {\wrongorerror{}}
                      \alt {\ReflectiveExp{}}
                      \alt {\NonReflectiveExp{}}
                      \alt {\MultimethodExp{}}
                      %\alt {\HintedExp{}}
                      \alt {\HashMapExp{}}
                &\mbox{Expressions} \\
  \val{} &::=&          \singletonmeta{}
                      \alt \classvaluemeta{}
                      \alt {\emptymap{}}
                      \alt {\const{}}
                      \alt {\num{}}
                      \alt {\str{}}
                      \alt \mapval{}
                      \alt {\closure {\openv{}} {\abs {\x{}} {\ty{}} {\e{}}}}
                      \alt {\multi {\val{}} {\disptable{}}}
                &\mbox{Values} \\
  \mapval{} &::=&  {\curlymapvaloverright{\val{}}{\val{}}}
                &\mbox{Map Values} \\
                \constantssyntax{}\\
%  \HintedExp{}             &::=& \typehintedexpsyntax{}
%                &\mbox{Type Hinted Expressions} \\
  \HashMapExp{}                &::=& \hmapexpressionsyntax{}
                &\mbox{Hash Maps} \\
  \NonReflectiveExp{}     &::=& \nonreflectiveexpsyntax{}
                &\mbox{Non-Reflective %Java
                       Interop} \\
  \ReflectiveExp{}     &::=& \reflectiveexpsyntax{}
                &\mbox{Reflective %Java
                       Interop} \\
  \MultimethodExp{}     &::=& \multimethodexpsyntax{}
                &\mbox{%Immutable First-Class 
                Multimethods}
                      \\%\\ save space...
  \s{}, \ty{}    &::=& \Top 
                      \alt \class{}
                      \alt {\Value \singletonmeta{}} 
                      \alt {\Unionsplice {\overrightarrow{\ty{}}}}
                      \alt
                      {\ArrowOne {\x{}} {\ty{}}
                                   {\ty{}}
                                   {\filterset {\prop{}} {\prop{}}}
                                   {\object{}}}
                      \\
                      &\alt& {\HMapgeneric {\mandatory{}} {\absent{}}}
                      \alt {\MultiFntype{\ty{}}{\ty{}}}
                      
                &\mbox{Types} \\
                \auxhmapsyntax{}\\
  \singletonallsyntax{}
                \\% \\ save space...
                \openvsyntax{}\\\\

                %\tatypesyntax{}\\\\

  \occurrencetypingsyntax{}
  %\pathelemsyntax{}\\
  \propenvsyntax{}
  \\\\

 \disptablesyntax{} \\
 %\typehintenvsyntax{} \\
\classtableallsyntax{} \\
               \classliteralallsyntax{}\\

               \classvaluesyntaxentry{}\\
                      \\
  \wrongorerror{} &::=& \wrong{} \alt \errorvalv{}
                &\mbox{Wrong or error}
                      \\
  \definedreduction{} &::=& \val{} \alt \wrongorerror{}
                 &\mbox{Defined reductions}
                 \\
  \polaritymeta{} &::=& \pluspolarityliteral \alt \minuspolarityliteral
                 &\mbox{Substitution Polarity}
\end{altgrammar}
$$
\caption{Syntax of Terms, Types, Propositions, and Objects}
\end{figure}

\begin{figure}
$$
\begin{array}{lllr}
  \Nil &\equiv& {\ValueNil}\\
  \True &\equiv& {\ValueTrue}\\
  \False &\equiv& {\ValueFalse}\\
\end{array}
$$
\caption{Type abbreviations}
\end{figure}

\begin{figure}
$$
\begin{array}{lllr}
  \judgementtwo{\propenv{}}{\e{}}{\ty{}} &\equiv& 
  \judgement{\propenv{}}{\e{}}{\ty{}}{\filterset{\thenprop{\prop{}}}{\elseprop{\prop{}}}}{\object{}}
  & \text{for some}\ {\thenprop{\prop{}}}, {\elseprop{\prop{}}} \text{and}\ {\object{}}

  \\
  {\replacefor{\ty{}}{\object{}}{\x{}}} &\equiv& {\pluspolarity{\replacefor{\ty{}}{\object{}}{\x{}}}}
  \\
  {\replacefor{\prop{}}{\object{}}{\x{}}} &\equiv&  {\pluspolarity{\replacefor{\prop{}}{\object{}}{\x{}}}}
  \\
  {\replacefor{\filterset{\prop{}}{\prop{}}}{\object{}}{\x{}}} &\equiv&  {\pluspolarity{\replacefor{\filterset{\prop{}}{\prop{}}}{\object{}}{\x{}}}}
  \\
  {\replacefor{\object{}}{\object{}}{\x{}}} &\equiv& {\pluspolarity{\replacefor{\object{}}{\object{}}{\x{}}}}

\end{array}
$$
\caption{Judgment abbreviations}
\end{figure}


\begin{figure*}
\begin{mathpar}

  {\TLocal}

{\TConst}

{\TTrue}

{\TFalse}

{\TNil}

{\TNum}

{\TDo}

{\TIf}

{\TLet}
                 
{\TApp}

{\TAbs}

\infer [T-Clos]
{ \exists {\propenv{}}. \satisfies{\openv{}}{\propenv{}}
  \ \text{and}\ 
\judgementrewrite {\propenv{}} {\abs {\x{}} {\ty{}} {\e{}}} {\s{}}
                 {\filterset {\thenprop {\prop{}}}
                             {\elseprop {\prop{}}}}
                 {\object{}}
                 {\abs {\x{}} {\ty{}} {\ep{}}}
              }
{ \judgementrewrite {}
            {\closure {\openv{}} {\abs {\x{}} {\ty{}} {\e{}}}} 
                      {\s{}}
             {\filterset {\thenprop {\prop{}}}
                         {\elseprop {\prop{}}}}
             {\object{}}
            {\closure {\openv{}} {\abs {\x{}} {\ty{}} {\ep{}}}}
          }

           {\TError}

         {\TSubsume}
\end{mathpar}
\caption{Standard Typing Rules}
\end{figure*}


\begin{figure*}
\begin{mathpar}

{\TNew}

{\TNewStatic}

{\TField}

{\TFieldStatic}

{\TMethod}

{\TMethodStatic}

{\TClass}

{\TInstance}
\end{mathpar}
\caption{Java Interop Typing Rules}
\end{figure*}

\begin{figure*}
\begin{mathpar}

  \TDefMulti{}

  \TDefMethod{}

\TIsA{}

\infer [T-Multi]
{ \judgementtworewrite {} {\val{}} {\ty{}} {\vp{}}
  \\
  \overrightarrow{\judgementtworewrite{}{\val{k}}{\Top}{\vp{k}}}
  \\
  \overrightarrow{\judgementtworewrite{}{\val{v}}{\s{}}{\vp{v}}}
}
{ \judgementrewrite {}
  {\multi {\val{}} {\curlymapvaloverright{\val{k}}{\val{v}}}}
                      {\MultiFntype {\s{}} {\ty{}}}
             {\filterset {\topprop{}} {\botprop{}}}
           {\emptyobject{}}
  {\multi {\vp{}} {\curlymapvaloverright{\vp{k}}{\vp{v}}}}
}

\end{mathpar}
\caption{Multimethod Typing Rules}
\end{figure*}

\begin{figure*}
\begin{mathpar}

%\infer [T-Get]
%{ \judgementtwo {\propenv{}} {\hastype {\e{m}} {\Map {\ty{k}}{\ty{v}}}}
%  \\
%  \judgementtwo {\propenv{}} {\hastype {\e{k}} {\Top}}}
%{ \judgement {\propenv{}} {\hastype {\getexp {\e{m}} {\e{k}}} {\Union{\ty{v}}{\nil{}}}}
%             {\filterset {\topprop{}} {\topprop{}}}
%           {\emptyobject{}}}
%
%\infer [T-Assoc]
%{ 
%  \judgementtwo {\propenv{}} {\hastype {\e{m}} {\Map{\ty{k}}{\ty{v}}}}
%  \\
%  \judgementtwo {\propenv{}} {\hastype {\e{k}} {\ty{k}}}
%  \\
%  \judgementtwo {\propenv{}} {\hastype {\e{v}} {\ty{v}}}
%}
%{ \judgement {\propenv{}} 
%             {\hastype {\assocexp {\e{m}} {\e{k}} {\e{v}}} {\Map {\ty{k}}{\ty{v}}}}
%             {\filterset {\topprop{}} {\botprop{}}}
%             {\emptyobject{}}
%}

\infer [T-HMap]
{ \overrightarrow{\judgementtworewrite {} {\val{k}}{\Value \kw{}}{\vp{k}}}\\
  \overrightarrow{\judgementtworewrite {} {\val{v}}{\ty{v}}{\vp{v}}}\\
  \mandatory{} = \mandatorysetoverright{\kw{}}{\ty{v}}
}
{ \judgementrewrite {}
             {\curlymapvaloverright{\val{k}}{\val{v}}}
                       {\HMapc {\mandatory{}}}
             {\filterset {\topprop{}} {\botprop{}}}
             {\emptyobject{}}
             {\curlymapvaloverright{\vp{k}}{\vp{v}}}
           }

    {\TKw}

    {\TGetHMap}

    {\TGetAbsent}

    {\TGetHMapPartialDefault}

    {\TAssoc}

\end{mathpar}
\caption{Map Typing Rules}
\end{figure*}


%\input{tools-analyzer-figure}

\begin{figure*}
\begin{mathpar}
\objectsub{}

\standardsubtyping{}
\SPMultiFn{}
\Multisubtyping{}

\HMapsubtyping{}
\end{mathpar}
\caption{Subtyping rules}
\end{figure*}


%$$
%\begin{tdisplay}{Evaluation Contexts}
%  \begin{altgrammar}
%    \E{} &::=& [ ] % application rules
%              \alt (\c{}\ \overrightarrow{\val{}}\ \E{}\ \overrightarrow{\exp{}}) % eval arguments left-to-right
%              % map rules
%              \alt \{\overrightarrow{\val{}\ \val{}}\ \E{}\ \exp{}\ \overrightarrow{\exp{}\ \exp{}} \} % key first
%              \alt \{\overrightarrow{\val{}\ \val{}}\ \val{}\ \E{}\ \overrightarrow{\exp{}\ \exp{}} \}   % value next
%              &\mbox{Evaluation Contexts}
%  \end{altgrammar}
%\end{tdisplay}
%$$ 

%{\classtablelookupfigure}

{\convertjavatypefigure{figure*}{}}

\begin{figure}
\begin{mathpar}
\constanttypefigure{}
\end{mathpar}
\caption{Constant Typing}
\end{figure}

\constantsemfigure{appendix}


\begin{figure*}
\isapropsfigure{}

\isaopsemfigure{}
\caption{Definition of isa?}
\end{figure*}

%\begin{figure*}
%$$
%\begin{array}{llrr}
%  \isacompare{\HVec{\overrightarrow{{\ty{}};{\prop{}};{\object{}}}^i}}
%             {\object{}}
%             {\HVec{\overrightarrow{{\ty{}};{\prop{}};{\object{}}}^j}}
%             {\replacefor
%              {\filtersetparen
%                {\isprop {\HVec{\overrightarrow{\isacomparethree{\ty{i}}{\object{i}}{\ty{j}}}}}{\x{}}}
%                {\notprop{\HVec{\overrightarrow{\isacomparethree{\ty{i}}{\object{i}}{\ty{j}}}}}{\x{}}}}
%              {\object{}}
%              {\x{}}}
%              & i = j
%\end{array}
%$$
%$$
%\begin{array}{lclr}
%  \isaopsem{\rtvector{\overrightarrow{\x{}}^i}}{\rtvector{\overrightarrow{\x{}}^j}} &=& {\true{}}
%                                                                                    & i = j, \overrightarrow{\isaopsem{\x{i}}{\x{j}} = {\true{}}}^{i,j}
%  \\
%\end{array}
%$$
%\caption{isa? Vector Extensions}
%\end{figure*}

\begin{figure*}
  \getmethodfigure{}
\caption{Definition of get-method}
\end{figure*}


%\clearpage

\begin{figure*}
\begin{mathpar}

\BLocal{}

\BDo{}

\BLet{}

\BVal{}

\BIfTrue{}

\BIfFalse{}

\BAbs{}

\BBetaClosure{}

\BDelta{}

\BBetaMulti{}

\BField{}

\BMethod{}

\BNew{}

       \BDefMulti{}

       \BDefMethod{}

       \BIsA{}

       {\BAssoc}

       {\BGet}

       {\BGetMissing}
\end{mathpar}
\caption{Operational Semantics}
\label{appendix:figure:opsem}
\end{figure*}

\begin{figure*}
\begin{mathpar}

\infer [BS-MethodRefl]
{}
{\opsem {\openv{}} {\methodexp {mth} {\e{}} {\overrightarrow{\e{}}}}
        {\wrong{}}}

\infer [BS-FieldRefl]
{}
{\opsem {\openv{}} {\fieldexp {\fld{}} {\e{}}}
        {\wrong{}}}

\infer [BS-NewRefl]
{}
{\opsem {\openv{}} {\fieldexp {\fld{}} {\e{}}}
        {\wrong{}}}


\infer [BS-Beta]
{ \opsem {\openv{}}
         {\e{f}}
         {\val{}}
         \\\\
  {\val{}} \not= {\const{}}
  \\
  {\val{}} \not= {\multi {\val{d}} {\disptable{}}}
  \\\\
  {\val{}} \not= {\closure {\openv{c}} {\abs {\x{}} {\ty{}} {\e{b}}}}
       }
{ \opsem {\openv{}}
         {\appexp {\e{f}} {\e{a}}}
         {\wrong{}}
       }

\infer [BS-BetaMulti]
{ \opsem {\openv{}}
         {\e{f}}
         {\multi {\val{}} {\disptable{}}}
         \\\\
  {\val{}} \not= {\const{}}
  \\
  {\val{}} \not= {\multi {\val{d}} {\disptable{}}}
  \\\\
  {\val{}} \not= {\closure {\openv{c}} {\abs {\x{}} {\ty{}} {\e{b}}}}
       }
{ \opsem {\openv{}}
         {\appexp {\e{f}} {\e{a}}}
         {\wrong{}}
       }

\infer [BS-FieldTarget]
{ \opsem {\openv{}}
         {\e{}} 
       {\val{1}}
         \\\\
         {\val{}} \not= {\classvalue{\classhint{1}} {\overrightarrow {\classfieldpair{\fld{i}} {\val{i}}}}}
       }
{ \opsem {\openv{}}
         {\fieldstaticexp {\classhint{1}} {\classhint{2}} {\fld{}} {\e{}}}
         {\wrong{}}
   }

\infer [BS-FieldMissing]
{ \opsem {\openv{}}
         {\e{}} 
       {\classvalue{\classhint{1}} {\overrightarrow {\classfieldpair{\fld{i}} {\val{i}}}}}
       \\
       \fld{} \not\in \{\overrightarrow{\fld{i}}\}
       }
{ \opsem {\openv{}}
         {\fieldstaticexp {\classhint{1}} {\classhint{2}} {\fld{}} {\e{}}}
         {\wrong{}}
   }


\infer [BS-MethodTarget]
{ \opsem {\openv{}}
         {\e{m}}
         {\val{}}
  \\
         {\val{}} \not= {\classvalue{\classhint{1}} {\overrightarrow {\classfieldpair{\fld{i}} {\val{i}}}}}
}
{\opsem {\openv{}}
        {\methodstaticexp {\classhint{1}} {\overrightarrow{\classhint{a}}} {\classhint{2}} {mth} {\e{m}} {\overrightarrow{\e{a}}}}
        {\wrong{}}
      }

\infer [BS-MethodArity]
{ i \not= a
}
{\opsem {\openv{}}
        {\methodstaticexp {\classhint{1}} {\overrightarrow{\classhint{i}}} {\classhint{2}} {mth} {\e{m}} {\overrightarrow{\e{a}}}}
        {\wrong{}}
      }

\infer [BS-MethodArg]
{ \opsem {\openv{}}
         {\e{m}}
         {\val{m}}
  \\
  \overrightarrow{
  \opsem {\openv{}}
         {\e{a}}
         {\val{a}}
       }
       \\\\
  \exists a.\ 
    \val{a} \not=\ {\classvalue{\classhint{a}} {\overrightarrow {\classfieldpair{\fld{i}} {\val{i}}}}}\ or\ \val{a} \not= \nil{}
}
{\opsem {\openv{}}
        {\methodstaticexp {\classhint{1}} {\overrightarrow{\classhint{a}}} {\classhint{2}} {mth} {\e{m}} {\overrightarrow{\e{a}}}}
        {\wrong{}}
      }

\infer [BS-NewArg]
{ \overrightarrow{
  \opsem {\openv{}}
         {\e{i}}
         {\val{i}}
     }
       \\\\
  \exists i.\ 
    \val{i} \not=\ {\classvalue{\classhint{i}} {\overrightarrow {\classfieldpair{\fld{i}} {\val{i}}}}}\ or\ \val{i} \not= \nil{}
}
{\opsem {\openv{}}
        {\newstaticexp {\overrightarrow{\classhint{i}}} {\classhint{1}} 
                       {\class{}} {\overrightarrow{\e{i}}}}
        {\wrong{}}
      }

\infer [BS-NewArity]
{ i \not= a
}
{\opsem {\openv{}}
        {\newstaticexp {\overrightarrow{\classhint{i}}} {\classhint{1}} 
                       {\class{}} {\overrightarrow{\e{a}}}}
        {\wrong{}}
      }

\infer [BS-AssocMap]
{\opsem {\openv{}}
        {\e{m}} {\val{}}
        \\
        \val{} \not= {\curlymap{\overrightarrow{({\val{a}}\ {\val{b}})}}}
}
{
 \opsem {\openv{}}
        {\assocexp {\e{m}} {\e{k}} {\e{v}}} 
        {\wrong{}}
                }

\infer [BS-AssocKey]
{\opsem {\openv{}}
        {\e{m}} {\curlymap{\overrightarrow{({\val{a}}\ {\val{b}})}}}
        \\
 \opsem {\openv{}} {\e{k}} {\val{k}}
 \\\\
 {\val{k}} \not= \kw{}
}
{
 \opsem {\openv{}}
        {\assocexp {\e{m}} {\e{k}} {\e{v}}} 
        {\wrong{}}
                }

\infer [BS-GetMap]
{ \opsem {\openv{}}
         {\e{m}} {\val{}}
        \\
        \val{} \not= {\curlymap{\overrightarrow{({\val{a}}\ {\val{b}})}}}
}
{\opsem {\openv{}}
        {\getexp {\e{m}} {\e{k}}}
        {\wrong{}}
}

\infer [BS-GetKey]
{ \opsem {\openv{}}
         {\e{m}} {\val{}}
        \\
 \opsem {\openv{}}
        {\e{k}} {\val{k}}
        \\\\
      \val{} \not= {\kw{}}
}
{\opsem {\openv{}}
        {\getexp {\e{m}} {\e{k}}}
        {\wrong{}}
}

\infer [BS-Local]
{ \notinopenv {\openv{}} {\x{}}}
{ \opsem {\openv{}} {\x{}} {\wrong{}} }

\infer [BS-DefMethod]
{ \opsem {\openv{}}
         {\e{m}}
         {\val{m}}
         \\
         \val{m} \not= {\multi {\val{d}} {\disptable{}}}
}
{\opsem {\openv{}}
        {\extendmultiexp {\e{m}} {\e{v}} {\e{f}}}
        {\wrong{}}
      }

\end{mathpar}
\caption{Stuck programs}
\end{figure*}

\begin{figure*}
\begin{mathpar}
\infer [BE-ErrorWrong]
{}
{ \opsem {\openv{}} 
         {\wrongorerror{}}
         {\wrongorerror{}}}

\infer [BE-Let]
{ \opsem {\openv{}} {\e{a}} {\wrongorerror{}}
 }
{ \opsem {\openv{}} 
         {\letexp {\x{}} {\e{a}} {\e{}}}
       {\wrongorerror{}}}

\infer [BE-Do1]
{ \opsem {\openv{}} {\e{1}} {\wrongorerror{}} }
{ \opsem {\openv{}} {\doexp{\e{1}}{\e{}}} {\wrongorerror{}}}

\infer [BE-Do2]
{ \opsem {\openv{}} {\e{1}} {\val{1}} 
  \\\\
  \opsem {\openv{}} {\e{}}  {\wrongorerror{}}
}
{ \opsem {\openv{}} {\doexp{\e{1}}{\e{}}} {\wrongorerror{}} }

\infer [BE-If]
{  \opsem {\openv{}} {\e{1}} {\wrongorerror{}}
}
{ \opsem {\openv{}}
         {\ifexp {\e1} {\e2} {\e3}}
         {\wrongorerror{}}
       }

\infer [BE-IfTrue]
{ \opsem {\openv{}} {\e{1}} {\val{1}}
  \\\\
  {\val{1}} \not= {\false{}}
  \\
  {\val{1}} \not= {\nil{}}
  \\\\
  \opsem {\openv{}} {\e{2}} {\wrongorerror{}}
}
{ \opsem {\openv{}}
         {\ifexp {\e1} {\e2} {\e3}}
         {\wrongorerror{}}
       }

\infer [BE-IfFalse]
{  \opsem {\openv{}} {\e{1}} {\false{}}
  \ \ \text{or}\ \ 
  \opsem {\openv{}} {\e{1}} {\nil{}}
  \\\\
  \opsem {\openv{}} {\e{3}} {\wrongorerror{}}
}
{ \opsem {\openv{}}
         {\ifexp {\e1} {\e2} {\e3}}
         {\wrongorerror{}}
       }

\infer [BE-Beta1]
{ \opsem {\openv{}}
         {\e{f}}
         {\wrongorerror{}}
       }
{ \opsem {\openv{}}
         {\appexp {\e{f}} {\e{a}}}
         {\wrongorerror{}}
       }

\infer [BE-Beta2]
{ \opsem {\openv{}}
         {\e{f}}
         {\val{f}}
         \\\\
  \opsem {\openv{}}
         {\e{a}}
         {\wrongorerror{}}
       }
{ \opsem {\openv{}}
         {\appexp {\e{f}} {\e{a}}}
         {\wrongorerror{}}
       }

\infer [BE-BetaClosure]
{ \opsem {\openv{}}
         {\e{f}}
         {\closure {\openv{c}} {\abs {\x{}} {\ty{}} {\e{b}}}}
         \\\\
  \opsem {\openv{}}
         {\e{a}}
         {\val{a}}
         \\\\
  \opsem {\extendopenv {\openv{c}} {\x{}} {\val{a}}}
         {\e{b}}
         {\wrongorerror{}}
       }
{ \opsem {\openv{}}
         {\appexp {\e{f}} {\e{a}}}
         {\wrongorerror{}}
       }

\infer [BE-BetaMulti1]
{ \opsem {\openv{}}
         {\e{f}}
         {\multi {\val{d}} {m}}
         \\\\
  \opsem {\openv{}}
         {\e{a}}
         {\val{a}}
         \\\\
  \opsem {\openv{}}
         {\appexp {\val{d}} {\val{a}}}
         {\wrongorerror{}}
       }
{ \opsem {\openv{}}
         {\appexp {\e{f}} {\e{a}}}
         {\wrongorerror{}}
       }

\infer [BE-BetaMulti2]
{ \opsem {\openv{}}
         {\e{f}}
         {\multi {\val{d}} {m}}
         \\\\
  \opsem {\openv{}}
         {\e{a}}
         {\val{a}}
         \\\\
  \opsem {\openv{}}
         {\appexp {\val{d}} {\val{a}}}
         {\val{e}}
         \\\\
  \getmethoderr {\disptable{}}
             {\val{e}}
             {\errorvalv{}}
       }
{ \opsem {\openv{}}
         {\appexp {\e{f}} {\e{a}}}
         {\errorvalv{}}
       }

\infer [BE-Delta]
{ \opsem {\openv{}} {\e{}} {\const{}}
  \\\\
  \opsem {\openv{}} {\ep{}} {\val{}}
  \\\\
  \constantopsem{\const{}}{\val{}} = \wrongorerror{}
}
{ \opsem {\openv{}}
         {\appexp {\e{}} {\ep{}}}
         {\wrongorerror{}}
       }

\infer [BE-Field]
{ \opsem {\openv{}}
         {\e{}} 
         {\wrongorerror{}}
       }
{ \opsem {\openv{}}
         {\fieldstaticexp {\classhint{1}} {\classhint{2}} {\fld{}} {\e{}}}
         {\wrongorerror{}}
   }

\infer [BE-Method1]
{ \opsem {\openv{}}
         {\e{m}}
         {\wrongorerror{}}
}
{\opsem {\openv{}}
        {\methodstaticexp {\classhint{1}} {\overrightarrow{\classhint{a}}} {\classhint{2}} {mth} {\e{m}} {\overrightarrow{\e{}}}}
        {\wrongorerror{}}
      }

\infer [BE-Method2]
{ \opsem {\openv{}}
         {\e{m}}
         {\val{m}}
  \\\\
  \overrightarrow{
  \opsem {\openv{}}
         {\e{n-1}}
         {\val{n-1}}
       }
         \\\\
  \opsem {\openv{}}
         {\e{n}}
         {\wrongorerror{}}
}
{\opsem {\openv{}}
        {\methodstaticexp {\classhint{1}} {\overrightarrow{\classhint{a}}} {\classhint{2}} {mth} {\e{m}} {\overrightarrow{\e{}}}}
        {\wrongorerror{}}
      }

\infer [BE-Method3]
{ \opsem {\openv{}}
         {\e{m}}
         {\val{m}}
  \\
  \overrightarrow{
  \opsem {\openv{}}
         {\e{a}}
         {\val{a}}
       }
  \\\\
  \invokejavamethod {\classhint{1}} {\val{m}} {mth}
                    {\overrightarrow{\classhint{a}}} {\overrightarrow{\val{a}}}
                    {\classhint{2}}
                    {\errorvalv{}}
}
{\opsem {\openv{}}
        {\methodstaticexp {\classhint{1}} {\overrightarrow{\classhint{a}}} {\classhint{2}} {mth} {\e{m}} {\overrightarrow{\e{a}}}}
        {\errorvalv{}}
      }

\infer [BE-New1]
{ \overrightarrow{
  \opsem {\openv{}}
         {\e{n-1}}
         {\val{n-1}}
       }
       \\\\
  \opsem {\openv{}}
         {\e{n}}
         {\wrongorerror{}}
       }
{ \opsem {\openv{}}
         {\newstaticexp {\overrightarrow{\classhint{i}}} {\classhint{1}} 
                        {\class{}} {\overrightarrow{\e{}}}}
         {\wrongorerror{}}
       }

\infer [BE-New2]
{ 
  \overrightarrow{
  \opsem {\openv{}}
         {\e{i}}
         {\val{i}}
       }
         \\\\
         \newjava {\classhint{1}}
                  {\overrightarrow{\classhint{i}}}
                  {\overrightarrow{\val{i}}}
                  {\errorvalv{}}
       }
{ \opsem {\openv{}}
         {\newstaticexp {\overrightarrow{\classhint{i}}} {\classhint{1}} 
                        {\class{}} {\overrightarrow{\e{i}}}}
         {\errorvalv{}}}

\infer [BE-DefMulti]
{ \opsem {\openv{}} {\e{d}} {\wrongorerror{}}
}
{\opsem {\openv{}}
        {\createmultiexp {\ty{}}
                         {\e{d}}}
        {\wrongorerror{}}
}

\infer [BE-DefMethod1]
{ \opsem {\openv{}}
         {\e{m}}
         {\wrongorerror{}}
}
{\opsem {\openv{}}
        {\extendmultiexp {\e{m}} {\e{v}} {\e{f}}}
        {\wrongorerror{}}
      }

\infer [BE-DefMethod2]
{ \opsem {\openv{}}
         {\e{m}}
         {\multi {\val{d}} {\disptable{}}}
         \\\\
  \opsem {\openv{}}
         {\e{v}}
         {\wrongorerror{}}
}
{\opsem {\openv{}}
        {\extendmultiexp {\e{m}} {\e{v}} {\e{f}}}
        {\wrongorerror{}}
      }

\infer [BE-DefMethod3]
{ \opsem {\openv{}}
         {\e{m}}
         {\multi {\val{d}} {\disptable{}}}
         \\\\
  \opsem {\openv{}}
         {\e{v}}
         {\val{v}}
         \\\\
  \opsem {\openv{}}
         {\e{f}}
         {\wrongorerror{}}
}
{\opsem {\openv{}}
        {\extendmultiexp {\e{m}} {\e{v}} {\e{f}}}
         {\wrongorerror{}}
      }
\end{mathpar}
\caption{Error and stuck propagation (continued in Figure~\ref{appendix:figure:errorstuck2})}
\label{appendix:figure:errorstuck1}
\end{figure*}

\begin{figure*}
\begin{mathpar}

\infer [BE-IsA1]
{ \opsem {\openv{}} {\e{1}} {\wrongorerror{}}
}
{\opsem {\openv{}} {\isaapp {\e{1}} {\e{2}}} {\wrongorerror{}}}

\infer [BE-IsA2]
{ \opsem {\openv{}} {\e{1}} {\val{1}}
  \\\\
  \opsem {\openv{}} {\e{2}} {\wrongorerror{}}
}
{\opsem {\openv{}} {\isaapp {\e{1}} {\e{2}}} {\wrongorerror{}}}

\infer [BE-Assoc1]
{\opsem {\openv{}}
        {\e{m}}{\wrongorerror{}} 
}
{
 \opsem {\openv{}}
        {\assocexp {\e{m}} {\e{k}} {\e{v}}} 
        {\wrongorerror{}}
                }

\infer [BE-Assoc2]
{\opsem {\openv{}}
        {\e{m}} {\curlymap{\overrightarrow{({\val{a}}\ {\val{b}})}}}
        \\
 \opsem {\openv{}}
        {\e{k}}{\wrongorerror{}}
}
{
 \opsem {\openv{}}
        {\assocexp {\e{m}} {\e{k}} {\e{v}}} 
        {\wrongorerror{}}
                }

\infer [BE-Assoc3]
{\opsem {\openv{}}
        {\e{m}} {\curlymap{\overrightarrow{({\val{a}}\ {\val{b}})}}}
        \\
 \opsem {\openv{}}
        {\e{k}} {\val{k}}
        \\
 \opsem {\openv{}}
        {\e{v}} {\wrongorerror{}}
}
{
 \opsem {\openv{}}
        {\assocexp {\e{m}} {\e{k}} {\e{v}}} 
        {\wrongorerror{}}
                }

\infer [BE-Get1]
{\opsem {\openv{}}
        {\e{m}} {\wrongorerror{}}
}
{
 \opsem {\openv{}}
        {\getexp {\e{m}} {\e{k}}}
        {\wrongorerror{}}
}

\infer [BE-Get2]
{\opsem {\openv{}}
        {\e{m}} {\curlymap{\overrightarrow{({\val{a}}\ {\val{b}})}}}
        \\
 \opsem {\openv{}}
        {\e{k}} {\wrongorerror{}}
}
{
 \opsem {\openv{}}
        {\getexp {\e{m}} {\e{k}}}
        {\wrongorerror{}}
}
\end{mathpar}
\caption{Error and stuck propagation (continued from Figure~\ref{appendix:figure:errorstuck1})}
\label{appendix:figure:errorstuck2}
\end{figure*}





\begin{figure*}
\begin{mathpar}

\begin{array}{lllll}
  \inopenvalign{\openv{}}{\x{}}{\val{} & {\roundpair{\x{}}{\val{}}} \in \openv{}}\\
  \inopenvalign{\openv{}}{\pth {\keype{k}} {\object{}}}{\getexp {{\openv{}}(\object{})}{\kw{}}}\\
  \inopenvalign{\openv{}}{\pth {\classpe{}} {\object{}}}{\appexp {\classconst{}} {{\openv{}}(\object{})}}

\end{array}

\end{mathpar}
\caption{Path translation}
\end{figure*}

\begin{figure*}
\begin{mathpar}

\begin{array}{lllll}
\updatefigure{}
\end{array}

\begin{array}{lllll}
\restrictremovefigure{}
\end{array}

\end{mathpar}
\caption{Type Update}
\label{appendix:updaterestrictremove}
\end{figure*}


\begin{figure*}
\begin{mathpar}
\infer [M-Or]
{ \satisfies{\openv{}}{\prop{1}}\ \text{or}\  \satisfies{\openv{}}{\prop{2}}}
{ \satisfies{\openv{}}{\orprop{\prop{1}}{\prop{2}}}
                   }

\infer [M-Imp]
{ \satisfies{\openv{}}{\prop{1}}\ \text{implies}\ \satisfies{\openv{}}{\prop{2}}}
{ \satisfies{\openv{}}{\impprop{\prop{1}}{\prop{2}}}
                   }

\infer [M-And]
{ \satisfies{\openv{}}{\prop{1}}
\\ \satisfies{\openv{}}{\prop{2}}}
{ \satisfies{\openv{}}{\andprop{\prop{1}}{\prop{2}}}
                   }


\infer [M-Top]
{}
{ \satisfies{\openv{}}{\topprop{}}
                   }

                   \\

\infer [M-Type]
{ \judgement {} {\openv{}({\pth{\pathelem{}}{\x{}}})} {\ty{}}{\filterset{\thenprop{\prop{}}}{\elseprop{\prop{}}}}{\object{}}}
{ \satisfies{\openv{}}{\isprop{\ty{}}{\pth{\pathelem{}}{\x{}}}}
                   }

\infer [M-NotType]
{ \judgement {} {\openv{}({\pth{\pathelem{}}{\x{}}})} {\s{}}{\filterset{\thenprop{\prop{}}}{\elseprop{\prop{}}}}{\object{}}
\\\\
\text{there is no}\ \val{}\ \text{such that}\ \judgement{}{\val{}}{\ty{}}{\filterset{\thenprop{\prop{1}}}{\elseprop{\prop{1}}}}{\object{1}}
\ \text{and}\ \judgement{}{\val{}}{\s{}}{\filterset{\thenprop{\prop{2}}}{\elseprop{\prop{2}}}}{\object{2}}
}
{ \satisfies{\openv{}}{\notprop{\ty{}}{\pth{\pathelem{}}{\x{}}}}
                   }
\end{mathpar}
\caption{Satisfaction Relation}
\end{figure*}

\begin{figure*}
\begin{mathpar}
\infer [L-Atom]
{ {\prop{}} \in {\propenv{}}}
{ \inpropenv {\propenv{}} {\prop{}}
}

\infer [L-True]
{}
{ \inpropenv {\propenv{}} {\topprop{}}}

\infer [L-False]
{ \inpropenv {\propenv{}} {\botprop{}}}
{ \inpropenv {\propenv{}} {\prop{}}}

\infer [L-AndI]
{ \inpropenv {\propenv{}} {\prop{1}}
  \\\\
  \inpropenv {\propenv{}} {\prop{2}}}
{ \inpropenv {\propenv{}} {\andprop {\prop{1}}{\prop{2}}}}

\infer [L-AndE]
{ \inpropenv {\propenv{}, {\prop{1}}, {\prop{2}}} {\prop{}} }
{ \inpropenv {\propenv{}, {\andprop {\prop{1}}{\prop{2}}}} {\prop{}}}

\infer [L-ImplI]
{ \inpropenv {\propenv{}, {\prop{1}}} {\prop{2}}}
{ \inpropenv {\propenv{}} {\impprop {\prop{1}}{\prop{2}}}}

\infer [L-ImplE]
{ \inpropenv {\propenv{}} {\prop{1}}
  \\\\
  \inpropenv {\propenv{}} {\impprop {\prop{1}}{\prop{2}}}}
{ \inpropenv {\propenv{}} {\prop{2}}}

\infer [L-OrI]
{ \inpropenv {\propenv{}} {\prop{1}}
  \ \text{or}\ 
  \inpropenv {\propenv{}} {\prop{2}}}
{ \inpropenv {\propenv{}} {\orprop {\prop{1}}{\prop{2}}}}


\infer [L-OrE]
{ \inpropenv {\propenv{}, {\prop{1}}}{\prop{}}
  \\\\
  \inpropenv {\propenv{}, {\prop{2}}}{\prop{}}}
{ \inpropenv {\propenv{}, {\orprop {\prop{1}}{\prop{2}}}}{\prop{}}}

\infer [L-Sub]
{ \inpropenv {\propenv{}} {\isprop {\ty{}}{\pth {\pathelem{}} {\x{}}}}
  \\
  \issubtypein {} {\ty{}}{\s{}}
}
{ \inpropenv {\propenv{}} {\isprop {\s{}}{\pth {\pathelem{}} {\x{}}}}}

\infer [L-SubNot]
{ \inpropenv {\propenv{}} {\notprop {\s{}}{\pth {\pathelem{}} {\x{}}}}
  \\
  \issubtypein {} {\ty{}}{\s{}}}
{ \inpropenv {\propenv{}} {\notprop {\ty{}}{\pth {\pathelem{}} {\x{}}}}}

\infer [L-Bot]
{ \inpropenv {\propenv{}} {\isprop {\Bot} {\pth {\pathelem{}} {\x{}}}}}
{ \inpropenv {\propenv{}} {\prop{}}}

{\LUpdate}

\\

\text{(The metavariable \propisnotmeta{} ranges over \ty{} and \nottype{\ty{}} (without variables).)}

\end{mathpar}
\caption{Proof System}
\label{appendix:figure:proofsystem}
\end{figure*}

\begin{figure*}
$$
\begin{array}{lclr}

{\withpolarity
  {\replacefor
    {\filterset {\thenprop {\prop{}}}{\elseprop {\prop{}}}}
    {\object{}}
    {\x{}}}
  {\polaritymeta{}}}
  &=&
{\filterset 
  {\withpolarity
    {\replacefor
      {\thenprop {\prop{}}}
      {\object{}}
      {\x{}}}
    {\polaritymeta{}}}
  {\withpolarity
    {\replacefor
      {\elseprop {\prop{}}}
      {\object{}}
      {\x{}}}
    {\polaritymeta{}}}}
\\\\
{\withpolarity
  {\replacefor
    {\isprop {\propisnotmeta{}} {\pth {\pathelem{}} {\x{}}}}
    {\pth {\pathelemp{}} {\y{}}}
    {\x{}}}
  {\polaritymeta{}}}
&=&
  {\isprop {({\withpolarity
              {\replacefor
               {\propisnotmeta{}}
               {\pth {\pathelemp{}} {\y{}}}
               {\x{}}}
              {\polaritymeta{}}})}
           {{\pathelem{}}({\pth {\pathelemp{}} {\y{}}})}}
           \\

{\pluspolarity
{\replacefor
  {\isprop {\propisnotmeta{}} {\pth {\pathelem{}} {\x{}}}}
  {\emptyobject{}}
  {\x{}}}
}
&=&
{\topprop{}}
\\
{\minuspolarity
{\replacefor
  {\isprop {\propisnotmeta{}} {\pth {\pathelem{}} {\x{}}}}
  {\emptyobject{}}
  {\x{}}}
}
&=&
{\botprop{}}

\\
{\withpolarity
{\replacefor
  {\isprop {\propisnotmeta{}} {\pth {\pathelem{}} {\x{}}}}
  {\object{}}
  {\z{}}}
{\polaritymeta{}}}
&=&
  {\isprop {\propisnotmeta{}} {\pth {\pathelem{}} {\x{}}}}
  & \x{} \not= \z{}\ \text{and}\ \z{} \not\in {\fv {\propisnotmeta{}}}

\\
{\pluspolarity
{\replacefor
  {\isprop {\propisnotmeta{}} {\pth {\pathelem{}} {\x{}}}}
  {\object{}}
  {\z{}}}
}
&=&
{\topprop{}}
  & \x{} \not= \z{}\ \text{and}\ \z{} \in {\fv {\propisnotmeta{}}}
\\
{\minuspolarity
{\replacefor
  {\isprop {\propisnotmeta{}} {\pth {\pathelem{}} {\x{}}}}
  {\object{}}
  {\z{}}}
}
&=&
{\botprop{}}
  & \x{} \not= \z{}\ \text{and}\ \z{} \in {\fv {\propisnotmeta{}}}

\\
{\withpolarity
{\replacefor
  {\topprop{}}
  {\object{}}
  {\x{}}}
{\polaritymeta{}}}
&=&
  {\topprop{}}

\\
{\withpolarity
{\replacefor
  {\botprop{}}
  {\object{}}
  {\x{}}}
{\polaritymeta{}}}
&=&
  {\botprop{}}

\\
{\pluspolarity
{\replacefor
  {({\impprop {\prop{1}} {\prop{2}}})}
  {\object{}}
  {\x{}}}
}
&=&
{\impprop 
  {\minuspolarity {\replacefor {\prop{1}} {\object{}} {\x{}}}}
  {\pluspolarity {\replacefor {\prop{2}} {\object{}} {\x{}}}}}
\\
{\minuspolarity
{\replacefor
  {({\impprop {\prop{1}} {\prop{2}}})}
  {\object{}}
  {\x{}}}
}
&=&
{\impprop 
  {\pluspolarity {\replacefor {\prop{1}} {\object{}} {\x{}}}}
  {\minuspolarity {\replacefor {\prop{2}} {\object{}} {\x{}}}}}
\\
{\withpolarity
{\replacefor
  {({\orprop {\prop{1}} {\prop{2}}})}
  {\object{}}
  {\x{}}}
{\polaritymeta{}}}
&=&
{\orprop 
  {\withpolarity
    {\replacefor {\prop{1}} {\object{}} {\x{}}}
    {\polaritymeta{}}}
  {\withpolarity
    {\replacefor {\prop{2}} {\object{}} {\x{}}}
    {\polaritymeta{}}}}
\\
{\withpolarity
{\replacefor
  {({\andprop {\prop{1}} {\prop{2}}})}
  {\object{}}
  {\x{}}}
{\polaritymeta{}}}
&=&
{\andprop 
{\withpolarity
  {\replacefor {\prop{1}} {\object{}} {\x{}}}
  {\polaritymeta{}}}
{\withpolarity
  {\replacefor {\prop{2}} {\object{}} {\x{}}}
  {\polaritymeta{}}}}

    \\\\

{\withpolarity
{\replacefor
  {\pth {\pathelem{}} {\x{}}}
  {\pth {\pathelemp{}} {\y{}}}
  {\x{}}}
{\polaritymeta{}}}
           &=&
{\pth{\pathelem{}}{\pth {\pathelemp{}} {\y{}}}}

    \\

{\withpolarity
{\replacefor
  {\pth {\pathelem{}} {\x{}}}
  {\emptyobject{}}
  {\x{}}}
{\polaritymeta{}}}
           &=&
{\emptyobject{}}

    \\

{\withpolarity
{\replacefor
  {\pth {\pathelem{}} {\x{}}}
  {\object{}}
  {\z{}}}
{\polaritymeta{}}}
           &=&
{\pth {\pathelem{}} {\x{}}}

& \x{} \not= \z{}
    \\

{\withpolarity
{\replacefor
  {\emptyobject{}}
  {\object{}}
  {\x{}}}
{\polaritymeta{}}}
           &=&
{\emptyobject{}}

\end{array}
$$
\center{\text{Substitution on types is capture-avoiding structural recursion.}}
\caption{Substitution}
\end{figure*}


\chapter{Soundness for \lambdatc{}}

{\javaassumptionsall{appendix}}

\begin{lemma} \label{appendix:lemma:envagree}
  If \openv{} and \openvp{} agree on \fv{\prop{}}
  and \satisfies{\openv{}}{\prop{}}
  then \satisfies{\openvp{}}{\prop{}}.
\begin{proof}
  Since the relevant parts of \openv{} and \openvp{} agree, the proof follows trivially.
\end{proof}
\end{lemma}

\begin{lemma} \label{appendix:lemma:substfilter}
  If 
  \begin{itemize}
    \item \prop{1} = {\replacefor {\prop{2}} {\object{}} {\x{}}},
    \item
  {\satisfies{\openv{2}}{\prop{2}}},
    \item
  $\forall v \in \fv{\prop{2}} - \x{}$.
                              {\inopenvnoeq{\openv{1}}{v}} = {\inopenvnoeq {\openv{2}}{v}},
    \item
  and {\inopenvnoeq{\openv{2}}{\x{}}} = {\inopenvnoeq{\openv{1}}{\object{}}}
  \end{itemize}
  then \satisfies{\openv{1}}{\prop{1}}.

  \begin{proof}
    By induction on the derivation of the model judgement.
  \end{proof}
\end{lemma}

\begin{lemma} \label{appendix:lemma:satisfies}
  If \satisfies{\openv{}}{\propenv{}} and \inpropenv{\propenv{}}{\prop{}} then \satisfies{\openv{}}{\prop{}}.

  \begin{proof}
    By structural induction on \inpropenv{\propenv{}}{\prop{}}.
%    \begin{itemize}
%      \item[]
%        \begin{case}[L-True]
%
%          Holds by M-Top.
%        \end{case}
%      \item[]
%        \begin{case}[L-False]
%          {\inpropenv{\propenv{}}{\botprop{}}}
%
%          ??? TODO
%        \end{case}
%      \item[]
%        \begin{case}[L-AndI]
%          \inpropenv{\propenv{}}{\andprop{\prop{1}}{\prop{2}}}, \satisfies{\openv{}}{\propenv{}}
%
%          By inversion on the proof system we know \inpropenv{\propenv{}}{\prop{1}}
%          and
%          \inpropenv{\propenv{}}{\prop{2}}.
%
%          By the induction hypothesis we know \satisfies{\openv{}}{\prop{1}}
%          and
%          \satisfies{\openv{}}{\prop{2}}.
%
%          By M-And we know \satisfies{\openv{}}{\andprop{\prop{1}}{\prop{2}}}
%          and we are done.
%        \end{case}
%      \item[]
%        \begin{case}[L-AndE]
%          \inpropenv{\propenv{},{\andprop{\prop{1}}{\prop{2}}}}{\prop{}}, \satisfies{\openv{}}{\propenv{},{\andprop{\prop{1}}{\prop{2}}}}
%
%
%          By inversion on the proof system we know  either
%          \inpropenv{\propenv{},{\prop{1}}}{\prop{}}
%          or
%          \inpropenv{\propenv{},{\prop{2}}}{\prop{}}.
%
%          %TODO
%         % By the induction hypothesis we know 
%         % either
%         % \satisfies{\openv{}}{\prop{1}}
%         % and
%         % \satisfies{\openv{}}{\prop{2}}.
%        \end{case}
%    \end{itemize}
  \end{proof}
\end{lemma}

\begin{lemma} \label{appendix:lemma:goodobjects+ve}
  If \inpropenv{\propenv{}}{\isprop{\ty{}}{\pth{\pathelem{}}{\x{}}}},
  \satisfies{\openv{}}{\propenv{}}
  and \inopenv{\openv{}}{\pth{\pathelem{}}{\x{}}}{\val{}}
  then
  \judgementselfrewrite{}{\val{}}{\ty{}}{\filterset{\thenprop{\propp{}}}{\elseprop{\propp{}}}}{\objectp{}}
  for some {\thenprop{\propp{}}}, {\elseprop{\propp{}}} and {\objectp{}}.
  \begin{proof}
    Corollary of lemma~\ref{appendix:lemma:satisfies}.
  \end{proof}
\end{lemma}

\begin{lemma}[Paths are independent] \label{appendix:lemma:pathindependent}
  If \inopenvnoeq{\openv{}}{\object{}} = \inopenvnoeq{\openv{1}}{\objectp{}}
  then \inopenvnoeq{\openv{}}{\pth{\pathelem{}}{\object{}}} =
       \inopenvnoeq{\openv{1}}{\pth{\pathelem{}}{\objectp{}}}
 \begin{proof}
   By induction on \pathelem{}.
   % FIXME
%   \begin{case}[\pathelem{} = \emptypath{}]
%     \inopenvnoeq{\openv{}}{\object{}} = {\inopenvnoeq{\openv{}}{\objectp{}}}
%
%     As 
%     \inopenvnoeq{\openv{}}{\pth{\emptypath{}}{\object{}}} = \inopenvnoeq{\openv{}}{\object{}}
%     and
%     \inopenvnoeq{\openv{}}{\pth{\emptypath{}}{\objectp{}}} = \inopenvnoeq{\openv{}}{\objectp{}}
%     we can conclude 
%     \inopenvnoeq{\openv{}}{\pth{\emptypath{}}{\object{}}} = \inopenvnoeq{\openv{}}{\pth{\emptypath{}}{\objectp{}}}.
%   \end{case}
%   \begin{case}[\pathelem{} = {\destructpath{\pesyntax{}}{\pathelem{1}}}]
%     \inopenvnoeq{\openv{}}{\object{}} = {\inopenvnoeq{\openv{}}{\objectp{}}}
%
%     By cases on \pesyntax{}.
%
%     \begin{itemize}
%       \item[]
%   \begin{subcase}[\pesyntax{} = {\keype{\kw{}}}] 
%
%%     TODO
%     By the induction hypothesis on {\pathelem{1}}
%     we know {\inopenvnoeq{\openv{}}{\pth{\pathelem{1}}{\object{}}}} =
%             {\inopenvnoeq{\openv{1}}{\pth{\pathelem{1}}{\objectp{}}}}.
%             By the definition of pth translation 
%             {\inopenvnoeq{\openv{}}{\pth{\pathelem{1}}{\object{}}}} = {\getexp {{\openv{}}(\object{})}{\kw{}}}
%             and 
%             {\inopenvnoeq{\openv{}}{\pth{\pathelem{1}}{\objectp{}}}} = {\getexp {{\openv{}}(\objectp{})}{\kw{}}}
%   \end{subcase} 
%     \end{itemize}
%%     TODO
%   \end{case}
 \end{proof}
\end{lemma}

\begin{lemma}[\classconst]\label{appendix:lemma:classconst}
  If
  {\opsem{\openv{}}{\appexp{\classconst{}}{\openv{}({\pth{\pathelem{}}{\x{}}})}}{\class{}}} then
  {\satisfies{\openv{}}{\isprop{\class{}}{\pth{\pathelem{}}{\x{}}}}}.

  \begin{proof}
    Induction on the definition of {\classconst{}}.
  \end{proof}
\end{lemma}

{\consistentwithdefinition{appendix}}

{\istruefalsedefinitions{appendix}}

%\begin{lemma}[Path substitution] \label{appendix:lemma:pathsubustitution}
%  If \satisfies{\openv{}}{\prop{}} and 
%  \openv(\object{}) = \openv(\objectp{})
%  then \satisfies{\openv{}}{\replacefor{\prop{}}{\object{}}{\objectp{}}}.
%  \begin{proof}
%    By straightforward induction on \prop{}.
%  \end{proof}
%\end{lemma}
%

\begin{lemma}[isa? has correct propositions] \label{appendix:lemma:isa}
  If
  \begin{itemize}
    \item
  \judgementrewrite {\propenv{}} {\val{1}} {\ty{1}}
             {\filterset {\thenprop {\prop{1}}}
                         {\elseprop {\prop{1}}}}
                       {\object{1}}
                       {\val{1}},
    \item
  \judgementrewrite {\propenv{}} {\val{2}} {\ty{2}}
             {\filterset {\thenprop {\prop{2}}}
                         {\elseprop {\prop{2}}}}
                       {\object{2}}
                       {\val{2}},
    \item
        \isaopsem{\val{1}}{\val{2}} = {\val{}}, 
    \item
        \satisfies{\openv{}}{\propenv{}},
    \item
  \isacompare{\ty{1}}{\object{1}}{\ty{2}}{\filterset {\thenprop {\propp{}}} {\elseprop {\propp{}}}},
    \item
        \inpropenv{\thenprop{\propp{}}}{\thenprop{\prop{}}}, and
    \item
        \inpropenv{\elseprop{\propp{}}}{\elseprop{\prop{}}},
    \end{itemize}
  then either
\begin{itemize}
  \item
        if
        \istrueval{\val{}}
        then {\satisfies{\openv{}}{\thenprop{\prop{}}}}, or
  \item
        if
        \isfalseval{\val{}}
        then {\satisfies{\openv{}}{\elseprop{\prop{}}}}.
\end{itemize}
\begin{proof}
        By cases on the definition of \isaopsemliteral
        and subcases on \isaopsemliteral.

        \begin{itemize} % isaopsem
          \item[]
            \begin{subcase}[\isaopsem{\val{1}}{\val{1}} = {\true{}}, \text{if} \val{1} \notequal\ {\class{}}]
              \ 

              \val{1} = \val{2}, \val{1} \notequal\ {\class{}}, \val{2} \notequal\ {\class{}}, \istrueval{\val{}}
              
              Since \istrueval{\val{}} we prove {\satisfies{\openv{}}{\thenprop{\prop{}}}}
              by cases on the definition of \isacompareliteral{}:
              \begin{itemize} % isacompare
                \item[]
                  \begin{subcase}[\isacompare{\s{}}{\pth{\classpe{}}{\pth{\pathelem{}}{\x{}}}}{\Value{\class{}}}
                                 {\filterset{\isprop{\class{}} {\pth{\pathelem{}}{\x{}}}}
                                            {\notprop{\class{}}{\pth{\pathelem{}}{\x{}}}}}]
                    \ 


                    \object{1} = {\pth{\classpe{}}{\pth{\pathelem{}}{\x{}}}},
                    \ty{2} = {\Value{\class{}}},
                    \inpropenv{\isprop{\class{}} {\pth{\pathelem{}}{\x{}}}}{\thenprop{\prop{}}}

                    Unreachable by inversion on the typing relation, since \ty{2} = {\Value{\class{}}},
                    yet \val{2} \notequal\ {\class{}}.

%                    By inversion on the typing relation, since \classpe{} is the last path element of \object{1}
%                    then \opsem{\openv{}}{\appexp{\classconst{}}{\openv{}({\pth{\pathelem{}}{\x{}}})}}{\val{1}}.
%
%                    Since {\val{1}} = {\val{2}} then {\ty{1}} = {\ty{2}}, and because {\ty{2}} = {\Value{\class{}}}
%                    then {\ty{1}} = {\Value{\class{}}}.
%
%                    By inversion {\val{1}} = {\class{}}, via T-Class.
%
%                    Since {\opsem{\openv{}}{\appexp{\classconst{}}{\openv{}({\pth{\pathelem{}}{\x{}}})}}{\class{}}}
%                    we conclude by lemma~\ref{appendix:lemma:classconst}
%                    with {\satisfies{\openv{}}{\isprop{\class{}} {\pth{\pathelem{}}{\x{}}}}}.

                  \end{subcase}
                \item[]
                  \begin{subcase}[\isacompare{\s{}}{\object{}}{\Value{\singletonmeta{}}}
                    {\replacefor
                      {\filtersetparen{\isprop{\Value{\singletonmeta{}}} {\x{}}}
                        {\notprop{\Value{\singletonmeta{}}}{\x{}}}}
                      {\object{}}
                      {\x{}}}\ 
                    \text{if}\ {\singletonmeta{}} \notequal \class{}]
                    \ 

                    \ty{2} = {\Value{\singletonmeta{}}}, 
                    {\singletonmeta{}} \notequal \class{},
                    \inpropenv{\replacefor{\isprop{\Value{\singletonmeta{}}} {\x{}}}
                                           {\object{1}}
                                           {\x{}}}{\thenprop{\prop{}}}
                    %\elseprop{\prop{}} = {\replacefor{\notprop{\Value{\singletonmeta{}}} {\x{}}}
                    %                       {\object{1}}
                    %                       {\x{}}}

                    Since \ty{2} = {\Value{\singletonmeta{}}} where {\singletonmeta{}} \notequal \class{},
                    by inversion on the typing judgement 
                    {\val{2}} is either \true{}, \false{}, \nil{} or \kw{}
                    by T-True, T-False, T-Nil or T-Kw.

                    Since \val{1} = {\val{2}} then \ty{1} = \ty{2}, and since \ty{2} = {\Value{\singletonmeta{}}}
                      then \ty{1} = {\Value{\singletonmeta{}}}, so
                    \judgementtwo {} {\val{1}} {\Value{\singletonmeta{}}}

                    If \object{1} = \emptyobject{} then \thenprop{\prop{}} = \topprop{} and
                    we derive
                    {\satisfies{\openv{}}{\topprop{}}} with M-Top.

                    Otherwise \object{1} = \pth{\pathelem{}}{\x{}} and 
                    \inpropenv{\isprop{\Value{\singletonmeta{}}}{\pth{\pathelem{}}{\x{}}}}{\thenprop{\prop{}}},
                    and since
                    \judgementtwo {} {\val{1}} {\Value{\singletonmeta{}}}
                    then
                    \judgementtwo {} {{\openv{}}(\pth{\pathelem{}}{\x{}})} {\Value{\singletonmeta{}}},
                    which we can use M-Type to derive
                    {\satisfies{\openv{}}{\isprop{\Value{\singletonmeta{}}}{\pth{\pathelem{}}{\x{}}}}}.
                  \end{subcase}
                \item[]
                  \begin{subcase}[\isacompare{\s{}}{\object{}}{\ty{}} {\filterset{\topprop{}} {\topprop{}}}]
                    \ 

                    {\thenprop{\prop{}}} = {\topprop{}}

                    {\satisfies{\openv{}}{\topprop{}}} holds by M-Top.

                  \end{subcase}
              \end{itemize}
            \end{subcase}
          \item[]
            \begin{subcase}[\isaopsem{\class{1}}{\class{2}} = {\true{}}, \text{if}\ \issubtypein{}{\class{1}}{\class{2}}]
              \ 

              \val{1} = \class{1}, \val{2} = \class{2},
              \issubtypein{}{\class{1}}{\class{2}},
              \istrueval{\val{}}
              
              Since \istrueval{\val{}} we prove {\satisfies{\openv{}}{\thenprop{\prop{}}}}
              by cases on the definition of \isacompareliteral{}:
              \begin{itemize} % isacompare
                \item[]
                  \begin{subcase}[\isacompare{\s{}}{\pth{\classpe{}}{\pth{\pathelem{}}{\x{}}}}{\Value{\class{}}}
                                 {\filterset{\isprop{\class{}} {\pth{\pathelem{}}{\x{}}}}
                                            {\notprop{\class{}}{\pth{\pathelem{}}{\x{}}}}}]
                    \ 


                    \object{1} = {\pth{\classpe{}}{\pth{\pathelem{}}{\x{}}}},
                    \ty{2} = {\Value{\class{2}}},
                    \inpropenv{\isprop{\class{2}} {\pth{\pathelem{}}{\x{}}}}{\thenprop{\prop{}}}

                    By inversion on the typing relation, since \classpe{} is the last path element of \object{1}
                    then \opsem{\openv{}}{\appexp{\classconst{}}{\openv{}({\pth{\pathelem{}}{\x{}}})}}{\val{1}}.

                    Since {\opsem{\openv{}}{\appexp{\classconst{}}{\openv{}({\pth{\pathelem{}}{\x{}}})}}{\class{1}}},
                    as {\val{1}} = {\class{1}},
                    we can derive from lemma~\ref{appendix:lemma:classconst}
                    {\satisfies{\openv{}}{\isprop{\class{1}} {\pth{\pathelem{}}{\x{}}}}}.

                    By the induction hypothesis we can derive 
                    {\inpropenv{\propenv{}}{\isprop{\class{1}} {\pth{\pathelem{}}{\x{}}}}},
                    and with the fact {\issubtypein{}{\class{1}}{\class{2}}}
                    we can use L-Sub to conclude 
                    {\inpropenv{\propenv{}}{\isprop{\class{2}} {\pth{\pathelem{}}{\x{}}}}},
                    and finally by lemma~\ref{appendix:lemma:satisfies}
                    we derive
                    {\satisfies{\openv{}}{\isprop{\class{2}} {\pth{\pathelem{}}{\x{}}}}}.

                  \end{subcase}
                \item[]
                  \begin{subcase}[\isacompare{\s{}}{\object{}}{\Value{\singletonmeta{}}}
                    {\replacefor
                      {\filtersetparen{\isprop{\Value{\singletonmeta{}}} {\x{}}}
                        {\notprop{\Value{\singletonmeta{}}}{\x{}}}}
                      {\object{}}
                      {\x{}}}\ 
                    \text{if}\ {\singletonmeta{}} \notequal \class{}]
                    \ 

                    \ty{2} = {\Value{\singletonmeta{}}}, 
                    {\singletonmeta{}} \notequal \class{},
                    \inpropenv{\replacefor{\isprop{\Value{\singletonmeta{}}} {\x{}}}
                                           {\object{1}}
                                           {\x{}}}{\thenprop{\prop{}}}

                    Unreachable case since 
                    \ty{2} = {\Value{\singletonmeta{}}} where 
                    {\singletonmeta{}} \notequal \class{},
                    but \val{2} = \class{2}.
                  \end{subcase}
                \item[]
                  \begin{subcase}[\isacompare{\s{}}{\object{}}{\ty{}} {\filterset{\topprop{}} {\topprop{}}}]
                    \ 

                    {\thenprop{\prop{}}} = {\topprop{}}

                    {\satisfies{\openv{}}{\topprop{}}} holds by M-Top.

                  \end{subcase}
              \end{itemize}
            \end{subcase}
          \item[]
            \begin{subcase}[\isaopsem{\val{1}}{\val{2}} = {\false{}}, otherwise]
              \ 

              \val{1} \notequal\ \val{2},
              \isfalseval{\val{}}
              
              Since \isfalseval{\val{}} we prove {\satisfies{\openv{}}{\elseprop{\prop{}}}}
              by cases on the definition of \isacompareliteral{}:
              \begin{itemize} % isacompare
                \item[]
                  \begin{subcase}[\isacompare{\s{}}{\pth{\classpe{}}{\pth{\pathelem{}}{\x{}}}}{\Value{\class{}}}
                                 {\filterset{\isprop{\class{}} {\pth{\pathelem{}}{\x{}}}}
                                            {\notprop{\class{}}{\pth{\pathelem{}}{\x{}}}}}]
                    \ 


                    \object{1} = {\pth{\classpe{}}{\pth{\pathelem{}}{\x{}}}},
                    \ty{2} = {\Value{\class{}}},
                    \inpropenv{\notprop{\class{}} {\pth{\pathelem{}}{\x{}}}}{\elseprop{\prop{}}}

                    By inversion on the typing relation, since \classpe{} is the last path element of \object{1}
                    then \opsem{\openv{}}{\appexp{\classconst{}}{\openv{}({\pth{\pathelem{}}{\x{}}})}}{\val{1}}.
                    
                    By the definition of {\classconst{}} either {\val{1}} = {\class{}} or {\val{1}} = \nil{}.

                    If {\val{1}} = \nil{}, then we know from the definition of \isaopsemliteral that 
                    {\openv{}({\pth{\pathelem{}}{\x{}}})} = \nil{}.

                    Since \judgementtwo{}{\openv{}({\pth{\pathelem{}}{\x{}}})}{\Nil},
                    and there is no \val{1} such that both \judgementtwo{}{\openv{}({\pth{\pathelem{}}{\x{}}})}{\class} and
                    \judgementtwo{}{\openv{}({\pth{\pathelem{}}{\x{}}})}{\Nil{}},
                    we use M-NotType to derive 
                    \satisfies{\openv{}}{\notprop{\class{}} {\pth{\pathelem{}}{\x{}}}}.

                    Similarly if {\val{1}} = \class{1}, by the definition of \isacompareliteral
                    we know {\notsubtypein{}{\class{1}}{\class{}}} and 
                    {\openv{}({\pth{\pathelem{}}{\x{}}})} = \class{1}.

                    Since \judgementtwo{}{\openv{}({\pth{\pathelem{}}{\x{}}})}{\class{1}},
                    and there is no \val{1} such that both 
                    \judgementtwo{}{\val{1}}{\class{}} and
                    \judgementtwo{}{\val{1}}{\class{1}},
                    we use M-NotType to derive 
                    \satisfies{\openv{}}{\notprop{\class{}} {\pth{\pathelem{}}{\x{}}}}.


                  \end{subcase}
                \item[]
                  \begin{subcase}[\isacompare{\s{}}{\object{}}{\Value{\singletonmeta{}}}
                    {\replacefor
                      {\filtersetparen{\isprop{\Value{\singletonmeta{}}} {\x{}}}
                        {\notprop{\Value{\singletonmeta{}}}{\x{}}}}
                      {\object{}}
                      {\x{}}}\ 
                    \text{if}\ {\singletonmeta{}} \notequal \class{}]
                    \ 

                    \ty{2} = {\Value{\singletonmeta{}}}, 
                    {\singletonmeta{}} \notequal \class{},
                    %\thenprop{\prop{}} = {\replacefor{\isprop{\Value{\singletonmeta{}}} {\x{}}}
                    %                       {\object{1}}
                    %                       {\x{}}}
                    \inpropenv{\replacefor{\notprop{\Value{\singletonmeta{}}} {\x{}}}
                                           {\object{1}}
                                           {\x{}}}{\elseprop{\prop{}}}

                    Since \ty{2} = {\Value{\singletonmeta{}}} where {\singletonmeta{}} \notequal \class{},
                    by inversion on the typing judgement 
                    {\val{2}} is either \true{}, \false{}, \nil{} or \kw{}
                    by T-True, T-False, T-Nil or T-Kw.

                    If \object{1} = \emptyobject{} then \elseprop{\prop{}} = \topprop{} and
                    we derive
                    {\satisfies{\openv{}}{\topprop{}}} with M-Top.

                    Otherwise \object{1} = \pth{\pathelem{}}{\x{}} and 
                    \inpropenv{\notprop{\Value{\singletonmeta{}}}{\pth{\pathelem{}}{\x{}}}}{\elseprop{\prop{}}}.
                    Noting that \val{1} \notequal\ \val{2},
                    we know \judgementtwo{}{\openv{}({\pth{\pathelem{}}{\x{}}})}{\s{}}
                    where \s{} \notequal\ {\Value{\singletonmeta{}}},
                    and there is no \val{1} such that both 
                    \judgementtwo{}{\val{1}}{\Value{\singletonmeta{}}} and
                    \judgementtwo{}{\val{1}}{\s{}}
                    so we can use M-NotType to derive
                    {\satisfies{\openv{}}{\notprop{\Value{\singletonmeta{}}}{\pth{\pathelem{}}{\x{}}}}}.
                  \end{subcase}
                \item[]
                  \begin{subcase}[\isacompare{\s{}}{\object{}}{\ty{}} {\filterset{\topprop{}} {\topprop{}}}]
                    \ 

                    {\elseprop{\prop{}}} = {\topprop{}}

                    {\satisfies{\openv{}}{\topprop{}}} holds by M-Top.

                  \end{subcase}
              \end{itemize}
            \end{subcase}
        \end{itemize}
      \end{proof}
\end{lemma}

\begin{lemma} \label{appendix:lemma:soundness}
{\soundnesslemmahypothesis}
\begin{proof}
By induction and cases on the derivation of \opsem {\openv{}} {\e{}} {\a{}},
and subcases on the penultimate rule of the derivation of
\judgementrewrite{\propenv{}}{\ep{}}{\ty{}}{\filterset{\thenprop{\prop{}}}{\elseprop{\prop{}}}}{\object{}}{\e{}}
followed by T-Subsume as the final rule.

% induction on the derivation of the evaluation semantics because we want to apply
% the induction hypothesis to subderivations of the eval sem. If eg. used the typing
% judgement, we couldn't use the induction hypothesis on applications of higher-order
% functions, since the subderivation of T-Abs wouldn't be present.

\begin{case}[B-Val]

  \begin{itemize}
    \item[] 
      \begin{subcase}[T-True]
        \val{} = \true{},
  \ep{} = \true{},
  \e{} = \true{}, \issubtypein{}{\True}{\ty{}}, \inpropenv{\topprop{}}{\thenprop{\prop{}}}, 
  \inpropenv{\botprop{}}{\elseprop{\prop{}}}, \issubtypein{}{\emptyobject{}}{\object{}}

        Proving part 1 is trivial: \object{} is a superobject of \emptyobject{}, which can only be \emptyobject{}.

        To prove part 2, we note that \val{} = \true{}
        and \inpropenv{\topprop{}}{\thenprop{\prop{}}},
        so \satisfies{\openv{}}{\thenprop{\prop{}}} by M-Top.

        Part 3 holds as \e{} can only be reduced to itself via B-Val.

        Part 4 holds vacuously.
      \end{subcase}
    \item[]
      \begin{subcase}[T-HMap] \val{} = {\curlymapvaloverright{\val{k}}{\val{v}}},
  \ep{} = {\curlymapvaloverright{\val{k}}{\val{v}}},
  \e{} = {\curlymapvaloverright{\val{k}}{\val{v}}},
  \issubtypein{}{\HMapc {\mandatory{}}}{\ty{}},
  \inpropenv{\topprop{}}{\thenprop{\prop{}}},
  \inpropenv{\botprop{}}{\elseprop{\prop{}}},
  \issubtypein{}{\emptyobject{}}{\object{}},
  $\overrightarrow{\judgementtwo {} {\val{k}}{\Value \kw{}}}$,
  $\overrightarrow{\judgementtwo {} {\val{v}}{\ty{v}}}$,
  \mandatory{} = \mandatorysetoverright{\kw{}}{\ty{v}}

        Similar to T-True.

        Part 4 holds by the induction hypothese on {\overr{\val{k}}} and {\overr{\val{v}}}.
      \end{subcase}
    \item[]
      \begin{subcase}[T-Kw] \val{} = {\kw{}},
  \ep{} = {\kw{}},
  \e{} = {\kw{}},
  \issubtypein{}{\Value{\kw{}}}{\ty{}},
  \inpropenv{\topprop{}}{\thenprop{\prop{}}},
  \inpropenv{\botprop{}}{\elseprop{\prop{}}},
  \issubtypein{}{\emptyobject{}}{\object{}}

        Similar to T-True.
      \end{subcase}
      \begin{subcase}[T-Str]
        Similar to T-Kw.
      \end{subcase}
  \item[] 
    \begin{subcase}[T-False]
      \val{} = \false{},
\ep{} = \false, 
\e{} = \false, 
\issubtypein{}{\False}{\ty{}},
\inpropenv{\botprop{}}{\thenprop{\prop{}}},
\inpropenv{\topprop{}}{\elseprop{\prop{}}},
\issubtypein{}{\emptyobject{}}{\object{}}

Proving part 1 is trivial: \object{} is a superobject of \emptyobject{}, which must be \emptyobject{}. 

To prove part 2, we note that \val{} = \false{}
and \inpropenv{\topprop{}}{\elseprop{\prop{}}}, so \satisfies{\openv{}}{\elseprop{\prop{}}} by M-Top. 

Part 3 holds as \e{} can only be reduced to itself via B-Val.

Part 4 holds vacuously.
\end{subcase}
    \item[]
      \begin{subcase}[T-Class] \val{} = {\class{}},
  \ep{} = {\class{}},
  \e{} = {\class{}},
  \issubtypein{}{\Value{\class{}}}{\ty{}},
  \inpropenv{\topprop{}}{\thenprop{\prop{}}},
  \inpropenv{\botprop{}}{\elseprop{\prop{}}},
  \issubtypein{}{\emptyobject{}}{\object{}}

        Similar to T-True.
      \end{subcase}
    \item[]
      \begin{subcase}[T-Instance]
        \val{} = {\classvalue{\classhint{}} {\overrightarrow {\classfieldpair{\fld{i}} {\val{i}}}}},
        \ep{} = {\classvalue{\classhint{}} {\overrightarrow {\classfieldpair{\fld{}} {\val{}}}}},
        \e{} = {\classvalue{\classhint{}} {\overrightarrow {\classfieldpair{\fld{}} {\val{}}}}},
        \issubtypein{}{\class{}}{\ty{}},
        \inpropenv{\topprop{}}{\thenprop{\prop{}}},
        \inpropenv{\botprop{}}{\elseprop{\prop{}}},
        \issubtypein{}{\emptyobject{}}{\object{}}


        Similar to T-True.

        Part 4 holds by the induction hypotheses on ${\overrightarrow{\val{i}}}$.
      \end{subcase}
  \item[] 
    \begin{subcase}[T-Nil] 
      \val{} = \nil{},
\ep{} = \nil, 
\e{} = \nil, 
\issubtypein{}{\Nil}{\ty{}},
\inpropenv{\botprop{}}{\thenprop{\prop{}}},
\inpropenv{\topprop{}}{\elseprop{\prop{}}},
\issubtypein{}{\emptyobject{}}{\object{}}

      Similar to T-False.
\end{subcase}
\item[]
\begin{subcase}[T-Multi] 
  \val{} = {\multi {\val{1}} {\curlymapvaloverright{\val{k}}{\val{v}}}}
  \ep{} = {\multi {\val{1}} {\curlymapvaloverright{\val{k}}{\val{v}}}},
  \judgementtworewrite {} {\val{1}} {\ty{1}}{\val{1}},
  \overr{\judgementtworewrite{}{\val{k}}{\Top{}}{\val{k}}},
  \overr{\judgementtworewrite{}{\val{v}}{\s{}}{\val{v}}},
  \e{} = {\multi {\val{1}} {\curlymapvaloverright{\val{k}}{\val{v}}}},
  \issubtypein{}{\MultiFntype {\s{}} {\ty{1}}}{\ty{}},
  \inpropenv{\topprop{}}{\thenprop{\prop{}}},
  \inpropenv{\botprop{}}{\elseprop{\prop{}}},
  \issubtypein{}{\emptyobject{}}{\object{}}

        Similar to T-True.
\end{subcase}
\item[]
\begin{subcase}[T-Const]
  \e{} = {\const{}},
  \issubtypein{}{\constanttype{\const{}}}{\ty{}},
  \inpropenv{\topprop{}}{\thenprop{\prop{}}},
  \inpropenv{\botprop{}}{\elseprop{\prop{}}},
  \issubobjin{}{\emptyobject{}}{\object{}}

        Parts 1, 2 and 3 hold for the same reasons as T-True. 
\end{subcase}


  \end{itemize}
\end{case}



\begin{case}[B-Local]
{ \inopenv {\openv{}} {\x{}} {\val{}} },
{ \opsem {\openv{}} {\x{}} {\val{}} }

\begin{itemize}
  \item[]
\begin{subcase}[T-Local]
  \ep{} = \x{}, 
  \e{} = \x{}, 
  \inpropenv{\notprop {\falsy{}} {\x{}}}{\thenprop{\prop{}}},
  \inpropenv{\isprop {\falsy{}} {\x{}}}{\elseprop{\prop{}}},
\issubtypein{}{\x{}}{\object{}},
\inpropenv{\propenv{}}{\isprop{\ty{}}{\x{}}}

Part 1 follows from \inopenv{\openv{}}{\object{}} {\val{}}, since either {\object{}} = \x{}
and \inopenv{\openv{}}{\x{}} {\val{}} is a premise of B-Local, or {\object{}} = {\emptyobject{}} which also
satisfies the goal.

Part 2 considers two cases: if \istrueval{\val{}}, then 
\satisfies{\openv{}}{\notprop{\falsy}{\x{}}} holds by M-NotType; if \isfalseval{\val{}}, then 
\satisfies{\openv{}}{\isprop{\falsy}{\x{}}} holds by M-Type.

We prove part 3 by observing
\inpropenv{\propenv{}}{\isprop{\ty{}}{\x{}}},
\satisfies{\openv{}}{\propenv{}},
and
\inopenv {\openv{}} {\x{}} {\val{}}
(by B-Local)
which gives us the desired result.

Part 4 holds vacuously.
\end{subcase}
\end{itemize}

\end{case}

\begin{case}[B-Do]
  \opsem {\openv{}} {\e{1}} {\val{1}},
  \opsem {\openv{}} {\e{2}} {\val{}}

\begin{itemize}
  \item[] \begin{subcase}[T-Do]
      \ep{} = {\doexp {\ep{1}} {\ep{2}}},
  \judgementrewrite {\propenv{}} 
             {\ep{1}} {\ty{1}}
             {\filterset {\thenprop {\prop{1}}} {\elseprop {\prop1}}} 
             {\object{1}}
             {\e{1}},
\judgementrewrite {\propenv{}, {\orprop {\thenprop {\prop{1}}} {\elseprop {\prop{1}}}}}
           {\ep{}} {\ty{}}
           {\filterset {\thenprop {\prop{}}} {\elseprop {\prop{}}}} 
           {\object{}}
           {\e{}},
    \e{} = {\doexp {\e{1}} {\e{2}}}

For all parts we note 
    since {\e{1}} can be either a true or false value
    then
    {\satisfies{\openv{}}{{\propenv{}},{\orprop {\thenprop {\prop{1}}} {\elseprop {\prop{1}}}}}}
    by M-Or,
    which together with 
\judgement {\propenv{}, {\orprop {\thenprop {\prop{1}}} {\elseprop {\prop{1}}}}}
           {\e{2}} {\ty{}}
           {\filterset {\thenprop {\prop{}}} {\elseprop {\prop{}}}} 
           {\object{}},
    and
  \opsem {\openv{}} {\e{2}} {\val{}}
    allows us to apply the induction hypothesis on \e{2}.

To prove part 1 we use the induction hypothesis on \e{2}
to show either \object{} = \emptyobject{} 
or \inopenv {\openv{}} {\object{}} {\val{}}, since \e{} always
evaluates to the result of \e{2}.

For part 2 we use the induction hypothesis on \e{2}
to show if \istrueval{\val{}} then
        {\satisfies{\openv{}}{\thenprop{\prop{}}}}
        or
  if \isfalseval{\val{}} then
        {\satisfies{\openv{}}{\elseprop{\prop{}}}}.

Parts 3 and 4 follow from the induction hypothesis on \e{2}.
    \end{subcase}
\end{itemize}
\end{case}

\begin{case}[BE-Do1]
  \opsem {\openv{}} {\e{1}} {\errorval{\val{e}}},
  \opsem {\openv{}} {\e{}} {\errorval{\val{}}}

        Trivially reduces to an error.
\end{case}

\begin{case}[BE-Do2]
  \opsem {\openv{}} {\e{1}} {\val{1}},
  \opsem {\openv{}} {\e{2}} {\errorvalv{}},
  \opsem {\openv{}} {\e{}} {\errorvalv{}}

        As above.
\end{case}

\begin{case}[B-New]
  $
  \overrightarrow{
  \opsem {\openv{}}
         {\e{i}}
         {\val{i}}
       }$,
         $\newjava {\classhint{1}}
                  {\overrightarrow{\classhint{i}}}
                  {\overrightarrow{\val{i}}}
                  {\val{}}$

\begin{itemize}
  \item[]
\begin{subcase}[T-New]
  \ep{} = {\newexp {\class{}} {\overrightarrow{\ep{i}}}},
  \inct{\ctctorentry{\overr{\class{i}}}}{\ctlookupctors{\ct{}}{\classhint{}}},
  \overr{\javatotcnil{\classhint{i}}{\ty{i}}},
  \overr{
  \judgementtworewrite {\propenv{}}
                    {\ep{i}} {\ty{i}}
                    {\e{i}}
                  },
  \e{} = {\newstaticexp {\overrightarrow{\classhint{i}}} {\classhint{}} 
                                                          {\class{}} {\overrightarrow{\e{i}}}},
  \issubtypein{}{\javatotcexp{\classhint{}}}{\ty{}},
  \inpropenv{\topprop{}}{\thenprop{\prop{}}},
  \inpropenv{\botprop{}}{\elseprop{\prop{}}},
  \issubobjin{}{\emptyobject{}}{\object{}}

Part 1 follows \object{} = \emptyobject{}.

Part 2 requires some explanation. The two false values in Typed Clojure
cannot be constructed with \newliteral{}, so the only case is \val{} $\not=$ \false\ (or \nil)
where \thenprop{\prop{}} = \topprop{} so \satisfies{\openv{}}{\thenprop{\prop{}}}.
\Void{} also lacks a constructor.

Part 3 holds as B-New reduces to a \emph{non-nilable}
instance of \class{} via \newjavaliteral (by assumption~\ref{appendix:assumption:new}), 
and {\ty{}} is a supertype of \javatotcexp{\classhint{}}.

\end{subcase}
\item[]
\begin{subcase}[T-NewStatic]
  {\ep{}} = {\newstaticexp {\overrightarrow{\classhint{i}}} {\classhint{}}
                                                          {\class{}} {\overrightarrow{\e{i}}}}

  Non-reflective constructors cannot be written directly by the user, so we can assume
  the class information attached to the syntax corresponds to an actual constructor by inversion
  from T-New.

  The rest of this case progresses like T-New.
\end{subcase}
\end{itemize}
\end{case}

\begin{case}[BE-New1] $\overrightarrow{
  \opsem {\openv{}}
         {\e{i-1}}
         {\val{i-1}}
       }$,
  \opsem {\openv{}}
         {\e{i}}
         {\errorvalv{}},
  \opsem {\openv{}} {\e{}} {\errorvalv{}}

        Trivially reduces to an error.

\end{case}

\begin{case}[BE-New2] 
  $\overrightarrow{
  \opsem {\openv{}}
         {\e{i}}
         {\val{i}}
       }$,
         \newjava {\classhint{1}}
                  {\overrightarrow{\classhint{i}}}
                  {\overrightarrow{\val{i}}}
                  {\errorvalv{}},
        \opsem {\openv{}} {\e{}} {\errorvalv{}}

        As above.

\end{case}

\begin{case}[B-Field]
  \opsem {\openv{}}
         {\e{1}} 
         {\classvalue{\classhint{1}} {\classfieldpair{\fld{}} {\val{}}}}

\begin{itemize}
  \item[]
\begin{subcase}[T-Field]
  \ep{} = {\fieldexp {\fld{}} {\ep{1}}},
  \judgementtworewrite {\propenv{}} {\ep{}} {\s{}} {\e{}},
  \issubtypein{}{\s{}}{\Object{}},
  \tctojava{\s{}}{\classhint{1}},
  \inct{\ctfldentry{\fld{}}{\classhint{2}}}{\ctlookupfields{\ct{}}{\classhint{1}}},
  \e{} = {\fieldstaticexp {\classhint{1}} {\classhint{2}} {\fld{}} {\e{1}}}
  \issubtypein{}{\javatotcnilexp{\classhint{2}}}{\ty{}},
  \inpropenv{\topprop{}}{\thenprop{\prop{}}},
  \inpropenv{\topprop{}}{\elseprop{\prop{}}},
  \issubobjin{}{\emptyobject{}}{\object{}}


Part 1 is trivial as \object{} is always \emptyobject{}.

Part 2 holds trivially; \val{} can be either a true or false value
and both {\thenprop{\prop{}}} and {\elseprop{\prop{}}}
are \topprop{}.

Part 3 relies on the semantics of \getfieldliteral (assumption~\ref{appendix:assumption:field})
in B-Field, which returns a \emph{nilable} instance of \classhint{2},
and \ty{} is a supertype of \javatotcnilexp{\classhint{2}}.
Notice \issubtypein{}{\s{}}{\Object{}} is required to guard from dereferencing \nil{},
as {\classhint{1}} erases occurrences of \Nil{} in \s{} via  \tctojava{\s{}}{\classhint{1}}.
\end{subcase}
  \item[]

\begin{subcase}[T-FieldStatic]
  {\ep{}} = {\fieldstaticexp {\classhint{1}} {\classhint{2}} {\fld{}} {\e{1}}}

  Non-reflective field lookups cannot be written directly by the user, so we can assume
  the class information attached to the syntax corresponds to an actual field by inversion
  from T-Field.

  The rest of this case progresses like T-Field.
\end{subcase}

\end{itemize}
\end{case}

\begin{case}[BE-Field]
  \opsem {\openv{}}
         {\e{1}} 
         {\errorvalv{}},
  \opsem {\openv{}}
         {\e{}}
         {\errorvalv{}}

         Trivially reduces to an error.

\end{case}

\begin{case}[B-Method]
  \opsem {\openv{}}
         {\e{m}}
         {\val{m}},
  $\overrightarrow{
  \opsem {\openv{}}
         {\e{a}}
         {\val{a}}}$,
  \invokejavamethod {\classhint{1}} {\val{m}} {mth}
                    {\overrightarrow{\classhint{a}}} {\overrightarrow{\val{a}}}
                    {\classhint{2}}
                    {\val{}}

\begin{itemize}
  \item[]
\begin{subcase}[T-Method]
  \judgementtworewrite {\propenv{}} {\ep{}} {\s{}} {\e{}},
             \issubtypein{}{\s{}}{\Object{}},
  \tctojava{\s{}}{\classhint{1}},
                  \inct{\ctmthentry{\mth{}}{\overrightarrow{\classhint{i}}}{\classhint{2}}}{\ctlookupmethods{\ct{}}{\classhint{1}}},
                  \overr{\javatotcnil{\classhint{i}}{\ty{i}}},
             \overr{
  \judgementtworewrite {\propenv{}} {\ep{i}} {\ty{i}} {\e{i}}
                  },
  \e{} = {\methodstaticexp {\classhint{1}} 
                          {\overr {\classhint{i}}} 
                          {\classhint{2}}
                          {\mth{}} {\e{m}} {\overr{\e{a}}}},
                        \issubtypein{}{\javatotcnilexp{\classhint{2}}}{\ty{}},
  \inpropenv{\topprop{}}{\thenprop{\prop{}}},
  \inpropenv{\topprop{}}{\elseprop{\prop{}}},
  \issubobjin{}{\emptyobject{}}{\object{}}


Part 1 is trivial as \object{} is always \emptyobject{}.

Part 2 holds trivially, \val{} can be either a true or false value
and both {\thenprop{\prop{}}} and {\elseprop{\prop{}}}
are \topprop{}.

Part 3 relies on the semantics of \invokejavamethodliteral (assumption~\ref{appendix:assumption:method})
in B-Method, which returns a \emph{nilable} instance of \classhint{2},
and \ty{} is a supertype of \javatotcnil{\classhint{2}}.
Notice \issubtypein{}{\s{}}{\Object{}} is required to guard from dereferencing \nil{},
as {\classhint{1}} erases occurrences of \Nil{} in \s{} via  \tctojava{\s{}}{\classhint{1}}.
\end{subcase}
\item[]
\begin{subcase}[T-MethodStatic]
  \ep{} = 
  {\methodstaticexp {\classhint{1}} 
        {\overrightarrow {\classhint{i}}} 
        {\classhint{2}}
        {\mth{}} {\e{1}} {\overrightarrow{\e{i}}}}

  Non-reflective method invocations cannot be written directly by the user, so we can assume
  the class information attached to the syntax corresponds to an actual method by inversion
  from T-Method.

  The rest of this case progresses like T-Method.
\end{subcase}


\end{itemize}

\end{case}

\begin{case}[BE-Method1]
  \opsem {\openv{}}
         {\e{m}}
         {\errorval{\val{}}},
  \opsem {\openv{}}
         {\e{}}
         {\errorval{\val{}}}

         Trivially reduces to an error.
\end{case}
\begin{case}[BE-Method2]
  \opsem {\openv{}}
         {\e{m}}
         {\val{m}},
 $\overrightarrow{
  \opsem {\openv{}}
         {\e{n-1}}
         {\val{n-1}}
       }$,
  \opsem {\openv{}}
         {\e{n}}
         {\errorval{\val{}}},
  \opsem {\openv{}}
         {\e{}}
         {\errorval{\val{}}}

  As above.
\end{case}
\begin{case}[BE-Method3]
  \opsem {\openv{}}
         {\e{m}}
         {\val{m}},
  $\overrightarrow{
  \opsem {\openv{}}
         {\e{a}}
         {\val{a}}
       }$,
  \invokejavamethod {\classhint{1}} {\val{m}} {mth}
                    {\overrightarrow{\classhint{a}}} {\overrightarrow{\val{a}}}
                    {\classhint{2}}
                    {\errorvalv{}},
  \opsem {\openv{}} {\e{}} {\errorvalv{}}

  As above.

\end{case}

\begin{case}[B-DefMulti]
  \val{} = {\multi {\val{d}} {\emptydisptable}},
  \opsem {\openv{}} {\e{d}} {\val{d}}



\begin{itemize}
  \item[]
\begin{subcase}[T-DefMulti]
  \ep{} = {\createmultiexp {\s{}} {\ep{d}}},
  \s{} = {\ArrowOne {\x{}} {\ty{1}} {\ty{2}}
                          {\filterset {\thenprop {\prop{1}}}
                                      {\elseprop {\prop{1}}}}
                          {\object{1}}},
  \ty{d} = {\ArrowOne {\x{}} {\ty{1}} {\ty{3}}
                          {\filterset {\thenprop {\prop{2}}}
                                      {\elseprop {\prop{2}}}}
                          {\object{2}}},
\judgementtworewrite {\propenv{}} {\ep{}} {\sp{}} {\e{}},
  \e{} = {\createmultiexp {\s{}} {\e{d}}},
  \issubtypein{}{\MultiFntype {\s{}} {\ty{d}}}{\ty{}},
  \inpropenv{\topprop{}}{\thenprop{\prop{}}},
  \inpropenv{\botprop{}}{\elseprop{\prop{}}},
  \issubobjin{}{\emptyobject{}}{\object{}}


Part 1 and 2 hold for the same reasons as T-True.
For part 3 we show \judgementtwo{}{\multi {\val{d}} {\emptydisptable}}{\MultiFntype {\s{}} {\ty{d}}}
by T-Multi, since \judgementtwo {} {\val{d}} {\ty{d}} by the inductive hypothesis on {\e{d}}
and {\emptydisptable} vacuously satisfies the other premises of T-Multi, so we are done.

\end{subcase}
\end{itemize}
\end{case}

\begin{case}[BE-DefMulti] \opsem {\openv{}} {\e{d}} {\errorvalv{}},
        \opsem {\openv{}} {\e{}} {\errorvalv{}}

        Trivially reduces to an error.

\end{case}

\begin{case}[B-DefMethod]

        \ 

        \begin{enumerate}
          \item
       \val{} = {\multi {\val{d}} {\disptablep{}}},
          \item
        \opsem {\openv{}}
               {\e{m}}
               {\multi {\val{d}} {\disptable{}}},
          \item
  \opsem {\openv{}}
         {\e{v}}
         {\val{v}},
          \item
  \opsem {\openv{}}
         {\e{f}}
         {\val{f}},
          \item
         \disptablep{} = {\extenddisptable {\disptable{}} 
                                {\val{v}}
                                {\val{f}}}
        \end{enumerate}

  \begin{itemize}
    \item[]
      \begin{subcase}[T-DefMethod]
        \ 
        
        \begin{enumerate}[resume]

          \item
  \ep{} = {\extendmultiexp {\ep{m}} {\ep{v}} {\ep{f}}},
          \item
  \ty{m} = {\ArrowOne {\x{}} {\ty{1}} {\s{}}
                     {\filterset {\thenprop {\prop{m}}}
                                 {\elseprop {\prop{m}}}}
                     {\object{m}}},
          \item
  \ty{d} = {\ArrowOne {\x{}} {\ty{1}} {\sp{}}
                     {\filterset {\thenprop {\prop{d}}}
                                 {\elseprop {\prop{d}}}}
                     {\object{d}}},
          \item
\judgementtworewrite {\propenv{}}
                  {\ep{m}} {\MultiFntype {\ty{m}} {\ty{d}}}
                  {\e{m}}
          \item
  \isacompare{\sp{}}{\object{d}}{\ty{v}}{\filterset {\thenprop {\prop{i}}} {\elseprop {\prop{i}}}},
          \item
\judgementtworewrite {\propenv{}}
           {\e{v}} {\ty{v}}
           {\e{v}}
          \item
  \judgementrewrite {\propenv{}, {\isprop{\ty{1}} {\x{}}}, {\thenprop {\prop{i}}}}
           {\ep{f}} {\s{}}
           {\filterset {\thenprop {\prop{m}}}
                       {\elseprop {\prop{m}}}}
           {\object{m}}
           {\e{f}}
          \item
  \e{} = {\extendmultiexp {\e{m}} {\e{v}} {\e{f}}},
          \item
  \e{f} = {\abs {\x{}} {\ty{1}} {\e{b}}},
\item
  \issubtypein{}{\MultiFntype {\ty{m}} {\ty{d}}}{\ty{}},
\item
  \inpropenv{\topprop{}}{\thenprop{\prop{}}},
\item
  \inpropenv{\botprop{}}{\elseprop{\prop{}}},
\item
  \issubobjin{}{\emptyobject{}}{\object{}}
        \end{enumerate}

                                Part 1 and 2 hold for the same reasons as T-True, noting that the propositions
                                and object agree with T-Multi.

For part 3 we show
\judgementtwo{}{\multi {\val{d}} {\extenddisptable {\disptable{}}{\val{v}}{\val{f}}}}{\MultiFntype {\ty{m}} {\ty{d}}}
by noting \judgementtwo {} {\val{d}} {\ty{d}},
  \judgementtwo{}{\val{v}}{\Top{}}
  and
  \judgementtwo{}{\val{f}}{\ty{m}}, and since \disptable{} is in the correct form by the inductive
  hypothesis on {\e{m}} we can satisfy all premises of T-Multi, so we are done.


      \end{subcase}

  \end{itemize}
\end{case}

      \begin{case}[BE-DefMethod1]
        \opsem {\openv{}}
               {\e{m}}
               {\errorval{\val{}}},
        \opsem {\openv{}}
                  {\e{}}
                {\errorval{\val{}}}

                Trivially reduces to an error.

      \end{case}
      \begin{case}[BE-DefMethod2]
        \opsem {\openv{}}
         {\e{m}}
         {\multi {\val{d}} {\disptable{}}},
  \opsem {\openv{}}
         {\e{v}}
         {\errorval{\val{}}},
        \opsem {\openv{}}
                  {\e{}}
                {\errorval{\val{}}}

                Trivially reduces to an error.
      \end{case}
      \begin{case}[BE-DefMethod3]
        \opsem {\openv{}}
         {\e{m}}
         {\multi {\val{d}} {\disptable{}}},
  \opsem {\openv{}}
         {\e{v}}
         {\val{v}},
  \opsem {\openv{}}
         {\e{f}}
         {\errorval{\val{}}},
        \opsem {\openv{}}
                  {\e{}}
                {\errorval{\val{}}}

                Trivially reduces to an error.

      \end{case}

\begin{case}[B-BetaClosure]
  \ 

  \begin{itemize}
    \item
  \opsem{\openv{}}{\e{}}{\val{}},
    \item
  \opsem {\openv{}}
         {\e{1}}
         {\closure {\openv{c}} {\abs {\x{}} {\s{}} {\e{b}}}},
    \item
  \opsem {\openv{}}
         {\e{2}}
         {\val{2}},
    \item
  \opsem {\extendopenv {\openv{c}} {\x{}} {\val{2}}}
         {\e{b}}
         {\val{}}
     \end{itemize}


\begin{itemize}
  \item[]
\begin{subcase}[T-App]
  \ 

  \begin{itemize}
    \item
  \ep{} = {\appexp {\ep{1}} {\ep{2}}},
    \item
  \judgementrewrite {\propenv{}} {\ep{1}} {\ArrowOne {\x{}} {\s{}}
                                                       {\ty{f}}
                                                       {\filterset {\thenprop {\prop{f}}}
                                                                   {\elseprop {\prop{f}}}}
                                                       {\object{f}}}
                {\filterset {\thenprop {\prop{1}}}
                            {\elseprop {\prop{1}}}}
                {\object{1}}
                {\e{1}},
    \item
  \judgementrewrite {\propenv{}}
                 {\ep{2}} {\s{}}
                 {\filterset {\thenprop {\prop{2}}}
                             {\elseprop {\prop{2}}}}
                 {\object{2}}
                 {\e{2}},
    \item
  \e{} = {\appexp {\e{1}} {\e{2}}},
    \item
      \issubtypein{}  {\replacefor {\ty{f}} {\object{2}} {\x{}}}{\ty{}},
    \item
      \inpropenv{\replacefor {\thenprop {\prop{f}}} {\object{2}} {\x{}}} {\thenprop {\prop{}}},
    \item
      \inpropenv{\replacefor {\elseprop {\prop{f}}} {\object{2}} {\x{}}} {\elseprop {\prop{}}},
    \item
      \issubobjin{}{\replacefor {\object{f}} {\object{2}} {\x{}}} {\object{}}
  \end{itemize}

         By inversion on \e{1} from T-Clos
         there is some environment {\propenvc{}} such that
         \begin{itemize}
           \item
              \satisfies{\openv{c}}{\propenvc{}} and
            \item
  \judgement {\propenvc{}} {\abs {\x{}} {\s{}} {\e{b}}} {\ArrowOne {\x{}} {\s{}}
                                                       {\ty{f}}
                                                       {\filterset {\thenprop {\prop{f}}}
                                                                   {\elseprop {\prop{f}}}}
                                                       {\object{f}}}
                {\filterset {\thenprop {\prop{1}}}
                            {\elseprop {\prop{1}}}}
                {\object{1}},
         \end{itemize}
         and also by inversion on \e{1} from T-Abs
         \begin{itemize}
           \item
  { \judgementrewrite {\propenvc{}, {\isprop {\s{}} {\x{}}}}
              {\ep{b}} {\ty{f}}
               {\filterset {\thenprop {\prop{f}}}
                           {\elseprop {\prop{f}}}}
               {\object{f}}
               {\e{b}}}.
         \end{itemize}

          From 
          \begin{itemize}
            \item
              \satisfies{\openv{c}}{\propenvc{}},
            \item
  \judgementrewrite {\propenv{}}
                 {\ep{2}} {\s{}}
                 {\filterset {\thenprop {\prop{2}}}
                             {\elseprop {\prop{2}}}}
                 {\object{2}}
                 {\e{2}} and 
            \item
  \opsem {\openv{}}
         {\e{2}}
         {\val{2}},
     \end{itemize}
              we know (by substitution)
              \satisfies{\extendopenv {\openv{c}} {\x{}} {\val{2}}}{\propenvc{},{\isprop{\s{}}{\x{}}}}.

              We want to prove
        \judgementrewrite {\propenvc{}}
                          {\replacefor{\ep{b}}{\val{2}}{\x{}}}
                          {\replacefor{\ty{f}}{\object{2}}{\x{}}}
               {\replacefor{\filterset {\thenprop {\prop{f}}}
                                       {\elseprop {\prop{f}}}}{\object{2}}{\x{}}}
                          {\replacefor{\object{f}}{\object{2}}{\x{}}}
                          {\replacefor{\e{b}}{\val{2}}{\x{}}}, 
                          which can be justified by noting 
          \begin{itemize}
            \item
  \judgementtworewrite {\propenvc{},{\isprop{\s{}}{\x{}}}}{\ep{b}}{\ty{f}}{\e{b}},
            \item
  \judgementrewrite {\propenv{}}
                 {\ep{2}} {\s{}}
                 {\filterset {\thenprop {\prop{2}}}
                             {\elseprop {\prop{2}}}}
                 {\object{2}}
                 {\e{2}} and 
            \item
  \opsem {\openv{}}
         {\e{2}}
         {\val{2}}.
     \end{itemize}

     From the previous fact and \satisfies{\openv{c}}{\propenvc{}},
              we know
  \opsem {\openv{c}}
         {\replacefor{\e{b}}{\val{2}}{\x{}}}
         {\val{}}.

                    Noting that 
      \issubtypein{}  {\replacefor {\ty{f}} {\object{2}} {\x{}}}{\ty{}},
      \inpropenv{\replacefor {\thenprop {\prop{f}}} {\object{2}} {\x{}}} {\thenprop {\prop{}}},
      \inpropenv{\replacefor {\elseprop {\prop{f}}} {\object{2}} {\x{}}} {\elseprop {\prop{}}}
      and
      \issubobjin{}{\replacefor {\object{f}} {\object{2}} {\x{}}} {\object{}},
                    we can use
         \begin{itemize}
           \item
        \judgementrewrite {\propenvc{}}
                          {\replacefor{\ep{b}}{\val{2}}{\x{}}}
                          {\replacefor{\ty{f}}{\object{2}}{\x{}}}
               {\replacefor{\filterset {\thenprop {\prop{f}}}
                                       {\elseprop {\prop{f}}}}{\object{2}}{\x{}}}
                          {\replacefor{\object{f}}{\object{2}}{\x{}}}
                          {\replacefor{\e{b}}{\val{2}}{\x{}}}, 
           \item
              \satisfies{\openv{c}}{\propenvc{}},
           \item
\isconsistent{\openv{c}} (via induction hypothesis on {\ep{1}}), and
           \item 
  \opsem {\openv{c}}
         {\replacefor{\e{b}}{\val{2}}{\x{}}}
         {\val{}}.
         \end{itemize}
         to apply the induction hypothesis on {\replacefor{\ep{b}}{\val{2}}{\x{}}} and satisfy
         all conditions.

\end{subcase}
\end{itemize}
\end{case}

\begin{case}[B-Delta]
  \opsem {\openv{}} {\e{1}} {\const{}},
  \opsem {\openv{}} {\e{2}} {\val{2}},
  \constantopsem{\const{}}{\val{2}} = \val{}

\begin{itemize}
  \item[]
\begin{subcase}[T-App]
  \ 

  \begin{itemize}
    \item
  \ep{} = {\appexp {\ep{1}} {\ep{2}}},
    \item
  \judgementrewrite {\propenv{}} {\ep{1}} {\ArrowOne {\x{}} {\s{}}
                                                       {\ty{f}}
                                                       {\filterset {\thenprop {\prop{f}}}
                                                                   {\elseprop {\prop{f}}}}
                                                       {\object{f}}}
                {\filterset {\thenprop {\prop{1}}}
                            {\elseprop {\prop{1}}}}
                {\object{1}}
                {\e{1}},
    \item
  \judgementrewrite {\propenv{}}
                 {\ep{2}} {\s{}}
                 {\filterset {\thenprop {\prop{2}}}
                             {\elseprop {\prop{2}}}}
                 {\object{2}}
                 {\e{2}},
    \item
  \e{} = {\appexp {\e{1}} {\e{2}}},
    \item
      \issubtypein{}  {\replacefor {\ty{f}} {\object{2}} {\x{}}}{\ty{}},
    \item
      \inpropenv{\replacefor {\thenprop {\prop{f}}} {\object{2}} {\x{}}} {\thenprop {\prop{}}},
    \item
      \inpropenv{\replacefor {\elseprop {\prop{f}}} {\object{2}} {\x{}}} {\elseprop {\prop{}}},
    \item
      \issubobjin{}{\replacefor {\object{f}} {\object{2}} {\x{}}} {\object{}}
  \end{itemize}

  % TODO do I need to prove anything about the argument in the definition
  % of the constant being under \s{}?

  Prove by cases on \const{}.
  \begin{itemize}
    \item[] \begin{subcase}[\const{} = \classconst]
        \issubtypein{}
  {\ArrowOne {\x{}} {\Top{}}
                                      {\Union{\Nil}{\Class}}
                                      {\filterset {\topprop{}}
                                                  {\topprop{}}}
                                      {\pth {\classpe{}} {\x{}}}}
    {\ArrowOne {\x{}} {\s{}}
                                                       {\ty{f}}
                                                       {\filterset {\thenprop {\prop{f}}}
                                                                   {\elseprop {\prop{f}}}}
                                                       {\object{f}}}

    Prove by cases on \val{2}.

        \begin{itemize}
          \item[] \begin{subcase}[\val{2} = \classvalue{\class{}} {\protect\overrightarrow {\classfieldpair{\fld{i}} {\val{i}}}}]
                    \val{} = \class{}

                    To prove part 1,
                    note
                    \issubobjin{}{\replacefor {\object{f}} {\object{2}} {\x{}}} {\object{}},
                    and \issubobjin{}{\pth {\classpe{}} {\x{}}}{\object{f}}.
                    Then either \object{} = \emptyobject{} and we are done,
                    or \object{} = {\pth {\classpe{}}{\object{2}}} and
                    by the induction hypothesis on \e{2} we know \inopenv {\openv{}} {\object{2}} {\val{2}}
                    and by the definition of path translation we know
                    {\openv{}}({\pth {\classpe{}} {\object{2}}}) = {\appexp {\classconst{}} {{\openv{}}(\object{2})}},
                    which evaluates to \val{}.

                    Part 2 is trivial since both propositions can only be \topprop{}.
                    
                    Part 3 holds because 
                    \val{} = \class{},
                    \issubtypein{}{\Union{\Nil}{\Class}}{\replacefor {\ty{f}} {\object{2}} {\x{}}}
                    and
                    \issubtypein{}{\replacefor {\ty{f}} {\object{2}} {\x{}}}{\ty{}},
                    so
                    {\judgementtwo{}{\val{}}{\ty{}}}
                    since
                    {\judgementtwo{}{\class{}}{\Union{\Nil}{\Class}}}.
                  \end{subcase}
          \item[] \begin{subcase}[\val{2} = \class{}] \val{} = \Class{}

              As above.
                  \end{subcase}
          \item[] \begin{subcase}[\val{2} = \true{}] \val{} = \Boolean{}

              As above.
                  \end{subcase}
          \item[] \begin{subcase}[\val{2} = \false{}] \val{} = \Boolean{}


              As above.
                  \end{subcase}
          \item[] \begin{subcase}[\val{2} = {\closure {\openv{}} {\abs {\x{}} {\ty{}} {\e{}}}}] \val{} = \IFn{}


              As above.
                  \end{subcase}
          \item[] \begin{subcase}[\val{2} = {\multi {\val{d}} {\disptable{}}}] \val{} = \HMapInstance{}


              As above.
                  \end{subcase}
          \item[] \begin{subcase}[\val{2} = {\protect\curlymapvaloverright{\val{1}}{\val{2}}}] \val{} = \Keyword{}


              As above.
                  \end{subcase}
          \item[] \begin{subcase}[\val{2} = {\nil{}}] \val{} = \nil{}

             Parts 1 and 2 as above.
                    Part 3 holds because \val{} = \nil{}
                    and {\judgementtwo{}{\nil{}}{\Union{\Nil}{\Class}}}.
                  \end{subcase}
        \end{itemize}
      \end{subcase}
    %\item[]
    %  \begin{subcase}[\const{} = \throwconst]
    %    {\ArrowOne {\x{}} {\s{}}
    %                                                   {\ty{f}}
    %                                                   {\filterset {\thenprop {\prop{f}}}
    %                                                               {\elseprop {\prop{f}}}}
    %                                                   {\object{f}}}
    %                                                   =
    %    {\ArrowOne {\x{}} {\Top{}}
    %                                  {\Bot{}}
    %                                  {\filterset {\botprop{}}
    %                                              {\botprop{}}}
    %                                  {\emptyobject{}}}

    %                                  Part 1 is trivial since \object{} = \emptyobject{} after substitution.
    %                                  Part 2 holds vacuously as both propositions are \botprop{} after substitution.
    %                                  Finally part 3 holds since {\judgementtwo{}{\hastype{\errorval{\val{2}}}{\Bot{}}}}.

    %  \end{subcase}
  \end{itemize}

\end{subcase}
\end{itemize}
\end{case}

\begin{case}[B-BetaMulti]
  \ 

  \begin{itemize}
    \item
  \opsem {\openv{}}
         {\e{1}}
         {\multi {\val{d}} {\disptable{}}},
    \item
  \opsem {\openv{}}
         {\e{2}}
         {\val{2}},
    \item
  \opsem {\openv{}}
         {\appexp {\val{d}} {\val{2}}}
         {\val{e}},
    \item
  \getmethod {\disptable{}}
             {\val{e}}
             {\val{l}}
             {\val{g}},
    \item
  \opsem {\openv{}}
         {\appexp {\val{g}} {\val{2}}}
         {\val{}},
       \item {\disptable{}} = {\curlymapvaloverright{\val{k}}{\val{v}}}
     \end{itemize}
     \begin{itemize}
       \item[]
\begin{subcase}[T-App]
  \ 

  \begin{itemize}
    \item
  \ep{} = {\appexp {\ep{1}} {\ep{2}}},
    \item
  \judgementrewrite {\propenv{}} {\ep{1}} {\ArrowOne {\x{}} {\s{}}
                                                       {\ty{f}}
                                                       {\filterset {\thenprop {\prop{f}}}
                                                                   {\elseprop {\prop{f}}}}
                                                       {\object{f}}}
                {\filterset {\thenprop {\prop{1}}}
                            {\elseprop {\prop{1}}}}
                {\object{1}}
                {\e{1}},
    \item
  \judgementrewrite {\propenv{}}
                 {\ep{2}} {\s{}}
                 {\filterset {\thenprop {\prop{2}}}
                             {\elseprop {\prop{2}}}}
                 {\object{2}}
                 {\e{2}},
    \item
  \e{} = {\appexp {\e{1}} {\e{2}}},
    \item
      \issubtypein{}  {\replacefor {\ty{f}} {\object{2}} {\x{}}}{\ty{}},
    \item
      \inpropenv{\replacefor {\thenprop {\prop{f}}} {\object{2}} {\x{}}} {\thenprop {\prop{}}},
    \item
      \inpropenv{\replacefor {\elseprop {\prop{f}}} {\object{2}} {\x{}}} {\elseprop {\prop{}}},
    \item
      \issubobjin{}{\replacefor {\object{f}} {\object{2}} {\x{}}} {\object{}},
  \end{itemize}

     By inversion on \e{1} via T-Multi we know 
     \begin{itemize}
       \item
         \judgementrewrite{\propenv{}}{\ep{1}}{\MultiFntype{\s{t}}{\s{d}}}
                {\filterset {\thenprop {\prop{1}}}
                            {\elseprop {\prop{1}}}}
                {\object{1}}{\e{1}},
           
         \item \s{t} = {\ArrowOne {\x{}} {\s{}}
                                                       {\ty{f}}
                                                       {\filterset {\thenprop {\prop{f}}}
                                                                   {\elseprop {\prop{f}}}}
                                                       {\object{f}}},
         \item \s{d} = {\ArrowOne {\x{}} {\s{}}
                                                       {\ty{d}}
                                                       {\filterset {\thenprop {\prop{d}}}
                                                                   {\elseprop {\prop{d}}}}
                                                       {\object{d}}},
       \item
         \judgementtwo{}{\val{d}}{\s{d}}
              \item
  $\overrightarrow{\judgementtwo{}{\val{k}}{\Top{}}}$, and 
\item
  $\overrightarrow{\judgementtwo{}{\val{v}}{\s{t}}}$.
  \end{itemize}

  % FIXME do we really know this? seems obvious but the IH says something subtly different, might need
  % a lemma to bridge this. Same problem in T-IsA case.
  By the inductive hypothesis on 
  \opsem {\openv{}}
         {\e{2}}
         {\val{2}}
  we know 
  \judgementrewrite {\propenv{}} {\val{2}} {\s{}}
             {\filterset {\thenprop {\prop{2}}}
                         {\elseprop {\prop{2}}}}
                       {\object{2}}
                       {\val{2}}.

We then consider applying the evaluated argument to the dispatch function:
  \opsem {\openv{}}
         {\appexp {\val{d}} {\val{2}}}
         {\val{e}}.

         Since we can satisfy T-App with
       \begin{itemize}
         \item
         \judgementtwo{}{\val{d}}{\ArrowOne {\x{}} {\s{}}
                                                       {\ty{d}}
                                                       {\filterset {\thenprop {\prop{d}}}
                                                                   {\elseprop {\prop{d}}}}
                                                       {\object{d}}}, and
         \item
  \judgementrewrite {\propenv{}} {\val{2}} {\s{}}
             {\filterset {\thenprop {\prop{2}}}
                         {\elseprop {\prop{2}}}}
                       {\object{2}}
                       {\val{2}},
       \end{itemize}
       we can apply the inductive hypothesis
       to derive
  \judgementrewrite {\propenv{}} {\val{e}} 
  {\replacefor{\ty{d}}
              {\object{2}}
              {\x{}}}
             {\replacefor{\filterset {\thenprop {\prop{d}}}
                                     {\elseprop {\prop{d}}}}
                         {\object{2}}
                         {\x{}}}
                       {\replacefor
                         {\object{d}}
                         {\object{2}}
                         {\x{}}}
                       {\val{e}}.

 Now we consider how we choose which method to dispatch to.

 As 
  \getmethod {\disptable{}}
             {\val{e}}
             {\val{l}}
             {\val{g}}, by inversion on \getmethodliteral
             we know
   there exists exactly one \val{k} such that 
   \entryinmap{\mapvalentry{\val{k}}{\val{g}}}{\disptable{}} and \isaopsem{\val{e}}{\val{k}} = {\true{}}.

   By inversion we know T-DefMethod must have extended \disptable{} 
   with the well-typed dispatch value \val{k},
   thus {\judgementtwo{}{\val{k}}{\ty{k}}}, and
   the well-typed method \val{g},
   so {\judgementtwo{}{\val{g}}{\s{t}}}.

  We can also prove that given
        \begin{itemize}
          \item
  \judgementrewrite {\propenv{}} {\val{e}} 
  {\replacefor{\ty{d}}
              {\object{2}}
              {\x{}}}
             {\replacefor{\filterset {\thenprop {\prop{d}}}
                                     {\elseprop {\prop{d}}}}
                         {\object{2}}
                         {\x{}}}
                       {\replacefor
                         {\object{d}}
                         {\object{2}}
                         {\x{}}}
                       {\val{e}}.
    \item
  \judgementtwo {\propenv{}} {\val{k}} {\ty{k}},
                     \item
        \isaopsem{\val{e}}{\val{k}} = {\true{}}, 
      \item
        \satisfies{\openv{}}{\propenv{}},
    \item
  \isacompare{\ty{d}}
                       {\replacefor
                         {\object{d}}
                         {\object{2}}
                         {\x{}}}
  {\ty{k}}{\filterset {\thenprop {\propp{}}} {\elseprop {\propp{}}}},
          \item
        \inpropenv{\thenprop{\propp{}}}{\thenprop{\propp{}}}, and
    \item
        \inpropenv{\elseprop{\propp{}}}{\elseprop{\propp{}}}.
        \end{itemize}
  we can apply \lemref{appendix:lemma:isa} to derive
  then {\satisfies{\openv{}}{\thenprop{\propp{}}}}.

   Now we consider applying the evaluated argument to the chosen method:
  \opsem {\openv{}}
         {\appexp {\val{g}} {\val{2}}}
         {\val{}}.

  By inversion via B-DefMethod we can assume {\val{g}} = {\abs{\x{}}{\s{}}{\e{b}}}, 
  ie. that we have chosen a method to dispatch to that is a closure.

  Because 
  \opsem {\openv{}}
         {\appexp {\val{g}} {\val{2}}}
         {\val{}}
         and
    \judgementtwo{\propenv{}}{\val{2}}{\s{}},
  by inversion via B-BetaClosure we know {\val{}} = {\replacefor{\e{b}}{\val{2}}{\x{}}}.

  With the following premises:
\begin{itemize}
  \item
{\judgementrewrite{{\propenv{}},{\thenprop{\propp{}}}}
                  {\replacefor{\ep{b}}{\val{2}}{\x{}}}
                  {\replacefor{\ty{f}} {\object{2}}{\x{}}}
                  {\replacefor
                   {\filterset {\thenprop {\prop{f}}}
                               {\elseprop {\prop{f}}}}
                             {\object{2}}
                             {\x{}}}
                  {\replacefor
                          {\object{f}}
                             {\object{2}}
                             {\x{}}}
                  {\replacefor{\e{b}}{\val{2}}{\x{}}}
                        },
    \begin{itemize}
      \item From
{\judgementrewrite{{\propenv{}},{\isprop{\s{}}{\x{}}}}
                  {\e{b}}
                  {\ty{f}}
               {\filterset {\thenprop {\prop{f}}}
                                       {\elseprop {\prop{f}}}}
                          {\object{f}}
                          {\e{b}}}
          via the inductive hypothesis on 
  \opsem {\openv{}}
         {\appexp {\abs{\x{}}{\s{}}{\e{b}}} {\val{2}}}
         {\val{}},
      \item then we can derive
{\judgementrewrite{{\propenv{}}}
                  {\replacefor{\ep{b}}{\val{2}}{\x{}}}
                  {\replacefor{\ty{f}} {\object{2}}{\x{}}}
                  {\replacefor
                   {\filterset {\thenprop {\prop{f}}}
                               {\elseprop {\prop{f}}}}
                             {\object{2}}
                             {\x{}}}
                  {\replacefor
                          {\object{f}}
                             {\object{2}}
                             {\x{}}}
                  {\replacefor{\e{b}}{\val{2}}{\x{}}}
                        } via substitution and the fact that {\x{}} is fresh 
                        therefore \x{} $\not\in$ \fv{\propenv{}} so we do not need to substitution for \x{} in \propenv{}. %TODO lemma for this
                        
      \item 
        \satisfies{\openv{}}{\propenv{}, {\thenprop{\propp{}}}}
        because
        \satisfies{\openv{}}{\propenv{}} and {\satisfies{\openv{}}{\thenprop{\propp{}}}} via M-And.
    \end{itemize}
  \item
              \satisfies{\openv{}}{{\propenv{}},{\thenprop{\propp{}}}},
    \begin{itemize}
      \item From \satisfies{\openv{}}{\propenv{}} and \item {\satisfies{\openv{}}{\thenprop{\propp{}}}}  via M-And.
    \end{itemize}
           \item
\isconsistent{\openv{}}, and
           \item 
  \opsem {\openv{}}
         {\replacefor{\e{b}}{\val{2}}{\x{}}}
         {\val{}}.
\end{itemize}

we can apply the inductive hypothesis to satisfy our overall goal for this subcase.
\end{subcase}
     \end{itemize}
\end{case}

\begin{case}[BE-Beta1]
  \ 

  Reduces to an error.
\end{case}
\begin{case}[BE-Beta2]
  \ 

  Reduces to an error.
\end{case}
\begin{case}[BE-BetaClosure]
  \ 

  Reduces to an error.
\end{case}
\begin{case}[BE-BetaMulti1]
  \ 

  Reduces to an error.
\end{case}
\begin{case}[BE-BetaMulti2]
  \ 

  Reduces to an error.
\end{case}
\begin{case}[BE-Delta]
  \ 

  Reduces to an error.
\end{case}

\begin{case}[B-IsA]
        \opsem {\openv{}} {\e{1}} {\val{1}},
        \opsem {\openv{}} {\e{2}} {\val{2}},
        \isaopsem{\val{1}}{\val{2}} = {\val{}}


  \begin{itemize}
    \item[]
      \begin{subcase}[T-IsA]
  \ep{} = {\isaapp {\ep{1}} {\ep{2}}},
  \judgementrewrite {\propenv{}} {\ep{1}} {\ty{1}}
             {\filterset {\thenprop {\prop{1}}}
                         {\elseprop {\prop{1}}}}
                       {\object{1}}
                       {\e{1}},
  \judgementrewrite {\propenv{}} {\ep{2}} {\ty{2}}
             {\filterset {\thenprop {\prop{2}}}
                         {\elseprop {\prop{2}}}}
                       {\object{2}}
                       {\e{2}},
  \e{} = {\isaapp {\e{1}} {\e{2}}},
  \issubtypein{}{\Boolean}{\ty{}},
  \isacompare{\ty{1}}{\object{1}}{\ty{2}}{\filterset {\thenprop {\propp{}}} {\elseprop {\propp{}}}},
  \inpropenv{\thenprop {\propp{}}}{\thenprop {\prop{}}},
  \inpropenv{\elseprop {\propp{}}}{\elseprop {\prop{}}},
  \issubobjin{}{\emptyobject{}}{\object{}}

        Part 1 holds trivially with \object{} = \emptyobject{}.

        For part 2, by the induction hypothesis on \e{1} and \e{2}
        we know
  \judgementrewrite {\propenv{}} {\val{1}} {\ty{1}}
             {\filterset {\thenprop {\prop{1}}}
                         {\elseprop {\prop{1}}}}
                       {\object{1}}
                       {\val{1}} and
  \judgementrewrite {\propenv{}} {\val{2}} {\ty{2}}
             {\filterset {\thenprop {\prop{2}}}
                         {\elseprop {\prop{2}}}}
                       {\object{2}}
                       {\val{2}},
                       so we can then apply
        \lemref{appendix:lemma:isa}
        to reach our goal.

        Part 3 holds because by the definition of \isaopsemliteral
        \val{} can only be \true\ or \false, 
        and since \judgementtwo{\propenv{}}{\true}{\ty{}}
        and
        \judgementtwo{\propenv{}}{\false}{\ty{}}
        we are done.
      \end{subcase}
  \end{itemize}
\end{case}

      \begin{case}[BE-IsA1]
        \opsem {\openv{}} {\e{1}} {\errorvalv{}}

        Trivially reduces to an error.
      \end{case}
      \begin{case}[BE-IsA2]
       \opsem {\openv{}} {\e{1}} {\val{1}},
       \opsem {\openv{}} {\e{2}} {\errorvalv{}}

        Trivially reduces to an error.
      \end{case}

\begin{case}[B-Get]
      $\opsem {\openv{}} {\e{m}}{\val{m}}$,
        $\val{m} = {\curlymap{\overrightarrow{({\val{a}}\ {\val{b}})}}}$,
         \opsem {\openv{}} {\e{k}} {\kw{}},
         $\keyinmap{\kw{}}{\curlymap{\overrightarrow{({\val{a}}\ {\val{b}})}}}$,
         \getmap{\curlymap{\overrightarrow{({\val{a}}\ {\val{b}})}}} {\kw{}} = {\val{}}

  \begin{itemize}
    \item[]
      \begin{subcase}[T-GetHMap]
  \ep{} = {\getexp {\ep{m}} {\ep{k}}},
  \judgementrewrite {\propenv{}} {\ep{m}} {\Unionsplice {\overrightarrow {\HMapgeneric {\mandatory{}} {\absent{}}}}}
           {\filterset {\thenprop {\prop{m}}} {\elseprop {\prop{m}}}}
           {\object{m}}
           {\e{m}},
  \judgementtworewrite {\propenv{}} {\ep{k}} {\Value {k}}{\e{k}},
  \overr{\inmandatory{\kw{}}{\ty{i}}{\mandatory{}},}
  \e{} = {\getexp {\e{m}} {\e{k}}},
  \issubtypein{}{\Unionsplice {\overrightarrow {\ty{i}}}}{\ty{}} ,
  \thenprop{\prop{}} = {\topprop{}},
  \elseprop{\prop{}} = {\topprop{}},
  \issubobjin{}{\replacefor {\pth {\keype{k}} {\x{}}}
                          {\object{m}}
                          {\x{}}}
                        {\object{}}


         To prove part 1 we consider two cases on the form of \object{m}: 
         \begin{itemize}
           \item
         if {\object{m}} = \emptyobject{}
         then \object{} = \emptyobject{} by substitution, which gives the desired result;
           \item
         if \object{m} = {\pth {\pathelem{m}} {\x{m}}}
         then \issubobjin{}{\pth {\keype{k}} {\object{m}}}{\object{}} by substitution.
         We note by the definition of path translation
         {\openv{}}({\pth {\keype{k}} {\object{m}}}) =
         {\getexp {{\openv{}}(\object{m})}{\kw{}}}
         and by the induction hypothesis on \e{m}
         {{\openv{}}(\object{m})} = {\curlymap{\overrightarrow{({\val{a}}\ {\val{b}})}}},
         which together imply 
         \inopenv {\openv{}} {\object{}} {\getexp {\curlymap{\overrightarrow{({\val{a}}\ {\val{b}})}}} {\kw{}}}.
         Since this is the same form as B-Get, we can apply the premise
         \getmap{\curlymap{\overrightarrow{({\val{a}}\ {\val{b}})}}} {\kw{}} = {\val{}}
         to derive \inopenv {\openv{}} {\object{}} {\val{}}.
         \end{itemize}
         
         Part 2 holds trivially as \thenprop{\prop{}} = {\topprop{}}
         and \elseprop{\prop{}} = {\topprop{}}.

         To prove part 3 we note that (by the induction hypothesis on \e{m})
         $\judgementtwo{}{\val{m}}{\Unionsplice{\overrightarrow {\HMapgeneric {\mandatory{}} {\absent{}}}}}$,
         where $\overrightarrow{\inmandatory{\kw{}}{\ty{i}}{\mandatory{}}}$, and 
         both
         $\keyinmap{\kw{}}{\curlymap{\overrightarrow{({\val{a}}\ {\val{b}})}}}$
         and
         \getmap{\curlymap{\overrightarrow{({\val{a}}\ {\val{b}})}}} {\kw{}} = {\val{}}
         imply \judgementtwo{}{\val{}}{\Unionsplice {\overrightarrow {\ty{i}}}}.

      \end{subcase}
    \item[]
      \begin{subcase}[T-GetHMapAbsent]
  \ep{} = {\getexp {\ep{m}} {\ep{k}}},
  \judgementtworewrite {\propenv{}} {\ep{k}} {\Value {k}} {\e{k}},
  \\
  \judgementrewrite {\propenv{}} {\ep{m}} {\HMapgeneric {\mandatory{}} {\absent}}
           {\filterset {\thenprop {\prop{m}}} {\elseprop {\prop{m}}}}
           {\object{m}}
           {\e{m}},
  {\inabsent{\kw{}}{\absent{}}},
  \e{} = {\getexp {\e{m}} {\e{k}}},
  \issubtypein{}{\Nil}{\ty{}},
  \thenprop{\prop{}} = {\topprop{}},
  \elseprop{\prop{}} = {\topprop{}},
  \issubobjin{}{\replacefor
               {\pth {\keype{k}} {\x{}}}
                          {\object{m}}
                          {\x{}}}
                        {\object{}}

       Unreachable subcase because 
         $\keyinmap{\kw{}}{\curlymap{\overrightarrow{({\val{a}}\ {\val{b}})}}}$,
         contradicts
                {\inabsent{\kw{}}{\absent{}}}.
      \end{subcase}
    \item[]
      \begin{subcase}[T-GetHMapPartialDefault]
  \ep{} = {\getexp {\ep{m}} {\ep{k}}},
  \judgementtworewrite {\propenv{}} {\ep{k}} {\Value {k}}{\e{k}},
  \\
 \judgementrewrite {\propenv{}} {\ep{m}} {\HMapp {\mandatory{}} {\absent}}
           {\filterset {\thenprop {\prop{m}}} {\elseprop {\prop{m}}}}
           {\object{m}}
           {\e{m}},
             {\notinmandatory{\kw{}}{\ty{}}{\mandatory{}}},
             {\notinabsent{\kw{}}{\absent{}}},
  \e{} = {\getexp {\e{m}} {\e{k}}},
  \ty{} = \Top,
  \thenprop{\prop{}} = {\topprop{}},
  \elseprop{\prop{}} = {\topprop{}},
  \issubobjin{}{\replacefor
               {\pth {\keype{k}} {\x{}}}
                          {\object{m}}
                          {\x{}}}{\object{}}

         Parts 1 and 2 are the same as the B-Get subcase.
         Part 3 is trivial as \ty{} = \Top.
      \end{subcase}
  \end{itemize}
\end{case}

\begin{case}[B-GetMissing]
        \val{} = \nil,
        $\opsem {\openv{}}
        {\e{m}} {\curlymap{\overrightarrow{({\val{a}}\ {\val{b}})}}}$,
       \opsem {\openv{}} {\e{k}} {\kw{}},
       \keynotinmap{\kw{}}{\curlymap{\overrightarrow{({\val{a}}\ {\val{b}})}}}

  \begin{itemize}
    \item[]
      \begin{subcase}[T-GetHMap]
  \ep{} = {\getexp {\ep{m}} {\ep{k}}},
  \judgementrewrite {\propenv{}} {\ep{m}} {\Unionsplice {\overrightarrow {\HMapgeneric {\mandatory{}} {\absent{}}}}}
           {\filterset {\thenprop {\prop{m}}} {\elseprop {\prop{m}}}}
           {\object{m}}
           {\e{m}},
  \judgementtworewrite {\propenv{}} {\ep{k}} {\Value {k}}{\e{k}},
  \overr{\inmandatory{\kw{}}{\ty{i}}{\mandatory{}},}
  \e{} = {\getexp {\e{m}} {\e{k}}},
  \issubtypein{}{\Unionsplice {\overrightarrow {\ty{i}}}}{\ty{}},
  \thenprop{\prop{}} = {\topprop{}},
  \elseprop{\prop{}} = {\topprop{}},
  \issubobjin{}{\replacefor {\pth {\keype{k}} {\x{}}}
                          {\object{m}}
                          {\x{}}}{\object{}}

       Unreachable subcase because 
       \keynotinmap{\kw{}}{\curlymap{\overrightarrow{({\val{a}}\ {\val{b}})}}}
       contradicts ${\inmandatory{\kw{}}{\ty{}}{\mandatory{}}}$.
      \end{subcase}
    \item[]
      \begin{subcase}[T-GetHMapAbsent]
  \ep{} = {\getexp {\ep{m}} {\ep{k}}},
  \judgementtworewrite {\propenv{}} {\ep{k}} {\Value {k}} {\e{k}},
  \\
  \judgementrewrite {\propenv{}} {\ep{m}} {\HMapgeneric {\mandatory{}} {\absent}}
           {\filterset {\thenprop {\prop{m}}} {\elseprop {\prop{m}}}}
           {\object{m}}
           {\e{m}},
  {\inabsent{\kw{}}{\absent{}}},
  \e{} = {\getexp {\e{m}} {\e{k}}},
  \issubtypein{}{\Nil}{\ty{}},
  \thenprop{\prop{}} = {\topprop{}},
  \elseprop{\prop{}} = {\topprop{}},
  \issubobjin{}{\replacefor
               {\pth {\keype{k}} {\x{}}}
                          {\object{m}}
                          {\x{}}}{\object{}}

         To prove part 1 we consider two cases on the form of \object{m}: 
         \begin{itemize}
           \item
         if {\object{m}} = \emptyobject{}
         then \object{} = \emptyobject{} by substitution, which gives the desired result;
           \item
         if \object{m} = {\pth {\pathelem{m}} {\x{m}}}
         then \issubobjin{}{\pth {\keype{k}} {\object{m}}}{\object{}} by substitution.
         We note by the definition of path translation
         {\openv{}}({\pth {\keype{k}} {\object{m}}}) =
         {\getexp {{\openv{}}(\object{m})}{\kw{}}}
         and by the induction hypothesis on \e{m}
         {{\openv{}}(\object{m})} = {\curlymap{\overrightarrow{({\val{a}}\ {\val{b}})}}},
         which together imply 
         \inopenv {\openv{}} {\object{}} {\getexp {\curlymap{\overrightarrow{({\val{a}}\ {\val{b}})}}} {\kw{}}}.
         Since this is the same form as B-GetMissing, we can apply the premise
        \val{} = \nil\ 
         to derive \inopenv {\openv{}} {\object{}} {\val{}}.
         \end{itemize}
         
         Part 2 holds trivially as \thenprop{\prop{}} = {\topprop{}}
         and \elseprop{\prop{}} = {\topprop{}}.

         To prove part 3 we note that \e{m} has type {\HMapgeneric {\mandatory{}} {\absent{}}}
         where {\inabsent{\kw{}}{\absent{}}}, and
         the premises of B-GetMissing
         \keynotinmap{\kw{}}{\curlymap{\overrightarrow{({\val{a}}\ {\val{b}})}}}
         and
          \val{} = \nil\ 
         tell us {\val{}} must be of type {\ty{}}.
      \end{subcase}
    \item[]
      \begin{subcase}[T-GetHMapPartialDefault]
  \ep{} = {\getexp {\ep{m}} {\ep{k}}},
  \judgementtworewrite {\propenv{}} {\ep{k}} {\Value {k}}{\e{k}},
  \\
 \judgementrewrite {\propenv{}} {\ep{m}} {\HMapp {\mandatory{}} {\absent}}
           {\filterset {\thenprop {\prop{m}}} {\elseprop {\prop{m}}}}
           {\object{m}}
           {\e{m}},
             {\notinmandatory{\kw{}}{\ty{}}{\mandatory{}}},
             {\notinabsent{\kw{}}{\absent{}}},
  \e{} = {\getexp {\e{m}} {\e{k}}},
  \ty{} = \Top,
  \thenprop{\prop{}} = {\topprop{}},
  \elseprop{\prop{}} = {\topprop{}},
  \issubobjin{}{\replacefor
               {\pth {\keype{k}} {\x{}}}
                          {\object{m}}
                          {\x{}}}{\object{}}

         Parts 1 and 2 are the same as the B-GetMissing subcase of T-GetHMapAbsent.
         Part 3 is trivial, since \ty{} = \Top.
      \end{subcase}
  \end{itemize}
\end{case}

\begin{case}[BE-Get1]
  \ 

  Reduces to an error.
\end{case}

\begin{case}[BE-Get2]
  \ 

  Reduces to an error.
\end{case}

\begin{case}[B-Assoc]
        \val{} = 
        {\extendmap{\curlymap{\overrightarrow{({\val{a}}\ {\val{b}})}}}
                {\kw{}}{\val{v}}},
        \opsem {\openv{}}
        {\e{m}} {\curlymap{\overrightarrow{({\val{a}}\ {\val{b}})}}},
        \opsem {\openv{}} {\e{k}} {\kw{}},
        \opsem {\openv{}} {\e{v}} {\val{v}}

  \begin{itemize}
    \item[]
      \begin{subcase}[T-AssocHMap]
  \judgementtworewrite {\propenv{}} {\ep{m}} {\HMapgeneric {\mandatory{}} {\absent}} {\e{m}},
  \judgementtworewrite {\propenv{}} {\ep{k}} {\Value{\kw{}}}{\e{k}},\\
  \judgementtworewrite {\propenv{}} {\ep{v}} {\ty{}}{\e{v}},
  {\kw{}} $\not\in$ {\absent{}},
  \ep{} = {\assocexp {\ep{m}} {\ep{k}} {\ep{v}}},\\
  \e{} = {\assocexp {\e{m}} {\e{k}} {\e{v}}},
  \issubtypein{}{\HMapgeneric {\extendmandatoryset {\mandatory{}}{\kw{}}{\ty{}}} {\absent}}{\ty{}},
  \thenprop{\prop{}} = {\topprop{}},
  \elseprop{\prop{}} = {\botprop{}},
  \object{} = \emptyobject{}

        Parts 1 and 2 hold for the same reasons as T-True.
        %TODO part 3
      \end{subcase}
  \end{itemize}
\end{case}

\begin{case}[BE-Assoc1]
  \ 

  Reduces to an error.
\end{case}

\begin{case}[BE-Assoc2]
  \ 

  Reduces to an error.
\end{case}

\begin{case}[BE-Assoc3]
  \ 

  Reduces to an error.
\end{case}

\begin{case}[B-IfFalse]
        \opsem {\openv{}} {\e{1}} {\false}
        \ \ \text{or}\ \ 
        \opsem {\openv{}} {\e{1}} {\nil},
        \opsem {\openv{}} {\e{3}} {\val{}}

  \begin{itemize}
    \item[]
      \begin{subcase}[T-If]
        \ep{} = {\ifexp {\ep{1}} {\ep{2}} {\ep{3}}},
        \judgementrewrite {\propenv{}} {\ep{1}} {\ty{1}} {\filterset {\thenprop {\prop{1}}} {\elseprop {\prop{1}}}}
                 {\object{1}}
                 {\e{1}},
  \judgementrewrite {\propenv{}, {\thenprop {\prop{1}}}}
                 {\ep{2}} {\ty{}} {\filterset {\thenprop {\prop{2}}} {\elseprop {\prop{2}}}}
                 {\object{}}
                 {\e{2}},
  \judgementrewrite {\propenv{}, {\elseprop {\prop{1}}}}
                 {\ep{3}} {\ty{}} {\filterset {\thenprop {\prop{3}}} {\elseprop {\prop{3}}}}
                 {\object{}}
                 {\e{3}},
        \e{} = {\ifexp {\e{1}} {\e{2}} {\e{3}}},
  \inpropenv{\orprop {\thenprop {\prop{2}}} {\thenprop {\prop{3}}}}{\thenprop{\prop{}}},
  \inpropenv{\orprop {\elseprop {\prop{2}}} {\elseprop {\prop{3}}}}{\elseprop{\prop{}}}

              For part 1, either \object{} = \emptyobject{}, or \e{} evaluates to the
              result of \e{3}.

              To prove part 2, we consider two cases:
              \begin{itemize}
                \item if \isfalseval{\val{}}
                  then \e{3} evaluates to a false value so {\satisfies{\openv{}}{\elseprop {\prop{3}}}}, and thus
                  {\satisfies{\openv{}}{\orprop {\elseprop {\prop{2}}} {\elseprop {\prop{3}}}}} by M-Or, 
                \item otherwise
                  \istrueval{\val{}},
                  so \e{3} evaluates to a true value so {\satisfies{\openv{}}{\thenprop {\prop{3}}}}, and thus
                  {\satisfies{\openv{}}{\orprop {\thenprop {\prop{2}}} {\thenprop {\prop{3}}}}} by M-Or.
              \end{itemize}

              Part 3 is trivial as
              \opsem {\openv{}} {\e{3}} {\val{}}
              and {\judgementtwo{}{\val{}}{\ty{}}} by the induction hypothesis on {\e{3}}.
      \end{subcase}
  \end{itemize}
\end{case}

\begin{case}[B-IfTrue]
        \opsem {\openv{}} {\e{1}} {\val{1}},
              ${\val{1}} \not= {\false}$,
              ${\val{1}} \not= {\nil}$,
              \opsem {\openv{}} {\e{2}} {\val{}}

  \begin{itemize}
    \item[]
      \begin{subcase}[T-If]
        \ep{} = {\ifexp {\ep{1}} {\ep{2}} {\ep{3}}},
        \judgementrewrite {\propenv{}} {\ep{1}} {\ty{1}} {\filterset {\thenprop {\prop{1}}} {\elseprop {\prop{1}}}}
                 {\object{1}}
                 {\e{1}},
  \judgementrewrite {\propenv{}, {\thenprop {\prop{1}}}}
                 {\ep{2}} {\ty{}} {\filterset {\thenprop {\prop{2}}} {\elseprop {\prop{2}}}}
                 {\object{}}
                 {\e{2}},
  \judgementrewrite {\propenv{}, {\elseprop {\prop{1}}}}
                 {\ep{3}} {\ty{}} {\filterset {\thenprop {\prop{3}}} {\elseprop {\prop{3}}}}
                 {\object{}}
                 {\e{3}},
        \e{} = {\ifexp {\e{1}} {\e{2}} {\e{3}}},
  \inpropenv{\orprop {\thenprop {\prop{2}}} {\thenprop {\prop{3}}}}{\thenprop{\prop{}}},
  \inpropenv{\orprop {\elseprop {\prop{2}}} {\elseprop {\prop{3}}}}{\elseprop{\prop{}}}

              For part 1, either \object{} = \emptyobject{}, or \e{} evaluates to the
              result of \e{2}.

              To prove part 2, we consider two cases:
              \begin{itemize}
                \item if \isfalseval{\val{}}
                  then \e{2} evaluates to a false value so {\satisfies{\openv{}}{\elseprop {\prop{2}}}}, and thus
                  {\satisfies{\openv{}}{\orprop {\elseprop {\prop{2}}} {\elseprop {\prop{3}}}}} by M-Or, 
                \item otherwise
                  \istrueval{\val{}},
                  so \e{2} evaluates to a true value so {\satisfies{\openv{}}{\thenprop {\prop{2}}}}, and thus
                  {\satisfies{\openv{}}{\orprop {\thenprop {\prop{2}}} {\thenprop {\prop{3}}}}} by M-Or.
              \end{itemize}

              Part 3 is trivial as
              \opsem {\openv{}} {\e{2}} {\val{}}
              and {\judgementtwo{}{\val{}}{\ty{}}} by the induction hypothesis on {\e{2}}.

      \end{subcase}
  \end{itemize}
\end{case}

\begin{case}[BE-If]
  \ 

  Reduces to an error.
\end{case}

\begin{case}[BE-IfFalse]
  \ 

  Reduces to an error.
\end{case}

\begin{case}[BE-IfTrue]
  \ 

  Reduces to an error.
\end{case}

\begin{case}[B-Let]
  \e{} = {\letexp {\x{}} {\e{1}} {\e{2}}},
        \opsem {\openv{}} {\e{1}} {\val{1}},
        \opsem {\extendopenv{\openv{}}{\x{}}{\val{1}}} {\e{2}} {\val{}}


  \begin{itemize}
    \item[]
      \begin{subcase}[T-Let]
  \ep{} = {\letexp {\x{}} {\ep{1}} {\ep{2}}},
  \judgementrewrite {\propenv{}} {\ep{1}} {\s{}} {\filterset {\thenprop {\prop{1}}} {\elseprop {\prop{1}}}}
             {\object{1}}
             {\ep{1}},
             \propp{} = {\impprop {\notprop {\falsy{}} {\x{}}} {\thenprop {\prop{1}}}},
             \proppp{} = {\impprop {\isprop {\falsy{}} {\x{}}} {\elseprop {\prop{1}}}},
  \judgementrewrite
       {\propenv{}, {\isprop {\s{}} {\x{}}},
         {\propp{}},
         {\proppp{}}}
             {\ep{2}} {\ty{}} {\filterset {\thenprop {\prop{}}} {\elseprop {\prop{}}}}
             {\object{}} 
             {\e{2}}

        For all the following cases (with a reminder that \x{} is fresh)
        we apply the induction hypothesis on \e{2}. We justify this by noting
        that occurrences of \x{} inside \e{2} have the same type as \e{1} and 
        simulate the propositions of \e{1}
        because 
        \begin{itemize}
          \item
  \judgementrewrite
       {\propenv{}, {\isprop {\s{}} {\x{}}},
         {\propp{}},
         {\proppp{}}}
             {\ep{2}} {\ty{}} {\filterset {\thenprop {\prop{}}} {\elseprop {\prop{}}}}
             {\object{}} 
             {\e{2}},
           \item
        \satisfies{\extendopenv{\openv{}}{\x{}}{\val{1}}}{\propenv{}, {\isprop {\s{}} {\x{}}}, \propp{}, \proppp{}},
           \item
        {\isconsistent{\extendopenv{\openv{}}{\x{}}{\val{1}}}},
        and
           \item
        \opsem {\extendopenv{\openv{}}{\x{}}{\val{1}}} {\e{2}} {\val{}}.
    \end{itemize}

        We prove parts 1, 2 and 3 by directly using the induction hypothesis on \e{2}.
      \end{subcase}
  \end{itemize}
\end{case}

\begin{case}[BE-Let]
  \ 
  
  Reduces to an error.
\end{case}

\begin{case}[B-Abs] 
        \val{} = {\closure {\openv{}} {\abs {\x{}} {\s{}} {\e{1}}}}

  \begin{itemize}
    \item[]
      \begin{subcase}[T-Clos]
  \ep{} = {\closure {\openv{}} {\abs {\x{}} {\s{}} {\e{1}}}},
  $\exists {\propenvp{}}. \satisfies{\openv{}}{\propenvp{}}$
  \ \text{and}\ 
\judgementrewrite {\propenvp{}} {\abs {\x{}} {\s{}} {\e{1}}} {\ty{}}
                 {\filterset {\thenprop {\prop{f}}}
                             {\elseprop {\prop{f}}}}
                 {\object{f}}
                 {\abs {\x{}} {\s{}} {\e{1}}},
  \e{} = {\closure {\openv{}} {\abs {\x{}} {\s{}} {\e{1}}}},
                 {\thenprop{\prop{}}} = \topprop{},
                 {\elseprop{\prop{}}} = \botprop{},
                 {\object{}} = \emptyobject{}

        We assume some \propenvp{}, such that
        \begin{itemize}
          \item \satisfies{\openv{}}{\propenvp{}}
          \item \judgement {\propenvp{}} {\abs {\x{}} {\s{}} {\e{1}}} {\ty{}}
                           {\filterset {\thenprop {\prop{}}}
                                       {\elseprop {\prop{}}}}
                           {\object{}}.
       \end{itemize}
       Note the last rule in the derivation of
          \judgement {\propenvp{}} {\abs {\x{}} {\s{}} {\e{1}}} {\ty{}}
                           {\filterset {\thenprop {\prop{}}}
                                       {\elseprop {\prop{}}}}
                           {\object{}}
                           must be T-Abs, so 
                           {\thenprop {\prop{}}} = {\topprop{}},
                           {\elseprop {\prop{}}} = {\botprop{}}
                           and {\object{}} = {\emptyobject{}}.
         Thus parts 1 and 2 hold for the same reasons as T-True.
         Part 3 holds as \val{} has the same type as {\abs {\x{}} {\s{}} {\e{1}}}
         under \propenvp{}.

      \end{subcase} 
  \end{itemize}
\end{case}

\begin{case}[B-Abs]
        \val{} = ${\closure {\openv{}} {\abs {\x{}} {\s{}} {\e{1}}}}$,
          { \opsem {\openv{}}
                   {\abs {\x{}} {\ty{}} {\e{1}}}
                   {\closure {\openv{}} {\abs {\x{}} {\s{}} {\e{1}}}}}

  \begin{itemize}
    \item[]
      %TODO
      \begin{subcase}[T-Abs]
  \ep{} = {\abs {\x{}} {\s{}} {\ep{1}}},
{ \judgementrewrite {\propenv{}, {\isprop {\s{}} {\x{}}}}
            {\ep{1}} {\ty{}}
             {\filterset {\thenprop {\prop{1}}}
                         {\elseprop {\prop{1}}}}
             {\object{1}}
             {\e{1}}},
           \issubtypein{}
           {\ArrowOne {\x{}} {\s{}}
                      {\ty{1}}
                      {\filterset {\thenprop {\prop{1}}}
                                  {\elseprop {\prop{1}}}}
                      {\object{1}}}
          {\ty{}},
          \inpropenv{\topprop{}}{\thenprop{\prop{}}},
          \inpropenv{\botprop{}}{\elseprop{\prop{}}},
          {\object{}} = {\emptyobject{}}

        Parts 1 and 2 hold for the same reasons as T-True.
        Part 3 holds directly via T-Clos, since \val{} must be a closure.
      \end{subcase}
  \end{itemize}
\end{case}

\begin{case}[BE-Error]
        \opsem {\openv{}} {\e{}} {\errorval{\val{1}}}


  \begin{itemize}
    \item[]
      \begin{subcase}[T-Error] 
  \ep{} = \errorval{\val{1}},
  \e{} = \errorval{\val{1}},
  \ty{} = \Bot,
  \thenprop{\prop{}} = \botprop{}, \elseprop{\prop{}} = \botprop{}, \object{} = \emptyobject{}

        Trivially reduces to an error.
      \end{subcase}
  \end{itemize}
\end{case}

\end{proof}

\end{lemma}

{\wrongtheorem{appendix}}

{\soundnesstheorem{appendix}}

 % include for newpage
%
%% Force CV to appear in TOC, but with no page number
%\addtocontents{toc}{\bigskip Curriculum Vitae\par}
%
%\includepdf[pages=-]{cv/ambrosebs-cv.pdf}


\end{document}
