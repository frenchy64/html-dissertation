%\documentclass[11pt]{iuthesis}
\documentclass{book}

%TODO use something like https://bitbucket.org/amiede/classicthesis/wiki/Home

% this is the dissertation, not the paper
\newcommand{\DISSERTATION}{}
\newcommand{\either}[2]{#1}

\usepackage{savesym}
\savesymbol{r}
\savesymbol{AA}
\usepackage{esop-common}
\usepackage{infer-common}
\usepackage{symb-common}
%\usepackage{quals-common}

\usepackage{hyperref}

\newcommand{\thesisauthor}[0]{Ambrose Bonnaire-Sergeant}
\newcommand{\thesistitle}[0]{Typed Clojure in Theory and Practice} % copied to \title for all-caps
%\newcommand{\thesiskeywords}[0]{Kwd1, Kwd2, Kwd3}

%% Setup for hyperref.
\hypersetup{
  pdftitle={\thesistitle{}},
  pdfauthor={\thesisauthor{}},
  colorlinks=true,
  linkcolor=black,
  citecolor=black,
  urlcolor=black,
}


\title{Typed Clojure in Theory and Practice}
\author{Ambrose Bonnaire-Sergeant}

% pdflatex
%\usepackage[backend=bibtex]{biblatex}
% latexmk
\usepackage[backend=biber]{biblatex}

% https://tex.stackexchange.com/questions/134191/line-breaks-of-long-urls-in-biblatex-bibliography
% wrap url's without spilling the margins
\usepackage{url}
\setcounter{biburllcpenalty}{7000}
\setcounter{biburlucpenalty}{8000}
\addbibresource{bibliography.bib}

\usepackage{thmtools}
%\declaretheorem[numberwithin=chapter]{example}
%\declaretheorem[numberwithin=chapter]{theorem}
%\declaretheorem[numberwithin=chapter]{lemma}
%\declaretheorem[numberwithin=chapter]{corollary}
%\declaretheorem[numberwithin=chapter]{definition}

\begin{document}

\frontmatter %turns off chapter numbering and uses roman numerals for page numbers

\maketitle

\tableofcontents

\newpage

%turns on chapter numbering, resets page numbering and uses arabic numerals for page numbers;
\mainmatter

\part{%Local Type Argument Synthesis with Symbolic Closures
Symbolic Closures}
\label{part:symbolic-closures}

\input{symbolic}

\part{Related and Future Work}
\label{part:related-future-work}

%\input{esop-related-work}

%\input{infer-related-work}

%% NOTE: Haven't pursued the following work yet

%\paragraph{How dynamic languages are used}
%Several languages have seen similar investigations
%into their idioms as I am proposing for Clojure.
%
%A popular motivation is to discover which type system features to support
%when retrofitting a type system.
%% FIXME the is \AAkerblom but there's an error.. also in the bibliography
%Akerblom et. al~\cite{Akerblom:2014:TDF:2597073.2597103} trace dynamic features in Python programs
%via instrumentation. They measured the prevalence of dynamic features in startup versus
%user code, and recorded usage frequencies for a set of dynamic features.
%They concluded dynamism is prevalent in Python, and thus should be supported
%in a retrofitted type system for Python.
%A study along similar lines is also applicable to Clojure, in particular analysing Typed
%Clojure's support for Clojure's dynamic features.
%
%Calla{\'u} et al. \cite{Callau2013} also conducted a large-scale study of
%dynamic Smalltalk idioms to inform future language extensions tooling support.
%Notably, they further perform a qualitative analysis aiming to identify
%the reasons why Smalltalk use these features in the first place, and
%whether they can be replaced with more predictable features. They also 
%measure which kinds of projects (e.g., testing frameworks, user-level libraries, or core system libraries) 
%use particular features more frequently.
%Due to the their prevalence in the open-source Clojure ecosystem,
%Typed Clojure has mainly been tested on user-level libraries.
%We could predict Typed Clojure's applicability to other kinds of projects
%by gathering similar data on how frequently different types of Clojure libraries use
%Clojure's various features.
%
%Andreasen et. al~\cite{Andreasen2016TraceTA} developed
%\emph{trace typing} to explore the design space of JavaScript type systems. 
%Using runtime observations, they studied which control flow techniques
%are used most often in JavaScript programs, and thus, which should
%be supported by an effective type system for JavaScript.
%Typed Clojure implements occurrence typing to reason about control
%flow in Clojure which seems to work well in practice, but a similar
%quantitative analysis could reveal further insights.

%Runtime analysis \cite{Mastrangelo:2015:UYO:2814270.2814313}

%\cite{Mastrangelo:2015:UYO:2814270.2814313} 

\input{symbolic-related}

\chapter{Future work}

The most pressing future work for Typed Clojure is part of the ongoing work
presented after \partref{part:types}.

\section{Future work for Automatic Annotations}
A larger scale investigation of Clojure usage patterns is now possible by
repurposing the automatic annotation tool described in \partref{part:autoann}
to generate and enforce clojure.spec annotations.
As well as testing the robustness of the tool's design, 
the resulting data would be
useful in investigating general questions like how effectively Clojure users utilize unit and generative testing,
how Clojure code evolves of code over time, and the prevalence of idioms that Typed Clojure and clojure.spec
have (and have not) been designed around.

\section{Future work for Extensible Types}
\partref{part:implementations} outlines a code analyzer that paves the way to a future implementation
of extensible typing rules for Typed Clojure.
The next steps in this direction involve deciding the user interface for such a system
and performing a survey of commonly used macros to determine which features must be supported.

\section{Future work for Symbolic Closures}
Symbolic closures (\partref{part:symbolic-closures})
show much promise in improving the user-experience of Typed Clojure.
However, our preliminary work is still not well understood.
\chapref{chapter:symbolic:metatheory} outlines several conjectures we hope to first prove.
Finally, the problem of integrating symbolic closures with type argument synthesis
is a crucial piece of future work, that (we hope) will prove symbolic closures
as indispensable in checking many common Clojure problems.

% Possible future work on Higher-rank types
% - see https://www.microsoft.com/en-us/research/publication/practical-type-inference-for-arbitrary-rank-types/
%   - looks like the journal version of boxy types?
%   - some notes from the paper
%     - a predicative type system only allows a polytype to be instantiated with monotypes
%     - ML_F is both impredicative and supports type inference (but costly to implement & formalize)
%       - also infers principal types
%     - higher-kinded types are orthogonal to higher-rank types, and Haskell's implementation of the former
%       happen to work well with higher-rank types (but no explanation)
%     - the concept of "syntax-directed" rules is given lots of a explanation
%     - \vdash^inst compares two _polytypes_
%     - Kfoury and Wells 1994 show that typeability of System F (with completely erased annotations) is decidable for rank 2 
%       but undecidable for rank 3>=
%     - LTI == "partial" type inference
%       - in the sense that it's not-complete (can't check all programs)
%     - nice discussion of partial type inference


\printbibliography


%\include{esop-appendix} % include for newpage
%
%% Force CV to appear in TOC, but with no page number
%\addtocontents{toc}{\bigskip Curriculum Vitae\par}
%
%\includepdf[pages=-]{cv/ambrosebs-cv.pdf}


\end{document}
