%\documentclass[11pt]{iuthesis}
\documentclass{book}

%TODO use something like https://bitbucket.org/amiede/classicthesis/wiki/Home

% this is the dissertation, not the paper
\newcommand{\DISSERTATION}{}
\newcommand{\either}[2]{#1}

\usepackage{savesym}
\savesymbol{r}
\savesymbol{AA}
\usepackage{esop-common}
\usepackage{infer-common}
\usepackage{symb-common}
%\usepackage{quals-common}

\usepackage{hyperref}

\newcommand{\thesisauthor}[0]{Ambrose Bonnaire-Sergeant}
\newcommand{\thesistitle}[0]{Typed Clojure in Theory and Practice} % copied to \title for all-caps
%\newcommand{\thesiskeywords}[0]{Kwd1, Kwd2, Kwd3}

%% Setup for hyperref.
\hypersetup{
  pdftitle={\thesistitle{}},
  pdfauthor={\thesisauthor{}},
  colorlinks=true,
  linkcolor=black,
  citecolor=black,
  urlcolor=black,
}


\title{Typed Clojure in Theory and Practice}
\author{Ambrose Bonnaire-Sergeant}

% pdflatex
%\usepackage[backend=bibtex]{biblatex}
% latexmk
\usepackage[backend=biber]{biblatex}

% https://tex.stackexchange.com/questions/134191/line-breaks-of-long-urls-in-biblatex-bibliography
% wrap url's without spilling the margins
\usepackage{url}
\setcounter{biburllcpenalty}{7000}
\setcounter{biburlucpenalty}{8000}
\addbibresource{bibliography.bib}

\usepackage{thmtools}
%\declaretheorem[numberwithin=chapter]{example}
%\declaretheorem[numberwithin=chapter]{theorem}
%\declaretheorem[numberwithin=chapter]{lemma}
%\declaretheorem[numberwithin=chapter]{corollary}
%\declaretheorem[numberwithin=chapter]{definition}

\begin{document}

\frontmatter %turns off chapter numbering and uses roman numerals for page numbers

\maketitle

\tableofcontents

\newpage

%turns on chapter numbering, resets page numbering and uses arabic numerals for page numbers;
\mainmatter

\part{%Local Type Argument Synthesis with Symbolic Closures
Symbolic Closures}
\label{part:symbolic-closures}

\chapter{Background}

As is inevitable for an optional type system, there are many
Clojure programs that Typed Clojure was not designed to type check.
These programs contain Clojure idioms that are often either intentionally
not supported by Typed Clojure's initial design, or 
were introduced to Clojure at a later date.
Regardless, programmers will inevitably want to use these features 
in their Typed Clojure programs---but crucially without breaking
support for existing idioms.
In this part, we explore what kinds of idioms are missing
support in Typed Clojure, and propose solutions in the form of
backwards-compatible extensions.

As we discussed in \partref{part:types}, Typed Clojure's initial
design is strongly influenced by Typed Racket. In particular,
Typed Clojure's static semantics of
combinining local type inference and occurrence typing
to check fully-expanded code
comes directly from Typed Racket.
This shared base is appropriate, given the similarities between
the base Clojure and Racket languages.
It is also effective, seamlessly handling many control flow
idioms, capturing many polymorphic idioms, and often yielding
predictable type error messages.
However, there are important tradeoffs to consider in this design---in the following
sections we introduce them and propose extensions to attempt to nullify
their downsides.

\section{Enhancing Local Type Inference}

``Local Type Inference''~\cite{PierceLTI} refers to the combination of
bidirectional type propagation 
and local type argument synthesis.
%specific approach
%to partial type inference for languages with subtyping and impredicative
%polymorphism.
%``Partial type inference'' here is
%problem of inferring 
%(Pfenning~\cite{Pfenning1988partial} describes the distinction more thoroughly).
%
Concerning the limitations of local type inference,
Hosoya and Pierce~\cite{hosoya1999good}
isolate two drawbacks.
The first is dealing with ``hard-to-synthesize arguments''.
To understand this, we must appreciate a key ingredient of local type inference
called \emph{bidirectional propagation}, which 
we use the example of type checking \clj{(inc 42)} to demonstrate.
If we have already checked \clj{inc} to have type \clj{[Int -> Int]}, we
now have a choice of how to check the argument \clj{42} is an \clj{Int}.
The first is to ascribe an expected type to \clj{42} of \clj{Int}
and rely on
bidirectional \emph{checking mode} to ensure \clj{42} has the correct type
once we check it.
The second is to infer the type of \clj{42} (without an expected type) using 
bidirectional \emph{synthesis mode}, and then ensure the inferred type
is compatible with \clj{Int} after the fact.
A useful analogy in terms of expressions is that checking mode propagates
information outside-in, and synthesis mode propagates inside-out.
A similar analogy in terms of a type derivation tree (that grows upwards)
relates checking and synthesis modes to information being passed
up and down the tree, respectively.

To best serve the purposes of local type inference, it is crucial to stay in
bidirectional \emph{checking} mode as much as possible.
The ``hard-to-synthesize arguments'' problem occurs when type argument
inference interferes with the ability to stay in checking mode, and
thus forces the bidirectional propagator into synthesis mode
for arguments that require checking mode.
For example, to type check

\clj{(map (fn [x] (inc x)) [1 2 3])},

where \clj{map} has type

\clj{(All [a b] [[a -> b] (Seqable a) -> (Seqable b)])},

we use type argument inference to determine how to instantiate type variables \clj{a}
and \clj{b} based on \clj{map}'s arguments.
Unfortunately, 
to answer this question,
the naive local type inference algorithm~\cite{PierceLTI}
uses synthesis mode to retrieve the argument types,
and so checks \clj{(fn [x] (inc x))} in synthesis mode.
No information is propagated about the type of \clj{x},
so this expression will fail to type check, demonstrating
why functions are hard-to-synthesize.

The second drawback noted by Hosoya and Pierce are
cases where there is no ``best'' type argument to infer.
This occurs when there is not enough information available
to determine how to instantiate a type such that the program
has the best chance of type checking, and so it must be guessed.
A representative case where this occurs is inferring the
type of a reference from just its instantiation, such
that optimal types are given to reads and writes.
For example, the following code creates a Clojure Atom
(a reference type) with initial value \clj{nil}, writes
\clj{0} to the Atom, and then increments the Atom's value.

{
\begin{lstlisting}[language=Clojure]
(let [r (atom nil)]
  (reset! r 0)
  (inc @r))
\end{lstlisting}
}

What type should \clj{r} be assigned? From its initial binding,
\clj{(Atom nil)} seems appropriate, but the subsequent write
would fail. Alternatively, assigning \clj{(Atom Any)} would
allow the write to succeed, but the the final read would fail
because it expects \clj{Int}.
This demonstrates difficulties of the ``no-best-type-argument'' problem.

Hosoya and Pierce report unsatisfactory results in their attempts to
fix these issues, in both the effectiveness and complexity
in their solutions.
They speculate that these difficulties might be better 
addressed at the language-design level---rather than algorithmically---in ways that
keep the bidirectional propagator in checking mode.
For the ``no-best-type-argument'' problem,
we agree with this assessment, since
addressing the problem mostly amounts to annotating 
all reference constructors.
To this end,
Typed Clojure offers several
wrappers for common functions where this problem
is common---the previous example might use the ``typed''
constructor \clj{(t/atom :- (U nil Int), nil)}.

However, the ``hard-to-synthesize arguments'' problem
is a deeper and more pervasive issue when checking Clojure code.
We don't have the luxury, desire, nor do we think it would be particularly
successful to introduce new core idioms to Clojure,
and so we attempt to solve the this problem algorithmically.

%\begin{lstlisting}[language=Clojure]
%(for [i (range 100)]
%  (map (fn [j] (inc j))
%       (range i)))
%\end{lstlisting}

Hosoya and Pierce outline the two main challenges that must be
addressed to solve the ``hard-to-synthesize arguments'' problem.
First, we must provide a strategy for identifying which arguments 
should be avoided.
For instance,
they provide a simple grammar for identifying hard-to-synthesize arguments,
which includes (for Standard ML) unannotated functions and unqualified constructors.
Second, an alternative (probably more complicated) algorithm
for inferring type arguments is needed that also handles
avoided arguments.
Their experiments show that the naive approach does not suffice,
and hint at the delicate handling needed to effectively maximize or minimize
instantiated types to stay in checking mode.
We will now use these challenges as a presentational framework to outline our own approach.

In our experience, the most common hard-to-synthesize expression in Clojure code
is the function.
Clojure's large standard library of higher-order functions and encouragement
of functional programming result in many usages of anonymous functions, which almost
always require annotations to check with Typed Clojure.
So, to answer Hosoya and Pierce's first challenge, 
we avoid checking hard-to-synthesize function expressions by
introducing a new function type: a \emph{symbolic closure type}.
A symbolic closure does not immediately check the function body. Instead,
the function's code along with its local type context is saved
until enough information is available to check
the function body in checking mode.
We present more details about symbolic closures in \chapref{chapter:symbolic:symbolic-closures}.

Now that we have delayed the checking of hard-to-check arguments,
Hosoya and Pierce's second challenge calls for an enhanced
type argument reconstruction algorithm to soundly handle
them.
We delegate this challenge to future work, and study symbolic closures
with an off-the-shelf
type argument reconstruction algorithm.
%Our investigation led us to create \emph{directed local type inference}
%(\chapref{chapter:symbolic:directed-lti}),
%which determines the possible data flows through a polymorphic function
%by noting the positions of type variable occurrences, and attempts to
%use this information to check its arguments in an optimal order for remaining
%in bidirectional checking mode.

%\section{Custom typing rules}
%
%Besides local type inference,
%another significant feature inherited from Typed Racket is that
%code is fully expanded before checking.
%This unfortunately means that macros with complex expansions
%are often uncheckable, and display cryptic error messages when attempting
%to do so.
%We investigate providing the user with \emph{custom typing rules} (\chapref{chapter:symbolic:custom-rules})
%as an extension point to customize how to type check a macro before
%it is expanded.
%As discussed in 
%\partref{part:implementations}, Typed Clojure's initial design does
%not support custom typing rules, so we exploit the infrastructure
%discussed there,
%and present our investigation into the user interface for the rules in
%\chapref{chapter:symbolic:custom-rules}.

% - "Avoiding hard-to-synthesize arguments"
%   1. need mechanism to decide which arguments to avoid
%   2. more complicated scheme for determining best type arguments

% - Problem
%   - many common idioms cannot be checked
%   - limitations of local type inference
%   - made harder by occurrence typing
%   - want general solutions available to all users
%   - preliminary investigation of several techniques
% - Possible solutions
%   - symbolic analysis
%     - symbolic closures
%       - deal with "obvious" local function annotations
%     - inlining
%   - directed local type inference
%     - derive data flow from polymorphic types for more aggressive local type variable inference
%   - custom typing rules
%     - interface for describing how to check an unexpanded macro call
%       - or complex functions
%     - custom errors
% - Constraints
%   - some speculation of how well they compose together
%   - small models without rigorous proofs
%   - case studies with real Clojure idioms

\chapter{Delayed checking for Unannotated Local Functions}
\label{chapter:symbolic:symbolic-closures}

Using bidirectional type checking, functions are hard-to-synthesize types for.
Put another way, to check a function body successfully using only locally available information,
types for its parameters are needed upfront.
For top-level function definitions, this is not a problem in many
optional type systems since top-level annotations would be provided
for each function.
However, for anonymous functions it's a different story.
The original local type inference algorithm~\cite{PierceLTI}
lacks a synthesis rule for unannotated functions, instead relying on bidirectional
propagation of types,
but due to the prevalence
of hard-to-synthesize anonymous functions in languages like JavaScript, Racket, and Clojure,
optional type systems for the languages add their own rules.

Typed Racket and Typed Clojure implement a simple but sound strategy
to check unannotated functions. The body of the function is checked
in a type context where its parameters are of type \clj{Any},
the \texttt{Top} type.
This helps check functions that don't use their arguments, or only
use them in positions that require type \clj{Any}.
For example, both \clj{(fn [x] "a")} and \clj{(fn [x] (str "x: " x))} 
synthesize to \clj{[Any -> String]} in Typed Clojure.
The downsides to this strategy are that unannotated functions are never
inferred as polymorphic, and functions that use their arguments
at types more specific than \clj{Any} are common.

TypeScript~\cite{typescript}, an optional type system for JavaScript,
takes a similar approach, but instead of annotating parameters with
TypeScript's least permissive type called \js{unknown},
by default it assigns parameters the unsound dynamic type \js{any}.
In TypeScript, \js{any} can be implicitly cast to any other type,
so the type checker will (unsoundly) allow any usage of unannotated arguments.
If this behavior is unsatisfactory,
the \js{noImplicitAny} flag removes special handling for unannotated
functions altogether, and TypeScript will demand explicit annotations for all arguments.

In this chapter, we present an alternative approach to checking unannotated functions
based on the insight that a function's body need only be type checked if and when it is called.
For example, the program \clj{(fn [x] (inc x))} cannot throw a runtime error because
the function is never called, and so a type system may soundly treat the function body as unreachable code.
On the other hand, wrapping the same program in the invocation
\clj{((fn [x] (inc x)) nil)}
makes the runtime error possible, and so a sound static type system must flag the error.

Exploiting this insight in the context of a bidirectional type checker using
local type inference requires many considerations.
First, we must decide in which situations is it desirable to delay checking a function.
Second, we must identify the information that must be saved in order to delay checking a function,
and then choose a suitable format for packaging that information.
Third, we must identify how a function is deemed ``reachable'',
and then which component of the type system is responsible for checking a function body.
Fourth, it is desirable to identify and handle the ways in which 
infinite loops are possible, such as the checking of a delayed function triggering
another delayed function to check, which triggers another delayed check, ad nauseam.
Fifth, we must determine how delayed functions interact with polymorphic types
during type argument reconstruction.

We address all these considerations in the following sections, except
for the final one, which we delegate to future work. %\chapref{chapter:symbolic:directed-lti}.

\section{Overview}
\label{symbolic:section:overview}

In this section, we explore some of the implications that come with delayed checks for local functions,
by example.
To explain the essense of the challenges we actually address, we avoid polymorphic functions in this section
%(we isolate those issues in \chapref{chapter:symbolic:directed-lti})
and restrict ourselves to non-recursive monomorphic functions.

First, let \clj{inc} be of type \clj{[Int -> Int]}.
The following, then, is well typed because \clj{1} is an \clj{Int}.

\begin{lstlisting}[language=Clojure]
(inc 1)
\end{lstlisting}

Using the standard bidirectional application type rule, \clj{inc} is checked first,
followed by \clj{1}.
However, eta-expanding the operator does not behave as nicely.

\begin{lstlisting}[language=Clojure]
((fn [x] (inc x)) 1)
\end{lstlisting}

Like usual, the standard application rule checks the function first.
However, there is no annotation for \clj{x}, so the function body will fail
to check.
This is unfortunate, especially in a type system that claims to be ``bidirectional'',
since the information that \clj{x} is an \clj{Int} is adjacent to the function
in the form of an argument.
One strategy to alleviate this problem is to always check arguments first~\cite{xie2018let}.
However, that nullifies the ability for the operator to propagate information
to its arguments, whose advantages are exploited to good effect in Colored Local Type Inference~\cite{coloredlti01}.

We combine both flavors by keeping the standard operator-first checking order
but delay the checking of unannotated functions.
Then, an additional application rule handles applications of
unannotated functions to force their checking.
So in this case, the checking of \clj{(inc x)}
is delayed until the argument \clj{1} is inferred as \clj{Int},
after which this information is used to check \clj{(inc x)}
in the extended type context where \clj{x : Int}.

We could imagine hard-coding a type rule that manually delays
direct applications of unannotated functions until after checking
its arguments.
However, that does not generalize to more complicated examples.
Take the following illustrative code, identical the previous
example, except the function is let-bound as \clj{f}.

{
%\lstset{numbers=left,xleftmargin=2em,framexleftmargin=1.5em}
\begin{lstlisting}[language=Clojure]
(let [f (fn [x] (inc x))](*@\label{symbolic:example:let-bound:def-f}@*)
  (f 1))(*@\label{symbolic:example:let-bound:app-f}@*)
\end{lstlisting}
}

Instead of following the brittle strategy of creating yet-another special rule to delay checking
let-bound functions, we generalize the idea.
We make a delayed function check a first-class concept in our type-system by
creating a new type for it.
Roughly, \clj{f} would have a delayed function type---introduced by
a type rule for unannotated functions---and \clj{(f 1)}
would force a check for the delayed function---by an application
rule that handles delayed function \emph{types} (not syntax-driven).

Now we must decide what a delayed function type consists of.
Clearly, the \emph{code} of the function must be preserved until
it is checked, otherwise the application rule would have nothing
to work with.
We note that our static semantics of saving
the code of a function to check later
is analogous to the runtime strategy of
evaluating a function as \emph{closure},
and using beta-reduction to extract the original
function from the closure and apply it to its arguments.

The trick in maintaining lexical scope during beta-reduction for closures
is to apply the function under the \emph{function definition's}
environment, instead of the application site's.
For example,
\figref{symbolic:example:closure-red}
evaluates
to \clj{2}
because
the occurrence of
\clj{y} on line \ref{symbolic:example:closure-red:y-usage}
is bound to \clj{1} by line \ref{symbolic:example:closure-red:y-def-site}.
If we used the local environment at the application site (line \ref{symbolic:example:closure-red:f-app}),
\clj{y} would be bound on line \ref{symbolic:example:closure-red:y-app-site}
to \clj{nil},
and would throw a runtime error.

% must save type context
\begin{figure}
{
\lstset{numbers=left,xleftmargin=2em,framexleftmargin=1.5em}
\begin{lstlisting}[language=Clojure]
(let [f (let [y 1](*@\label{symbolic:example:closure-red:y-def-site}@*)
          (fn [x] (+ x y)))](*@\label{symbolic:example:closure-red:y-usage}@*)
  (let [y nil](*@\label{symbolic:example:closure-red:y-app-site}@*)
    (f 1)))(*@\label{symbolic:example:closure-red:f-app}@*)
\end{lstlisting}
}
  \caption{This example evaluates to \clj{2} with lexically scoped variables.}
  \label{symbolic:example:closure-red}
\end{figure}

The crucial insight is that
the same trick applies to \emph{checking} delayed function types,
except at the \emph{type}-level.
Specifically, the occurrence of \clj{y}
on line \ref{symbolic:example:closure-red:y-usage}
must be checked as type \clj{Int} (from line \ref{symbolic:example:closure-red:y-def-site}),
and not type \clj{nil} (from line \ref{symbolic:example:closure-red:y-app-site}).
So, a delayed function type pairs a function's code with the type environment
at the function definition site.
This strongly resembles a ``type-level'' closure that is reduced symbolically,
and so we call this new type a \emph{symbolic closure}.

We can use symbolic closures to inline higher-order-function definitions.
In the following example, \clj{app} would normally need a higher-order
or polymorphic
annotation to handle the application on the final line.
Instead, with symbolic closures, type checking reduces in a few steps to simply checking
\clj{(inc x)} where \clj{x : Int}.

% more beta reduction
\begin{lstlisting}[language=Clojure]
(let [f (fn [x] (inc x))
      app (fn [g y] (g y))]
  (app f 1))
\end{lstlisting}

As alluded to in the previous section, we must identify
all type system components who are responsible for checking symbolic closures,
and ensure they perform their obligations correctly.
The following example uses a higher-order function
\clj{app-int} to increment the value \clj{1}.
Since \clj{app-int} is annotated, it will be checked
by the standard application rule.
However, its first argument will be delayed as a symbolic
closure---now we must identify who is responsible for checking it.

% using subtyping to check symbolic closures.
\begin{lstlisting}[language=Clojure]
(ann app-int [[Int -> Int] Int -> Int])
(defn app-int [f x] (f x))
...
(app-int (fn [x] (inc x)) 1)
\end{lstlisting}

The type signature of \clj{app-int},
clearly says that its first argument may be called with an \clj{Int}.
Therefore, to maintain soundness, applications of \clj{app-int}
must ensure its first argument accepts \clj{Int}.
The standard application type rule uses subtyping to ensure
provided arguments are compatible with the formal parameter types of
the operator.
To handle symbolic closures, we preserve the standard application rule
and instead add a subtyping case for symbolic closures.

In this case, the subtyping relation would be asked to verify if
``the symbolic closure type representing \clj{(fn [x] (inc x))}
is a subtype of \clj{[Int -> Int]}''.
This can be answered by checking the symbolic closure
returns \clj{Int} when 
\clj{x} is type \clj{Int}---and so this subtyping case
delegates to checking if the symbolic closures inhabits the given type.
The subtype relationship is true if the check succeeds without type error,
otherwise it is false.

The correct ``contravariant subtyping left-of-the-arrow''
is naturally preserved.
In this case, the left-of-the-arrow check is ``\clj{Int} is a subtype of \clj{x}'s type'', and
annotating \clj{x} as \clj{Int} turns this statement into the reflexively true ``\clj{Int} is a subtype of \clj{Int}''.
At a glance, it may seem that we are wasting the benefits
of this contravariant rule---after all, it enables \clj{x} to be \emph{any} supertype of
\clj{Int}, such as \clj{Num} or even \clj{Any}.
However, it is in our interest to propagate the most precise parameter types
so then function bodies have the best chance to check without error.
Since symbolic closures are designed to support rechecking their bodies at different argument types,
a symbolic function can simply be rechecked with the less-precise types
when it comes time to broaden its domain.

This scheme extends to subtyping with arbitrarily-nested function types.
To demonstrate nesting to the right of an arrow,
the following code sums 1 with itself via
\clj{curried-app-int}, which accepts a curried
function of two arguments \clj{f} and a number \clj{x}, and 
provides \clj{x} as both arguments to \clj{f}.

% using subtyping to check symbolic closures.
\begin{lstlisting}[language=Clojure]
(ann curried-app-int [[Int -> [Int -> Int]] Int -> Int])
(defn curried-app-int [f x] ((f x) x))
...
(curried-app-int (fn [y] (fn [x] (+ x y))) 1)
\end{lstlisting}

The standard application rule will ensure 
``the symbolic closure of \clj{(fn [y] (fn [x] (+ x y)))}
is a subtype of
\clj{[Int -> [Int -> Int]]}'', which involves assuming
\clj{y : Int} and then checking the \emph{code} \clj{(fn [x] (+ x y))}
at type \clj{[Int -> Int]}---which just uses the standard
function rule.

To demonstrate nesting to the left of an arrow,
\clj{app-inc} again computes \clj{(inc 1)}
in an even more convoluted way with \clj{app-inc}---by accepting a function
\clj{f} that it passes both \clj{inc} and its second argument to.

% using subtyping to check symbolic closures.
\begin{lstlisting}[language=Clojure]
(ann app-inc [[[Int -> Int] Int -> Int] Int -> Int])
(defn app-inc [f x] (f inc x))
...
(app-inc (fn [g y] (g y)) 1)
\end{lstlisting}

Importantly, \clj{app-inc}'s first argument has a function
type to the left on an arrow, in particular \clj{[Int -> Int]}.
Under these conditions, subtyping asserts ``the symbolic
closure \clj{(fn [g y] (g y))} is a subtype of \clj{[[Int -> Int] Int -> Int]}''
by assuming \clj{g : [Int -> Int]} and \clj{y : Int} and
verifying that \clj{(g y)} checks as \clj{Int}---which is almost immediate by
the standard application rule.

We leverage some syntactic restrictions
to avoid the need for further subtyping cases for symbolic closures.
First, symbolic closures cannot be annotated by the programmer,
and can only be introduced by the ``unannotated function'' typing rule.
Second (as discussed in \secref{analyzer:extensibility:side-effects}),
top-level variables are not allowed to inherit the types of their initial
values, and must be explicitly annotated.
These restrictions ensure symbolic closures both cannot occur to the
left of an arrow type, and 
cannot propagate beyond the top-level form it was defined in.
%This stretches the metaphor of ``local'' type inference
%beyond just a single tree walk using
%bidirectional propagation,

\subsection{Performance and error messages}

% FIXME need to be more precise about "undecidable". What problem are
% we deciding? See Wells '94 for some details. I think so far I
% mean "type checking always terminates (conservatively)"

While useful, allowing the type system to perform beta-reduction
requires careful planning: type checking time is now proportional 
to the running time of the program!
Unsurprisingly, this makes type checking with a naive implementation of symbolic
closures undecidable.
Without intervention,
the next program (an infinite loop using the y-combinator that computes \clj{(inc (inc (inc ...)))})
would send the \emph{type system} into an infinite loop.

%TODO much simpler example: ((fn [x] (x x)) (fn [x] (x x)))

% infinite loops
\begin{lstlisting}[language=Clojure]
(let [Y (fn [f]
          ((fn [g] (fn [x] (f (g g) x)))
           (fn [g] (fn [x] (f (g g) x)))))]
  (let [compute (Y (fn [f x] (inc (f x))))]
    (compute 1)))
\end{lstlisting}

To prevent such loops, we limit the number of symbolic reductions
done at type-checking time.
As a conservative solution to the halting
problem, this limit will prematurely halt some programs that would
otherwise fully reduce in a finite number of steps.
For example, if we set the reduction limit to 5\ in
the following code,
during the 6th reduction of \clj{f} the type system will
throw an error.

% premature halting
\begin{lstlisting}[language=Clojure]
(let [f (fn [x] x)]
  (f (f (f (f (f (f 1)))))))
\end{lstlisting}

In simple cases like these, the error message 
can guide the user to fixing the error.
For example, the type system would suggest 
annotating \clj{f} as \clj{[Int -> Int]} (by collecting
argument and return types as the program is reduced),
which would cause the program to check successfully
under the same conditions.
For cases with more heterogeneous argument and return types---like the y-combinator---the 
error message would just note which function caused
the reduction quota to be depleted.

As Wells~\cite{wells1994typability} remarks,
stopgap measures such as this to circumvent undecidable
type inference algorithms negatively affect
program portability.
For example, a different reduction limit may cause
a program to fail to type check that otherwise type checked
in a previous version.
We hope to learn reasonable defaults for the reduction limit
by experience.

%Note that using the type checker to decide subtyping
%has unfortunate implications for 
%the aforementioned annotation suggestions
%for reduction-limit error messages.
%An ``obviously-failing'' subtyping check might trigger a
%check for irrelevant arguments, and then provide them to the user.
%A curious aside: if symbolic closures are identified just by their code and definition
%type environments, suggestions may also be merged for functions with
%identical code and scope.

% TODO performance
% - undecidable 
%   - heuristics needed to halt search
%   - type checking time proportional to running time of program
% - for finitely running programs:
%   - degenerate case checking time complexity becomes at least exponential time in the size of the program because we can recheck a function
%     body multiple times, and a symbolic closure can be duplicated
%
% eg. (let [pair (fn [f g] (f (g) (g)))] (pair (fn [x y] (+ x y)) (fn [] 1)))
% - (fn [] 1) is checked twice
%   - can "stack" these recheckings, worst case is infinite
% - Damas-Milner algorithm checks a function definition once to determine its principle type scheme
%   - exponential time & space
%     - because principle type schemes can become very large
%     - also exponential time to print a type
%     - symbolic closures are also exponential time to print a type (naively)
%       since they can be duplicated
%       - I think these are similar reasons to Milner's algorithm

% do we need a story for runtime casting from Any to [Int -> Int]?
%\begin{lstlisting}[language=Clojure]
%(ann dynapp-int [Any Int -> Int])
%...
%(dynapp-int (fn [x] (inc x)) 1)
%\end{lstlisting}

% no idea what to do with negation function types 
%\begin{lstlisting}[language=Clojure]
%(ann app-int [(U [Int -> Int] (I Any (Not [Int -> Int]))) Int -> Int])
%...
%(app-int (fn [x] (inc x)) 1)
%\end{lstlisting}


\section{Formal model}
\label{symbolic:section:formal-model}

We formalize a restriction of symbolic closure types by defining an explicitly typed internal language,
providing an external languages that can omit annotations,
and formulating a type inference algorithm based on symbolic closures to recover omitted annotations.
This approach is similar to Local Type Inference~\cite{PierceLTI}
and Colored Local Type Inference~\cite{coloredlti01},
except where they utilize bidirectional type propagation to locally determine function parameter types,
we instead use symbolic closures to propagate type information
(since we have a synthesis rule for functions).

\subsection{Internal Language}

\begin{figure}[h]
$$
\begin{array}{lrll}
  \ltiE{}, \ltiF{} &::=& \ltivar{} \alt
                         \ltifuntparamargtype{\ova{\ltitvar{}}}
                                             {\ltivar{}}
                                             {\ltiT{}}
                                             {\ltiE{}}
                         \alt
                         \ltiappinst{\ltiF{}}{\ova{\ltiR{}}}{\ltiE{}} \alt
                         \ltisel{\ltiE{}}{\ltivar{}} \alt
                         \ltiRec{\ova{\ltivar{} = \ltiE{}}}
                      &\mbox{Terms} \\
  \ltiT{}, \ltiS{}, \ltiR{} &::=& \ltitvar{} 
                         \alt
                         \ltiTop
                         \alt
                         \ltiBot
                         \alt \ltiPolyFn{\ltiT{}}{\ova{\ltitvar{}}}{\ltiT{}}
                         \alt
                         \ltiRec{\ova{\hastype{\ltivar{}}{\ltiT{}}}}
                      &\mbox{Types} \\
  \ltiEnv{} &::=& \ltiEmptyEnv \alt
                  \ltiEnvConcat{\ltiEnv{}}{\hastype{\ltivar{}}{\ltiT{}}} \alt
                  \ltiEnvConcat{\ltiEnv{}}{\ltitvar{}}
                      &\mbox{Type Environments} \\
\end{array}
$$
\caption{Internal Language Syntax}
\label{symbolic:figure:internal-language}
\end{figure}

Our internal language is based on System \ltiFsub extended with records, and is functionally identical
to that used to model Colored Local Type Inference~\cite{coloredlti01}, except
our lambda terms require full (return) type annotations.
\figref{symbolic:figure:internal-language} shows the syntax
for the internal language.
Terms \ltiE{} and \ltiF{} range over 
variables \ltivar{},
explicitly typed polymorphic functions
                         \ltifuntparamargtype{\ova{\ltitvar{}}}
                                             {\ltivar{}}
                                             {\ltiT{}}
                                             {\ltiE{}},
function application
with explicit type arguments
\ltiappinst{\ltiF{}}{\ova{\ltiR{}}}{\ltiE{}},
record selectors
\ltisel{\ltiE{}}{\ltivar{}},
and record constructors
\ltiRec{\ova{\ltivar{} = \ltiE{}}}.
Types \ltiT{}, \ltiS{}, and \ltiR{} are 
type variables \ltitvar{},
top type \ltiTop,
bottom type \ltiBot,
polymorphic function types \ltiPolyFn{\ltiT{}}{\ova{\ltitvar{}}}{\ltiS{}}
(where bound type variables are enumerated over the arrow),
and
record types \ltiRec{\ova{\hastype{\ltivar{}}{\ltiT{}}}}.
Type environments \ltiEnv{}
consist of 
the empty environment
\ltiEmptyEnv,
concatenation of variable typings
``\ltiEnvConcat{\ltiEnv{}}{\hastype{\ltivar{}}{\ltiT{}}}'',
and 
concatenation of type variables
``\ltiEnvConcat{\ltiEnv{}}{\ltitvar{}}''.

We assume different term and type variables are distinct,
and treat terms and types that are equal up to alpha-renaming as equivalent.
Record terms and types have unordered fields.
We treat primitive types (like \sml{String}) as free type variables.

\begin{figure}
  \begin{mathpar}

    \boxed{
    \infer[]
    {
      \ltitjudgementNoElab{\ltiEnv{}}{\ltiE{}}{\ltiT{}}
      \\\\
      \text{\ltiE{} is of type \ltiT{}
      }
      \\\\
      \text{ in context \ltiEnv{}.}
                 }
                 {}
                 }

    \infer [\ltiIVar]
    {}
    {
    \ltitjudgementNoElab
                    {\ltiEnv{}}
                    {\ltivar{}}
                    {\ltiEnvLookup{\ltiEnv{}}{\ltivar{}}}
                 }

    \infer [\ltiISel]
    {
    \ltitjudgementNoElab{\ltiEnv{}}
                  {\ltiE{}}
                  {\ltiRec{\hastype{\ltivar{1}}{\ltiT{1}}, ..., \hastype{\ltivar{i}}{\ltiT{i}} , ..., \hastype{\ltivar{n}}{\ltiT{n}}}}
    }
    {
    \ltitjudgementNoElab{\ltiEnv{}}
                  {\ltisel{\ltiE{}}{\ltivar{i}}}
                  {\ltiT{i}}
    }

    \infer [\ltiISelBot]
    {
    \ltitjudgementNoElab{\ltiEnv{}}
                     {\ltiE{}}
                     {\ltiBot}
    }
    {
    \ltitjudgementNoElab{\ltiEnv{}}
                  {\ltisel{\ltiE{}}{\ltivar{i}}}
                  {\ltiBot}
    }

    \infer [\ltiIAbs]
    { 
    \ltitjudgementNoElab{\ltiEnvConcat{\ltiEnv{}}
                                {\ltiEnvConcat{\ova{\ltitvar{}}}
                                              {\hastype{\ltivar{}}{\ltiT{}}}}}
                  {\ltiE{}}
                  {\ltiS{}}
    }
    {
    \ltitjudgementNoElab{\ltiEnv{}}
                  {\ltifuntparamargtype{\ova{\ltitvar{}}}
                                   {\ltivar{}}
                                   {\ltiT{}}
                                   {\ltiE{}}}
                  {\ltiPoly{\ova{\ltitvar{}}}{\ltiFn{\ltiT{}}{\ltiS{}}}}
                 }
                 \ \ \ \
%
    \infer [\ltiIAppInst]
    {
    \ltitjudgementNoElab{\ltiEnv{}}
                  {\ltiF{}}
                  {\ltiPolyFn{\ltiT{}}{\ova{\ltitvar{}}}{\ltiS{}}}
                    \\\\
    \ltitjudgementNoElab{\ltiEnv{}}
                  {\ltiE{}}
                  {\ltiTp{}}
                  \\
                  \ltiisubtype{\ltiEnv{}}{\ltiTp{}}{\ltireplace{\ova{\ltiR{}}}{\ova{\ltitvar{}}}{\ltiT{}}}
    }
    {
      \ltitjudgementNoElab{\ltiEnv{}}
                    {\ltiappinst{\ltiF{}}
                                {\ova{\ltiR{}}}
                                {\ltiE{}}}
                    {\ltireplace{\ova{\ltiR{}}}{\ova{\ltitvar{}}}{\ltiS{}}}
    }
                 \ \ \ \
%
    \infer [\ltiIAppInstBot]
    {
    \ltitjudgementNoElab{\ltiEnv{}}
                  {\ltiF{}}
                  {\ltiBot}
                  \\\\
    \ltitjudgementNoElab{\ltiEnv{}}
                  {\ltiE{}}
                  {\ltiS{}}
    }
    {
    \ltitjudgementNoElab{\ltiEnv{}}
                  {\ltiappinst{\ltiF{}}{\ova{\ltiR{}}}{\ltiE{}}}
                  {\ltiBot{}}
    }
                 \ \ \ \
%
    \infer [\ltiIRec]
    {
    \overrightarrow{
    \ltitjudgementNoElab{\ltiEnv{}}
                  {\ltiE{}}
                  {\ltiT{}}
                  }
    }
    {
    \ltitjudgementNoElab{\ltiEnv{}}
                  {\ltiRec{\ova{\ltivar{} = \ltiE{}}}}
                  {\ltiRec{\ova{\hastype{\ltivar{}}{\ltiT{}}}}}
    }

  \end{mathpar}
  \caption{Internal language type system
  }
  \label{symbolic:figure:internal-language-type-system}
\end{figure}

\figref{symbolic:figure:internal-language-type-system}
presents the type system for the internal language
\ltitjudgementNoElab{\ltiEnv{}}{\ltiE{}}{\ltiT{}},
pronounced ``\ltiE{} is of type \ltiT{} in context \ltiEnv{}.''
\ltiIVar is the normal variable lookup rule.
\ltiISel selects a field already present in a record.
\ltiISelBot allows selecting fields from \ltiBot.
\ltiIAbs checks a function definition at its annotated type.
\ltiIAppInst checks a function application with explicit type arguments.
\ltiIAppInstBot allows applying operators of type \ltiBot.
\ltiIRec checks record constructors.

\begin{figure}
  \begin{mathpar}
    \boxed{
    \infer[]
    {}
    {
      \ltiisubtype{\ltiEnv{}}{\ltiT{}}{\ltiS{}}
      \\\\
      \text{\ltiT{} is a subtype of \ltiS{}.
      }
                 }
                 }

    \infer [\ltiSTVar]
    {}
    {
    \ltiisubtype{\ltiEnv{}}{\ltitvar{}}{\ltitvar{}}
    }

    \infer [\ltiSTop]
    {}
    { \ltiisubtype{\ltiEnv{}}{\ltiT{}}{\ltiTop}}

    \infer [\ltiSBot]
    {}
    { \ltiisubtype{\ltiEnv{}}{\ltiBot}{\ltiT{}}}

    \infer [\ltiSRec]
    {
    \overrightarrow{\ltiisubtype{\ltiEnv{}}{\ltiT{}}{\ltiS{}}}
    }
    {
    \ltiisubtype{\ltiEnv{}}
                    {\ltiRec{\ova{\hastype{\ltivar{}}{\ltiT{}}},
                             \ova{\hastype{\ltivarp{}}{\ltiTp{}}}}}
                    {\ltiRec{\ova{\hastype{\ltivar{}}{\ltiS{}}}}}
    }

    \infer [\ltiSFn]
    {
          \ltiisubtype{\ltiEnv{}}{\ltiS{}}{\ltiSp{}}
          \\
          \ltiisubtype{\ltiEnv{}}{\ltiT{}}{\ltiTp{}}
          }
    {\ltiisubtype {\ltiEnv{}}
                     {\ltiPolyFn{\ltiSp{}}{\ova{\ltitvar{}}}{\ltiT{}}}
                     {\ltiPolyFn{\ltiS{}}{\ova{\ltitvar{}}}{\ltiTp{}}}
                     }

  \end{mathpar}
  \caption{Internal language subtyping
  }
  \label{symbolic:figure:internal-language-subtyping}
\end{figure}

\figref{symbolic:figure:internal-language-subtyping}
presents the subtyping for the internal language
\ltiisubtype{\ltiEnv{}}{\ltiT{}}{\ltiS{}}, pronounced
``\ltiT{} is a subtype of \ltiS{}.''
\ltiSTVar says type variables are subtypes of themselves.
\ltiSTop and \ltiSBot establish \ltiTop and \ltiBot as maximal and
minimal types.
\ltiSRec says record types may forget or upcast their fields.
\ltiSFn relates types contravariantly to the left of an arrow
and covariantly to the right.

\subsection{External language}

\begin{figure}
$$
\begin{array}{lrll}
  \ltiE{}, \ltiF{} &::=& ... \alt \ltiufun{\ltivar{}}{\ltiE{}}
                         \alt \ltiapp{\ltiF{}}{\ltiE{}}
                      &\mbox{Terms}
\end{array}
$$
\caption{External Language Syntax
  (extends \figref{symbolic:figure:internal-language})
  }
\label{symbolic:figure:external-language-syntax}
\end{figure}

The syntax for the external language
\figref{symbolic:figure:external-language-syntax}
is a superset of the internal language, with unannotated functions 
\ltiufun{\ltivar{}}{\ltiE{}},
and ``lightweight'' applications with implicit type arguments
\ltiapp{\ltiF{}}{\ltiE{}}.

\begin{figure}
  \begin{mathpar}
    \infer [\ltiEAppInf]
    {
    \ltitjudgementNoElab{\ltiEnv{}}
                    {\ltiF{}}
                    {\ltiPolyFn{\ltiT{}}{\ova{\ltitvar{}}}{\ltiS{}}}
                    %{\ltiFp{}}
                    \\
    \ltitjudgementNoElab{\ltiEnv{}}
                    {\ltiE{}}
                    {\ltiSp{}}
                    %{\ltiEp{}}
                  \\
                       |\ova{\ltitvar{}}|>0
                  \\\\
                  \forall \ltiRp{}.
                    \left(
                    \begin{array}{lll}
                      \ltiisubtype{\ltiEnv{}}{\ltiSp{}}{\ltireplace{\ova{\ltiRp{}}}{\ova{\ltitvar{}}}{\ltiT{}}}
                      \text{ implies }
                      %\arcr
                      \ltiisubtype{\ltiEnv{}}{\ltireplace{\ova{\ltiR{}}}{\ova{\ltitvar{}}}{\ltiSp{}}}
                                   {\ltireplace{\ova{\ltiRp{}}}{\ova{\ltitvar{}}}{\ltiSp{}}}
                    \end{array}
                  \right)
    }
    {
    \ltitjudgementNoElab{\ltiEnv{}}
                    {\ltiapp{\ltiF{}}{\ltiE{}}}
                    {\ltireplace{\ova{\ltiR{}}}{\ova{\ltitvar{}}}{\ltiS{}}}
                    %{\ltiappinst{\ltiFp{}}
                    %            {\ova{\ltiR{}}}
                    %            {\ltiEp{}}}
    }

    \infer [\ltiEAppInfBot]
    {
    \ltitjudgementNoElab{\ltiEnv{}}
                  {\ltiF{}}
                  {\ltiBot}
                  \\\\
    \ltitjudgementNoElab{\ltiEnv{}}
                  {\ltiE{}}
                  {\ltiS{}}
    }
    {
    \ltitjudgementNoElab{\ltiEnv{}}
                  {\ltiapp{\ltiF{}}{\ltiE{}}}
                  {\ltiT{}}
    }

    \infer [\ltiEUAbs]
    { 
    \ltitjudgementNoElab{\ltiEnvConcat{\ltiEnv{}}
                                {\ltiEnvConcat{\ova{\ltitvar{}}}
                                              {\hastype{\ltivar{}}{\ltiT{}}}}}
                  {\ltiE{}}
                  {\ltiS{}}
                  \\\\
              \ova{\ltitvar{}} \cap \ltitv{\ltiE{}} = \varnothing
    }
    {
    \ltitjudgementNoElab{\ltiEnv{}}
                  {\ltiufun{\ltivar{}}{\ltiE{}}}
                  {\ltiPolyFn{\ltiT{}}{\ova{\ltitvar{}}}{\ltiS{}}}
                 }
  \end{mathpar}

  \caption{External Language Specification (extends 
  \figref{symbolic:figure:internal-language-type-system})
  }
  \label{symbolic:figure:external-language-declarative-type-system}
\end{figure}

The external language type system is (declaratively) specified in
\figref{symbolic:figure:external-language-declarative-type-system}
as a superset of the (algorithmic) internal language.
\ltiEAppInf says we must pick the most general type arguments when elaborating
an application without explicit type arguments.
This is identical to the corresponding rule in Local Type Inference.
\ltiEUAbs says that an untyped function must type check at the interface
chosen for its elaboration.
Since we also infer type parameters, the rule also requires the type variables chosen must
not capture type variables that occur free in the body of the function.
A similar rule is included in Colored Local Type Inference, except
colored types enforce that parameter types and type parameters only be inherited
from its surrounding context.
In constrast, our rule uses an oracle to synthesize both.
This is because our type inference algorithm based on symbolic closures
is not restricted to local reasoning.

\begin{figure}[h]
$$
\begin{array}{lrlll}
  \ltiFn{\ova{\ltiT{}}^n}{\ltiS{}} &\Leftrightarrow&
  \ltiFn{\ltiRec{\overrightarrowcaption{\hastype{\texttt{arg}i}{\ltiT{i}}}^{1 \leq i \leq n}}}{\ltiS{}}
  & \text{where } n \not= 1
                      &\mbox{Type abbreviations} \\
  \ltiufun{\ova{\ltivar{}}^n}{\ltiE{}} &\Leftrightarrow&
  \ltiufun{\ltivarp{}}{\ltireplaceoverrightarrowcaption{\ltisel{\ltivarp{}}{\texttt{arg}i}}{\ltivar{i}}{1 \leq i \leq n}{\ltiE{}}}
  & \text{where } n \not= 1, \ltivarp{} \not\in \ltifvLHS{\ltiE{}}
                      &\mbox{Term abbreviations}
  \\
  \ltifuntparamargtype{\ova{\ltitvar{}}}{\ova{\ltivar{}}^n}{\ova{\ltiT{}}^n}{\ltiE{}} &\Leftrightarrow&
  \ltifuntparamargtype{\ova{\ltitvar{}}}
                      {\ltivarp{}}
                      {\ltiRec{\overrightarrowcaption{\hastype{\texttt{arg}i}{\ltiT{i}}}^{1 \leq i \leq n}}}
                      {\\ && \ \ \ltireplaceoverrightarrowcaption{\ltisel{\ltivarp{}}{\texttt{arg}i}}{\ltivar{i}}{1 \leq i \leq n}{\ltiE{}}}
  & \text{where } n \not= 1, \ltivarp{} \not\in \ltifvLHS{\ltiE{}}
  \\
  \ltiapp{\ltiF{}}{\ova{\ltiE{}}^n} &\Leftrightarrow&
  \ltiapp{\ltiF{}}{\ltiRec{\overrightarrowcaption{\texttt{arg}i = \ltiE{i}}^{1 \leq i \leq n}}}
  & \text{where } n \not= 1
  \\
  \ltilet{\ova{\ltivar{}}}{\ova{\ltiE{}}}{\ltiF{}} &\Leftrightarrow& \ltiappParens{\ltiufun{\ova{\ltivar{}}}{\ltiF{}}}{\ova{\ltiE{}}}
  \\
  \ltifunargtype{\ova{\ltivar{}}}{\ova{\ltiT{}}}{\ltiE{}} &\Leftrightarrow&
  \ltifuntparamargtype{}{\ova{\ltivar{}}}{\ova{\ltiT{}}}{\ltiE{}}
  \\
  \ltifuntparamargrettype{\ova{\ltitvar{}}}{\ova{\ltivar{}}}{\ova{\ltiT{}}}{\ltiS{}}{\ltiE{}} &\Leftrightarrow&
  \ltifuntparamargtype{\ova{\ltitvar{}}}{\ova{\ltivar{}}}{\ova{\ltiT{}}}{\ltianncolon{\ltiE{}}{\ltiS{}}}
  \\
  \ltianncolon{\ltiE{}}{\ltiS{}} &\Leftrightarrow&
  \ltiappParens{\ltifunargtype{\ltivar{}}{\ltiS{}}{\ltivar{}}}{\ltiE{}}
  \\
  \ltifunargrettype{\ova{\ltivar{}}}{\ova{\ltiT{}}}{\ltiS{}}{\ltiE{}} &\Leftrightarrow&
  \ltifuntparamargrettype{}{\ova{\ltivar{}}}{\ova{\ltiT{}}}{\ltiS{}}{\ltiE{}}
\end{array}
$$
\caption{External Language Syntax abbreviations
  }
\label{symbolic:figure:external-language-syntax-abbreviations}
\end{figure}

We use several syntax abbreviations, enumerated in
\figref{symbolic:figure:external-language-syntax-abbreviations}.
The first four abbreviations use record terms and types to represent multi-parameter functions
and applications.
For example, the function type
\ltiFn{\ltiT{1},\ltiT{2}}{\ltiS{}}
stands for
\ltiFn{\ltiRec{\hastype{\texttt{arg1}}{\ltiT{1}}, {\hastype{\texttt{arg2}}{\ltiT{2}}}}}{\ltiS{}},
the function term
\ltiufun{\text{x}, \text{y}}{\ltiE{}}
stands for 
\ltiufun{\ltivarp{}}{\ltireplaceSingle{\ltireplaceentry{\ltisel{\ltivarp{}}{\texttt{arg1}}}{\text{x}},
                                       \ltireplaceentry{\ltisel{\ltivarp{}}{\texttt{arg2}}}{\text{y}}}
                                      {\ltiE{}}}
(where {\ltivarp{}} does not occur free in {\ltiE{}})
and the application term
\ltiapp{\ltiF{}}{\ltiE{1},\ltiE{2}}
stands for
  \ltiapp{\ltiF{}}{\ltiRec{\texttt{arg1} = \ltiE{1},{\texttt{arg2} = \ltiE{2}}}}.
Symbolic closures permit the lambda-encoding of let, so 
\ltilet{\ltivar{}}{\ltiE{}}{\ltiF{}}
stands for 
\ltiappParens{\ltiufun{\ltivar{}}{\ltiF{}}}{\ltiE{}}.
We allow the type parameters to be omitted from function terms if they are empty.
Type ascription
\ltianncolon{\ltiE{}}{\ltiS{}}
stands for
\ltiappParens{\ltifunargtype{\ltivar{}}{\ltiS{}}{\ltivar{}}}{\ltiE{}},
which we use to represent return types for functions
\ltifuntparamargrettype{\ltitvar{}}{\ltivar{}}{\ltiT{}}{\ltiS{}}{\ltiE{}}
as
\ltifuntparamargtype{\ltitvar{}}{\ltivar{}}{\ltiT{}}{\ltianncolon{\ltiE{}}{\ltiS{}}}.

\subsection{Type Inference Algorithm}

We now define a type inference algorithm based on symbolic closures
that recovers types from terms written in the external language.
First, we give the syntax for symbolic closures, then we describe
the organization of type inference, and finally fill in the missing details.

%In order to reliably elaborate to \ltiFsub, we model a restriction of symbolic closure types.
%
%First, the elaborated type of a symbolic closure is chosen greedily,
%when it is first symbolically executed, and each symbolic closure
%must be exercised at least once to elaborate its body.
%A polymorphic type may also be chosen, but the type arguments bound by
%the function must also be chosen at this time.
%This is for several reasons.
%Most obviously, we lack
%intersection types, and so have no natural way of enumerating more than one interface
%type for a function, with each interface
%corresponding to a symbolic execution.
%Adding intersection types would require further considerations, namely
%the machinery needed to annotate a function that is ascribed many interfaces is quite involved
%(e.g., branching types~\cite{wells2002branching},
%contextual subtyping~\cite{Dunfield2004Tridirectional},
%and ``lazy'' type substitutions~\cite{polyduce1})
%and obscures the (orthogonal) idea of symbolic closures, which we hope to present
%in its essense.
%We might also attempt to infer a single polymorphic function type that combines all interfaces,
%however that is a separate problem we have not attempted.
%
%Second, we restrict a symbolic closure to the type-variable scope in which it was defined.
%Relaxing this restriction raises questions about scoping also solved by
%contextual subtyping~\cite{Dunfield2004Tridirectional}.
%Alternatively, we could quantify over out-of-scope variables
%with a polymorphic type, but this would most likely require also
%inferring type arguments for arguments~\cite{polyduce1},
%which we do not cover here.
%
%Third, we disallow a symbolic closure type to be passed to themselves.
%We must fully erase symbolic closure types to elaborate to \ltiFsub,
%however we do not model the equi-recursive type binders that are the 
%natural encoding for such types. On the other hand, since it does not require
%recursive types to elaborate, symbolic closures
%be passed to \emph{other} symbolic closures, even if the former elaborates to a polymorphic
%type (since \ltiFsub is impredicative, i.e., does not restrict the places a polymorphic type
%may occur).
%
%Fourth, we add a global symbolic reduction limit, called ``fuel'', to make
%type inference decidable.
%Since a symbolic closure may be symbolically executed an unbounded
%number of times, this restriction is also useful for practical implementations
%using symbolic closures.
%
%As we will see, symbolic closures are useful even with these restrictions.

\begin{figure}
$$
\begin{array}{lrll}
  \ltiE{}, \ltiF{} &::=& ... \alt
                         \ltiufunelab{\ltiufunelabentry{\ltiClosureID{}}}
                                     {\ltivar{}}
                                     {\ltiE{}}
                      &\mbox{Terms} \\
  \ltiT{}, \ltiS{}, \ltiR{} &::=& ... \alt \ltiClosureWithStkID{\ltiEnv{}}{\ltiClosureID{}}{\ltiufun{\ltivar{}}{\ltiE{}}}
                      &\mbox{Types} \\
  \ltiClosureID{} &::=& \ltitvar{}
                      &\mbox{Symbolic Closure Identifiers} \\
  \ltiFuel{} &::=& \ltinat{}
                      &\mbox{Symbolic Reduction Fuel} \\
  \ltiClosureCache{} &::=& \ova{\ltiClosureCacheEntry
                                {\ltiClosureID{}}
                                {\ltiClosure{\ltiEnv{}}{\ltiE{}}}}
                      &\mbox{Elaboration Caches}
                      \\
  \ltiCombinedThreadedEnv{} &::=& \ltimakeCombinedThreadedEnv{\ltiFuel{}}{\ltiClosureCache{}}
                      &\mbox{Threaded Environments}
\end{array}
$$
\caption{Symbolic Closure Language (SCL) Syntax (extends \figref{symbolic:figure:external-language-syntax})}
\label{symbolic:figure:SC-language-syntax}
\end{figure}

The syntax for the Symbolic Closure Language (SCL)
is given in 
\figref{symbolic:figure:SC-language-syntax}, which
is a superset of the external language syntax.
We introduce a new term and type, both which act as
a sort of placeholder for an explicitly typed function term or type (respectively)
which will be filled in after type checking.
The term
\ltiufunelab{\ltiufunelabentry{\ltiClosureID{}}}
            {\ltivar{}}
            {\ltiE{}}
is a tagged function,
which says the unannotated function 
\ltiufun{\ltivar{}}{\ltiE{}}
was assigned the symbolic closure type identified by
\ltiClosureID{}.
A symbolic closure type
\ltiClosureWithStkID{\ltiEnv{}}{\ltiClosureID{}}{\ltiufun{\ltivar{}}{\ltiE{}}},
then,
says the unannotated function term \ltiufun{\ltivar{}}{\ltiE{}}
is closed under definition type context \ltiEnv{}, with identifier \ltiClosureID{}.
We say \ltiufun{\ltivar{}}{\ltiE{}} is \emph{closed} because all free type and term variables in \ltiufun{\ltivar{}}{\ltiE{}}
are bound by \ltiEnv{}.
In terms of the internal language,
a loose first-intuition of a symbolic closure type's meaning
is a function type
\ltiPolyFn{\ltiT{}}{\ova{\ltitvar{}}}{\ltiS{}}
where
\ltitjudgementNoElab{\ltiEnvConcat{\ltiEnv{}}
                    {\ltiEnvConcat{\ova{\ltitvar{}}}
                                  {\hastype{\ltivar{}}{\ltiT{}}}}}
                    {\ltiE{}}
                    {\ltiS{}}.
This analogy is inadequate because (in small part) \ltiEnv{}, \ltiT{}, \ltiE{}, and \ltiS{}
can contain SCL types and terms.

An important part of using SCL for type inference is making the meaning of
symbolic closure types explicit by replacing them with concrete internal types and terms.
To this end, the remaining syntax is in service to the bookkeeping necessary to decide
how to achieve this.
To prevent infinite loops when checking symbolic closures, we introduce
symbolic reduction fuel \ltiFuel{}, a natural number that represents
the remaining number of symbolic reductions allowed.
The elaboration of an unannotated function checked with symbolic closures could
be determined at any time, so an elaboration cache \ltiClosureCache{} is maintained 
that associates a symbolic closure \ltiClosureID{}
with its scoped elaboration \ltiClosure{\ltiEnv{}}{\ltiE{}}, which says
SCL term \ltiE{} is closed under SCL type environment \ltiEnv{}.
For convienience, we use threaded environments \ltiCombinedThreadedEnv{} to stand for
the pair \ltimakeCombinedThreadedEnv{\ltiFuel{}}{\ltiClosureCache{}}.

We now describe the organization of type inference.
The typing judgment for SCL is written
    \ltitSstkjudgement{\ltimakeCombinedThreadedEnv{\ltiFuel{}}{\ltiClosureCache{}}}
                      {\ltiEnv{}}
                      {\ltiE{}}
                      {\ltiT{}}
                      {\ltimakeCombinedThreadedEnv{\ltiFuelp{}}{\ltiClosureCachep{}}}
                      {\ltiEp{}}
                      and
says with initial fuel \ltiFuel{} and elaboration cache \ltiClosureCache{},
external term \ltiE{} is of SCL type \ltiT{}
in SCL context \ltiEnv{}, elaborating to SCL term \ltiEp{} with 
updated fuel \ltiFuelp{} and elaboration cache \ltiClosureCachep{}.
It performs a depth-first traversal of the syntax tree.

\begin{figure}[h]
  \begin{mathpar}
  \infer[Infer]
  {
    \exists \ltiFuel{}.
     \ltitSstkjudgement{\ltimakeCombinedThreadedEnv{\ltiFuel{}}{\ltiEmptyClosureCache}}
                       {\ltiEnv{}}
                       {\ltiE{}}
                       {\ltiS{}}
                       {\ltimakeCombinedThreadedEnv{\ltiFuelp{}}{\ltiClosureCache{}}}
                       {\ltiEp{}}
                       \\
                  \ltielimClosT{\varnothing}{\ltiClosureCache{}}{\ltiS{}}{\ltiT{}}
                  \\
                  \ltielimClos{\ltiClosureCache{}}{\ltiEp{}}{\ltiF{}}
  }
  {
    \ltiinferTL{\ltiEnv{}}{\ltiE{}}{\ltiT{}}{\ltiF{}}
  }
  \end{mathpar}
\end{figure}

The top-level driver for type inference
\ltiinferTL{\ltiEnv{}}{\hastype{\ltiE{}}{\ltiT{}}}{\ltiF{}}, presented above,
says external term \ltiE{} has internal type \ltiT{}
in external environment \ltiEnv{}, with internal elaboration \ltiF{}.
It requires an initial fuel \ltiFuel{} to be provided to SCL, 
and then uses the output elaboration cache \ltiClosureCache{}
to erase SCL terms and types using the metafunctions \ltielimClossymbol
and \ltielimClosTsymbol.
Next, we present the SCL type system and subtyping
%(except for the SCL implementation of \ltiEAppInf, described in \chapref{chapter:symbolic:directed-lti}),
then provide definitions for the elaboration metafunctions.

\begin{figure}
  \begin{mathpar}
    \boxed
    {
    \infer[]
    {}
    {
    \ltitSstkjudgement{\ltiCombinedThreadedEnv{}}
                      {\ltiEnv{}}
                      {\ltiE{}}
                      {\ltiT{}}
                      {\ltiCombinedThreadedEnvp{}}
                      {\ltiEp{}}
                     \\\\
                     \text{Given symbolic closure environment \ltiCombinedThreadedEnv{}
                     and SCL context \ltiEnv{}, external term \ltiE{}
                     has SCL type \ltiT{}
                     }
                     \\\\
                     \text{in environment \ltiCombinedThreadedEnvp{},
                     with SCL elaboration \ltiEp{} (omitted when obvious from subderivations).
                     }
                     }
                     }

    \begin{array}{c}
    \infer [\ltiSCVar]
    {}
    {
    \ltitSstkjudgementNoElab{\ltiCombinedThreadedEnv{}}
                      {\ltiEnv{}}
                      {\ltivar{}}
                      {\ltiEnvLookup{\ltiEnv{}}{\ltivar{}}}
                      {\ltiCombinedThreadedEnv{}}
                      {\ltivar{}}
                 }
\ \ \ 
    \infer [\ltiSCRec]
    {
    \overrightarrowcaption{
    \ltitSstkjudgementNoElab{\ltiCombinedThreadedEnv{i-1}}
                      {\ltiEnv{}}
                      {\ltiF{i}}
                      {\ltiT{i}}
                      {\ltiCombinedThreadedEnv{i}}
                      {\ltiFp{i}}
                      }^{ 1 \leq i \leq n}
    }
    {
    \ltitSstkjudgementNoElab{\ltiCombinedThreadedEnv{0}}
                      {\ltiEnv{}}
                      {\ltiRec{\ova{\ltivar{} = \ltiF{}}^n}}
                      {\ltiRec{\ova{\hastype{\ltivar{}}{\ltiT{}}}^n}}
                      {\ltiCombinedThreadedEnv{n}}
                      {\ltiRec{\ova{\ltivar{} = \ltiFp{}}^n}}
    }

                 \\\\
    \infer [\ltiSCSel]
    {
    \ltitSstkjudgementNoElab{\ltiCombinedThreadedEnv{}}
                      {\ltiEnv{}}
                      {\ltiF{}}
                      {\ltiRec{\hastype{\ltivar{1}}{\ltiT{1}},..., \hastype{\ltivar{i}}{\ltiT{i}},..., \hastype{\ltivar{n}}{\ltiT{n}}}}
                      {\ltiCombinedThreadedEnvp{}}
                      {\ltiFp{}}
    }
    {
    \ltitSstkjudgementNoElab{\ltiCombinedThreadedEnv{}}
                      {\ltiEnv{}}
                      {\ltisel{\ltiF{}}{\ltivar{i}}}
                      {\ltiT{i}}
                      {\ltiCombinedThreadedEnvp{}}
                      {\ltisel{\ltiFp{}}{\ltivar{i}}}
    }
    \end{array}

    \begin{array}{ll}
    \infer [\ltiSCAbs]
    {
    \left(
    \begin{array}{llll}
      \text{$|\ova{\ltitvar{}}|>0$ implies \ltiEnv{} and \ltiS{}}
                     \arcr
                     \text{contain no symbolic closures}
    \end{array}
    \right)
                     \\\\
     \ltitSstkjudgementNoElab{\ltiCombinedThreadedEnv{}}
                    {\ltiEnvConcat{\ltiEnv{}}
                                   {\ltiEnvConcat{\ova{\ltitvar{}}}
                                                 {\hastype{\ltivar{}}
                                                          {\ltiT{}}}}}
                     {\ltiE{}}
                     {\ltiS{}}
                     {\ltiCombinedThreadedEnvp{}}
                     {\ltiEp{}}
    }
    {
    \ltitSstkjudgementNoElab{\ltiCombinedThreadedEnv{}}
                    {\ltiEnv{}}
                    {\ltifuntparamargtype{\ova{\ltitvar{}}}
                                           {\ltivar{}}
                                           {\ltiT{}}
                                           {\ltiE{}}}
                    {\ltiPolyFn{\ltiT{}}{\ova{\ltitvar{}}}{\ltiS{}}}
                    {\ltiCombinedThreadedEnvp{}}
                    {\ltifuntparaminterface{\ova{\ltitvar{}}}
                                           {\ltiFn{\ltiT{}}{\ltiS{}}}
                                           {\ltivar{}}
                                           {\ltiEp{}}}
                 }
    \end{array}

    \infer [\ltiSCAppInst]
    {
    \ltitSstkjudgementNoElab{\ltiCombinedThreadedEnv{1}}
                      {\ltiEnv{}}
                      {\ltiF{}}
                      {\ltiPolyFn{\ltiT{}}{\ova{\ltitvar{}}}{\ltiS{}}}
                      {\ltiCombinedThreadedEnv{2}}
                      {\ltiFp{}}
                  \\\\
    \ltitSstkjudgementNoElab{\ltiCombinedThreadedEnv{2}}
                      {\ltiEnv{}}
                      {\ltiE{}}
                      {\ltiTp{}}
                      {\ltiCombinedThreadedEnv{3}}
                      {\ltiEp{}}
    \\\\
                       \ltiSsubtype{\ltiCombinedThreadedEnv{3}}
                                   {\ltiEnv{}}
                                   {\ltiTp{}}
                                   {\ltireplace{\ova{\ltiR{}}}{\ova{\ltitvar{}}}{\ltiT{}}}
                                   {\ltiCombinedThreadedEnv{4}}
    }
    {
    \ltitSstkjudgementNoElab{\ltiCombinedThreadedEnv{1}}
                      {\ltiEnv{}}
                      {\ltiappinst{\ltiF{}}
                                  {\ova{\ltiR{}}}
                                  {\ltiE{}}}
                      {\ltireplace{\ova{\ltiR{}}}{\ova{\ltitvar{}}}{\ltiS{}}}
                      {\ltiCombinedThreadedEnv{4}}
                      {\ltiappinst{\ltiFp{}}
                                  {\ova{\ltiR{}}}
                                  {\ltiEp{}}}
    }
                 \ \ \ \ \ 
                 %
    \infer [\ltiSCAppInstBot]
    {
    \ltitSstkjudgementNoElab{\ltiCombinedThreadedEnv{}}
                      {\ltiEnv{}}
                      {\ltiF{}}
                      {\ltiBot}
                      {\ltiCombinedThreadedEnvpp{}}
                      {\ltiFp{}}
                  \\\\
    \ltitSstkjudgementNoElab{\ltiCombinedThreadedEnvpp{}}
                      {\ltiEnv{}}
                      {\ltiE{}}
                      {\ltiS{}}
                      {\ltiCombinedThreadedEnvp{}}
                      {\ltiEp{}}
    }
    {
    \ltitSstkjudgementNoElab{\ltiCombinedThreadedEnv{}}
                      {\ltiEnv{}}
                      {\ltiappinst{\ltiF{}}
                                  {\ova{\ltiR{}}}
                                  {\ltiE{}}}
                      {\ltiT{}}
                      {\ltiCombinedThreadedEnvp{}}
                      {\ltiappinst{\ltiFp{}}
                                  {\ova{\ltiR{}}}
                                  {\ltiEp{}}}
    }
                 \ \ \ \ \ 
                 %
    \infer [\ltiSCSelBot]
    {
    \ltitSstkjudgementNoElab{\ltiCombinedThreadedEnv{}}
                      {\ltiEnv{}}
                      {\ltiF{}}
                      {\ltiBot}
                      {\ltiCombinedThreadedEnvp{}}
                      {\ltiFp{}}
    }
    {
    \ltitSstkjudgementNoElab{\ltiCombinedThreadedEnv{}}
                      {\ltiEnv{}}
                      {\ltisel{\ltiF{}}{\ltivar{}}}
                      {\ltiBot}
                      {\ltiCombinedThreadedEnvp{}}
                      {\ltisel{\ltiFp{}}{\ltivar{}}}
    }
                 \ \ \ \ \ 
                 %
    \infer [\ltiSCAppInfBot]
    {
    \ltitSstkjudgementNoElab{\ltiCombinedThreadedEnv{}}
                      {\ltiEnv{}}
                      {\ltiF{}}
                      {\ltiBot}
                      {\ltiCombinedThreadedEnvpp{}}
                      {\ltiFp{}}
                  \\\\
    \ltitSstkjudgementNoElab{\ltiCombinedThreadedEnvpp{}}
                      {\ltiEnv{}}
                      {\ltiE{}}
                      {\ltiS{}}
                      {\ltiCombinedThreadedEnvp{}}
                      {\ltiEp{}}
    }
    {
    \ltitSstkjudgementNoElab{\ltiCombinedThreadedEnv{}}
                      {\ltiEnv{}}
                      {\ltiapp{\ltiF{}}{\ltiE{}}}
                      {\ltiBot}
                      {\ltiCombinedThreadedEnvp{}}
                      {\ltiappinst{\ltiFp{}}
                                  {}
                                  {\ltiEp{}}}
    }

    \infer[\ltiSCAppInfPT]
    {
    \ltitSstkjudgementNoElab{\ltiCombinedThreadedEnv{1}}
                      {\ltiEnv{}}
                      {\ltiF{}}
                      {\ltiPoly{\ova{\ltitvar{}}}
                               {\ltiFn{\ltiT{}}{\ltiS{}}}}
                      {\ltiCombinedThreadedEnv{2}}
                      {\ltiFp{}}
                  \\
    \ltitSstkjudgementNoElab{\ltiCombinedThreadedEnv{2}}
                      {\ltiEnv{}}
                      {\ltiE{}}
                      {\ltiTp{}}
                      {\ltiCombinedThreadedEnv{3}}
                      {\ltiEp{}}
                      \\
      \ltiT{}, \ltiS{}, \ltiTp{}
                     \text{ contain no symbolic closures}
                  \\
                       |\ova{\ltitvar{}}|>0
           \\
           \ltigenconstraint{\varnothing}{\ova{\ltitvar{}}}{\ltiTp{}}{\ltiT{}}{\ltiC{}}
           \\
           \ltiSubst{\ltiC{}}{\ltiFn{\ltiT{}}{\ltiS{}}}{\ltisubst{}}
    }
    {
    \ltitSstkjudgementNoElab{\ltiCombinedThreadedEnv{1}}
                      {\ltiEnv{}}
                      {\ltiapp{\ltiF{}}{\ltiE{}}}
                      {\ltiApplySubst{\ltisubst{}}{\ltiS{}}}
                      {\ltiCombinedThreadedEnv{3}}
                      {\ltiappinst{\ltiFp{}}
                                  {\ova{\ltiApplySubst{\ltisubst{}}
                                                      {\ltitvar{}}}}
                                  {\ltiEp{}}}
    }


    \infer [\ltiSCUAbs]
    {
    \ltiCombinedThreadedEnv{} = \ltimakeCombinedThreadedEnv{\ltiFuel{}}{\ltiClosureCache{}}
    \\
    \ltiClosureID{} \not\in dom(\ltiClosureCache{})
    \\\\
    \ltiCombinedThreadedEnvp{}
    =
    \ltimakeCombinedThreadedEnv{\ltiFuel{}}
    {\ltimapsto{\ltiClosureCache{}}
               {\ltiClosureID{}}
               {\ltiClosure{\ltiEnv{}}{\ltiufun{\ltivar{}}{\ltiE{}}}}}
    }
    {
    \ltitSstkjudgement{\ltiCombinedThreadedEnv{}}
                      {\ltiEnv{}}
                      {\ltiufun{\ltivar{}}{\ltiE{}}}
                      {\ltiClosureWithStkID{\ltiEnv{}}
                                           {\ltiClosureID{}}
                                           {\ltiufun{\ltivar{}}{\ltiE{}}}}
                      {\ltiCombinedThreadedEnvp{}}
                      {\ltiufunelab{\ltiClosureID{}}
                                   {\ltivar{}}
                                   {\ltiE{}}}
                 }
    \ \ \ \ 
%
    \infer [\ltiSCAppInfClosure]
    {
    \ltitSstkjudgement{\ltiCombinedThreadedEnv{1}}
                      {\ltiEnv{}}
                      {\ltiF{}}
                      {\ltiClosureWithStkID{\ltiEnvp{}}
                                           {\ltiClosureID{}}
                                           {\ltiufun{\ltivar{}}{\ltiEp{}}}}
                      {\ltiCombinedThreadedEnv{2}}
                      {\ltiFp{}}
                  \\
    \ltitSstkjudgement{\ltiCombinedThreadedEnv{2}}
                      {\ltiEnv{}}
                      {\ltiE{}}
                      {\ltiT{}}
                      {\ltimakeCombinedThreadedEnv{\ltiFuel{3}}{\ltiClosureCache{3}}}
                      {\ltiEpp{}}
                  \\\\
    0 < \ltiFuel{3}
    \\
    \ltitSstkjudgement{\ltimakeCombinedThreadedEnv
                       {\ltiFuel{3}-1}
                       {\ltiClosureCache{3}}}
                      {\ltiEnvConcat{\ltiEnvp{}}{\hastype{\ltivar{}}{\ltiT{}}}}
                      {\ltiEp{}}
                      {\ltiS{}}
                      {\ltimakeCombinedThreadedEnv{\ltiFuel{4}}{\ltiClosureCache{4}}}
                      {\ltiFpp{}}
    }
    {
    \ltitSstkjudgement{\ltiCombinedThreadedEnv{1}}
                      {\ltiEnv{}}
                      {\ltiapp{\ltiF{}}{\ltiE{}}}
                      {\ltiS{}}
                      {\ltimakeCombinedThreadedEnv{\ltiFuel{4}}
                          {\ltiupdateClosureCacheSingleLHS{\ltiClosureCache{4}}
                                {\ltiClosureID{}}
                                {\ltifuntparamargrettype
                                 {}
                                 {\ltivar{}}
                                 {\ltiT{}}
                                 {\ltiS{}}
                                 {\ltiFpp{}}}}}
                      {\ltiappinst{\ltiFp{}}
                                  {}
                                  {\ltiEpp{}}}
    }
  \end{mathpar}
  \caption{Type inference algorithm% (\textsc{AppInf} omitted)
  }
  \label{symbolic:figure:SC-language-algorithmic-type-system}
\end{figure}

The SCL type system is given in \figref{symbolic:figure:SC-language-algorithmic-type-system}.
We abbreviate
    \ltitSstkjudgement{\ltimakeCombinedThreadedEnv{\ltiFuel{}}{\ltiClosureCache{}}}
                      {\ltiEnv{}}
                      {\ltiE{}}
                      {\ltiT{}}
                      {\ltimakeCombinedThreadedEnv{\ltiFuelp{}}{\ltiClosureCachep{}}}
                      {\ltiEp{}}
                      as
    \ltitSstkjudgement{\ltiCombinedThreadedEnv{}}
                      {\ltiEnv{}}
                      {\ltiE{}}
                      {\ltiT{}}
                      {\ltiCombinedThreadedEnvp{}}
                      {\ltiEp{}}
                      where
${\ltiCombinedThreadedEnv{}} = {\ltimakeCombinedThreadedEnv{\ltiFuel{}}{\ltiClosureCache{}}}$
and 
                      ${\ltiCombinedThreadedEnvp{}} = {\ltimakeCombinedThreadedEnv{\ltiFuelp{}}{\ltiClosureCachep{}}}$.
Further, we sometimes omit the elaborated term as
    \ltitSstkjudgementNoElab{\ltiCombinedThreadedEnv{}}
                      {\ltiEnv{}}
                      {\ltiE{}}
                      {\ltiT{}}
                      {\ltiCombinedThreadedEnvp{}}
                      {\ltiEp{}},
                      but only when
                      {\ltiEp{}}
                      can be obviously derived from subderivations.
The first seven rules correspond to the internal language type system,
straightfowardly extended with threaded environments.
The extra condition in \ltiSCAbs
helps ensure a symbolic closure type only reasons about type variables
in its definition scope.
The first rule for lightweight applications \ltiSCAppInfBot implements \ltiEAppInfBot.
The \ltiSCAppInfPT rule uses Pierce and Turner's type argument synthesis
algorithm~\cite{PierceLTI} off-the-shelf. Since it does not handle them,
we ensure that it cannot be fed a symbolic closure.

The remaining rules are more interesting.
The symbolic closure introduction rule
\ltiSCUAbs creates a symbolic closure type with a fresh identifier \ltiClosureID{}
for the unannotated function term \ltiufun{\ltivar{}}{\ltiE{}}.
The return type is symbolic closure type
                       \ltiClosureWithStkID{\ltiEnv{}}
                                           {\ltiClosureID{}}
                                           {\ltiufun{\ltivar{}}{\ltiE{}}},
which holds enough information to both check its body at some later time
and link its elaboration to the originating term.
The elaboration cache entry for \ltiClosureID{} is initialized 
with an unannotated function term, signifying that the body has yet to be type checked,
and is passed on as part of \ltiCombinedThreadedEnvp{}.
Finally, the elaboration
                      {\ltiufunelab{\ltiClosureID{}}
                                   {\ltivar{}}
                                   {\ltiE{}}}
tags the original term with its symbolic closure identifier \ltiClosureID{}.
Intuitively, this rule is sound because, from the type checker's perspective,
there is no witness (yet) to \ltiufun{\ltivar{}}{\ltiE{}}
being called, and so there is no opportunity to ``get stuck'' or
``go wrong''. 
The next rule handles one kind of witness to its invocation: application.

The application rule for symbolic closures
\ltiSCAppInfClosure
checks term \ltiapp{\ltiF{}}{\ltiE{}}
where \ltiF{} has symbolic closure type
                      {\ltiClosureWithStkID{\ltiEnvp{}}
                                           {\ltiClosureID{}}
                                           {\ltiufun{\ltivar{}}{\ltiEp{}}}}
                                           and 
\ltiE{} has type \ltiT{}.
As with the normal application rule,
we must ensure \ltiF{}'s domain is permissive enough to be
applied to terms of type \ltiT{}, which
we verify with a symbolic reduction.
After consuming fuel, the final premise checks \ltiClosureID{}'s
function body \ltiEp{} in its definition context \ltiEnvp{},
extended with \hastype{\ltivar{}}{\ltiT{}} (to account for
\ltiE{} of type \ltiT{} being passed as an argument), giving result type \ltiS{}
and elaboration \ltiFpp{}.
The type of the entire application is simply \ltiS{}, since
the condition in \ltiSCAbs (and in other rules, given later) ensure that
\ltiEnv{} and \ltiEnvp{} share the same type variable scope---there
is no opportunity for \ltiS{} to introduce an out-of-scope type variable.
We now pick the elaboration for \ltiClosureID{}
to be
{\ltifuntparamargrettype
                                 {}
                                 {\ltivar{}}
                                 {\ltiT{}}
                                 {\ltiS{}}
                                 {\ltiFpp{}}}
(an abbreviation for a function with return type \ltiS{}, defined in
\figref{symbolic:figure:external-language-syntax-abbreviations})
using the \ltiupdateClosureCacheSinglesymbol metafunction.
Since we choose \ltiClosureID{} to be monomorphic, it does
not require type arguments
and so
the entire application elaborates to
                      {\ltiappinst{\ltiFp{}}
                                  {}
                                  {\ltiEpp{}}}.
A symbolic closure's elaboration may be picked exactly once,
so it is an error for any past or future elaborations decide
on different numbers of type arguments.
We elide the almost-identical rule \ltiSCAppClosure for applying a symbolic closure
with explicit type arguments,
since it can only be of the form \ltiappinst{\ltiF{}}{}{\ltiE{}}.

To illustrate how \ltiSCUAbs and \ltiSCAppInfClosure interact,
we give simplified definitions for each, using judgments resembling the internal language
and a simplified symbolic closure
type \ltiClosure{\ltiEnv{}}{\ltiufun{\ltivar{}}{\ltiF{}}} that omits identifiers.
\ltiSimpUAbs immediately packages up a function with its environment in a symbolic closure type.
\ltiSimpAppInfClosure unpacks the symbolic closure, and checks its body in a context extended
with the argument's type.
Highlighting is used to convey how a symbolic closure is assembled, disassembled, and checked.

\begin{mathpar}
  \inferrule*[lab=\boxed{\ltiSimpUAbs}]
   {}
   {
   \ltitSstkjudgementNoElabCombined{\varnothing}
                                   {\colorbox{pink}{\ltiEnv{}}}
                                   {\colorbox{pink}{\ltiufun{\ltivar{}}{\ltiF{}}}}
                                   {\colorbox{pink}{
                                           {\ltiClosure{\ltiEnv{}}{\ltiufun{\ltivar{}}{\ltiF{}}}}}}
                                   {\ltiCombinedThreadedEnv{}}
                                   {\ltiufun{\ltivar{}}{\ltiF{}}}
   }

  \inferrule*[lab=\boxed{\ltiSimpAppInfClosure}]
  {
   \ltitSstkjudgementNoElabCombined{\varnothing}
                      {\ltiEnvp{}}
                      {\ltiE{1}}
                      {\colorbox{pink}{\ltiClosure{{\ltiEnv{}}}
                                           {\ltiufun{{\ltivar{}}}
                                                    {\ltiF{}}}}}
                      {\ltiClosureCache{}}
                      {\ltiufunelab{\ltiInferred{\text{c1}}}{\text{x}}{\text{x}}}
   \\
   \ltitSstkjudgementNoElabCombined{\ltiClosureCache{}}
                      {\ltiEnvp{}}
                      {\ltiE{2}}
                      {{\colorbox{pink}{\ltiS{}}}}
                      {\ltiClosureCache{}}
                      {\ltiEp{}}
   \\
    \ltitSstkjudgementNoElabCombined{\ltiClosureCache{}}
                      {\ltiEnvConcat{\colorbox{pink}{\ltiEnv{}}
                                     }
                                    {\hastypesmall{\colorbox{pink}{\ltivar{}}}
                                                  {{\colorbox{pink}{\ltiS{}}}}}}
                      {\colorbox{pink}{\ltiF{}}}
                      {\colorbox{pink}{\ltiT{}}}
                      {\ltiClosureCache{}}
                      {\ltiFp{}}
  }
  {
    \ltitSstkjudgementNoElabCombined{\ltiCombinedThreadedEnv{}}
                      {\ltiEnvp{}}
                      {\ltiapp{\ltiE{1}}
                              {\ltiE{2}}}
                      {\colorbox{pink}{\ltiT{}}}
                      {\ltiCombinedThreadedEnvp{}}
                      {\ltiapp{\ltiFp{}}{\ltiEp{}}}
  }
\end{mathpar}

Using these rules, we illustrate inferring the \emph{obfuscated} term
                      ``{\ltilet{\text{g}}{\ltiRec{\text{d}={{\ltiufun{\ltivar{}}{\ltiF{}}}}}}
                              {\ltiapp{\ltisel{\text{g}}{\text{d}}}{\ltiE{}}}}'',
which is like the \emph{unobfuscated} term ``{\ltiappParens{\ltiufun{\ltivar{}}{\ltiF{}}}{\ltiE{}}}'',
except the function is briefly stored in a let-bound record.
Bidirectional propagation of types is insufficient to recover the type of {\ltivar{}},
however symbolic closures provide the necessary indirection to successfully infer the term.
To simplify its presentation, the following derivation uses the standard explicit typing rule for let,
where the left premise checks the bound term and the right premise checks the body in an extended context---we use 
the lambda encoding of lets everywhere else (\figref{symbolic:figure:external-language-syntax-abbreviations}).
We also liberally elide \ltiEnv{} (when uninteresting)
and the subderivation trees of \ltiSimpAppInfClosure.

\begin{mathpar}
  \inferrule*[]
  {
  \inferrule*
  {
  \inferrule*[lab=\boxed{\ltiSimpUAbs}]
   {}
   {
   \ltitSstkjudgementNoElabCombined{\varnothing}
                                   {\colorbox{pink}{\ltiEnv{}}^{\tikz[overlay,remember picture] \node [] (c1) {};}}
                                   {{\tikz[overlay,remember picture] \node [] (c) {};}\colorbox{pink}{\ltiufun{\ltivar{}}{\ltiF{}}}}
                                   {\colorbox{pink}{
                                           {\ltiClosure{\text{}^{\tikz[overlay,remember picture] \node [] (d1) {};}\ltiEnv{}}
                                                                {{\tikz[overlay,remember picture] \node [] (d2) {};}\ltiufun{\ltivar{}}{\ltiF{}}}}}
                                           {\tikz[overlay,remember picture] \node [] (d) {};}}
                                   {\ltiCombinedThreadedEnv{}}
                                   {\ltiufun{\ltivar{}}{\ltiF{}}}
   }
   }
   {
   \ltitSstkjudgementJustType{\varnothing}
                                   {...}
                                   {\ltiRec{\text{d}={{\tikz[overlay,remember picture] \node [] (b) {};}\colorbox{pink}{\ltiufun{\ltivar{}}{\ltiF{}}}}}}
                                   {\ltiRec{\hastype{\text{d}}
                                           {\colorbox{pink}
                                            {\ltiClosure{\ltiEnv{}}
                                                                {\ltiufun{\ltivar{}}
                                                                         {\ltiF{}}}}
                                                                         ^
                                                                         {\tikz[overlay,remember picture] \node [] (e) {};}
                                                                         _
                                                                         {\tikz[overlay,remember picture] \node [] (e1) {};}}}}
                                   {\ltiCombinedThreadedEnv{}}
                                   {\ltiufun{\ltivar{}}{\ltiF{}}}
                                   }
   \\
  \inferrule*[lab={\boxed{\ltiSimpAppInfClosure}}]
  {
  \inferrule*[]
   {}
   {
   \ltitSstkjudgementJustType{\varnothing}
                      {...}
                      {{\ltisel{\text{g}}{\text{d}}}}
                      {\text{}_{\tikz[overlay,remember picture] \node [] (i) {};}
                      \colorbox{pink}{\ltiClosure{{\ltiEnv{}}^{\tikz[overlay,remember picture] \node [] (j) {};}}
                                           {\ltiufun{{\ltivar{}}_{\tikz[overlay,remember picture] \node [] (p) {};}}
                                                    {\ltiF{}^{\tikz[overlay,remember picture] \node [] (l) {};}}}}
                                                    {\tikz[overlay,remember picture] \node [] (h) {};}}
                      {\ltiClosureCache{}}
                      {\ltiufunelab{\ltiInferred{\text{c1}}}{\text{x}}{\text{x}}}
   }
   \\
   \inferrule*
   {}
   { \ltitSstkjudgementJustType{\ltiClosureCache{}}
                      {...}
                      {\ltiE{}}
                      {{\colorbox{pink}{\ltiS{}}}^{\tikz[overlay,remember picture] \node [] (n) {};}}
                      {\ltiClosureCache{}}
                      {\ltiEp{}}
   }
   \\
   \inferrule*[]
   {}
   {
    \ltitSstkjudgementNoElabCombined{\ltiClosureCache{}}
                      {\ltiEnvConcat{\text{}^{\tikz[overlay,remember picture] \node [] (k) {};}
                                     \colorbox{pink}{\ltiEnv{}}
                                     }
                                    {\hastypesmall{\colorbox{pink}{\ltivar{}}_{\tikz[overlay,remember picture] \node [] (q) {};}}
                                                  {{\colorbox{pink}{\ltiS{}}}^{\tikz[overlay,remember picture] \node [] (o) {};}}}}
                      {\text{}^{\tikz[overlay,remember picture] \node [] (m) {};}
                       \colorbox{pink}{\ltiF{}}}
                      {\colorbox{pink}{\ltiT{}}{\tikz[overlay,remember picture] \node [] (r) {};}}
                      {\ltiClosureCache{}}
                      {\ltisel{\text{g}}{\text{d}}}
                      }
  }
  {
    \ltitSstkjudgementNoElabCombined{\ltiCombinedThreadedEnv{}}
                      {%\ltiEnvConcat{...}
                                    {\hastype{\text{g}}{\ltiRec{\hastype{\text{d}}
                                                                        {{}^{\tikz[overlay,remember picture] \node [] (g) {};}
                                                                                _{\tikz[overlay,remember picture] \node [] (f) {};}
                                                                        \colorbox{pink}{\ltiClosure{\ltiEnv{}}
                                                                                             {\ltiufun{\ltivar{}}{\ltiF{}}}}}}}}}
                      {\ltiapp{
                               {\ltisel{\text{g}}{\text{d}}}}
                              {\ltiE{}}}
                      {\colorbox{pink}{\ltiT{}}{\tikz[overlay,remember picture] \node [] (s) {};}}
                      {\ltiCombinedThreadedEnvp{}}
                      {\ltiapp{\ltiFp{}}{\ltiEp{}}}
  }
  }
  {
    \ltitSstkjudgementJustType{\ltiCombinedThreadedEnv{}}
                      {...}
                      {\ltilet{\text{g}}{\ltiRec{\text{d}={{\tikz[overlay,remember picture] \node [] (a) {};}
                                                           \colorbox{pink}{\ltiufun{\ltivar{}}{\ltiF{}}}}}}
                              {\ltiapp{\ltisel{\text{g}}{\text{d}}}{\ltiE{}}}}
                      {\colorbox{pink}{\ltiT{}}{\tikz[overlay,remember picture] \node [] (t) {};}}
                      {\ltiCombinedThreadedEnvp{}}
                      {\ltiapp{\ltiFp{}}{\ltiEp{}}}
  }
\begin{tikzpicture}[remember picture, overlay,
                  text width = 2.5cm ]
  \coordinate (Start1) at (a);
  \coordinate (End1) at (b);
  \coordinate (Start2) at (b);
  \coordinate (End2) at (c);
  \coordinate (Start3) at (c);
  \coordinate (End3) at (d2);
  \coordinate (Start3p1) at (c1);
  \coordinate (End3p1) at (d1);
  \coordinate (Start4) at (d);
  \coordinate (End4) at (e);
  \coordinate (Start6) at (e1);
  \coordinate (End6) at (f);
  \coordinate (Start7) at (g);
  \coordinate (End7) at (h);
  \coordinate (Start8) at (j);
  \coordinate (End8) at (k);
  \coordinate (Start9) at (l);
  \coordinate (End9) at (m);
  \coordinate (Start10) at (n);
  \coordinate (End10) at (o);
  \coordinate (Start11) at (p);
  \coordinate (End11) at (q);
  \coordinate (Start12) at (r);
  \coordinate (End12) at (s);
  \coordinate (End13) at (t);
  \draw[pink,-](Start1.north) to [bend left] (End1.south);
  \draw[pink,-](Start2.north) to (End2.south);
  \draw[pink,-](Start3.north) to [bend right] (End3.south);
  \draw[pink,-](Start3p1.north) to [bend left] (End3p1.south);
  \draw[pink,-](Start4.north) to (End4.south);
  \draw[pink,-](Start6.north) to (End6.south);
  \draw[pink,-](Start7.north) to (End7.south);
  \draw[pink,-](Start8.north) to [bend left] (End8.south);
  \draw[pink,-](Start9.north) to [bend left] (End9.south);
  \draw[pink,-](Start10.north) to [bend left] (End10.south);
  \draw[pink,-](Start11.north) to [out=356, in=184] (End11.south);
  \draw[pink,-](Start12.north) to (End12.south);
  \draw[pink,-](End12.north) to (End13.south);
\end{tikzpicture} 
\end{mathpar}
% instructions for in/out https://cremeronline.com/LaTeX/minimaltikz.pdf

The derivation proceeds as follows.
First, the record term 
``{\ltiRec{\text{d}={\colorbox{pink}{\ltiufun{\ltivar{}}{\ltiF{}}}}}}''
is checked using the left subderivation, which ends with \ltiSimpUAbs.
There, to delay its checking, {\colorbox{pink}{\ltiufun{\ltivar{}}{\ltiF{}}}}
is packaged with its definition context {\colorbox{pink}{\ltiEnv{}}}
in the symbolic closure {\colorbox{pink}{\ltiClosure{\ltiEnv{}}{\ltiufun{\ltivar{}}{\ltiF{}}}}},
which finds itself in the resulting record type of the left subderivation
``{\ltiRec{\hastype{\text{d}}{\colorbox{pink}{\ltiClosure{\ltiEnv{}}{\ltiufun{\ltivar{}}{\ltiF{}}}}}}}''.
The right subderivation then binds this record type as $\text{g}$ in the type environment
to check the body ``{\ltiapp{\ltisel{\text{g}}{\text{d}}}{\ltiE{}}}''.
The first rule in the right subderivation is \ltiSimpAppInfClosure,
since the operator {\ltisel{\text{g}}{\text{d}}}
has our symbolic closure type
{\colorbox{pink}{\ltiClosure{\ltiEnv{}}{\ltiufun{\ltivar{}}{\ltiF{}}}}}
via a rule like \ltiISel.
Next, the argument \ltiE{} is checked as type {\colorbox{pink}{\ltiS{}}}.
Now, {\colorbox{pink}{\ltiEnv{}}}, {\colorbox{pink}{\ltivar{}}},
and {\colorbox{pink}{\ltiF{}}}
are extracted from the symbolic closure
and, along with {\colorbox{pink}{\ltiS{}}}, used to check (effectively)
our type-decorated unobfuscated term
``{\ltiappParens{\ltifunargtype{\ltivar{}}{\colorbox{pink}{\ltiS{}}}{\ltiF{}}}{\ltiE{}}}''
with the derivation
    ``\ltitSstkjudgementNoElabCombined{\ltiClosureCache{}}
                      {\ltiEnvConcat{\colorbox{pink}{\ltiEnv{}}
                                     }
                                    {\hastypesmall{\colorbox{pink}{\ltivar{}}}
                                                  {{\colorbox{pink}{\ltiS{}}}}}}
                      {\colorbox{pink}{\ltiF{}}}
                      {\colorbox{pink}{\ltiT{}}}
                      {\ltiClosureCache{}}
                      {\ltisel{\text{g}}{\text{d}}}.''
The result type {\colorbox{pink}{\ltiT{}}}
is then propagated to be the type of the entire derivation.

Local Type Inference and Colored Local Type Inference
would have failed to infer the obfuscated term exactly
at \ltiSimpUAbs, because it requires the type of \ltivar{}
to be known from its context at the point \ltiufun{\ltivar{}}{\ltiF{}}
is checked.
Indeed, they also fail to check our unobfuscated term ``{\ltiappParens{\ltiufun{\ltivar{}}{\ltiF{}}}{\ltiE{}}}''
for similar reasons.
In that case, their application rules require a type for the operator before checking its operand,
and so no information may be propagated from operand to operator.
This is fatal to checking the unobfuscated term, since they cannot synthesize the type of functions
from nothing (unlike \ltiSimpUAbs).


\begin{figure}
  \begin{mathpar}
    \boxed{
    \infer[]
    {}
    {\ltiSsubtype{\ltiCombinedThreadedEnv{}}
                 {\ltiEnv{}}
                 {\ltiS{}}
                 {\ltiT{}}
                 {\ltiCombinedThreadedEnvp{}}
                 \\\\
                 \text{
                 With symbolic closure environment \ltiCombinedThreadedEnv{},
                 \ltiS{} is a subtype of \ltiT{}
                 in updated environment \ltiCombinedThreadedEnvp{}.
                 }
    }
    }

    \infer [\ltiSCSTVar]
    {}
    {
     \ltiSsubtype{\ltiCombinedThreadedEnv{}}
                 {\ltiEnv{}}
                 {\ltitvar{}}
                 {\ltitvar{}}
                 {\ltiCombinedThreadedEnv{}}
    }

    \infer [\ltiSCSTop]
    {}
    { \ltiSsubtype{\ltiCombinedThreadedEnv{}}{\ltiEnv{}}{\ltiT{}}{\ltiTop}{\ltiCombinedThreadedEnv{}}}

    \infer [\ltiSCSBot]
    {}
    { \ltiSsubtype{\ltiCombinedThreadedEnv{}}{\ltiEnv{}}{\ltiBot}{\ltiT{}}{\ltiCombinedThreadedEnv{}}}

    \infer [\ltiSCSRec]
    {
    \overrightarrowcaption{\ltiSsubtype{\ltiCombinedThreadedEnv{i-1}}{\ltiEnv{}}
                                {\ltiT{}}
                                {\ltiS{}}
                                {\ltiCombinedThreadedEnv{i}}
                                }^{1 \leq i \leq n}
    }
    {
    \ltiSsubtype{\ltiCombinedThreadedEnv{0}}
                {\ltiEnv{}}
                {\ltiRec{\ova{\hastype{\ltivar{}}{\ltiT{}}}^n,
                         \ova{\hastype{\ltivarp{}}{\ltiTp{}}}}}
                {\ltiRec{\ova{\hastype{\ltivar{}}{\ltiS{}}}^n}}
                {\ltiCombinedThreadedEnv{n}}
    }

    % eg (IFn [Int -> Int] [Number -> Number]) <: [Nothing -> Any]
    \infer [\ltiSCSFn]
    {
    \left(
    \begin{array}{lll}
      |\ova{\ltitvar{}}|>0 \text{ implies \ltiT{}, \ltiTp{}, \ltiS{}, \ltiSp{}}
    \arcr
      \text{contain no symbolic closures}
    \end{array}
    \right)
    \\\\
    \ltiSsubtype{\ltiCombinedThreadedEnv{}}{\ltiEnv{}}{\ltiS{}}{\ltiSp{}}{\ltiCombinedThreadedEnvpp{}}
      \\\\
      \ltiSsubtype{\ltiCombinedThreadedEnvpp{}}{\ltiEnv{}}{\ltiT{}}{\ltiTp{}}{\ltiCombinedThreadedEnvp{}}
    }
    { \ltiSsubtype{\ltiCombinedThreadedEnv{}}{\ltiEnv{}}
                  {\ltiPolyFn{\ltiSp{}}{\ova{\ltitvar{}}}{\ltiT{}}}
                  {\ltiPolyFn{\ltiS{}}{\ova{\ltitvar{}}}{\ltiTp{}}}
                  {\ltiCombinedThreadedEnvp{}}
       }

    \infer [\ltiSCSClosure]
    {
    \ltiCombinedThreadedEnv{1} = {\ltimakeCombinedThreadedEnv{\ltiFuel{1}}{\ltiClosureCache{1}}}
    \\
    \ltitv{\ltiE{}} \cap \ova{\ltitvar{}} = \varnothing
    \\\\
    0 < \ltiFuel{1}
    \\
    \ltitSstkjudgement{\ltimakeCombinedThreadedEnv{\ltiFuel{1}-1}{\ltiClosureCache{1}}}
                      {\ltiEnvConcat{\ltiEnv{}}
                                    {\ltiEnvConcat{\ova{\ltitvar{}}}
                                                  {\hastype{\ltivar{}}{\ltiT{}}}}}
                      {\ltiE{}}
                      {\ltiSp{}}
                      {\ltimakeCombinedThreadedEnv{\ltiFuel{2}}{\ltiClosureCache{2}}}
                      {\ltiEp{}}
                      \\\\
    \ltiSsubtype{\ltimakeCombinedThreadedEnv
                 {\ltiFuel{2}}
                 {\ltiupdateClosureCacheSingleLHS{\ltiClosureCache{2}}
                                                  {\ltiClosureID{}}
                                                  {\ltifuntparamargrettype
                                                   {\ova{\ltitvar{}}}
                                                   {\ltivar{}}
                                                   {\ltiT{}}
                                                   {\ltiSp{}}
                                                   {\ltiEp{}}}}}
                {\ltiEnv{}}{\ltiSp{}}{\ltiS{}}
                {\ltiCombinedThreadedEnv{3}}
    }
    { \ltiSsubtype{\ltiCombinedThreadedEnv{1}}
                  {\ltiEnvp{}}
                  {\ltiClosureWithStkID{\ltiEnv{}}
                                       {\ltiClosureID{}}
                                       {\ltiufun{\ltivar{}}{\ltiE{}}}}
                  {\ltiPolyFn{\ltiT{}}{\ova{\ltitvar{}}}{\ltiS{}}}
                  {\ltiCombinedThreadedEnv{3}}
                  }
  \end{mathpar}

  \caption{Symbolic Closure Language Subtyping}
  \label{symbolic:figure:SC-language-subtype}
\end{figure}

Subtyping for SCL is given in
  \figref{symbolic:figure:SC-language-subtype}.
The judgment
\ltiSsubtype{\ltimakeCombinedThreadedEnv{\ltiFuel{}}{\ltiClosureCache{}}}
            {\ltiEnv{}}
            {\ltiS{}}
            {\ltiT{}}
            {\ltimakeCombinedThreadedEnv{\ltiFuelp{}}{\ltiClosureCachep{}}}
            says with fuel \ltiFuel{} and elaboration cache \ltiClosureCache{},
            \ltiS{} is a subtype of 
            {\ltiT{}}
            with updated fuel \ltiFuelp{} and elaboration cache \ltiClosureCachep{}.
Similar to the typing judgment, we abbreviate subtyping
with threaded environments as
\ltiSsubtype{\ltiCombinedThreadedEnv{}}
            {\ltiEnv{}}
            {\ltiS{}}
            {\ltiT{}}
            {\ltiCombinedThreadedEnvp{}}.
The first five rules correspond to the internal language subtyping rules,
extended with threaded environments.
The extra condition in \ltiSCSFn helps contain symbolic closures
to the type-variable scope they were defined in.
The rule \ltiSCSClosure relates symbolic closures with polymorphic function types.
It follows the idea that \ltiClosureWithStkID{\ltiEnv{}}
                             {\ltiClosureID{}}
                             {\ltiufun{\ltivar{}}{\ltiE{}}}
                             is a subtype of
\ltiPolyFn{\ltiT{}}{\ova{\ltitvar{}}}{\ltiS{}}
if \ltifuntparamargrettype{\ova{\ltitvar{}}}
                          {\ltivar{}}
                          {\ltiT{}}
                          {\ltiSp{}}
                          {\ltiE{}}
is well typed under \ltiEnv{} and 
\ltiSp{}
is a subtype of
\ltiS{}.
The rule proceeds similarly to \ltiSCAppInfClosure, except we may choose a polymorphic
type for \ltiClosureID{}, and we must check the return type is under \ltiS{}.
Adding a type binder to a term invites the possibility of unintentional 
variable capture, and so 
the condition on \ova{\ltitvar{}} avoids capturing free type variables in \ltiE{}.

To illustrate \ltiSCSClosure's role, we again aggressively simplify our presentation.
The simplified rule \ltiSimpSClosure extracts symbolic closure's terms and definition environment
to check it with the provided input type and type arguments.

\begin{mathpar}
    \infer [\ltiSimpSClosure]
    {
    \ltitSstkjudgementNoElabCombined{\ltimakeCombinedThreadedEnv{\ltiFuel{1}-1}{\ltiClosureCache{1}}}
                      {\ltiEnvConcat{\colorbox{pink}{\ltiEnv{}}}
                                    {\ltiEnvConcat{\colorbox{pink}{\ova{\ltitvar{}}}}
                                                  {\hastype{\colorbox{pink}{\ltivar{}}}
                                                           {\colorbox{pink}{\ltiT{}}}}}}
                      {\colorbox{pink}{\ltiE{}}}
                      {\colorbox{pink}{\ltiSp{}}}
                      {\ltimakeCombinedThreadedEnv{\ltiFuel{2}}{\ltiClosureCache{2}}}
                      {\ltiEp{}}
                      \\
    \ltiSsubtypeJustTypes{\ltimakeCombinedThreadedEnv
                 {\ltiFuel{2}}
                 {\ltiupdateClosureCacheSingleLHS{\ltiClosureCache{2}}
                                                  {\ltiClosureID{}}
                                                  {\ltifuntparamargrettype
                                                   {\ova{\ltitvar{}}}
                                                   {\ltivar{}}
                                                   {\ltiT{}}
                                                   {\ltiSp{}}
                                                   {\ltiEp{}}}}}
                {\ltiEnv{}}{\colorbox{pink}{\ltiSp{}}}{\ltiS{}}
                {\ltiCombinedThreadedEnv{3}}
    }
    { \ltiSsubtypeJustTypes{\ltiCombinedThreadedEnv{1}}
                  {\ltiEnvp{}}
                  {\colorbox{pink}{\ltiClosure{\ltiEnv{}}
                                       {\ltiufun{\ltivar{}}{\ltiE{}}}}}
                  {\ltiPolyFn{\colorbox{pink}{\ltiT{}}}{\colorbox{pink}{\ova{\ltitvar{}}}}{\ltiS{}}}
                  {\ltiCombinedThreadedEnv{3}}
                  }
\end{mathpar}

We now illustrate how to check
\ltianncolon{\ltiRec{\text{d}=\ltiufun{\ltivar{}}{\ltiF{}}}}
            {\ltiPolyFn{\ltiT{}}{\ova{\ltitvar{}}}{\ltiS{}}}
a slightly more complicated version of
\ltianncolon{(\ltiufun{\ltivar{}}{\ltiF{}})}{\ltiPolyFn{\ltiT{}}{\ova{\ltitvar{}}}{\ltiS{}}}.
To streamline presentation, we temporarily ignore the fact that type ascription
is syntactic sugar (defined \figref{symbolic:figure:external-language-syntax-abbreviations})
and use its standard typing rule, which first synthesizes a type for the term, then checks
it is a subtype of the provided type.

\begin{mathpar}
  \inferrule*[]
  {
  \inferrule*
  {
  \inferrule*[lab=\boxed{\ltiSimpUAbs}]
   {}
   {
   \ltitSstkjudgementNoElabCombined{\varnothing}
                                   {\colorbox{pink}{\ltiEnv{}}^{\tikz[overlay,remember picture] \node [] (c1) {};}}
                                   {{\tikz[overlay,remember picture] \node [] (c) {};}\colorbox{pink}{\ltiufun{\ltivar{}}{\ltiF{}}}}
                                   {\colorbox{pink}{
                                           {\ltiClosure{\text{}^{\tikz[overlay,remember picture] \node [] (d1) {};}\ltiEnv{}}
                                                                {{\tikz[overlay,remember picture] \node [] (d2) {};}\ltiufun{\ltivar{}}{\ltiF{}}}}}
                                           {\tikz[overlay,remember picture] \node [] (d) {};}}
                                   {\ltiCombinedThreadedEnv{}}
                                   {\ltiufun{\ltivar{}}{\ltiF{}}}
   }
   }
   {
   \ltitSstkjudgementJustType{\varnothing}
                                   {...}
                                   {\ltiRec{\text{d}={{\tikz[overlay,remember picture] \node [] (b) {};}\colorbox{pink}{\ltiufun{\ltivar{}}{\ltiF{}}}}}}
                                   {\ltiRec{\hastype{\text{d}}
                                           {\colorbox{pink}
                                            {\ltiClosure{\ltiEnv{}}
                                                                {\ltiufun{\ltivar{}}
                                                                         {\ltiF{}}}}
                                                                         ^
                                                                         {\tikz[overlay,remember picture] \node [] (e) {};}
                                                                         _
                                                                         {\tikz[overlay,remember picture] \node [] (e1) {};}}}}
                                   {\ltiCombinedThreadedEnv{}}
                                   {\ltiufun{\ltivar{}}{\ltiF{}}}
                                   }
   \\
  \inferrule*[%T-Ann
  ]
  {
  \inferrule*[]
   {}
   {
   \ltitSstkjudgementJustType{\varnothing}
                      {...}
                      {{\ltisel{\text{g}}{\text{d}}}}
                      {\colorbox{pink}{\ltiClosure{{\ltiEnv{}}}
                                           {\ltiufun{{\ltivar{}}}
                                                    {\ltiF{}}}}
                                                    {\tikz[overlay,remember picture] \node [] (h) {};}}
                      {\ltiClosureCache{}}
                      {\ltiufunelab{\ltiInferred{\text{c1}}}{\text{x}}{\text{x}}}
                      \\
    \inferrule* [lab=\boxed{\ltiSimpSClosure}]
    {
    \ltitSstkjudgementNoElabCombined{\ltiClosureCache{}}
                      {\ltiEnvConcat{\colorbox{pink}{\ltiEnv{}}_{\tikz[overlay,remember picture] \node [] (k) {};}}
                                    {\ltiEnvConcat
                                      {{\colorbox{pink}{\ova{\ltitvar{}}}}_{\tikz[overlay,remember picture] \node [] (u) {};}}
                                      {\hastypesmall{\colorbox{pink}{\ltivar{}}_{\tikz[overlay,remember picture] \node [] (q) {};}}
                                                    {{\colorbox{pink}{\ltiT{}}}_{\tikz[overlay,remember picture] \node [] (o) {};}}}}}
                      {\text{}_{\tikz[overlay,remember picture] \node [] (m) {};}
                       \colorbox{pink}{\ltiF{}}}
                      {{\ltiSp{}}}
                      {\ltiClosureCache{}}
                      {\ltisel{\text{g}}{\text{d}}}
                      \\
    \ltiSsubtypeJustTypes{\ltimakeCombinedThreadedEnv
                 {\ltiFuel{2}}
                 {\ltiupdateClosureCacheSingleLHS{\ltiClosureCache{2}}
                                                  {\ltiClosureID{}}
                                                  {\ltifuntparamargrettype
                                                   {\ova{\ltitvar{}}}
                                                   {\ltivar{}}
                                                   {\ltiT{}}
                                                   {\ltiSp{}}
                                                   {\ltiEp{}}}}}
                {\ltiEnv{}}{{\ltiSp{}}}
                           {\colorbox{pink}{\ltiS{}}{\tikz[overlay,remember picture] \node [] (r) {};}}
                {\ltiCombinedThreadedEnv{3}}
    }
    { 
      \ltiSsubtypeJustTypes{\ltiCombinedThreadedEnv{1}}
                  {\ltiEnvp{}}
                  {\text{}_{\tikz[overlay,remember picture] \node [] (i) {};}
                      \colorbox{pink}{\ltiClosure{{\tikz[overlay,remember picture] \node [] (j) {};}{\ltiEnv{}}}
                                           {\ltiufun{{\ltivar{}}^{\tikz[overlay,remember picture] \node [] (p) {};}}
                                                    {{\tikz[overlay,remember picture] \node [] (l) {};}\ltiF{}}}}}
                  {{\colorbox{pink}
                    {\ltiPolyFn{{\ltiT{}}^{\tikz[overlay,remember picture] \node [] (n) {};}}
                               {{\tikz[overlay,remember picture] \node [] (t) {};}{\ova{\ltitvar{}}}}
                               {{\ltiS{}}{\tikz[overlay,remember picture] \node [] (s) {};}}}}
                    _{\tikz[overlay,remember picture] \node [] (v) {};}}
                  {\ltiCombinedThreadedEnv{3}}
                  }
   }
  }
  {
    \ltitSstkjudgementNoElabCombined{\ltiCombinedThreadedEnv{}}
                      {%\ltiEnvConcat{...}
                                    {\hastype{\text{g}}{\ltiRec{\hastype{\text{d}}
                                                                        {{}^{\tikz[overlay,remember picture] \node [] (g) {};}
                                                                                _{\tikz[overlay,remember picture] \node [] (f) {};}
                                                                        \colorbox{pink}{\ltiClosure{\ltiEnv{}}
                                                                                             {\ltiufun{\ltivar{}}{\ltiF{}}}}}}}}}
                      {\ltianncolon{\ltisel{\text{g}}{\text{d}}}
                                   {{}_{\tikz[overlay,remember picture] \node [] (x) {};}
                                   {\colorbox{pink}{\ltiPolyFn{\ltiT{}}
                                                              {\ova{\ltitvar{}}}
                                                              {\ltiS{}}}}^{\tikz[overlay,remember picture] \node [] (w) {};}}}
                      {{\ltiPolyFn{\ltiT{}}{\ova{\ltitvar{}}}{\ltiS{}}}}
                      {\ltiCombinedThreadedEnvp{}}
                      {\ltiapp{\ltiFp{}}{\ltiEp{}}}
  }
  }
  {
    \ltitSstkjudgementJustType{\ltiCombinedThreadedEnv{}}
                      {...}
                      {\ltilet{\text{g}}{\ltiRec{\text{d}={{\tikz[overlay,remember picture] \node [] (a) {};}
                                                           {\colorbox{pink}{\ltiufun{\ltivar{}}{\ltiF{}}}}}}}
                              {\ltianncolon{\ltisel{\text{g}}{\text{d}}}
                                           {{\colorbox{pink}{\ltiPolyFn{\ltiT{}}
                                                                      {\ova{\ltitvar{}}}
                                                                      {\ltiS{}}}}
                                                                      ^{\tikz[overlay,remember picture] \node [] (y) {};}}}}
                      {\ltiPolyFn{\ltiT{}}{\ova{\ltitvar{}}}{\ltiS{}}}
                      {\ltiCombinedThreadedEnvp{}}
                      {\ltiapp{\ltiFp{}}{\ltiEp{}}}
  }
\begin{tikzpicture}[remember picture, overlay,
                  text width = 2.5cm ]
  \coordinate (Start1) at (a);
  \coordinate (End1) at (b);
  \coordinate (Start2) at (b);
  \coordinate (End2) at (c);
  \coordinate (Start3) at (c);
  \coordinate (End3) at (d2);
  \coordinate (Start3p1) at (c1);
  \coordinate (End3p1) at (d1);
  \coordinate (Start4) at (d);
  \coordinate (End4) at (e);
  \coordinate (Start6) at (e1);
  \coordinate (End6) at (f);
  \coordinate (Start7) at (g);
  \coordinate (End7) at (h);
  \coordinate (Start7p1) at (h);
  \coordinate (End7p1) at (i);
  \coordinate (Start8) at (j);
  \coordinate (End8) at (k);
  \coordinate (Start9) at (l);
  \coordinate (End9) at (m);
  \coordinate (Start10) at (n);
  \coordinate (End10) at (o);
  \coordinate (Start11) at (p);
  \coordinate (End11) at (q);
  \coordinate (Start12) at (r);
  \coordinate (End12) at (s);
  \coordinate (Start13) at (t);
  \coordinate (End13) at (u);
  \coordinate (Start14) at (v);
  \coordinate (End14) at (w);
  \coordinate (Start15) at (x);
  \coordinate (End15) at (y);
  \draw[pink,-](Start1.north) to [bend left] (End1.south);
  \draw[pink,-](Start2.north) to (End2.south);
  \draw[pink,-](Start3.north) to [bend right] (End3.south);
  \draw[pink,-](Start3p1.north) to [bend left] (End3p1.south);
  \draw[pink,-](Start4.north) to (End4.south);
  \draw[pink,-](Start6.north) to (End6.south);
  \draw[pink,-](Start7.north) to (End7.south);
  \draw[pink,-](Start7p1.north) to (End7p1.south);
  \draw[pink,-](Start8.north) to (End8.south);
  \draw[pink,-](Start9.north) to (End9.south);
  \draw[pink,-](Start10.north) to (End10.south);
  \draw[pink,-](Start11.north) to (End11.south);
  \draw[pink,-](Start12.north) to (End12.south);
  \draw[pink,-](Start13.north) to (End13.south);
  \draw[pink,-](Start14.north) to (End14.south);
  \draw[pink,-](Start15.north) to (End15.south);
\end{tikzpicture} 
\end{mathpar}

The derivation begins in a similar fashion to the previous one, 
with the left subderivation using \ltiSimpUAbs
to delay the type checking of {\colorbox{pink}{\ltiufun{\ltivar{}}{\ltiF{}}}}
via the symbolic closure
\colorbox{pink}
{\ltiClosure{{\ltiEnv{}}}
           {\ltiufun{{\ltivar{}}}
                    {\ltiF{}}}}.
The right subderivation
begins with a rule checking type ascription, specifically that  term
``{\ltisel{\text{g}}{\text{d}}}''
has function type
{\colorbox{pink}{\ltiPolyFn{\ltiT{}}
                           {\ova{\ltitvar{}}}
                           {\ltiS{}}}}.
First, a type is synthesized for ``{\ltisel{\text{g}}{\text{d}}}'',
which reveals our symbolic closure type from the left subderivation.
Then, we check that the symbolic closure is a subtype of 
our desired function type via the symbolic reduction
    ``\ltitSstkjudgementNoElabCombined{\ltiClosureCache{}}
                      {\ltiEnvConcat{\colorbox{pink}{\ltiEnv{}}}
                                    {\ltiEnvConcat
                                      {{\colorbox{pink}{\ova{\ltitvar{}}}}}
                                      {\hastypesmall{\colorbox{pink}{\ltivar{}}}
                                                    {{\colorbox{pink}{\ltiT{}}}}}}}
                      {\colorbox{pink}{\ltiF{}}}
                      {{\ltiSp{}}}
                      {\ltiClosureCache{}}
                      {\ltisel{\text{g}}{\text{d}}}.''
Finally, the return type of the symbolic reduction is checked to be compatible
with our desired function type with
    \ltiSsubtypeJustTypes{\ltimakeCombinedThreadedEnv
                 {\ltiFuel{2}}
                 {\ltiupdateClosureCacheSingleLHS{\ltiClosureCache{2}}
                                                  {\ltiClosureID{}}
                                                  {\ltifuntparamargrettype
                                                   {\ova{\ltitvar{}}}
                                                   {\ltivar{}}
                                                   {\ltiT{}}
                                                   {\ltiSp{}}
                                                   {\ltiEp{}}}}}
                {\ltiEnv{}}{{\ltiSp{}}}
                           {\colorbox{pink}{\ltiS{}}}
                {\ltiCombinedThreadedEnv{3}}.


\begin{figure}

%  \[
%    \boxed{\ltielabDriver{\ltiE{}}{\ltiEp{}}{\ltiT{}}
%    \text{ Elaborates external language term \ltiE{} to internal language term \ltiEp{} and type \ltiT{}.
%    }
%    }
%  \]
%
%  \[
%  \begin{array}{lll}
%    \ltielabDriver{\ltiE{1}}
%                  {\ltielimClosLHS{\ltiClosureCache{}}{\ltiE{2}}}
%                  {\ltielimClosTLHS{\varnothing}{\ltiClosureCache{}}{\ltiT{}}}
%                  , &\text{where }
%    \exists \ltiFuel{}.
%     \ltitSstkjudgement{\ltimakeCombinedThreadedEnv{\ltiFuel{}}{\ltiEmptyClosureCache}}
%                       {\ltiEmptyEnv}
%                       {\ltiE{1}}
%                       {\ltiT{}}
%                       {\ltimakeCombinedThreadedEnv{\ltiFuelp{}}{\ltiClosureCache{}}}
%                       {\ltiE{2}}
%  \end{array}
%  \]

  \begin{mathpar}
    \boxed{
    \infer[]
    {}
    {\ltiupdateClosureCacheSingle{\ltiClosureCache{}}{\ltiClosureID{}}{\ltiE{}}{\ltiClosureCachep{}}
    \\\\
    \text{Pick SCL elaboration \ltiE{} for symbolic closure identifier \ltiClosureID{}.
    }
    }
    }

    \begin{array}{llll}
      \ltiupdateClosureCacheSinglealign{\ltiClosureCache{}}{\ltiClosureID{}}{\ltiE{}}
                           {\ltimapsto{\ltiClosureCache{}}
                                      {\ltiClosureID{}}
                                      {\ltiClosure{\ltiEnv{}}
                                                  {\ltiE{}}}}
                                                 , &
                                                 \text{where }
    (\ltilookup{\ltiClosureCache{}}{\ltiClosureID{}} = 
                       {\ltiClosure{\ltiEnv{}}
                                   {\ltiufun{\ltivar{}}{\ltiF{}}}})
                                   \text{ or }
    (\ltilookup{\ltiClosureCache{}}{\ltiClosureID{}} = 
              {\ltiClosure{\ltiEnv{}}{\ltiE{}}})
    \end{array}
  \end{mathpar}

  \[
    \boxed{\ltielimClos{\ltiClosureCache{}}{\ltiE{}}{\ltiEp{}}
    \text{ Converts symbolic closures in \ltiE{} to explicit types in \ltiEp{}}
    }
  \]

  \[
  \begin{array}{llll}
    \ltielimClosalign{\ltiClosureCache{}}{\ltivar{}}
                     {\ltivar{}}
                     \\
    \ltielimClosalign{\ltiClosureCache{}}
                     {\ltiappinst{\ltiF{}}
                                 {\ova{\ltiR{}}}
                                 {\ltiE{}}}
                     {\ltiappinst{\ltielimClosLHS{\ltiClosureCache{}}{\ltiF{}}}
                                 {\ova{\ltielimClosTLHS{\varnothing}{\ltiClosureCache{}}{\ltiR{}}}}
                                 {\ltielimClosLHS{\ltiClosureCache{}}{\ltiE{}}}}
                             \\
    \ltielimClosalign{\ltiClosureCache{}}{\ltisel{\ltiF{}}{\ltivar{}}}
                     {\ltisel{\ltielimClosLHS{\ltiClosureCache{}}{\ltiF{}}}{\ltivar{}}}
                     \\
    \ltielimClosalign{\ltiClosureCache{}}{\ltiRec{\ova{\ltivar{} = \ltiF{}}}}
                     {\ltiRec{\ova{\ltivar{} = \ltielimClosLHS{\ltiClosureCache{}}{\ltiF{}}}}}
                     \\
    \ltielimClosalign{\ltiClosureCache{}}
                     {\ltifuntparamargtype{\ova{\ltitvar{}}}
                                          {\ltivar{}}
                                          {\ltiT{}}
                                          {\ltiE{}}}
                     {\ltifuntparamargtype{\ova{\ltitvar{}}}
                                          {\ltivar{}}
                                          {\ltielimClosTLHS{\varnothing}{\ltiClosureCache{}}{\ltiT{}}}
                                          {\ltielimClosLHS{\ltiClosureCache{}}{\ltiE{}}}}
                     \\
    \ltielimClosalign{\ltiClosureCache{}}
                     {\ltiufunelab{\ltiufunelabentry{\ltiClosureID{}}}
                                  {\ltivar{}}
                                  {\ltiE{}}}
                     {\ltielimClosLHS{\ltiClosureCache{}}
                                     {\ltiF{}}},
                     &\text{where } \ltilookup{\ltiClosureCache{}}{\ltiClosureID{}}
                                      = \ltiClosure{\ltiEnv{}}{\ltiF{}}
  \end{array}
  \]


  \begin{mathpar}
    % spread boxed over two lines for dissertation
    \boxed{
    \infer[]
    {}
    {\ltielimClosT{\ova{\ltiClosureID{}}}{\ltiClosureCache{}}{\ltiT{}}{\ltiTp{}}
    \\
    \text{ Converts symbolic closures in \ltiT{} to explicit types in \ltiTp{},
    with seen symbolic closures \ova{\ltiClosureID{}}.}
    }}

  \begin{array}{llll}
    \ltielimClosTalign{\ova{\ltiClosureID{}}}{\ltiClosureCache{}}{\ltiTop}{\ltiTop}
                      \\
    \ltielimClosTalign{\ova{\ltiClosureID{}}}{\ltiClosureCache{}}{\ltiBot}{\ltiBot}
                      \\
    \ltielimClosTalign{\ova{\ltiClosureID{}}}{\ltiClosureCache{}}{\ltitvar{}}{\ltitvar{}}
                      \\
    \ltielimClosTalign{\ova{\ltiClosureID{}}}{\ltiClosureCache{}}
                      {\ltiPolyFn{\ltiT{}}{\ova{\ltitvar{}}}{\ltiS{}}}
                      {\ltiPolyFn{\ltielimClosTLHS{\ova{\ltiClosureID{}}}{\ltiClosureCache{}}{\ltiT{}}}{\ova{\ltitvar{}}}
                             {\ltielimClosTLHS{\ova{\ltiClosureID{}}}{\ltiClosureCache{}}{\ltiS{}}}}
                                          \\
    \ltielimClosTalign{\ova{\ltiClosureID{}}}{\ltiClosureCache{}}
                      {\ltiRec{\ova{\hastype{\ltivar{}}{\ltiT{}}}}}
                      {\ltiRec{\ova{\hastype{\ltivar{}}{\ltielimClosTLHS{\ova{\ltiClosureID{}}}{\ltiClosureCache{}}{\ltiT{}}}}}}
                      \\
    \ltielimClosTalign{\ova{\ltiClosureID{}}}{\ltiClosureCache{}}
                      {\ltiClosureWithStkID{\ltiEnv{}}{\ltiClosureIDp{}}{\ltiE{}}}
                      {\ltiPolyFn{\ltielimClosTLHS{\ova{\ltiClosureID{}}\ltiClosureIDp{}}
                                                  {\ltiClosureCache{}}
                                                  {\ltiT{}}}
                                 {\ova{\ltitvar{}}}
                                 {\ltielimClosTLHS{\ova{\ltiClosureID{}}\ltiClosureIDp{}}
                                                  {\ltiClosureCache{}}
                                                  {\ltiS{}}}}
                      , & 
                      \text{where }
                      \ltiClosureIDp{} \not\in \ova{\ltiClosureID{}},
                      \ltilookup{\ltiClosureCache{}}{\ltiClosureIDp{}}
                      = \ltiClosure{\ltiEnv{}}
                                   {\ltifuntparamargrettype{\ova{\ltitvar{}}}{\ltivar{}}{\ltiT{}}{\ltiS{}}{\ltiE{}}}
                                   % this condition is checked in S-Closure and constraint system
                                   %, \ova{\ltitvar{}} \cap \ltitv{\ltiE{}} = \varnothing
  \end{array}
  \end{mathpar}
  \caption{Elaboration Metafunctions for SCL Terms and Types}
  \label{symbolic:figure:SC-language-elaboration}
\end{figure}

Now that we have covered the type system and subtyping, we turn to the elaboration rules,
which are given in \figref{symbolic:figure:SC-language-elaboration}.
They are split into two metafunctions \ltielimClossymbol and \ltielimClosTsymbol,
elaborating terms and types respectively, along with
\ltiupdateClosureCacheSinglesymbol, which manages when a symbolic closure's elaboration may
be chosen.

For terms, \ltielimClos{\ltiClosureCache{}}{\ltiE{}}{\ltiEp{}}
elaborates \ltiE{} to \ltiEp{} using elaboration cache \ltiClosureCache{}.
The case for tagged unannotated functions 
{\ltiufunelab{\ltiufunelabentry{\ltiClosureID{}}}
                                  {\ltivar{}}
                                  {\ltiE{}}}
simply uses the elaboration entry for \ltiClosureID{} to continue elaboration.
The metafunction is undefined for the external language's unannotated terms
\ltiufun{\ltivar{}}{\ltiE{}} and \ltiapp{\ltiF{}}{\ltiE{}}.
This enforces that each symbolic closure must be symbolically executed at least once
to elaborate away these terms.
Unannotated applications \ltiapp{\ltiF{}}{\ltiE{}}
are elaborated away with local type argument synthesis.
%, covered in \chapref{chapter:symbolic:directed-lti}.

For elaborating types, \ltielimClossymbol
uses \ltielimClosT{\ova{\ltiClosureID{}}}{\ltiClosureCache{}}{\ltiT{}}{\ltiTp{}}, which
elaborates symbolic closures in \ltiT{} using \ltiClosureCache{}.
Some extra bookkeeping is needed to prevent infinitely generating
types. Since symbolic closures may be passed to other symbolic closures,
a seen-set \ova{\ltiClosureID{}} handles the case where it is passed to itself.
Since we do not model equi-recursive types, we simply disallow that situation here.

To highlight the elaboration rules,
we demonstrate the lambda-encoding of let in our system by
checking term ``\ltilet{\ltivar{}}{42}{\ltiF{}}''---which desugars to
``{\ltiappParens{\ltiufun{\ltivar{}}{\ltiF{}}}{\text{42}}}''---at some
some overall type \ltiS{}
with $\text{42}$ having type $\text{Int}$.
To streamline presentation, we assume that checking \ltiF{}
does not introduce any symbolic closures
(to keep the elaboration cache compact), liberally
remove uninteresting type environments and elaboration caches,
and omit symbolic reduction fuel from the derivation.

\begin{mathpar}
  \inferrule*[left=\ltiSCAppInfClosure
  ]
  {\infer[\ltiSCUAbs]
   {
   %\colorbox{pink}{\ltiClosureID{}} \not\in dom(\varnothing)\\
   %\\\\
   \ltiClosureCache{} =
                       {\ltiClosureCacheEntry{\colorbox{pink}{\ltiClosureID{}}}
                                           {\ltiClosure{\ltiEmptyEnv}
                                                        {\ltiufun{\ltivar{}}{\ltiF{}}}}}
   }
   { \ltitSstkjudgement{\varnothing}
                      {\ltiEmptyEnv}
                      {\tikz[overlay,remember picture] \node [] (b) {};\ltiufun{\ltivar{}}{\ltiF{}}}
                      {\ltiClosureWithStkID{\ltiEmptyEnv}
                                           {{}_{\tikz[overlay,remember picture] \node [] (m) {};}{\colorbox{pink}{\ltiClosureID{}}}}
                                           {\ltiufun{\ltivar{}^{\tikz[overlay,remember picture] \node [] (e) {};}}
                                                    {\ltiF{}^{\tikz[overlay,remember picture] \node [] (g) {};}}}}
                      {\ltiClosureCache{}}
                      {\colorbox{pink}{\ltiufunelab{\ltiClosureID{}}{\ltivar{}}{{\ltiF{}}_{\tikz[overlay,remember picture] \node [] (o) {};}}}}
   }
   \\
    \begin{array}{ccc}
   \ltitSstkjudgementJustType{\ltiClosureCache{}}
                      {\ltiEmptyEnv}
                      {\tikz[overlay,remember picture] \node [] (d) {};\text{42}}
                      {\colorbox{pink}{\text{Int}\tikz[overlay,remember picture] \node [] (i) {};}}
                      {\ltiClosureCache{}}
                      {\ltiEp{}}
                      \ \ \ \ \ 
    \ltitSstkjudgementNoCombined{\ltiClosureCache{}}
                      {\hastype{\text{}^{\tikz[overlay,remember picture] \node [] (f) {};}
                                \ltivar{}}
                               {{\tikz[overlay,remember picture] \node [] (j) {};}{\text{Int}}}}
                      {\text{}^{\tikz[overlay,remember picture] \node [] (h) {};}
                       \ltiF{}}
                      {{\colorbox{pink}{\ltiS{}}}_{\tikz[overlay,remember picture] \node [] (k) {};}}
                      {\ltiClosureCache{}}
                      {{\colorbox{pink}{\ltiFp{}}}_{\tikz[overlay,remember picture] \node [] (s) {};}}
    \end{array}
  }
  {
    \ltitSstkjudgement{\varnothing}
                      {\ltiEmptyEnv}
                      {\ltiappParens{\text{}^{\tikz[overlay,remember picture] \node [] (a) {};}
                                     \ltiufun{\ltivar{}}{\ltiF{}}}
                                    {\text{42}^{\tikz[overlay,remember picture] \node [] (c) {};}}}
                      {\ltiS{}}
                      {\ltiClosureCacheEntry
                       {{}^{{}^{\tikz[overlay,remember picture] \node [] (n) {};}}{\colorbox{pink}{\ltiClosureID{}}}}
                       {\ltiClosure{\ltiEmptyEnv}
                                         {\ltifunargrettype{\ltivar{}}
                                                           {{\colorbox{pink}
                                                            {\text{Int}}}^{\tikz[overlay,remember picture] \node [] (j2) {};}}
                                                           {
                                                            {\colorbox{pink}{\ltiS{}}}^{\tikz[overlay,remember picture] \node [] (l) {};}
                                                            \ }
                                                           {{\colorbox{pink}{\ltiFp{}}}^{\tikz[overlay,remember picture] \node [] (t) {};}}}}}
                      {\ltiappParens{{\colorbox{pink}{\ltiufunelab{\ltiClosureID{}}{\ltivar{}}{\ltiF{}}}}^{\tikz[overlay,remember picture] \node [] (p) {};}}{\text{42}}}
  }
\begin{tikzpicture}[remember picture, overlay,
                  text width = 2.5cm ]
  \coordinate (Start1) at (a);
  \coordinate (End1) at (b);
  \coordinate (Start2) at (c);
  \coordinate (End2) at (d);
  \coordinate (Start3) at (e);
  \coordinate (End3) at (f);
  \coordinate (Start4) at (g);
  \coordinate (End4) at (h);
  \coordinate (Start5) at (i);
  \coordinate (End5) at (j);
  \coordinate (Start6) at (i);
  \coordinate (End6) at (j2);
  \coordinate (Start7) at (k);
  \coordinate (End7) at (l);
  \coordinate (Start8) at (m);
  \coordinate (End8) at (n);
  \coordinate (Start9) at (o);
  \coordinate (End9) at (p);
  \coordinate (Start11) at (s);
  \coordinate (End11) at (t);
  % (lambda (x) f) -> (lambda (x) f)
  %\draw[pink,thick,dotted,->](Start1.north) to (End1.south);
  % 42 -> 42
  %\draw[pink,thick,dotted,->](Start2.north) to (End2.south);
  \draw[pink,thick,dotted,->](Start3.north) to [bend left] (End3.south);
  \draw[pink,thick,dotted,->](Start4.north) to [bend left] (End4.south);
  \draw[pink,thick,dotted,->](Start5.north) to [bend right] (End5.south);
  % a little to the right (Int -> Int)
  \draw[pink,thick,solid,->,transform canvas={xshift=1pt}](Start6.north) to (End6.south);
  \draw[pink,thick,solid,->](Start7.north) to [bend right=30] (End7.south);
  \draw[pink,thick,solid,->](Start8.north) to (End8.north);
  % a little higher
  \draw[pink,thick,solid,->,transform canvas={yshift=4pt}](Start9.north) to [bend left=10] (End9.north);
  \draw[pink,thick,solid,->](Start11.north) to (End11.north);
\end{tikzpicture} 
\end{mathpar}
%
The derivation starts with the \ltiSCAppInfClosure rule (chosen because the operator
\ltiufun{\ltivar{}}{\ltiF{}}
has a symbolic closure type)
with empty elaboration cache \ltiEmptyClosureCache
and empty type environment \ltiEmptyEnv.
The distinct identifer {{\colorbox{pink}{\ltiClosureID{}}}}
is chosen by \ltiSCUAbs in the left subderivation, and is utilized there in three ways.
First, the elaboration cache \ltiClosureCache{}'s entry for
{{\colorbox{pink}{\ltiClosureID{}}}} is initialized
to
{\ltiClosure{\ltiEmptyEnv}{\ltiufun{\ltivar{}}{\ltiF{}}}},
an untyped ``placeholder'' that signals that the final elaboration of {{\colorbox{pink}{\ltiClosureID{}}}}
has yet to be picked.
Second, it identifies the rule's overall symbolic closure type
                      {\ltiClosureWithStkID{\ltiEmptyEnv}
                                           {{\colorbox{pink}{\ltiClosureID{}}}}
                                           {\ltiufun{\ltivar{}}
                                                    {\ltiF{}}}}
for later elaboration.
Third, it tags the rule's elaboration {\colorbox{pink}{\ltiufunelab{\ltiClosureID{}}{\ltivar{}}{\ltiF{}}}}
which, again, links the term to its elaboration cache entry.
Like a traditional application rule, the middle subderivation checks the argument.
The right subderivation performs a symbolic reduction
    \ltitSstkjudgementNoCombined{\ltiClosureCache{}}
                      {\hastype{\ltivar{}}
                               {{\text{Int}}}}
                      {\ltiF{}}
                      {{\colorbox{pink}{\ltiS{}}}}
                      {\ltiClosureCache{}}
                      {{\colorbox{pink}{\ltiFp{}}}},
where \ltivar{} and \ltiF{} come from the symbolic closure type
and $\text{Int}$ from the argument's type,
with the result type \colorbox{pink}{\ltiS{}} being the type of the entire derivation.
The elaboration \colorbox{pink}{\ltiFp{}}
will (eventually) be inserted in \ltiF{}'s place.

Now the goal is to stash enough information in the elaborated term and elaboration cache
to eliminate unannotated terms using \ltielimClossymbol after type checking.
For the entire derivation's elaborated term,
tagged functions are preserved
by combining the operator and operand elaborations in
{\ltiappParens{{\colorbox{pink}{\ltiufunelab{\ltiClosureID{}}{\ltivar{}}{\ltiF{}}}}}{\text{42}}}.
For the elaboration cache,
the (omitted) call
                 \ltiupdateClosureCacheSingleLHS{\ltiClosureCache{}}
                                                {\ltiClosureID{}}
                       {\ltiClosure{\ltiEmptyEnv}
                                         {\ltifunargrettype{\ltivar{}}
                                                           {{\text{Int}}}
                                                           {{\colorbox{pink}{\ltiS{}}}
                                                            \ }
                                                           {{\colorbox{pink}{\ltiFp{}}}}}}
updates the elaboration cache's ``placeholder'' entry
for {{\colorbox{pink}{\ltiClosureID{}}}}
to be 
                      {\ltiClosureCacheEntry
                       {{\colorbox{pink}{\ltiClosureID{}}}}
                       {\ltiClosure{\ltiEmptyEnv}
                                         {\ltifunargrettype{\ltivar{}}
                                                           {{\colorbox{pink}
                                                            {\text{Int}}}}
                                                           {
                                                            {\colorbox{pink}{\ltiS{}}}
                                                            \ }
                                                           {{\colorbox{pink}{\ltiFp{}}}}}}},
where {{\colorbox{pink}{\ltiClosureID{}}}},
{\ltiEmptyEnv}, and {\ltivar{}} come from the symbolic closure type,
{{\colorbox{pink}{\text{Int}}}} from the argument's type, 
and {\colorbox{pink}{\ltiS{}}},
{{\colorbox{pink}{\ltiFp{}}}}
from the symbolic reduction.

% Now explain how to SC elaborate a term

The following call to \ltielimClossymbol then eliminates all SCL terms
and types, performing a full elaboration into the internal language.
Once a tagged function term is encountered, its elaboration is simply read
off the cache and inserted.
In this case, there is no need
to also traverse \ltiFp{} since we assumed it does not contain symbolic closures,
however in general a recursive \ltielimClossymbol call on
\ltiFp{} would be needed to eliminate its symbolic closures.

\[
\begin{array}{llll}
\ltielimClosalign{\ltiClosureCacheEntry
                       {{\colorbox{pink}{\ltiClosureID{}}}{\tikz[overlay,remember picture] \node [] (b) {};}}
                       {{\ltiClosure{\ltiEmptyEnv}
                                   {\colorbox{pink}
                                         {\ltifunargrettype{\ltivar{}}
                                                           {\text{Int}}
                                                           {\ltiS{} \ }
                                                           {\ltiFp{}}}}}^
                        {\tikz[overlay,remember picture] \node [] (c) {};}}}
{\ \ \ltiappParens{{\tikz[overlay,remember picture] \node [] (a) {};}{\colorbox{pink}{\ltiufunelab{\ltiClosureID{}}{\ltivar{}}{\ltiF{}}}}}{\text{42}}}
{\ltiappParens{{}^{\tikz[overlay,remember picture] \node [] (d) {};}
               \colorbox{pink}{\ltifunargrettype{\ltivar{}}
                                                {\text{Int}}
                                                {\ltiS{} \ }
                                                {\ltiFp{}}}}
              {\text{42}}}
\begin{tikzpicture}[remember picture, overlay,
                  text width = 2.5cm ]
  \coordinate (Start1) at (a);
  \coordinate (End1) at (b);
  \coordinate (Start2) at (c);
  \coordinate (End2) at (d);
  \draw[pink,thick,solid,->](Start1.north) to [bend left] (End1.south);
  \draw[pink,thick,solid,->](Start2.north) to [bend left] (End2.south);
\end{tikzpicture} 
\end{array}
\]

%\section{Examples without Type Argument Synthesis}

%% let x = 1 in x
%{
%\begin{lstlisting}[language=ml,mathescape=true]
%let x = 1 in x
%(* Desugared *)
%$\ltiappParens{\ltiufun{\text{x}}{\text{x}}}{\text{1}}$
%(* SC annotated *)
%(* $\ltiInferred{\ltiClosureCache{} =%
%      \ltiClosureCacheEntry{\text{c1}}%
%                           {\ltiClosure{\ltiEmptyEnv}%
%                                       {\ltiNotInferred%
%                                        {\ltifunargrettype{\text{x}}%
%                                                          {\ltiInferred{\text{Int}}}%
%                                                          {\ltiInferred{\text{Int}},}%
%                                                          {\text{x}}}}}}$ *)
%$\ltiappParens{\ltiufunelab{\ltiInferred{\text{c1}}}{\text{x}}{\text{x}}}{\text{1}}$
%(* fully annotated *)
%$\ltiappParens{\ltifunargrettype{\text{x}}%
%                                {\ltiInferred{\text{Int}}}%
%                                {\ltiInferred{\text{Int}},}%
%                                {\text{x}}}%
%              {\text{1}}$
%\end{lstlisting}
%}


\chapter{Symbolic Closure Metatheory Conjectures}
\label{chapter:symbolic:metatheory}

We now outline the relationships between the internal, external, and symbolic closure
languages that would be desirable as a series of conjectures.
We leave the proofs as future work.

%\begin{lemma}[Subtyping (External Language)]
%   \ltiSdsubtypeseen{\ltiSubtypeSeen{}}{\ltiEnv{}}{\ltiT{}}{\ltiS{}}
%    iff
%   \ltiisubtypeseen{\ltiSubtypeSeen{}}{\ltiEnv{}}{\ltiT{}}{\ltiS{}}
%\end{lemma}

%\begin{lemma}[Soundness (External Language)]
%  If \ltitSdjudgement{\ltiEnv{}}
%                     {\ltiE{}}
%                     {\ltiT{}}
%                     {\ltiEp{}}
%                     and
%                     \ltiSdsubtypeseen{\varnothing}{\ltiEnv{}}{\ltiT{}}{\ltiS{}}
%                     then
%    \ltitjudgement{\ltiEnv{}}
%                  {\ltiEp{}}
%                  {\ltiS{}}
%                  {\ltiEpp{}}
%\end{lemma}

%\begin{lemma}[Completeness (External Language)]
%  If \ltitjudgement{\ltiEnv{}}
%                   {\ltiE{}}
%                   {\ltiT{}}
%                   {\ltiEp{}}
%                  then
%  \ltitSdjudgement{\ltiEnv{}}
%                     {\ltiE{}}
%                     {\ltiT{}}
%                     {\ltiEp{}}
%\end{lemma}

Ideally, SCL always infers sound annotations and types for external terms.

\begin{conjecture}[SCL Soundness]
  If there exists fuel {\ltiFuel{}} such that
     \ltitSstkjudgement{\ltimakeCombinedThreadedEnv{\ltiFuel{}}{\ltiEmptyClosureCache}}
                       {\ltiEmptyEnv}
                       {\ltiE{}}
                       {\ltiT{}}
                       {\ltimakeCombinedThreadedEnv{\ltiFuelp{}}{\ltiClosureCache{}}}
                       {\ltiEp{}}
  for external term \ltiE{},
                       SCL type \ltiT{},
                       and SCL term \ltiEp{},
                      then
    \ltitjudgementNoElab{}
                  {\ltielimClosLHS{\ltiClosureCache{}}{\ltiEp{}}}
                  {\ltiTp{}}
                   in the internal language,
                  where 
                  \ltiisubtype{}{\ltiTp{}}{\ltielimClosTLHS{\varnothing}{\ltiClosureCache{}}{\ltiT{}}}.
\end{conjecture}

%Several key 

%\begin{conjecture}[SCL Completeness 1]
%  If \ltitjudgementNoElab{\ltiEnv{}}
%                   {\ltiE{}}
%                   {\ltiT{}}
%    then
%    either
%    \forall
%    and fuel \ltiFuel{}
%    such that
%    \ltitSstkjudgementNoElab{\ltimakeCombinedThreadedEnv{\ltiFuel{}}{\ltiEmptyClosureCache}}
%                      {\ltiEnv{}}
%                      {\ltiF{}}
%                      {\ltiT{}}
%                      {\ltimakeCombinedThreadedEnv{\ltiFuelp{}}{\ltiClosureCache{}}}
%                      {\ltiFp{}}.
%\end{conjecture}

%% TODO timeout rules for type system
%% TODO 

Completeness for SCL says that there always exists some amount of annotations
we can add to a term that type checks in the external language so it can type
check in SCL.

\begin{conjecture}[SCL Weak Completeness]
  If \ltitjudgementNoElab{}
                     {\ltiE{}}
                     {\ltiT{}}
                     for external term {\ltiE{}},
                     then
    there exists a term \ltiF{} such that \ltiE{} is a partial erasure of \ltiF{}
    and fuel \ltiFuel{}
    such that
    \ltitSstkjudgementNoElab{\ltimakeCombinedThreadedEnv{\ltiFuel{}}{\ltiEmptyClosureCache}}
                      {\ltiEmptyEnv}
                      {\ltiF{}}
                      {\ltiT{}}
                      {\ltimakeCombinedThreadedEnv{\ltiFuelp{}}{\ltiClosureCache{}}}
                      {\ltiFp{}}.
\end{conjecture}

Since SCL essentially contains the rules of the internal language, we can always
use the annotations chosen by the external language's oracle. This way, we can
erase all symbolic closure introduction rules and reuse previous results for systems
based on \ltiFsub, so this theorem does not say much about symbolic closures.
It would be nice to prove a stronger theorem
based on fuel, such as the following.

\begin{conjecture}[SCL Strong Completeness]
  If \ltitjudgementNoElab{}
                     {\ltiE{}}
                     {\ltiT{}}
                     for external term {\ltiE{}},
                     then
    either, for all initial fuel \ltiFuel{}, SCL gets stuck at a symbolic reduction with zero fuel when checking {\ltiE{}},
    or there exists fuel \ltiFuel{}
    such that
    \ltitSstkjudgementNoElab{\ltimakeCombinedThreadedEnv{\ltiFuel{}}{\ltiEmptyClosureCache}}
                      {\ltiEmptyEnv}
                      {\ltiE{}}
                      {\ltiTp{}}
                      {\ltimakeCombinedThreadedEnv{\ltiFuelp{}}{\ltiClosureCache{}}}
                      {\ltiFp{}},
          and \ltiisubtype{\ltiEnv{}}{\ltielimClosTLHS{\varnothing}{\ltiClosureCache{}}{\ltiTp{}}}{\ltiT{}}.
\end{conjecture}

Unfortunately, this does not hold in the presented model of SCL, at least in part because of the restrictions placed on
symbolic closures and our use of ``off-the-shelf'' type argument synthesis.
Furthermore, we would need to distinguish type errors from running out of fuel.
We now discuss some specific issues we face when trying to achieve a system with Strong Completeness,
and speculate on how to fix them.

The restriction in \ltiSCAbs disallows symbolic closures to cross into a new type variable
scope, like
``\ltilet{\text{f}}{\ltiufun{\ltivar{}}{\ltiE{}}}
       {\ltifun{\ltitvar{}}{\ltivar{}}{\ltitvar{}}{\ltiapp{\text{f}}{\ltivar{}}}}.''
We have considered two fixes, both with their own tradeoffs.
The first is to infer a polymorphic type for $\text{f}$---we simply quantify
over each type variable that occurs out-of-scope with respect to $\text{f}$'s
definition context.
In the present example, instead of recording $\text{f}$'s type as
``\ltiPolyFn{\ltitvar{}}{}{...}'' at the application site,
we would record the polymorphic type
``\ltiPolyFn{\ltitvar{}}{\ltitvar{}}{...}'',
since {\ltitvar{}} is not in scope at $\text{f}$'s definition.
The complication here is related to type argument synthesis.
Since $\text{f}$ now has a polymorphic type, 
the application must elaborate to \ltiappinst{\text{f}}{\ltitvar{}}{\ltivar{}}.
Furthermore, if $\text{f}$ is passed as an argument to a polymorphic function
such as \ltiapp{\text{map}}{\text{f},\ltiE{}},
we would have to also infer type arguments for operands
(like CDuce~\cite{polyduce2}).

Another approach to handling out-of-scope type variables is to 
enrich types with type contexts.
Here, $\text{f}$ would have type 
``(\ltistackmapping{\ltitvarp{}}{\ltiPolyFn{\ltitvarp{}}{}{...}})'',
which says ``in a type context that starts with some
type variable \ltitvar{},
expands to \ltiPolyFn{\ltitvar{}}{}{...}''.
This way, checking the body of ``\ltifun{\ltitvar{}}{\ltivar{}}{\ltitvar{}}{\ltiapp{\text{f}}{\ltivar{}}}''
would effectively update $\text{f}$'s type to 
``\ltiPolyFn{\ltitvar{}}{}{...}''.
This approach resembles contextual subtyping~\cite{Dunfield2004Tridirectional},
except their dependence of types on type contexts is purely syntactic.
Type contexts are eliminated too early to check our example,
because the $\text{f}$'s annotation must occur at its definition, but
there the type environment is empty.

Another major restriction is that we may only pick a single
elaboration for each symbolic closure. This disallows simple programs like 
\ltilet{\text{f}}{\ltiufun{\ltivar{}}{\ltiE{}}}
       {\ltiRec{\text{left}=\ltiapp{\text{f}}{1},
                \text{right}=\ltiapp{\text{f}}{\text{``a''}}}}.
We have considered several approaches to lifting this restriction.
First is to infer an \emph{intersection type} for \text{f}.
That way, \text{f}'s type would be
\ltiIFn{\ltiFn{\text{Int}}{\text{Int}},{\ltiFn{\text{Str}}{\text{Str}}}},
which says ``returns an Int when given an Int, and returns a Str
when given a Str.''
Now choosing the final elaboration for {\ltiE{}} becomes more complicated,
but the intersection type checking literature presents several solutions~\cite{wells2002branching,Dunfield2004Tridirectional,polyduce1}.

We could also try and guess a polymorphic type for \text{f}
based on its use sites,
in an approach resembling Trace Typing~\cite{Andreasen2016TraceTA}.
Here, say we know 
\text{f} is type \ltiFn{\text{Int}}{\text{Int}}
and we learn it is also of type \ltiFn{\text{Str}}{\text{Str}},
we could guess a generalization like
``\ltiPolyFn{\ltitvar{}}{\ltitvar{}}{\ltitvar{}}''
based on the shape of both observations.
Since the definition type environment of \text{f} is always handy
in the elaboration cache, we could check if \text{f} inhabits
the generalized type (Trace Typing only checks use sites to
verify a polymorphic type).

The restriction in \ltiSCAppInfPT that type argument synthesis
may not reason about symbolic closures rules out checking
\ltiapp{\text{map}}{\ltiufun{\ltivar{}}{\ltiF{}}, \ltiE{}}.
Supporting symbolic closures in type argument synthesis requires
a following a notion of data flow in polymorphic types.
For example, the free theorems~\cite{wadler1989theorems}
of the type for \text{map} implies that
its function argument may be invoked only with elements of
its collection argument.
We can visualize the data flow \text{map} must adhere to.

\[
\ltiPolyFn{\ltiFn{\text{a}_{\tikz[overlay,remember picture] \node [] (b) {};}}
                 {\text{b}^{\tikz[overlay,remember picture] \node [] (c) {};}}
                 ,
          {\text{List[a}_{\tikz[overlay,remember picture] \node [] (a) {};}
                 \text{]}}
                 }
          {\text{a,b}}
          {\text{List[}^{\tikz[overlay,remember picture] \node [] (d) {};}
           \text{b]}}
\begin{tikzpicture}[remember picture, overlay,
                  text width = 2.5cm ]
  \coordinate (Start1) at (a);
  \coordinate (End1) at (b);
  \coordinate (Start2) at (c);
  \coordinate (End2) at (d);
  \draw[red,->,bend right=-45](Start1.south) to (End1.east);
  \draw[red,->,bend right=-45](Start2.east) to (End2.north east);
\end{tikzpicture} 
\]

Now, the role of type argument synthesis is to integrate symbolic closures
into this data flow.
For our example, the type of \ltiE{} flows to \text{List[a]},
which informs the symbolic closure of its input, and then \text{List[b]}
is derived from the output of the symbolic closure.

\[
\begin{array}{c}
\ltiapp{\text{map}}{\ltiufun{{\tikz[overlay,remember picture] \node [] (b2) {};}\ltivar{}}
                             {\ltiF{}}{\tikz[overlay,remember picture] \node [] (c0) {};},
                             \ltiE{}{\tikz[overlay,remember picture] \node [] (a0) {};}} \\\\
\ltiPolyFn{\ltiFn{\text{a}_{\tikz[overlay,remember picture] \node [] (b) {};}}
                 {\text{b}^{\tikz[overlay,remember picture] \node [] (c) {};}}
                 ,
          {\text{List[a}_{\tikz[overlay,remember picture] \node [] (a) {};}
                 \text{]}}
                 }
          {\text{a,b}}
          {\text{List[}^{\tikz[overlay,remember picture] \node [] (d) {};}
           \text{b]}}
\begin{tikzpicture}[remember picture, overlay,
                  text width = 2.5cm ]
  \coordinate (Start0) at (a0);
  \coordinate (End0) at (a);
  \coordinate (Start1) at (a);
  \coordinate (End1) at (b);
  \coordinate (Start1b) at (b);
  \coordinate (End1b) at (b2);
  \coordinate (Start2a) at (c0);
  \coordinate (End2a) at (c);
  \coordinate (Start2) at (c);
  \coordinate (End2) at (d);
  \draw[blue,dotted,thick,->](Start0.south) to (End0.east);
  \draw[red,->,bend right=-45](Start1.south) to (End1.east);
  \draw[blue,dotted,thick,->](Start1b.south) to (End1b.east);
  \draw[blue,dotted,thick,->](Start2a.south) to (End2a.east);
  \draw[red,->,bend right=-45](Start2.east) to (End2.north east);
\end{tikzpicture} 
\end{array}
\]

Formalizing this intuition into a type-argument synthesis algorithm is work-in-progress.

% Recursive types
Our SCL model does not feature equi-recursive recursive types. Adding them would allow us to
lift the restriction that a symbolic closure may not be passed to itself.
In the elaboration phase, we introduce a recursive type when we discover
a symbolic closure that we are currently elaborating.
For example, the program ``\ltilet{\ltivar{}}{\ltiufun{\ltivar{}}{\ltivar{}}}{\ltiapp{\ltivar{}}{\ltivar{}}}'',
{\ltivar{}} is passed to itself.
Using this method of elaboration, {\ltivar{}} will be
assigned the type
\ltiFn{\ltiMu{\ltitvar{}}{\ltiFn{\ltitvar{}}{\ltitvar{}}}}
      {\ltiMu{\ltitvar{}}{\ltiFn{\ltitvar{}}{\ltitvar{}}}},
where
{\ltiMu{\ltitvar{}}{\ltiFn{\ltitvar{}}{\ltitvar{}}}}
stands for
the given type of {\ltivar{}}.


Distinguishing between a type error
and running out of fuel in SCL would significantly complicate the model.
One way this might be able to be achieved is based on a technique for
proving type soundness for big-step reduction relations.
For each type and subtype rule, we add extra rules to explicitly
handle all cases where the derivation can ``get stuck,''
and return a special fuel value denoting ``ran out of fuel'' in rules
that get stuck when checking for sufficient fuel.
Then, another set of rules will propagate this special fuel value
back to the root of the derivation, which allows us to distinguish
the two cases.

% something totally different? Symbolic closures as primitives + don't go wrong

Up till now, our conjectures and discussion have been centered around compiling SCL to \ltiFsub.
Another direction we could take is to lift all restrictions
on SCL (except symbolic reduction fuel, to keep it decidable),
ignore elaborations, and treat SCL as a type \emph{checker} rather than a
type inferencer.
Then, we could attempt to prove a type soundness theorem
like the following.

\begin{conjecture}[Unrestricted SCL Type Soundness]
  If there exists fuel {\ltiFuel{}} such that
     \ltitSstkjudgementNoElab{\ltiFuel{}}
                       {\ltiEmptyEnv}
                       {\ltiE{}}
                       {\ltiT{}}
                       {\ltiFuelp{}}
                       {\ltiEp{}}
  for external term \ltiE{}, and
                       SCL type \ltiT{},
                      then
  evaluating {\ltiE{}}
                       yields a value $v$, whose type \ltiTp{}
                       is a subtype of
                       {\ltiT{}} up-to symbolic closure types.
\end{conjecture}

This better reflects the original intended use-case of symbolic closures:
optional type systems with evaluation semantics based on type-erasure.
Indeed, if symbolic closures were to be added to a language like Typed Clojure,
there is no compelling reason to implement elaboration rules and so it is
natural to leave symbolic closures unrestricted. While we believe unrestricted
symbolic closures are type sound, we are unsure how to approach proving such
a theorem.

% - Solution
%   - "obvious" function annotations
%     - can be derived from usage context
%   - introduce "symbolic" closure types
%     - a function's type is its code + typed local scope
%   - don't need to check a function that isn't called
% - Constraints
%   - wildcard "?" type
%     - needed to provide argument types while inferring body
%     - from Colored LTI
%   - Infinite loops
%     - subtyping
%     - type generalization
%     - term reduction limits
%   - user-level story
%     - symbolic closures enabled by flag
%     - users cannot write a symbolic closure
%     - that way, global annotations cannot contain a symbolic closure
%       - helps with polymorphism story
%         - constraint solving
%           - hypothesis/goal: only one side of contraint solving can have a symbolic closure
%             - one side is from global annotation, other side from local inference
%     - compatible with occurrence typing
%   - reporting errors
%     - suggesting types
%     - avoid showing inlining to users
%   - checking fn with arguments at type Bot is equivalent (?) to not checking at all
%     - what about strange disjoint ordered intersection types like `into`
%     - do they break? do they need initial Bot arities?
%   - 0-n checks to same function
%     - avoid double expansion
%       - many copies of symbolic closures are made, could be expanded at different times
%         - how to synchronize?
%     - skipping unreachable functions
%       - potential for latent bugs, if type checker turned off in future and fn is made reachable
%     - performance
%   - consistent evaluation results
%     - how to ensure correct inlining?
%     - relationship between inlined and evaluated code?
%       - do we want to "undo" the inlining when finally evaluating?
%   - when to use a closure type?
%     - partial annotations
%   - polymorphism
%     - postpone discussion to next chapter
%   - applying symbolic analysis to infer loop/recur annotations
%     - similar issues
%     - different type generalization story?
%   - comparison to let-polymorphism
%     - expressiveness
%     - performance
%   - help check macros?
%     - not directly applicable, since too much context would be lost
%       - would help check *more* of an expansion, but error messages
%         are still unrelated to original code
%   - is this a sound strategy?
%     - faithfully simulates beta reduction
%     - termination story?
%   - case studies
%     - criteria:
%       - good errors?
%       - predictable behavior?
%       - performance?
%     - simple eta expansions
%       - (+ 1 2)
%       - ((fn [x y] (+ x y)) 1 2)
%     - let-bound functions
%       - (+ 1 2)
%       - (let [plus (fn [x y] (+ x y))]
%           (plus 1 2))
%     - y-combinator
%       - stress test
%     - let-polymorphism worst case (exponential) comparison
%       - stress test
%     - completely inlined transducers
%       - case study: inlining map + comp
%         - why: non recursive polymorphic functions
%           - common idiom
%         - how to report errors?
%   - polymorphic function-intersection types
%     - how to handle, do we need backtracking?
%     - do we need to recheck arguments? 

%\chapter{Type Argument Synthesis with Symbolic Closures}
\label{chapter:symbolic:directed-lti}

\subsection{Type-argument synthesis for the Symbolic Closure Language}

\begin{figure}
  \begin{mathpar}
    \infer[AppInf]
    {
    \ltitSstkjudgementNoElab{\ltiCombinedThreadedEnv{1}}
                      {\ltiEnv{}}
                      {\ltiF{}}
                      {\ltiPoly{\ova{\ltitvar{}}}
                               {\ltiFn{\ltiT{}}{\ltiS{}}}}
                      {\ltiCombinedThreadedEnv{2}}
                      {\ltiFp{}}
                  \\
    \ltitSstkjudgementNoElab{\ltiCombinedThreadedEnv{2}}
                      {\ltiEnv{}}
                      {\ltiE{}}
                      {\ltiTp{}}
                      {\ltiCombinedThreadedEnv{3}}
                      {\ltiEp{}}
                  \\
                       |\ova{\ltitvar{}}|>0
           \\
           \ltigenconstraint{\varnothing}{\ova{\ltitvar{}}}{\ltiTp{}}{\ltiT{}}{\ltiCp{}}
           \\
           \ltiprocessDelays{\ltiCombinedThreadedEnv{3}}
                            %{\ltiEnvConcatParen{\ltiEnv{}}{\ova{\ltitvar{}}}}
                            {\ltiCp{}}
                            {\ltiC{}}
                            {\ltiCombinedThreadedEnv{4}}
           \\
           \ltiSubst{\ltiC{}}{\ltiFn{\ltiT{}}{\ltiS{}}}{\ltisubst{}}
    }
    {
    \ltitSstkjudgementNoElab{\ltiCombinedThreadedEnv{1}}
                      {\ltiEnv{}}
                      {\ltiapp{\ltiF{}}{\ltiE{}}}
                      {\ltiApplySubst{\ltisubst{}}{\ltiS{}}}
                      {\ltiCombinedThreadedEnv{4}}
                      {\ltiappinst{\ltiFp{}}
                                  {\ova{\ltiApplySubst{\ltisubst{}}
                                                      {\ltitvar{}}}}
                                  {\ltiEp{}}}
    }
  \end{mathpar}
  \caption{Type argument synthesis for the Symbolic Closure Language}
\end{figure}

\begin{figure}
$$
\begin{array}{lrll}
  \ltiC{} &::=& \ltiCSet{\ova{\ltiCEntry{\ltiT{}}{\ltitvar{}}{\ltiT{}}}\ 
                         \ova{\ltiDEntryVX{\ltiV{}}{\ova{\ltitvar{}}}{\ltiT{}}{\ltiT{}}}
                          }
                      &\mbox{Constraint sets}\\
                      % TODO talk about X/V constraint sets
   \ltiCEmpty &\Leftrightarrow& \ltiCSet{\ova{\ltiCEntry{\ltiBot}{\ltitvar{}}{\ltiTop}}}
                      &\mbox{Constraint abbreviations}
\end{array}
$$
  \caption{Syntax for Constraint generation}
\end{figure}

\begin{figure}
  \begin{mathpar}
    \infer [CG-Top]
    {}
    {
    \ltigenconstraint{\ltiV{}}{\overline{\ltitvar{}}}{\ltiT{}}{\ltiTop}{\ltiCEmpty}
    }

    \infer [CG-Upper]
    {
    \ltitvar{1} \in \overline{\ltitvar{}}
    \\
    \ltidemote{\ltiS{}}{\ltiV{}}{\ltiT{}}
    \\\\
    \ltitv{\ltiS{}} \cap \overline{\ltitvar{}} = \varnothing
    }
    {
    \ltigenconstraint{\ltiV{}}{\overline{\ltitvar{}}}{\ltitvar{1}}{\ltiS{}}
                     {\ltiCSet{\ltiCEntry{\ltiBot}{\ltitvar{1}}{\ltiT{}}}}
    }

    \infer [CG-Lower]
    {
    \ltitvar{1} \in \overline{\ltitvar{}}
    \\
    \ltipromote{\ltiS{}}{\ltiV{}}{\ltiT{}}
    \\\\
    \ltitv{\ltiS{}} \cap \overline{\ltitvar{}} = \varnothing
    }
    {
    \ltigenconstraint{\ltiV{}}{\overline{\ltitvar{}}}{\ltiS{}}{\ltitvar{1}}
                     {\ltiCSet{\ltiCEntry{\ltiT{}}{\ltitvar{1}}{\ltiTop}}}
    }

    \infer [CG-Bot]
    {}
    {
    \ltigenconstraint{\ltiV{}}{\overline{\ltitvar{}}}{\ltiBot}{\ltiT{}}{\ltiCEmpty}
    }
    \ \ \ 
    %
    \infer [CG-Refl]
    {
      \ltitvarp{}
      \not\in
      \overline{\ltitvar{}}
    }
    {
    \ltigenconstraint{\ltiV{}}
                     {\overline{\ltitvar{}}}
                     {\ltitvarp{}}
                     {\ltitvarp{}}
                     {\ltiCEmpty}
    }

    \infer [CG-Fun]
    {
    |\ova{\ltitvar{}}|>0 \text{ implies \ova{\ltiTp{}}, \ova{\ltiT{}} contain no symbolic closures}
    \\
    \ltigenconstraint{\ltiV{} \cup \overline{\ltitvar{}}}
                     {\ova{\ltitvarp{}}}
                     {\ltiT{1}}
                     {\ltiTp{1}}
                     {\ltiC{1}}
    \\
    \ltigenconstraint{\ltiV{} \cup \overline{\ltitvar{}}}
                     {\ova{\ltitvarp{}}}
                     {\ltiT{2}}
                     {\ltiTp{2}}
                     {\ltiC{2}}
    \\\\
    \overline{\ltitvar{}} \cap (\ltiV{} \cup \overline{\ltitvarp{}}) = \varnothing
    }
    {
    \ltigenconstraint{\ltiV{}}
                     {\overline{\ltitvarp{}}}
                     {\ltiPolyFn{\ltiTp{1}}{\ova{\ltitvar{}}}{\ltiT{2}}}
                     {\ltiPolyFn{\ltiT{1}}{\ova{\ltitvar{}}}{\ltiTp{2}}}
                     {\ltiCIntersect{\ltiC{1}}{\ltiC{2}}}
    }

    \infer [CG-Closure]
    {}
    {
    \ltigenconstraint{\ltiV{}}
                     {\overline{\ltitvarp{}}}
                     {\ltiClosureWithStkID{\ltiEnv{}}{\ltiClosureID{}}{\ltiufun{\ltivar{}}{\ltiE{}}}}
                     {\ltiPolyFn{\ltiS{}}{\ova{\ltitvar{}}}{\ltiT{}}}
                     {\ltiCSet{\ltiDEntryVX{\ltiV{}}
                                         {\overline{\ltitvarp{}}}
                                         {\ltiClosureWithStkID{\ltiEnv{}}{\ltiClosureID{}}{\ltiufun{\ltivar{}}{\ltiE{}}}}
                                         {\ltiPoly{\ova{\ltitvar{}}}{\ltiFn{\ltiS{}}{\ltiT{}}}}}}
    }
  \end{mathpar}
  \caption{Constraint generation system
                 \ltigenconstraint{\ltiV{}}{\overline{\ltitvar{}}}{\ltiT{}}{\ltiS{}}{\ltiC{}}
                 where $\ltiV{} \cap {\overline{\ltitvar{}}} = \varnothing$.
  }
\end{figure}

\begin{figure}
  \begin{mathpar}

    \boxed{
    \infer[]
    {
      \ltiprocessDelays{\ltiCombinedThreadedEnv{}}
                       {\ltiC{}}
                       {\ltiCp{}}
                       {\ltiCombinedThreadedEnvp{}}
      \\\\
      \text{Process all delayed constraints in
                       \ltiC{}, yielding a new constraint set \ltiCp{}.
      }
    }
    {}
    }

    \infer[]
    {
      \ltiorderDelays{\ltiC{0}}
                     {\ova
                       {\ltiDEntryVX{\ltiV{}}
                                    {\ova{\ltitvarp{}}}
                                    {\ltiT{}}
                                    {\ltiS{}}}^n}
                                    \\
      \overrightarrowcaption{
      \ltiprocessDelay{\ltiCombinedThreadedEnv{i-1}}
                      {\ltiDEntryVX{\ltiV{i}}
                                   {\ova{\ltitvarp{}}_i}
                                   {\ltiT{i}}
                                   {\ltiS{i}}}
                      {\ltiC{i-1}}
                      {\ltiC{i}}
                      {\ltiCombinedThreadedEnv{i}}
                      }^{1 \leq i \leq n}
    }
    {
      \ltiprocessDelays{\ltiCombinedThreadedEnv{0}}
                       {\ltiC{0}}
                       {\ltiC{n}}
                       {\ltiCombinedThreadedEnv{n}}
    }

    \boxed{
    \infer[]
    {
      \ltiorderDelays{\ltiC{}}{\ova{\ltiDEntryVX{\ltiV{}}{\ova{\ltitvar{}}}{\ltiT{}}{\ltiS{}}}}
      \\\\
      \text{Returns the delayed constraints in constraint set \ltiC{} topologically
      sorted by type variable dependency.
      }
      \\\\
      \text{eg. \ltiDEntryVX{}{}{...}{\ltiFn{\ltitvar{1}}{\ltitvar{2}}}
      goes before 
            \ltiDEntryVX{}{}{...}{\ltiFn{\ltitvar{2}}{\ltitvar{3}}}
      }
    }
    {}
    }

    \infer[]
    {
      \ova{\ltitvar{}}^m = \ova{\ltitvarp{}}_1 = \ova{\ltitvarp{}}_2 = ... = \ova{\ltitvarp{}}_{i-1} = \ova{\ltitvarp{}}_i
      \\
      \forall i \in 1...n, j \in 1...m.
          \{\ltivariance{\ltitvar{j}}{\ltiS{i}}, \ltivariance{\ltitvar{j}}{\ltiT{i}}\} \subseteq \{\ltivconstant, \ltivcovariant\}
      \\
      \text{let \ova{k} be a permutation of 1...n st. }
        \forall i,j \in 1...n.
        \text{ if }
        \ltitv{\ltiT{{k_i}}} \cap \ltitv{\ltiS{{k_j}}} \cap \ova{\ltitvarp{}} \not= \varnothing
        \text{ then }
        % not \leq, eg. [a -> a] depends on itself
        i < j 
    }
    {
      \ltiorderDelays{\ltiCSet{...,\ova{\ltiDEntryVX{\ltiV{}}
                                             {\ova{\ltitvarp{}}^m}
                                             {\ltiClosureWithStkID{\ltiEnv{}}{\ltiClosureID{}}{\ltiufun{\ltivar{}}{\ltiE{}}}}
                                             {\ltiFn{\ltiS{}}{\ltiT{}}}}^n}}
                                             {
      [\ltiDEntryVX{\ltiV{}}
                   {\ova{\ltitvarp{}}}
                   {\ltiClosureWithStkID{\ltiEnv{}}{\ltiClosureID{}}{\ltiufun{\ltivar{}}{\ltiE{}}}}
                   {\ltiFn{\ltiS{}}{\ltiT{}}}_i
                   |
                   i \in \ova{k}]
                   }
    }

    \boxed{
    \infer[]
    {
      \ltiprocessDelay{\ltiCombinedThreadedEnv{}}
                      {\ltiDEntryVX{\ltiV{}}
                                   {\ova{\ltitvarp{}}}
                                   {\ltiS{}}
                                   {\ltiT{}}}
                      {\ltiC{}}
                      {\ltiCp{}}
                      {\ltiCombinedThreadedEnvp{}}
      \\\\
      \text{Process delayed constraint 
                      {\ltiDEntryVX{\ltiV{}}
                                   {\ova{\ltitvarp{}}}
                                   {\ltiS{}}
                                   {\ltiT{}}},
                                   with current constraint set \ltiC{},
      }
      \\\\
      \text{yielding a new constraint set \ltiCp{}.}
    }
    {}
    }

    \infer[]
    {
    \ltitv{\ltiE{}} \cap \ova{\ltitvarpp{}} = \varnothing
    \\
            0 < \ltiFuel{} \\
            \ltiSubst{\ltiC{}}{\ltiPolyFn{\ltiS{}}{}{\ltiT{}}}{\ltisubst{}}\\
            \ltitSstkjudgement{\ltimakeCombinedThreadedEnv{\ltiFuel{}-1}
                                                          {\ltiClosureCache{1}}}
                              {\ltiEnvConcat{\ltiEnv{}}
                                            {\hastype{\ltivar{}}
                                                     {\ltiApplySubst{\ltisubst{}}{\ltiS{}}}}}
                              {\ltiE{}}
                              {\ltiTp{}}
                              {\ltimakeCombinedThreadedEnv{\ltiFuelp{}}
                                                          {\ltiClosureCache{2}}}
                              {\ltiFpp{}}
                              \\
          \ltiupdateClosureCacheSingle{\ltiClosureCache{2}}
                                {\ltiClosureID{}}
                                {\ltifuntparamargrettype
                                 {\ova{\ltitvarpp{}}}
                                 {\ltivar{}}
                                 {\ltiApplySubst{\ltisubst{}}{\ltiS{}}}
                                 {\ltiTp{}}
                                 {\ltiFpp{}}}
                                {\ltiClosureCache{3}}
                                \\
          \ltigenconstraint{\ltiV{} \cup \overline{\ltitvarpp{}}}
                           {\ova{\ltitvar{}}}
                           {\ltiTp{}}
                           {\ltiApplySubst{\ltisubst{}}{\ltiT{}}}
                           {\ltiCpp{}}
                           \\
                           \ltiCpp{} \text{ does not contain delayed constraints}
    }
    {
      \ltiprocessDelay{\ltimakeCombinedThreadedEnv{\ltiFuel{}}{\ltiClosureCache{1}}}
                      {\ltiDEntryVX{\ltiV{}}
                                   {\ova{\ltitvarp{}}}
                                   {\ltiClosureWithStkID{\ltiEnv{}}{\ltiClosureID{}}{\ltiufun{\ltivar{}}{\ltiE{}}}}
                                   {\ltiPolyFn{\ltiS{}}{\ova{\ltitvarpp{}}}{\ltiT{}}}}
                      {\ltiC{}}
                      {\ltiCIntersect{\ltiC{}}{\ltiCpp{}}}
                      {\ltimakeCombinedThreadedEnv{\ltiFuelp{}}{\ltiClosureCache{3}}}
    }

  \end{mathpar}
  \caption{Processing delayed constraints
  }
\end{figure}

{
\begin{lstlisting}[language=ml,mathescape=true]
type Option[a] = {match : $\ltiPoly{\text{r}}%
                                   {\ltiFn{\text{OptionVisitor[a,r]}}%
                                          {\text{r}}}$}
type OptionVisitor[a,r] =
  {caseNone : $\ltiFn{}{\text{r}}$,
   caseSome : $\ltiFn{\text{a}}{\text{r}}$}
\end{lstlisting}
}

{
\begin{lstlisting}[language=ml,mathescape=true]
None = $\ltifuntparaminterfaceLHS{\text{s}}%
                                 {\ltiFn{}%
                                        {\text{Option[s]}}}%
                                 {}$
         {match = $\ltiufun{\text{v}}%
                           {\text{v.caseNone()}}$}
(* SC annotated *)
(* $\ltiInferred{\ltiClosureCache{} =%
     \ltiClosureCacheEntry{\text{c1}}%
                          {\ltiClosure{\text{s}}%
                                      {\ltiNotInferred{\ltifuntparaminterface{\ltiInferred{\text{r}}}%
                                                                             {\ltiInferred{\ltiFn{\text{OptionVisitor[s,r]}}{\text{r}}}}%
                                                                             {\text{v}}%
                                                                             {\text{v.caseNone()}}}}}}$ *)
None = $\ltifuntparaminterfaceLHS{\text{s}}%
                                 {\ltiFn{}{\text{Option[s]}}}%
                                 {}$
         {match = $\ltiufunelab{\ltiInferred{\text{c1}}}%
                               {\text{v}}%
                               {\text{v.caseNone()}}$}
(* fully annotated *)
None = $\ltifuntparaminterfaceLHS{\text{s}}%
                                 {\ltiFn{}%
                                        {\text{Option[s]}}}%
                                 {}$
         {match = $\ltifuntparaminterface{\ltiInferred{\text{r}}}%
                                         {\ltiInferred{\ltiFn{\text{OptionVisitor[s,r]}}{\text{r}}}}%
                                         {\text{v}}%
                                         {\text{v.caseNone()}}$}
\end{lstlisting}
}

{
\begin{lstlisting}[language=ml,mathescape=true]
Some = $\ltifuntparaminterfaceLHS{\text{t}}{\ltiFn{\text{t}}{\text{Option[t]}}}{\text{y}}$
         {match = $\ltiufun{\text{v}}{\text{v.caseSome(y)}}$}
(* SC annotated *)
(* $\ltiInferred{\ltiClosureCache{} = \ltiClosureCacheEntry{\text{c1}}{\ltiClosure{\ltiEnvConcat{\text{t}}{\hastype{\text{y}}{\text{t}}}}{\ltiNotInferred{\ltifuntparaminterface{\ltiInferred{\text{r}}}{\ltiInferred{\ltiFn{\text{OptionVisitor[t,r]}}{\text{r}}}}{\text{v}}{\text{v.caseSome(y)}}}}}}$ *)
Some = $\ltifuntparaminterfaceLHS{\text{t}}{\ltiFn{\text{t}}{\text{Option[t]}}}{\text{y}}$
         {match = $\ltiufunelab{\ltiInferred{\text{c1}}}{\text{v}}{\text{v.caseSome(y)}}$}
(* fully annotated *)
Some = $\ltifuntparaminterfaceLHS{\text{t}}{\ltiFn{\text{t}}{\text{Option[t]}}}{\text{y}}$
         {match = $\ltifuntparaminterface{\ltiInferred{\text{r}}}{\ltiInferred{\ltiFn{\text{OptionVisitor[t,r]}}{\text{r}}}}{\text{v}}{\text{v.caseSome(y)}}$}
\end{lstlisting}
}

{
\begin{lstlisting}[language=ml,mathescape=true]
map = $\ltifuntparaminterfaceLHS{\text{c,d}}{\ltiFn{\ltiFn{\text{c}}{\text{d}},\text{Option[c]}}{\text{Option[d]}}}{\text{f,x}}$
        x.match({caseNone = $\ltiufun{}{\text{None()}}$,
                 caseSome = $\ltiufun{\text{y}}{\text{Some(f(y))}}$})
(* SC annotated *)
(* $\ltiInferred{\ltiEnv{} = {\ltiEnvConcat{\text{c}}{\ltiEnvConcat{\text{d}}{\ltiEnvConcat{\hastype{\text{f}}{\ltiFn{\text{c}}{\text{d}}}}{\hastype{\text{x}}{\text{Option[c]}}}}}}}$ *)
(* $\ltiInferred{\ltiClosureCache{} =}$
     $\ltiInferred{\ltiClosureCacheEntry{\text{c1}}{\ltiClosure{\ltiEnv{}}{\ltiNotInferred{\ltifuninterface{\ltiInferred{\ltiFn{}{\ltilstOption{\ltiBot}}}}{}{\text{None[\ltiInferred{\ltiBot}]()}}}}}}$,
     $\ltiInferred{\ltiClosureCacheEntry{\text{c2}}{\ltiClosure{\ltiEnv{}}{\ltiNotInferred{\ltifuninterface{\ltiInferred{\ltiFn{\text{c}}{\ltilstOption{\text{d}}}}}{\text{y}}{\text{Some[\ltiInferred{\text{d}}](f(y))}}}}}}$
*)
map = $\ltifuntparaminterfaceLHS{\text{c,d}}{\ltiFn{\ltiFn{\text{c}}{\text{d}},\text{Option[c]}}{\text{Option[d]}}}{\text{f,x}}$
        x.match[$\ltiInferred{\text{d}}$]({caseNone = $\ltiufunelab{\ltiInferred{\text{c1}}}{}{\text{None()}}$,
                    caseSome = $\ltiufunelab{\ltiInferred{\text{c2}}}{\text{y}}{\text{Some(f(y))}}$})
(* fully annotated *)
map = $\ltifuntparaminterfaceLHS{\text{c,d}}{\ltiFn{\ltiFn{\text{c}}{\text{d}},\text{Option[c]}}{\text{Option[d]}}}{\text{f,x}}$
        x.match[$\ltiInferred{\text{d}}$]({caseNone = $\ltifuninterface{\ltiInferred{\ltiFn{}{\ltilstOption{\ltiBot}}}}{}{\text{None[\ltiInferred{\ltiBot}]()}}$,
                    caseSome = $\ltifuninterface{\ltiInferred{\ltiFn{\text{c}}{\ltilstOption{\text{d}}}}}{\text{y}}{\text{Some[\ltiInferred{\text{d}}](f(y))}}$})
\end{lstlisting}
}

{
\begin{lstlisting}[language=ml,mathescape=true]
map($\ltiufun{\text{y}}{\text{1+y}}$, Some(42))
(* SC annotated *)
(* $\ltiInferred{\ltiClosureCache{} =%
      \ltiClosureCacheEntry{\text{c1}}%
                           {\ltiClosure{\ltiEmptyEnv}%
                                       {\ltiNotInferred%
                                        {\ltifuninterface{\ltiInferred{\ltiFn{\text{Int}}{\text{Int}}}}%
                                                         {\text{y}}%
                                                         {\text{1+y}}}}}}$ *)
map[$\ltiInferred{\text{Int, Int}}$]($\ltiufunelab{\ltiInferred{\text{c1}}}{\text{y}}{\text{1+y}}$, Some[$\ltiInferred{\text{Int}}$](42))
(* fully annotated *)
map[$\ltiInferred{\text{Int, Int}}$]($\ltifuninterface{\ltiInferred{\ltiFn{\text{Int}}{\text{Int}}}}{\text{y}}{\text{1+y}}$, Some[$\ltiInferred{\text{Int}}$](42))
\end{lstlisting}
}



{
\begin{lstlisting}[language=ml,mathescape=true]
id = $\ltifuntparaminterface{\text{a}}{\ltiFn{\text{a}}{\text{a}}}{\text{x}}{\text{x}}$

let app = $\ltiufun{\text{f},\text{x}}{\ltiapp{\text{f}}{\text{x}}}$ in
  $\ltiapp{\text{app}}%
          {\text{id}, \text{1}}$
(* SC annotated *)
(* $\ltiInferred{\ltiClosureCache{} =%
      \ltiClosureCacheEntry{\text{c1}}%
                           {\ltiClosure{\ltiEmptyEnv}%
                                       {\ltiNotInferred%
                                        {\ltifuninterface{\ltiInferred{\ltiFn{\ltiPoly{\text{a}}{\ltiFn{\text{a}}{\text{a}}},\text{Int}}%
                                                                             {\text{Int}}}}%
                                                         {\text{f,x}}%
                                                         {\ltiappinst{\text{f}}{\ltiInferred{\text{Int}}}{\text{x}}}}}}}$ *)
let app = $\ltiufunelab{\text{c1}}{\text{f},\text{x}}{\ltiapp{\text{f}}{\text{x}}}$ in
  $\ltiapp{\text{app}}%
          {\text{id}, \text{1}}$
(* Fully annotated *)
let app = $\ltifuninterface{\ltiInferred{\ltiFn{\ltiPoly{\text{a}}{\ltiFn{\text{a}}{\text{a}}},\text{Int}}%
                                               {\text{Int}}}}%
                           {\text{f},\text{x}}%
                           {\ltiappinst{\text{f}}{\ltiInferred{\text{Int}}}{\text{x}}}$ in
  $\ltiapp{\text{app}}%
          {\text{id}, \text{1}}$
\end{lstlisting}
}


%\chapter{Recursive types and Intersection types}

\subsection{Compiling to \ltiFsub}

There are a least two ways to approach our external language
with symbolic closures compiling to \ltiFsub.
The main challenge and incentive is to remove all symbolic
closure types from our program and to replace them with
explicit annotations.
Our first approach is to copy function code and annotate
parameter types at each usage.
This approach seems promising, except there is a tension between
copying function code and creating runtime closures.
Our second approach is to extend \ltiFsub with intersection types
and ascribe each function an intersection type that 
describes all the ways it was used as a symbolic closure.

We now explore our first approach, to copy function code as they
are checked.
First, for those functions ascribed symbolic closure types, we
copy their code and insert them at their usage sites.
For example, \clj{f} in 

\begin{lstlisting}[language=Clojure]
(let [f (fn [x] x)]
  (f 1)
  (f "a"))
\end{lstlisting}

would be given a symbolic closure type, and
the program would be expanded like so:

\begin{lstlisting}[language=Clojure]
(let [f (fn [x] x)]
  ((fn [x] x) 1)
  ((fn [x] x) "a"))
\end{lstlisting}

First, we notice that all occurrences of \clj{f} have disappeared,
so it is safe to assume \clj{f} will never be called. It seems
reasonable to us to annotate its argument as \clj{Bot}.

\begin{lstlisting}[language=Clojure]
(let [f (fn [x :- Bot] x)]
  ((fn [x] x) 1)
  ((fn [x] x) "a"))
\end{lstlisting}

Next, the two unannotated functions would be checked as symbolic closures.
These checks would succeed, and then we could ascribe a parameter type on
each function.

\begin{lstlisting}[language=Clojure]
(let [f (fn [x :- Bot] x)]
  ((fn [x :- Int] x) 1)
  ((fn [x :- Str] x) "a"))
\end{lstlisting}

This program type checks with the rules of \ltiFsub.
However, subtle variations on this program are much
more puzzling to account for.

If \clj{f} closes over a variable, like \clj{v}
here

\begin{lstlisting}[language=Clojure]
(let [f (let [v 1] (fn [x] (print v) x))]
  (f 1)
  (f "a"))
\end{lstlisting}

then simply copying its function code will
not suffice.
If we do so, \clj{v} is no longer in scope:

\begin{lstlisting}[language=Clojure]
(let [f (let [v 1] (fn [x] (print v) x))]
  ((fn [x] (print v) x) 1)
  ((fn [x] (print v) x) "a"))
\end{lstlisting}

We could imagine inlining the value of 
\clj{v} to work around this issue, but this is not a
full solution.
If instead, \clj{v} was an annotated function parameter
as in the next example

\begin{lstlisting}[language=Clojure]
(let [f (fn [v :- Int]
          (fn [x] (print v) x))]
  ((f 42) 1)
  ((f 42) "a"))
\end{lstlisting}

it's even unclear how to inline the
unannotated function at all.
It seems the only choice is to inline \clj{f}
in its entirety, regardless if it was a symbolic closure.

\begin{lstlisting}[language=Clojure]
(let [f (fn [v :- Int]
          (fn [x] (print v) x))]
  (((fn [v :- Int]
      (fn [x] (print v) x))
    42)
   1)
  (((fn [v :- Int]
      (fn [x] (print v) x))
    42)
   "a"))
\end{lstlisting}

This way we can at least annotate the missing \clj{Int} and \clj{Str}
parameter types.
It is now tempting to inline \emph{all} local variables with their
definitions.
This doesn't work in a language with side effects.
For example, inlining \clj{f} to ``fix'' the issues
in one of our previous examples would repeat side
effects, like printing \clj{"I only print once"} in the following
program (it, instead, prints thrice if \clj{f} is inlined).

\begin{lstlisting}[language=Clojure]
(let [f (let [v 1]
          (print "I only print once.")
          (fn [x] (print v) x))]
  (f 1)
  (f "a"))
\end{lstlisting}

Symbolic closures can also get their types by being passed
to annotated functions.
For example, \clj{id}
gets the type \clj{[Int -> Int]}
by being passed to \clj{f}, whose definition we will treat
as opaque, emulating a top-level function.

\begin{lstlisting}[language=Clojure]
(let [f (fn [g :- [Int -> Int]] ...)
      id (fn [x] x)]
  (f id))
\end{lstlisting}

By inlining \clj{id} and annotating its parameter \clj{Int},
this program does not pose any particular challenge.

\subsubsection{Intersection types}

If we allow ordered function intersection types,
featured in several optional type systems,
we arrive at an impasse.
Here, we assert that \clj{g} must \emph{both}
be
\clj{[Int -> Int]}
and
\clj{[Num -> Num]}.

\begin{lstlisting}[language=Clojure]
(let [f (fn [g :- (IFn [Int -> Int] [Num -> Num])] ...)
      id (fn [x] x)]
  (f id))
\end{lstlisting}

Clearly \clj{id} inhabits both these types,
however attempting to inline its definition
gets us nowhere.
Introducing intersection types gives us
one more trick up our sleeve: ascribing
\clj{id} as an \emph{intersection} of function types.

\begin{lstlisting}[language=Clojure]
(let [f (fn [g :- (IFn [Int -> Int] [Num -> Num])] ...)
      id (fn [x :- Bot] x)]
  (f (ann (fn [x] x)
          (IFn [Int -> Int]
               [Num -> Num]))))
\end{lstlisting}

This seems to help immensely, but now it seems a waste
to inline \clj{id} at all.
Instead, we could simplify this program by only annotating \clj{id}'s
right-hand-side.

\begin{lstlisting}[language=Clojure]
(let [f (fn [g :- (IFn [Int -> Int] [Num -> Num])] ...)
      id (ann (fn [x] x)
              (IFn [Int -> Int]
                   [Num -> Num]))]
  (f id))
\end{lstlisting}

We can use this technique to help check our previous examples
without having to work around closed-over variables.
For example, our thrice-printing example requires
a single intersection type annotation, based on
the two symbolic reductions of \clj{f}.

\begin{lstlisting}[language=Clojure]
(let [f (let [v 1]
          (print "I only print once.")
          (ann (fn [x] (print v) x)
               (IFn [Int -> Int]
                    [Str -> Str])))]
  (f 1)
  (f "a"))
\end{lstlisting}

Unfortunately, inferring ordered intersection types for unannotated
functions introduce other issues,
most prominently determining the ``best'' ordering of
arities.

Compared to unordered intersections, ordered intersections
have a simple application rule: first arity wins.
A ``best'' ordering would yield the same or more accurate return type
for every possible application, compared to every other ordering.
For now, we assume that a best ordering
exists for every ordered function intersection type, although
we are not sure.

There are several interesting cases we sketch to give an idea
of the character of this algorithm.
First, if arity 1's domain is a subtype of arity 2's
domain, then arity 1 should come first,
using subtyping of the range to break ties in a similar fashion.
For example, \clj{[Int -> Int]}
goes before  \clj{[Num -> Int]},
but
\clj{[Int -> Int]}
precedes
\clj{[Int -> Num]}.
Second, if the domains of two arities
are incomparable via subtyping, their ordering does not matter.
For example,
\clj{[(Pair Int Num) -> Int]},
\clj{[(Pair Num Int) -> Num]},
and
\clj{[Int -> Num]}
may occur in any order in relation to each other.
This also accounts for multiple arguments,
by considering them as a list passed to a single argument.

Going back to checking programs, we now explore some
other ways in which inferring ordered function intersections has
interesting interactions with \ltiFsub.
The next example is similar to the one that helped motivate
intersection types earlier, except we omit the annotation
on \clj{f}.

\begin{lstlisting}[language=Clojure]
(let [f (fn [g]
          (g 1)
          (g "a"))
      id (fn [x] x)]
  (f id))
\end{lstlisting}

Here, the annotation on \clj{g} is the interesting part.

\begin{lstlisting}[language=Clojure]
(let [f (fn [g :- (IFn [Int -> Int] [Str -> Str])]
          (g 1)
          (g "a"))
      id (ann (fn [x] x)
              (IFn [Int -> Int] [Str -> Str]))]
  (f id))
\end{lstlisting}

Unfortunately, we can't retroactively annotate all programs in this way.
Take the following program.

\begin{lstlisting}[language=Clojure]
(let [f (fn [g]
          (fn [x]
            (g x)))]
  ((f (fn [y] y)) 1)
  ((f (fn [z] z)) "a"))
\end{lstlisting}

Ideally, we would give \clj{f} the polymorphic type
\clj{(All [a] [[a -> a] -> [a -> a]])}.
Then, we would use the type variable to annotate its return
as \clj{[a -> a]}.
Instead, we infer \clj{f} as type

\begin{lstlisting}[language=Clojure]
(IFn [[Int -> Int] -> [Int -> Int]]
     [[Str -> Str] -> [Str -> Str]])
\end{lstlisting}

This means that \clj{f}'s body will be checked twice.
The first time, \clj{g} will be assumed \clj{[Int -> Int]},
and then return checked as \clj{[Int -> Int]}.
The second, \clj{g} will be assumed \clj{[Str -> Str]},
and then return checked as \clj{[Str -> Str]}.

The problem now is annotating the function in \clj{f}'s body once-and-for-all.
It inhabits the type \clj{(U [Int -> Int] [Str -> Str])},
but that is too broad to be compatible with \clj{f}'s return---on the other hand,
it does not inhabit \clj{(IFn [Int -> Int] [Str -> Str])},
because \clj{g} cannot accept both \clj{Int} and \clj{Str}.
To handle these cases, we borrow \emph{conditional types} from TypeScript.

A conditional type is type-level dependency between types.
It is of the form \clj{(if (subtype? S T) U V)},
and returns type \clj{U} if \clj{S} is a subtype of \clj{T},
and \clj{V} otherwise.
This construct is particularly useful in combination with
the ability to reference the types of \emph{variables}.
In TypeScript, the type \clj{typeof f} resolves to the type of \clj{f}
in the current type environment. Here, we equivalently write \clj{(TypeOf f)}.

Applying these new type constructs to our example, we get the following annotation:

\begin{lstlisting}[language=Clojure]
(let [f (ann (fn [g]
               (fn [x :- (if (subtype? (TypeOf g) [Int -> Int]) Int Str)]
                 (g x)))
             (IFn [[Int -> Int] -> [Int -> Int]]
                  [[Str -> Str] -> [Str -> Str]]))]
  ((f (fn [y :- Int] y)) 1)
  ((f (fn [z :- Str] z)) "a"))
\end{lstlisting}

Now when \clj{f}'s first function type is checked,
\clj{g} will be of type \clj{[Int -> Int]}, which annotates
\clj{x} as \clj{Int}.
Correspondingly for the second function type,
\clj{g} will be of type \clj{[Str -> Str]}, which annotates
\clj{x} as \clj{Str} via the conditional type's else-branch.
We note that special consideration of variable shadowing is required when using \clj{TypeOf}---for
example, if \clj{g} was shadowed above, we would be branching on the wrong type.

\subsubsection{Polymorphism}

We have not addressed how symbolic closures interact with polymorphic types.
For now, we consider a restricted subset of polymorphic functions, but which happens to be
common in Clojure code.
Anecdotally, they are higher-order functions that take in functions cannot be iterated.

For example, the \clj{map} function is roughly of type:

\begin{lstlisting}[language=Clojure]
(All [a b]
  [[a -> b] (Seqable a) -> (Seqable b)])
\end{lstlisting}

We can immediately see the data flow by the occurrences of type variables.
The function argument takes an \clj{a} from the collection argument,
and then returns a \clj{b} to the return collection.
The function argument cannot be called on its own output in the body
of \clj{map} because \clj{b} is not compatible with \clj{a}.
We can draw these dependencies as arrows---notice that there is no arrow
from \clj{b} to \clj{a}. There are implicit dependencies from
\clj{a} to \clj{b} because everything to the right of an arrow type depends
on everything to the left of the arrow (or, output values depend
on input values).


\begin{lstlisting}
(All [a b]
  [[(*@\tikz[overlay,remember picture] \node [] (b) {};@*)a -> (*@\tikz[overlay,remember picture] \node [] (c) {};@*)b] (Seqable (*@\tikz[overlay,remember picture] \node [] (a) {};@*)a) -> (Seq (*@\tikz[overlay,remember picture] \node [] (d) {};@*)b)])
\end{lstlisting}
\begin{tikzpicture}[remember picture, overlay,
                  text width = 2.5cm ]
  \coordinate (Start1) at (a);
  \coordinate (End1) at (b);
  \coordinate (Start2) at (c);
  \coordinate (End2) at (d);
  \draw[red,->,bend right=-45](Start1.south) to (End1.east);
  \draw[blue,->,bend right=-45](Start2.east) to (End2.north east);
\end{tikzpicture} 

Inferring the data flow is crucial to checking symbolic closures.
Take the following example, where the function argument is inferred
as a symbolic closure.

\begin{lstlisting}[language=Clojure]
(map (fn [x] x) [1 2 3])
\end{lstlisting}

We now have two jobs: to infer the type arguments to \clj{map}
and to infer the type of \clj{x}.
Both can be found simultaneously by solving constraints
to find optimal instantiations for \clj{a} and \clj{b}.
First, we collect the constraints that make
\clj{(Closure \{\} (fn [x] x))}
a subtype of
\clj{[a -> b]}.
By checking the function with annotations
\clj{(fn [x :- a] x)},
we know that \clj{a} flows into \clj{b}, so
we get the constraint
\clj{Bot <: a <: b}.
For the second argument to \clj{map}
we infer
\clj{Int <: a <: Top}.
Since both type variables occur invariantly, we use their smallest instantiations,
so the optimal solution to these constraints
is \clj{a = Int, b = Int}.
We use this substitution to both
both provide the type arguments to \clj{map} (via \clj{inst})
and function argument (by substituting away the \clj{a} that
exercised the symbolic closure).


\begin{lstlisting}[language=Clojure]
((inst map Int Int) (fn [x :- Int] x) [1 2 3])
\end{lstlisting}

We can combine this inference technique with the same approach
we used to check our previous let-bound function examples,
like in the following code.

\begin{lstlisting}[language=Clojure]
(let [f (fn [x] x)]
  (map f [1 2 3])
  (map f ["a" "b" "c"]))
\end{lstlisting}

Here, we infer type arguments for each usage of \clj{map},
and combine the information collected for \clj{f} from both
inferences into an intersection type, yielding:

\begin{lstlisting}[language=Clojure]
(let [f (ann (fn [x] x)
             (IFn [Int -> Int]
                  [Str -> Str]))]
  ((inst map Int Int) f [1 2 3])
  ((inst map Str Str) f ["a" "b" "c"]))
\end{lstlisting}

Furthermore, this approach plays nicely with inferring conditional types,
like in the next example:

\begin{lstlisting}[language=Clojure]
(let [f (fn [g]
          (fn [x]
            (map g x)))]
  ((f (fn [y] y)) [1 2 3])
  ((f (fn [z] z)) ["a" "b" "c"]))
\end{lstlisting}

Now we must both infer an annotation for \clj{x} and
the type arguments to \clj{map}, but, as
in example motivating conditional types,
this is aggravated by \clj{f} being given a
intersection type, and thus forcing its body to be checked twice.
The solution is to to use more conditional types, particularly
as in the instantiation of \clj{map}:

\begin{lstlisting}[language=Clojure]
(let [f (ann (fn [g]
               (fn [x :- (Seqable (if (subtype? (TypeOf g) [Int -> Int]) Int Str))]
                 ((inst map
                        (if (subtype? (TypeOf x) (Seqable Int)) Int Str)
                        (if (subtype? (TypeOf x) (Seqable Int)) Int Str))
                  g x)))
             (IFn [[Int -> Int] -> [(Seqable Int) -> (Seqable Int)]]
                  [[Str -> Str] -> [(Seqable Str) -> (Seqable Str)]]))]
  ((f (fn [y :- Int] y)) [1 2 3])
  ((f (fn [z :- Str] z)) ["a" "b" "c"]))
\end{lstlisting}

\subsubsection{Inferring Conditional types}

The most attractive use of conditional types is to check the
same piece of code at different types.

\begin{figure}
$$
\begin{array}{lrll}
  \ltiE{}, \ltiF{} &::=& ... \alt
                         \ltifuntparaminterface{\ova{\ltitvar{}}}
                                               {\ova
                                                {\ltistackmapping{\ltiEnv{}}
                                                                 {\ltiIFn{\ova{\ltiFn{\ltiT{}}{\ltiT{}}}}}}}
                                               {\ltivar{}}
                                               {\ltiE{}}
                         \alt
                         \ltiappinst{\ltiF{}}{\ova{\ltistackmapping{\ltiEnv{}}{\ova{\ltiR{}}}}}{\ltiE{}} \alt
                      &\mbox{Terms} \\
  \ltiappinst{\ltiF{}}{\ova{\ltiR{}}}{\ltiE{}} &\Leftrightarrow&
         \ltiappinst{\ltiF{}}{\ltistackmapping{\ltiEmptyEnv{}}{\ova{\ltiR{}}}}{\ltiE{}}\\
   \ltifuntparaminterface{\ova{\ltitvar{}}}
                         {\ltiT{}}
                         {\ltivar{}}
                         {\ltiE{}}
         &\Leftrightarrow&
   \ltifuntparaminterface{\ova{\ltitvar{}}}
                         {\ltistackmapping{\ltiEmptyEnv{}}{\ltiT{}}}
                         {\ltivar{}}
                         {\ltiE{}}
                      &\mbox{Term abbreviations} \\
  \ltiT{}, \ltiS{}, \ltiR{} &::=& ...
                         \alt
                         \ltiMu{\ltitvar{}}{\ltiT{}}
                         \alt 
                         \ltiIFn{\ova{\ltiFn{\ltiT{}}{\ltiT{}}}}
                      &\mbox{Types} \\
  \ltiSubtypeSeen{} &::=& \ova{\ltiSeenEntry{\ltiT{}}{\ltiT{}}}
                      &\mbox{Subtype Seen List} \\

\end{array}
$$
\caption{Internal Language Syntax Extensions}
\label{symbolic:figure:internal-language-mu-intersection}
\end{figure}

\begin{figure}
  \begin{mathpar}
    \infer [I-AppInst]
    {
    \ltitjudgement{\ltiEnv{}}
                  {\ltiF{}}
                  {\ltiT{}^f}
                  {\ltiFp{}}
                    \\
    \ltitjudgement{\ltiEnv{}}
                  {\ltiE{}}
                  {\ltiT{}^a}
                  {\ltiEp{}}
                  \\
         \ltiunfold{\ltiT{}^f}
                   {\ltiPoly{\ova{\ltitvar{}}}{\ltiIFn{\ova{\ltiFn{\ltiT{}}{\ltiS{}}}^n}}}
                  \\\\
                  \exists i \in 1...m.
                        \ltiunifyContexts{\ltiInternalOrExternalLang{}}{\ltistackmapping{\ltiEnvp{i}}{\ova{\ltiRp{}}_i}}{\ltiEnv{}}{\ova{\ltiR{}}} \}
                  \\\\
                  \ova{\ltiSp{}}
                  =
                  \left\{ {\ltireplace{\ova{\ltiR{}}}{\ova{\ltitvar{}}}{\ltiS{i}}}
                  \middle| i \in 1 ... n, 
                  \text{ if }
                  \ltiisubtype{\ltiEnv{}}{\ltiT{}^a}{\ltireplace{\ova{\ltiR{}}}{\ova{\ltitvar{}}}{\ltiT{i}}}
                  \right\}
                  \\
                  |\ova{\ltiSp{}}|>0
    }
    {
      \ltitjudgement{\ltiEnv{}}
                    {\ltiappinst{\ltiF{}}
                                {\ova{\ltistackmapping{\ltiEnvp{}}{\ova{\ltiRp{}}}}^m}
                                {\ltiE{}}}
                    {\ltiMeetMany{\ova{\ltiSp{}}}}
                    {\ltiappinst{\ltiFp{}}
                                {\ova{\ltistackmapping{\ltiEnvp{}}{\ova{\ltiRp{}}}}^m}
                                {\ltiEp{}}}
    }

    \infer [I-App\Bot]
    {
    \ltitjudgement{\ltiEnv{}}
                  {\ltiF{}}
                  {\ltiT{}}
                  {\ltiFp{}}
                  \\\\
    \ltitjudgement{\ltiEnv{}}
                  {\ltiE{}}
                  {\ltiS{}}
                  {\ltiEp{}}
                  \\\\
    \ltiunfold{\ltiT{}}{\ltiBot}
    }
    {
    \ltitjudgement{\ltiEnv{}}
                  {\ltiappinst{\ltiF{}}{\ova{\ltistackmapping{\ltiEnvp{}}{\ova{\ltiR{}}}}}{\ltiE{}}}
                  {\ltiBot{}}
                  {\ltiappinst{\ltiFp{}}{\ova{\ltistackmapping{\ltiEnvp{}}{\ova{\ltiR{}}}}}{\ltiEp{}}}
    }

    \infer [I-Abs]
    { 
    \exists i \in 1...m.
    \ltiunifyContexts{\ltiInternalOrExternalLang{}}
                     {\ltistackmapping{\ltiEnvp{i}}{\ltiTp{i}}}
                     {\ltiEnv{}}
                     {\ltiIFn{\ova{\ltiFn{\ltiT{}}{\ltiS{}}}^n}}
    \\\\
    n>0
    \\
    \overrightarrow{
    \ltitjudgement{\ltiEnvConcat{\ltiEnv{}}
                                {\ltiEnvConcat{\ova{\ltitvar{}}}
                                              {\hastype{\ltivar{}}{\ltiT{i}}}}}
                  {\ltiE{}}
                  {\ltiSp{i}}
                  {\ltiF{i}}
                  \ \ \
          \ltiisubtype{\ltiEnv{}}{\ltiSp{i}}{\ltiS{i}}
                  }^{1 \leq i \leq n}
          \\\\
          \ltimergeTaggedTermsLHS{\ltiE{}}{\ltimergeTaggedTermsLHS{\ltiF{1}}{\ltimergeTaggedTermsLHS{...}{\ltiF{n}}}}
          = \ltiEp{}
    }
    {
    \ltitjudgement{\ltiEnv{}}
                  {\ltifuntparaminterface{\ova{\ltitvar{}}}
                                         {\ova{\ltistackmapping{\ltiEnvp{}}{\ltiTp{}}}^m}
                                         {\ltivar{}}
                                         {\ltiE{}}}
                  {\ltiPoly{\ova{\ltitvar{}}}{\ltiIFn{\ova{\ltiFn{\ltiT{}}{\ltiS{}}}^n}}}
                  {\ltifuntparaminterface{\ova{\ltitvar{}}}
                                         {\ova{\ltistackmapping{\ltiEnvp{}}{\ltiTp{}}}^m}
                                         {\ltivar{}}
                                         {\ltiEp{}}}
                 }

    \infer [I-Sel]
    {
    \ltitjudgement{\ltiEnv{}}
                  {\ltiE{}}
                  {\ltiS{}}
                  {\ltiF{}}
                     \\\\
    \ltiunfold{\ltiS{}}
              {\ltiRec{\hastype{\ltivar{1}}{\ltiT{1}}, ..., \hastype{\ltivar{i}}{\ltiT{i}} , ..., \hastype{\ltivar{n}}{\ltiT{n}}}}
    }
    {
    \ltitjudgement{\ltiEnv{}}
                  {\ltisel{\ltiE{}}{\ltivar{i}}}
                  {\ltiT{i}}
                  {\ltisel{\ltiF{}}{\ltivar{i}}}
    }

    \infer [I-Sel\ltiBot]
    {
    \ltitjudgement{\ltiEnv{}}
                     {\ltiE{}}
                     {\ltiT{}}
                     {\ltiF{}}
                     \\\\
                     \ltiunfold{\ltiT{}}{\ltiBot}
    }
    {
    \ltitjudgement{\ltiEnv{}}
                  {\ltisel{\ltiE{}}{\ltivar{}}}
                  {\ltiBot}
                  {\ltisel{\ltiF{}}{\ltivar{}}}
    }
  \end{mathpar}
  \caption{Internal language type system extensions
  }
  \label{symbolic:figure:internal-language-type-system-mu-intersection}
\end{figure}

\begin{figure}
  \begin{mathpar}
    \boxed{
    \infer[]
    {}
    {
    \ltiunfold{\ltiT{}}{\ltiS{}}
    \\\\
    \text{ Unfold top-level recursion in \ltiT{} to \ltiS{}.}
    }
    }

    \begin{array}{lcl}
      \ltiunfoldalign{\ltiPoly{\ova{\ltitvar{}}}{\ltiT{}}}
                     {\ltiPoly{\ova{\ltitvar{}}}{\ltiunfoldLHS{\ltiT{}}}}\\
      \ltiunfoldalign{\ltiMu{\ltitvar{}}{\ltiT{}}}
                     {\ltireplace{\ltiMu{\ltitvar{}}{\ltiT{}}}
                                 {\ltitvar{}}
                                 {\ltiunfoldLHS{\ltiT{}}}}
                                                                \\
      \ltiunfoldalign{\ltiT{}}{\ltiT{}} \text{, otherwise}\\
    \end{array}

    \boxed{
    \infer[]
    {}
    {
    \ltimergeTaggedTerms{\ltiE{1}}{\ltiE{2}}{\ltiF{}}
    \\\\
    \text{Merge terms \ltiE{i} as \ltiF{}.}
    }
    }

    \boxed{
    \infer[]
    {}
    {
    \ltiunifyContexts{\ltiInternalOrExternalLang{}}{\ltistackmapping{\ltiEnvp{}}{\ova{\ltiT{}}}}{\ltiEnv{}}{\ova{\ltiS{}}}
    \\\\
    \text{Prepare \ova{\ltiT{}} for use in current context \ltiEnv{}.}
    }
    }

    \begin{array}{lllll}
      \ltiunifyContextsalign{\ltiInternalOrExternalLang{}}
                             {\ltistackmapping{\ltiEmptyEnv{}}{\ova{\ltiT{}}}}
                             {\ltiEnv{}}
                             {\ova{\ltiT{}}} \\
      \ltiunifyContextsalign{\ltiInternalOrExternalLang{}}
                             {\ltistackmapping{\ltiEnvConcatParen{\ltitvarp{}}{\ltiEnvp{}}}{\ova{\ltiT{}}}}
                             {\ltiEnvConcatParen{\ltitvar{}}{\ltiEnv{}}}
                             {\ltiunifyContextsLHS{\ltiInternalOrExternalLang{}}
                                                   {\ltireplace{\ltitvar{}}{\ltitvarp{}}
                                                               {(\ltistackmapping{\ltiEnvp{}}{\ova{\ltiT{}}})}}
                                                   {\ltiEnv{}}}\\
      \ltiunifyContextsalign{\ltiInternalOrExternalLang{}}
                             {\ltistackmapping{\ltiEnvConcatParen{\hastype{\ltivarp{}}{\ltiSp{}}}{\ltiEnvp{}}}{\ova{\ltiT{}}}}
                            {\ltiEnvConcatParen{\hastype{\ltivar{}}{\ltiS{}}}{\ltiEnv{}}}
                            {\ltiunifyContextsLHS{\ltiInternalOrExternalLang{}}
                                                  {\ltistackmapping{\ltiEnvp{}}{\ova{\ltiT{}}}}
                                                  {\ltiEnv{}}},
                                                 &\text{ if } \ltiisubtype{\ltiEnvpp{}}{\ltiS{}}{\ltiSp{}}
    \end{array}

  \begin{array}{llll}
    \ltimergeTaggedTermsalign{\ltifuntparaminterface{\ova{\ltitvar{}}}{\ltiT{}}{\ltivar{}}{\ltiE{1}}}
                             {\ltifuntparaminterface{\ova{\ltitvar{}}}{\ltiT{}}
                                              {\ltivar{}}
                                              {\ltiE{2}}}
                             {\ltifuntparaminterface{\ova{\ltitvar{}}}{\ltiT{}}
                                              {\ltivar{}}
                                              {\ltimergeTaggedTermsLHS{\ltiE{1}}{\ltiE{2}}}}
                             \\
    \ltimergeTaggedTermsalign{\ltiappinst{\ltiF{}}{\ova{\ltistackmapping{\ltiEnv{}}{\ova{\ltiR{}}}}}{\ltiE{}}}
                             {\ltiappinst{\ltiFp{}}{\ova{\ltistackmapping{\ltiEnv{}}{\ova{\ltiR{}}}}}{\ltiEp{}}}
                             {\ltiappinst{\ltimergeTaggedTermsLHS{\ltiF{}}{\ltiFp{}}}
                                         {\ova{\ltistackmapping{\ltiEnv{}}{\ova{\ltiR{}}}}}
                                         {\ltimergeTaggedTermsLHS{\ltiE{}}{\ltiEp{}}}}
                                     \\
    \ltimergeTaggedTermsalign{\ltisel{\ltiE{1}}{\ltivar{}}}
                             {\ltisel{\ltiE{2}}{\ltivar{}}}
                             {\ltisel{\ltimergeTaggedTermsLHS{\ltiE{1}}{\ltiE{2}}}{\ltivar{}}}
                                     \\
    \ltimergeTaggedTermsalign{\ltiRec{\ova{\ltivar{} = \ltiE{}}}}
                             {\ltiRec{\ova{\ltivar{} = \ltiF{}}}}
                             {\ltiRec{\ova{\ltivar{} = \ltimergeTaggedTermsLHS{\ltiE{}}{\ltiF{}}}}}
                                     \\
    \ltimergeTaggedTermsalign{\ltivar{}}
                             {\ltivar{}}
                             {\ltivar{}}
  \end{array}

  \end{mathpar}
  \caption{Extended Type System Metafunctions}
  \label{symbolic:figure:internal-language-metafunctions}
\end{figure}

\begin{figure}
  \begin{mathpar}
    \boxed{
    \infer[]
    {}
    {
      \ltiisubtypeseen{\ltiSubtypeSeen{}}{\ltiEnv{}}{\ltiT{}}{\ltiS{}}
      \\\\
      \text{\ltiT{} is a subtype of \ltiS{},
      }
      \\\\
      \text{with seen queries \ltiSubtypeSeen{}.
                 }
                 }
                 }

    \infer [S-MuL]
    {
     \ltiSeenEntry{\ltiMu{\ltitvar{}}{\ltiT{}}}{\ltiS{}} \in \ltiSubtypeSeen{}
     \\\\
    \text{ or }
     \\\\
    \ltiisubtypeseen{\ltiSeenConcat{\ltiSeenEntry{\ltiMu{\ltitvar{}}{\ltiT{}}}
                                                 {\ltiS{}}}
                                   {\ltiSubtypeSeen{}}}
                    {\ltiEnv{}}
                    {\ltireplace{\ltiMu{\ltitvar{}}{\ltiT{}}}{\ltitvar{}}{\ltiT{}}}{\ltiS{}}
    }
    {
    \ltiisubtypeseen{\ltiSubtypeSeen{}}{\ltiEnv{}}
                    {\ltiMu{\ltitvar{}}{\ltiT{}}}
                    {\ltiS{}}
    }

    \infer [S-MuR]
    {
     \ltiSeenEntry{\ltiS{}}{\ltiMu{\ltitvar{}}{\ltiT{}}} \in \ltiSubtypeSeen{}
     \\\\
    \text{ or }
     \\\\
    \ltiisubtypeseen{\ltiSeenConcat{\ltiSeenEntry{\ltiS{}}
                                                 {{\ltiMu{\ltitvar{}}{\ltiT{}}}}}
                                   {\ltiSubtypeSeen{}}}
                    {\ltiEnv{}}
                    {\ltiS{}}
                    {\ltireplace{\ltiMu{\ltitvar{}}{\ltiT{}}}{\ltitvar{}}{\ltiT{}}}
    }
    {
    \ltiisubtypeseen{\ltiSubtypeSeen{}}
                    {\ltiEnv{}}
                    {\ltiS{}}
                    {\ltiMu{\ltitvar{}}{\ltiT{}}}
    }

    \infer [S-IFn]
    { 
      \forall i \in 1...n.\ 
        \exists j \in 1...m.\ 
          \ltiisubtypeseen{\ltiSubtypeSeen{}}{\ltiEnv{}}{\ltiS{j}}{\ltiT{i}}
    }
    { \ltiisubtypeseen{\ltiSubtypeSeen{}}{\ltiEnv{}}
                      {\ltiIFn{\ova{\ltiS{}}^m}}
                      {\ltiIFn{\ova{\ltiT{}}^n}}
                   }

%    \infer [SF-ContextBoth]
%    {
%     \ltiisubtypeseen{\ltiSubtypeSeen{}}
%                     {\ltiEnvpp{}}
%                     {\ltiunifyContextsLHS{\ltiinternallabel}{\ltistackmapping{\ltiEnvpp{}}{\ltiS{}}}{\ltiEnvp{}}}
%                     {\ltiT{}}
%    }
%    {\ltiisubtypeseen{\ltiSubtypeSeen{}}
%                     {\ltiEnv{}}
%                     {(\ltistackmapping{\ltiEnvp{}}{\ltiS{}})}
%                     {(\ltistackmapping{\ltiEnvpp{}}{\ltiT{}})}
%    }
%
%    \infer [SF-ContextL]
%    {
%     \ltiisubtypeseen{\ltiSubtypeSeen{}}
%                     {\ltiEnv{}}
%                     {\ltiunifyContextsLHS{\ltiinternallabel}{\ltistackmapping{\ltiEnvp{}}{\ltiS{}}}{\ltiEnv{}}}
%                     {\ltiT{}}
%    }
%    {\ltiisubtypeseen{\ltiSubtypeSeen{}}
%                     {\ltiEnv{}}
%                     {(\ltistackmapping{\ltiEnvp{}}{\ltiS{}})}
%                     {\ltiT{}}
%    }
%
%    \infer [SF-ContextR]
%    {
%     \ltiisubtypeseen{\ltiSubtypeSeen{}}
%                     {\ltiEnv{}}
%                     {\ltiS{}}
%                     {\ltiunifyContextsLHS{\ltiinternallabel}{\ltistackmapping{\ltiEnvp{}}{\ltiT{}}}{\ltiEnv{}}}
%    }
%    {\ltiisubtypeseen{\ltiSubtypeSeen{}}
%                     {\ltiEnv{}}
%                     {\ltiS{}}
%                     {(\ltistackmapping{\ltiEnvp{}}{\ltiT{}})}
%    }

  \end{mathpar}
  \caption{Internal language subtyping extensions
  }
  \label{symbolic:figure:internal-language-subtyping-mu-intersection}
\end{figure}

\begin{figure}
  \begin{mathpar}

%    \infer [Var]
%    {}
%    {
%       \ltitSdjudgement{\ltiEnv{}}
%                       {\ltivar{}}
%                       {\ltiEnvLookup{\ltiEnv{}}{\ltivar{}}}
%                       {\ltivar{}}
%                 }

%    \infer [Sel]
%    {
%    \ltitSdjudgement{\ltiEnv{}}
%                    {\ltiF{}}
%                    {\ltiS{}}
%                    {\ltiFp{}}
%                     \\\\
%    \ltiSdsubtype{\ltiEnv{}}{\ltiS{}}{\ltiRec{\hastype{\ltivar{1}}{\ltiT{1}},...,\hastype{\ltivar{i}}{\ltiT{i}},...,\hastype{\ltivar{n}}{\ltiT{n}}}}
%    }
%    {
%    \ltitSdjudgement{\ltiEnv{}}
%                  {\ltisel{\ltiF{}}{\ltivar{i}}}
%                  {\ltiT{i}}
%                  {\ltisel{\ltiFp{}}{\ltivar{i}}}
%    }
%
%    \infer [Rec]
%    {
%    \overrightarrow{
%    \ltitSdjudgement{\ltiEnv{}}
%                    {\ltiF{i}}
%                    {\ltiT{i}}
%                    {\ltiFp{i}}
%                    }
%                    ^{1 \leq i \leq n}
%    }
%    {
%    \ltitSdjudgement{\ltiEnv{}}
%                    {\ltiRec{\ova{\ltivar{} = \ltiF{}}^n}}
%                    {\ltiRec{\ova{\hastype{\ltivar{}}{\ltiT{}}}^n}}
%                    {\ltiRec{\ova{\ltivar{} = \ltiFp{}}^n}}
%    }

    \infer [E-AppInf]
    {
    \ltitjudgement{\ltiEnv{}}
                    {\ltiF{}}
                    {\ltiS{}^f}
                    {\ltiFp{}}
                    \\
    \ltitjudgement{\ltiEnv{}}
                    {\ltiE{}}
                    {\ltiS{}^a}
                    {\ltiEp{}}
                    \\\\
          \ltiisubtype{\ltiEnv{}}
                    {\ltiS{}^f}
                    {\ltiPoly{\ova{\ltitvar{}}}
                             {\ltiIFn{\ltiFn{\ltiT{}}{\ltiS{}}}}}
                  \\
                       |\ova{\ltitvar{}}|>0
                  \\\\
                  \forall \ltiRp{}.
                    \left(
                    \begin{array}{lll}
                      \ltiisubtype{\ltiEnv{}}{\ltiS{}^a}{\ltireplace{\ova{\ltiRp{}}}{\ova{\ltitvar{}}}{\ltiT{}}}
                      \text{ implies}
                      \arcr
                      \ltiisubtype{\ltiEnv{}}{\ltireplace{\ova{\ltiR{}}}{\ova{\ltitvar{}}}{\ltiS{}^a}}
                                   {\ltireplace{\ova{\ltiRp{}}}{\ova{\ltitvar{}}}{\ltiS{}^a}}
                    \end{array}
                  \right)
    }
    {
    \ltitjudgement{\ltiEnv{}}
                    {\ltiapp{\ltiF{}}{\ltiE{}}}
                    {\ltireplace{\ova{\ltiR{}}}{\ova{\ltitvar{}}}{\ltiS{}}}
                    {\ltiappinst{\ltiFp{}}
                                {\ltistackmapping{\ltiEnv{}}{\ova{\ltiR{}}}}
                                {\ltiEp{}}}
    }

    \infer [E-UAbs]
    { 
    \exists i \in 1...m.
    \ltiunifyContexts{\ltiInternalOrExternalLang{}}
                     {\ltistackmapping{\ltiEnvp{i}}{\ltiTp{i}}}
                     {\ltiEnv{}}
                     {\ltiIFn{\ova{\ltiFn{\ltiT{}}{\ltiS{}}}^n}}
    \\\\
    n>0
    \\\\
    \overrightarrow{
    \ltitjudgement{\ltiEnvConcat{\ltiEnv{}}
                                {\ltiEnvConcat{\ova{\ltitvar{}}}
                                              {\hastype{\ltivar{}}{\ova{\ltiT{i}}}}}}
                  {\ltiE{}}
                  {\ltiSp{i}}
                  {\ltiF{i}}
                  \ \ \
          \ltiisubtype{\ltiEnv{}}{\ltiSp{i}}{\ltiS{i}}
                  }^{1 \leq i \leq n}
    \\\\
          \ltimergeTaggedTermsLHS{\ltiE{}}{\ltimergeTaggedTermsLHS{\ltiF{1}}{\ltimergeTaggedTermsLHS{...}{\ltiF{n}}}} = \ltiEp{}
    }
    {
    \ltitjudgement{\ltiEnv{}}
                  {\ltiufun{\ltivar{}}{\ltiE{}}}
                  {\ltiPoly{\ova{\ltitvar{}}}{\ltiIFn{\ova{\ltiFn{\ltiT{}}{\ltiS{}}}^n}}}
                  {\ltifuntparaminterface{\ova{\ltitvar{}}}
                                         {\ova{\ltistackmapping{\ltiEnvp{}}{\ltiTp{}}}}
                                         {\ltivar{}}
                                         {\ltiEp{}}}
                 }

%    \infer [AppInst]
%    {
%    \ltitSdjudgement{\ltiEnv{}}
%                    {\ltiF{}}
%                    {\ltiT{}^f}
%                    {\ltiFp{}}
%                    \\
%    \ltitSdjudgement{\ltiEnv{}}
%                    {\ltiE{}}
%                    {\ltiTp{}}
%                    {\ltiEp{}}
%                  \\\\
%                  \ltiSdsubtype{\ltiEnv{}}{
%         \ltiresolveLHS{\ltiexternallanglabel}
%                    {\ltiEnv{}}
%                    {\ltiT{}^f}}
%                    {
%                    {\ltiPoly{\ova{\ltitvar{}}}{\ltiSplitIFn{\ova{\ltiFn{\ltiT{}}{\ltiS{}}}^n}
%                                                            {\ova{\ltiContextualFn{\ltiEnvpp{}}{\ltiTpp{}}{\ltiSpp{}}}}}}
%                                                            }
%                  \\\\
%                  m > 0
%                  \\
%                  \ltiLfindTA{\ltiexternallanglabel}{\ltiEnv{}}{\ova{\ltitvar{}}}{\ova{\ltistackmapping{\ltiEnvp{}}{\ova{\ltiRp{}}}}}{\ova{\ltiR{}}}
%                  \\\\
%                  \ova{\ltiSp{}}^m
%                  =
%                  \{ \ltiS{i}\ |\ i \in 1 ... n, \ltiSdsubtype{\ltiEnv{}}{\ltiT{}^a}{\ltireplace{\ova{\ltiR{}}}{\ova{\ltitvar{}}}{\ltiT{i}}}
%                  \}
%    }
%    {
%    \ltitSdjudgement{\ltiEnv{}}
%                    {\ltiappinst{\ltiF{}}
%                                {\ova{\ltistackmapping{\ltiEnvp{}}{\ova{\ltiRp{}}}}}
%                                {\ltiE{}}}
%                    {\ltiMeetMany{\ova{\ltireplace{\ova{\ltiR{}}}{\ova{\ltitvar{}}}{\ltiSp{}}}}}
%                    {\ltiappinst{\ltiFp{}}
%                                {\ova{\ltistackmapping{\ltiEnvp{}}{\ova{\ltiRp{}}}}}
%                                {\ltiEp{}}}
%    }


%    \infer [Abs]
%    {
%    \ltiunfold{\ltiTp{}}{\ltiPoly{\ova{\ltitvar{}}}
%                                    {\ltiSplitIFn{\ova{\ltiFn{\ltiT{}}{\ltiS{}}}^{1...n}}
%                                                 {\ova{\ltiContextualFn{\ltiEnvp{}}{\ltiT{}}{\ltiS{}}}^{n+1...m}}}}
%    \\\\
%                     m>0
%                     \\
%    \overrightarrowcaption{
%     \ltitSdjudgement{\ltiEnvConcat{\ltiEnv{}}
%                                   {\ltiEnvConcat{\ova{\ltitvar{}}}
%                                                 {\hastype{\ltivar{}}{\ltiT{i}}}}}
%                     {\ltiEp{0}}
%                     {\ltiSp{i}}
%                     {\ltiE{i}}
%                     }^{1 \leq i \leq n}
%                     \\\\
%    \overrightarrowcaption{
%     \ltitSdjudgement{\ltiEnvConcat{\ltiEnvMissingTVarsLHS{\ltiEnv{}}{\ltiEnvp{i}}}
%                                   {\ltiEnvConcat{\ova{\ltitvar{}}}
%                                                 {\hastype{\ltivar{}}
%                                                          {\ltiT{i}}}}}
%                     {\ltiEp{0}}
%                     {\ltiSp{i}}
%                     {\ltiE{i}}
%                     }^{n < i \leq m}
%                     \\\\
%                     \overrightarrowcaption{\ltiSdsubtype{\ltiEnv{}}{\ltiSp{i}}
%                                                  {\ltiS{i}}
%                                                  ,
%                                                  \ 
%                                                  \ltimergeTaggedTerms{\ltiEp{i-1}}{\ltiE{i}}{\ltiEp{i}}
%                                                  }^{1 \leq i \leq m}
%    }
%    {
%    \ltitSdjudgement{\ltiEnv{}}
%                    {\ltifuninterface{\ltiTp{}}{\ltivar{}}{\ltiEp{0}}}
%                    {\ltiTp{}}
%                    {\ltifuninterface{\ltiTp{}}{\ltivar{}}{\ltiEp{m}}}
%                 }
  \end{mathpar}

  \caption{Extended External Language Specification
  }
  \label{symbolic:figure:external-language-declarative-type-system-mu-intersection}
\end{figure}


\begin{figure}
  \begin{mathpar}
    \boxed{
    \infer[]
    {}
    {
    \ltimergeTaggedTerms{\ltiE{1}}{\ltiE{2}}{\ltiF{}}
    \\\\
    \text{Merge terms \ltiE{i} as \ltiF{} (extends \figref{symbolic:figure:internal-language-metafunctions}).}
    }
    }

  \begin{array}{llll}
    \ltimergeTaggedTermsalign{\ltifuntparaminterface{\ova{\ltitvar{}}}
                                                    {\ova{\ltistackmapping{\ltiEnv{}}{\ltiT{}}}^n}
                                                    {\ltivar{}}{\ltiE{1}}}
                             {\ltifuntparaminterface{\ova{\ltitvar{}}}
                                                    {\ova{\ltistackmapping{\ltiEnv{}}{\ltiT{}}}^{n+1...m}}
                                                    {\ltivar{}}
                                                    {\ltiE{2}}}
                             {\ltifuntparaminterface{\ova{\ltitvar{}}}
                                                    {\ova{\ltistackmapping{\ltiEnv{}}{\ltiT{}}}^m}
                                                    {\ltivar{}}
                                                    {\ltimergeTaggedTermsLHS{\ltiE{1}}{\ltiE{2}}}}
                             \\
    \ltimergeTaggedTermsalign{\ltiappinst{\ltiF{}}{\ova{\ltistackmapping{\ltiEnv{}}{\ova{\ltiR{}}}}^n}{\ltiE{}}}
                             {\ltiappinst{\ltiFp{}}{\ova{\ltistackmapping{\ltiEnv{}}{\ova{\ltiR{}}}}^{n+1...m}}{\ltiEp{}}}
                             {\ltiappinst{\ltimergeTaggedTermsLHS{\ltiF{}}{\ltiFp{}}}
                                         {\ova{\ltistackmapping{\ltiEnv{}}{\ova{\ltiR{}}}}^m}
                                         {\ltimergeTaggedTermsLHS{\ltiE{}}{\ltiEp{}}}}
    \end{array}
  \end{mathpar}

  \caption{Extended External Language Metafunctions
  }
  \label{symbolic:figure:external-language-metafunctions}
\end{figure}

\begin{figure}
$$
\begin{array}{lrll}
  \ltiE{}, \ltiF{} &::=& ... \alt
                         \ltiufunelab{\ova{\ltiufunelabentry{\ltiClosureID{}}}}
                                     {\ltivar{}}
                                     {\ltiE{}}
                      &\mbox{Terms} \\
  \ltiClosureCache{} &::=& \ova{\ltiClosureCacheEntry
                                {\ltiClosureID{}}
                                {\ltiClosure{\ltiEnv{}}
                                            {\ltifuntparaminterface
                                             {\ova{\ltitvar{}}}
                                             {\ova{\ltistackmapping{\ltiEnv{}}{\ltiT{}}}}
                                             {\ltiE{}}}}}
                      &\mbox{Closure Cache} \\
  \ltiCombinedThreadedEnv{} &::=& \ltimakeCombinedThreadedEnv{\ltiFuel{}}{\ltiClosureCache{}}
                      &\mbox{Combined Threaded Environments} \\
\end{array}
$$
\caption{Extended Symbolic Closure Language Syntax (extends \figref{symbolic:figure:external-language-syntax-mu-intersection})}
\label{symbolic:figure:SC-language-syntax}
\end{figure}

\begin{figure}
  \begin{mathpar}
    % TODO thread seen subtypings
    \boxed
    {
    \infer[]
    {}
    {
    \ltitSstkjudgement{\ltiClosureCache{}}
                      {\ltiEnv{}}
                      {\ltiE{}}
                      {\ltiT{}}
                      {\ltiClosureCachep{}}
                      {\ltiEp{}}
                     \\\\
                     \text{Given symbolic closure cache \ltiClosureCache{}
                     and context \ltiEnv{}, external term \ltiE{} 
                     is of symbolic-closure-language type \ltiT{}
                     }
                     \\\\
                     \text{
                     with updated symbolic closure cache \ltiClosureCachep{},
                     and elaborated symbolic-closure-language term \ltiEp{}.
                     }
                     }
                     }

    \infer [AppInst]
    {
    \ltitSstkjudgement{\ltiClosureCache{1}}
                      {\ltiEnvp{}}
                      {\ltiF{}}
                      {\ltiT{}^f}
                      {\ltiClosureCache{2}}
                      {\ltiFp{}}
                  \\
    \ltitSstkjudgement{\ltiClosureCache{2}}
                      {\ltiEnvp{}}
                      {\ltiE{}}
                      {\ltiT{}^a}
                      {\ltiClosureCachep{0}}
                      {\ltiEp{}}
                  \\
    \ltiunfold{\ltiT{}^f}
              {\ltiPoly{\ova{\ltitvar{}}}
                       {\ltiIFn{\ova{\ltiFn{\ltiT{}}{\ltiS{}}}^n}}}
                  \\\\
                  \exists i \in 1...m.
                        \ltiunifyContextsSC{\ltiClosureCachep{0}}
                                           {\ltistackmapping{\ltiEnv{i}}{\ova{\ltiRp{}}_i}}
                                           {\ltiEnvp{}}
                                           {\ova{\ltiR{}}}
                                           {\ltiClosureCachep{1}}
    \\
    \ltiClosureCachepp{0} = \ltiClosureCachep{n}
    \\
    \ova{\ltiSpp{}}_0 = \varnothing
                   \\
                   |\ova{\ltiSpp{}}_n| > 0
    \\\\
    \overrightarrowcaption{
      (\ova{\ltiSpp{}}_i, \ltiClosureCachepp{i})
        = \left\{
                     \begin{array}{llll}
                       (\ova{\ltiSpp{}}_{i-1}{\ltireplace{\ova{\ltiR{}}}{\ova{\ltitvar{}}}{\ltiSp{i}}}, \ltiClosureCachepp{i})
                       , &\text{if } 
                       \ltiSsubtype{\ltiClosureCachepp{i-1}}
                                   {\ltiEnvp{}}
                                   {\ltiT{}^a}
                                   {\ltireplace{\ova{\ltiR{}}}{\ova{\ltitvar{}}}{\ltiTp{i}}}
                                   {\ltiClosureCachepp{i}}
                       \arcr
                       (\ova{\ltiSpp{}}_{i-1}, \ltiClosureCachepp{i-1}),  &\text{otherwise}
                     \end{array}
                   \right.
                   }^{1 \leq i \leq n}
    }
    {
    \ltitSstkjudgement{\ltiClosureCache{1}}
                      {\ltiEnvp{}}
                      {\ltiappinst{\ltiF{}}
                                  {\ova{\ltistackmapping{\ltiEnvp{}}{\ova{\ltiRp{}}}}^m}
                                  {\ltiE{}}}
                      {\ltiMeetMany{\ova{\ltiSpp{}}_n}}
                      {\ltiClosureCachepp{n}}
                      {\ltiappinst{\ltiFp{}}
                                  {\ova{\ltistackmapping{\ltiEnvp{}}{\ova{\ltiRp{}}}}^m}
                                  {\ltiEp{}}}
    }

    \infer [AppInf-Closure]
    {
    \ltitSstkjudgement{\ltiClosureCache{1}}
                      {\ltiEnv{}}
                      {\ltiF{}}
                      {\ltiTpp{}}
                      {\ltiClosureCache{2}}
                      {\ltiFp{}}
                  \\
    \ltitSstkjudgement{\ltiClosureCache{2}}
                      {\ltiEnv{}}
                      {\ltiE{}}
                      {\ltiTp{}}
                      {\ltiClosureCache{3}}
                      {\ltiEp{}}
                  \\\\
    \ltiunfold{\ltiTpp{}}{\ltiClosureWithStkID{\ltiEnvp{}}
                                              {\ltiClosureID{}}
                                              {\ltiufun{\ltivar{}}{\ltiEpp{}}}}
                  \\\\
                  \ltilookup{\ltiClosureCache{3}}{\ltiClosureID{}} =
                  \ltiClosureCacheVal{\ltiFuel{}}{\ltiClosureElab{}}
                  \\
    0 < \ltiFuel{}
    \\\\
    \ltitSstkjudgement{\ltimapsto{\ltiClosureCache{3}}{\ltiClosureID{}}{\ltiClosureCacheVal{\ltiFuel{}-1}{\ltiClosureElab{}}}}
                      {\ltiEnvConcat{\ltiEnvp{}}{\hastype{\ltivar{}}{\ltiTp{}}}}
                      {\ltiEpp{}}
                      {\ltiS{}}
                      {\ltiClosureCache{4}}
                      {\ltiFpp{}}
                      \\\\
    \ltiupdateClosureCache{\ltiClosureCache{4}}{\ltiEnv{}}{\ltiClosureID{}}{\varnothing}{\ltiTp{}}{\ltiS{}}{\ltiFpp{}}{\ltiClosureCache{5}}
    }
    {
    \ltitSstkjudgement{\ltiClosureCache{1}}
                      {\ltiEnv{}}
                      {\ltiapp{\ltiF{}}{\ltiE{}}}
                      {\ltiS{}}
                      {\ltiClosureCache{5}}
                      {\ltiappinst{\ltiFp{}}
                                  {}
                                  {\ltiEp{}}}
    }

    \infer [AppInf\Bot]
    {
    \ltitSstkjudgement{\ltiClosureCache{1}}
                      {\ltiEnv{}}
                      {\ltiF{}}
                      {\ltiT{}}
                      {\ltiClosureCache{2}}
                      {\ltiFp{}}
                  \\\\
    \ltitSstkjudgement{\ltiClosureCache{2}}
                      {\ltiEnv{}}
                      {\ltiE{}}
                      {\ltiS{}}
                      {\ltiClosureCache{3}}
                      {\ltiEp{}}
                  \\\\
    \ltiunfold{\ltiT{}}{\ltiBot}
    }
    {
    \ltitSstkjudgement{\ltiClosureCache{1}}
                      {\ltiEnv{}}
                      {\ltiapp{\ltiF{}}{\ltiE{}}}
                      {\ltiTpp{}}
                      {\ltiClosureCache{3}}
                      {\ltiappinst{\ltiFp{}}
                                  {}
                                  {\ltiEp{}}}
    }

    \infer [AppInst\Bot]
    {
    \ltitSstkjudgement{\ltiClosureCache{1}}
                      {\ltiEnv{}}
                      {\ltiF{}}
                      {\ltiT{}}
                      {\ltiClosureCache{2}}
                      {\ltiFp{}}
                  \\
    \ltitSstkjudgement{\ltiClosureCache{2}}
                      {\ltiEnv{}}
                      {\ltiE{}}
                      {\ltiS{}}
                      {\ltiClosureCache{3}}
                      {\ltiEp{}}
                  \\\\
    \ltiunfold{\ltiT{}}{\ltiBot}
    }
    {
    \ltitSstkjudgement{\ltiClosureCache{1}}
                      {\ltiEnv{}}
                      {\ltiappinst{\ltiF{}}
                                  {\ova{\ltistackmapping{\ltiEnv{}}{\ova{\ltiR{}}}}}
                                  {\ltiE{}}}
                      {\ltiTpp{}}
                      {\ltiClosureCache{3}}
                      {\ltiappinst{\ltiFp{}}
                                  {\ova{\ltistackmapping{\ltiEnv{}}{\ova{\ltiR{}}}}}
                                  {\ltiEp{}}}
    }

    \infer [Sel]
    {
    \ltitSstkjudgement{\ltiClosureCache{}}
                      {\ltiEnv{}}
                      {\ltiF{}}
                      {\ltiS{}}
                      {\ltiClosureCachep{}}
                      {\ltiFp{}}
                      \\\\
    \ltiunfold{\ltiS{}}
              {\ltiRec{\hastype{\ltivar{1}}{\ltiT{1}},..., \hastype{\ltivar{i}}{\ltiT{i}},..., \hastype{\ltivar{n}}{\ltiT{n}}}}
    }
    {
    \ltitSstkjudgement{\ltiClosureCache{}}
                      {\ltiEnv{}}
                      {\ltisel{\ltiF{}}{\ltivar{i}}}
                      {\ltiT{i}}
                      {\ltiClosureCachep{}}
                      {\ltisel{\ltiFp{}}{\ltivar{i}}}
    }

    \infer [Sel\Bot]
    {
    \ltitSstkjudgement{\ltiClosureCache{}}
                      {\ltiEnv{}}
                      {\ltiF{}}
                      {\ltiS{}}
                      {\ltiClosureCachep{}}
                      {\ltiFp{}}
                      \\\\
    \ltiunfold{\ltiS{}}{\ltiBot}
    }
    {
    \ltitSstkjudgement{\ltiClosureCache{}}
                      {\ltiEnv{}}
                      {\ltisel{\ltiF{}}{\ltivar{i}}}
                      {\ltiBot}
                      {\ltiClosureCachep{}}
                      {\ltisel{\ltiFp{}}{\ltivar{i}}}
    }

    \infer [Abs]
    {
    \exists i \in 1...m.
    \ltiunifyContexts{\ltiInternalOrExternalLang{}}
                     {\ltistackmapping{\ltiEnvp{i}}{\ltiTp{i}}}
                     {\ltiEnv{}}
                     {\ltiIFn{\ova{\ltiFn{\ltiT{}}{\ltiS{}}}^n}}
                     \\\\
                     n>0
                     \\\\
                     \text{TODO if $|\ova{\ltitvar{}}|>0$, erase SC's in \ltiEnv{}}
                     \\\\
    \overrightarrowcaption{
     \ltitSstkjudgement{\ltiClosureCache{i-1}}
                    {\ltiEnvConcat{\ltiEnv{}}
                                   {\ltiEnvConcat{\ova{\ltitvar{}}}
                                                 {\hastype{\ltivar{}}
                                                          {\ltiT{i}}}}}
                     {\ltiE{}}
                     {\ltiSp{i}}
                     {\ltiF{i}}
                     {\ltiClosureCache{i}}
                     }^{1 \leq i \leq n}
                     \\\\
                     \overrightarrowcaption{
                        \ltiSsubtype{\ltiClosureCache{n+i-1}}
                                                  {\ltiEnv{}}
                                                  {\ltiSp{i}}
                                                  {\ltiS{i}}
                                                  {\ltiClosureCache{n+i}}
                                                  }^{1 \leq i \leq n}
                                                  \\
        \ltimergeTaggedTermsLHS{\ltiE{}}{\ltimergeTaggedTermsLHS{\ltiF{1}}{\ltimergeTaggedTermsLHS{...}{\ltiF{n}}}} = \ltiEp{}
    }
    {
    \ltitSstkjudgement{\ltiClosureCache{0}}
                    {\ltiEnv{}}
                    {\ltifuntparaminterface{\ova{\ltitvar{}}}
                                           {\ova{\ltistackmapping{\ltiEnvp{}}{\ltiTp{}}}^m}
                                           {\ltivar{}}
                                           {\ltiE{}}}
                    {\ltiPoly{\ova{\ltitvar{}}}{\ltiIFn{\ova{\ltiFn{\ltiT{}}{\ltiS{}}}^n}}}
                    {\ltifuntparaminterface{\ova{\ltitvar{}}}
                                           {\ova{\ltistackmapping{\ltiEnvp{}}{\ltiTp{}}}^m}
                                           {\ltivar{}}
                                           {\ltiEp{}}}
                    {\ltiClosureCache{2n}}
                 }

    \infer [UAbs]
    {
    \ltiClosureID{} \not\in dom(\ltiClosureCache{})
    \\\\
    \ltiClosureCachep{}
    =
    \ltimapsto{\ltiClosureCache{}}
              {\ltiClosureID{}}
              {\ltiClosureCacheVal
               {\ltiFuel{0}}
               {\ltiufun{\ltivar{}}{\ltiE{}}}}
               \\
               for some initial fuel {\ltiFuel{0}}
    }
    {
    \ltitSstkjudgement{\ltiClosureCache{}}
                      {\ltiEnv{}}
                      {\ltiufun{\ltivar{}}{\ltiE{}}}
                      {\ltiClosureWithStkID
                                           {\ltiEnv{}}
                                           {\ltiClosureID{}}
                                           {\ltiufun{\ltivar{}}{\ltiE{}}}}
                      {\ltiClosureCachep{}}
                      {\ltiufunelab{\ltiClosureID{}}
                                   {\ltivar{}}
                                   {\ltiE{}}}
                 }
  \end{mathpar}

  \caption{Extended Algorithmic Type system for Symbolic Closure Language (\textsc{AppInf} omitted)
  }
  \label{symbolic:figure:SC-language-algorithmic-type-system-mu-intersection}
\end{figure}

\begin{figure}
  \begin{mathpar}
    \boxed{
    \infer[]
    {}
    {\ltiSsubtypeseen{\ltiSubtypeSeen{}}
                 {\ltiCombinedThreadedEnv{}}
                 {\ltiEnv{}}
                 {\ltiS{}}
                 {\ltiT{}}
                 {\ltiCombinedThreadedEnvp{}}
                 \\\\
                 \text{
                 With closure cache \ltiCombinedThreadedEnv{}
                 and seen subtypings \ltiSubtypeSeen{},
                 \ltiS{} is a subtype of \ltiT{}
                 with updated cache
                 \ltiCombinedThreadedEnvp{}.
                 }
    }
    }

    \infer [S-TVar]
    {}
    {
     \ltiSsubtypeseen{\ltiSubtypeSeen{}}
                 {\ltiCombinedThreadedEnv{}}
                 {\ltiEnv{}}
                 {\ltitvar{}}
                 {\ltitvar{}}
                 {\ltiCombinedThreadedEnv{}}
    }

    \infer [S-Top]
    {}
    { \ltiSsubtypeseen{\ltiSubtypeSeen{}}{\ltiCombinedThreadedEnv{}}{\ltiEnv{}}{\ltiT{}}{\Top}{\ltiCombinedThreadedEnv{}}}

    \infer [S-Bot]
    {}
    { \ltiSsubtypeseen{\ltiSubtypeSeen{}}{\ltiCombinedThreadedEnv{}}{\ltiEnv{}}{\Bot}{\ltiT{}}{\ltiCombinedThreadedEnv{}}}

    \infer [S-MuL]
    {
     (\ltiSeenEntry{\ltiMu{\ltitvar{}}{\ltiT{}}}{\ltiS{}} \in \ltiSubtypeSeen{}
     \text{ and } \ltiCombinedThreadedEnv{} = \ltiCombinedThreadedEnvp{}
     )
     \\\\
    \text{ or }
     \\\\
    \ltiSsubtypeseen{\ltiSeenConcat{\ltiSeenEntry{\ltiMu{\ltitvar{}}{\ltiT{}}}
                                                 {\ltiS{}}}
                                   {\ltiSubtypeSeen{}}}
                    {\ltiCombinedThreadedEnv{}}
                    {\ltiEnv{}}
                    {\ltireplace{\ltiMu{\ltitvar{}}{\ltiT{}}}{\ltitvar{}}{\ltiT{}}}{\ltiS{}}
                    {\ltiCombinedThreadedEnvp{}}
    }
    {
    \ltiSsubtypeseen{\ltiSubtypeSeen{}}
                    {\ltiCombinedThreadedEnv{}}
                    {\ltiEnv{}}
                    {\ltiMu{\ltitvar{}}{\ltiT{}}}
                    {\ltiS{}}
                    {\ltiCombinedThreadedEnvp{}}
    }

    \infer [S-MuR]
    {
     (\ltiSeenEntry{\ltiS{}}{\ltiMu{\ltitvar{}}{\ltiT{}}} \in \ltiSubtypeSeen{}
     \text{ and } \ltiCombinedThreadedEnv{} = \ltiCombinedThreadedEnvp{})
     \\\\
    \text{ or }
     \\\\
    \ltiSsubtypeseen{\ltiSeenConcat{\ltiSeenEntry{\ltiS{}}
                                                 {{\ltiMu{\ltitvar{}}{\ltiT{}}}}}
                                   {\ltiSubtypeSeen{}}}
                    {\ltiCombinedThreadedEnv{}}
                    {\ltiEnv{}}
                    {\ltiS{}}
                    {\ltireplace{\ltiMu{\ltitvar{}}{\ltiT{}}}{\ltitvar{}}{\ltiT{}}}
                    {\ltiCombinedThreadedEnvp{}}
    }
    {
    \ltiSsubtypeseen{\ltiSubtypeSeen{}}
                    {\ltiCombinedThreadedEnv{}}
                    {\ltiEnv{}}
                    {\ltiS{}}
                    {\ltiMu{\ltitvar{}}{\ltiT{}}}
                    {\ltiCombinedThreadedEnvp{}}
    }

    \infer [S-Rec]
    {
    \overrightarrow{\ltiSsubtypeseen{\ltiSubtypeSeen{}}{\ltiCombinedThreadedEnv{i-1}}{\ltiEnv{}}
                                {\ltiT{}}
                                {\ltiS{}}
                                {\ltiCombinedThreadedEnv{i}}
                                }
    }
    {
    \ltiSsubtypeseen{\ltiSubtypeSeen{}}{\ltiCombinedThreadedEnv{0}}
                {\ltiEnv{}}
                {\ltiRec{\ova{\hastype{\ltivar{}}{\ltiT{}}}^n,
                         \ova{\hastype{\ltivarp{}}{\ltiTp{}}}}}
                {\ltiRec{\ova{\hastype{\ltivar{}}{\ltiS{}}}^n}}
                {\ltiCombinedThreadedEnv{n}}
    }

    \infer [S-Poly]
    {
    \text{TODO erase SC's in \ltiS{} and \ltiT{}}
    \\
    \ltiSsubtypeseen{\ltiSubtypeSeen{}}{\ltiCombinedThreadedEnv{}}
                {\ltiEnvConcat{\ltiEnv{}}{\ova{\ltitvar{}}}}
                {\ltiT{}}
                {\ltiS{}}
                {\ltiCombinedThreadedEnvp{}}
    }
    {
    \ltiSsubtypeseen{\ltiSubtypeSeen{}}{\ltiCombinedThreadedEnv{}}
                {\ltiEnv{}}
                {\ltiPoly{\ova{\ltitvar{}}}{\ltiT{}}}
                {\ltiPoly{\ova{\ltitvar{}}}{\ltiS{}}}
                {\ltiCombinedThreadedEnvp{}}
    }

    \infer [S-IFn]
    { 
      \overrightarrowcaption{
        \exists j \in 1...m.\ 
          \ltiSsubtypeseen{\ltiSubtypeSeen{}}
                          {\ltiCombinedThreadedEnv{i-1}}
                          {\ltiEnv{}}
                          {\ltiS{j}}
                          {\ltiT{i}}
                          {\ltiCombinedThreadedEnv{i}}
                          }^{1 \leq i \leq n}
    }
    { \ltiSsubtypeseen{\ltiSubtypeSeen{}}{\ltiCombinedThreadedEnv{0}}{\ltiEnv{}}
                      {\ltiIFn{\ova{\ltiS{}}^m}}
                      {\ltiIFn{\ova{\ltiT{}}^n}}
                      {\ltiCombinedThreadedEnv{n}}
                   }

                   % FIXME don't push in the Poly, just duplicate the logic in SF-Closure
                   % this is because we only get to choose the type arguments once
    \infer [S-Closure]
    {
    \overrightarrowcaption{
    \ltiSsubtypeseen{\ltiSubtypeSeen{}}
                    {\ltiCombinedThreadedEnv{i-1}}
                    {\ltiEnvpp{}}
                    {\ltiClosureWithStkID{\ltiEnv{}}
                                         {\ltiClosureID{}}
                                         {\ltiufun{\ltivar{}}{\ltiE{}}}}
                    {\ltiPoly{\ova{\ltitvar{}}}{\ltiFn{\ltiT{i}}{\ltiS{i}}}}
                    {\ltiCombinedThreadedEnv{i}}
                    }^{1 \leq i \leq n}
    }
    { \ltiSsubtypeseen{\ltiSubtypeSeen{}}{\ltiCombinedThreadedEnv{0}}
                  {\ltiEnvpp{}}
                  {\ltiClosureWithStkID{\ltiEnv{}}
                                       {\ltiClosureID{}}
                                       {\ltiufun{\ltivar{}}{\ltiE{}}}}
                  {\ltiPoly{\ova{\ltitvar{}}}{\ltiIFn{\ova{\ltiFn{\ltiT{}}{\ltiS{}}}^n}}}
                  {\ltiCombinedThreadedEnv{n}}
                  }

    % eg (IFn [Int -> Int] [Number -> Number]) <: [Nothing -> Any]
    \infer [SF-Fn]
    { \ltiSsubtypeseen{\ltiSubtypeSeen{}}{\ltiCombinedThreadedEnv{1}}{\ltiEnv{}}{\ltiS{}}{\ltiSp{}}{\ltiCombinedThreadedEnv{2}}
      \\\\
      \ltiSsubtypeseen{\ltiSubtypeSeen{}}{\ltiCombinedThreadedEnv{2}}{\ltiEnv{}}{\ltiT{}}{\ltiTp{}}{\ltiCombinedThreadedEnv{3}}
    }
    { \ltiSsubtypeseen{\ltiSubtypeSeen{}}{\ltiCombinedThreadedEnv{1}}{\ltiEnv{}}
                  {\ltiFn{\ltiSp{}}{\ltiT{}}}
                  {\ltiFn{\ltiS{}}{\ltiTp{}}}
                  {\ltiCombinedThreadedEnv{3}}
       }

    \infer [SF-Closure]
    {
    0 < \ltiFuel{}
    \\
    \text{TODO propagate \ltiSubtypeSeen{} through type system} 
    \\
    \ltitSstkjudgement{\ltimakeCombinedThreadedEnv{\ltiFuel{}-1}
                                                  {\ltiClosureCache{}}}
                      {\ltiEnvConcat{\ltiEnv{}}{\hastype{\ltivar{}}{\ltiT{}}}}
                      {\ltiE{}}
                      {\ltiSp{}}
                      {\ltiCombinedThreadedEnv{}}
                      {\ltiEp{}}
                      \\\\
    \ltiSsubtypeseen{\ltiSubtypeSeen{}}{\ltiCombinedThreadedEnv{}}{\ltiEnv{}}{\ltiSp{}}{\ltiS{}}
                {\ltimakeCombinedThreadedEnv{\ltiFuelp{}}
                                            {\ltiClosureCachep{}}}
                      \\
    \ltiupdateClosureCache{\ltiClosureCachep{}}{\ltiEnv{}}{\ltiClosureID{}}{\ova{\ltitvar{}}}{\ltiT{}}{\ltiS{}}{\ltiEp{}}{\ltiClosureCachepp{}}
    }
    { \ltiSsubtypeseen{\ltiSubtypeSeen{}}{\ltimakeCombinedThreadedEnv{\ltiFuel{}}
                                              {\ltiClosureCache{}}}
                  {\ltiEnvp{}}
                  {\ltiClosureWithStkID
                                       {\ltiEnv{}}
                                       {\ltiClosureID{}}
                                       {\ltiufun{\ltivar{}}{\ltiE{}}}}
                  {\ltiPoly{\ova{\ltitvar{}}}{\ltiFn{\ltiT{}}{\ltiS{}}}}
                  {\ltimakeCombinedThreadedEnv{\ltiFuelp{}}
                                              {\ltiClosureCachepp{}}}
                  }

%    \infer [SF-ContextBoth]
%    {
%     \ltiunifyContextsSC{\ltiCombinedThreadedEnv{1}}
%                        {\ltistackmapping{\ltiEnvpp{}}{\ltiS{}}}
%                        {\ltiEnvp{}}
%                        {\ltiSp{}}
%                        {\ltiCombinedThreadedEnv{2}}
%     \\\\
%     \ltiSsubtypeseen{\ltiSubtypeSeen{}}
%                     {\ltiCombinedThreadedEnv{2}}
%                     {\ltiEnvpp{}}
%                     {\ltiSp{}}
%                     {\ltiT{}}
%                     {\ltiCombinedThreadedEnv{3}}
%    }
%    {\ltiSsubtypeseen{\ltiSubtypeSeen{}}
%                     {\ltiCombinedThreadedEnv{1}}
%                     {\ltiEnv{}}
%                     {(\ltistackmapping{\ltiEnvp{}}{\ltiS{}})}
%                     {(\ltistackmapping{\ltiEnvpp{}}{\ltiT{}})}
%                     {\ltiCombinedThreadedEnv{3}}
%    }
%    \ 
%%
%    \infer [SF-ContextL]
%    {
%    \ltiunifyContextsSC{\ltiCombinedThreadedEnv{1}}
%                       {\ltistackmapping{\ltiEnvp{}}{\ltiS{}}}
%                       {\ltiEnv{}}
%                       {\ltiSp{}}
%                       {\ltiCombinedThreadedEnv{2}}
%                       \\\\
%     \ltiSsubtypeseen{\ltiSubtypeSeen{}}
%                     {\ltiCombinedThreadedEnv{2}}
%                     {\ltiEnv{}}
%                     {\ltiSp{}}
%                     {\ltiT{}}
%                     {\ltiCombinedThreadedEnv{3}}
%    }
%    {\ltiSsubtypeseen{\ltiSubtypeSeen{}}
%                     {\ltiCombinedThreadedEnv{1}}
%                     {\ltiEnv{}}
%                     {(\ltistackmapping{\ltiEnvp{}}{\ltiS{}})}
%                     {\ltiT{}}
%                     {\ltiCombinedThreadedEnv{3}}
%    }
%    \ 
%%
%    \infer [SF-ContextR]
%    {
%     \ltiunifyContextsSC{\ltiCombinedThreadedEnv{1}}
%                        {\ltistackmapping{\ltiEnvp{}}{\ltiT{}}}
%                        {\ltiEnv{}}
%                        {\ltiTp{}}
%                        {\ltiCombinedThreadedEnv{2}}
%     \\\\
%     \ltiSsubtypeseen{\ltiSubtypeSeen{}}
%                     {\ltiCombinedThreadedEnv{2}}
%                     {\ltiEnv{}}
%                     {\ltiS{}}
%                     {\ltiTp{}}
%                     {\ltiCombinedThreadedEnv{3}}
%    }
%    {\ltiSsubtypeseen{\ltiSubtypeSeen{}}
%                     {\ltiCombinedThreadedEnv{1}}
%                     {\ltiEnv{}}
%                     {\ltiS{}}
%                     {(\ltistackmapping{\ltiEnvp{}}{\ltiT{}})}
%                     {\ltiCombinedThreadedEnv{3}}
%    }

  \end{mathpar}

  \caption{Extended Symbolic Closure Language Subtyping}
  \label{symbolic:figure:SC-language-subtype-mu-intersection}
\end{figure}

\begin{figure}
  \begin{mathpar}
   \boxed{
   \infer[]
   {}
   {
   \ltimergeTaggedTerms{\ltiE{1}}{\ltiE{2}}{\ltiE{3}}
   \\\\
    \text{Merge terms \ltiE{1} and \ltiE{2}}
    \\\\
    \text{
  (extends \figref{symbolic:figure:external-language-metafunctions}).}
   }}

  \begin{array}{llll}
    \ltimergeTaggedTermsalign{\ltiufunelab{\ova{\ltiufunelabentry
                                                            {\ltiClosureID{}}}^n}
                                     {\ltivar{}}
                                     {\ltiE{1}}}
                             {\ltiufunelab{\ova{\ltiufunelabentry
                                                            {\ltiClosureID{}}}^{n+1...m}}
                                     {\ltivar{}}
                                     {\ltiE{2}}}
                             {\ltiufunelab{\ova{\ltiufunelabentry
                                                            {\ltiClosureID{}}}^m
                                                            }
                                     {\ltivar{}}
                                     {\ltimergeTaggedTermsLHS{\ltiE{1}}{\ltiE{2}}}}
  \end{array}

    \boxed{
    \infer[]
    {}
    {\ltiupdateClosureCache{\ltiClosureCache{}}{\ltiEnv{}}{\ltiClosureID{}}{\ova{\ltitvar{}}}{\ltiT{}}{\ltiS{}}{\ltiE{}}{\ltiClosureCachep{}}
    \\\\
    \text{Record symbolic closure \ltiClosureID{} as \ltiPoly{\ova{\ltitvar{}}}{\ltiFn{\ltiT{}}{\ltiS{}}} under}
    \\\\
    \text{application context \ltiEnv{}, with elaboration \ltiE{}.}
    }}

    \infer[]
    {
    % we want to only set type variables once, we probably need to distinguish
    % between zero type variables and a never-exercised closure
    \text{TODO merge type variables}
    \\
    \ltilookup{\ltiClosureCache{}}{\ltiClosureID{}} = 
    \ltiClosure{\ltiEnv{}}
               {\ltifuntparaminterface{\ova{\ltitvar{}}}
                                      {\ltiIFn{\ova{\ltiFn{\ltiTp{}}{\ltiSp{}}}}}
                                      {\ltivar{}}
                                      {\ltiF{}}}
    \\\\
    \ltiClosureCachep{} = 
    \ltimapsto{\ltiClosureCache{}}
              {\ltiClosureID{}}
              {\ltiClosure{\ltiEnv{}}
                          {\ltifuntparaminterface{\ova{\ltitvar{}}}
                                                 {\ltiIFn{\ltiFn{\ltiT{}}{\ltiS{}} \ova{\ltiFn{\ltiTp{}}{\ltiSp{}}}}}
                                                 {\ltivar{}}
                                                 {\ltimergeTaggedTermsLHS{\ltiE{}}{\ltiF{}}}}}
    }
    {\ltiupdateClosureCache{\ltiClosureCache{}}{\ltiEnvp{}}{\ltiClosureID{}}{\ova{\ltitvar{}}}{\ltiT{}}{\ltiS{}}{\ltiE{}}{\ltiClosureCachep{}}
    }

    \boxed{
    \infer[]
    {}
    {
    \ltiunifyContextsSC{\ltiCombinedThreadedEnv{}}{\ltistackmapping{\ltiEnvp{}}{\ova{\ltiT{}}}}{\ltiEnv{}}{\ova{\ltiS{}}}{\ltiCombinedThreadedEnvp{}}
    \text{ Use \ova{\ltiT{}}'s context \ltiEnvp{} to prepare \ova{\ltiT{}} for use in current context \ltiEnv{}.
    }
    }
    }

    \begin{array}{lllll}
      \ltiunifyContextsSCalign{\ltiCombinedThreadedEnv{}}
                              {\ltistackmapping{\ltiEmptyEnv{}}{\ova{\ltiT{}}}}
                              {\ltiEnv{}}
                              {\ltiunifyContextsSCRHS{\ova{\ltiT{}}}
                                                     {\ltiCombinedThreadedEnv{}}}
                                                     \\
      \ltiunifyContextsSCalign{\ltiCombinedThreadedEnv{}}
                              {\ltistackmapping{\ltiEnvConcatParen{\ltitvarp{}}{\ltiEnvp{}}}{\ova{\ltiT{}}}}
                              {\ltiEnvConcatParen{\ltitvar{}}{\ltiEnv{}}}
                              {\ltiunifyContextsSCLHS{\ltiCombinedThreadedEnv{}}
                                                    {\ltireplace{\ltitvar{}}{\ltitvarp{}}
                                                                {(\ltistackmapping{\ltiEnvp{}}{\ova{\ltiT{}}})}}
                                                    {\ltiEnv{}}}\\
      \ltiunifyContextsSCalign{\ltiCombinedThreadedEnv{}}
                              {\ltistackmapping{\ltiEnvConcatParen{\hastype{\ltivarp{}}{\ltiSp{}}}{\ltiEnvp{}}}{\ova{\ltiT{}}}}
                              {\ltiEnvConcatParen{\hastype{\ltivar{}}{\ltiS{}}}{\ltiEnv{}}}
                              {\ltiunifyContextsSCLHS{\ltiCombinedThreadedEnvp{}}
                                                     {\ltistackmapping{\ltiEnvp{}}{\ova{\ltiT{}}}}
                                                     {\ltiEnv{}}},
                                                 \text{ if } \ltiSsubtype{\ltiCombinedThreadedEnv{}}
                                                                          {\ltiEnvpp{}}
                                                                          {\ltiS{}}
                                                                          {\ltiSp{}}
                                                                          {\ltiCombinedThreadedEnvp{}}
    \end{array}
  \end{mathpar}
  \caption{Metafunctions for Extended Symbolic Closure language}
\end{figure}

\begin{figure}

  \[
    \boxed{\ltielabDriver{\ltiE{}}{\ltiEp{}}{\ltiT{}}
    \text{ Elaborates external language term \ltiE{} to internal language term \ltiEp{} and type \ltiT{}.
    }
    }
  \]

  \begin{mathpar}
    \infer[ElabDriver]
    {
     \exists \ltiFuel{}.\ 
     \ltitSstkjudgement{\ltimakeCombinedThreadedEnv{\ltiFuel{}}{\ltiEmptyClosureCache}}
                       {\ltiEmptyEnv}
                       {\ltiE{1}}
                       {\ltiT{}}
                       {\ltimakeCombinedThreadedEnv{\ltiFuelp{}}
                                                   {\ltiClosureCache{}}}
                       {\ltiE{2}}
                       \\
     \ltielimClos{\ltiClosureCache{}}{\ltiE{2}}{\ltiE{3}}
     \\
     \ltielimClosT{\varnothing}{\ltiClosureCache{}}{\ltiT{}}{\ltiTp{}}
    }
    {
    \ltielabDriver{\ltiE{1}}{\ltiE{3}}{\ltiTp{}}
    }
  \end{mathpar}

  \[
    \boxed{\ltielimClos{\ltiClosureCache{}}{\ltiE{}}{\ltiEp{}}
    \text{ Converts symbolic closures in \ltiE{} to explicit types in \ltiEp{}}
    }
  \]

  \[
  \begin{array}{llll}
    \ltielimClosalign{\ltiClosureCache{}}{\ltivar{}}
                     {\ltivar{}}
                     \\
    \ltielimClosalign{\ltiClosureCache{}}
                     {\ltiappinst{\ltiF{}}
                                 {\ova{\ltistackmapping{\ltiEnv{}}{\ova{\ltiR{}}}}}
                                 {\ltiE{}}}
                     {\ltiappinst{\ltielimClosLHS{\ltiClosureCache{}}{\ltiF{}}}
                                 {\ova{\ltistackmapping{\ltielimClosEnvLHS{\ltiClosureCache{}}{\ltiEnv{}}}
                                                       {\ova{\ltielimClosTLHS{\varnothing}{\ltiClosureCache{}}{\ltiR{}}}}}}
                                 {\ltielimClosLHS{\ltiClosureCache{}}{\ltiE{}}}}
                             \\
    \ltielimClosalign{\ltiClosureCache{}}{\ltisel{\ltiF{}}{\ltivar{}}}
                     {\ltisel{\ltielimClosLHS{\ltiClosureCache{}}{\ltiF{}}}{\ltivar{}}}
                     \\
    \ltielimClosalign{\ltiClosureCache{}}{\ltiRec{\ova{\ltivar{} = \ltiF{}}}}
                     {\ltiRec{\ova{\ltivar{} = \ltielimClosLHS{\ltiClosureCache{}}{\ltiF{}}}}}
                     \\
    \ltielimClosalign{\ltiClosureCache{}}
                     {\ltifuntparaminterface{\ova{\ltitvar{}}}
                                            {\ova{\ltistackmapping{\ltiEnv{}}{\ltiT{}}}}
                                            {\ltivar{}}
                                            {\ltiE{}}}
                     {\ltifuntparaminterface{\ova{\ltitvar{}}}
                                            {\ova{\ltistackmapping{\ltielimClosEnvLHS{\ltiClosureCache{}}{\ltiEnv{}}}
                                                                  {\ltielimClosTLHS{\varnothing}{\ltiClosureCache{}}{\ltiT{}}}}}
                                            {\ltivar{}}
                                            {\ltielimClosLHS{\ltiClosureCache{}}{\ltiE{}}}}
                     \\
    \ltielimClosalign{\ltiClosureCache{}}
                     {\ltiufunelab{\ova{\ltiufunelabentry{\ltiClosureID{}}}^n}
                                  {\ltivar{}}
                                  {\ltiE{}}}
                     {\ltielimClosLHS{\ltiClosureCache{}}
                                     {\ltimergeTaggedTermsLHS{\ltiF{1}}
                                                             {\ltimergeTaggedTermsLHS{...}{\ltiF{n}}}}}
                    , &n>0
                     \\
                     &&\text{where } \overrightarrowcaption{
                                      \ltilookup{\ltiClosureCache{}}{\ltiClosureID{i}}
                                      = \ltistackmapping{\ltiEnv{i}}{\ltiF{i}}
                                      }^{1 \leq i \leq n}
  \end{array}
  \]


  \[
    \boxed{\ltielimClosEnv{\ltiClosureCache{}}{\ltiEnv{}}{\ltiEnvp{}}
    \text{ Eliminates symbolic closures in \ltiEnv{} using \ltiClosureCache{}.
    }
    }
  \]

  \[
  \begin{array}{llllll}
    \ltielimClosEnvalign{\ltiClosureCache{}}{\ltiEmptyEnv}{\ltiEmptyEnv}
    \\
    \ltielimClosEnvalign{\ltiClosureCache{}}
                        {\ltiEnvConcatParen{\hastype{\ltivar{}}{\ltiT{}}}{\ltiEnv{}}}
                        {\ltiEnvConcat{\hastype{\ltivar{}}{\ltielimClosTLHS{\varnothing}{\ltiClosureCache{}}{\ltiT{}}}}
                                      {\ltielimClosEnvLHS{\ltiClosureCache{}}{\ltiEnv{}}}}
    \\
    \ltielimClosEnvalign{\ltiClosureCache{}}
                        {\ltiEnvConcatParen{\ltitvar{}}{\ltiEnv{}}}
                        {\ltiEnvConcat{\ltitvar{}}
                                      {\ltielimClosEnvLHS{\ltiClosureCache{}}{\ltiEnv{}}}}
  \end{array}
  \]

  \[
    \boxed{\ltielimClosT{\ova{\ltiClosureID{}}}{\ltiClosureCache{}}{\ltiT{}}{\ltiTp{}}
    \text{ Converts symbolic closures in \ltiT{} to explicit types in \ltiTp{}}
    }
  \]

  \[
  \begin{array}{llll}
    \ltielimClosTalign{\ova{\ltiClosureID{}}}{\ltiClosureCache{}}{\ltitvar{}}{\ltitvar{}}\\
    \ltielimClosTalign{\ova{\ltiClosureID{}}}{\ltiClosureCache{}}{\ltiTop}{\ltiTop}\\
    \ltielimClosTalign{\ova{\ltiClosureID{}}}{\ltiClosureCache{}}{\ltiBot}{\ltiBot}\\
    \ltielimClosTalign{\ova{\ltiClosureID{}}}{\ltiClosureCache{}}
                      {\ltiIFn{\ova{\ltiFn{\ltiT{}}{\ltiS{}}}}}
                      {\ltiIFn{\ova{\ltiFn{\ltielimClosTLHS{\ova{\ltiClosureID{}}}{\ltiClosureCache{}}{\ltiT{}}}
                                          {\ltielimClosTLHS{\ova{\ltiClosureID{}}}{\ltiClosureCache{}}{\ltiS{}}}}}}
                                          \\
    \ltielimClosTalign{\ova{\ltiClosureID{}}}{\ltiClosureCache{}}
                      {\ltiRec{\ova{\hastype{\ltivar{}}{\ltiT{}}}}}
                      {\ltiRec{\ova{\hastype{\ltivar{}}{\ltielimClosTLHS{\ova{\ltiClosureID{}}}{\ltiClosureCache{}}{\ltiT{}}}}}}
                                          \\
    \ltielimClosTalign{\ova{\ltiClosureID{}}}{\ltiClosureCache{}}
                      {\ltiMu{\ltitvar{}}{\ltiT{}}}
                      {\ltiMu{\ltitvar{}}{\ltielimClosTLHS{\ova{\ltiClosureID{}}}{\ltiClosureCache{}}{\ltiT{}}}}
                      , &\ltitvar{} \not\in \ova{\ltiClosureID{}}
                      \\
    \ltielimClosTalign{\ova{\ltiClosureID{}}}{\ltiClosureCache{}}
                      {\ltiPoly{\ova{\ltitvar{}}}{\ltiT{}}}
                      {\ltiPoly{\ova{\ltitvar{}}}{\ltielimClosTLHS{\ova{\ltiClosureID{}}}{\ltiClosureCache{}}{\ltiT{}}}}
                      , &\ova{\ltitvar{}} \cap \ova{\ltiClosureID{}} = \varnothing
                      \\
    \ltielimClosTalign{\ova{\ltiClosureID{}}}{\ltiClosureCache{}}
                      {\ltiClosureWithStkID{\ltiEnv{}}{\ltiClosureIDp{}}{\ltiufun{\ltivar{}}{\ltiE{}}}}
                      {\ltiClosureIDp{}}
                      , & \ltiClosureIDp{} \in \ova{\ltiClosureID{}}
                      \\
    \ltielimClosTalign{\ova{\ltiClosureID{}}}{\ltiClosureCache{}}
                      {\ltiClosureWithStkID{\ltiEnv{}}{\ltiClosureIDp{}}{\ltiufun{\ltivar{}}{\ltiE{}}}}
                      {\ltiPoly{\ova{\ltitvar{}}}
                               {\ltiMu{\ltiClosureIDp{}}
                                      {\ltielimClosTLHS{(\ltiClosureIDp{}, \ova{\ltiClosureID{}})}
                                                       {\ltiClosureCache{}}
                                                       {\ltiT{}}}}}
                      , & \ltiClosureIDp{} \not\in \ova{\ltiClosureID{}},
                          \ltiClosureID{} \not\in \ltifvLHS{\ltiT{}}
                      \\
                      &&\text{where } 
                      \ltilookup{\ltiClosureCache{}}{\ltiClosureIDp{}}
                      = \ltiClosure{\ltiEnv{}}
                                   {\ltifuntparaminterface{\ova{\ltitvar{}}}
                                                          {\ltiT{}}
                                                          {\ltivar{}}
                                                          {\ltiE{}}}

  \end{array}
  \]
  \caption{Elaboration Metafunctions for Extended Symbolic Closure language}
\end{figure}

{
\begin{lstlisting}[language=ml,mathescape=true]
let f = $\ltiufun{\text{x}}{\text{x}}$ in
  {left = $\ltiapp{\text{f}}{\text{1}}$, right = $\ltiapp{\text{f}}{\text{"a"}}$}
(* SC annotated *)
(* $\ltiInferred{\ltiClosureCache{} =%
      \ltiClosureCacheEntry{\text{c1}}%
                           {\ltiClosure{\ltiEmptyEnv}%
                                       {\ltiNotInferred%
                                        {\ltifuninterface{\ltiInferred{\ltiIFn{\ltiFn{\text{Int}}{\text{Int}} \ltiFn{\text{Str}}{\text{Str}}}}}%
                                                         {\text{x}}%
                                                         {\text{x}}}}}}$ *)
let f = $\ltiufunelab{\ltiInferred{\text{c1}}}{\text{x}}{\text{x}}$ in
  {left = $\ltiapp{\text{f}}{\text{1}}$, right = $\ltiapp{\text{f}}{\text{"a"}}$}
(* fully annotated *)
let f = $\ltifuninterface{\ltiInferred{\ltiIFn{\ltiFn{\text{Int}}{\text{Int}} \ltiFn{\text{Str}}{\text{Str}}}}}{\text{x}}{\text{x}}$ in
  {left = $\ltiapp{\text{f}}{\text{1}}$, right = $\ltiapp{\text{f}}{\text{"a"}}$}
\end{lstlisting}
}

{
\begin{lstlisting}[language=ml,mathescape=true]
let f = ${\ltiufun{\text{x}}{\text{x}}}$ in
  $\ltiapp{\text{f}}{\text{f}}$
(* SC annotated *)
(* $\ltiInferred{\ltiClosureCache{} =%
      \ltiClosureCacheEntry{\text{c1}}%
                           {\ltiClosure{\ltiEmptyEnv}%
                                       {\ltiNotInferred%
                                        {\ltifuninterface{\ltiInferred{\ltiFn{\ltiClosureWithStkIDParens{\ltiEmptyEnv}{\text{c1}}{\ltiufun{\text{x}}{\text{x}}}}%
                                                                             {\ltiClosureWithStkIDParens{\ltiEmptyEnv}{\text{c1}}{\ltiufun{\text{x}}{\text{x}}}}}}%
                                                         {\text{x}}%
                                                         {\text{x}}}}}}$ *)
let f = $\ltiufunelab{\ltiInferred{\text{c1}}}{\text{x}}{\text{x}}$ in
  $\ltiapp{\text{f}}{\text{f}}$
(* Fully annotated *)
let f = $\ltifuninterface{\ltiInferred{\ltiFn{\ltiMu{\text{a}}{\ltiFn{\text{a}}{\text{a}}}}{\ltiMu{\text{a}}{\ltiFn{\text{a}}{\text{a}}}}}}{\text{x}}{\text{x}}$ in
  $\ltiapp{\text{f}}{\text{f}}$
\end{lstlisting}
}

{
\begin{lstlisting}[language=ml,mathescape=true]
let f = $\ltiufun{\text{x}}{\text{x}}$ in
  {left  = $\ltiapp{\text{map}}{\text{f},\ltiapp{\text{Some}}{\text{1}}}$,
   right = $\ltiapp{\text{map}}{\text{f},\ltiapp{\text{Some}}{\text{"a"}}}$}
(* SC annotated *)
(* $\ltiInferred{\ltiClosureCache{} =%
      \ltiClosureCacheEntry{\text{c1}}%
                           {\ltiClosure{\ltiEmptyEnv}%
                                       {\ltiNotInferred%
                                        {\ltifuninterface{\ltiInferred{\ltiIFn{\ltiFn{\text{Int}}{\text{Int}}%
                                                                               \ltiFn{\text{Str}}{\text{Str}}}}}%
                                                         {\text{f,x}}%
                                                         {\ltiappinst{\text{f}}{\ltiInferred{\text{Int}}}{\text{x}}}}}}}$ *)
let f = $\ltiufunelab{\text{c1}}{\text{x}}{\text{x}}$ in
  {left  = $\ltiappinst{\text{map}}{\ltiInferred{\text{Int,Int}}}{\text{f},\ltiappinst{\text{Some}}{\ltiInferred{\text{Int}}}{\text{1}}}$,
   right = $\ltiappinst{\text{map}}{\ltiInferred{\text{Str,Str}}}{\text{f},\ltiappinst{\text{Some}}{\ltiInferred{\text{Str}}}{\text{"a"}}}$}
(* Fully annotated *)
let f = $\ltifuninterface{\ltiInferred{\ltiIFn{\ltiFn{\text{Int}}{\text{Int}}%
                                               \ltiFn{\text{Str}}{\text{Str}}}}}%
                         {\text{x}}%
                         {\text{x}}$ in
  {left  = $\ltiappinst{\text{map}}{\ltiInferred{\text{Int,Int}}}{\text{f},\ltiappinst{\text{Some}}{\ltiInferred{\text{Int}}}{\text{1}}}$,
   right = $\ltiappinst{\text{map}}{\ltiInferred{\text{Str,Str}}}{\text{f},\ltiappinst{\text{Some}}{\ltiInferred{\text{Str}}}{\text{"a"}}}$}
\end{lstlisting}
}

{
\begin{lstlisting}[language=ml,mathescape=true]
let f = $\ltiufun{\text{x}}{\ltiapp{\text{map}}{\ltiufun{\text{y}}{\text{y}},\text{x}}}$ in
  {left  = $\ltiapp{\text{f}}{\ltiapp{\text{Some}}{\text{1}}}$,
   right = $\ltiapp{\text{f}}{\ltiapp{\text{Some}}{\text{"a"}}}$}
(* SC annotated *)
(* $\ltiInferred{\ltiEnv{1} = {\hastype{\text{x}}{\text{Option[Int]}}}}$ *)
(* $\ltiInferred{\ltiEnv{2} = {\hastype{\text{x}}{\text{Option[Str]}}}}$ *)
(* $\ltiInferred{\ltiClosureCache{}} =$
     $\ltiInferred{%
      \ltiClosureCacheEntry{\text{c1}}%
                           {\ltiClosure{\ltiEmptyEnv}%
                                       {\ltiNotInferred%
                                        {\ltifuninterfaceLHS{\ltiInferred{\ltiIFn{\ltiFn{\text{Option[Int]}}{\text{Option[Int]}}%
                                                                                  \ltiFn{\text{Option[Str]}}{\text{Option[Str]}}}}}%
                                                         {\text{x}}}}}}$
              $\ltiappinst{\text{map}}% <- do not change indentation!
                           {\ltiInferred%
                            {\ltistackmapping{\ltiEnv{1}}{\text{[Int,Int]}},%
                             \ltistackmapping{\ltiEnv{2}}{\text{[Str,Str]}}}}%
                           {\ltiufunelab{\text{c1-1,c1-2}}{\text{y}}{\text{y}},\text{x}}$
     $\ltiInferred{%
      \ltiClosureCacheEntry{\text{c1-1}}%
                           {\ltiClosure{\ltiEnv{1}}%
                                       {\ltiNotInferred%
                                        {\ltifuninterface{\ltiInferred{\ltiFn{\text{Int}}{\text{Int}}}}%
                                                         {\text{y}}%
                                                         {\text{y}}}}}}$
     $\ltiInferred{%
      \ltiClosureCacheEntry{\text{c1-2}}%
                           {\ltiClosure{\ltiEnv{2}}%
                                       {\ltiNotInferred%
                                        {\ltifuninterface{\ltiInferred{\ltiFn{\text{Str}}{\text{Str}}}}%
                                                         {\text{y}}%
                                                         {\text{y}}}}}}$ *)
let f = $\ltiufunelab{\ltiInferred{\text{c1}}}{\text{x}}{\ltiapp{\text{map}}{\ltiufunelab{\ltiInferred{\text{c1-1,c1-2}}}{\text{y}}{\text{y}},\text{x}}}$ in
  {left  = $\ltiapp{\text{f}}{\ltiappinst{\text{Some}}{\ltiInferred{\text{Int}}}{\text{1}}}$,
   right = $\ltiapp{\text{f}}{\ltiappinst{\text{Some}}{\ltiInferred{\text{Str}}}{\text{"a"}}}$}
(* Fully elaborated *)
let f = $\ltifuninterfaceLHS{\ltiInferred{\ltiIFn{\ltiFn{\text{Option[Int]}}{\text{Option[Int]}}%
                                               \ltiFn{\text{Option[Str]}}{\text{Option[Str]}}}}}%
                         {\text{x}}$
          ${\ltiappinst{\text{map}}%<- do not change indentation!
                       {\ltiInferred%
                        {\ltistackmapping{\ltiEnv{1}}{\text{[Int,Int]}},%
                         \ltistackmapping{\ltiEnv{2}}{\text{[Str,Str]}}}}%
                       {\ltifuninterface{\ltiInferred%
                                         {\ltistackmapping{\ltiEnv{1}}{\ltiFn{\text{Int}}{\text{Int}}},%
                                          \ltistackmapping{\ltiEnv{2}}{\ltiFn{\text{Str}}{\text{Str}}}}}%
                                        {\text{y}}%
                                        {\text{y}},%
                        \text{x}}}$ in
  {left  = $\ltiapp{\text{f}}{\ltiappinst{\text{Some}}{\ltiInferred{\text{Int}}}{\text{1}}}$,
   right = $\ltiapp{\text{f}}{\ltiappinst{\text{Some}}{\ltiInferred{\text{Str}}}{\text{"a"}}}$}
\end{lstlisting}
}

%\input{colored}

% - Solution
%   - extend colored LTI with directed inference
%   - introduce constrained types
%   - derive data flow from (variances of) polymorphic variable occurrences 
%   - simple example
%     - (identity 1)
%     - demonstrate how this is checked with colored LTI
%     - compare to directed LTI:
%       - 2/----v
%         [x -> x]  Int
%          ^--------/1
%
%         1. Int flows to contravariant position
%         2. contravariant position flows to covariant position (because it's on the other side of ->)
%       - no loop, because variables not under different numbers of function types
%         - (we don't know the precise rule yet)
%   - complex example
% - Constraints
%   - advantages over colored LTI
%     - aids symbolic analysis
%       - because we derive potential dataflows, we don't need to over-approximate,
%         and thus trigger unneeded symbolic analysis
%         - which might then fail because of not enough contextual information
%   - disadvantages over colored LTI
%     - significant deviation from LTI
%       - constrained types
%       - aggressive local inference based on data flows
%     - not obvious how to prove soundness
%   - infinite loops
%     - how to manage cycles in inferred data flow 
%   - constraint solving
%     - constrained types
%       - literature (see symb.tex)
%   - flow diagrams
%     - see symb.tex
%   - relationship to colored LTI model
%     - see symb.tex
%   - related work
%     - ML_sub
%     - see: symb.tex
% - investigate implications 
%   - Remy ICFP '05
%     - (seems to) propagate information simultaneously in both directions like CLTI
%     - intro prose does a nice job explaining ML moving towards System F & challenges
%   - Joe B Wells 1994
%     - explains Church vs Curry style System F formulations
%     - some mentions of decidable fragments of System F
%   - Boxy types, ICFP '06
%     - explains higher-rank types
%       - types with forall quantifiers nested inside function types
%     - explains impredicativity
%       - being allowed to instantiate a type variable with a polytype (polymorphic type)
%     - explains "local type inference"
%       - a partial inference technique for a language with bounded, impredicative quantification,
%         and higher-rank types.
%     - explains "CLTI"
%       - reformulated bidirectional checking for F_sub so that the _type_ and not the _judgment form_
%         describes the direction in which type information flows
%     - CLTI's colors inspired their "boxy" types
%       - they outline differences in related work

%\chapter{Custom Typing rules}
%\label{chapter:symbolic:custom-rules}

% - Solution
%   - allow users to provide custom typing rules
%   - 
% - Constraints
%   - wildcard type from colored LTI useful
%   - custom error messages
%     - propagation via expected types
%       - outer-most wins
%   - using clojure.spec to conform/unform
%     - to rip apart and put syntax back together
%     - more robust than manual parsing
%   - differences with Turnstile
%     - in Turnstile, the macro *is* the rule
%       - here, we separate the two
%       - we preserve the macro call until evaluation
%       - use the typing rule to expand "under" the macro as many times as we want
%         - can do this 0-n times, thus compatible with directed LTI & symbolic analysis

%\include{hm-comparison}


\part{Related and Future Work}
\label{part:related-future-work}

%\chapter{Related Work to Typed Clojure}

% Cite a few of the early papers here.
%http://www.cs.washington.edu/research/projects/cecil/www/pubs/
\paragraph{Multimethods} 
\cite{MS02} and collaborators present a sequence of
systems~\cite{Chambers:1992:OMC,Chambers:1994:TMM,MS02} with statically-typed multimethods
and modular type checking.  In contrast to Typed Clojure, in these
system methods declare the types of arguments that they expect which
corresponds to exclusively using \clj{class} as the dispatch function
in Typed Clojure. However, Typed Clojure does not attempt to rule out
failed dispatches.

% one sentence
% TC based on TR, already covered

%\paragraph{Occurrence Typing} 
%Occurrence typing~\cite{TF08,TF10} extends the type 
%system with a \emph{proposition environment} that represents 
%the information on the types of bindings down conditional branches.
%These propositions are then used to update the types associated
%with bindings in the \emph{type environment} down branches
%so binding occurrences are given different types 
%depending on the branches they appear in, and the conditionals
%that lead to that branch.

% What's diff about TC from the related work
% small summary for deisel....
% - diesel supports x
%- - calculus supports some subset of x
% we support y, which covers most of x but also foo

% eg. multiple dispatch
%     nominal vs structural

% eg. run abritrary metaprogramming over dispatch in CLOS
%  more expressive

% type systems for mm or rows
% rows vs HMap
% - no poly in HMap
% - based on subtyping
% - rows based on polymorphism

\paragraph{Record Types} Row polymorphism~\cite{Wand89typeinference,CM91,HP91}, used
in systems such as the OCaml object system, provides many of the
features of HMap types, but defined using universally-quantified row
variables. HMaps in Typed Clojure are instead designed to be used with
subtyping, but nonetheless provide similar expressiveness, including
the ability to require presence and absence of certain keys. 

Dependent JavaScript~\cite{Chugh:2012:DTJ} can track similar
invariants as HMaps with types for JS objects. They must deal with
mutable objects, they feature refinement types and strong updates to
the heap to track changes to objects.

TeJaS~\cite{TeJaS}, another type system for JavaScript,
also supports similar HMaps, with the ability to
record the presence and absence of entries, but lacks a compositional
flow-checking approach like occurrence typing.

Typed Lua~\cite{Maidl:2014:TLO} has \emph{table types} which track
entries in a mutable Lua table.  Typed Lua changes the dynamic
semantics of Lua to accommodate mutability: Typed Lua raises a runtime
error for lookups on missing keys---HMaps consider lookups on missing
keys normal.

\paragraph{Java Interoperability in Statically Typed Languages}
Scala~\cite{OCD+} has nullable references for compatibility with Java.
Programmers must manually check for
\java{null} as in Java to avoid null-pointer exceptions. 


\paragraph{Other optional and gradual type systems}
%In addition to Typed Racket, 
Several other gradual type
systems have been developed for existing
dynamically-typed languages.  Reticulated Python~\cite{Vitousek14} is
an experimental gradually typed system for Python, implemented as a
source-to-source translation that inserts dynamic checks at language
boundaries and supporting Python's first-class object system. 
Clojure's nominal classes avoids the need to support
first-class object system in Typed Clojure, however HMaps offer an alternative to
the structural objects offered by Reticulated. Similarly,
Gradualtalk~\cite{gradualtalk} offers gradual typing for Smalltalk,
with nominal classes.

Optional types
%, requiring less implementation effort and avoiding
%runtime cost, 
have been  adopted in industry, including Hack~\cite{hack}, and Flow~\cite{flow} and
TypeScript~\cite{typescript}, two extensions of JavaScript. These
systems  support  limited forms of occurrence typing,
and do not include the other features we
present.

%  \item GradualTalk
%  \item Flow
%\end{itemize}




%\Dchapter{Related work\either{ to Automatic Annotations}{}}

%The field of dynamic analysis has a rich history.
%Ball~\cite{ball1999concept}
%introduces frequency spectrum analysis,
%an approach that observes a running program
%that is similar to our instrumentation approach.
%Mock~\cite{mock2003dynamic}
%makes the case for efficient profiling of programs
%to better facilitate usage of instrumentation.
%
%Value Profiling is another related area which characterises
%programs based on their running entities.
%
%Daikon~\cite{ernst2001dynamically}
%uses dynamic analysis to recover likely program invariants
%in C programs.
%
%Dynamic type inference has been attempted in many different
%areas.
%Rubydust~\cite{An10dynamicinference}
%infers static types for Ruby. They prove the generated types
%sound, which we do not. 
%\begin{verbatim}
%Conversely, they do not generate
%recursive types, but recursive types in ruby are probably
%nominal, so how different are we?
%\end{verbatim}
%
%In the context of JavaScript, several usages of this technique
%can be found.
%JSTrace~\cite{saftoiu2010jstrace}
%generates types for (?).
%Separately, work has been done to generate JSDoc-like annotations~\cite{odgaard2014}.
%TypeDevil~\cite{pradel2015typedevil}
%uses dynamic analysis to warn JavaScript programmers of possible inconsistencies
%in their programs.
%
%Work in recovering context-free grammars is most related to our algorithm
%to recover recursive types.
%% TODO Shamir~\cite{shamir1962remark} notes that it is impossible
%% TODO \cite{knobe1976method}
%
%In the context of machine learning, 
%this area is called grammar induction or language learning.  % according to vcrepinvsek2005inferring
%% TODO Wang\cite{wang1998grammar} summarises 
%{\v{C}}repin{\v{s}}ek et. al~\cite{vcrepinvsek2005inferring}
%use genetic programming to infer context-free grammars
%for domain-specific languages.
%Most work in this area assume both positive and negative
%examples. We cannot distinguish between these two in
%our system, so we assume all examples are positive.
%
%Chen~\cite{chen1995bayesian} uses Bayesian inference to converge
%on a suitable grammar, given examples.
%
%There has been recent interest in approximate type inference.
%
%Pluquet et. al~\cite{marot2009fast} investigate heuristics
%to quickly infer types in dynamic programs.
%So does Milojkovi{\'c}
%\cite{milojkovic2016exploring}.
%Spasojevi{\'c} et. al~\cite{spasojevic2014mining}
%compare types across a cross section of projects to improve
%inference.
%
%Adamsen et. al~\cite{adamsen2016analyzing} verify test suite completeness using a hybrid approach of lightweight dependency analysis, static type checking and dynamic instrumentation.
%
%% Inference and Evolution of TypeScript Declaration Files
%% - they submit pull requests from their tool's output
%% https://cs.au.dk/~amoeller/papers/tstools/paper.pdf
%
%% Automatic TS annotations from JSON (including recursive types)
%% https://github.com/shakyShane/json-ts

\paragraph{Automatic annotations}
There are two common implementation strategies for automatic annotation tools. The first
strategy, ``ruling-out'' (for invariant detection), assumes all invariants are true 
and then use runtime analysis results to rule out
impossible invariants. The second ``building-up'' strategy (for dynamic type inference)
assumes nothing and uses runtime analysis results to build up invariant/type knowledge.

Examples of invariant detection tools include Daikon~\infercitep{ernst2001dynamically},
DIDUCE~\infercitep{hangal2002tracking}, and Carrot~\infercitep{pytlik2003automated}, and
typically enhance statically typed languages with more expressive types or contracts.
Examples of dynamic type inference include our tool, Rubydust \infercitep{An10dynamicinference},
JSTrace~\infercitep{saftoiu2010jstrace}, and TypeDevil~\infercitep{pradel2015typedevil},
and typically target untyped languages.

Both strategies have different space behavior with respect to representing
the set of known invariants.
The ruling-out strategy typically uses a lot of memory at the beginning,
but then can free memory as it rules out invariants. For example, if
\texttt{odd(x)} and \texttt{even(x)} are assumed, observing \texttt{x = 1}
means we can delete and free the memory recording \texttt{even(x)}.
Alternatively, the building-up strategy uses the least memory storing
known invariants/types at the beginning, but increases memory usage
as more the more samples are collected. For example, if we know
\texttt{x : Bottom}, and we observe \texttt{x = "a"} and \texttt{x = 1}
at different points in the program, we must use more memory to
store the union \texttt{x : String $\cup$ Integer} in our set of known invariants.

\paragraph{Daikon}
Daikon can reason about very expressive relationships between variables
using properties like ordering ($x < y$), linear relationships ($y = ax + b$),
and containment ($x \in y$). It also supports reasoning with ``derived variables''
like fields ($x.f$), and array accesses ($a[i]$).
%
Typed Clojure's dynamic inference can record heterogeneous data structures
like vectors and hash-maps, but otherwise cannot express relationships
between variables.

There are several reasons for this. The most prominent is that Daikon
primarily targets Java-like languages, so inferring simple type information
would be redundant with the explicit typing disciplines of these languages.
On the other hand, the process of moving from Clojure to Typed Clojure
mostly involves writing simple type signatures without dependencies
between variables. Typed Clojure recovers relevant dependent information
via occurrence typing~\infercitep{TF10}, and gives the option to manually annotate necessary
dependencies in function signatures when needed.

\paragraph{Reverse Engineering Programs with Static Analysis}
Rigi~\infercitep{muller1992reverse} analyzes
the structure of large software systems,
combining static analysis 
with a user-facing graphical environment to allow users to view and manipulate
the in-progress reverse engineering results.
We instead use a static type system as a feedback mechanism,
which forces more aggressive compacting of generated annotations.

Lackwit~\infercitep{o1997lackwit} uses static analysis to identify abstract 
data types in C programs. Like our work, they share representations between
values, except they use type inference with representations encoded as types.
Recursive representations are inferred via Felice and Coppos's
work on type inference with recursive types~\infercitep{cardone1991type},
where we rely on our imprecise ``squashing'' algorithms over incomplete runtime samples.

Soft Typing~\infercitep{CF91} uses static analysis to insert runtime checks into untyped
programs for invariants that cannot be proved statically. Our approach is instead to let
the user check the generated annotations with a static type system, with static type errors
guiding the user to manually add casts when needed.

\paragraph{Schema Inference}
\infercitet{baazizi2017schema}
infer structural properties of JSON data using a custom JSON schema format.
Their schema inference algorithm proceeds in two stages:
schema inference and schema fusion.
This resembles our collection and naive type environment construction phases.
There are slight differences between schema fusion and our approach.
Schema fusion upcasts heterogeneous array types to be homogeneous, where
we maintain heterogeneous vector types until a differently-sized
vector type is found in the same position.
We also support function types, which JSON lacks.
While they support nested data, they do not attempt to factor out common types as names
or create recursive types like our squashing algorithms.

\infercitet{discala2016automatic}
present a machine learning algorithm to translate denormalized
and nested data that is commonly found in NoSQL databases to traditional
relational formats used by standard RDBMS.
A key component is a schema generation algorithm which arranges related
data into tables via a matching algorithm which discovers related attributes.
Phases 1 and 2 of their algorithm are similar to our local and global
squashing algorithms, respectively, in that first locally accessible information
is combined, and then global information.
%TODO do they infer recursive schemas?
They identify groups of attributes that have (possibly cyclic) relationships.
%They choose a loose method of relating entities (soft functional dependencies) to compensate
%for data inconsistencies and to help users learn about their data via higher-level abstractions.
Where our squashing algorithms for map types are based on (sets of) keysets---on the 
assumption that related entities use similar keysets---they also join attributes
based on their similar values.
This enables more effective entity matching via equivalent attributes
with different names (e.g., ``Email'' vs ``UserEmail'').
Our approach instead assumes programs are somewhat internally consistent, and instead
optimizes to handle missing samples from incomplete dynamic analysis.

%TODO Inferring XML Schema Definitions from XML Data - Bex, Neven, Vansummeren

% Inference and Evolution of TypeScript Declaration Files
% - they submit pull requests from their tool's output
% https://cs.au.dk/~amoeller/papers/tstools/paper.pdf
\paragraph{Other Annotation Tools}
Static analyzers
for JavaScript
(TSInfer~\infercitep{kristensen2017inference}) and for Python (Typpete~\infercitep{typette18}
and PyType~\infercitep{pytype})
automatically annotate code with types.
PyType and Typpete inferred \texttt{nodes}
as \texttt{(? -> int)}
and \texttt{Dict[(Sequence, object)] -> int}, respectively---our tool 
infers it as \clj{[Op -> Int]} by also generating a compact recursive
type.
Similarly, a class-based translation of
inferred both \texttt{left} and \texttt{right}
fields
as \texttt{Any} by PyType, and as \texttt{Leaf} by Typpete---our tool
uses \clj{Op},
a compact recursive type containing \emph{both} \clj{Leaf} and \clj{Node}.
This is similar to our experience with TypeWiz in \Dchapref{infer:chapter:intro}.
(We were unable to install TSInfer.)

NoRegrets~\infercitep{noregrets2018} uses dynamic analysis to learn how a program
is used, and automatically runs the tests of downstream projects to
improve test coverage.
Their \emph{dynamic access paths} represented as
a series of \emph{actions} are analogous to our paths of path elements.

% distinguishes public/private API

% Python
% - MaxSMT-Based Type Inference for Python 3
%  - cites other python based projects
%  - https://link.springer.com/content/pdf/10.1007%2F978-3-319-96142-2_2.pdf
% - pytype
%  - static analysis to generate python annotations
%  - https://github.com/google/pytype
% - pyannotate
%   - dynamic analysis
%   - https://github.com/dropbox/pyannotate

% A Survey of Dynamic Analysis and Test Generation for JavaScript
%  - http://mp.binaervarianz.de/js_survey_2017.pdf


%% NOTE: Haven't pursued the following work yet

%\paragraph{How dynamic languages are used}
%Several languages have seen similar investigations
%into their idioms as I am proposing for Clojure.
%
%A popular motivation is to discover which type system features to support
%when retrofitting a type system.
%% FIXME the is \AAkerblom but there's an error.. also in the bibliography
%Akerblom et. al~\cite{Akerblom:2014:TDF:2597073.2597103} trace dynamic features in Python programs
%via instrumentation. They measured the prevalence of dynamic features in startup versus
%user code, and recorded usage frequencies for a set of dynamic features.
%They concluded dynamism is prevalent in Python, and thus should be supported
%in a retrofitted type system for Python.
%A study along similar lines is also applicable to Clojure, in particular analysing Typed
%Clojure's support for Clojure's dynamic features.
%
%Calla{\'u} et al. \cite{Callau2013} also conducted a large-scale study of
%dynamic Smalltalk idioms to inform future language extensions tooling support.
%Notably, they further perform a qualitative analysis aiming to identify
%the reasons why Smalltalk use these features in the first place, and
%whether they can be replaced with more predictable features. They also 
%measure which kinds of projects (e.g., testing frameworks, user-level libraries, or core system libraries) 
%use particular features more frequently.
%Due to the their prevalence in the open-source Clojure ecosystem,
%Typed Clojure has mainly been tested on user-level libraries.
%We could predict Typed Clojure's applicability to other kinds of projects
%by gathering similar data on how frequently different types of Clojure libraries use
%Clojure's various features.
%
%Andreasen et. al~\cite{Andreasen2016TraceTA} developed
%\emph{trace typing} to explore the design space of JavaScript type systems. 
%Using runtime observations, they studied which control flow techniques
%are used most often in JavaScript programs, and thus, which should
%be supported by an effective type system for JavaScript.
%Typed Clojure implements occurrence typing to reason about control
%flow in Clojure which seems to work well in practice, but a similar
%quantitative analysis could reveal further insights.

%Runtime analysis \cite{Mastrangelo:2015:UYO:2814270.2814313}

%\cite{Mastrangelo:2015:UYO:2814270.2814313} 

\chapter{Related Work to Extensible Type Systems%and Interleaving Type Checking with Expansion
}

Turnstile~\cite{Chang2017TSM} type checks a program during expansion
by repurposing the Racket macro system.
Instead of the more standard approach of providing separate rules to check a macro, Turnstile
typing rules specify both the expansion and checking semantics, and so ensuring the
two are compatibile becomes automatic.
On the other hand, Typed Clojure does not have the goal of allowing users to override
how language primitives type check. Instead, our goal is to provide
a simple interface to write type rules for library functions and macros
in a style that hides the necessary bookkeeping surrounding occurrence
typing and scope management.

SugarJ~\cite{Erdweg2011SJ}
adds syntactic language extensibility to languages like Java, such as pair
syntax, embedded XML, and closures.
Desugarings are expressed as rewrite rules to plain Java.
Similarly, work on \emph{type-specific languages}~\cite{omar2014safely}
adds extensible systems for the definitions of specialized syntax literals
to existing languages.
The \emph{type} of an expression determines how it is parsed and elaborated.

% this paper has a great related works section that differentiates
% the strategies of several typed metaprogramming techniques
SoundX~\cite{Lorenzen2016STS} presents a solution to a common
dilemma in typed metaprogramming: whether to desugar before
type checking, or vice-versa.
The authors present a system where a form is type checked before 
being desugared, with a guarantee that only well-typed code is generated.
Programmers specify desugarings with a combination of typing and rewriting rules, 
which are then connected to form a valid type derivation
in a process called \emph{forwarding}.
We will explore whether we can get the same effect in Typed Clojure
without requiring the user to understand typing rules.
%For example, Scala macros~\cite{Burmako2013SML} interleave type checking and
%desugaring

Ziggurat~\cite{Fisher06staticanalysis} allows programmers to define
the static and dynamic semantics of macros separately. To demonstrate its
broad applicability, they choose Scheme-like macros that generate assembly code
for the dynamic semantics.
They advocate building towers of static analyses, so
macros can be statically checked in terms the static semantics of other macros, instead
of just their assembly code expansions which would otherwise be too difficult to check.
This idea resembles our prototypes in defining custom typing rules for functions and macros in Typed Clojure,
where the dynamic semantics are defined by runtime Clojure constructs (\texttt{defn}
and \texttt{defmacro}), and towers of static semantics are progressively specified in terms of the static
analysis of other Clojure forms.

Type Tailoring~\cite{greenmanttailoring} is an approach to provide more information
to a host type system than it might be capable of by itself.
In particular, the authors use the host platform's metaprogramming functionality
to refine the types of calls based on the program syntax alone, as well as improve
error messages by incorporating surface syntax. Their experiments are based in Typed Racket, that fully expands
syntax before checking it. Since Typed Clojure recently changed to interleave macroexpansion
and type checking, we could extend this technique to also refine calls based on the
types of their arguments (like SoundX).

Other work is relevant to our investigations of improving the user experience
of Typed Clojure. SweetT~\cite{pombrio2018inferring} automatically infers type rules
for syntactic sugar. Helium~\cite{Heeren2003STI} provides hooks into the type inference
process for domain-specific type error messages.

\chapter{Related Work to Symbolic Closures% and Combining Type Checking with Symbolic Analysis
}

\paragraph{Local Type Inference}
Symbolic closures were originally designed as an extension of Local Type Inference~\cite{PierceLTI}.
Our presentation omitted their bidirectional checking (we did not propagate type information down the syntax
tree using synthesis/checking rules)
and so was not a superset of Local Type Inference---in
particular, it does not take advantage of the relaxed optimality conditions when inferring type
arguments in checking mode.
However, adding back bidirectional checking should be possible by starting with the rules of Local
Type Inference, adding a \emph{synthesis} rule for functions that introduces a symbolic
closure, application and subtyping rules for symbolic closures, and some side conditions to restrict 
how a symbolic closure may be reasoned about (like our ``must not contain symbolic closures''
conditions scattered in various rules).
This way, a symbolic closure should only be introduced where Local Type Inference fails---when
a type of a function must be synthesized---and so seems more likely to be a superset of 
Local Type Inference.

Colored Local Type Inference~\cite{coloredlti01} extends Local Type Inference with partial
information propagation. Their type inference algorithm does not use explicit synthesis/checking
rules, instead passing ``prototypes'' \emph{P}
down the syntax tree that containing partial expected type information used for type checking.
A prototype is a type \emph{T} extended with the wildcard ``?'', denoting unknown information,
and the specific shape of a prototype denotes which type rule to use.
The rule inferring unannotated functions \ltiufun{\ltivar{}}{\ltiE{}} requires a prototype \ltiPolyFn{T}{}{P}, where
\emph{T} is the fully known expected type for {\ltivar{}}.
A symbolic closure could be introduced when checking an unannotated function with prototype
\ltiPolyFn{P}{}{P'} or ``?'' (the equivalent of Local Type Inference's ``synthesis'' rules).
In the more complicated case of \ltiPolyFn{P}{}{P'}, a symbolic reduction of 
\ltiufun{\ltivar{}}{\ltiE{}} is required to ensure it at least conforms its the prototype.
For example, inferring the type of \ltiapp{\text{map}}{\ltiufun{\ltivar{}}{\ltiE{}},[1,2,3]} with
Colored Local Type Inference (where \text{map} has type ``\ltiPolyFn{\ltiFn{\text{a}}{\text{b}},\text{List[a]}}{\text{a,b}}{\text{List[b]}}'')
checks \ltiufun{\ltivar{}}{\ltiE{}} with prototype \ltiPolyFn{\text{?}}{}{\text{?}}.
We can be optimistic and check the function 
at the largest (most specific) subtype of 
\ltiPolyFn{\ltiBot}{}{\ltiTop}
that matches \ltiPolyFn{\text{?}}{}{\text{?}}, which is \ltiPolyFn{\ltiBot}{}{\ltiTop}.
This ensures that the function at least conforms to the most optimistic interpretation of its
prototype, and then by returning a symbolic closure type instead of \ltiPolyFn{\ltiBot}{}{\ltiTop}
allows us to check more specific requirements later.
Of course, to fully check this example, it requires that we specify how type argument synthesis works with
symbolic closures, but it at least illustrates how symbolic closures relate to the rest of the system.

Spine-local type inference~\cite{Jenkins:2018:STI:3310232.3310233}
explores Local Type Inference in the context of System F (without subtyping).
They present a greedy type argument synthesis algorithm
which more aggressively propagates type information
to an application's arguments.
To check arguments, type variable instantiations are guessed
based on the expected type of the application.
For example, when checking \ltiapp{\text{id}}{\ltiufun{\ltivar{}}{\ltiE{}}}
with expected type \ltiFn{\ltiT{}}{\ltiS{}},
where \text{id} has type \ltiPolyFn{\ltitvar{}}{\ltitvar{}}{\ltitvar{}},
\ltitvar{} would be guessed to have type \ltiFn{\ltiT{}}{\ltiS{}}
and then {\ltiufun{\ltivar{}}{\ltiE{}}} would be checked at that type.
This would fail if the application was in synthesis mode.
In this specific example, symbolic closures would allow the checking
of \ltiufun{\ltivar{}}{\ltiE{}} to be delayed to when more type information
is available, in either checking or synthesis modes.
Unfortunately, it does not seem that their algorithm can check
\ltiapp{\text{map}}{\ltiufun{\ltivar{}}{\ltiE{}},[1,2,3]}
even in checking mode, and so does not seem to assist us in solving similar
problems with symbolic closures.
This case does not check because only the type of \ltiE{}
would be apparent from an expected type, not the type of \ltivar{}.

% Spine-local type inference
% - Judgement \vdash^P digs down an application to find the head
%   - happens naturally with symbolic closures
% - they use metavariables to solve direct applications
%   - can they check things like (let [x (fn [y] (inc y))] (x 1)) ?
% - they have polymorphism but not subtyping (plain System F)
%   - they speculate about extending to Fsub in related works
%   - they mention Hosoya & Pierce's "challenges" to fix hard-to-synthesize terms
% - they type check arguments left-to-right in a polymorphic application
%   - can't tell if that's different from inferring the data flow from a polytype 
%     and then checking in that order
% - their sense of "locality" is less ambitious than symbolic closures
%   - see "Type Inference Failures" section
% - good related works section for "Impredicative Polymorphism"

\paragraph{Mixing Symbolic Execution and Type Checking}
Mix~\cite{Khoo2010MTC} allows an interplay of symbolic execution~\cite{King1976SEP} with type checking
by providing syntactic regions,
with terms
\MixTregion{\ltiE{}} signaling to use type checking for {\ltiE{}},
and
\MixSregion{e} for symbolic execution.
In Mix, for example, the term
%
\[
\MixSregion{\ltilet{\text{id}}{\ltiufun{\ltivar{}}{\ltivar{}}}
                  {\MixTregion{  ...\ \MixSregion{\ltiapp{\text{id}}{3}}
                               \ ...\ \MixSregion{\ltiapp{\text{id}}{3.0}}
                               \ ... }}}
\]
%
symbolically executes \MixSregion{\ltiapp{\text{id}}{3}}
and
\MixSregion{\ltiapp{\text{id}}{3.0}}, propagating result types
\text{Int} and \text{Real} back to the typed regions, respectively.
Comparatively, symbolic closures integrates only a small amount of
symbolic execution with a type system, but in such a way that delayed symbolic computations
may pass between typed regions.
Since Mix cleanly separates symbolic execution and type checking and its formalism does not support
function types, it is difficult to compare the two approaches.
In rough terms, symbolic closures use typed regions by default and automatically adds symbolic regions
around unannotated functions.
%
\[
\MixTregion{\ltilet{\text{id}}{\MixSregion{\ltiufun{\ltivar{}}{\ltivar{}}}}
                   {  ...\ \ltiapp{\text{id}}{3}
                    \ ...\ \ltiapp{\text{id}}{3.0}
                    \ ... }}
\]
%
Typing rules are then added to introduce a symbolic closure type
and also to apply them, which involves checking the delayed body in a typed region.
%
\begin{mathpar}
\infer[]
  {}
  { \ltitjudgementNoElab{\ltiEnv{}}{\MixSregion{\ltiufun{\ltivar{}}{\ltiE{}}}}
                        {\ltiClosure{\ltiEnv{}}{\ltiufun{\ltivar{}}{\ltiE{}}}}
  }

\infer[]
  { \ltitjudgementNoElab{\ltiEnv{}}{\MixTregion{\ltiF{}}}
                        {\ltiClosure{\ltiEnvp{}}{\ltiufun{\ltivar{}}{\ltiEp{}}}}
    \\\\
    \ltitjudgementNoElab{\ltiEnv{}}{\MixTregion{\ltiE{}}}{\ltiS{}}
    \\
    \ltitjudgementNoElab{\ltiEnvConcat{\ltiEnvp{}}{\hastype{\ltivar{}}{\ltiS{}}}}
                        {\MixTregion{\ltiEp{}}}
                        {\ltiT{}}
  }
  { \ltitjudgementNoElab{\ltiEnv{}}{\MixTregion{\ltiapp{\ltiF{}}{\ltiE{}}}}{\ltiT{}}
  }
\end{mathpar}

When forced to delineate type checking from symbolic execution like this, it interesting to ask to what degree symbolic
closures even uses symbolic execution.
Our view is that (at least) symbolic closures symbolically execute the runtime-closure introduction rule.
%
\begin{mathpar}
\infer[]
  {}
  {
  \opsem {\openv{}}
         {\ltiufun{\x{}}{\e{}}}
         {\closurenosuffix{\openv{}}{\ltiufun{\x{}}{\e{}}}}
         }
\end{mathpar}
%
The symbolic closure type
{\ltiClosure{\ltiEnv{}}{\ltiufun{\ltivar{}}{\ltiE{}}}}
is then the symbolic value of the runtime closure
{\closurenosuffix{\openv{}}{\ltiufun{\x{}}{\e{}}}},
related by the following typing rule.
%
\begin{mathpar}
\infer []
{ 
  \overrightarrowcaption{
  \ltitjudgementNoElab{}{\ltiEnvLookup{\openv{}}{y}}{\ltiEnvLookup{\ltiEnv{}}{y}}
  }
  ^{y \in dom(\openv{})}
              }
{ \ltitjudgementNoElab {\ltiEnvp{}}
                       {\closurenosuffix
                        {\openv{}}
                        {\ltiufun{\ltivar{}}{\ltiE{}}}}
                       {\ltiClosure{\ltiEnv{}}{\ltiufun{\ltivar{}}{\ltiE{}}}}
          }
\end{mathpar}
%
As evidenced by the lack of symbolic regions in the above application rule,
a ``symbolic reduction'' of a symbolic closure is not particularly related
to symbolic execution---it merely kicks off some delayed type checking.
However, \ltiEnv{}, \ltiS{}, and \ltiT{} in that rule may contain symbolic closure
types, so symbolic values are being used to reason about the program.

%\begin{mathpar}
%\infer[]
%{ \opsem {\openv{}}
%         {\e{f}}
%         {\closurenosuffix {\openv{c}} {\abs {\x{}} {\t{}} {\e{b}}}}
%         \\
%  \opsem {\openv{}}
%         {\e{a}}
%         {\v{a}}
%         \\
%  \opsem {\extendopenv {\openv{c}} {\x{}} {\v{a}}}
%         {\e{b}}
%         {\v{}}
%       }
%{ \opsem {\openv{}}
%         {\appexp {\e{f}} {\e{a}}}
%         {\v{}}
%       }
%\end{mathpar}

%Symbolic closures type check an anonymous function if annotated, otherwise it is treated symbolically.
%As the authors envision, this is akin to automatically inserting
%the mode of a code region based on its context, with a Mix-like language
%becoming the intermediate language.

Mix also uses symbolic execution to enhance simple type systems with flow-sensitivity.
For example, the following Mix program uses symbolic execution 
to flow-sensitively reason about \text{int?}, a predicate that returns true only for integer values.
%
\[
\MixSregion{\ltilet{\text{f}}{\ltiufun{\ltivar{}}{(\ltiif{\ltiapp{\text{int?}}{\ltivar{}}}{\ltivar{}}{\textsf{nil}})}}
                  {\MixTregion{  ...\ \MixSregion{\ltiapp{\text{f}}{3}}
                               \ ...\ \MixSregion{\ltiapp{\text{f}}{3.0}}
                               \ ... }}}
\]
%
The symbolic regions determine
\MixSregion{\ltiapp{\text{f}}{3}} has type \text{Int} and
\MixSregion{\ltiapp{\text{f}}{3.0}} type \text{Nil} via symbolic execution.
Symbolic closures are instead designed to be compatible with flow-sensitive type systems like occurrence typing~\cite{TF10}.
Here is the analogous program using symbolic closures.
%
\[
\MixTregion{\ltilet{\text{f}}{\MixSregion{\ltiufun{\ltivar{}}{(\ltiif{\ltiapp{\text{int?}}{\ltivar{}}}{\ltivar{}}{\textsf{nil}})}}}
                  {  ...\ {\ltiapp{\text{f}}{3}}
                               \ ...\ {\ltiapp{\text{f}}{3.0}}
                               \ ... }}
\]
%
Now, let us assume occurrence typing is also used to check this program, and
that \text{int?} is typed as a predicate for integers.
The call to
{\ltiapp{\text{f}}{3}}
triggers the symbolic reduction
%
\[
\ltitjudgementNoElab{\hastype{\ltivar{}}{\text{Int}}}
                    {(\ltiif{\ltiapp{\text{int?}}{\ltivar{}}}{\ltivar{}}{\textsf{nil}})}
                    {\text{Int}}
\]
%
which has type \text{Int}, because the else-branch is unreachable, and 
similarly {\ltiapp{\text{f}}{3.0}} triggers
%
\[
\ltitjudgementNoElab{\hastype{\ltivar{}}{\text{Real}}}
                    {(\ltiif{\ltiapp{\text{int?}}{\ltivar{}}}{\ltivar{}}{\textsf{nil}})}
                    {\text{Nil}}
\]
%
which has type \text{Nil}, because the then-branch is unreachable.

% >> talk about occurrence tpying.

% Let arguments go first

% Dunfield works I need to compare to
% - Greedy Bidirectional Polymorphism
% - Sound and complete bidirectional typechecking for higher-rank polymorphism with existentials and indexed types
% - Complete and Easy Bidirectional Typechecking for Higher-Rank Polymorphism

% Other works with undecidable type checking
% - Hybrid type checking - Knowles, Flanagan

\paragraph{Intersection Type Checking}
In hindsight, the idea behind symbolic closures resembles intersection type checking,
where the same code may be checked at multiple types.
Carlier and Wells~\cite{carlier2005expansion} give an approachable explanation of ``expansion'',
a mechanism that informs an intersection type system when it should
check the same term at different types.
This is achieved by splicing typing rules (like intersection-introduction) into existing typing
derivations that are derived from the principal typings of subterms.
In contrast, symbolic closures do not assume principal types are available, and
delays the construction of typing derivation(s) for a delayed term
altogether until it is obvious how to construct it.
Then, it is matter of combining a symbolic closure's typing derivations
to recover the (intersection) type it was used at.

%In Section 3.1, they provide a motivating example for expansion, constructed to be untypable 
%with simple (non-recursive) types, and show how expansion assigns it a type.

%TODO
%With the full power of intersection type systems, a principal type expresses

%TODO how do SC relate to this statement?
% - However, computing these principal typings is as expensive as evaluation, 
%   for the simple reason that principal typings for a term in the full system
%   express all of the information in the term’s β-normal form
% - do SC perform less reductions? eg. (lambda (x) x) does not reduce, but I assume
%   the Coppo algorithm constructs a principal type for it. Then, what if it is
%   use like (lambda (z) (let ([f (lambda (x) x)]) (f (f (f z))))) ?
%   Does it still infer an intersection type for f? Obviously, SC's do nothing here.

% survey
% Expansion: the Crucial Mechanism for Type Inference with Intersection Types: A Survey and Explanation1
% https://www.sciencedirect.com/science/article/pii/S1571066105050656
% - "expansion" seems similar to inferring intersection types via symbolic closures
%   - Section 6.2 talks about type inference for rank-2 intersection types
%     - advantage is that expansion never has to introduce intersections under ->
%       - do we do this with symbolic closures?
% - "Expansion is an operation on typings that simulates the effect of splicing in typing rules
%    uses at nested positions in some derivation of that typing."
% - omega expansion looks very similar to symbolic closures types (Section 4)
%   - at least in that it embeds the term in the type to track its origin 
% - Section 5.2 talks about cost of type inference == beta reduction

% (Intersection type systems)
% Principal Types and Unification for Simple Intersection Type Systems
% https://www.sciencedirect.com/science/article/pii/S0890540185711418

\paragraph{Higher-order Control Flow Analysis}

%% (found via the expansion survey as an application of intersection types)
% http://delivery.acm.org/10.1145/260000/258951/p1-banerjee.pdf?ip=140.182.72.36&id=258951&acc=ACTIVE%20SERVICE&key=EA62C54EFA59E1BA%2EEC3C9CD27046E2ED%2E4D4702B0C3E38B35%2E4D4702B0C3E38B35&__acm__=1553786311_eb5536bb95ba54f3a2979f7a216e898e
% A Modular, Polyvariant, and Type-Based Closure Analysis, Anindya Banerjee

%TODO aka. Closure-analysis 
% http://citeseerx.ist.psu.edu/viewdoc/summary?doi=10.1.1.36.6128
% Analysis and Efficient Implementation of Functional Programs (1991), Peter Sestoft
Closure analysis~\cite{sestoft1991analysis} approximates the set of arguments which a given
function may be applied, as well as which functions a given term
may evaluate to.
Each function term is labelled 
\Sesoftlambda{\Sesoftlabel}{\ltivar{}}{\ltiE{}}, where the label {\Sesoftlabel}
abstracts over the set of all runtime closures
\closurenosuffix{\openv{}}{\Sesoftlambda{}{\ltivar{}}{\ltiE{}}}
where runtime environment {\openv{}} can choose arbitrary bindings for {\ltiE{}}'s
free term variables.
In contrast, a ``tagged'' symbolic closure term
\ltiufunelab{\ltiClosureID{}}{\ltivar{}}{\ltiE{}}
uses identifier {\ltiClosureID{}} to stand for
\closurenosuffix{\openv{}}{\ltiufun{\ltivar{}}{\ltiE{}}}
where the bindings in the runtime environment {\openv{}} are of the types
given in the type environment \ltiEnv{} where the term was encountered
by the type checker.
In an unrestricted setting of symbolic closures, the same term
may be used with different identifiers.
For example, in
%
\[
\ltilet{\text{f}}{\ltiufun{\text{x}}{\ltiufun{\text{y}}{\text{x}}}}{...{\ltiapp{\text{f}}{3}}...{\ltiapp{\text{f}}{3.0}}...}
\]
%
the first call to \text{f} tags the inner function as {\ltiufunelab{\ltiClosureID{1}}{\text{y}}{\text{x}}}
with {\ltiClosureID{1}} standing for the set of closures
whose runtime environments bind \text{x} to a value of type \text{Int},
and the second call tags it as {\ltiufunelab{\ltiClosureID{2}}{\text{y}}{\text{x}}}
with {\ltiClosureID{2}} standing for the set of closures
who similarly bind \text{x} to a value of type \text{Real}.


% https://dl.acm.org/citation.cfm?id=201001
% Closure analysis in constraint form -	Jens Palsberg

Giannini and Rocca~\cite{giannini1988characterization}
provide the following strongly-normalizing term is not typable in System F,
which we write in Clojure and refer to as \GRterm.

%separate to preserve spacing
\begin{cljlisting}
(let [I (fn [a] a)
      K (fn [b] (fn [c] b))
      D (fn [d] (d d))]
  ((fn [x] (fn [y] ((y (x I))
                    (x K))))
   D))
\end{cljlisting}
%separate to preserve spacing

Palsberg~\cite{Palsberg:1995:CAC:200994.201001}
uses \GRterm to motivate program analyses
that answer basic questions like:
\begin{itemize}
  \item For every application point, which abstractions can be applied?
  \item For every abstraction, to which arguments can it be applied?
\end{itemize}
Symbolic closures answer neither of these questions.
Instead, they provide answers relevant to checking and inferring types:
\begin{itemize}
  \item Can \GRterm accept an argument of type \ltiT{}?
  \item When given an argument of type \ltiT{}, what type is the value returned by \GRterm?
  \item Does \GRterm inhabit \ltiFn{\ltiT{}}{\ltiS{}}?
\end{itemize}

To illustrate, we turn to
our preliminary implementation of symbolic closures~\footnote{https://github.com/frenchy64/lti-model}
to explore \GRterm.
It exposes the type checking query \clj{(tc p e)},
which returns the type of checking \clj{e} at expected prototype \clj{p},
where a prototype is a type that can contain ``wildcards'' \clj{?}.
Now we can query the type of \GRterm as if it had a type.
The caveat: without a rich enough prototype,
a benign symbolic closure type may be provided as an answer---you only get out what you put in
(for this reason, symbolic closures perform particularly well when top-level types are always provided).

For example, \clj{(tc ? GR)} asks to synthesize a type for \GRterm.
Unsurprising, a symbolic closure type greets us (below).

% full output
%(Closure
% {I (Closure {} (fn [a] a)),
%  K (Closure {I (Closure {} (fn [a] a))} (fn [b] (fn [c] b))),
%  D
%  (Closure
%   {I (Closure {} (fn [a] a)),
%    K (Closure {I (Closure {} (fn [a] a))} (fn [b] (fn [c] b)))}
%   (fn [d] (d d))),
%  x
%  (Closure
%   {I (Closure {} (fn [a] a)),
%    K (Closure {I (Closure {} (fn [a] a))} (fn [b] (fn [c] b)))}
%   (fn [d] (d d)))}
% (fn [y] ((y (x I)) (x K))))

%abbreviated
%(Closure
% {I (Closure {} I),
%  K (Closure {I (Closure {} I)} K),
%  D (Closure {I (Closure {} I), K (Closure {I (Closure {} I)} K)} D),
%  x (Closure {I (Closure {} I), K (Closure {I (Closure {} I)} K)} D)}
% (fn [y] ((y (x I)) (x K))))

% where Ic = (Closure {} I)
%       Kc = (Closure {I Ic} K)
%       Dc = (Closure {I Ic, K Kc} D)
%(Closure
% {I Ic,
%  K Kc,
%  D Dc,
%  x Dc}
% (fn [y] ((y (x I)) (x K))))
{
\[
\begin{array}{lll}
\ltiClosure{\ltiEnv{}}{\text{\clj{(fn [y] ((y (x I)) (x K)))}}},
              \text{ where }&
{\ltiEnv{}} =  \ltiEnvConcat{\hastype{\text{\clj{I}}}{{\text{\clj{I}}}_c}}
              {\ltiEnvConcat{\hastype{\text{\clj{K}}}{{\text{\clj{K}}}_c}}
              {\ltiEnvConcat{\hastype{\text{\clj{D}}}{{\text{\clj{D}}}_c}}
              {\ltiEnvConcat{\hastype{\text{\clj{x}}}{{\text{\clj{D}}}_c}}}}}\\&
{\GRclosuretag{I}} = \ltiClosure{\ltiEnv{I}}{\text{\clj{I}}}\\&
%{\GRclosuretag{K}} = \ltiClosure{\hastype{\text{\clj{I}}}{{\text{\clj{I}}}_c}}{\text{\clj{K}}}\\&
{\GRclosuretag{K}} = \ltiClosure{\ltiEnv{k}}{\text{\clj{K}}}\\&
%{\GRclosuretag{D}} = \ltiClosure{\ltiEnvConcat{\hastype{\text{\clj{I}}}{{\text{\clj{I}}}_c}}{\hastype{\text{\clj{K}}}{{\text{\clj{K}}}_c}}}
%                                  {\text{\clj{D}}}
{\GRclosuretag{D}} = \ltiClosure{\ltiEnv{D}}{\text{\clj{D}}}\\&
\end{array}
\]
}

The term of \GRclosure is the (call-by-value) normal form of \GRterm, 
derived by applying the symbolic closure of \clj{(fn [x] ...)} to  \clj{D}.
The type environment
\ltiEnv{} captures the type environment at the point the \clj{(fn [y] ...)} term
was type checked.
There, the bindings \clj{I}, \clj{K}, and \clj{D} are all symbolic closure types
{\GRclosuretag{I}},
{\GRclosuretag{K}}, and
{\GRclosuretag{D}}, respectively,
with \clj{x} also having type {\GRclosuretag{D}}
as a result of the application.
%Due to Clojure's left-to-right semantics for \clj{let} bindings, \clj{I} is also bound by
%{\ltiEnv{k}} and {\ltiEnv{D}}, and 
%\clj{K} by {\ltiEnv{D}}.

As explained in \secref{symbolic:section:formal-model}, two ways
to query a symbolic closure are by applying it or using subtyping.
We can experimentally discover what shape of argument \GRterm accepts 
by querying it at different prototypes, and using error messages and
visual inspections of \GRterm and its normal form (calculated by \GRclosure) as guidance.
We started with the query \clj{(tc [Any -> ?] GR)}, which
gives the error message \clj{Cannot invoke Any}.
Then we inspected \GRclosure, and noticed \clj{y}
must have shape \clj{[? -> [? -> ?]]} based on its usage.
Incorporating that information results in our first interesting query result.

\begin{cljlisting}
(tc [[? -> [? -> ?]] -> ?]
    GR)
;=> [[Any -> [Any -> Nothing]] -> Nothing]
\end{cljlisting}

This was calculated by observing a result type of \clj{Nothing}
from the application of \GRclosure to an argument of type \clj{[Any -> [Any -> Nothing]]}
(derived by minimizing/maximising wildcards 
in covariant/contravariant positions, respectively, with respect to the relevant part of the prototype).
We can find other interesting types \GRclosure inhabits by varying the query.

\begin{cljlisting}
(tc [[? -> [? -> Int]] -> ?]
    GR)
;=> [[Any -> [Any -> Int]] -> Int]

(tc (All [a] [[? -> [? -> a]] -> ?])
    GR)
;=> (All [a] [[Any -> [Any -> a]] -> a])
\end{cljlisting}

With the last query, we stumble on how to use \GRterm as a glorified identity function.
We can verify this by evaluating a few terms.

\begin{cljlisting}
(GR (fn [_] (fn [_] 42))) ;=> 42
(GR (fn [_] (fn [_] 24))) ;=> 24
\end{cljlisting}

We can also synthesize the types of these calls.

\begin{cljlisting}
(tc ? (GR (fn [_] (fn [_] 42))))
;=> Int
\end{cljlisting}

The original use case of symbolic closures is to type check
top-level functions against provided types,
but whose bodies are too difficult to type check with traditional means.
The following (extreme) example shows how checking the definition
of a simple identity function can be thwarted, and how symbolic closures
can make checking the definition of functions much more flexible.

\begin{cljlisting}
(tc (All [a] [a -> a])
    (fn [z]
      (GR (fn [_] (fn [_] z)))))
;=> (All [a] [a -> a])
\end{cljlisting}

This illustrates the promise of symbolic closures to treat previously untypable terms
as ``black-boxes'' during type checking, especially in a setting where top-level
type information is always provided.

%http://web.cs.ucla.edu/~palsberg/tba/papers/banerjee-icfp97.pdf
% A modular, polyvariant and type-based closure analysis - Banerjee
Banerjee~\cite{banerjee1997modular}
achieves a similar effect by
instrumenting the rank 2 fragment of the intersection type discipline
with flow information of closure values.
Function terms are labelled and
arrow types are annotated with sets of labels which denote
which functions it may represent.
They demonstrate by analyzing the following term
$(\l f. (\l x. f I)(f 0))I$
where $I$ represents the identity function.
They label the term
$(\l^1 f. (\l^2 x. f (\l^3 u. u))(f 0))(\l^4 v. v)$
and infer overall type
$t \xrightarrow[]{\{3\}} t$,
which says values of this type originate from the lambda labeled $3$,
with fresh type variable $t$.

Their system inherits the principal typing property of intersection
types, which we lack in Typed Clojure.
To compensate for this, our prototype for unrestricted symbolic closures
types returns the full code and type environment for the corresponding closure
of lambda $3$, so it may
be further checked later when more type information is available.
For example, plugging this example into our prototype gives the
following symbolic closure type (using their labelled lambda syntax)
$\ltiClosure{\ltiEnv{}}{\l^3 u. u}$,
where $\ltiEnv{} = \{f : \ltiClosure{\{\}}{\l^4 v. v}, x : \text{Int}\}$.

%\begin{cljlisting}
%(tc ? ((fn [f] ((fn [x] (f (fn [u] u))) (f 0))) (fn [v] v)))
%\end{cljlisting}


\paragraph{Hindley-Milner and Let-polymorphism}
%TODO
%\input{hm-comparison}
Kanellakis and Mitchell~\cite{kanellakis1989polymorphic}
provide a set of (pathological) ML programs that exhibit exponential
growth in the size of their principal type schemes.
%Later, Mairson~\cite{Mairson:1989} confirmed that
%the problem of ML type-checking is \textsc{DExpTime}-complete.
We use their benchmarks to compare symbolic closures with
global type inference in the style of Milner~\cite{milner1978theory}.

Example 3.1 of~\cite{kanellakis1989polymorphic} uses a lambda-encoding of pairs to create an ML principal type
which appears to grow exponentially in length, however has a linear time representation as a directed acyclic graph.
It is designed to avoid ML's let-construct to remove the influence of let-polymorphism.
The idea behind the program is to duplicate types \ltiS{} by placing them in (lambda-encoded) tuples,
following the pattern \ltiS{}, $\langle\ltiS{},\ltiS{}\rangle$, $\langle\langle\ltiS{},\ltiS{}\rangle,\langle\ltiS{},\ltiS{}\rangle\rangle$,
and so on.
To compare with symbolic closures,
let $P$ stand for \clj{(fn [x] (fn [z] (z x x)))}
and $P_z$ for \clj{(fn [z] (z x x))} in the following, where the left
hand side term has the right hand side type.
We can see the size of the type grows linearly in the number of occurrences of $P$---specifically,
the outermost symbolic closure's \emph{environment} increases in size.

{
\[
\begin{array}{c@{}c@{}c@{}c@{}ccr@{}r@{}r@{}r@{}r@{}r@{}r@{}r@{}r}
  &&{(P\ 1)}&&        & : &    \ltiClosure{\hastype{\text{x}}{&&&\text{Int}&&}&}{P_z}\\
  &{(P\ } &{(P\ 1)}& {)}&    & : &   \ltiClosure{\hastype{\text{x}}
                                                 {&&(\ltiClosure{\hastype{\text{x}}{&\text{Int}&}}{{P_z}})&&}}
                                                {P_z}\\
{(P\ }&{(P\ }&{(P\ 1)}&{)}&{)}    & : &   \ltiClosure{\hastype{\text{x}}
                                                      {&(\ltiClosure{\hastype{\text{x}}{&(\ltiClosure{\hastype{\text{x}}{&\text{Int}&}}{P_z})&}}{P_z})}&}
                                                     {P_z}
\end{array}
\]
}

Example 3.4 of~\cite{kanellakis1989polymorphic}
exhibits exponential growth in the size of ML principal types by exploiting
let-polymorphism's ability to copy type variables and assign new names to them.
It follows the following pattern, which is similar to the previous benchmark, except
intermediate values are let-bound.

\begin{minipage}[t]{0.3\linewidth}
\begin{cljlisting}
(let [x0 1
      x1 (P x0)]
  x1)
\end{cljlisting}
\end{minipage}
%
\begin{minipage}[t]{0.3\linewidth}
\begin{cljlisting}
(let [x0 1
      x1 (P x0)
      x2 (P x1)]
  x2)
\end{cljlisting}
\end{minipage}
%
\begin{minipage}[t]{0.3\linewidth}
\begin{cljlisting}
(let [x0 1
      x1 (P x0)
      x2 (P x1)
      x3 (P x2)]
  x3)
\end{cljlisting}
\end{minipage}

Unlike ML, we find that symbolic closures are rather sensitive to whether $P$
is let-bound or copied.
If let-bound at the top of each term, the types of each term are
identical to the previous example, and so grow linearly in size.
This is because the resulting type of each term is a symbolic closure of the
$P_z$ term occuring in $P$, whose definition type environment
never increases to include new variables (in particular, \clj{x1}, \clj{x2}, and \clj{x3}
are never in-scope there).
If $P$ is copied, however, the number of symbolic closures types reachable from
the innermost occurrence of $P$ grows exponentially, and so the resulting type also
grows exponentially.
We can reduce this to linear growth with sharing as below, where \clj{xi} has type \clj{Pci},
because the exponential growth happens by duplicating symbolic closure types.
Each $P_z$ term comes from a different copy of $P$, in particular
the $P_z$ term of symbolic closure type \clj{Pci} originates from the $P$ occurring on the right-hand-side of \clj{xi}.

\begin{cljlisting}
Pc1 = {x0 Int,                         x Int}@Pz
Pc2 = {x0 Int, x1 Pc1,                 x Pc1}@Pz
Pc3 = {x0 Int, x1 Pc1, x2 Pc2,         x Pc2}@Pz
Pc4 = {x0 Int, x1 Pc1, x2 Pc2, x3 Pc3, x Pc3}@Pz
\end{cljlisting}

Example 3.5 of~\cite{kanellakis1989polymorphic} gives a series of terms whose ML principal
type is doubly-exponential in the size of the term,
reduced to exponential when converted to a directed acyclic graph.
The pattern is below, which, for $i>1$ and $j=i-1$, binds \clj{xi} to \clj{(fn [y] (xj (xj y)))}.

\begin{minipage}[t]{0.31\linewidth}
\begin{cljlisting}
(let [x1 (fn [x] (P x))
      x2 (fn [y]
           (x1 (x1 y)))]
  (x2 1))
\end{cljlisting}
\end{minipage}
%
\begin{minipage}[t]{0.31\linewidth}
\begin{cljlisting}
(let [x1 (fn [x] (P x))
      x2 (fn [y]
           (x1 (x1 y)))
      x3 (fn [y]
           (x2 (x2 y)))]
  (x3 1))
\end{cljlisting}
\end{minipage}
%
\begin{minipage}[t]{0.31\linewidth}
\begin{cljlisting}
(let [x1 (fn [x] (P x))
      x2 (fn [y]
           (x1 (x1 y)))
      x3 (fn [y]
           (x2 (x2 y)))
      x4 (fn [y]
           (x3 (x3 y)))]
  (x4 1))
\end{cljlisting}
\end{minipage}

The symbolic closure type for these terms grows exponentially in size, again because 
the number of reachable closures grows exponentially. However, each symbolic closure
is distinct, so no sharing is possible. The term ending in \clj{xi}
is given type \clj{Pci}, below.

\begin{cljlisting}
Pc1 = {x Int}                                      @Pz
Pc2 = {x Pc1}                                      @Pz
Pc3 = {x {x Pc2}@Pz}                               @Pz
Pc4 = {x {x {x {x {x Pc3}@Pz}@Pz}@Pz}@Pz}          @Pz
Pc5 = {x {x {x {x {x {x {x {x {x {x {x {x Pc4}
       @Pz}@Pz}@Pz}@Pz}@Pz}@Pz}@Pz}@Pz}@Pz}@Pz}@Pz}@Pz
                                        
\end{cljlisting}

% https://link.springer.com/content/pdf/10.1007%2FBFb0032745.pdf
% Type-Directed Flow Analysis for Typed Intermediate Languages - Suresh Jagannathan, Stephen Week~s, and Andrew Wright 
% - seems like their "abstract-closures" probably relate to symbolic closures
% - they "exploit types to control a flow analysis algorithm"
\paragraph{Flow analysis for typed languages}
Jagannathan, Weeks and Wright~\cite{jagannathan1997type}
give a flow analysis for a typed intermediate language.
Their ``abstract closures'' resemble our symbolic closures,
and, like ours, their algorithm is not guaranteed to terminate.
Our work explicitly integrates symbolic closures as a new type
in the language and therefore assists in type inference,
whereas their main result is a separate flow analysis that 
exploit types to increase flow accuracy.

% Faithful Translations between Polyvariant Flows and Polymorphic Types
% http://people.cs.ksu.edu/~tamtoft/Papers/Amt+Tur:FTPFPT-2000/short.pdf
% Torben Amtoft and Franklyn Turbak

% Gilray's thesis
% pg 44 has a long list of references to check
% https://thomas.gilray.org/pdf/thesis-gilray.pdf

%TODO
%\paragraph{Directional Polymorphism}

% MLsub
% https://www.cl.cam.ac.uk/~sd601/thesis.pdf
% https://www.cl.cam.ac.uk/~sd601/papers/mlsub-preprint.pdf
%
% Polar type system (Jim)
% http://citeseerx.ist.psu.edu/viewdoc/download?doi=10.1.1.123.8718&rep=rep1&type=pdf

% Pottier
% Simplifying Subtyping Constraints: A Theory
% http://citeseerx.ist.psu.edu/viewdoc/download?doi=10.1.1.41.7032&rep=rep1&type=pdf
% A Framework for Type Inference with Subtyping%
% http://citeseerx.ist.psu.edu/viewdoc/download?doi=10.1.1.55.2364&rep=rep1&type=pdf

% ML_F
% http://gallium.inria.fr/~remy/mlf/mlf-type-inference-long.pdf


%TODO
%Xie and Oliveira~\cite{xie2018let} present a type system where
%argument type information flows to the function position in applications.
%Then, defining `let` as sugar propagates enough information to avoid
%a custom rule for `let`.
%No information is propagated from functions to applications, so the benefits
%of Colored Local Type Inference are negated.

\chapter{Future work}

The most pressing future work for Typed Clojure is part of the ongoing work
presented after \partref{part:types}.

\section{Future work for Automatic Annotations}
A larger scale investigation of Clojure usage patterns is now possible by
repurposing the automatic annotation tool described in \partref{part:autoann}
to generate and enforce clojure.spec annotations.
As well as testing the robustness of the tool's design, 
the resulting data would be
useful in investigating general questions like how effectively Clojure users utilize unit and generative testing,
how Clojure code evolves of code over time, and the prevalence of idioms that Typed Clojure and clojure.spec
have (and have not) been designed around.

\section{Future work for Extensible Types}
\partref{part:implementations} outlines a code analyzer that paves the way to a future implementation
of extensible typing rules for Typed Clojure.
The next steps in this direction involve deciding the user interface for such a system
and performing a survey of commonly used macros to determine which features must be supported.

\section{Future work for Symbolic Closures}
Symbolic closures (\partref{part:symbolic-closures})
show much promise in improving the user-experience of Typed Clojure.
However, our preliminary work is still not well understood.
\chapref{chapter:symbolic:metatheory} outlines several conjectures we hope to first prove.
Finally, the problem of integrating symbolic closures with type argument synthesis
is a crucial piece of future work, that (we hope) will prove symbolic closures
as indispensable in checking many common Clojure problems.

% Possible future work on Higher-rank types
% - see https://www.microsoft.com/en-us/research/publication/practical-type-inference-for-arbitrary-rank-types/
%   - looks like the journal version of boxy types?
%   - some notes from the paper
%     - a predicative type system only allows a polytype to be instantiated with monotypes
%     - ML_F is both impredicative and supports type inference (but costly to implement & formalize)
%       - also infers principal types
%     - higher-kinded types are orthogonal to higher-rank types, and Haskell's implementation of the former
%       happen to work well with higher-rank types (but no explanation)
%     - the concept of "syntax-directed" rules is given lots of a explanation
%     - \vdash^inst compares two _polytypes_
%     - Kfoury and Wells 1994 show that typeability of System F (with completely erased annotations) is decidable for rank 2 
%       but undecidable for rank 3>=
%     - LTI == "partial" type inference
%       - in the sense that it's not-complete (can't check all programs)
%     - nice discussion of partial type inference


\printbibliography


%\counterwithin{figure}{section}
\counterwithin{assumption}{section}
\counterwithin{theorem}{section}
\counterwithin{lemma}{section}
\counterwithin{definition}{section}

% instead of writing a wrapper for \part* (see thesis-format.sty),
% I'll just comment this out
%\part*{Appendices}

\appendix

\chapter{Full rules for \lambdatc{}}

\begin{figure}[h]
$$
\begin{altgrammar}
  \expd{}, \e{} &::=& \x{}
                      \alt \val{} 
                      \alt {\comb {\e{}} {\e{}}} 
                      \alt {\abs {\x{}} {\ty{}} {\e{}}}
                      \alt {\ifexp {\e{}} {\e{}} {\e{}}}
                      \alt {\doexp {\e{}} {\e{}}}
                      \\
                      &\alt& {\letexp {\x{}} {\e{}} {\e{}}}
                      \alt {\wrongorerror{}}
                      \alt {\ReflectiveExp{}}
                      \alt {\NonReflectiveExp{}}
                      \alt {\MultimethodExp{}}
                      %\alt {\HintedExp{}}
                      \alt {\HashMapExp{}}
                &\mbox{Expressions} \\
  \val{} &::=&          \singletonmeta{}
                      \alt \classvaluemeta{}
                      \alt {\emptymap{}}
                      \alt {\const{}}
                      \alt {\num{}}
                      \alt {\str{}}
                      \alt \mapval{}
                      \alt {\closure {\openv{}} {\abs {\x{}} {\ty{}} {\e{}}}}
                      \alt {\multi {\val{}} {\disptable{}}}
                &\mbox{Values} \\
  \mapval{} &::=&  {\curlymapvaloverright{\val{}}{\val{}}}
                &\mbox{Map Values} \\
                \constantssyntax{}\\
%  \HintedExp{}             &::=& \typehintedexpsyntax{}
%                &\mbox{Type Hinted Expressions} \\
  \HashMapExp{}                &::=& \hmapexpressionsyntax{}
                &\mbox{Hash Maps} \\
  \NonReflectiveExp{}     &::=& \nonreflectiveexpsyntax{}
                &\mbox{Non-Reflective %Java
                       Interop} \\
  \ReflectiveExp{}     &::=& \reflectiveexpsyntax{}
                &\mbox{Reflective %Java
                       Interop} \\
  \MultimethodExp{}     &::=& \multimethodexpsyntax{}
                &\mbox{%Immutable First-Class 
                Multimethods}
                      \\%\\ save space...
  \s{}, \ty{}    &::=& \Top 
                      \alt \class{}
                      \alt {\Value \singletonmeta{}} 
                      \alt {\Unionsplice {\overrightarrow{\ty{}}}}
                      \alt
                      {\ArrowOne {\x{}} {\ty{}}
                                   {\ty{}}
                                   {\filterset {\prop{}} {\prop{}}}
                                   {\object{}}}
                      \\
                      &\alt& {\HMapgeneric {\mandatory{}} {\absent{}}}
                      \alt {\MultiFntype{\ty{}}{\ty{}}}
                      
                &\mbox{Types} \\
                \auxhmapsyntax{}\\
  \singletonallsyntax{}
                \\% \\ save space...
                \openvsyntax{}\\\\

                %\tatypesyntax{}\\\\

  \occurrencetypingsyntax{}
  %\pathelemsyntax{}\\
  \propenvsyntax{}
  \\\\

 \disptablesyntax{} \\
 %\typehintenvsyntax{} \\
\classtableallsyntax{} \\
               \classliteralallsyntax{}\\

               \classvaluesyntaxentry{}\\
                      \\
  \wrongorerror{} &::=& \wrong{} \alt \errorvalv{}
                &\mbox{Wrong or error}
                      \\
  \definedreduction{} &::=& \val{} \alt \wrongorerror{}
                 &\mbox{Defined reductions}
                 \\
  \polaritymeta{} &::=& \pluspolarityliteral \alt \minuspolarityliteral
                 &\mbox{Substitution Polarity}
\end{altgrammar}
$$
\caption{Syntax of Terms, Types, Propositions, and Objects}
\end{figure}

\begin{figure}
$$
\begin{array}{lllr}
  \Nil &\equiv& {\ValueNil}\\
  \True &\equiv& {\ValueTrue}\\
  \False &\equiv& {\ValueFalse}\\
\end{array}
$$
\caption{Type abbreviations}
\end{figure}

\begin{figure}
$$
\begin{array}{lllr}
  \judgementtwo{\propenv{}}{\e{}}{\ty{}} &\equiv& 
  \judgement{\propenv{}}{\e{}}{\ty{}}{\filterset{\thenprop{\prop{}}}{\elseprop{\prop{}}}}{\object{}}
  & \text{for some}\ {\thenprop{\prop{}}}, {\elseprop{\prop{}}} \text{and}\ {\object{}}

  \\
  {\replacefor{\ty{}}{\object{}}{\x{}}} &\equiv& {\pluspolarity{\replacefor{\ty{}}{\object{}}{\x{}}}}
  \\
  {\replacefor{\prop{}}{\object{}}{\x{}}} &\equiv&  {\pluspolarity{\replacefor{\prop{}}{\object{}}{\x{}}}}
  \\
  {\replacefor{\filterset{\prop{}}{\prop{}}}{\object{}}{\x{}}} &\equiv&  {\pluspolarity{\replacefor{\filterset{\prop{}}{\prop{}}}{\object{}}{\x{}}}}
  \\
  {\replacefor{\object{}}{\object{}}{\x{}}} &\equiv& {\pluspolarity{\replacefor{\object{}}{\object{}}{\x{}}}}

\end{array}
$$
\caption{Judgment abbreviations}
\end{figure}


\begin{figure*}
\begin{mathpar}

  {\TLocal}

{\TConst}

{\TTrue}

{\TFalse}

{\TNil}

{\TNum}

{\TDo}

{\TIf}

{\TLet}
                 
{\TApp}

{\TAbs}

\infer [T-Clos]
{ \exists {\propenv{}}. \satisfies{\openv{}}{\propenv{}}
  \ \text{and}\ 
\judgementrewrite {\propenv{}} {\abs {\x{}} {\ty{}} {\e{}}} {\s{}}
                 {\filterset {\thenprop {\prop{}}}
                             {\elseprop {\prop{}}}}
                 {\object{}}
                 {\abs {\x{}} {\ty{}} {\ep{}}}
              }
{ \judgementrewrite {}
            {\closure {\openv{}} {\abs {\x{}} {\ty{}} {\e{}}}} 
                      {\s{}}
             {\filterset {\thenprop {\prop{}}}
                         {\elseprop {\prop{}}}}
             {\object{}}
            {\closure {\openv{}} {\abs {\x{}} {\ty{}} {\ep{}}}}
          }

           {\TError}

         {\TSubsume}
\end{mathpar}
\caption{Standard Typing Rules}
\end{figure*}


\begin{figure*}
\begin{mathpar}

{\TNew}

{\TNewStatic}

{\TField}

{\TFieldStatic}

{\TMethod}

{\TMethodStatic}

{\TClass}

{\TInstance}
\end{mathpar}
\caption{Java Interop Typing Rules}
\end{figure*}

\begin{figure*}
\begin{mathpar}

  \TDefMulti{}

  \TDefMethod{}

\TIsA{}

\infer [T-Multi]
{ \judgementtworewrite {} {\val{}} {\ty{}} {\vp{}}
  \\
  \overrightarrow{\judgementtworewrite{}{\val{k}}{\Top}{\vp{k}}}
  \\
  \overrightarrow{\judgementtworewrite{}{\val{v}}{\s{}}{\vp{v}}}
}
{ \judgementrewrite {}
  {\multi {\val{}} {\curlymapvaloverright{\val{k}}{\val{v}}}}
                      {\MultiFntype {\s{}} {\ty{}}}
             {\filterset {\topprop{}} {\botprop{}}}
           {\emptyobject{}}
  {\multi {\vp{}} {\curlymapvaloverright{\vp{k}}{\vp{v}}}}
}

\end{mathpar}
\caption{Multimethod Typing Rules}
\end{figure*}

\begin{figure*}
\begin{mathpar}

%\infer [T-Get]
%{ \judgementtwo {\propenv{}} {\hastype {\e{m}} {\Map {\ty{k}}{\ty{v}}}}
%  \\
%  \judgementtwo {\propenv{}} {\hastype {\e{k}} {\Top}}}
%{ \judgement {\propenv{}} {\hastype {\getexp {\e{m}} {\e{k}}} {\Union{\ty{v}}{\nil{}}}}
%             {\filterset {\topprop{}} {\topprop{}}}
%           {\emptyobject{}}}
%
%\infer [T-Assoc]
%{ 
%  \judgementtwo {\propenv{}} {\hastype {\e{m}} {\Map{\ty{k}}{\ty{v}}}}
%  \\
%  \judgementtwo {\propenv{}} {\hastype {\e{k}} {\ty{k}}}
%  \\
%  \judgementtwo {\propenv{}} {\hastype {\e{v}} {\ty{v}}}
%}
%{ \judgement {\propenv{}} 
%             {\hastype {\assocexp {\e{m}} {\e{k}} {\e{v}}} {\Map {\ty{k}}{\ty{v}}}}
%             {\filterset {\topprop{}} {\botprop{}}}
%             {\emptyobject{}}
%}

\infer [T-HMap]
{ \overrightarrow{\judgementtworewrite {} {\val{k}}{\Value \kw{}}{\vp{k}}}\\
  \overrightarrow{\judgementtworewrite {} {\val{v}}{\ty{v}}{\vp{v}}}\\
  \mandatory{} = \mandatorysetoverright{\kw{}}{\ty{v}}
}
{ \judgementrewrite {}
             {\curlymapvaloverright{\val{k}}{\val{v}}}
                       {\HMapc {\mandatory{}}}
             {\filterset {\topprop{}} {\botprop{}}}
             {\emptyobject{}}
             {\curlymapvaloverright{\vp{k}}{\vp{v}}}
           }

    {\TKw}

    {\TGetHMap}

    {\TGetAbsent}

    {\TGetHMapPartialDefault}

    {\TAssoc}

\end{mathpar}
\caption{Map Typing Rules}
\end{figure*}


%\input{tools-analyzer-figure}

\begin{figure*}
\begin{mathpar}
\objectsub{}

\standardsubtyping{}
\SPMultiFn{}
\Multisubtyping{}

\HMapsubtyping{}
\end{mathpar}
\caption{Subtyping rules}
\end{figure*}


%$$
%\begin{tdisplay}{Evaluation Contexts}
%  \begin{altgrammar}
%    \E{} &::=& [ ] % application rules
%              \alt (\c{}\ \overrightarrow{\val{}}\ \E{}\ \overrightarrow{\exp{}}) % eval arguments left-to-right
%              % map rules
%              \alt \{\overrightarrow{\val{}\ \val{}}\ \E{}\ \exp{}\ \overrightarrow{\exp{}\ \exp{}} \} % key first
%              \alt \{\overrightarrow{\val{}\ \val{}}\ \val{}\ \E{}\ \overrightarrow{\exp{}\ \exp{}} \}   % value next
%              &\mbox{Evaluation Contexts}
%  \end{altgrammar}
%\end{tdisplay}
%$$ 

%{\classtablelookupfigure}

{\convertjavatypefigure{figure*}{}}

\begin{figure}
\begin{mathpar}
\constanttypefigure{}
\end{mathpar}
\caption{Constant Typing}
\end{figure}

\constantsemfigure{appendix}


\begin{figure*}
\isapropsfigure{}

\isaopsemfigure{}
\caption{Definition of isa?}
\end{figure*}

%\begin{figure*}
%$$
%\begin{array}{llrr}
%  \isacompare{\HVec{\overrightarrow{{\ty{}};{\prop{}};{\object{}}}^i}}
%             {\object{}}
%             {\HVec{\overrightarrow{{\ty{}};{\prop{}};{\object{}}}^j}}
%             {\replacefor
%              {\filtersetparen
%                {\isprop {\HVec{\overrightarrow{\isacomparethree{\ty{i}}{\object{i}}{\ty{j}}}}}{\x{}}}
%                {\notprop{\HVec{\overrightarrow{\isacomparethree{\ty{i}}{\object{i}}{\ty{j}}}}}{\x{}}}}
%              {\object{}}
%              {\x{}}}
%              & i = j
%\end{array}
%$$
%$$
%\begin{array}{lclr}
%  \isaopsem{\rtvector{\overrightarrow{\x{}}^i}}{\rtvector{\overrightarrow{\x{}}^j}} &=& {\true{}}
%                                                                                    & i = j, \overrightarrow{\isaopsem{\x{i}}{\x{j}} = {\true{}}}^{i,j}
%  \\
%\end{array}
%$$
%\caption{isa? Vector Extensions}
%\end{figure*}

\begin{figure*}
  \getmethodfigure{}
\caption{Definition of get-method}
\end{figure*}


%\clearpage

\begin{figure*}
\begin{mathpar}

\BLocal{}

\BDo{}

\BLet{}

\BVal{}

\BIfTrue{}

\BIfFalse{}

\BAbs{}

\BBetaClosure{}

\BDelta{}

\BBetaMulti{}

\BField{}

\BMethod{}

\BNew{}

       \BDefMulti{}

       \BDefMethod{}

       \BIsA{}

       {\BAssoc}

       {\BGet}

       {\BGetMissing}
\end{mathpar}
\caption{Operational Semantics}
\label{appendix:figure:opsem}
\end{figure*}

\begin{figure*}
\begin{mathpar}

\infer [BS-MethodRefl]
{}
{\opsem {\openv{}} {\methodexp {mth} {\e{}} {\overrightarrow{\e{}}}}
        {\wrong{}}}

\infer [BS-FieldRefl]
{}
{\opsem {\openv{}} {\fieldexp {\fld{}} {\e{}}}
        {\wrong{}}}

\infer [BS-NewRefl]
{}
{\opsem {\openv{}} {\fieldexp {\fld{}} {\e{}}}
        {\wrong{}}}


\infer [BS-Beta]
{ \opsem {\openv{}}
         {\e{f}}
         {\val{}}
         \\\\
  {\val{}} \not= {\const{}}
  \\
  {\val{}} \not= {\multi {\val{d}} {\disptable{}}}
  \\\\
  {\val{}} \not= {\closure {\openv{c}} {\abs {\x{}} {\ty{}} {\e{b}}}}
       }
{ \opsem {\openv{}}
         {\appexp {\e{f}} {\e{a}}}
         {\wrong{}}
       }

\infer [BS-BetaMulti]
{ \opsem {\openv{}}
         {\e{f}}
         {\multi {\val{}} {\disptable{}}}
         \\\\
  {\val{}} \not= {\const{}}
  \\
  {\val{}} \not= {\multi {\val{d}} {\disptable{}}}
  \\\\
  {\val{}} \not= {\closure {\openv{c}} {\abs {\x{}} {\ty{}} {\e{b}}}}
       }
{ \opsem {\openv{}}
         {\appexp {\e{f}} {\e{a}}}
         {\wrong{}}
       }

\infer [BS-FieldTarget]
{ \opsem {\openv{}}
         {\e{}} 
       {\val{1}}
         \\\\
         {\val{}} \not= {\classvalue{\classhint{1}} {\overrightarrow {\classfieldpair{\fld{i}} {\val{i}}}}}
       }
{ \opsem {\openv{}}
         {\fieldstaticexp {\classhint{1}} {\classhint{2}} {\fld{}} {\e{}}}
         {\wrong{}}
   }

\infer [BS-FieldMissing]
{ \opsem {\openv{}}
         {\e{}} 
       {\classvalue{\classhint{1}} {\overrightarrow {\classfieldpair{\fld{i}} {\val{i}}}}}
       \\
       \fld{} \not\in \{\overrightarrow{\fld{i}}\}
       }
{ \opsem {\openv{}}
         {\fieldstaticexp {\classhint{1}} {\classhint{2}} {\fld{}} {\e{}}}
         {\wrong{}}
   }


\infer [BS-MethodTarget]
{ \opsem {\openv{}}
         {\e{m}}
         {\val{}}
  \\
         {\val{}} \not= {\classvalue{\classhint{1}} {\overrightarrow {\classfieldpair{\fld{i}} {\val{i}}}}}
}
{\opsem {\openv{}}
        {\methodstaticexp {\classhint{1}} {\overrightarrow{\classhint{a}}} {\classhint{2}} {mth} {\e{m}} {\overrightarrow{\e{a}}}}
        {\wrong{}}
      }

\infer [BS-MethodArity]
{ i \not= a
}
{\opsem {\openv{}}
        {\methodstaticexp {\classhint{1}} {\overrightarrow{\classhint{i}}} {\classhint{2}} {mth} {\e{m}} {\overrightarrow{\e{a}}}}
        {\wrong{}}
      }

\infer [BS-MethodArg]
{ \opsem {\openv{}}
         {\e{m}}
         {\val{m}}
  \\
  \overrightarrow{
  \opsem {\openv{}}
         {\e{a}}
         {\val{a}}
       }
       \\\\
  \exists a.\ 
    \val{a} \not=\ {\classvalue{\classhint{a}} {\overrightarrow {\classfieldpair{\fld{i}} {\val{i}}}}}\ or\ \val{a} \not= \nil{}
}
{\opsem {\openv{}}
        {\methodstaticexp {\classhint{1}} {\overrightarrow{\classhint{a}}} {\classhint{2}} {mth} {\e{m}} {\overrightarrow{\e{a}}}}
        {\wrong{}}
      }

\infer [BS-NewArg]
{ \overrightarrow{
  \opsem {\openv{}}
         {\e{i}}
         {\val{i}}
     }
       \\\\
  \exists i.\ 
    \val{i} \not=\ {\classvalue{\classhint{i}} {\overrightarrow {\classfieldpair{\fld{i}} {\val{i}}}}}\ or\ \val{i} \not= \nil{}
}
{\opsem {\openv{}}
        {\newstaticexp {\overrightarrow{\classhint{i}}} {\classhint{1}} 
                       {\class{}} {\overrightarrow{\e{i}}}}
        {\wrong{}}
      }

\infer [BS-NewArity]
{ i \not= a
}
{\opsem {\openv{}}
        {\newstaticexp {\overrightarrow{\classhint{i}}} {\classhint{1}} 
                       {\class{}} {\overrightarrow{\e{a}}}}
        {\wrong{}}
      }

\infer [BS-AssocMap]
{\opsem {\openv{}}
        {\e{m}} {\val{}}
        \\
        \val{} \not= {\curlymap{\overrightarrow{({\val{a}}\ {\val{b}})}}}
}
{
 \opsem {\openv{}}
        {\assocexp {\e{m}} {\e{k}} {\e{v}}} 
        {\wrong{}}
                }

\infer [BS-AssocKey]
{\opsem {\openv{}}
        {\e{m}} {\curlymap{\overrightarrow{({\val{a}}\ {\val{b}})}}}
        \\
 \opsem {\openv{}} {\e{k}} {\val{k}}
 \\\\
 {\val{k}} \not= \kw{}
}
{
 \opsem {\openv{}}
        {\assocexp {\e{m}} {\e{k}} {\e{v}}} 
        {\wrong{}}
                }

\infer [BS-GetMap]
{ \opsem {\openv{}}
         {\e{m}} {\val{}}
        \\
        \val{} \not= {\curlymap{\overrightarrow{({\val{a}}\ {\val{b}})}}}
}
{\opsem {\openv{}}
        {\getexp {\e{m}} {\e{k}}}
        {\wrong{}}
}

\infer [BS-GetKey]
{ \opsem {\openv{}}
         {\e{m}} {\val{}}
        \\
 \opsem {\openv{}}
        {\e{k}} {\val{k}}
        \\\\
      \val{} \not= {\kw{}}
}
{\opsem {\openv{}}
        {\getexp {\e{m}} {\e{k}}}
        {\wrong{}}
}

\infer [BS-Local]
{ \notinopenv {\openv{}} {\x{}}}
{ \opsem {\openv{}} {\x{}} {\wrong{}} }

\infer [BS-DefMethod]
{ \opsem {\openv{}}
         {\e{m}}
         {\val{m}}
         \\
         \val{m} \not= {\multi {\val{d}} {\disptable{}}}
}
{\opsem {\openv{}}
        {\extendmultiexp {\e{m}} {\e{v}} {\e{f}}}
        {\wrong{}}
      }

\end{mathpar}
\caption{Stuck programs}
\end{figure*}

\begin{figure*}
\begin{mathpar}
\infer [BE-ErrorWrong]
{}
{ \opsem {\openv{}} 
         {\wrongorerror{}}
         {\wrongorerror{}}}

\infer [BE-Let]
{ \opsem {\openv{}} {\e{a}} {\wrongorerror{}}
 }
{ \opsem {\openv{}} 
         {\letexp {\x{}} {\e{a}} {\e{}}}
       {\wrongorerror{}}}

\infer [BE-Do1]
{ \opsem {\openv{}} {\e{1}} {\wrongorerror{}} }
{ \opsem {\openv{}} {\doexp{\e{1}}{\e{}}} {\wrongorerror{}}}

\infer [BE-Do2]
{ \opsem {\openv{}} {\e{1}} {\val{1}} 
  \\\\
  \opsem {\openv{}} {\e{}}  {\wrongorerror{}}
}
{ \opsem {\openv{}} {\doexp{\e{1}}{\e{}}} {\wrongorerror{}} }

\infer [BE-If]
{  \opsem {\openv{}} {\e{1}} {\wrongorerror{}}
}
{ \opsem {\openv{}}
         {\ifexp {\e1} {\e2} {\e3}}
         {\wrongorerror{}}
       }

\infer [BE-IfTrue]
{ \opsem {\openv{}} {\e{1}} {\val{1}}
  \\\\
  {\val{1}} \not= {\false{}}
  \\
  {\val{1}} \not= {\nil{}}
  \\\\
  \opsem {\openv{}} {\e{2}} {\wrongorerror{}}
}
{ \opsem {\openv{}}
         {\ifexp {\e1} {\e2} {\e3}}
         {\wrongorerror{}}
       }

\infer [BE-IfFalse]
{  \opsem {\openv{}} {\e{1}} {\false{}}
  \ \ \text{or}\ \ 
  \opsem {\openv{}} {\e{1}} {\nil{}}
  \\\\
  \opsem {\openv{}} {\e{3}} {\wrongorerror{}}
}
{ \opsem {\openv{}}
         {\ifexp {\e1} {\e2} {\e3}}
         {\wrongorerror{}}
       }

\infer [BE-Beta1]
{ \opsem {\openv{}}
         {\e{f}}
         {\wrongorerror{}}
       }
{ \opsem {\openv{}}
         {\appexp {\e{f}} {\e{a}}}
         {\wrongorerror{}}
       }

\infer [BE-Beta2]
{ \opsem {\openv{}}
         {\e{f}}
         {\val{f}}
         \\\\
  \opsem {\openv{}}
         {\e{a}}
         {\wrongorerror{}}
       }
{ \opsem {\openv{}}
         {\appexp {\e{f}} {\e{a}}}
         {\wrongorerror{}}
       }

\infer [BE-BetaClosure]
{ \opsem {\openv{}}
         {\e{f}}
         {\closure {\openv{c}} {\abs {\x{}} {\ty{}} {\e{b}}}}
         \\\\
  \opsem {\openv{}}
         {\e{a}}
         {\val{a}}
         \\\\
  \opsem {\extendopenv {\openv{c}} {\x{}} {\val{a}}}
         {\e{b}}
         {\wrongorerror{}}
       }
{ \opsem {\openv{}}
         {\appexp {\e{f}} {\e{a}}}
         {\wrongorerror{}}
       }

\infer [BE-BetaMulti1]
{ \opsem {\openv{}}
         {\e{f}}
         {\multi {\val{d}} {m}}
         \\\\
  \opsem {\openv{}}
         {\e{a}}
         {\val{a}}
         \\\\
  \opsem {\openv{}}
         {\appexp {\val{d}} {\val{a}}}
         {\wrongorerror{}}
       }
{ \opsem {\openv{}}
         {\appexp {\e{f}} {\e{a}}}
         {\wrongorerror{}}
       }

\infer [BE-BetaMulti2]
{ \opsem {\openv{}}
         {\e{f}}
         {\multi {\val{d}} {m}}
         \\\\
  \opsem {\openv{}}
         {\e{a}}
         {\val{a}}
         \\\\
  \opsem {\openv{}}
         {\appexp {\val{d}} {\val{a}}}
         {\val{e}}
         \\\\
  \getmethoderr {\disptable{}}
             {\val{e}}
             {\errorvalv{}}
       }
{ \opsem {\openv{}}
         {\appexp {\e{f}} {\e{a}}}
         {\errorvalv{}}
       }

\infer [BE-Delta]
{ \opsem {\openv{}} {\e{}} {\const{}}
  \\\\
  \opsem {\openv{}} {\ep{}} {\val{}}
  \\\\
  \constantopsem{\const{}}{\val{}} = \wrongorerror{}
}
{ \opsem {\openv{}}
         {\appexp {\e{}} {\ep{}}}
         {\wrongorerror{}}
       }

\infer [BE-Field]
{ \opsem {\openv{}}
         {\e{}} 
         {\wrongorerror{}}
       }
{ \opsem {\openv{}}
         {\fieldstaticexp {\classhint{1}} {\classhint{2}} {\fld{}} {\e{}}}
         {\wrongorerror{}}
   }

\infer [BE-Method1]
{ \opsem {\openv{}}
         {\e{m}}
         {\wrongorerror{}}
}
{\opsem {\openv{}}
        {\methodstaticexp {\classhint{1}} {\overrightarrow{\classhint{a}}} {\classhint{2}} {mth} {\e{m}} {\overrightarrow{\e{}}}}
        {\wrongorerror{}}
      }

\infer [BE-Method2]
{ \opsem {\openv{}}
         {\e{m}}
         {\val{m}}
  \\\\
  \overrightarrow{
  \opsem {\openv{}}
         {\e{n-1}}
         {\val{n-1}}
       }
         \\\\
  \opsem {\openv{}}
         {\e{n}}
         {\wrongorerror{}}
}
{\opsem {\openv{}}
        {\methodstaticexp {\classhint{1}} {\overrightarrow{\classhint{a}}} {\classhint{2}} {mth} {\e{m}} {\overrightarrow{\e{}}}}
        {\wrongorerror{}}
      }

\infer [BE-Method3]
{ \opsem {\openv{}}
         {\e{m}}
         {\val{m}}
  \\
  \overrightarrow{
  \opsem {\openv{}}
         {\e{a}}
         {\val{a}}
       }
  \\\\
  \invokejavamethod {\classhint{1}} {\val{m}} {mth}
                    {\overrightarrow{\classhint{a}}} {\overrightarrow{\val{a}}}
                    {\classhint{2}}
                    {\errorvalv{}}
}
{\opsem {\openv{}}
        {\methodstaticexp {\classhint{1}} {\overrightarrow{\classhint{a}}} {\classhint{2}} {mth} {\e{m}} {\overrightarrow{\e{a}}}}
        {\errorvalv{}}
      }

\infer [BE-New1]
{ \overrightarrow{
  \opsem {\openv{}}
         {\e{n-1}}
         {\val{n-1}}
       }
       \\\\
  \opsem {\openv{}}
         {\e{n}}
         {\wrongorerror{}}
       }
{ \opsem {\openv{}}
         {\newstaticexp {\overrightarrow{\classhint{i}}} {\classhint{1}} 
                        {\class{}} {\overrightarrow{\e{}}}}
         {\wrongorerror{}}
       }

\infer [BE-New2]
{ 
  \overrightarrow{
  \opsem {\openv{}}
         {\e{i}}
         {\val{i}}
       }
         \\\\
         \newjava {\classhint{1}}
                  {\overrightarrow{\classhint{i}}}
                  {\overrightarrow{\val{i}}}
                  {\errorvalv{}}
       }
{ \opsem {\openv{}}
         {\newstaticexp {\overrightarrow{\classhint{i}}} {\classhint{1}} 
                        {\class{}} {\overrightarrow{\e{i}}}}
         {\errorvalv{}}}

\infer [BE-DefMulti]
{ \opsem {\openv{}} {\e{d}} {\wrongorerror{}}
}
{\opsem {\openv{}}
        {\createmultiexp {\ty{}}
                         {\e{d}}}
        {\wrongorerror{}}
}

\infer [BE-DefMethod1]
{ \opsem {\openv{}}
         {\e{m}}
         {\wrongorerror{}}
}
{\opsem {\openv{}}
        {\extendmultiexp {\e{m}} {\e{v}} {\e{f}}}
        {\wrongorerror{}}
      }

\infer [BE-DefMethod2]
{ \opsem {\openv{}}
         {\e{m}}
         {\multi {\val{d}} {\disptable{}}}
         \\\\
  \opsem {\openv{}}
         {\e{v}}
         {\wrongorerror{}}
}
{\opsem {\openv{}}
        {\extendmultiexp {\e{m}} {\e{v}} {\e{f}}}
        {\wrongorerror{}}
      }

\infer [BE-DefMethod3]
{ \opsem {\openv{}}
         {\e{m}}
         {\multi {\val{d}} {\disptable{}}}
         \\\\
  \opsem {\openv{}}
         {\e{v}}
         {\val{v}}
         \\\\
  \opsem {\openv{}}
         {\e{f}}
         {\wrongorerror{}}
}
{\opsem {\openv{}}
        {\extendmultiexp {\e{m}} {\e{v}} {\e{f}}}
         {\wrongorerror{}}
      }
\end{mathpar}
\caption{Error and stuck propagation (continued in Figure~\ref{appendix:figure:errorstuck2})}
\label{appendix:figure:errorstuck1}
\end{figure*}

\begin{figure*}
\begin{mathpar}

\infer [BE-IsA1]
{ \opsem {\openv{}} {\e{1}} {\wrongorerror{}}
}
{\opsem {\openv{}} {\isaapp {\e{1}} {\e{2}}} {\wrongorerror{}}}

\infer [BE-IsA2]
{ \opsem {\openv{}} {\e{1}} {\val{1}}
  \\\\
  \opsem {\openv{}} {\e{2}} {\wrongorerror{}}
}
{\opsem {\openv{}} {\isaapp {\e{1}} {\e{2}}} {\wrongorerror{}}}

\infer [BE-Assoc1]
{\opsem {\openv{}}
        {\e{m}}{\wrongorerror{}} 
}
{
 \opsem {\openv{}}
        {\assocexp {\e{m}} {\e{k}} {\e{v}}} 
        {\wrongorerror{}}
                }

\infer [BE-Assoc2]
{\opsem {\openv{}}
        {\e{m}} {\curlymap{\overrightarrow{({\val{a}}\ {\val{b}})}}}
        \\
 \opsem {\openv{}}
        {\e{k}}{\wrongorerror{}}
}
{
 \opsem {\openv{}}
        {\assocexp {\e{m}} {\e{k}} {\e{v}}} 
        {\wrongorerror{}}
                }

\infer [BE-Assoc3]
{\opsem {\openv{}}
        {\e{m}} {\curlymap{\overrightarrow{({\val{a}}\ {\val{b}})}}}
        \\
 \opsem {\openv{}}
        {\e{k}} {\val{k}}
        \\
 \opsem {\openv{}}
        {\e{v}} {\wrongorerror{}}
}
{
 \opsem {\openv{}}
        {\assocexp {\e{m}} {\e{k}} {\e{v}}} 
        {\wrongorerror{}}
                }

\infer [BE-Get1]
{\opsem {\openv{}}
        {\e{m}} {\wrongorerror{}}
}
{
 \opsem {\openv{}}
        {\getexp {\e{m}} {\e{k}}}
        {\wrongorerror{}}
}

\infer [BE-Get2]
{\opsem {\openv{}}
        {\e{m}} {\curlymap{\overrightarrow{({\val{a}}\ {\val{b}})}}}
        \\
 \opsem {\openv{}}
        {\e{k}} {\wrongorerror{}}
}
{
 \opsem {\openv{}}
        {\getexp {\e{m}} {\e{k}}}
        {\wrongorerror{}}
}
\end{mathpar}
\caption{Error and stuck propagation (continued from Figure~\ref{appendix:figure:errorstuck1})}
\label{appendix:figure:errorstuck2}
\end{figure*}





\begin{figure*}
\begin{mathpar}

\begin{array}{lllll}
  \inopenvalign{\openv{}}{\x{}}{\val{} & {\roundpair{\x{}}{\val{}}} \in \openv{}}\\
  \inopenvalign{\openv{}}{\pth {\keype{k}} {\object{}}}{\getexp {{\openv{}}(\object{})}{\kw{}}}\\
  \inopenvalign{\openv{}}{\pth {\classpe{}} {\object{}}}{\appexp {\classconst{}} {{\openv{}}(\object{})}}

\end{array}

\end{mathpar}
\caption{Path translation}
\end{figure*}

\begin{figure*}
\begin{mathpar}

\begin{array}{lllll}
\updatefigure{}
\end{array}

\begin{array}{lllll}
\restrictremovefigure{}
\end{array}

\end{mathpar}
\caption{Type Update}
\label{appendix:updaterestrictremove}
\end{figure*}


\begin{figure*}
\begin{mathpar}
\infer [M-Or]
{ \satisfies{\openv{}}{\prop{1}}\ \text{or}\  \satisfies{\openv{}}{\prop{2}}}
{ \satisfies{\openv{}}{\orprop{\prop{1}}{\prop{2}}}
                   }

\infer [M-Imp]
{ \satisfies{\openv{}}{\prop{1}}\ \text{implies}\ \satisfies{\openv{}}{\prop{2}}}
{ \satisfies{\openv{}}{\impprop{\prop{1}}{\prop{2}}}
                   }

\infer [M-And]
{ \satisfies{\openv{}}{\prop{1}}
\\ \satisfies{\openv{}}{\prop{2}}}
{ \satisfies{\openv{}}{\andprop{\prop{1}}{\prop{2}}}
                   }


\infer [M-Top]
{}
{ \satisfies{\openv{}}{\topprop{}}
                   }

                   \\

\infer [M-Type]
{ \judgement {} {\openv{}({\pth{\pathelem{}}{\x{}}})} {\ty{}}{\filterset{\thenprop{\prop{}}}{\elseprop{\prop{}}}}{\object{}}}
{ \satisfies{\openv{}}{\isprop{\ty{}}{\pth{\pathelem{}}{\x{}}}}
                   }

\infer [M-NotType]
{ \judgement {} {\openv{}({\pth{\pathelem{}}{\x{}}})} {\s{}}{\filterset{\thenprop{\prop{}}}{\elseprop{\prop{}}}}{\object{}}
\\\\
\text{there is no}\ \val{}\ \text{such that}\ \judgement{}{\val{}}{\ty{}}{\filterset{\thenprop{\prop{1}}}{\elseprop{\prop{1}}}}{\object{1}}
\ \text{and}\ \judgement{}{\val{}}{\s{}}{\filterset{\thenprop{\prop{2}}}{\elseprop{\prop{2}}}}{\object{2}}
}
{ \satisfies{\openv{}}{\notprop{\ty{}}{\pth{\pathelem{}}{\x{}}}}
                   }
\end{mathpar}
\caption{Satisfaction Relation}
\end{figure*}

\begin{figure*}
\begin{mathpar}
\infer [L-Atom]
{ {\prop{}} \in {\propenv{}}}
{ \inpropenv {\propenv{}} {\prop{}}
}

\infer [L-True]
{}
{ \inpropenv {\propenv{}} {\topprop{}}}

\infer [L-False]
{ \inpropenv {\propenv{}} {\botprop{}}}
{ \inpropenv {\propenv{}} {\prop{}}}

\infer [L-AndI]
{ \inpropenv {\propenv{}} {\prop{1}}
  \\\\
  \inpropenv {\propenv{}} {\prop{2}}}
{ \inpropenv {\propenv{}} {\andprop {\prop{1}}{\prop{2}}}}

\infer [L-AndE]
{ \inpropenv {\propenv{}, {\prop{1}}, {\prop{2}}} {\prop{}} }
{ \inpropenv {\propenv{}, {\andprop {\prop{1}}{\prop{2}}}} {\prop{}}}

\infer [L-ImplI]
{ \inpropenv {\propenv{}, {\prop{1}}} {\prop{2}}}
{ \inpropenv {\propenv{}} {\impprop {\prop{1}}{\prop{2}}}}

\infer [L-ImplE]
{ \inpropenv {\propenv{}} {\prop{1}}
  \\\\
  \inpropenv {\propenv{}} {\impprop {\prop{1}}{\prop{2}}}}
{ \inpropenv {\propenv{}} {\prop{2}}}

\infer [L-OrI]
{ \inpropenv {\propenv{}} {\prop{1}}
  \ \text{or}\ 
  \inpropenv {\propenv{}} {\prop{2}}}
{ \inpropenv {\propenv{}} {\orprop {\prop{1}}{\prop{2}}}}


\infer [L-OrE]
{ \inpropenv {\propenv{}, {\prop{1}}}{\prop{}}
  \\\\
  \inpropenv {\propenv{}, {\prop{2}}}{\prop{}}}
{ \inpropenv {\propenv{}, {\orprop {\prop{1}}{\prop{2}}}}{\prop{}}}

\infer [L-Sub]
{ \inpropenv {\propenv{}} {\isprop {\ty{}}{\pth {\pathelem{}} {\x{}}}}
  \\
  \issubtypein {} {\ty{}}{\s{}}
}
{ \inpropenv {\propenv{}} {\isprop {\s{}}{\pth {\pathelem{}} {\x{}}}}}

\infer [L-SubNot]
{ \inpropenv {\propenv{}} {\notprop {\s{}}{\pth {\pathelem{}} {\x{}}}}
  \\
  \issubtypein {} {\ty{}}{\s{}}}
{ \inpropenv {\propenv{}} {\notprop {\ty{}}{\pth {\pathelem{}} {\x{}}}}}

\infer [L-Bot]
{ \inpropenv {\propenv{}} {\isprop {\Bot} {\pth {\pathelem{}} {\x{}}}}}
{ \inpropenv {\propenv{}} {\prop{}}}

{\LUpdate}

\\

\text{(The metavariable \propisnotmeta{} ranges over \ty{} and \nottype{\ty{}} (without variables).)}

\end{mathpar}
\caption{Proof System}
\label{appendix:figure:proofsystem}
\end{figure*}

\begin{figure*}
$$
\begin{array}{lclr}

{\withpolarity
  {\replacefor
    {\filterset {\thenprop {\prop{}}}{\elseprop {\prop{}}}}
    {\object{}}
    {\x{}}}
  {\polaritymeta{}}}
  &=&
{\filterset 
  {\withpolarity
    {\replacefor
      {\thenprop {\prop{}}}
      {\object{}}
      {\x{}}}
    {\polaritymeta{}}}
  {\withpolarity
    {\replacefor
      {\elseprop {\prop{}}}
      {\object{}}
      {\x{}}}
    {\polaritymeta{}}}}
\\\\
{\withpolarity
  {\replacefor
    {\isprop {\propisnotmeta{}} {\pth {\pathelem{}} {\x{}}}}
    {\pth {\pathelemp{}} {\y{}}}
    {\x{}}}
  {\polaritymeta{}}}
&=&
  {\isprop {({\withpolarity
              {\replacefor
               {\propisnotmeta{}}
               {\pth {\pathelemp{}} {\y{}}}
               {\x{}}}
              {\polaritymeta{}}})}
           {{\pathelem{}}({\pth {\pathelemp{}} {\y{}}})}}
           \\

{\pluspolarity
{\replacefor
  {\isprop {\propisnotmeta{}} {\pth {\pathelem{}} {\x{}}}}
  {\emptyobject{}}
  {\x{}}}
}
&=&
{\topprop{}}
\\
{\minuspolarity
{\replacefor
  {\isprop {\propisnotmeta{}} {\pth {\pathelem{}} {\x{}}}}
  {\emptyobject{}}
  {\x{}}}
}
&=&
{\botprop{}}

\\
{\withpolarity
{\replacefor
  {\isprop {\propisnotmeta{}} {\pth {\pathelem{}} {\x{}}}}
  {\object{}}
  {\z{}}}
{\polaritymeta{}}}
&=&
  {\isprop {\propisnotmeta{}} {\pth {\pathelem{}} {\x{}}}}
  & \x{} \not= \z{}\ \text{and}\ \z{} \not\in {\fv {\propisnotmeta{}}}

\\
{\pluspolarity
{\replacefor
  {\isprop {\propisnotmeta{}} {\pth {\pathelem{}} {\x{}}}}
  {\object{}}
  {\z{}}}
}
&=&
{\topprop{}}
  & \x{} \not= \z{}\ \text{and}\ \z{} \in {\fv {\propisnotmeta{}}}
\\
{\minuspolarity
{\replacefor
  {\isprop {\propisnotmeta{}} {\pth {\pathelem{}} {\x{}}}}
  {\object{}}
  {\z{}}}
}
&=&
{\botprop{}}
  & \x{} \not= \z{}\ \text{and}\ \z{} \in {\fv {\propisnotmeta{}}}

\\
{\withpolarity
{\replacefor
  {\topprop{}}
  {\object{}}
  {\x{}}}
{\polaritymeta{}}}
&=&
  {\topprop{}}

\\
{\withpolarity
{\replacefor
  {\botprop{}}
  {\object{}}
  {\x{}}}
{\polaritymeta{}}}
&=&
  {\botprop{}}

\\
{\pluspolarity
{\replacefor
  {({\impprop {\prop{1}} {\prop{2}}})}
  {\object{}}
  {\x{}}}
}
&=&
{\impprop 
  {\minuspolarity {\replacefor {\prop{1}} {\object{}} {\x{}}}}
  {\pluspolarity {\replacefor {\prop{2}} {\object{}} {\x{}}}}}
\\
{\minuspolarity
{\replacefor
  {({\impprop {\prop{1}} {\prop{2}}})}
  {\object{}}
  {\x{}}}
}
&=&
{\impprop 
  {\pluspolarity {\replacefor {\prop{1}} {\object{}} {\x{}}}}
  {\minuspolarity {\replacefor {\prop{2}} {\object{}} {\x{}}}}}
\\
{\withpolarity
{\replacefor
  {({\orprop {\prop{1}} {\prop{2}}})}
  {\object{}}
  {\x{}}}
{\polaritymeta{}}}
&=&
{\orprop 
  {\withpolarity
    {\replacefor {\prop{1}} {\object{}} {\x{}}}
    {\polaritymeta{}}}
  {\withpolarity
    {\replacefor {\prop{2}} {\object{}} {\x{}}}
    {\polaritymeta{}}}}
\\
{\withpolarity
{\replacefor
  {({\andprop {\prop{1}} {\prop{2}}})}
  {\object{}}
  {\x{}}}
{\polaritymeta{}}}
&=&
{\andprop 
{\withpolarity
  {\replacefor {\prop{1}} {\object{}} {\x{}}}
  {\polaritymeta{}}}
{\withpolarity
  {\replacefor {\prop{2}} {\object{}} {\x{}}}
  {\polaritymeta{}}}}

    \\\\

{\withpolarity
{\replacefor
  {\pth {\pathelem{}} {\x{}}}
  {\pth {\pathelemp{}} {\y{}}}
  {\x{}}}
{\polaritymeta{}}}
           &=&
{\pth{\pathelem{}}{\pth {\pathelemp{}} {\y{}}}}

    \\

{\withpolarity
{\replacefor
  {\pth {\pathelem{}} {\x{}}}
  {\emptyobject{}}
  {\x{}}}
{\polaritymeta{}}}
           &=&
{\emptyobject{}}

    \\

{\withpolarity
{\replacefor
  {\pth {\pathelem{}} {\x{}}}
  {\object{}}
  {\z{}}}
{\polaritymeta{}}}
           &=&
{\pth {\pathelem{}} {\x{}}}

& \x{} \not= \z{}
    \\

{\withpolarity
{\replacefor
  {\emptyobject{}}
  {\object{}}
  {\x{}}}
{\polaritymeta{}}}
           &=&
{\emptyobject{}}

\end{array}
$$
\center{\text{Substitution on types is capture-avoiding structural recursion.}}
\caption{Substitution}
\end{figure*}


\chapter{Soundness for \lambdatc{}}

{\javaassumptionsall{appendix}}

\begin{lemma} \label{appendix:lemma:envagree}
  If \openv{} and \openvp{} agree on \fv{\prop{}}
  and \satisfies{\openv{}}{\prop{}}
  then \satisfies{\openvp{}}{\prop{}}.
\begin{proof}
  Since the relevant parts of \openv{} and \openvp{} agree, the proof follows trivially.
\end{proof}
\end{lemma}

\begin{lemma} \label{appendix:lemma:substfilter}
  If 
  \begin{itemize}
    \item \prop{1} = {\replacefor {\prop{2}} {\object{}} {\x{}}},
    \item
  {\satisfies{\openv{2}}{\prop{2}}},
    \item
  $\forall v \in \fv{\prop{2}} - \x{}$.
                              {\inopenvnoeq{\openv{1}}{v}} = {\inopenvnoeq {\openv{2}}{v}},
    \item
  and {\inopenvnoeq{\openv{2}}{\x{}}} = {\inopenvnoeq{\openv{1}}{\object{}}}
  \end{itemize}
  then \satisfies{\openv{1}}{\prop{1}}.

  \begin{proof}
    By induction on the derivation of the model judgement.
  \end{proof}
\end{lemma}

\begin{lemma} \label{appendix:lemma:satisfies}
  If \satisfies{\openv{}}{\propenv{}} and \inpropenv{\propenv{}}{\prop{}} then \satisfies{\openv{}}{\prop{}}.

  \begin{proof}
    By structural induction on \inpropenv{\propenv{}}{\prop{}}.
%    \begin{itemize}
%      \item[]
%        \begin{case}[L-True]
%
%          Holds by M-Top.
%        \end{case}
%      \item[]
%        \begin{case}[L-False]
%          {\inpropenv{\propenv{}}{\botprop{}}}
%
%          ??? TODO
%        \end{case}
%      \item[]
%        \begin{case}[L-AndI]
%          \inpropenv{\propenv{}}{\andprop{\prop{1}}{\prop{2}}}, \satisfies{\openv{}}{\propenv{}}
%
%          By inversion on the proof system we know \inpropenv{\propenv{}}{\prop{1}}
%          and
%          \inpropenv{\propenv{}}{\prop{2}}.
%
%          By the induction hypothesis we know \satisfies{\openv{}}{\prop{1}}
%          and
%          \satisfies{\openv{}}{\prop{2}}.
%
%          By M-And we know \satisfies{\openv{}}{\andprop{\prop{1}}{\prop{2}}}
%          and we are done.
%        \end{case}
%      \item[]
%        \begin{case}[L-AndE]
%          \inpropenv{\propenv{},{\andprop{\prop{1}}{\prop{2}}}}{\prop{}}, \satisfies{\openv{}}{\propenv{},{\andprop{\prop{1}}{\prop{2}}}}
%
%
%          By inversion on the proof system we know  either
%          \inpropenv{\propenv{},{\prop{1}}}{\prop{}}
%          or
%          \inpropenv{\propenv{},{\prop{2}}}{\prop{}}.
%
%          %TODO
%         % By the induction hypothesis we know 
%         % either
%         % \satisfies{\openv{}}{\prop{1}}
%         % and
%         % \satisfies{\openv{}}{\prop{2}}.
%        \end{case}
%    \end{itemize}
  \end{proof}
\end{lemma}

\begin{lemma} \label{appendix:lemma:goodobjects+ve}
  If \inpropenv{\propenv{}}{\isprop{\ty{}}{\pth{\pathelem{}}{\x{}}}},
  \satisfies{\openv{}}{\propenv{}}
  and \inopenv{\openv{}}{\pth{\pathelem{}}{\x{}}}{\val{}}
  then
  \judgementselfrewrite{}{\val{}}{\ty{}}{\filterset{\thenprop{\propp{}}}{\elseprop{\propp{}}}}{\objectp{}}
  for some {\thenprop{\propp{}}}, {\elseprop{\propp{}}} and {\objectp{}}.
  \begin{proof}
    Corollary of lemma~\ref{appendix:lemma:satisfies}.
  \end{proof}
\end{lemma}

\begin{lemma}[Paths are independent] \label{appendix:lemma:pathindependent}
  If \inopenvnoeq{\openv{}}{\object{}} = \inopenvnoeq{\openv{1}}{\objectp{}}
  then \inopenvnoeq{\openv{}}{\pth{\pathelem{}}{\object{}}} =
       \inopenvnoeq{\openv{1}}{\pth{\pathelem{}}{\objectp{}}}
 \begin{proof}
   By induction on \pathelem{}.
   % FIXME
%   \begin{case}[\pathelem{} = \emptypath{}]
%     \inopenvnoeq{\openv{}}{\object{}} = {\inopenvnoeq{\openv{}}{\objectp{}}}
%
%     As 
%     \inopenvnoeq{\openv{}}{\pth{\emptypath{}}{\object{}}} = \inopenvnoeq{\openv{}}{\object{}}
%     and
%     \inopenvnoeq{\openv{}}{\pth{\emptypath{}}{\objectp{}}} = \inopenvnoeq{\openv{}}{\objectp{}}
%     we can conclude 
%     \inopenvnoeq{\openv{}}{\pth{\emptypath{}}{\object{}}} = \inopenvnoeq{\openv{}}{\pth{\emptypath{}}{\objectp{}}}.
%   \end{case}
%   \begin{case}[\pathelem{} = {\destructpath{\pesyntax{}}{\pathelem{1}}}]
%     \inopenvnoeq{\openv{}}{\object{}} = {\inopenvnoeq{\openv{}}{\objectp{}}}
%
%     By cases on \pesyntax{}.
%
%     \begin{itemize}
%       \item[]
%   \begin{subcase}[\pesyntax{} = {\keype{\kw{}}}] 
%
%%     TODO
%     By the induction hypothesis on {\pathelem{1}}
%     we know {\inopenvnoeq{\openv{}}{\pth{\pathelem{1}}{\object{}}}} =
%             {\inopenvnoeq{\openv{1}}{\pth{\pathelem{1}}{\objectp{}}}}.
%             By the definition of pth translation 
%             {\inopenvnoeq{\openv{}}{\pth{\pathelem{1}}{\object{}}}} = {\getexp {{\openv{}}(\object{})}{\kw{}}}
%             and 
%             {\inopenvnoeq{\openv{}}{\pth{\pathelem{1}}{\objectp{}}}} = {\getexp {{\openv{}}(\objectp{})}{\kw{}}}
%   \end{subcase} 
%     \end{itemize}
%%     TODO
%   \end{case}
 \end{proof}
\end{lemma}

\begin{lemma}[\classconst]\label{appendix:lemma:classconst}
  If
  {\opsem{\openv{}}{\appexp{\classconst{}}{\openv{}({\pth{\pathelem{}}{\x{}}})}}{\class{}}} then
  {\satisfies{\openv{}}{\isprop{\class{}}{\pth{\pathelem{}}{\x{}}}}}.

  \begin{proof}
    Induction on the definition of {\classconst{}}.
  \end{proof}
\end{lemma}

{\consistentwithdefinition{appendix}}

{\istruefalsedefinitions{appendix}}

%\begin{lemma}[Path substitution] \label{appendix:lemma:pathsubustitution}
%  If \satisfies{\openv{}}{\prop{}} and 
%  \openv(\object{}) = \openv(\objectp{})
%  then \satisfies{\openv{}}{\replacefor{\prop{}}{\object{}}{\objectp{}}}.
%  \begin{proof}
%    By straightforward induction on \prop{}.
%  \end{proof}
%\end{lemma}
%

\begin{lemma}[isa? has correct propositions] \label{appendix:lemma:isa}
  If
  \begin{itemize}
    \item
  \judgementrewrite {\propenv{}} {\val{1}} {\ty{1}}
             {\filterset {\thenprop {\prop{1}}}
                         {\elseprop {\prop{1}}}}
                       {\object{1}}
                       {\val{1}},
    \item
  \judgementrewrite {\propenv{}} {\val{2}} {\ty{2}}
             {\filterset {\thenprop {\prop{2}}}
                         {\elseprop {\prop{2}}}}
                       {\object{2}}
                       {\val{2}},
    \item
        \isaopsem{\val{1}}{\val{2}} = {\val{}}, 
    \item
        \satisfies{\openv{}}{\propenv{}},
    \item
  \isacompare{\ty{1}}{\object{1}}{\ty{2}}{\filterset {\thenprop {\propp{}}} {\elseprop {\propp{}}}},
    \item
        \inpropenv{\thenprop{\propp{}}}{\thenprop{\prop{}}}, and
    \item
        \inpropenv{\elseprop{\propp{}}}{\elseprop{\prop{}}},
    \end{itemize}
  then either
\begin{itemize}
  \item
        if
        \istrueval{\val{}}
        then {\satisfies{\openv{}}{\thenprop{\prop{}}}}, or
  \item
        if
        \isfalseval{\val{}}
        then {\satisfies{\openv{}}{\elseprop{\prop{}}}}.
\end{itemize}
\begin{proof}
        By cases on the definition of \isaopsemliteral
        and subcases on \isaopsemliteral.

        \begin{itemize} % isaopsem
          \item[]
            \begin{subcase}[\isaopsem{\val{1}}{\val{1}} = {\true{}}, \text{if} \val{1} \notequal\ {\class{}}]
              \ 

              \val{1} = \val{2}, \val{1} \notequal\ {\class{}}, \val{2} \notequal\ {\class{}}, \istrueval{\val{}}
              
              Since \istrueval{\val{}} we prove {\satisfies{\openv{}}{\thenprop{\prop{}}}}
              by cases on the definition of \isacompareliteral{}:
              \begin{itemize} % isacompare
                \item[]
                  \begin{subcase}[\isacompare{\s{}}{\pth{\classpe{}}{\pth{\pathelem{}}{\x{}}}}{\Value{\class{}}}
                                 {\filterset{\isprop{\class{}} {\pth{\pathelem{}}{\x{}}}}
                                            {\notprop{\class{}}{\pth{\pathelem{}}{\x{}}}}}]
                    \ 


                    \object{1} = {\pth{\classpe{}}{\pth{\pathelem{}}{\x{}}}},
                    \ty{2} = {\Value{\class{}}},
                    \inpropenv{\isprop{\class{}} {\pth{\pathelem{}}{\x{}}}}{\thenprop{\prop{}}}

                    Unreachable by inversion on the typing relation, since \ty{2} = {\Value{\class{}}},
                    yet \val{2} \notequal\ {\class{}}.

%                    By inversion on the typing relation, since \classpe{} is the last path element of \object{1}
%                    then \opsem{\openv{}}{\appexp{\classconst{}}{\openv{}({\pth{\pathelem{}}{\x{}}})}}{\val{1}}.
%
%                    Since {\val{1}} = {\val{2}} then {\ty{1}} = {\ty{2}}, and because {\ty{2}} = {\Value{\class{}}}
%                    then {\ty{1}} = {\Value{\class{}}}.
%
%                    By inversion {\val{1}} = {\class{}}, via T-Class.
%
%                    Since {\opsem{\openv{}}{\appexp{\classconst{}}{\openv{}({\pth{\pathelem{}}{\x{}}})}}{\class{}}}
%                    we conclude by lemma~\ref{appendix:lemma:classconst}
%                    with {\satisfies{\openv{}}{\isprop{\class{}} {\pth{\pathelem{}}{\x{}}}}}.

                  \end{subcase}
                \item[]
                  \begin{subcase}[\isacompare{\s{}}{\object{}}{\Value{\singletonmeta{}}}
                    {\replacefor
                      {\filtersetparen{\isprop{\Value{\singletonmeta{}}} {\x{}}}
                        {\notprop{\Value{\singletonmeta{}}}{\x{}}}}
                      {\object{}}
                      {\x{}}}\ 
                    \text{if}\ {\singletonmeta{}} \notequal \class{}]
                    \ 

                    \ty{2} = {\Value{\singletonmeta{}}}, 
                    {\singletonmeta{}} \notequal \class{},
                    \inpropenv{\replacefor{\isprop{\Value{\singletonmeta{}}} {\x{}}}
                                           {\object{1}}
                                           {\x{}}}{\thenprop{\prop{}}}
                    %\elseprop{\prop{}} = {\replacefor{\notprop{\Value{\singletonmeta{}}} {\x{}}}
                    %                       {\object{1}}
                    %                       {\x{}}}

                    Since \ty{2} = {\Value{\singletonmeta{}}} where {\singletonmeta{}} \notequal \class{},
                    by inversion on the typing judgement 
                    {\val{2}} is either \true{}, \false{}, \nil{} or \kw{}
                    by T-True, T-False, T-Nil or T-Kw.

                    Since \val{1} = {\val{2}} then \ty{1} = \ty{2}, and since \ty{2} = {\Value{\singletonmeta{}}}
                      then \ty{1} = {\Value{\singletonmeta{}}}, so
                    \judgementtwo {} {\val{1}} {\Value{\singletonmeta{}}}

                    If \object{1} = \emptyobject{} then \thenprop{\prop{}} = \topprop{} and
                    we derive
                    {\satisfies{\openv{}}{\topprop{}}} with M-Top.

                    Otherwise \object{1} = \pth{\pathelem{}}{\x{}} and 
                    \inpropenv{\isprop{\Value{\singletonmeta{}}}{\pth{\pathelem{}}{\x{}}}}{\thenprop{\prop{}}},
                    and since
                    \judgementtwo {} {\val{1}} {\Value{\singletonmeta{}}}
                    then
                    \judgementtwo {} {{\openv{}}(\pth{\pathelem{}}{\x{}})} {\Value{\singletonmeta{}}},
                    which we can use M-Type to derive
                    {\satisfies{\openv{}}{\isprop{\Value{\singletonmeta{}}}{\pth{\pathelem{}}{\x{}}}}}.
                  \end{subcase}
                \item[]
                  \begin{subcase}[\isacompare{\s{}}{\object{}}{\ty{}} {\filterset{\topprop{}} {\topprop{}}}]
                    \ 

                    {\thenprop{\prop{}}} = {\topprop{}}

                    {\satisfies{\openv{}}{\topprop{}}} holds by M-Top.

                  \end{subcase}
              \end{itemize}
            \end{subcase}
          \item[]
            \begin{subcase}[\isaopsem{\class{1}}{\class{2}} = {\true{}}, \text{if}\ \issubtypein{}{\class{1}}{\class{2}}]
              \ 

              \val{1} = \class{1}, \val{2} = \class{2},
              \issubtypein{}{\class{1}}{\class{2}},
              \istrueval{\val{}}
              
              Since \istrueval{\val{}} we prove {\satisfies{\openv{}}{\thenprop{\prop{}}}}
              by cases on the definition of \isacompareliteral{}:
              \begin{itemize} % isacompare
                \item[]
                  \begin{subcase}[\isacompare{\s{}}{\pth{\classpe{}}{\pth{\pathelem{}}{\x{}}}}{\Value{\class{}}}
                                 {\filterset{\isprop{\class{}} {\pth{\pathelem{}}{\x{}}}}
                                            {\notprop{\class{}}{\pth{\pathelem{}}{\x{}}}}}]
                    \ 


                    \object{1} = {\pth{\classpe{}}{\pth{\pathelem{}}{\x{}}}},
                    \ty{2} = {\Value{\class{2}}},
                    \inpropenv{\isprop{\class{2}} {\pth{\pathelem{}}{\x{}}}}{\thenprop{\prop{}}}

                    By inversion on the typing relation, since \classpe{} is the last path element of \object{1}
                    then \opsem{\openv{}}{\appexp{\classconst{}}{\openv{}({\pth{\pathelem{}}{\x{}}})}}{\val{1}}.

                    Since {\opsem{\openv{}}{\appexp{\classconst{}}{\openv{}({\pth{\pathelem{}}{\x{}}})}}{\class{1}}},
                    as {\val{1}} = {\class{1}},
                    we can derive from lemma~\ref{appendix:lemma:classconst}
                    {\satisfies{\openv{}}{\isprop{\class{1}} {\pth{\pathelem{}}{\x{}}}}}.

                    By the induction hypothesis we can derive 
                    {\inpropenv{\propenv{}}{\isprop{\class{1}} {\pth{\pathelem{}}{\x{}}}}},
                    and with the fact {\issubtypein{}{\class{1}}{\class{2}}}
                    we can use L-Sub to conclude 
                    {\inpropenv{\propenv{}}{\isprop{\class{2}} {\pth{\pathelem{}}{\x{}}}}},
                    and finally by lemma~\ref{appendix:lemma:satisfies}
                    we derive
                    {\satisfies{\openv{}}{\isprop{\class{2}} {\pth{\pathelem{}}{\x{}}}}}.

                  \end{subcase}
                \item[]
                  \begin{subcase}[\isacompare{\s{}}{\object{}}{\Value{\singletonmeta{}}}
                    {\replacefor
                      {\filtersetparen{\isprop{\Value{\singletonmeta{}}} {\x{}}}
                        {\notprop{\Value{\singletonmeta{}}}{\x{}}}}
                      {\object{}}
                      {\x{}}}\ 
                    \text{if}\ {\singletonmeta{}} \notequal \class{}]
                    \ 

                    \ty{2} = {\Value{\singletonmeta{}}}, 
                    {\singletonmeta{}} \notequal \class{},
                    \inpropenv{\replacefor{\isprop{\Value{\singletonmeta{}}} {\x{}}}
                                           {\object{1}}
                                           {\x{}}}{\thenprop{\prop{}}}

                    Unreachable case since 
                    \ty{2} = {\Value{\singletonmeta{}}} where 
                    {\singletonmeta{}} \notequal \class{},
                    but \val{2} = \class{2}.
                  \end{subcase}
                \item[]
                  \begin{subcase}[\isacompare{\s{}}{\object{}}{\ty{}} {\filterset{\topprop{}} {\topprop{}}}]
                    \ 

                    {\thenprop{\prop{}}} = {\topprop{}}

                    {\satisfies{\openv{}}{\topprop{}}} holds by M-Top.

                  \end{subcase}
              \end{itemize}
            \end{subcase}
          \item[]
            \begin{subcase}[\isaopsem{\val{1}}{\val{2}} = {\false{}}, otherwise]
              \ 

              \val{1} \notequal\ \val{2},
              \isfalseval{\val{}}
              
              Since \isfalseval{\val{}} we prove {\satisfies{\openv{}}{\elseprop{\prop{}}}}
              by cases on the definition of \isacompareliteral{}:
              \begin{itemize} % isacompare
                \item[]
                  \begin{subcase}[\isacompare{\s{}}{\pth{\classpe{}}{\pth{\pathelem{}}{\x{}}}}{\Value{\class{}}}
                                 {\filterset{\isprop{\class{}} {\pth{\pathelem{}}{\x{}}}}
                                            {\notprop{\class{}}{\pth{\pathelem{}}{\x{}}}}}]
                    \ 


                    \object{1} = {\pth{\classpe{}}{\pth{\pathelem{}}{\x{}}}},
                    \ty{2} = {\Value{\class{}}},
                    \inpropenv{\notprop{\class{}} {\pth{\pathelem{}}{\x{}}}}{\elseprop{\prop{}}}

                    By inversion on the typing relation, since \classpe{} is the last path element of \object{1}
                    then \opsem{\openv{}}{\appexp{\classconst{}}{\openv{}({\pth{\pathelem{}}{\x{}}})}}{\val{1}}.
                    
                    By the definition of {\classconst{}} either {\val{1}} = {\class{}} or {\val{1}} = \nil{}.

                    If {\val{1}} = \nil{}, then we know from the definition of \isaopsemliteral that 
                    {\openv{}({\pth{\pathelem{}}{\x{}}})} = \nil{}.

                    Since \judgementtwo{}{\openv{}({\pth{\pathelem{}}{\x{}}})}{\Nil},
                    and there is no \val{1} such that both \judgementtwo{}{\openv{}({\pth{\pathelem{}}{\x{}}})}{\class} and
                    \judgementtwo{}{\openv{}({\pth{\pathelem{}}{\x{}}})}{\Nil{}},
                    we use M-NotType to derive 
                    \satisfies{\openv{}}{\notprop{\class{}} {\pth{\pathelem{}}{\x{}}}}.

                    Similarly if {\val{1}} = \class{1}, by the definition of \isacompareliteral
                    we know {\notsubtypein{}{\class{1}}{\class{}}} and 
                    {\openv{}({\pth{\pathelem{}}{\x{}}})} = \class{1}.

                    Since \judgementtwo{}{\openv{}({\pth{\pathelem{}}{\x{}}})}{\class{1}},
                    and there is no \val{1} such that both 
                    \judgementtwo{}{\val{1}}{\class{}} and
                    \judgementtwo{}{\val{1}}{\class{1}},
                    we use M-NotType to derive 
                    \satisfies{\openv{}}{\notprop{\class{}} {\pth{\pathelem{}}{\x{}}}}.


                  \end{subcase}
                \item[]
                  \begin{subcase}[\isacompare{\s{}}{\object{}}{\Value{\singletonmeta{}}}
                    {\replacefor
                      {\filtersetparen{\isprop{\Value{\singletonmeta{}}} {\x{}}}
                        {\notprop{\Value{\singletonmeta{}}}{\x{}}}}
                      {\object{}}
                      {\x{}}}\ 
                    \text{if}\ {\singletonmeta{}} \notequal \class{}]
                    \ 

                    \ty{2} = {\Value{\singletonmeta{}}}, 
                    {\singletonmeta{}} \notequal \class{},
                    %\thenprop{\prop{}} = {\replacefor{\isprop{\Value{\singletonmeta{}}} {\x{}}}
                    %                       {\object{1}}
                    %                       {\x{}}}
                    \inpropenv{\replacefor{\notprop{\Value{\singletonmeta{}}} {\x{}}}
                                           {\object{1}}
                                           {\x{}}}{\elseprop{\prop{}}}

                    Since \ty{2} = {\Value{\singletonmeta{}}} where {\singletonmeta{}} \notequal \class{},
                    by inversion on the typing judgement 
                    {\val{2}} is either \true{}, \false{}, \nil{} or \kw{}
                    by T-True, T-False, T-Nil or T-Kw.

                    If \object{1} = \emptyobject{} then \elseprop{\prop{}} = \topprop{} and
                    we derive
                    {\satisfies{\openv{}}{\topprop{}}} with M-Top.

                    Otherwise \object{1} = \pth{\pathelem{}}{\x{}} and 
                    \inpropenv{\notprop{\Value{\singletonmeta{}}}{\pth{\pathelem{}}{\x{}}}}{\elseprop{\prop{}}}.
                    Noting that \val{1} \notequal\ \val{2},
                    we know \judgementtwo{}{\openv{}({\pth{\pathelem{}}{\x{}}})}{\s{}}
                    where \s{} \notequal\ {\Value{\singletonmeta{}}},
                    and there is no \val{1} such that both 
                    \judgementtwo{}{\val{1}}{\Value{\singletonmeta{}}} and
                    \judgementtwo{}{\val{1}}{\s{}}
                    so we can use M-NotType to derive
                    {\satisfies{\openv{}}{\notprop{\Value{\singletonmeta{}}}{\pth{\pathelem{}}{\x{}}}}}.
                  \end{subcase}
                \item[]
                  \begin{subcase}[\isacompare{\s{}}{\object{}}{\ty{}} {\filterset{\topprop{}} {\topprop{}}}]
                    \ 

                    {\elseprop{\prop{}}} = {\topprop{}}

                    {\satisfies{\openv{}}{\topprop{}}} holds by M-Top.

                  \end{subcase}
              \end{itemize}
            \end{subcase}
        \end{itemize}
      \end{proof}
\end{lemma}

\begin{lemma} \label{appendix:lemma:soundness}
{\soundnesslemmahypothesis}
\begin{proof}
By induction and cases on the derivation of \opsem {\openv{}} {\e{}} {\a{}},
and subcases on the penultimate rule of the derivation of
\judgementrewrite{\propenv{}}{\ep{}}{\ty{}}{\filterset{\thenprop{\prop{}}}{\elseprop{\prop{}}}}{\object{}}{\e{}}
followed by T-Subsume as the final rule.

% induction on the derivation of the evaluation semantics because we want to apply
% the induction hypothesis to subderivations of the eval sem. If eg. used the typing
% judgement, we couldn't use the induction hypothesis on applications of higher-order
% functions, since the subderivation of T-Abs wouldn't be present.

\begin{case}[B-Val]

  \begin{itemize}
    \item[] 
      \begin{subcase}[T-True]
        \val{} = \true{},
  \ep{} = \true{},
  \e{} = \true{}, \issubtypein{}{\True}{\ty{}}, \inpropenv{\topprop{}}{\thenprop{\prop{}}}, 
  \inpropenv{\botprop{}}{\elseprop{\prop{}}}, \issubtypein{}{\emptyobject{}}{\object{}}

        Proving part 1 is trivial: \object{} is a superobject of \emptyobject{}, which can only be \emptyobject{}.

        To prove part 2, we note that \val{} = \true{}
        and \inpropenv{\topprop{}}{\thenprop{\prop{}}},
        so \satisfies{\openv{}}{\thenprop{\prop{}}} by M-Top.

        Part 3 holds as \e{} can only be reduced to itself via B-Val.

        Part 4 holds vacuously.
      \end{subcase}
    \item[]
      \begin{subcase}[T-HMap] \val{} = {\curlymapvaloverright{\val{k}}{\val{v}}},
  \ep{} = {\curlymapvaloverright{\val{k}}{\val{v}}},
  \e{} = {\curlymapvaloverright{\val{k}}{\val{v}}},
  \issubtypein{}{\HMapc {\mandatory{}}}{\ty{}},
  \inpropenv{\topprop{}}{\thenprop{\prop{}}},
  \inpropenv{\botprop{}}{\elseprop{\prop{}}},
  \issubtypein{}{\emptyobject{}}{\object{}},
  $\overrightarrow{\judgementtwo {} {\val{k}}{\Value \kw{}}}$,
  $\overrightarrow{\judgementtwo {} {\val{v}}{\ty{v}}}$,
  \mandatory{} = \mandatorysetoverright{\kw{}}{\ty{v}}

        Similar to T-True.

        Part 4 holds by the induction hypothese on {\overr{\val{k}}} and {\overr{\val{v}}}.
      \end{subcase}
    \item[]
      \begin{subcase}[T-Kw] \val{} = {\kw{}},
  \ep{} = {\kw{}},
  \e{} = {\kw{}},
  \issubtypein{}{\Value{\kw{}}}{\ty{}},
  \inpropenv{\topprop{}}{\thenprop{\prop{}}},
  \inpropenv{\botprop{}}{\elseprop{\prop{}}},
  \issubtypein{}{\emptyobject{}}{\object{}}

        Similar to T-True.
      \end{subcase}
      \begin{subcase}[T-Str]
        Similar to T-Kw.
      \end{subcase}
  \item[] 
    \begin{subcase}[T-False]
      \val{} = \false{},
\ep{} = \false, 
\e{} = \false, 
\issubtypein{}{\False}{\ty{}},
\inpropenv{\botprop{}}{\thenprop{\prop{}}},
\inpropenv{\topprop{}}{\elseprop{\prop{}}},
\issubtypein{}{\emptyobject{}}{\object{}}

Proving part 1 is trivial: \object{} is a superobject of \emptyobject{}, which must be \emptyobject{}. 

To prove part 2, we note that \val{} = \false{}
and \inpropenv{\topprop{}}{\elseprop{\prop{}}}, so \satisfies{\openv{}}{\elseprop{\prop{}}} by M-Top. 

Part 3 holds as \e{} can only be reduced to itself via B-Val.

Part 4 holds vacuously.
\end{subcase}
    \item[]
      \begin{subcase}[T-Class] \val{} = {\class{}},
  \ep{} = {\class{}},
  \e{} = {\class{}},
  \issubtypein{}{\Value{\class{}}}{\ty{}},
  \inpropenv{\topprop{}}{\thenprop{\prop{}}},
  \inpropenv{\botprop{}}{\elseprop{\prop{}}},
  \issubtypein{}{\emptyobject{}}{\object{}}

        Similar to T-True.
      \end{subcase}
    \item[]
      \begin{subcase}[T-Instance]
        \val{} = {\classvalue{\classhint{}} {\overrightarrow {\classfieldpair{\fld{i}} {\val{i}}}}},
        \ep{} = {\classvalue{\classhint{}} {\overrightarrow {\classfieldpair{\fld{}} {\val{}}}}},
        \e{} = {\classvalue{\classhint{}} {\overrightarrow {\classfieldpair{\fld{}} {\val{}}}}},
        \issubtypein{}{\class{}}{\ty{}},
        \inpropenv{\topprop{}}{\thenprop{\prop{}}},
        \inpropenv{\botprop{}}{\elseprop{\prop{}}},
        \issubtypein{}{\emptyobject{}}{\object{}}


        Similar to T-True.

        Part 4 holds by the induction hypotheses on ${\overrightarrow{\val{i}}}$.
      \end{subcase}
  \item[] 
    \begin{subcase}[T-Nil] 
      \val{} = \nil{},
\ep{} = \nil, 
\e{} = \nil, 
\issubtypein{}{\Nil}{\ty{}},
\inpropenv{\botprop{}}{\thenprop{\prop{}}},
\inpropenv{\topprop{}}{\elseprop{\prop{}}},
\issubtypein{}{\emptyobject{}}{\object{}}

      Similar to T-False.
\end{subcase}
\item[]
\begin{subcase}[T-Multi] 
  \val{} = {\multi {\val{1}} {\curlymapvaloverright{\val{k}}{\val{v}}}}
  \ep{} = {\multi {\val{1}} {\curlymapvaloverright{\val{k}}{\val{v}}}},
  \judgementtworewrite {} {\val{1}} {\ty{1}}{\val{1}},
  \overr{\judgementtworewrite{}{\val{k}}{\Top{}}{\val{k}}},
  \overr{\judgementtworewrite{}{\val{v}}{\s{}}{\val{v}}},
  \e{} = {\multi {\val{1}} {\curlymapvaloverright{\val{k}}{\val{v}}}},
  \issubtypein{}{\MultiFntype {\s{}} {\ty{1}}}{\ty{}},
  \inpropenv{\topprop{}}{\thenprop{\prop{}}},
  \inpropenv{\botprop{}}{\elseprop{\prop{}}},
  \issubtypein{}{\emptyobject{}}{\object{}}

        Similar to T-True.
\end{subcase}
\item[]
\begin{subcase}[T-Const]
  \e{} = {\const{}},
  \issubtypein{}{\constanttype{\const{}}}{\ty{}},
  \inpropenv{\topprop{}}{\thenprop{\prop{}}},
  \inpropenv{\botprop{}}{\elseprop{\prop{}}},
  \issubobjin{}{\emptyobject{}}{\object{}}

        Parts 1, 2 and 3 hold for the same reasons as T-True. 
\end{subcase}


  \end{itemize}
\end{case}



\begin{case}[B-Local]
{ \inopenv {\openv{}} {\x{}} {\val{}} },
{ \opsem {\openv{}} {\x{}} {\val{}} }

\begin{itemize}
  \item[]
\begin{subcase}[T-Local]
  \ep{} = \x{}, 
  \e{} = \x{}, 
  \inpropenv{\notprop {\falsy{}} {\x{}}}{\thenprop{\prop{}}},
  \inpropenv{\isprop {\falsy{}} {\x{}}}{\elseprop{\prop{}}},
\issubtypein{}{\x{}}{\object{}},
\inpropenv{\propenv{}}{\isprop{\ty{}}{\x{}}}

Part 1 follows from \inopenv{\openv{}}{\object{}} {\val{}}, since either {\object{}} = \x{}
and \inopenv{\openv{}}{\x{}} {\val{}} is a premise of B-Local, or {\object{}} = {\emptyobject{}} which also
satisfies the goal.

Part 2 considers two cases: if \istrueval{\val{}}, then 
\satisfies{\openv{}}{\notprop{\falsy}{\x{}}} holds by M-NotType; if \isfalseval{\val{}}, then 
\satisfies{\openv{}}{\isprop{\falsy}{\x{}}} holds by M-Type.

We prove part 3 by observing
\inpropenv{\propenv{}}{\isprop{\ty{}}{\x{}}},
\satisfies{\openv{}}{\propenv{}},
and
\inopenv {\openv{}} {\x{}} {\val{}}
(by B-Local)
which gives us the desired result.

Part 4 holds vacuously.
\end{subcase}
\end{itemize}

\end{case}

\begin{case}[B-Do]
  \opsem {\openv{}} {\e{1}} {\val{1}},
  \opsem {\openv{}} {\e{2}} {\val{}}

\begin{itemize}
  \item[] \begin{subcase}[T-Do]
      \ep{} = {\doexp {\ep{1}} {\ep{2}}},
  \judgementrewrite {\propenv{}} 
             {\ep{1}} {\ty{1}}
             {\filterset {\thenprop {\prop{1}}} {\elseprop {\prop1}}} 
             {\object{1}}
             {\e{1}},
\judgementrewrite {\propenv{}, {\orprop {\thenprop {\prop{1}}} {\elseprop {\prop{1}}}}}
           {\ep{}} {\ty{}}
           {\filterset {\thenprop {\prop{}}} {\elseprop {\prop{}}}} 
           {\object{}}
           {\e{}},
    \e{} = {\doexp {\e{1}} {\e{2}}}

For all parts we note 
    since {\e{1}} can be either a true or false value
    then
    {\satisfies{\openv{}}{{\propenv{}},{\orprop {\thenprop {\prop{1}}} {\elseprop {\prop{1}}}}}}
    by M-Or,
    which together with 
\judgement {\propenv{}, {\orprop {\thenprop {\prop{1}}} {\elseprop {\prop{1}}}}}
           {\e{2}} {\ty{}}
           {\filterset {\thenprop {\prop{}}} {\elseprop {\prop{}}}} 
           {\object{}},
    and
  \opsem {\openv{}} {\e{2}} {\val{}}
    allows us to apply the induction hypothesis on \e{2}.

To prove part 1 we use the induction hypothesis on \e{2}
to show either \object{} = \emptyobject{} 
or \inopenv {\openv{}} {\object{}} {\val{}}, since \e{} always
evaluates to the result of \e{2}.

For part 2 we use the induction hypothesis on \e{2}
to show if \istrueval{\val{}} then
        {\satisfies{\openv{}}{\thenprop{\prop{}}}}
        or
  if \isfalseval{\val{}} then
        {\satisfies{\openv{}}{\elseprop{\prop{}}}}.

Parts 3 and 4 follow from the induction hypothesis on \e{2}.
    \end{subcase}
\end{itemize}
\end{case}

\begin{case}[BE-Do1]
  \opsem {\openv{}} {\e{1}} {\errorval{\val{e}}},
  \opsem {\openv{}} {\e{}} {\errorval{\val{}}}

        Trivially reduces to an error.
\end{case}

\begin{case}[BE-Do2]
  \opsem {\openv{}} {\e{1}} {\val{1}},
  \opsem {\openv{}} {\e{2}} {\errorvalv{}},
  \opsem {\openv{}} {\e{}} {\errorvalv{}}

        As above.
\end{case}

\begin{case}[B-New]
  $
  \overrightarrow{
  \opsem {\openv{}}
         {\e{i}}
         {\val{i}}
       }$,
         $\newjava {\classhint{1}}
                  {\overrightarrow{\classhint{i}}}
                  {\overrightarrow{\val{i}}}
                  {\val{}}$

\begin{itemize}
  \item[]
\begin{subcase}[T-New]
  \ep{} = {\newexp {\class{}} {\overrightarrow{\ep{i}}}},
  \inct{\ctctorentry{\overr{\class{i}}}}{\ctlookupctors{\ct{}}{\classhint{}}},
  \overr{\javatotcnil{\classhint{i}}{\ty{i}}},
  \overr{
  \judgementtworewrite {\propenv{}}
                    {\ep{i}} {\ty{i}}
                    {\e{i}}
                  },
  \e{} = {\newstaticexp {\overrightarrow{\classhint{i}}} {\classhint{}} 
                                                          {\class{}} {\overrightarrow{\e{i}}}},
  \issubtypein{}{\javatotcexp{\classhint{}}}{\ty{}},
  \inpropenv{\topprop{}}{\thenprop{\prop{}}},
  \inpropenv{\botprop{}}{\elseprop{\prop{}}},
  \issubobjin{}{\emptyobject{}}{\object{}}

Part 1 follows \object{} = \emptyobject{}.

Part 2 requires some explanation. The two false values in Typed Clojure
cannot be constructed with \newliteral{}, so the only case is \val{} $\not=$ \false\ (or \nil)
where \thenprop{\prop{}} = \topprop{} so \satisfies{\openv{}}{\thenprop{\prop{}}}.
\Void{} also lacks a constructor.

Part 3 holds as B-New reduces to a \emph{non-nilable}
instance of \class{} via \newjavaliteral (by assumption~\ref{appendix:assumption:new}), 
and {\ty{}} is a supertype of \javatotcexp{\classhint{}}.

\end{subcase}
\item[]
\begin{subcase}[T-NewStatic]
  {\ep{}} = {\newstaticexp {\overrightarrow{\classhint{i}}} {\classhint{}}
                                                          {\class{}} {\overrightarrow{\e{i}}}}

  Non-reflective constructors cannot be written directly by the user, so we can assume
  the class information attached to the syntax corresponds to an actual constructor by inversion
  from T-New.

  The rest of this case progresses like T-New.
\end{subcase}
\end{itemize}
\end{case}

\begin{case}[BE-New1] $\overrightarrow{
  \opsem {\openv{}}
         {\e{i-1}}
         {\val{i-1}}
       }$,
  \opsem {\openv{}}
         {\e{i}}
         {\errorvalv{}},
  \opsem {\openv{}} {\e{}} {\errorvalv{}}

        Trivially reduces to an error.

\end{case}

\begin{case}[BE-New2] 
  $\overrightarrow{
  \opsem {\openv{}}
         {\e{i}}
         {\val{i}}
       }$,
         \newjava {\classhint{1}}
                  {\overrightarrow{\classhint{i}}}
                  {\overrightarrow{\val{i}}}
                  {\errorvalv{}},
        \opsem {\openv{}} {\e{}} {\errorvalv{}}

        As above.

\end{case}

\begin{case}[B-Field]
  \opsem {\openv{}}
         {\e{1}} 
         {\classvalue{\classhint{1}} {\classfieldpair{\fld{}} {\val{}}}}

\begin{itemize}
  \item[]
\begin{subcase}[T-Field]
  \ep{} = {\fieldexp {\fld{}} {\ep{1}}},
  \judgementtworewrite {\propenv{}} {\ep{}} {\s{}} {\e{}},
  \issubtypein{}{\s{}}{\Object{}},
  \tctojava{\s{}}{\classhint{1}},
  \inct{\ctfldentry{\fld{}}{\classhint{2}}}{\ctlookupfields{\ct{}}{\classhint{1}}},
  \e{} = {\fieldstaticexp {\classhint{1}} {\classhint{2}} {\fld{}} {\e{1}}}
  \issubtypein{}{\javatotcnilexp{\classhint{2}}}{\ty{}},
  \inpropenv{\topprop{}}{\thenprop{\prop{}}},
  \inpropenv{\topprop{}}{\elseprop{\prop{}}},
  \issubobjin{}{\emptyobject{}}{\object{}}


Part 1 is trivial as \object{} is always \emptyobject{}.

Part 2 holds trivially; \val{} can be either a true or false value
and both {\thenprop{\prop{}}} and {\elseprop{\prop{}}}
are \topprop{}.

Part 3 relies on the semantics of \getfieldliteral (assumption~\ref{appendix:assumption:field})
in B-Field, which returns a \emph{nilable} instance of \classhint{2},
and \ty{} is a supertype of \javatotcnilexp{\classhint{2}}.
Notice \issubtypein{}{\s{}}{\Object{}} is required to guard from dereferencing \nil{},
as {\classhint{1}} erases occurrences of \Nil{} in \s{} via  \tctojava{\s{}}{\classhint{1}}.
\end{subcase}
  \item[]

\begin{subcase}[T-FieldStatic]
  {\ep{}} = {\fieldstaticexp {\classhint{1}} {\classhint{2}} {\fld{}} {\e{1}}}

  Non-reflective field lookups cannot be written directly by the user, so we can assume
  the class information attached to the syntax corresponds to an actual field by inversion
  from T-Field.

  The rest of this case progresses like T-Field.
\end{subcase}

\end{itemize}
\end{case}

\begin{case}[BE-Field]
  \opsem {\openv{}}
         {\e{1}} 
         {\errorvalv{}},
  \opsem {\openv{}}
         {\e{}}
         {\errorvalv{}}

         Trivially reduces to an error.

\end{case}

\begin{case}[B-Method]
  \opsem {\openv{}}
         {\e{m}}
         {\val{m}},
  $\overrightarrow{
  \opsem {\openv{}}
         {\e{a}}
         {\val{a}}}$,
  \invokejavamethod {\classhint{1}} {\val{m}} {mth}
                    {\overrightarrow{\classhint{a}}} {\overrightarrow{\val{a}}}
                    {\classhint{2}}
                    {\val{}}

\begin{itemize}
  \item[]
\begin{subcase}[T-Method]
  \judgementtworewrite {\propenv{}} {\ep{}} {\s{}} {\e{}},
             \issubtypein{}{\s{}}{\Object{}},
  \tctojava{\s{}}{\classhint{1}},
                  \inct{\ctmthentry{\mth{}}{\overrightarrow{\classhint{i}}}{\classhint{2}}}{\ctlookupmethods{\ct{}}{\classhint{1}}},
                  \overr{\javatotcnil{\classhint{i}}{\ty{i}}},
             \overr{
  \judgementtworewrite {\propenv{}} {\ep{i}} {\ty{i}} {\e{i}}
                  },
  \e{} = {\methodstaticexp {\classhint{1}} 
                          {\overr {\classhint{i}}} 
                          {\classhint{2}}
                          {\mth{}} {\e{m}} {\overr{\e{a}}}},
                        \issubtypein{}{\javatotcnilexp{\classhint{2}}}{\ty{}},
  \inpropenv{\topprop{}}{\thenprop{\prop{}}},
  \inpropenv{\topprop{}}{\elseprop{\prop{}}},
  \issubobjin{}{\emptyobject{}}{\object{}}


Part 1 is trivial as \object{} is always \emptyobject{}.

Part 2 holds trivially, \val{} can be either a true or false value
and both {\thenprop{\prop{}}} and {\elseprop{\prop{}}}
are \topprop{}.

Part 3 relies on the semantics of \invokejavamethodliteral (assumption~\ref{appendix:assumption:method})
in B-Method, which returns a \emph{nilable} instance of \classhint{2},
and \ty{} is a supertype of \javatotcnil{\classhint{2}}.
Notice \issubtypein{}{\s{}}{\Object{}} is required to guard from dereferencing \nil{},
as {\classhint{1}} erases occurrences of \Nil{} in \s{} via  \tctojava{\s{}}{\classhint{1}}.
\end{subcase}
\item[]
\begin{subcase}[T-MethodStatic]
  \ep{} = 
  {\methodstaticexp {\classhint{1}} 
        {\overrightarrow {\classhint{i}}} 
        {\classhint{2}}
        {\mth{}} {\e{1}} {\overrightarrow{\e{i}}}}

  Non-reflective method invocations cannot be written directly by the user, so we can assume
  the class information attached to the syntax corresponds to an actual method by inversion
  from T-Method.

  The rest of this case progresses like T-Method.
\end{subcase}


\end{itemize}

\end{case}

\begin{case}[BE-Method1]
  \opsem {\openv{}}
         {\e{m}}
         {\errorval{\val{}}},
  \opsem {\openv{}}
         {\e{}}
         {\errorval{\val{}}}

         Trivially reduces to an error.
\end{case}
\begin{case}[BE-Method2]
  \opsem {\openv{}}
         {\e{m}}
         {\val{m}},
 $\overrightarrow{
  \opsem {\openv{}}
         {\e{n-1}}
         {\val{n-1}}
       }$,
  \opsem {\openv{}}
         {\e{n}}
         {\errorval{\val{}}},
  \opsem {\openv{}}
         {\e{}}
         {\errorval{\val{}}}

  As above.
\end{case}
\begin{case}[BE-Method3]
  \opsem {\openv{}}
         {\e{m}}
         {\val{m}},
  $\overrightarrow{
  \opsem {\openv{}}
         {\e{a}}
         {\val{a}}
       }$,
  \invokejavamethod {\classhint{1}} {\val{m}} {mth}
                    {\overrightarrow{\classhint{a}}} {\overrightarrow{\val{a}}}
                    {\classhint{2}}
                    {\errorvalv{}},
  \opsem {\openv{}} {\e{}} {\errorvalv{}}

  As above.

\end{case}

\begin{case}[B-DefMulti]
  \val{} = {\multi {\val{d}} {\emptydisptable}},
  \opsem {\openv{}} {\e{d}} {\val{d}}



\begin{itemize}
  \item[]
\begin{subcase}[T-DefMulti]
  \ep{} = {\createmultiexp {\s{}} {\ep{d}}},
  \s{} = {\ArrowOne {\x{}} {\ty{1}} {\ty{2}}
                          {\filterset {\thenprop {\prop{1}}}
                                      {\elseprop {\prop{1}}}}
                          {\object{1}}},
  \ty{d} = {\ArrowOne {\x{}} {\ty{1}} {\ty{3}}
                          {\filterset {\thenprop {\prop{2}}}
                                      {\elseprop {\prop{2}}}}
                          {\object{2}}},
\judgementtworewrite {\propenv{}} {\ep{}} {\sp{}} {\e{}},
  \e{} = {\createmultiexp {\s{}} {\e{d}}},
  \issubtypein{}{\MultiFntype {\s{}} {\ty{d}}}{\ty{}},
  \inpropenv{\topprop{}}{\thenprop{\prop{}}},
  \inpropenv{\botprop{}}{\elseprop{\prop{}}},
  \issubobjin{}{\emptyobject{}}{\object{}}


Part 1 and 2 hold for the same reasons as T-True.
For part 3 we show \judgementtwo{}{\multi {\val{d}} {\emptydisptable}}{\MultiFntype {\s{}} {\ty{d}}}
by T-Multi, since \judgementtwo {} {\val{d}} {\ty{d}} by the inductive hypothesis on {\e{d}}
and {\emptydisptable} vacuously satisfies the other premises of T-Multi, so we are done.

\end{subcase}
\end{itemize}
\end{case}

\begin{case}[BE-DefMulti] \opsem {\openv{}} {\e{d}} {\errorvalv{}},
        \opsem {\openv{}} {\e{}} {\errorvalv{}}

        Trivially reduces to an error.

\end{case}

\begin{case}[B-DefMethod]

        \ 

        \begin{enumerate}
          \item
       \val{} = {\multi {\val{d}} {\disptablep{}}},
          \item
        \opsem {\openv{}}
               {\e{m}}
               {\multi {\val{d}} {\disptable{}}},
          \item
  \opsem {\openv{}}
         {\e{v}}
         {\val{v}},
          \item
  \opsem {\openv{}}
         {\e{f}}
         {\val{f}},
          \item
         \disptablep{} = {\extenddisptable {\disptable{}} 
                                {\val{v}}
                                {\val{f}}}
        \end{enumerate}

  \begin{itemize}
    \item[]
      \begin{subcase}[T-DefMethod]
        \ 
        
        \begin{enumerate}[resume]

          \item
  \ep{} = {\extendmultiexp {\ep{m}} {\ep{v}} {\ep{f}}},
          \item
  \ty{m} = {\ArrowOne {\x{}} {\ty{1}} {\s{}}
                     {\filterset {\thenprop {\prop{m}}}
                                 {\elseprop {\prop{m}}}}
                     {\object{m}}},
          \item
  \ty{d} = {\ArrowOne {\x{}} {\ty{1}} {\sp{}}
                     {\filterset {\thenprop {\prop{d}}}
                                 {\elseprop {\prop{d}}}}
                     {\object{d}}},
          \item
\judgementtworewrite {\propenv{}}
                  {\ep{m}} {\MultiFntype {\ty{m}} {\ty{d}}}
                  {\e{m}}
          \item
  \isacompare{\sp{}}{\object{d}}{\ty{v}}{\filterset {\thenprop {\prop{i}}} {\elseprop {\prop{i}}}},
          \item
\judgementtworewrite {\propenv{}}
           {\e{v}} {\ty{v}}
           {\e{v}}
          \item
  \judgementrewrite {\propenv{}, {\isprop{\ty{1}} {\x{}}}, {\thenprop {\prop{i}}}}
           {\ep{f}} {\s{}}
           {\filterset {\thenprop {\prop{m}}}
                       {\elseprop {\prop{m}}}}
           {\object{m}}
           {\e{f}}
          \item
  \e{} = {\extendmultiexp {\e{m}} {\e{v}} {\e{f}}},
          \item
  \e{f} = {\abs {\x{}} {\ty{1}} {\e{b}}},
\item
  \issubtypein{}{\MultiFntype {\ty{m}} {\ty{d}}}{\ty{}},
\item
  \inpropenv{\topprop{}}{\thenprop{\prop{}}},
\item
  \inpropenv{\botprop{}}{\elseprop{\prop{}}},
\item
  \issubobjin{}{\emptyobject{}}{\object{}}
        \end{enumerate}

                                Part 1 and 2 hold for the same reasons as T-True, noting that the propositions
                                and object agree with T-Multi.

For part 3 we show
\judgementtwo{}{\multi {\val{d}} {\extenddisptable {\disptable{}}{\val{v}}{\val{f}}}}{\MultiFntype {\ty{m}} {\ty{d}}}
by noting \judgementtwo {} {\val{d}} {\ty{d}},
  \judgementtwo{}{\val{v}}{\Top{}}
  and
  \judgementtwo{}{\val{f}}{\ty{m}}, and since \disptable{} is in the correct form by the inductive
  hypothesis on {\e{m}} we can satisfy all premises of T-Multi, so we are done.


      \end{subcase}

  \end{itemize}
\end{case}

      \begin{case}[BE-DefMethod1]
        \opsem {\openv{}}
               {\e{m}}
               {\errorval{\val{}}},
        \opsem {\openv{}}
                  {\e{}}
                {\errorval{\val{}}}

                Trivially reduces to an error.

      \end{case}
      \begin{case}[BE-DefMethod2]
        \opsem {\openv{}}
         {\e{m}}
         {\multi {\val{d}} {\disptable{}}},
  \opsem {\openv{}}
         {\e{v}}
         {\errorval{\val{}}},
        \opsem {\openv{}}
                  {\e{}}
                {\errorval{\val{}}}

                Trivially reduces to an error.
      \end{case}
      \begin{case}[BE-DefMethod3]
        \opsem {\openv{}}
         {\e{m}}
         {\multi {\val{d}} {\disptable{}}},
  \opsem {\openv{}}
         {\e{v}}
         {\val{v}},
  \opsem {\openv{}}
         {\e{f}}
         {\errorval{\val{}}},
        \opsem {\openv{}}
                  {\e{}}
                {\errorval{\val{}}}

                Trivially reduces to an error.

      \end{case}

\begin{case}[B-BetaClosure]
  \ 

  \begin{itemize}
    \item
  \opsem{\openv{}}{\e{}}{\val{}},
    \item
  \opsem {\openv{}}
         {\e{1}}
         {\closure {\openv{c}} {\abs {\x{}} {\s{}} {\e{b}}}},
    \item
  \opsem {\openv{}}
         {\e{2}}
         {\val{2}},
    \item
  \opsem {\extendopenv {\openv{c}} {\x{}} {\val{2}}}
         {\e{b}}
         {\val{}}
     \end{itemize}


\begin{itemize}
  \item[]
\begin{subcase}[T-App]
  \ 

  \begin{itemize}
    \item
  \ep{} = {\appexp {\ep{1}} {\ep{2}}},
    \item
  \judgementrewrite {\propenv{}} {\ep{1}} {\ArrowOne {\x{}} {\s{}}
                                                       {\ty{f}}
                                                       {\filterset {\thenprop {\prop{f}}}
                                                                   {\elseprop {\prop{f}}}}
                                                       {\object{f}}}
                {\filterset {\thenprop {\prop{1}}}
                            {\elseprop {\prop{1}}}}
                {\object{1}}
                {\e{1}},
    \item
  \judgementrewrite {\propenv{}}
                 {\ep{2}} {\s{}}
                 {\filterset {\thenprop {\prop{2}}}
                             {\elseprop {\prop{2}}}}
                 {\object{2}}
                 {\e{2}},
    \item
  \e{} = {\appexp {\e{1}} {\e{2}}},
    \item
      \issubtypein{}  {\replacefor {\ty{f}} {\object{2}} {\x{}}}{\ty{}},
    \item
      \inpropenv{\replacefor {\thenprop {\prop{f}}} {\object{2}} {\x{}}} {\thenprop {\prop{}}},
    \item
      \inpropenv{\replacefor {\elseprop {\prop{f}}} {\object{2}} {\x{}}} {\elseprop {\prop{}}},
    \item
      \issubobjin{}{\replacefor {\object{f}} {\object{2}} {\x{}}} {\object{}}
  \end{itemize}

         By inversion on \e{1} from T-Clos
         there is some environment {\propenvc{}} such that
         \begin{itemize}
           \item
              \satisfies{\openv{c}}{\propenvc{}} and
            \item
  \judgement {\propenvc{}} {\abs {\x{}} {\s{}} {\e{b}}} {\ArrowOne {\x{}} {\s{}}
                                                       {\ty{f}}
                                                       {\filterset {\thenprop {\prop{f}}}
                                                                   {\elseprop {\prop{f}}}}
                                                       {\object{f}}}
                {\filterset {\thenprop {\prop{1}}}
                            {\elseprop {\prop{1}}}}
                {\object{1}},
         \end{itemize}
         and also by inversion on \e{1} from T-Abs
         \begin{itemize}
           \item
  { \judgementrewrite {\propenvc{}, {\isprop {\s{}} {\x{}}}}
              {\ep{b}} {\ty{f}}
               {\filterset {\thenprop {\prop{f}}}
                           {\elseprop {\prop{f}}}}
               {\object{f}}
               {\e{b}}}.
         \end{itemize}

          From 
          \begin{itemize}
            \item
              \satisfies{\openv{c}}{\propenvc{}},
            \item
  \judgementrewrite {\propenv{}}
                 {\ep{2}} {\s{}}
                 {\filterset {\thenprop {\prop{2}}}
                             {\elseprop {\prop{2}}}}
                 {\object{2}}
                 {\e{2}} and 
            \item
  \opsem {\openv{}}
         {\e{2}}
         {\val{2}},
     \end{itemize}
              we know (by substitution)
              \satisfies{\extendopenv {\openv{c}} {\x{}} {\val{2}}}{\propenvc{},{\isprop{\s{}}{\x{}}}}.

              We want to prove
        \judgementrewrite {\propenvc{}}
                          {\replacefor{\ep{b}}{\val{2}}{\x{}}}
                          {\replacefor{\ty{f}}{\object{2}}{\x{}}}
               {\replacefor{\filterset {\thenprop {\prop{f}}}
                                       {\elseprop {\prop{f}}}}{\object{2}}{\x{}}}
                          {\replacefor{\object{f}}{\object{2}}{\x{}}}
                          {\replacefor{\e{b}}{\val{2}}{\x{}}}, 
                          which can be justified by noting 
          \begin{itemize}
            \item
  \judgementtworewrite {\propenvc{},{\isprop{\s{}}{\x{}}}}{\ep{b}}{\ty{f}}{\e{b}},
            \item
  \judgementrewrite {\propenv{}}
                 {\ep{2}} {\s{}}
                 {\filterset {\thenprop {\prop{2}}}
                             {\elseprop {\prop{2}}}}
                 {\object{2}}
                 {\e{2}} and 
            \item
  \opsem {\openv{}}
         {\e{2}}
         {\val{2}}.
     \end{itemize}

     From the previous fact and \satisfies{\openv{c}}{\propenvc{}},
              we know
  \opsem {\openv{c}}
         {\replacefor{\e{b}}{\val{2}}{\x{}}}
         {\val{}}.

                    Noting that 
      \issubtypein{}  {\replacefor {\ty{f}} {\object{2}} {\x{}}}{\ty{}},
      \inpropenv{\replacefor {\thenprop {\prop{f}}} {\object{2}} {\x{}}} {\thenprop {\prop{}}},
      \inpropenv{\replacefor {\elseprop {\prop{f}}} {\object{2}} {\x{}}} {\elseprop {\prop{}}}
      and
      \issubobjin{}{\replacefor {\object{f}} {\object{2}} {\x{}}} {\object{}},
                    we can use
         \begin{itemize}
           \item
        \judgementrewrite {\propenvc{}}
                          {\replacefor{\ep{b}}{\val{2}}{\x{}}}
                          {\replacefor{\ty{f}}{\object{2}}{\x{}}}
               {\replacefor{\filterset {\thenprop {\prop{f}}}
                                       {\elseprop {\prop{f}}}}{\object{2}}{\x{}}}
                          {\replacefor{\object{f}}{\object{2}}{\x{}}}
                          {\replacefor{\e{b}}{\val{2}}{\x{}}}, 
           \item
              \satisfies{\openv{c}}{\propenvc{}},
           \item
\isconsistent{\openv{c}} (via induction hypothesis on {\ep{1}}), and
           \item 
  \opsem {\openv{c}}
         {\replacefor{\e{b}}{\val{2}}{\x{}}}
         {\val{}}.
         \end{itemize}
         to apply the induction hypothesis on {\replacefor{\ep{b}}{\val{2}}{\x{}}} and satisfy
         all conditions.

\end{subcase}
\end{itemize}
\end{case}

\begin{case}[B-Delta]
  \opsem {\openv{}} {\e{1}} {\const{}},
  \opsem {\openv{}} {\e{2}} {\val{2}},
  \constantopsem{\const{}}{\val{2}} = \val{}

\begin{itemize}
  \item[]
\begin{subcase}[T-App]
  \ 

  \begin{itemize}
    \item
  \ep{} = {\appexp {\ep{1}} {\ep{2}}},
    \item
  \judgementrewrite {\propenv{}} {\ep{1}} {\ArrowOne {\x{}} {\s{}}
                                                       {\ty{f}}
                                                       {\filterset {\thenprop {\prop{f}}}
                                                                   {\elseprop {\prop{f}}}}
                                                       {\object{f}}}
                {\filterset {\thenprop {\prop{1}}}
                            {\elseprop {\prop{1}}}}
                {\object{1}}
                {\e{1}},
    \item
  \judgementrewrite {\propenv{}}
                 {\ep{2}} {\s{}}
                 {\filterset {\thenprop {\prop{2}}}
                             {\elseprop {\prop{2}}}}
                 {\object{2}}
                 {\e{2}},
    \item
  \e{} = {\appexp {\e{1}} {\e{2}}},
    \item
      \issubtypein{}  {\replacefor {\ty{f}} {\object{2}} {\x{}}}{\ty{}},
    \item
      \inpropenv{\replacefor {\thenprop {\prop{f}}} {\object{2}} {\x{}}} {\thenprop {\prop{}}},
    \item
      \inpropenv{\replacefor {\elseprop {\prop{f}}} {\object{2}} {\x{}}} {\elseprop {\prop{}}},
    \item
      \issubobjin{}{\replacefor {\object{f}} {\object{2}} {\x{}}} {\object{}}
  \end{itemize}

  % TODO do I need to prove anything about the argument in the definition
  % of the constant being under \s{}?

  Prove by cases on \const{}.
  \begin{itemize}
    \item[] \begin{subcase}[\const{} = \classconst]
        \issubtypein{}
  {\ArrowOne {\x{}} {\Top{}}
                                      {\Union{\Nil}{\Class}}
                                      {\filterset {\topprop{}}
                                                  {\topprop{}}}
                                      {\pth {\classpe{}} {\x{}}}}
    {\ArrowOne {\x{}} {\s{}}
                                                       {\ty{f}}
                                                       {\filterset {\thenprop {\prop{f}}}
                                                                   {\elseprop {\prop{f}}}}
                                                       {\object{f}}}

    Prove by cases on \val{2}.

        \begin{itemize}
          \item[] \begin{subcase}[\val{2} = \classvalue{\class{}} {\protect\overrightarrow {\classfieldpair{\fld{i}} {\val{i}}}}]
                    \val{} = \class{}

                    To prove part 1,
                    note
                    \issubobjin{}{\replacefor {\object{f}} {\object{2}} {\x{}}} {\object{}},
                    and \issubobjin{}{\pth {\classpe{}} {\x{}}}{\object{f}}.
                    Then either \object{} = \emptyobject{} and we are done,
                    or \object{} = {\pth {\classpe{}}{\object{2}}} and
                    by the induction hypothesis on \e{2} we know \inopenv {\openv{}} {\object{2}} {\val{2}}
                    and by the definition of path translation we know
                    {\openv{}}({\pth {\classpe{}} {\object{2}}}) = {\appexp {\classconst{}} {{\openv{}}(\object{2})}},
                    which evaluates to \val{}.

                    Part 2 is trivial since both propositions can only be \topprop{}.
                    
                    Part 3 holds because 
                    \val{} = \class{},
                    \issubtypein{}{\Union{\Nil}{\Class}}{\replacefor {\ty{f}} {\object{2}} {\x{}}}
                    and
                    \issubtypein{}{\replacefor {\ty{f}} {\object{2}} {\x{}}}{\ty{}},
                    so
                    {\judgementtwo{}{\val{}}{\ty{}}}
                    since
                    {\judgementtwo{}{\class{}}{\Union{\Nil}{\Class}}}.
                  \end{subcase}
          \item[] \begin{subcase}[\val{2} = \class{}] \val{} = \Class{}

              As above.
                  \end{subcase}
          \item[] \begin{subcase}[\val{2} = \true{}] \val{} = \Boolean{}

              As above.
                  \end{subcase}
          \item[] \begin{subcase}[\val{2} = \false{}] \val{} = \Boolean{}


              As above.
                  \end{subcase}
          \item[] \begin{subcase}[\val{2} = {\closure {\openv{}} {\abs {\x{}} {\ty{}} {\e{}}}}] \val{} = \IFn{}


              As above.
                  \end{subcase}
          \item[] \begin{subcase}[\val{2} = {\multi {\val{d}} {\disptable{}}}] \val{} = \HMapInstance{}


              As above.
                  \end{subcase}
          \item[] \begin{subcase}[\val{2} = {\protect\curlymapvaloverright{\val{1}}{\val{2}}}] \val{} = \Keyword{}


              As above.
                  \end{subcase}
          \item[] \begin{subcase}[\val{2} = {\nil{}}] \val{} = \nil{}

             Parts 1 and 2 as above.
                    Part 3 holds because \val{} = \nil{}
                    and {\judgementtwo{}{\nil{}}{\Union{\Nil}{\Class}}}.
                  \end{subcase}
        \end{itemize}
      \end{subcase}
    %\item[]
    %  \begin{subcase}[\const{} = \throwconst]
    %    {\ArrowOne {\x{}} {\s{}}
    %                                                   {\ty{f}}
    %                                                   {\filterset {\thenprop {\prop{f}}}
    %                                                               {\elseprop {\prop{f}}}}
    %                                                   {\object{f}}}
    %                                                   =
    %    {\ArrowOne {\x{}} {\Top{}}
    %                                  {\Bot{}}
    %                                  {\filterset {\botprop{}}
    %                                              {\botprop{}}}
    %                                  {\emptyobject{}}}

    %                                  Part 1 is trivial since \object{} = \emptyobject{} after substitution.
    %                                  Part 2 holds vacuously as both propositions are \botprop{} after substitution.
    %                                  Finally part 3 holds since {\judgementtwo{}{\hastype{\errorval{\val{2}}}{\Bot{}}}}.

    %  \end{subcase}
  \end{itemize}

\end{subcase}
\end{itemize}
\end{case}

\begin{case}[B-BetaMulti]
  \ 

  \begin{itemize}
    \item
  \opsem {\openv{}}
         {\e{1}}
         {\multi {\val{d}} {\disptable{}}},
    \item
  \opsem {\openv{}}
         {\e{2}}
         {\val{2}},
    \item
  \opsem {\openv{}}
         {\appexp {\val{d}} {\val{2}}}
         {\val{e}},
    \item
  \getmethod {\disptable{}}
             {\val{e}}
             {\val{l}}
             {\val{g}},
    \item
  \opsem {\openv{}}
         {\appexp {\val{g}} {\val{2}}}
         {\val{}},
       \item {\disptable{}} = {\curlymapvaloverright{\val{k}}{\val{v}}}
     \end{itemize}
     \begin{itemize}
       \item[]
\begin{subcase}[T-App]
  \ 

  \begin{itemize}
    \item
  \ep{} = {\appexp {\ep{1}} {\ep{2}}},
    \item
  \judgementrewrite {\propenv{}} {\ep{1}} {\ArrowOne {\x{}} {\s{}}
                                                       {\ty{f}}
                                                       {\filterset {\thenprop {\prop{f}}}
                                                                   {\elseprop {\prop{f}}}}
                                                       {\object{f}}}
                {\filterset {\thenprop {\prop{1}}}
                            {\elseprop {\prop{1}}}}
                {\object{1}}
                {\e{1}},
    \item
  \judgementrewrite {\propenv{}}
                 {\ep{2}} {\s{}}
                 {\filterset {\thenprop {\prop{2}}}
                             {\elseprop {\prop{2}}}}
                 {\object{2}}
                 {\e{2}},
    \item
  \e{} = {\appexp {\e{1}} {\e{2}}},
    \item
      \issubtypein{}  {\replacefor {\ty{f}} {\object{2}} {\x{}}}{\ty{}},
    \item
      \inpropenv{\replacefor {\thenprop {\prop{f}}} {\object{2}} {\x{}}} {\thenprop {\prop{}}},
    \item
      \inpropenv{\replacefor {\elseprop {\prop{f}}} {\object{2}} {\x{}}} {\elseprop {\prop{}}},
    \item
      \issubobjin{}{\replacefor {\object{f}} {\object{2}} {\x{}}} {\object{}},
  \end{itemize}

     By inversion on \e{1} via T-Multi we know 
     \begin{itemize}
       \item
         \judgementrewrite{\propenv{}}{\ep{1}}{\MultiFntype{\s{t}}{\s{d}}}
                {\filterset {\thenprop {\prop{1}}}
                            {\elseprop {\prop{1}}}}
                {\object{1}}{\e{1}},
           
         \item \s{t} = {\ArrowOne {\x{}} {\s{}}
                                                       {\ty{f}}
                                                       {\filterset {\thenprop {\prop{f}}}
                                                                   {\elseprop {\prop{f}}}}
                                                       {\object{f}}},
         \item \s{d} = {\ArrowOne {\x{}} {\s{}}
                                                       {\ty{d}}
                                                       {\filterset {\thenprop {\prop{d}}}
                                                                   {\elseprop {\prop{d}}}}
                                                       {\object{d}}},
       \item
         \judgementtwo{}{\val{d}}{\s{d}}
              \item
  $\overrightarrow{\judgementtwo{}{\val{k}}{\Top{}}}$, and 
\item
  $\overrightarrow{\judgementtwo{}{\val{v}}{\s{t}}}$.
  \end{itemize}

  % FIXME do we really know this? seems obvious but the IH says something subtly different, might need
  % a lemma to bridge this. Same problem in T-IsA case.
  By the inductive hypothesis on 
  \opsem {\openv{}}
         {\e{2}}
         {\val{2}}
  we know 
  \judgementrewrite {\propenv{}} {\val{2}} {\s{}}
             {\filterset {\thenprop {\prop{2}}}
                         {\elseprop {\prop{2}}}}
                       {\object{2}}
                       {\val{2}}.

We then consider applying the evaluated argument to the dispatch function:
  \opsem {\openv{}}
         {\appexp {\val{d}} {\val{2}}}
         {\val{e}}.

         Since we can satisfy T-App with
       \begin{itemize}
         \item
         \judgementtwo{}{\val{d}}{\ArrowOne {\x{}} {\s{}}
                                                       {\ty{d}}
                                                       {\filterset {\thenprop {\prop{d}}}
                                                                   {\elseprop {\prop{d}}}}
                                                       {\object{d}}}, and
         \item
  \judgementrewrite {\propenv{}} {\val{2}} {\s{}}
             {\filterset {\thenprop {\prop{2}}}
                         {\elseprop {\prop{2}}}}
                       {\object{2}}
                       {\val{2}},
       \end{itemize}
       we can apply the inductive hypothesis
       to derive
  \judgementrewrite {\propenv{}} {\val{e}} 
  {\replacefor{\ty{d}}
              {\object{2}}
              {\x{}}}
             {\replacefor{\filterset {\thenprop {\prop{d}}}
                                     {\elseprop {\prop{d}}}}
                         {\object{2}}
                         {\x{}}}
                       {\replacefor
                         {\object{d}}
                         {\object{2}}
                         {\x{}}}
                       {\val{e}}.

 Now we consider how we choose which method to dispatch to.

 As 
  \getmethod {\disptable{}}
             {\val{e}}
             {\val{l}}
             {\val{g}}, by inversion on \getmethodliteral
             we know
   there exists exactly one \val{k} such that 
   \entryinmap{\mapvalentry{\val{k}}{\val{g}}}{\disptable{}} and \isaopsem{\val{e}}{\val{k}} = {\true{}}.

   By inversion we know T-DefMethod must have extended \disptable{} 
   with the well-typed dispatch value \val{k},
   thus {\judgementtwo{}{\val{k}}{\ty{k}}}, and
   the well-typed method \val{g},
   so {\judgementtwo{}{\val{g}}{\s{t}}}.

  We can also prove that given
        \begin{itemize}
          \item
  \judgementrewrite {\propenv{}} {\val{e}} 
  {\replacefor{\ty{d}}
              {\object{2}}
              {\x{}}}
             {\replacefor{\filterset {\thenprop {\prop{d}}}
                                     {\elseprop {\prop{d}}}}
                         {\object{2}}
                         {\x{}}}
                       {\replacefor
                         {\object{d}}
                         {\object{2}}
                         {\x{}}}
                       {\val{e}}.
    \item
  \judgementtwo {\propenv{}} {\val{k}} {\ty{k}},
                     \item
        \isaopsem{\val{e}}{\val{k}} = {\true{}}, 
      \item
        \satisfies{\openv{}}{\propenv{}},
    \item
  \isacompare{\ty{d}}
                       {\replacefor
                         {\object{d}}
                         {\object{2}}
                         {\x{}}}
  {\ty{k}}{\filterset {\thenprop {\propp{}}} {\elseprop {\propp{}}}},
          \item
        \inpropenv{\thenprop{\propp{}}}{\thenprop{\propp{}}}, and
    \item
        \inpropenv{\elseprop{\propp{}}}{\elseprop{\propp{}}}.
        \end{itemize}
  we can apply \lemref{appendix:lemma:isa} to derive
  then {\satisfies{\openv{}}{\thenprop{\propp{}}}}.

   Now we consider applying the evaluated argument to the chosen method:
  \opsem {\openv{}}
         {\appexp {\val{g}} {\val{2}}}
         {\val{}}.

  By inversion via B-DefMethod we can assume {\val{g}} = {\abs{\x{}}{\s{}}{\e{b}}}, 
  ie. that we have chosen a method to dispatch to that is a closure.

  Because 
  \opsem {\openv{}}
         {\appexp {\val{g}} {\val{2}}}
         {\val{}}
         and
    \judgementtwo{\propenv{}}{\val{2}}{\s{}},
  by inversion via B-BetaClosure we know {\val{}} = {\replacefor{\e{b}}{\val{2}}{\x{}}}.

  With the following premises:
\begin{itemize}
  \item
{\judgementrewrite{{\propenv{}},{\thenprop{\propp{}}}}
                  {\replacefor{\ep{b}}{\val{2}}{\x{}}}
                  {\replacefor{\ty{f}} {\object{2}}{\x{}}}
                  {\replacefor
                   {\filterset {\thenprop {\prop{f}}}
                               {\elseprop {\prop{f}}}}
                             {\object{2}}
                             {\x{}}}
                  {\replacefor
                          {\object{f}}
                             {\object{2}}
                             {\x{}}}
                  {\replacefor{\e{b}}{\val{2}}{\x{}}}
                        },
    \begin{itemize}
      \item From
{\judgementrewrite{{\propenv{}},{\isprop{\s{}}{\x{}}}}
                  {\e{b}}
                  {\ty{f}}
               {\filterset {\thenprop {\prop{f}}}
                                       {\elseprop {\prop{f}}}}
                          {\object{f}}
                          {\e{b}}}
          via the inductive hypothesis on 
  \opsem {\openv{}}
         {\appexp {\abs{\x{}}{\s{}}{\e{b}}} {\val{2}}}
         {\val{}},
      \item then we can derive
{\judgementrewrite{{\propenv{}}}
                  {\replacefor{\ep{b}}{\val{2}}{\x{}}}
                  {\replacefor{\ty{f}} {\object{2}}{\x{}}}
                  {\replacefor
                   {\filterset {\thenprop {\prop{f}}}
                               {\elseprop {\prop{f}}}}
                             {\object{2}}
                             {\x{}}}
                  {\replacefor
                          {\object{f}}
                             {\object{2}}
                             {\x{}}}
                  {\replacefor{\e{b}}{\val{2}}{\x{}}}
                        } via substitution and the fact that {\x{}} is fresh 
                        therefore \x{} $\not\in$ \fv{\propenv{}} so we do not need to substitution for \x{} in \propenv{}. %TODO lemma for this
                        
      \item 
        \satisfies{\openv{}}{\propenv{}, {\thenprop{\propp{}}}}
        because
        \satisfies{\openv{}}{\propenv{}} and {\satisfies{\openv{}}{\thenprop{\propp{}}}} via M-And.
    \end{itemize}
  \item
              \satisfies{\openv{}}{{\propenv{}},{\thenprop{\propp{}}}},
    \begin{itemize}
      \item From \satisfies{\openv{}}{\propenv{}} and \item {\satisfies{\openv{}}{\thenprop{\propp{}}}}  via M-And.
    \end{itemize}
           \item
\isconsistent{\openv{}}, and
           \item 
  \opsem {\openv{}}
         {\replacefor{\e{b}}{\val{2}}{\x{}}}
         {\val{}}.
\end{itemize}

we can apply the inductive hypothesis to satisfy our overall goal for this subcase.
\end{subcase}
     \end{itemize}
\end{case}

\begin{case}[BE-Beta1]
  \ 

  Reduces to an error.
\end{case}
\begin{case}[BE-Beta2]
  \ 

  Reduces to an error.
\end{case}
\begin{case}[BE-BetaClosure]
  \ 

  Reduces to an error.
\end{case}
\begin{case}[BE-BetaMulti1]
  \ 

  Reduces to an error.
\end{case}
\begin{case}[BE-BetaMulti2]
  \ 

  Reduces to an error.
\end{case}
\begin{case}[BE-Delta]
  \ 

  Reduces to an error.
\end{case}

\begin{case}[B-IsA]
        \opsem {\openv{}} {\e{1}} {\val{1}},
        \opsem {\openv{}} {\e{2}} {\val{2}},
        \isaopsem{\val{1}}{\val{2}} = {\val{}}


  \begin{itemize}
    \item[]
      \begin{subcase}[T-IsA]
  \ep{} = {\isaapp {\ep{1}} {\ep{2}}},
  \judgementrewrite {\propenv{}} {\ep{1}} {\ty{1}}
             {\filterset {\thenprop {\prop{1}}}
                         {\elseprop {\prop{1}}}}
                       {\object{1}}
                       {\e{1}},
  \judgementrewrite {\propenv{}} {\ep{2}} {\ty{2}}
             {\filterset {\thenprop {\prop{2}}}
                         {\elseprop {\prop{2}}}}
                       {\object{2}}
                       {\e{2}},
  \e{} = {\isaapp {\e{1}} {\e{2}}},
  \issubtypein{}{\Boolean}{\ty{}},
  \isacompare{\ty{1}}{\object{1}}{\ty{2}}{\filterset {\thenprop {\propp{}}} {\elseprop {\propp{}}}},
  \inpropenv{\thenprop {\propp{}}}{\thenprop {\prop{}}},
  \inpropenv{\elseprop {\propp{}}}{\elseprop {\prop{}}},
  \issubobjin{}{\emptyobject{}}{\object{}}

        Part 1 holds trivially with \object{} = \emptyobject{}.

        For part 2, by the induction hypothesis on \e{1} and \e{2}
        we know
  \judgementrewrite {\propenv{}} {\val{1}} {\ty{1}}
             {\filterset {\thenprop {\prop{1}}}
                         {\elseprop {\prop{1}}}}
                       {\object{1}}
                       {\val{1}} and
  \judgementrewrite {\propenv{}} {\val{2}} {\ty{2}}
             {\filterset {\thenprop {\prop{2}}}
                         {\elseprop {\prop{2}}}}
                       {\object{2}}
                       {\val{2}},
                       so we can then apply
        \lemref{appendix:lemma:isa}
        to reach our goal.

        Part 3 holds because by the definition of \isaopsemliteral
        \val{} can only be \true\ or \false, 
        and since \judgementtwo{\propenv{}}{\true}{\ty{}}
        and
        \judgementtwo{\propenv{}}{\false}{\ty{}}
        we are done.
      \end{subcase}
  \end{itemize}
\end{case}

      \begin{case}[BE-IsA1]
        \opsem {\openv{}} {\e{1}} {\errorvalv{}}

        Trivially reduces to an error.
      \end{case}
      \begin{case}[BE-IsA2]
       \opsem {\openv{}} {\e{1}} {\val{1}},
       \opsem {\openv{}} {\e{2}} {\errorvalv{}}

        Trivially reduces to an error.
      \end{case}

\begin{case}[B-Get]
      $\opsem {\openv{}} {\e{m}}{\val{m}}$,
        $\val{m} = {\curlymap{\overrightarrow{({\val{a}}\ {\val{b}})}}}$,
         \opsem {\openv{}} {\e{k}} {\kw{}},
         $\keyinmap{\kw{}}{\curlymap{\overrightarrow{({\val{a}}\ {\val{b}})}}}$,
         \getmap{\curlymap{\overrightarrow{({\val{a}}\ {\val{b}})}}} {\kw{}} = {\val{}}

  \begin{itemize}
    \item[]
      \begin{subcase}[T-GetHMap]
  \ep{} = {\getexp {\ep{m}} {\ep{k}}},
  \judgementrewrite {\propenv{}} {\ep{m}} {\Unionsplice {\overrightarrow {\HMapgeneric {\mandatory{}} {\absent{}}}}}
           {\filterset {\thenprop {\prop{m}}} {\elseprop {\prop{m}}}}
           {\object{m}}
           {\e{m}},
  \judgementtworewrite {\propenv{}} {\ep{k}} {\Value {k}}{\e{k}},
  \overr{\inmandatory{\kw{}}{\ty{i}}{\mandatory{}},}
  \e{} = {\getexp {\e{m}} {\e{k}}},
  \issubtypein{}{\Unionsplice {\overrightarrow {\ty{i}}}}{\ty{}} ,
  \thenprop{\prop{}} = {\topprop{}},
  \elseprop{\prop{}} = {\topprop{}},
  \issubobjin{}{\replacefor {\pth {\keype{k}} {\x{}}}
                          {\object{m}}
                          {\x{}}}
                        {\object{}}


         To prove part 1 we consider two cases on the form of \object{m}: 
         \begin{itemize}
           \item
         if {\object{m}} = \emptyobject{}
         then \object{} = \emptyobject{} by substitution, which gives the desired result;
           \item
         if \object{m} = {\pth {\pathelem{m}} {\x{m}}}
         then \issubobjin{}{\pth {\keype{k}} {\object{m}}}{\object{}} by substitution.
         We note by the definition of path translation
         {\openv{}}({\pth {\keype{k}} {\object{m}}}) =
         {\getexp {{\openv{}}(\object{m})}{\kw{}}}
         and by the induction hypothesis on \e{m}
         {{\openv{}}(\object{m})} = {\curlymap{\overrightarrow{({\val{a}}\ {\val{b}})}}},
         which together imply 
         \inopenv {\openv{}} {\object{}} {\getexp {\curlymap{\overrightarrow{({\val{a}}\ {\val{b}})}}} {\kw{}}}.
         Since this is the same form as B-Get, we can apply the premise
         \getmap{\curlymap{\overrightarrow{({\val{a}}\ {\val{b}})}}} {\kw{}} = {\val{}}
         to derive \inopenv {\openv{}} {\object{}} {\val{}}.
         \end{itemize}
         
         Part 2 holds trivially as \thenprop{\prop{}} = {\topprop{}}
         and \elseprop{\prop{}} = {\topprop{}}.

         To prove part 3 we note that (by the induction hypothesis on \e{m})
         $\judgementtwo{}{\val{m}}{\Unionsplice{\overrightarrow {\HMapgeneric {\mandatory{}} {\absent{}}}}}$,
         where $\overrightarrow{\inmandatory{\kw{}}{\ty{i}}{\mandatory{}}}$, and 
         both
         $\keyinmap{\kw{}}{\curlymap{\overrightarrow{({\val{a}}\ {\val{b}})}}}$
         and
         \getmap{\curlymap{\overrightarrow{({\val{a}}\ {\val{b}})}}} {\kw{}} = {\val{}}
         imply \judgementtwo{}{\val{}}{\Unionsplice {\overrightarrow {\ty{i}}}}.

      \end{subcase}
    \item[]
      \begin{subcase}[T-GetHMapAbsent]
  \ep{} = {\getexp {\ep{m}} {\ep{k}}},
  \judgementtworewrite {\propenv{}} {\ep{k}} {\Value {k}} {\e{k}},
  \\
  \judgementrewrite {\propenv{}} {\ep{m}} {\HMapgeneric {\mandatory{}} {\absent}}
           {\filterset {\thenprop {\prop{m}}} {\elseprop {\prop{m}}}}
           {\object{m}}
           {\e{m}},
  {\inabsent{\kw{}}{\absent{}}},
  \e{} = {\getexp {\e{m}} {\e{k}}},
  \issubtypein{}{\Nil}{\ty{}},
  \thenprop{\prop{}} = {\topprop{}},
  \elseprop{\prop{}} = {\topprop{}},
  \issubobjin{}{\replacefor
               {\pth {\keype{k}} {\x{}}}
                          {\object{m}}
                          {\x{}}}
                        {\object{}}

       Unreachable subcase because 
         $\keyinmap{\kw{}}{\curlymap{\overrightarrow{({\val{a}}\ {\val{b}})}}}$,
         contradicts
                {\inabsent{\kw{}}{\absent{}}}.
      \end{subcase}
    \item[]
      \begin{subcase}[T-GetHMapPartialDefault]
  \ep{} = {\getexp {\ep{m}} {\ep{k}}},
  \judgementtworewrite {\propenv{}} {\ep{k}} {\Value {k}}{\e{k}},
  \\
 \judgementrewrite {\propenv{}} {\ep{m}} {\HMapp {\mandatory{}} {\absent}}
           {\filterset {\thenprop {\prop{m}}} {\elseprop {\prop{m}}}}
           {\object{m}}
           {\e{m}},
             {\notinmandatory{\kw{}}{\ty{}}{\mandatory{}}},
             {\notinabsent{\kw{}}{\absent{}}},
  \e{} = {\getexp {\e{m}} {\e{k}}},
  \ty{} = \Top,
  \thenprop{\prop{}} = {\topprop{}},
  \elseprop{\prop{}} = {\topprop{}},
  \issubobjin{}{\replacefor
               {\pth {\keype{k}} {\x{}}}
                          {\object{m}}
                          {\x{}}}{\object{}}

         Parts 1 and 2 are the same as the B-Get subcase.
         Part 3 is trivial as \ty{} = \Top.
      \end{subcase}
  \end{itemize}
\end{case}

\begin{case}[B-GetMissing]
        \val{} = \nil,
        $\opsem {\openv{}}
        {\e{m}} {\curlymap{\overrightarrow{({\val{a}}\ {\val{b}})}}}$,
       \opsem {\openv{}} {\e{k}} {\kw{}},
       \keynotinmap{\kw{}}{\curlymap{\overrightarrow{({\val{a}}\ {\val{b}})}}}

  \begin{itemize}
    \item[]
      \begin{subcase}[T-GetHMap]
  \ep{} = {\getexp {\ep{m}} {\ep{k}}},
  \judgementrewrite {\propenv{}} {\ep{m}} {\Unionsplice {\overrightarrow {\HMapgeneric {\mandatory{}} {\absent{}}}}}
           {\filterset {\thenprop {\prop{m}}} {\elseprop {\prop{m}}}}
           {\object{m}}
           {\e{m}},
  \judgementtworewrite {\propenv{}} {\ep{k}} {\Value {k}}{\e{k}},
  \overr{\inmandatory{\kw{}}{\ty{i}}{\mandatory{}},}
  \e{} = {\getexp {\e{m}} {\e{k}}},
  \issubtypein{}{\Unionsplice {\overrightarrow {\ty{i}}}}{\ty{}},
  \thenprop{\prop{}} = {\topprop{}},
  \elseprop{\prop{}} = {\topprop{}},
  \issubobjin{}{\replacefor {\pth {\keype{k}} {\x{}}}
                          {\object{m}}
                          {\x{}}}{\object{}}

       Unreachable subcase because 
       \keynotinmap{\kw{}}{\curlymap{\overrightarrow{({\val{a}}\ {\val{b}})}}}
       contradicts ${\inmandatory{\kw{}}{\ty{}}{\mandatory{}}}$.
      \end{subcase}
    \item[]
      \begin{subcase}[T-GetHMapAbsent]
  \ep{} = {\getexp {\ep{m}} {\ep{k}}},
  \judgementtworewrite {\propenv{}} {\ep{k}} {\Value {k}} {\e{k}},
  \\
  \judgementrewrite {\propenv{}} {\ep{m}} {\HMapgeneric {\mandatory{}} {\absent}}
           {\filterset {\thenprop {\prop{m}}} {\elseprop {\prop{m}}}}
           {\object{m}}
           {\e{m}},
  {\inabsent{\kw{}}{\absent{}}},
  \e{} = {\getexp {\e{m}} {\e{k}}},
  \issubtypein{}{\Nil}{\ty{}},
  \thenprop{\prop{}} = {\topprop{}},
  \elseprop{\prop{}} = {\topprop{}},
  \issubobjin{}{\replacefor
               {\pth {\keype{k}} {\x{}}}
                          {\object{m}}
                          {\x{}}}{\object{}}

         To prove part 1 we consider two cases on the form of \object{m}: 
         \begin{itemize}
           \item
         if {\object{m}} = \emptyobject{}
         then \object{} = \emptyobject{} by substitution, which gives the desired result;
           \item
         if \object{m} = {\pth {\pathelem{m}} {\x{m}}}
         then \issubobjin{}{\pth {\keype{k}} {\object{m}}}{\object{}} by substitution.
         We note by the definition of path translation
         {\openv{}}({\pth {\keype{k}} {\object{m}}}) =
         {\getexp {{\openv{}}(\object{m})}{\kw{}}}
         and by the induction hypothesis on \e{m}
         {{\openv{}}(\object{m})} = {\curlymap{\overrightarrow{({\val{a}}\ {\val{b}})}}},
         which together imply 
         \inopenv {\openv{}} {\object{}} {\getexp {\curlymap{\overrightarrow{({\val{a}}\ {\val{b}})}}} {\kw{}}}.
         Since this is the same form as B-GetMissing, we can apply the premise
        \val{} = \nil\ 
         to derive \inopenv {\openv{}} {\object{}} {\val{}}.
         \end{itemize}
         
         Part 2 holds trivially as \thenprop{\prop{}} = {\topprop{}}
         and \elseprop{\prop{}} = {\topprop{}}.

         To prove part 3 we note that \e{m} has type {\HMapgeneric {\mandatory{}} {\absent{}}}
         where {\inabsent{\kw{}}{\absent{}}}, and
         the premises of B-GetMissing
         \keynotinmap{\kw{}}{\curlymap{\overrightarrow{({\val{a}}\ {\val{b}})}}}
         and
          \val{} = \nil\ 
         tell us {\val{}} must be of type {\ty{}}.
      \end{subcase}
    \item[]
      \begin{subcase}[T-GetHMapPartialDefault]
  \ep{} = {\getexp {\ep{m}} {\ep{k}}},
  \judgementtworewrite {\propenv{}} {\ep{k}} {\Value {k}}{\e{k}},
  \\
 \judgementrewrite {\propenv{}} {\ep{m}} {\HMapp {\mandatory{}} {\absent}}
           {\filterset {\thenprop {\prop{m}}} {\elseprop {\prop{m}}}}
           {\object{m}}
           {\e{m}},
             {\notinmandatory{\kw{}}{\ty{}}{\mandatory{}}},
             {\notinabsent{\kw{}}{\absent{}}},
  \e{} = {\getexp {\e{m}} {\e{k}}},
  \ty{} = \Top,
  \thenprop{\prop{}} = {\topprop{}},
  \elseprop{\prop{}} = {\topprop{}},
  \issubobjin{}{\replacefor
               {\pth {\keype{k}} {\x{}}}
                          {\object{m}}
                          {\x{}}}{\object{}}

         Parts 1 and 2 are the same as the B-GetMissing subcase of T-GetHMapAbsent.
         Part 3 is trivial, since \ty{} = \Top.
      \end{subcase}
  \end{itemize}
\end{case}

\begin{case}[BE-Get1]
  \ 

  Reduces to an error.
\end{case}

\begin{case}[BE-Get2]
  \ 

  Reduces to an error.
\end{case}

\begin{case}[B-Assoc]
        \val{} = 
        {\extendmap{\curlymap{\overrightarrow{({\val{a}}\ {\val{b}})}}}
                {\kw{}}{\val{v}}},
        \opsem {\openv{}}
        {\e{m}} {\curlymap{\overrightarrow{({\val{a}}\ {\val{b}})}}},
        \opsem {\openv{}} {\e{k}} {\kw{}},
        \opsem {\openv{}} {\e{v}} {\val{v}}

  \begin{itemize}
    \item[]
      \begin{subcase}[T-AssocHMap]
  \judgementtworewrite {\propenv{}} {\ep{m}} {\HMapgeneric {\mandatory{}} {\absent}} {\e{m}},
  \judgementtworewrite {\propenv{}} {\ep{k}} {\Value{\kw{}}}{\e{k}},\\
  \judgementtworewrite {\propenv{}} {\ep{v}} {\ty{}}{\e{v}},
  {\kw{}} $\not\in$ {\absent{}},
  \ep{} = {\assocexp {\ep{m}} {\ep{k}} {\ep{v}}},\\
  \e{} = {\assocexp {\e{m}} {\e{k}} {\e{v}}},
  \issubtypein{}{\HMapgeneric {\extendmandatoryset {\mandatory{}}{\kw{}}{\ty{}}} {\absent}}{\ty{}},
  \thenprop{\prop{}} = {\topprop{}},
  \elseprop{\prop{}} = {\botprop{}},
  \object{} = \emptyobject{}

        Parts 1 and 2 hold for the same reasons as T-True.
        %TODO part 3
      \end{subcase}
  \end{itemize}
\end{case}

\begin{case}[BE-Assoc1]
  \ 

  Reduces to an error.
\end{case}

\begin{case}[BE-Assoc2]
  \ 

  Reduces to an error.
\end{case}

\begin{case}[BE-Assoc3]
  \ 

  Reduces to an error.
\end{case}

\begin{case}[B-IfFalse]
        \opsem {\openv{}} {\e{1}} {\false}
        \ \ \text{or}\ \ 
        \opsem {\openv{}} {\e{1}} {\nil},
        \opsem {\openv{}} {\e{3}} {\val{}}

  \begin{itemize}
    \item[]
      \begin{subcase}[T-If]
        \ep{} = {\ifexp {\ep{1}} {\ep{2}} {\ep{3}}},
        \judgementrewrite {\propenv{}} {\ep{1}} {\ty{1}} {\filterset {\thenprop {\prop{1}}} {\elseprop {\prop{1}}}}
                 {\object{1}}
                 {\e{1}},
  \judgementrewrite {\propenv{}, {\thenprop {\prop{1}}}}
                 {\ep{2}} {\ty{}} {\filterset {\thenprop {\prop{2}}} {\elseprop {\prop{2}}}}
                 {\object{}}
                 {\e{2}},
  \judgementrewrite {\propenv{}, {\elseprop {\prop{1}}}}
                 {\ep{3}} {\ty{}} {\filterset {\thenprop {\prop{3}}} {\elseprop {\prop{3}}}}
                 {\object{}}
                 {\e{3}},
        \e{} = {\ifexp {\e{1}} {\e{2}} {\e{3}}},
  \inpropenv{\orprop {\thenprop {\prop{2}}} {\thenprop {\prop{3}}}}{\thenprop{\prop{}}},
  \inpropenv{\orprop {\elseprop {\prop{2}}} {\elseprop {\prop{3}}}}{\elseprop{\prop{}}}

              For part 1, either \object{} = \emptyobject{}, or \e{} evaluates to the
              result of \e{3}.

              To prove part 2, we consider two cases:
              \begin{itemize}
                \item if \isfalseval{\val{}}
                  then \e{3} evaluates to a false value so {\satisfies{\openv{}}{\elseprop {\prop{3}}}}, and thus
                  {\satisfies{\openv{}}{\orprop {\elseprop {\prop{2}}} {\elseprop {\prop{3}}}}} by M-Or, 
                \item otherwise
                  \istrueval{\val{}},
                  so \e{3} evaluates to a true value so {\satisfies{\openv{}}{\thenprop {\prop{3}}}}, and thus
                  {\satisfies{\openv{}}{\orprop {\thenprop {\prop{2}}} {\thenprop {\prop{3}}}}} by M-Or.
              \end{itemize}

              Part 3 is trivial as
              \opsem {\openv{}} {\e{3}} {\val{}}
              and {\judgementtwo{}{\val{}}{\ty{}}} by the induction hypothesis on {\e{3}}.
      \end{subcase}
  \end{itemize}
\end{case}

\begin{case}[B-IfTrue]
        \opsem {\openv{}} {\e{1}} {\val{1}},
              ${\val{1}} \not= {\false}$,
              ${\val{1}} \not= {\nil}$,
              \opsem {\openv{}} {\e{2}} {\val{}}

  \begin{itemize}
    \item[]
      \begin{subcase}[T-If]
        \ep{} = {\ifexp {\ep{1}} {\ep{2}} {\ep{3}}},
        \judgementrewrite {\propenv{}} {\ep{1}} {\ty{1}} {\filterset {\thenprop {\prop{1}}} {\elseprop {\prop{1}}}}
                 {\object{1}}
                 {\e{1}},
  \judgementrewrite {\propenv{}, {\thenprop {\prop{1}}}}
                 {\ep{2}} {\ty{}} {\filterset {\thenprop {\prop{2}}} {\elseprop {\prop{2}}}}
                 {\object{}}
                 {\e{2}},
  \judgementrewrite {\propenv{}, {\elseprop {\prop{1}}}}
                 {\ep{3}} {\ty{}} {\filterset {\thenprop {\prop{3}}} {\elseprop {\prop{3}}}}
                 {\object{}}
                 {\e{3}},
        \e{} = {\ifexp {\e{1}} {\e{2}} {\e{3}}},
  \inpropenv{\orprop {\thenprop {\prop{2}}} {\thenprop {\prop{3}}}}{\thenprop{\prop{}}},
  \inpropenv{\orprop {\elseprop {\prop{2}}} {\elseprop {\prop{3}}}}{\elseprop{\prop{}}}

              For part 1, either \object{} = \emptyobject{}, or \e{} evaluates to the
              result of \e{2}.

              To prove part 2, we consider two cases:
              \begin{itemize}
                \item if \isfalseval{\val{}}
                  then \e{2} evaluates to a false value so {\satisfies{\openv{}}{\elseprop {\prop{2}}}}, and thus
                  {\satisfies{\openv{}}{\orprop {\elseprop {\prop{2}}} {\elseprop {\prop{3}}}}} by M-Or, 
                \item otherwise
                  \istrueval{\val{}},
                  so \e{2} evaluates to a true value so {\satisfies{\openv{}}{\thenprop {\prop{2}}}}, and thus
                  {\satisfies{\openv{}}{\orprop {\thenprop {\prop{2}}} {\thenprop {\prop{3}}}}} by M-Or.
              \end{itemize}

              Part 3 is trivial as
              \opsem {\openv{}} {\e{2}} {\val{}}
              and {\judgementtwo{}{\val{}}{\ty{}}} by the induction hypothesis on {\e{2}}.

      \end{subcase}
  \end{itemize}
\end{case}

\begin{case}[BE-If]
  \ 

  Reduces to an error.
\end{case}

\begin{case}[BE-IfFalse]
  \ 

  Reduces to an error.
\end{case}

\begin{case}[BE-IfTrue]
  \ 

  Reduces to an error.
\end{case}

\begin{case}[B-Let]
  \e{} = {\letexp {\x{}} {\e{1}} {\e{2}}},
        \opsem {\openv{}} {\e{1}} {\val{1}},
        \opsem {\extendopenv{\openv{}}{\x{}}{\val{1}}} {\e{2}} {\val{}}


  \begin{itemize}
    \item[]
      \begin{subcase}[T-Let]
  \ep{} = {\letexp {\x{}} {\ep{1}} {\ep{2}}},
  \judgementrewrite {\propenv{}} {\ep{1}} {\s{}} {\filterset {\thenprop {\prop{1}}} {\elseprop {\prop{1}}}}
             {\object{1}}
             {\ep{1}},
             \propp{} = {\impprop {\notprop {\falsy{}} {\x{}}} {\thenprop {\prop{1}}}},
             \proppp{} = {\impprop {\isprop {\falsy{}} {\x{}}} {\elseprop {\prop{1}}}},
  \judgementrewrite
       {\propenv{}, {\isprop {\s{}} {\x{}}},
         {\propp{}},
         {\proppp{}}}
             {\ep{2}} {\ty{}} {\filterset {\thenprop {\prop{}}} {\elseprop {\prop{}}}}
             {\object{}} 
             {\e{2}}

        For all the following cases (with a reminder that \x{} is fresh)
        we apply the induction hypothesis on \e{2}. We justify this by noting
        that occurrences of \x{} inside \e{2} have the same type as \e{1} and 
        simulate the propositions of \e{1}
        because 
        \begin{itemize}
          \item
  \judgementrewrite
       {\propenv{}, {\isprop {\s{}} {\x{}}},
         {\propp{}},
         {\proppp{}}}
             {\ep{2}} {\ty{}} {\filterset {\thenprop {\prop{}}} {\elseprop {\prop{}}}}
             {\object{}} 
             {\e{2}},
           \item
        \satisfies{\extendopenv{\openv{}}{\x{}}{\val{1}}}{\propenv{}, {\isprop {\s{}} {\x{}}}, \propp{}, \proppp{}},
           \item
        {\isconsistent{\extendopenv{\openv{}}{\x{}}{\val{1}}}},
        and
           \item
        \opsem {\extendopenv{\openv{}}{\x{}}{\val{1}}} {\e{2}} {\val{}}.
    \end{itemize}

        We prove parts 1, 2 and 3 by directly using the induction hypothesis on \e{2}.
      \end{subcase}
  \end{itemize}
\end{case}

\begin{case}[BE-Let]
  \ 
  
  Reduces to an error.
\end{case}

\begin{case}[B-Abs] 
        \val{} = {\closure {\openv{}} {\abs {\x{}} {\s{}} {\e{1}}}}

  \begin{itemize}
    \item[]
      \begin{subcase}[T-Clos]
  \ep{} = {\closure {\openv{}} {\abs {\x{}} {\s{}} {\e{1}}}},
  $\exists {\propenvp{}}. \satisfies{\openv{}}{\propenvp{}}$
  \ \text{and}\ 
\judgementrewrite {\propenvp{}} {\abs {\x{}} {\s{}} {\e{1}}} {\ty{}}
                 {\filterset {\thenprop {\prop{f}}}
                             {\elseprop {\prop{f}}}}
                 {\object{f}}
                 {\abs {\x{}} {\s{}} {\e{1}}},
  \e{} = {\closure {\openv{}} {\abs {\x{}} {\s{}} {\e{1}}}},
                 {\thenprop{\prop{}}} = \topprop{},
                 {\elseprop{\prop{}}} = \botprop{},
                 {\object{}} = \emptyobject{}

        We assume some \propenvp{}, such that
        \begin{itemize}
          \item \satisfies{\openv{}}{\propenvp{}}
          \item \judgement {\propenvp{}} {\abs {\x{}} {\s{}} {\e{1}}} {\ty{}}
                           {\filterset {\thenprop {\prop{}}}
                                       {\elseprop {\prop{}}}}
                           {\object{}}.
       \end{itemize}
       Note the last rule in the derivation of
          \judgement {\propenvp{}} {\abs {\x{}} {\s{}} {\e{1}}} {\ty{}}
                           {\filterset {\thenprop {\prop{}}}
                                       {\elseprop {\prop{}}}}
                           {\object{}}
                           must be T-Abs, so 
                           {\thenprop {\prop{}}} = {\topprop{}},
                           {\elseprop {\prop{}}} = {\botprop{}}
                           and {\object{}} = {\emptyobject{}}.
         Thus parts 1 and 2 hold for the same reasons as T-True.
         Part 3 holds as \val{} has the same type as {\abs {\x{}} {\s{}} {\e{1}}}
         under \propenvp{}.

      \end{subcase} 
  \end{itemize}
\end{case}

\begin{case}[B-Abs]
        \val{} = ${\closure {\openv{}} {\abs {\x{}} {\s{}} {\e{1}}}}$,
          { \opsem {\openv{}}
                   {\abs {\x{}} {\ty{}} {\e{1}}}
                   {\closure {\openv{}} {\abs {\x{}} {\s{}} {\e{1}}}}}

  \begin{itemize}
    \item[]
      %TODO
      \begin{subcase}[T-Abs]
  \ep{} = {\abs {\x{}} {\s{}} {\ep{1}}},
{ \judgementrewrite {\propenv{}, {\isprop {\s{}} {\x{}}}}
            {\ep{1}} {\ty{}}
             {\filterset {\thenprop {\prop{1}}}
                         {\elseprop {\prop{1}}}}
             {\object{1}}
             {\e{1}}},
           \issubtypein{}
           {\ArrowOne {\x{}} {\s{}}
                      {\ty{1}}
                      {\filterset {\thenprop {\prop{1}}}
                                  {\elseprop {\prop{1}}}}
                      {\object{1}}}
          {\ty{}},
          \inpropenv{\topprop{}}{\thenprop{\prop{}}},
          \inpropenv{\botprop{}}{\elseprop{\prop{}}},
          {\object{}} = {\emptyobject{}}

        Parts 1 and 2 hold for the same reasons as T-True.
        Part 3 holds directly via T-Clos, since \val{} must be a closure.
      \end{subcase}
  \end{itemize}
\end{case}

\begin{case}[BE-Error]
        \opsem {\openv{}} {\e{}} {\errorval{\val{1}}}


  \begin{itemize}
    \item[]
      \begin{subcase}[T-Error] 
  \ep{} = \errorval{\val{1}},
  \e{} = \errorval{\val{1}},
  \ty{} = \Bot,
  \thenprop{\prop{}} = \botprop{}, \elseprop{\prop{}} = \botprop{}, \object{} = \emptyobject{}

        Trivially reduces to an error.
      \end{subcase}
  \end{itemize}
\end{case}

\end{proof}

\end{lemma}

{\wrongtheorem{appendix}}

{\soundnesstheorem{appendix}}

 % include for newpage
%
%% Force CV to appear in TOC, but with no page number
%\addtocontents{toc}{\bigskip Curriculum Vitae\par}
%
%\includepdf[pages=-]{cv/ambrosebs-cv.pdf}


\end{document}
