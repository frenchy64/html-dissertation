\Dchapter{Overview}
\label{infer:sec:overview}

%\input{infer-old-pldi-overview}

We demonstrate our approach by synthesizing a Typed Clojure annotation for \clj{nodes}.
The following presentation is somewhat loose to keep from being bogged down by details---interested
readers may follow the pointers to subsequent sections where they are made precise.

We use dynamic analysis to observe the execution of functions, so we give an
explicit test suite for \clj{nodes}.

\begin{lstlisting}[language=Clojure]
(def t1 {:op :node, :left {:op :leaf, :val 1}, :right {:op :leaf, :val 2}})
(deftest nodes-test (is (= (nodes t1) 3)))
\end{lstlisting}

The first step is the instrumentation phase
(formalized in \secref{infer:sec:formal:collection-phase}), which 
monitors the inputs and outputs of \clj{nodes}
by redefining it to use the \clj{track} function like so (where \clj{<nodes-body>} begins the \clj{case} expression of
the original \clj{nodes} definition):

\begin{lstlisting}[language=Clojure]
(def nodes (fn [t'] (track ((fn [t] <nodes-body>) (track t' ['nodes :dom]))
                           ['nodes :rng])))
\end{lstlisting}

The \clj{track} function (given later in \figref{infer:fig:trackmeta})
takes a value to track and a
\emph{path} that represents its origin, and returns an instrumented value
along with recording some runtime samples about the value.
A path is represented as a vector of \emph{path elements},
and describes the source of the value in question.
For example, \clj{(track 3 ['nodes :rng])}
returns \clj{3} and records the sample
\resentry{\text{\clj{['nodes :rng]}}}{\text{\clj{Int}}}
which says ``\clj{Int} was recorded at \clj{nodes}'s range.''
%
Running our test suite \clj{nodes-test} with an instrumented \clj{nodes}
results in more samples like this, most which use the path element
\clj{\{:key :kw\}} which represents a map lookup on the \clj{:kw} entry.

\inferrule[]
{}
{
\resentry{\cljmath{['nodes :dom \{:key :op\}]}}{\cljmath{':leaf}} \\
\resentry{\cljmath{['nodes :dom \{:key :op\}]}}{\cljmath{':node}} \\
\resentry{\cljmath{['nodes :dom \{:key :val\}]}}{\cljmath{?}} \\
\resentry{\cljmath{['nodes :dom \{:key :left\}]}}{\cljmath{?}}\\
\resentry{\cljmath{['nodes :dom \{:key :right\}]}}{\cljmath{?}}\\
\resentry{\cljmath{['nodes :dom \{:key :left\} \{:key :op\}]}}{\cljmath{':leaf}}\\
\resentry{\cljmath{['nodes :dom \{:key :right\} \{:key :op\}]}}{\cljmath{':leaf}}\\
\resentry{\cljmath{['nodes :dom \{:key :left\} \{:key :val\}]}}{\cljmath{Int}}\\
\resentry{\cljmath{['nodes :dom \{:key :right\} \{:key :val\}]}}{\cljmath{Int}}
%\\ \resentry{\cljmath{['nodes :rng]}}{\cljmath{Int}}
}

Now, our task is to transform these samples into a readable and useful annotation.
This is the function of the inference phase (formalized in \secref{infer:sec:formal:inference-phase}),
which is split into three passes: first it generates a naive type from samples, then it
combines types that occur syntactically near eachother (``squash locally''),
and then aggressively across different function annotations (``squash globally'').

The initial naive type generated from these samples resembles TypeWiz's
``verbatim'' annotation given in \Dchapref{infer:chapter:intro}, except
the \clj{?} placeholder represents incomplete information about a path
(this process is formalized as \generatetenv{} in \figref{infer:fig:generatetenv}).

\begin{lstlisting}[language=Clojure]
(ann nodes [(U '{:op ':leaf, :val ?} '{:op ':node,
                                       :left '{:op ':leaf, :val Int},
                                       :right '{:op ':leaf, :val Int}}) -> Int])
\end{lstlisting}

Next, the two ``squashing'' phases.
The intuition behind both are based on seeing types as directed graphs,
where vertices are type aliases, and an edge connects
two vertices $u$ and $v$ if $u$ is mentioned in $v$'s type.

Local squashing (\squashlocal{} in \figref{infer:fig:squashlocal})
constructs such a graph by creating type aliases from map types
using a post-order traversal of the original types.
In this example, the previous annotations become:

\begin{lstlisting}[language=Clojure]
(defalias op-leaf1 '{:op ':leaf, :val ?})
(defalias op-leaf2 '{:op ':leaf, :val Int})
(defalias op-leaf3 '{:op ':leaf, :val Int})
(defalias op-node '{:op ':node, :left op-leaf2, :right op-leaf3})
(ann nodes [(U op-leaf1 op-node) -> Int])
\end{lstlisting}

As a graph, this becomes the left-most graph below. The dotted edge
from \clj{op-leaf2} to \clj{op-leaf1} signifies that they are to be merged,
based on the similar structure of the types they point to.

\begin{tikzpicture}[
        > = stealth, % arrow head style
        shorten > = 1pt, % don't touch arrow head to node
        auto,
        %node distance = 1.5cm, % distance between nodes
        semithick % line style
    ]
  \tikzstyle{invalias}=[
      draw = none,
      thick,
      fill = none,
      rounded rectangle,
      text opacity=0
  ]
  \tikzstyle{alias}=[
      draw = black,
      thick,
      fill = white,
      rounded rectangle,
  ]
  \tikzstyle{ann}=[
      draw = gray,
      thick,
      fill = white,
      rectangle,
  ]

  \begin{scope}

    \node[ann]   (nodes) {\clj{nodes}};
    \node[alias] (leaf1) [right = 0.7cm of nodes] {\clj{op-leaf1}};
    \node[alias] (leaf2) [below = 0.1cm of leaf1] {\clj{op-leaf2}};
    \node[alias] (leaf3) [below = 0.1cm of leaf2] {\clj{op-leaf3}};
    \node[alias] (op-node) [below of=nodes]       {\clj{op-node}};

    \path[->] (nodes) edge node {} (leaf1);
    \path[->] (nodes) edge node {} (op-node);
    \path[->] (op-node) edge node  {} (leaf2.west);
    \path[->] (op-node) edge node  {} (leaf3.west);

    \draw[->,dotted,thick,red] (leaf2.east) to [bend right] (leaf1.east);
  \end{scope}

  \begin{scope}[xshift=4.5cm]
    \node[ann]   (nodes) {\clj{nodes}};
    \node[alias] (leaf1) [right = 0.7cm of nodes] {\clj{op-leaf1}};
    \node[invalias] (leaf2) [below = 0.1cm of leaf1] {\clj{op-leaf2}};
    \node[alias] (leaf3) [below = 0.1cm of leaf2] {\clj{op-leaf3}};
    \node[alias] (op-node) [below of=nodes]       {\clj{op-node}};

    \path[->] (nodes) edge node {} (leaf1);
    \path[->] (nodes) edge node {} (op-node);
    \path[->] (op-node) edge node  {} (leaf1);
    \path[->] (op-node) edge node  {} (leaf3);

    \draw[->,dotted,thick,red] (leaf3.east) to [bend right] (leaf1.east);
  \end{scope}

  \begin{scope}[xshift=9cm]
    \node[ann]   (nodes) {\clj{nodes}};
    \node[alias] (leaf1) [right = 0.7cm of nodes] {\clj{op-leaf1}};
    \node[invalias] (leaf2) [below = 0.1cm of leaf1] {\clj{op-leaf2}};
    \node[invalias] (leaf3) [below = 0.1cm of leaf2] {\clj{op-leaf3}};
    \node[alias] (op-node) [below of=nodes]       {\clj{op-node}};

    \path[->] (nodes) edge node {} (leaf1);
    \path[->] (nodes) edge node {} (op-node);
    \path[->] (op-node) edge node  {} (leaf1);
  \end{scope}
\end{tikzpicture}

After several merges (reading the graphs left-to-right), local squashing results in the following:

\begin{lstlisting}[language=Clojure]
(defalias op-leaf '{:op ':leaf, :val Int})
(defalias op-node '{:op ':node, :left op-leaf, :right op-leaf})
(ann nodes [(U op-leaf op-node) -> Int])
\end{lstlisting}

All three duplications of the \clj{':leaf} type in the naive annotation have
been consolidated into their own name,
with the \clj{?} placeholder for the \clj{:val} entry being absorbed into \clj{Int}.

Now, the global squashing phase (\squashglobal{} in \figref{infer:fig:squashglobal})
proceeds similarly, except the notion of a vertex is expanded to also include
\emph{unions} of map types, calculated, again, with a post-order traversal of the types
giving:

\begin{lstlisting}[language=Clojure]
(defalias op-leaf '{:op ':leaf, :val Int})
(defalias op-node '{:op ':node, :left op-leaf, :right op-leaf})
(defalias op-leaf-node (U op-leaf op-node))
(ann nodes [op-leaf-node -> Int])
\end{lstlisting}

This creates \clj{op-leaf-node}, giving the left-most graph below.

\begin{tikzpicture}[
        > = stealth, % arrow head style
        shorten > = 1pt, % don't touch arrow head to node
        auto,
        %node distance = 1.5cm, % distance between nodes
        semithick % line style
    ]
  \tikzstyle{invalias}=[
      draw = none,
      thick,
      fill = none,
      rounded rectangle,
      text opacity=0
  ]
  \tikzstyle{alias}=[
      draw = black,
      thick,
      fill = white,
      rounded rectangle,
  ]
  \tikzstyle{ann}=[
      draw = gray,
      thick,
      fill = white,
      rectangle,
  ]

  \begin{scope}
    \node[ann]   (nodes) {\clj{nodes}};
    \node[alias] (op-leaf-node) [below = 0.7cm of nodes] {\clj{op-leaf-node}};
    \node[alias] (op-leaf) [right = 0.7cm of nodes] {\clj{op-leaf}};
    \node[alias] (op-node) [right = 0.5cm of op-leaf-node] {\clj{op-node}};

    \path[->] (nodes) edge node {} (op-leaf-node);
    \path[->] (op-leaf-node) edge node  {} (op-leaf);
    \path[->] (op-leaf-node) edge node  {} (op-node);
    \path[->] (op-node) edge node  {} (op-leaf);

    \draw[->,dotted,thick,red] (op-leaf-node) to [bend right] (op-leaf);
  \end{scope}

  \begin{scope}[xshift=5cm]
    \node[ann]   (nodes) {\clj{nodes}};
    \node[alias] (op-leaf-node) [below = 0.7cm of nodes] {\clj{op-leaf-node}};
    \node[invalias] (op-leaf) [right = 0.7cm of nodes] {\clj{op-leaf}};
    \node[alias] (op-node) [right = 0.5cm of op-leaf-node] {\clj{op-node}};

    \path[->] (nodes) edge node {} (op-leaf-node);
    \draw[->] (op-leaf-node) to [bend right=20] (op-node);
    \draw[->] (op-node) to [bend right=20] (op-leaf-node);

    \draw[->,dotted,thick,red] (op-node) to (op-leaf-node);
  \end{scope}

  \begin{scope}[xshift=10cm]
    \node[ann]   (nodes) {\clj{nodes}};
    \node[alias] (op-leaf-node) [below = 0.7cm of nodes] {\clj{op-leaf-node}};

    \path[->] (nodes) edge node {} (op-leaf-node);
    \path[->] (op-leaf-node.east) edge [loop right] node {} (op-leaf-node.east);
  \end{scope}
\end{tikzpicture}

Now, type aliases are merged based on overlapping \emph{sets} of top-level keysets and likely tags.
Since \clj{op-leaf} and \clj{op-leaf-node} refer to maps with identical keysets
(\clj{:op} and \clj{:val}) and whose likely tags agree (the \clj{:op} entry
is probably a tag, and they are both \clj{':leaf}),
they are merged and all occurrences of \clj{op-leaf}
are renamed to \clj{op-leaf-node}, creating a \emph{mutually recursive type}
between the remaining aliases in the middle graph:

\begin{lstlisting}[language=Clojure]
(defalias op-node '{:op ':node, :left op-leaf-node, :right op-leaf-node})
(defalias op-leaf-node (U '{:op ':leaf, :val Int} op-node))
(ann nodes [op-leaf-node -> Int])
\end{lstlisting}

In the right-most graph, the aliases \clj{op-node} and \clj{op-leaf-node} are merged for similar reasons:

\begin{lstlisting}[language=Clojure]
(defalias op-leaf-node
  (U '{:op ':leaf, :val Int}
     '{:op ':node, :left op-leaf-node, :right op-leaf-node}))
(ann nodes [op-leaf-node -> Int])
\end{lstlisting}

All that remains is to choose a recognizable name for the alias.
Since all its top-level types seem to use the \clj{:op} entry for
tags, we choose the name \clj{Op} and output the final annotation:

\begin{lstlisting}[language=Clojure]
(defalias Op (U '{:op ':leaf, :val Int}
                '{:op ':node, :left Op, :right Op}))
(ann nodes [Op -> Int])
\end{lstlisting}

The rest of the porting workflow involves the programmer repeatedly type checking
their code and gradually tweaking the generated annotations until they type check.
It turns out that this annotation immediately type checks the definition of \clj{nodes}
and all its valid usages, so we turn to a more complicated function \clj{visit-leaf}
to demonstrate a typical scenario.

\begin{lstlisting}[language=Clojure]
(defn visit-leaf "Updates :leaf nodes in tree t with function f."
  [f t] (case (:op t)
          :node (assoc t :left (visit-leaf f (:left t))
                         :right (visit-leaf f (:right t)))
          :leaf (f t)))
\end{lstlisting}

This higher-order function uses \clj{assoc} to associate new children
as it recurses down a given tree to update leaf nodes with the provided function.
The following test simply increments the leaf values of the previously-defined \clj{t1}.

\begin{lstlisting}[language=Clojure]
(deftest visit-leaf-test 
  (is (= (visit-leaf (fn [leaf] (assoc leaf :val (inc (:val leaf)))) t1)
         {:op :node, :left {:op :leaf, :val 2}, :right {:op :leaf, :val 3}})))
\end{lstlisting}

Running this test under instrumentation yields some interesting runtime samples
whose calculation is made efficient by space-efficient tracking (\secref{infer:sec:space-efficient-tracking}),
which ensures a function is not repeatedly tracked unnecessarily.
The following two samples demonstrate how to handle multiple arguments
(by parameterizing the \clj{:dom} path element)
and higher-order functions
(by nesting \clj{:dom} or \clj{:rng} path elements).

\inferrule[]
{}
{
\resentry{\cljmath{['visit-leaf \{:dom 1\} \{:key :op\}]}}{\cljmath{':leaf}}\\
\resentry{\cljmath{['visit-leaf \{:dom 0\} \{:dom 0\} \{:key :op\}]}}{\cljmath{':leaf}}
}

Here is our automatically generated initial annotation.

% initial annotation
\begin{lstlisting}[language=Clojure]
(defalias Op (U '{:op ':leaf, :val t/Int} '{:op ':node, :left Op, :right Op}))
(ann visit-leaf [[Op -> Any] Op -> Any])
\end{lstlisting}

Notice the surprising occurrences of \clj{Any}. They originate
from \clj{?} placeholders due to the lazy tracking of maps 
(\secref{infer:sec:lazy-tracking}).
Since \clj{visit-leaf} does not traverse the results of \clj{f},
nor does anything traverse \clj{visit-leaf}'s results (hash-codes are used for equality checking)
neither tracking is realized.
Also notice the first argument of \clj{visit-leaf} is underprecise.
These could trigger type errors on usages of \clj{visit-leaf},
so manual intervention is needed (highlighted). We factor out and use a new alias \clj{Leaf}
and replace occurrences of \clj{Any} with \clj{Op}.

\begin{lstlisting}[language=Clojure]
(*@\colorbox{pink}{(defalias Leaf}@*) '{:op ':leaf, :val Int}(*@\colorbox{pink}{)}@*)
(defalias Op (U (*@\colorbox{pink}{Leaf}@*) '{:op ':node, :left Op, :right Op}))
(ann visit-leaf [[(*@\colorbox{pink}{Leaf}@*) -> (*@\colorbox{pink}{Op}@*)] Op -> (*@\colorbox{pink}{Op}@*)])
\end{lstlisting}

% TODO Point to experiments
We measure the success of our workflow by using it to type check real Clojure programs.
Experiment 1 (\secref{infer:sec:experiment1}) manually inspects a selection of inferred types.
Experiment 2 (\secref{infer:sec:experiment2}) classifies and quantifies the kinds of changes needed.
Experiment 3 (\secref{infer:sec:experiment3}) enforces initial annotations at runtime to ensure
they are meaningfully underprecise.
