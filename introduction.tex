\section{My Thesis}

\emph{Typed Clojure is a sound and practical optional type system for Clojure.}

\section{Structure of this Dissertation}

This document progresses in several parts that support my thesis statement.

\partref{part:types} motivates and presents the design of Typed Clojure.
It addresses both parts of my thesis statement.

\begin{itemize}
  \item \emph{Typed Clojure is sound} I formalize Typed Clojure, including
    its characteristic features like hash-maps, multimethods, and Java interoperability,
    and prove the model type sound.
  \item \emph{Typed Clojure is practical} 
      I present an empirical study of real-world Typed Clojure usage
        in over 19,000 lines of code, showing its features correspond to actual usage patterns.
\end{itemize}

The results and industry feedback of this work inspired three distinct research directions
to help improve the experience of using Typed Clojure.

\begin{itemize}
  \item
\partref{part:autoann} presents a solution to lower the annotation burden in real-world Typed Clojure programs.
I formalize and implement a tool to automatically annotate types for top-level
user and library definitions, and empirically study the manual changes needed for the generated annotations
to pass type checking.
  \item
\partref{part:implementations} describes the design and implementation of a 
new code analyzer for Clojure, in service of enabling user-provided type rules for Clojure macros
    to help make type checking complex macro usages more robust.
\item \partref{part:symbolic-closures} motivates and describes \emph{symbolic closure types},
      a technique that enhances type checking with symbolic execution, that helps check some
      common Clojure idioms via a compatible extension of Typed Clojure's original design.
\end{itemize}

Finally, \partref{part:related-future-work} presents the related work and future directions for each part.

\section{Previously Published Work}

\partref{part:types} has been published:

\begin{itemize}
  \item Ambrose Bonnaire-Sergeant, Rowan Davies, and Sam Tobin-Hochstadt. 
        Practical Optional Types for Clojure. In
        \emph{Proceedings of the 25th European Symposium on Programming}, 2016.
        (ESOP '16)
\end{itemize}

\partref{part:autoann} is in submission:

\begin{itemize}
  \item Ambrose Bonnaire-Sergeant, and Sam Tobin-Hochstadt.
        Squash the work: A Workflow for Typing Untyped Programs that use Ad-Hoc Data Structures.
        \emph{In Submission}
\end{itemize}
