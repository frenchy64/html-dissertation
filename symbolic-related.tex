\chapter{Related Work to Extensible Type Systems%and Interleaving Type Checking with Expansion
}

Turnstile~\cite{Chang2017TSM} type checks a program during expansion
by repurposing the Racket macro system.
Instead of the more standard approach of providing separate rules to check a macro, Turnstile
typing rules specify both the expansion and checking semantics, and so ensuring the
two are compatibile becomes automatic.
On the other hand, Typed Clojure does not have the goal of allowing users to override
how language primitives type check. Instead, our goal is to provide
a simple interface to write type rules for library functions and macros
in a style that hides the necessary bookkeeping surrounding occurrence
typing and scope management.

SugarJ~\cite{Erdweg2011SJ}
adds syntactic language extensibility to languages like Java, such as pair
syntax, embedded XML, and closures.
Desugarings are expressed as rewrite rules to plain Java.
Similarly, work on \emph{type-specific languages}~\cite{omar2014safely}
adds extensible systems for the definitions of specialized syntax literals
to existing languages.
The \emph{type} of an expression determines how it is parsed and elaborated.

% this paper has a great related works section that differentiates
% the strategies of several typed metaprogramming techniques
SoundX~\cite{Lorenzen2016STS} presents a solution to a common
dilemma in typed metaprogramming: whether to desugar before
type checking, or vice-versa.
The authors present a system where a form is type checked before 
being desugared, with a guarantee that only well-typed code is generated.
Programmers specify desugarings with a combination of typing and rewriting rules, 
which are then connected to form a valid type derivation
in a process called \emph{forwarding}.
We will explore whether we can get the same effect in Typed Clojure
without requiring the user to understand typing rules.
%For example, Scala macros~\cite{Burmako2013SML} interleave type checking and
%desugaring

Ziggurat~\cite{Fisher06staticanalysis} allows programmers to define
the static and dynamic semantics of macros separately. To demonstrate its
broad applicability, they choose Scheme-like macros that generate assembly code
for the dynamic semantics.
They advocate building towers of static analyses, so
macros can be statically checked in terms the static semantics of other macros, instead
of just their assembly code expansions which would otherwise be too difficult to check.
This idea resembles our prototypes in defining custom typing rules for functions and macros in Typed Clojure,
where the dynamic semantics are defined by runtime Clojure constructs (\texttt{defn}
and \texttt{defmacro}), and towers of static semantics are progressively specified in terms of the static
analysis of other Clojure forms.

Type Tailoring~\cite{greenmanttailoring} is an approach to provide more information
to a host type system than it might be capable of by itself.
In particular, the authors use the host platform's metaprogramming functionality
to refine the types of calls based on the program syntax alone, as well as improve
error messages by incorporating surface syntax. Their experiments are based in Typed Racket, that fully expands
syntax before checking it. Since Typed Clojure recently changed to interleave macroexpansion
and type checking, we could extend this technique to also refine calls based on the
types of their arguments (like SoundX).

Other work is relevant to our investigations of improving the user experience
of Typed Clojure. SweetT~\cite{pombrio2018inferring} automatically infers type rules
for syntactic sugar. Helium~\cite{Heeren2003STI} provides hooks into the type inference
process for domain-specific type error messages.

\chapter{Related Work to Symbolic Closures% and Combining Type Checking with Symbolic Analysis
}

\paragraph{Local Type Inference}
Symbolic closures were originally designed as an extension of Local Type Inference~\cite{PierceLTI}.
Our presentation omitted their bidirectional checking (we did not propagate type information down the syntax
tree using synthesis/checking rules)
and so was not a superset of Local Type Inference---in
particular, it does not take advantage of the relaxed optimality conditions when inferring type
arguments in checking mode.
However, adding back bidirectional checking should be possible by starting with the rules of Local
Type Inference, adding a \emph{synthesis} rule for functions that introduces a symbolic
closure, application and subtyping rules for symbolic closures, and some side conditions to restrict 
how a symbolic closure may be reasoned about (like our ``must not contain symbolic closures''
conditions scattered in various rules).
This way, a symbolic closure should only be introduced where Local Type Inference fails---when
a type of a function must be synthesized---and so seems more likely to be a superset of 
Local Type Inference.

Colored Local Type Inference~\cite{coloredlti01} extends Local Type Inference with partial
information propagation. Their type inference algorithm does not use explicit synthesis/checking
rules, instead passing ``prototypes'' \emph{P}
down the syntax tree that containing partial expected type information used for type checking.
A prototype is a type \emph{T} extended with the wildcard ``?'', denoting unknown information,
and the specific shape of a prototype denotes which type rule to use.
The rule inferring unannotated functions \ltiufun{\ltivar{}}{\ltiE{}} requires a prototype \ltiPolyFn{T}{}{P}, where
\emph{T} is the fully known expected type for {\ltivar{}}.
A symbolic closure could be introduced when checking an unannotated function with prototype
\ltiPolyFn{P}{}{P'} or ``?'' (the equivalent of Local Type Inference's ``synthesis'' rules).
In the more complicated case of \ltiPolyFn{P}{}{P'}, a symbolic reduction of 
\ltiufun{\ltivar{}}{\ltiE{}} is required to ensure it at least conforms its the prototype.
For example, inferring the type of \ltiapp{\text{map}}{\ltiufun{\ltivar{}}{\ltiE{}},[1,2,3]} with
Colored Local Type Inference (where \text{map} has type ``\ltiPolyFn{\ltiFn{\text{a}}{\text{b}},\text{List[a]}}{\text{a,b}}{\text{List[b]}}'')
checks \ltiufun{\ltivar{}}{\ltiE{}} with prototype \ltiPolyFn{\text{?}}{}{\text{?}}.
We can be optimistic and check the function 
at the largest (most specific) subtype of 
\ltiPolyFn{\ltiBot}{}{\ltiTop}
that matches \ltiPolyFn{\text{?}}{}{\text{?}}, which is \ltiPolyFn{\ltiBot}{}{\ltiTop}.
This ensures that the function at least conforms to the most optimistic interpretation of its
prototype, and then by returning a symbolic closure type instead of \ltiPolyFn{\ltiBot}{}{\ltiTop}
allows us to check more specific requirements later.
Of course, to fully check this example, it requires that we specify how type argument synthesis works with
symbolic closures, but it at least illustrates how symbolic closures relate to the rest of the system.

Spine-local type inference~\cite{Jenkins:2018:STI:3310232.3310233}
explores Local Type Inference in the context of System F (without subtyping).
They present a greedy type argument synthesis algorithm
which more aggressively propagates type information
to an application's arguments.
To check arguments, type variable instantiations are guessed
based on the expected type of the application.
For example, when checking \ltiapp{\text{id}}{\ltiufun{\ltivar{}}{\ltiE{}}}
with expected type \ltiFn{\ltiT{}}{\ltiS{}},
where \text{id} has type \ltiPolyFn{\ltitvar{}}{\ltitvar{}}{\ltitvar{}},
\ltitvar{} would be guessed to have type \ltiFn{\ltiT{}}{\ltiS{}}
and then {\ltiufun{\ltivar{}}{\ltiE{}}} would be checked at that type.
This would fail if the application was in synthesis mode.
In this specific example, symbolic closures would allow the checking
of \ltiufun{\ltivar{}}{\ltiE{}} to be delayed to when more type information
is available, in either checking or synthesis modes.
Unfortunately, it does not seem that their algorithm can check
\ltiapp{\text{map}}{\ltiufun{\ltivar{}}{\ltiE{}},[1,2,3]}
even in checking mode, and so does not seem to assist us in solving similar
problems with symbolic closures.
This case does not check because only the type of \ltiE{}
would be apparent from an expected type, not the type of \ltivar{}.

% Spine-local type inference
% - Judgement \vdash^P digs down an application to find the head
%   - happens naturally with symbolic closures
% - they use metavariables to solve direct applications
%   - can they check things like (let [x (fn [y] (inc y))] (x 1)) ?
% - they have polymorphism but not subtyping (plain System F)
%   - they speculate about extending to Fsub in related works
%   - they mention Hosoya & Pierce's "challenges" to fix hard-to-synthesize terms
% - they type check arguments left-to-right in a polymorphic application
%   - can't tell if that's different from inferring the data flow from a polytype 
%     and then checking in that order
% - their sense of "locality" is less ambitious than symbolic closures
%   - see "Type Inference Failures" section
% - good related works section for "Impredicative Polymorphism"

\paragraph{Mixing Symbolic Execution and Type Checking}
Mix~\cite{Khoo2010MTC} allows an interplay of symbolic execution~\cite{King1976SEP} with type checking
by providing syntactic regions,
with terms
\MixTregion{\ltiE{}} signaling to use type checking for {\ltiE{}},
and
\MixSregion{e} for symbolic execution.
In Mix, for example, the term
%
\[
\MixSregion{\ltilet{\text{id}}{\ltiufun{\ltivar{}}{\ltivar{}}}
                  {\MixTregion{  ...\ \MixSregion{\ltiapp{\text{id}}{3}}
                               \ ...\ \MixSregion{\ltiapp{\text{id}}{3.0}}
                               \ ... }}}
\]
%
symbolically executes \MixSregion{\ltiapp{\text{id}}{3}}
and
\MixSregion{\ltiapp{\text{id}}{3.0}}, propagating result types
\text{Int} and \text{Real} back to the typed regions, respectively.
Comparatively, symbolic closures integrates only a small amount of
symbolic execution with a type system, but in such a way that delayed symbolic computations
may pass between typed regions.
Since Mix cleanly separates symbolic execution and type checking and its formalism does not support
function types, it is difficult to compare the two approaches.
In rough terms, symbolic closures use typed regions by default and automatically adds symbolic regions
around unannotated functions.
%
\[
\MixTregion{\ltilet{\text{id}}{\MixSregion{\ltiufun{\ltivar{}}{\ltivar{}}}}
                   {  ...\ \ltiapp{\text{id}}{3}
                    \ ...\ \ltiapp{\text{id}}{3.0}
                    \ ... }}
\]
%
Typing rules are then added to introduce a symbolic closure type
and also to apply them, which involves checking the delayed body in a typed region.
%
\begin{mathpar}
\infer[]
  {}
  { \ltitjudgementNoElab{\ltiEnv{}}{\MixSregion{\ltiufun{\ltivar{}}{\ltiE{}}}}
                        {\ltiClosure{\ltiEnv{}}{\ltiufun{\ltivar{}}{\ltiE{}}}}
  }

\infer[]
  { \ltitjudgementNoElab{\ltiEnv{}}{\MixTregion{\ltiF{}}}
                        {\ltiClosure{\ltiEnvp{}}{\ltiufun{\ltivar{}}{\ltiEp{}}}}
    \\\\
    \ltitjudgementNoElab{\ltiEnv{}}{\MixTregion{\ltiE{}}}{\ltiS{}}
    \\
    \ltitjudgementNoElab{\ltiEnvConcat{\ltiEnvp{}}{\hastype{\ltivar{}}{\ltiS{}}}}
                        {\MixTregion{\ltiEp{}}}
                        {\ltiT{}}
  }
  { \ltitjudgementNoElab{\ltiEnv{}}{\MixTregion{\ltiapp{\ltiF{}}{\ltiE{}}}}{\ltiT{}}
  }
\end{mathpar}

When forced to delineate type checking from symbolic execution like this, it interesting to ask to what degree symbolic
closures even uses symbolic execution.
Our view is that (at least) symbolic closures symbolically execute the runtime-closure introduction rule.
%
\begin{mathpar}
\infer[]
  {}
  {
  \opsem {\openv{}}
         {\ltiufun{\x{}}{\e{}}}
         {\closurenosuffix{\openv{}}{\ltiufun{\x{}}{\e{}}}}
         }
\end{mathpar}
%
The symbolic closure type
{\ltiClosure{\ltiEnv{}}{\ltiufun{\ltivar{}}{\ltiE{}}}}
is then the symbolic value of the runtime closure
{\closurenosuffix{\openv{}}{\ltiufun{\x{}}{\e{}}}},
related by the following typing rule.
%
\begin{mathpar}
\infer []
{ 
  \overrightarrowcaption{
  \ltitjudgementNoElab{}{\ltiEnvLookup{\openv{}}{y}}{\ltiEnvLookup{\ltiEnv{}}{y}}
  }
  ^{y \in dom(\openv{})}
              }
{ \ltitjudgementNoElab {\ltiEnvp{}}
                       {\closurenosuffix
                        {\openv{}}
                        {\ltiufun{\ltivar{}}{\ltiE{}}}}
                       {\ltiClosure{\ltiEnv{}}{\ltiufun{\ltivar{}}{\ltiE{}}}}
          }
\end{mathpar}
%
As evidenced by the lack of symbolic regions in the above application rule,
a ``symbolic reduction'' of a symbolic closure is not particularly related
to symbolic execution---it merely kicks off some delayed type checking.
However, \ltiEnv{}, \ltiS{}, and \ltiT{} in that rule may contain symbolic closure
types, so symbolic values are being used to reason about the program.

%\begin{mathpar}
%\infer[]
%{ \opsem {\openv{}}
%         {\e{f}}
%         {\closurenosuffix {\openv{c}} {\abs {\x{}} {\ty{}} {\e{b}}}}
%         \\
%  \opsem {\openv{}}
%         {\e{a}}
%         {\val{a}}
%         \\
%  \opsem {\extendopenv {\openv{c}} {\x{}} {\val{a}}}
%         {\e{b}}
%         {\val{}}
%       }
%{ \opsem {\openv{}}
%         {\appexp {\e{f}} {\e{a}}}
%         {\val{}}
%       }
%\end{mathpar}

%Symbolic closures type check an anonymous function if annotated, otherwise it is treated symbolically.
%As the authors envision, this is akin to automatically inserting
%the mode of a code region based on its context, with a Mix-like language
%becoming the intermediate language.

Mix also uses symbolic execution to enhance simple type systems with flow-sensitivity.
For example, the following Mix program uses symbolic execution 
to flow-sensitively reason about \text{int?}, a predicate that returns true only for integer values.
%
\[
\MixSregion{\ltilet{\text{f}}{\ltiufun{\ltivar{}}{(\ltiif{\ltiapp{\text{int?}}{\ltivar{}}}{\ltivar{}}{\textsf{nil}})}}
                  {\MixTregion{  ...\ \MixSregion{\ltiapp{\text{f}}{3}}
                               \ ...\ \MixSregion{\ltiapp{\text{f}}{3.0}}
                               \ ... }}}
\]
%
The symbolic regions determine
\MixSregion{\ltiapp{\text{f}}{3}} has type \text{Int} and
\MixSregion{\ltiapp{\text{f}}{3.0}} type \text{Nil} via symbolic execution.
Symbolic closures are instead designed to be compatible with flow-sensitive type systems like occurrence typing~\cite{TF10}.
Here is the analogous program using symbolic closures.
%
\[
\MixTregion{\ltilet{\text{f}}{\MixSregion{\ltiufun{\ltivar{}}{(\ltiif{\ltiapp{\text{int?}}{\ltivar{}}}{\ltivar{}}{\textsf{nil}})}}}
                  {  ...\ {\ltiapp{\text{f}}{3}}
                               \ ...\ {\ltiapp{\text{f}}{3.0}}
                               \ ... }}
\]
%
Now, let us assume occurrence typing is also used to check this program, and
that \text{int?} is typed as a predicate for integers.
The call to
{\ltiapp{\text{f}}{3}}
triggers the symbolic reduction
%
\[
\ltitjudgementNoElab{\hastype{\ltivar{}}{\text{Int}}}
                    {(\ltiif{\ltiapp{\text{int?}}{\ltivar{}}}{\ltivar{}}{\textsf{nil}})}
                    {\text{Int}}
\]
%
which has type \text{Int}, because the else-branch is unreachable, and 
similarly {\ltiapp{\text{f}}{3.0}} triggers
%
\[
\ltitjudgementNoElab{\hastype{\ltivar{}}{\text{Real}}}
                    {(\ltiif{\ltiapp{\text{int?}}{\ltivar{}}}{\ltivar{}}{\textsf{nil}})}
                    {\text{Nil}}
\]
%
which has type \text{Nil}, because the then-branch is unreachable.

% >> talk about occurrence tpying.

% Let arguments go first

% Dunfield works I need to compare to
% - Greedy Bidirectional Polymorphism
% - Sound and complete bidirectional typechecking for higher-rank polymorphism with existentials and indexed types
% - Complete and Easy Bidirectional Typechecking for Higher-Rank Polymorphism

% Other works with undecidable type checking
% - Hybrid type checking - Knowles, Flanagan

\paragraph{Intersection Type Checking}
In hindsight, the idea behind symbolic closures resembles intersection type checking,
where the same code may be checked at multiple types.
Carlier and Wells~\cite{carlier2005expansion} give an approachable explanation of ``expansion'',
a mechanism that informs an intersection type system when it should
check the same term at different types.
This is achieved by splicing typing rules (like intersection-introduction) into existing typing
derivations that are derived from the principal typings of subterms.
In contrast, symbolic closures do not assume principal types are available, and
delays the construction of typing derivation(s) for a delayed term
altogether until it is obvious how to construct it.
Then, it is matter of combining a symbolic closure's typing derivations
to recover the (intersection) type it was used at.

%In Section 3.1, they provide a motivating example for expansion, constructed to be untypable 
%with simple (non-recursive) types, and show how expansion assigns it a type.

%TODO
%With the full power of intersection type systems, a principal type expresses

%TODO how do SC relate to this statement?
% - However, computing these principal typings is as expensive as evaluation, 
%   for the simple reason that principal typings for a term in the full system
%   express all of the information in the term’s β-normal form
% - do SC perform less reductions? eg. (lambda (x) x) does not reduce, but I assume
%   the Coppo algorithm constructs a principal type for it. Then, what if it is
%   use like (lambda (z) (let ([f (lambda (x) x)]) (f (f (f z))))) ?
%   Does it still infer an intersection type for f? Obviously, SC's do nothing here.

% survey
% Expansion: the Crucial Mechanism for Type Inference with Intersection Types: A Survey and Explanation1
% https://www.sciencedirect.com/science/article/pii/S1571066105050656
% - "expansion" seems similar to inferring intersection types via symbolic closures
%   - Section 6.2 talks about type inference for rank-2 intersection types
%     - advantage is that expansion never has to introduce intersections under ->
%       - do we do this with symbolic closures?
% - "Expansion is an operation on typings that simulates the effect of splicing in typing rules
%    uses at nested positions in some derivation of that typing."
% - omega expansion looks very similar to symbolic closures types (Section 4)
%   - at least in that it embeds the term in the type to track its origin 
% - Section 5.2 talks about cost of type inference == beta reduction

% (Intersection type systems)
% Principal Types and Unification for Simple Intersection Type Systems
% https://www.sciencedirect.com/science/article/pii/S0890540185711418

\paragraph{Higher-order Control Flow Analysis}

%% (found via the expansion survey as an application of intersection types)
% http://delivery.acm.org/10.1145/260000/258951/p1-banerjee.pdf?ip=140.182.72.36&id=258951&acc=ACTIVE%20SERVICE&key=EA62C54EFA59E1BA%2EEC3C9CD27046E2ED%2E4D4702B0C3E38B35%2E4D4702B0C3E38B35&__acm__=1553786311_eb5536bb95ba54f3a2979f7a216e898e
% A Modular, Polyvariant, and Type-Based Closure Analysis, Anindya Banerjee

%TODO aka. Closure-analysis 
% http://citeseerx.ist.psu.edu/viewdoc/summary?doi=10.1.1.36.6128
% Analysis and Efficient Implementation of Functional Programs (1991), Peter Sestoft
Closure analysis~\cite{sestoft1991analysis} approximates the set of arguments which a given
function may be applied, as well as which functions a given term
may evaluate to.
Each function term is labelled 
\Sesoftlambda{\Sesoftlabel}{\ltivar{}}{\ltiE{}}, where the label {\Sesoftlabel}
abstracts over the set of all runtime closures
\closurenosuffix{\openv{}}{\Sesoftlambda{}{\ltivar{}}{\ltiE{}}}
where runtime environment {\openv{}} can choose arbitrary bindings for {\ltiE{}}'s
free term variables.
In contrast, a ``tagged'' symbolic closure term
\ltiufunelab{\ltiClosureID{}}{\ltivar{}}{\ltiE{}}
uses identifier {\ltiClosureID{}} to stand for
\closurenosuffix{\openv{}}{\ltiufun{\ltivar{}}{\ltiE{}}}
where the bindings in the runtime environment {\openv{}} are of the types
given in the type environment \ltiEnv{} where the term was encountered
by the type checker.
In an unrestricted setting of symbolic closures, the same term
may be used with different identifiers.
For example, in
%
\[
\ltilet{\text{f}}{\ltiufun{\text{x}}{\ltiufun{\text{y}}{\text{x}}}}{...{\ltiapp{\text{f}}{3}}...{\ltiapp{\text{f}}{3.0}}...}
\]
%
the first call to \text{f} tags the inner function as {\ltiufunelab{\ltiClosureID{1}}{\text{y}}{\text{x}}}
with {\ltiClosureID{1}} standing for the set of closures
whose runtime environments bind \text{x} to a value of type \text{Int},
and the second call tags it as {\ltiufunelab{\ltiClosureID{2}}{\text{y}}{\text{x}}}
with {\ltiClosureID{2}} standing for the set of closures
who similarly bind \text{x} to a value of type \text{Real}.


% https://dl.acm.org/citation.cfm?id=201001
% Closure analysis in constraint form -	Jens Palsberg

Giannini and Rocca~\cite{giannini1988characterization}
provide the following strongly-normalizing term is not typable in System F,
which we write in Clojure and refer to as \GRterm.

%separate to preserve spacing
\begin{lstlisting}[language=Clojure]
(let [I (fn [a] a)
      K (fn [b] (fn [c] b))
      D (fn [d] (d d))]
  ((fn [x] (fn [y] ((y (x I))
                    (x K))))
   D))
\end{lstlisting}
%separate to preserve spacing

Palsberg~\cite{Palsberg:1995:CAC:200994.201001}
uses \GRterm to motivate program analyses
that answer basic questions like:
\begin{itemize}
  \item For every application point, which abstractions can be applied?
  \item For every abstraction, to which arguments can it be applied?
\end{itemize}
Symbolic closures answer neither of these questions.
Instead, they provide answers relevant to checking and inferring types:
\begin{itemize}
  \item Can \GRterm accept an argument of type \ltiT{}?
  \item When given an argument of type \ltiT{}, what type is the value returned by \GRterm?
  \item Does \GRterm inhabit \ltiFn{\ltiT{}}{\ltiS{}}?
\end{itemize}

To illustrate, we turn to
our preliminary implementation of symbolic closures~\footnote{https://github.com/frenchy64/lti-model}
to explore \GRterm.
It exposes the type checking query \clj{(tc p e)},
which returns the type of checking \clj{e} at expected prototype \clj{p},
where a prototype is a type that can contain ``wildcards'' \clj{?}.
Now we can query the type of \GRterm as if it had a type.
The caveat: without a rich enough prototype,
a benign symbolic closure type may be provided as an answer---you only get out what you put in
(for this reason, symbolic closures perform particularly well when top-level types are always provided).

For example, \clj{(tc ? GR)} asks to synthesize a type for \GRterm.
Unsurprising, a symbolic closure type greets us (below).

% full output
%(Closure
% {I (Closure {} (fn [a] a)),
%  K (Closure {I (Closure {} (fn [a] a))} (fn [b] (fn [c] b))),
%  D
%  (Closure
%   {I (Closure {} (fn [a] a)),
%    K (Closure {I (Closure {} (fn [a] a))} (fn [b] (fn [c] b)))}
%   (fn [d] (d d))),
%  x
%  (Closure
%   {I (Closure {} (fn [a] a)),
%    K (Closure {I (Closure {} (fn [a] a))} (fn [b] (fn [c] b)))}
%   (fn [d] (d d)))}
% (fn [y] ((y (x I)) (x K))))

%abbreviated
%(Closure
% {I (Closure {} I),
%  K (Closure {I (Closure {} I)} K),
%  D (Closure {I (Closure {} I), K (Closure {I (Closure {} I)} K)} D),
%  x (Closure {I (Closure {} I), K (Closure {I (Closure {} I)} K)} D)}
% (fn [y] ((y (x I)) (x K))))

% where Ic = (Closure {} I)
%       Kc = (Closure {I Ic} K)
%       Dc = (Closure {I Ic, K Kc} D)
%(Closure
% {I Ic,
%  K Kc,
%  D Dc,
%  x Dc}
% (fn [y] ((y (x I)) (x K))))
{
\[
\begin{array}{lll}
\ltiClosure{\ltiEnv{}}{\text{\clj{(fn [y] ((y (x I)) (x K)))}}},
              \text{ where }&
{\ltiEnv{}} =  \ltiEnvConcat{\hastype{\text{\clj{I}}}{{\text{\clj{I}}}_c}}
              {\ltiEnvConcat{\hastype{\text{\clj{K}}}{{\text{\clj{K}}}_c}}
              {\ltiEnvConcat{\hastype{\text{\clj{D}}}{{\text{\clj{D}}}_c}}
              {\ltiEnvConcat{\hastype{\text{\clj{x}}}{{\text{\clj{D}}}_c}}}}}\\&
{\GRclosuretag{I}} = \ltiClosure{\ltiEnv{I}}{\text{\clj{I}}}\\&
%{\GRclosuretag{K}} = \ltiClosure{\hastype{\text{\clj{I}}}{{\text{\clj{I}}}_c}}{\text{\clj{K}}}\\&
{\GRclosuretag{K}} = \ltiClosure{\ltiEnv{k}}{\text{\clj{K}}}\\&
%{\GRclosuretag{D}} = \ltiClosure{\ltiEnvConcat{\hastype{\text{\clj{I}}}{{\text{\clj{I}}}_c}}{\hastype{\text{\clj{K}}}{{\text{\clj{K}}}_c}}}
%                                  {\text{\clj{D}}}
{\GRclosuretag{D}} = \ltiClosure{\ltiEnv{D}}{\text{\clj{D}}}\\&
\end{array}
\]
}

The term of \GRclosure is the (call-by-value) normal form of \GRterm, 
derived by applying the symbolic closure of \clj{(fn [x] ...)} to  \clj{D}.
The type environment
\ltiEnv{} captures the type environment at the point the \clj{(fn [y] ...)} term
was type checked.
There, the bindings \clj{I}, \clj{K}, and \clj{D} are all symbolic closure types
{\GRclosuretag{I}},
{\GRclosuretag{K}}, and
{\GRclosuretag{D}}, respectively,
with \clj{x} also having type {\GRclosuretag{D}}
as a result of the application.
%Due to Clojure's left-to-right semantics for \clj{let} bindings, \clj{I} is also bound by
%{\ltiEnv{k}} and {\ltiEnv{D}}, and 
%\clj{K} by {\ltiEnv{D}}.

As explained in \secref{symbolic:section:formal-model}, two ways
to query a symbolic closure are by applying it or using subtyping.
We can experimentally discover what shape of argument \GRterm accepts 
by querying it at different prototypes, and using error messages and
visual inspections of \GRterm and its normal form (calculated by \GRclosure) as guidance.
We started with the query \clj{(tc [Any -> ?] GR)}, which
gives the error message \clj{Cannot invoke Any}.
Then we inspected \GRclosure, and noticed \clj{y}
must have shape \clj{[? -> [? -> ?]]} based on its usage.
Incorporating that information results in our first interesting query result.

\begin{lstlisting}[language=Clojure]
(tc [[? -> [? -> ?]] -> ?]
    GR)
;=> [[Any -> [Any -> Nothing]] -> Nothing]
\end{lstlisting}

This was calculated by observing a result type of \clj{Nothing}
from the application of \GRclosure to an argument of type \clj{[Any -> [Any -> Nothing]]}
(derived by minimizing/maximising wildcards 
in covariant/contravariant positions, respectively, with respect to the relevant part of the prototype).
We can find other interesting types \GRclosure inhabits by varying the query.

\begin{lstlisting}[language=Clojure]
(tc [[? -> [? -> Int]] -> ?]
    GR)
;=> [[Any -> [Any -> Int]] -> Int]

(tc (All [a] [[? -> [? -> a]] -> ?])
    GR)
;=> (All [a] [[Any -> [Any -> a]] -> a])
\end{lstlisting}

With the last query, we stumble on how to use \GRterm as a glorified identity function.
We can verify this by evaluating a few terms.

\begin{lstlisting}[language=Clojure]
(GR (fn [_] (fn [_] 42))) ;=> 42
(GR (fn [_] (fn [_] 24))) ;=> 24
\end{lstlisting}

We can also synthesize the types of these calls.

\begin{lstlisting}[language=Clojure]
(tc ? (GR (fn [_] (fn [_] 42))))
;=> Int
\end{lstlisting}

The original use case of symbolic closures is to type check
top-level functions against provided types,
but whose bodies are too difficult to type check with traditional means.
The following (extreme) example shows how checking the definition
of a simple identity function can be thwarted, and how symbolic closures
can make checking the definition of functions much more flexible.

\begin{lstlisting}[language=Clojure]
(tc (All [a] [a -> a])
    (fn [z]
      (GR (fn [_] (fn [_] z)))))
;=> (All [a] [a -> a])
\end{lstlisting}

This illustrates the promise of symbolic closures to treat previously untypable terms
as ``black-boxes'' during type checking, especially in a setting where top-level
type information is always provided.

%http://web.cs.ucla.edu/~palsberg/tba/papers/banerjee-icfp97.pdf
% A modular, polyvariant and type-based closure analysis - Banerjee
Banerjee~\cite{banerjee1997modular}
achieves a similar effect by
instrumenting the rank 2 fragment of the intersection type discipline
with flow information of closure values.
Function terms are labelled and
arrow types are annotated with sets of labels which denote
which functions it may represent.
They demonstrate by analyzing the following term
$(\l f. (\l x. f I)(f 0))I$
where $I$ represents the identity function.
They label the term
$(\l^1 f. (\l^2 x. f (\l^3 u. u))(f 0))(\l^4 v. v)$
and infer overall type
$t \xrightarrow[]{\{3\}} t$,
which says values of this type originate from the lambda labeled $3$,
with fresh type variable $t$.

Their system inherits the principal typing property of intersection
types, which we lack in Typed Clojure.
To compensate for this, our prototype for unrestricted symbolic closures
types returns the full code and type environment for the corresponding closure
of lambda $3$, so it may
be further checked later when more type information is available.
For example, plugging this example into our prototype gives the
following symbolic closure type (using their labelled lambda syntax)
$\ltiClosure{\ltiEnv{}}{\l^3 u. u}$,
where $\ltiEnv{} = \{f : \ltiClosure{\{\}}{\l^4 v. v}, x : \text{Int}\}$.

%\begin{lstlisting}[language=Clojure]
%(tc ? ((fn [f] ((fn [x] (f (fn [u] u))) (f 0))) (fn [v] v)))
%\end{lstlisting}


\paragraph{Hindley-Milner and Let-polymorphism}
%TODO
%\input{hm-comparison}
Kanellakis and Mitchell~\cite{kanellakis1989polymorphic}
provide a set of (pathological) ML programs that exhibit exponential
growth in the size of their principal type schemes.
%Later, Mairson~\cite{Mairson:1989} confirmed that
%the problem of ML type-checking is \textsc{DExpTime}-complete.
We use their benchmarks to compare symbolic closures with
global type inference in the style of Milner~\cite{milner1978theory}.

Example 3.1 of~\cite{kanellakis1989polymorphic} uses a lambda-encoding of pairs to create an ML principal type
which appears to grow exponentially in length, however has a linear time representation as a directed acyclic graph.
It is designed to avoid ML's let-construct to remove the influence of let-polymorphism.
The idea behind the program is to duplicate types \ltiS{} by placing them in (lambda-encoded) tuples,
following the pattern \ltiS{}, $\langle\ltiS{},\ltiS{}\rangle$, $\langle\langle\ltiS{},\ltiS{}\rangle,\langle\ltiS{},\ltiS{}\rangle\rangle$,
and so on.
To compare with symbolic closures,
let $P$ stand for \clj{(fn [x] (fn [z] (z x x)))}
and $P_z$ for \clj{(fn [z] (z x x))} in the following, where the left
hand side term has the right hand side type.
We can see the size of the type grows linearly in the number of occurrences of $P$---specifically,
the outermost symbolic closure's \emph{environment} increases in size.

{
\[
\begin{array}{c@{}c@{}c@{}c@{}ccr@{}r@{}r@{}r@{}r@{}r@{}r@{}r@{}r}
  &&{(P\ 1)}&&        & : &    \ltiClosure{\hastype{\text{x}}{&&&\text{Int}&&}&}{P_z}\\
  &{(P\ } &{(P\ 1)}& {)}&    & : &   \ltiClosure{\hastype{\text{x}}
                                                 {&&(\ltiClosure{\hastype{\text{x}}{&\text{Int}&}}{{P_z}})&&}}
                                                {P_z}\\
{(P\ }&{(P\ }&{(P\ 1)}&{)}&{)}    & : &   \ltiClosure{\hastype{\text{x}}
                                                      {&(\ltiClosure{\hastype{\text{x}}{&(\ltiClosure{\hastype{\text{x}}{&\text{Int}&}}{P_z})&}}{P_z})}&}
                                                     {P_z}
\end{array}
\]
}

Example 3.4 of~\cite{kanellakis1989polymorphic}
exhibits exponential growth in the size of ML principal types by exploiting
let-polymorphism's ability to copy type variables and assign new names to them.
It follows the following pattern, which is similar to the previous benchmark, except
intermediate values are let-bound.

\begin{minipage}[t]{0.3\linewidth}
\begin{lstlisting}[language=Clojure]
(let [x0 1
      x1 (P x0)]
  x1)
\end{lstlisting}
\end{minipage}
%
\begin{minipage}[t]{0.3\linewidth}
\begin{lstlisting}[language=Clojure]
(let [x0 1
      x1 (P x0)
      x2 (P x1)]
  x2)
\end{lstlisting}
\end{minipage}
%
\begin{minipage}[t]{0.3\linewidth}
\begin{lstlisting}[language=Clojure]
(let [x0 1
      x1 (P x0)
      x2 (P x1)
      x3 (P x2)]
  x3)
\end{lstlisting}
\end{minipage}

Unlike ML, we find that symbolic closures are rather sensitive to whether $P$
is let-bound or copied.
If let-bound at the top of each term, the types of each term are
identical to the previous example, and so grow linearly in size.
This is because the resulting type of each term is a symbolic closure of the
$P_z$ term occuring in $P$, whose definition type environment
never increases to include new variables (in particular, \clj{x1}, \clj{x2}, and \clj{x3}
are never in-scope there).
If $P$ is copied, however, the number of symbolic closures types reachable from
the innermost occurrence of $P$ grows exponentially, and so the resulting type also
grows exponentially.
We can reduce this to linear growth with sharing as below, where \clj{xi} has type \clj{Pci},
because the exponential growth happens by duplicating symbolic closure types.
Each $P_z$ term comes from a different copy of $P$, in particular
the $P_z$ term of symbolic closure type \clj{Pci} originates from the $P$ occurring on the right-hand-side of \clj{xi}.

\begin{lstlisting}[language=Clojure]
Pc1 = {x0 Int,                         x Int}@Pz
Pc2 = {x0 Int, x1 Pc1,                 x Pc1}@Pz
Pc3 = {x0 Int, x1 Pc1, x2 Pc2,         x Pc2}@Pz
Pc4 = {x0 Int, x1 Pc1, x2 Pc2, x3 Pc3, x Pc3}@Pz
\end{lstlisting}

Example 3.5 of~\cite{kanellakis1989polymorphic} gives a series of terms whose ML principal
type is doubly-exponential in the size of the term,
reduced to exponential when converted to a directed acyclic graph.
The pattern is below, which, for $i>1$ and $j=i-1$, binds \clj{xi} to \clj{(fn [y] (xj (xj y)))}.

\begin{minipage}[t]{0.31\linewidth}
\begin{lstlisting}[language=Clojure]
(let [x1 (fn [x] (P x))
      x2 (fn [y]
           (x1 (x1 y)))]
  (x2 1))
\end{lstlisting}
\end{minipage}
%
\begin{minipage}[t]{0.31\linewidth}
\begin{lstlisting}[language=Clojure]
(let [x1 (fn [x] (P x))
      x2 (fn [y]
           (x1 (x1 y)))
      x3 (fn [y]
           (x2 (x2 y)))]
  (x3 1))
\end{lstlisting}
\end{minipage}
%
\begin{minipage}[t]{0.31\linewidth}
\begin{lstlisting}[language=Clojure]
(let [x1 (fn [x] (P x))
      x2 (fn [y]
           (x1 (x1 y)))
      x3 (fn [y]
           (x2 (x2 y)))
      x4 (fn [y]
           (x3 (x3 y)))]
  (x4 1))
\end{lstlisting}
\end{minipage}

The symbolic closure type for these terms grows exponentially in size, again because 
the number of reachable closures grows exponentially. However, each symbolic closure
is distinct, so no sharing is possible. The term ending in \clj{xi}
is given type \clj{Pci}, below.

\begin{lstlisting}[language=Clojure]
Pc1 = {x Int}                                      @Pz
Pc2 = {x Pc1}                                      @Pz
Pc3 = {x {x Pc2}@Pz}                               @Pz
Pc4 = {x {x {x {x {x Pc3}@Pz}@Pz}@Pz}@Pz}          @Pz
Pc5 = {x {x {x {x {x {x {x {x {x {x {x {x Pc4}
       @Pz}@Pz}@Pz}@Pz}@Pz}@Pz}@Pz}@Pz}@Pz}@Pz}@Pz}@Pz
\end{lstlisting}

% https://link.springer.com/content/pdf/10.1007%2FBFb0032745.pdf
% Type-Directed Flow Analysis for Typed Intermediate Languages - Suresh Jagannathan, Stephen Week~s, and Andrew Wright 
% - seems like their "abstract-closures" probably relate to symbolic closures
% - they "exploit types to control a flow analysis algorithm"
\paragraph{Flow analysis for typed languages}
Jagannathan, Weeks and Wright~\cite{jagannathan1997type}
give a flow analysis for a typed intermediate language.
Their ``abstract closures'' resemble our symbolic closures,
and, like ours, their algorithm is not guaranteed to terminate.
Our work explicitly integrates symbolic closures as a new type
in the language and therefore assists in type inference,
whereas their main result is a separate flow analysis that 
exploit types to increase flow accuracy.

% Faithful Translations between Polyvariant Flows and Polymorphic Types
% http://people.cs.ksu.edu/~tamtoft/Papers/Amt+Tur:FTPFPT-2000/short.pdf
% Torben Amtoft and Franklyn Turbak

% Gilray's thesis
% pg 44 has a long list of references to check
% https://thomas.gilray.org/pdf/thesis-gilray.pdf

%TODO
%\paragraph{Directional Polymorphism}

% MLsub
% https://www.cl.cam.ac.uk/~sd601/thesis.pdf
% https://www.cl.cam.ac.uk/~sd601/papers/mlsub-preprint.pdf
%
% Polar type system (Jim)
% http://citeseerx.ist.psu.edu/viewdoc/download?doi=10.1.1.123.8718&rep=rep1&type=pdf

% Pottier
% Simplifying Subtyping Constraints: A Theory
% http://citeseerx.ist.psu.edu/viewdoc/download?doi=10.1.1.41.7032&rep=rep1&type=pdf
% A Framework for Type Inference with Subtyping%
% http://citeseerx.ist.psu.edu/viewdoc/download?doi=10.1.1.55.2364&rep=rep1&type=pdf

% ML_F
% http://gallium.inria.fr/~remy/mlf/mlf-type-inference-long.pdf


%TODO
%Xie and Oliveira~\cite{xie2018let} present a type system where
%argument type information flows to the function position in applications.
%Then, defining `let` as sugar propagates enough information to avoid
%a custom rule for `let`.
%No information is propagated from functions to applications, so the benefits
%of Colored Local Type Inference are negated.

